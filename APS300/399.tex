\section{399 --- Evaluate Division}

\textbf{Medium}

Equations are given in the format $A / B = k$, where $A$ and $B$ are variables represented as strings, and $k$ is a real number (floating point number). Given some queries, return the answers. If the answer does not exist, return $-1.0$.

\paragraph{Example:}

\textbf{Input}:

\begin{flushleft}
\fcj{equations = [ ["a", "b"], ["b", "c"] ]},

\fcj{values = [2.0, 3.0]},

\fcj{queries = [ ["a", "c"], ["b", "a"], ["a", "e"], ["a", "a"], ["x", "x"] ]}. 

\textbf{Ouput}: \fcj{[6.0, 0.5, -1.0, 1.0, -1.0 ]}

\end{flushleft} 

The input is always valid. You may assume that evaluating the queries will result in no division by zero and there is no contradiction.

\subsection{BFS}
We can treat the numerator and denumerator as the nodes in a graph, and the weight is the divide value. Based on the graph, we can use BFS or DFS to find the path for the given query.

\setcounter{lstlisting}{0}
\begin{lstlisting}[style=customc, caption={DFS}]
vector<double> calcEquation( vector<vector<string>>& equations, vector<double>& values, vector<vector<string>>& queries )
{
    //create graph
    //g: [numerator, denumberator] --> value
    map<pair<string_view, string_view>, double> g;
    //nodes: [numerator]-> list of denumerators
    unordered_map<string_view, vector<string_view>> nodes;
    size_t i{};
    for( const auto& eq : equations )
    {
        string_view x( eq[0].c_str(), eq[0].size() );
        string_view y( eq[1].c_str(), eq[1].size() );
        auto p = make_pair( x, y );
        g[p] = values[i];
        p = make_pair( y, x );
        g[p] = 1.0 / values[i];
        ++i;
        nodes[x].push_back( y );
        nodes[y].push_back( x );
    }
    vector<double> ans( queries.size(), -1.0 );
    i = {};
    //BFS to find the weight of path
    //from query[0] to query[1]
    for( const auto& query : queries )
    {
        string_view start( query[0].c_str(), query[0].size() );
        string_view dest( query[1].c_str(), query[1].size() );
        if( !nodes.count( start ) || !nodes.count( dest ) )
        {
            ++i;
            continue;
        }
        queue<pair<string_view, double>> q;
        q.emplace( start, 1.0 );
        //use seen to avoid infinite loop
        unordered_set<string_view> seen;
        while( !q.empty() )
        {
            auto [k, v] = q.front();
            q.pop();
            if( k == dest )
            {
                ans[i] = v;
                break;
            }
            seen.emplace( k );
            for( auto next : nodes[k] )
            {
                if( seen.count( next ) )
                {
                    continue;
                }
                auto key = make_pair( k, next );
                auto w = g[key];
                q.emplace( next, w * v );
            }
        }
        ++i;
    }
    return ans;
}
\end{lstlisting}