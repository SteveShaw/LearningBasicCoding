\section{300. Longest Increasing Subsequence}
Given an unsorted array of integers $A$, find the length of longest increasing subsequence.

\paragraph{Example:}

\begin{flushleft}
\textbf{Input}: [10, 9, 2, 5, 3, 7, 101, 18]
\\
\textbf{Output}: 4 
\\
\textbf{Explanation}: The longest increasing subsequence is [2, 3, 7, 101], therefore the length is 4.
\end{flushleft} 

\paragraph{Note:}
\begin{itemize}
\item There may be more than one LIS combination, it is only necessary for you to return the length.
\item Your algorithm should run in $O(n^2)$ complexity.
\end{itemize}

\paragraph{Follow up:} 
\begin{itemize}
\item Could you improve it to $O(n \log n)$ time complexity?
\end{itemize}

\subsection{Dynamic Programming}
\begin{itemize}
\item 这是典型的循环式DP。
\item Suppose $F[i]$表示以 $A[i]$为结尾的LIS的长度,
\item 对于每一个$A[i]$,从$A[0]$循环到$A[i-1]$,如果发现其中某个数$A[j] < A[i]$, update $F[i]$ as $F[i]\gets \max(F[i], F[j] + 1)$。
\item 在上述update $F[i]$时,同时记录current maximum $F[i]$ as $\ell$。
\item At the end,$\ell$就是global LIS的长度。
\end{itemize}

\setcounter{algorithm}{0}
\begin{algorithm}[H]
\caption{Dynamic Programming}
\begin{algorithmic}[1]
\Procedure{LengthOfLIS}{$A, L$}
\State $\star$ create an array $F$ with size $L$ as DP array
\State $\star$ Set all elements in $F$ as 1
\State $\star$ Set result LIS $\ell$ as 1
\For{$i:=1$ \textbf{to} $L-1$}
\For{$j:=0$ \textbf{to} $i-1$}
\algstore{300algo}
\end{algorithmic}
\end{algorithm}
\begin{algorithm}[H]
\begin{algorithmic}[1]
\algrestore{300algo}
\If{$A[j] < A[i]$}
\State $F[i]\gets \max(F[i], F[j]+1)$
\State $\ell\gets\max(\ell, F[i])$
\EndIf
\EndFor
\EndFor
\State \Return $\ell$
\EndProcedure
\end{algorithmic}
\end{algorithm}

\setcounter{lstlisting}{0}
\begin{lstlisting}[style=customc, caption={Dynamic Programming}]
int lengthOfLIS( vector<int>& nums )
{
    if( nums.empty() )
    {
        return 0;
    }

    vector<int> F( nums.size(), 1 );

    int ans = 1;

    for( size_t i = 1; i < nums.size(); ++i )
    {
        for( size_t j = 0; j < i; ++j )
        {
            if( nums[j] < nums[i] )
            {
                F[i] = ( max )( F[i], F[j] + 1 );
                //update global maximum LIS
                ans = ( max )( ans, F[i] );
            }
        }
    }

    return ans;
}
\end{lstlisting}

\subsection{Binary Search}
\begin{enumerate}
\item 初始化一个array $F$,最初为empty。
\item 遍历输入数组$A$,对于每个$A[i]$,用leftmost binary search在$F$中查找第一个不小于$A[i]$的值 $F[\omega]$,
\item 如果这个值不存在,就把$A[i]$加入到$F$
\item 否则,将$F[\omega]$替换为$A[i]$。
\item 最后返回$F$的size。
\item 特别注意的是$F$可能不是一个真实的LIS,只是长度和LIS相等。
\end{enumerate}

\begin{algorithm}[H]
\caption{Binary Search}
\begin{algorithmic}[1]
\Procedure{LengthOfLIS}{$A, L$}
\State $\star$ Create an empty array $F$
\State $\star$ Add $A[0]$ to $F$
\For{$i:=1$ \textbf{to} $L-1$}
\State $\star$ leftmost binary search $F$ to find the first index $\omega$ where $F[\omega]\geq A[i]$
\If{$\omega=\lvert F\vert$} \Comment $A[i]$ is largest for all elements in $F$
\State  $\star$ Add $A[i]$ to $F$
\Else
\State $F[\omega]\gets A[i]$ \Comment Update $F[\omega]$ to $A[i]$
\EndIf
\EndFor
\State \Return $ \lvert F\vert $ \Comment Return the size of $F$
\EndProcedure
\end{algorithmic}
\end{algorithm}

\begin{lstlisting}[style=customc, caption={Binary Search}]
int lengthOfLIS( vector<int>& nums )
{

    if( nums.empty() )
    {
        return 0;
    }

    vector<int> F;
    F.reserve( nums.size() );

    F.push_back( nums[0] );

    for( size_t i = 1; i < nums.size(); ++i )
    {

        auto p = leftmost( F, nums[i] );

        if( p == F.size() )
        {
            //no any number in F is no less than nums[i]
            F.push_back( nums[i] );
        }
        else
        {
            //update F[p] as nums[i]
            F[p] = nums[i];
        }
    }

    //The size of F is the length
    //but F itself maynot be LIS
    return F.size();
}

//leftmost binary search
size_t leftmost( vector<int>& A, int x )
{
    size_t l = 0;
    size_t r = A.size();

    while( l < r )
    {
        auto mid = ( l + r ) / 2;
        if( A[mid] < x )
        {
            l = mid + 1;
        }
        else
        {
            r = mid;
        }
    }

    return l;
}
\end{lstlisting}

\paragraph{Related Problems}
\begin{itemize}
\item \textbf{334. Increasing Triplet Subsequence}
\item \textbf{354. Russian Doll Envelopes}
\item \textbf{646. Maximum Length of Pair Chain}
\item \textbf{673. Number of Longest Increasing Subsequence}
\end{itemize}