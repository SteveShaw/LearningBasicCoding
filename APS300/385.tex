\section{385 --- Mini Parser}
Given a nested list of integers represented as a string, implement a parser to deserialize it.

Each element is either an integer, or a list -- whose elements may also be integers or other lists.

\paragraph{Note:} 
\begin{itemize}
\item You may assume that the string is well-formed:
\begin{itemize}
\item String is non-empty.
\item String does not contain white spaces.
\item String contains only digits 0-9, left and right bracket, comma and dash.
\end{itemize}

\end{itemize}


\paragraph{Example 1:}
\begin{flushleft}
\textbf{Input}: $S=324$
\\
\textbf{Output}:
\\
You should return a \texttt{NestedInteger} object which contains a single integer 324.
\end{flushleft}

\paragraph{Example 2:}
\begin{flushleft}
\textbf{Input}: $[123,[456,[789]]]$
\\
\textbf{Output}:
\\
Return a \texttt{NestedInteger} object containing a nested list with 2 elements:
\begin{enumerate}
\item An integer containing value 123.
\item A nested list containing two elements:
\begin{enumerate}
\item An integer containing value 456.
\item A nested list with one element:
\begin{enumerate}
\item An integer containing value 789.
\end{enumerate}
\end{enumerate}
\end{enumerate}
\end{flushleft}

\subsection{Stack}
\begin{enumerate}
\item 个人觉得NestedInteger这个类必须是copyable的
\item 如果字符串只包含数字,那么stack是emtpy,需要考虑这种情况。
\item 遇到一个left bracket,就往stack中放入一个新的\texttt{NestedInteger}。
\item 遇到逗号,要么前面是数字,这时候创建一个包含该数字的\texttt{NestedInteger},并加入到stack的top element的list中。不是数字的话,则已经创建出了一个list的\texttt{NestedInteger},并已经加入到stack的top element中,所以可以跳过。
\item 遇到right bracket,检查stack是否包含两个或两个以上的元素,如果是,则将当前的top element弹出,加入到其下一层的\texttt{NestedInteger} object中。如果不是,则表示我们已经搜寻到了字符串的末尾了。可以直接跳过
\item 最后返回stack的top element
\end{enumerate}

\setcounter{lstlisting}{0}
\begin{lstlisting}[style=customc, caption={Iterative With Stack}]
NestedInteger deserialize( string s )
{
    stack<NestedInteger> stk;

    int x = 0;
    int sign = 1;

    char last = 0;

    for( auto c : s )
    {
        if( c == '[' )
        {
            //add a new NestedInteger object
            stk.push( NestedInteger() );
            continue;
        }

        if( c == '-' )
        {
            //the sign symbol
            sign = -1;
            last = c;
            continue;
        }

        if( ( c >= '0' ) && ( c <= '9' ) )
        {
            x = x * 10 + ( c - '0' );
            last = c;
            continue;
        }

        if( c == ',' )
        {
            if( ( last >= '0' ) && ( last <= '9' ) )
            {
                //last element is number
                //add to top element
                stk.top().add( NestedInteger( x * sign ) );
                x = 0;
                sign = 1;
            }

            last = c;
            continue;
        }

        if( c == ']' )
        {
            if( ( last >= '0' ) && ( last <= '9' ) )
            {
                //add a new NestedInteger object to top element
                stk.top().add( NestedInteger( x * sign ) );

                x = 0;
                sign = 1;
            }

            if( stk.size() > 1 )
            {
                //pop top and add to its parent
                auto t = stk.top();
                stk.pop();

                stk.top().add( t );
            }

            last = c;
        }
    }

    if( stk.empty() )
    {
        //s only contain numbers
        return NestedInteger( x * sign );
    }

    return stk.top();
}
\end{lstlisting}

\subsection{Recursive}
\begin{enumerate}
\item Iterate over $s$, if a left bracket is met, recursively parsing the substring starting from next index.
\item otherwise, extract a number from $s$ and add to current \texttt{NestInteger} object.
\item For comma, just skip.
\item For right bracket, end current build process
\end{enumerate}

\begin{lstlisting}[style=customc, caption={Recursion}]
NestedInteger deserialize( string s )
{
    size_t p = 0;

    if( s[p] == '[' )
    {
        //call function to create list
        return parse_list( s, p );
    }

    //call function to create number
    return parse_digit( s, p );
}

NestedInteger parse_list( string& s, size_t& start )
{
    ++start; //skip '['
    NestedInteger obj_ni;

    while( start < s.size() )
    {
        auto c = s[start];
        if( c == '[' )
        {
            obj_ni.add( parse_list( s, start ) );
        }
        else if( ( c == '-' ) || ( ( c <= '9' ) && ( c >= '0' ) ) )
        {
            obj_ni.add( parse_digit( s, start ) );
        }
        else if( c == ',' )
        {
            //skip
            ++start;
        }
        else if( c == ']' )
        {
            ++start; //skip ']' and then break
            break;
        }
    }

    return obj_ni;
}

NestedInteger parse_digit( string& s, size_t& start )
{
    int x = 0;
    int sign = 1;
    //check if the number is negative
    if( s[start] == '-' )
    {
        sign = -1;
    }
    else
    {
        x = s[start] - '0';
    }

    ++start;

    while( start < s.size() )
    {
        auto c = s[start];

        if( ( c > '9' ) || ( c < '0' ) )
        {
            break;
        }

        x = x * 10 + ( c - '0' );

        ++start;
    }

    return NestedInteger( x * sign );
}
\end{lstlisting}

\paragraph{Related Problems}
\begin{itemize}
\item \textbf{341. Flatten Nested List Iterator}
\item \textbf{439. Ternary Expression Parser}
\item \textbf{722. Remove Comments}
\end{itemize}
