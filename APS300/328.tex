\section{328 --- Odd Even Linked List}
Given a singly linked list $L$, group all odd nodes together followed by the even nodes. Please note here we are talking about the node number and not the value in the nodes.
\par
You should try to do it in place. The program should run in $ O(1) $ space complexity and $ O(N) $ ($ N $ is the number of nodes in $ L $) time complexity.

\paragraph{Example 1:}

\begin{flushleft}
\textbf{Input}: 
\begin{figure}[H]
\begin{tikzpicture}
[my/.style={draw, circle, fill=gray!20!, minimum size=5mm}]
\node[my] (1) at (0,0) {1};
\node[my] (2) [right=8mm of 1] {2};
\node[my] (3) [right=8mm of 2] {3};
\node[my] (4) [right=8mm of 3] {4};
\node[my] (5) [right=8mm of 4] {5};
\node[my] (6) [right=8mm of 4] {$\varnothing$};
\draw[>=stealth,->] (1) -- (2);
\draw[>=stealth,->] (2) -- (3);
\draw[>=stealth,->] (3) -- (4);
\draw[>=stealth,->] (4) -- (5);
\draw[>=stealth,->] (5) -- (6);
\end{tikzpicture}
\end{figure}
\textbf{Output}: 
\begin{figure}[H]
\begin{tikzpicture}
[my/.style={draw, circle, fill=gray!20!, minimum size=5mm}]
\node[my] (1) at (0,0) {1};
\node[my] (2) [right=8mm of 1] {3};
\node[my] (3) [right=8mm of 2] {5};
\node[my] (4) [right=8mm of 3] {2};
\node[my] (5) [right=8mm of 4] {4};
\node[my] (6) [right=8mm of 4] {$\varnothing$};
\draw[>=stealth,->] (1) -- (2);
\draw[>=stealth,->] (2) -- (3);
\draw[>=stealth,->] (3) -- (4);
\draw[>=stealth,->] (4) -- (5);
\draw[>=stealth,->] (5) -- (6);
\end{tikzpicture}
\end{figure}

\end{flushleft}
\paragraph{Example 2:}
\begin{flushleft}
\textbf{Input}: 
\begin{figure}[H]
\begin{tikzpicture}
[my/.style={draw, circle, fill=gray!20!, minimum size=5mm}]
\node[my] (1) at (0,0) {2};
\node[my] (2) [right=8mm of 1] {1};
\node[my] (3) [right=8mm of 2] {3};
\node[my] (4) [right=8mm of 3] {5};
\node[my] (5) [right=8mm of 4] {6};
\node[my] (6) [right=8mm of 5] {4};
\node[my] (7) [right=8mm of 6] {7};
\node[my] (8) [right=8mm of 7] {$\varnothing$};
\draw[>=stealth,->] (1) -- (2);
\draw[>=stealth,->] (2) -- (3);
\draw[>=stealth,->] (3) -- (4);
\draw[>=stealth,->] (4) -- (5);
\draw[>=stealth,->] (5) -- (6);
\draw[>=stealth,->] (6) -- (7);
\draw[>=stealth,->] (7) -- (8);
\end{tikzpicture}
\end{figure}
\textbf{Output}: 
\begin{figure}[H]
\begin{tikzpicture}
[my/.style={draw, circle, fill=gray!20!, minimum size=5mm}]
\node[my] (1) at (0,0) {2};
\node[my] (2) [right=8mm of 1] {3};
\node[my] (3) [right=8mm of 2] {6};
\node[my] (4) [right=8mm of 3] {7};
\node[my] (5) [right=8mm of 4] {1};
\node[my] (6) [right=8mm of 5] {5};
\node[my] (7) [right=8mm of 6] {4};
\node[my] (8) [right=8mm of 7] {$\varnothing$};
\draw[>=stealth,->] (1) -- (2);
\draw[>=stealth,->] (2) -- (3);
\draw[>=stealth,->] (3) -- (4);
\draw[>=stealth,->] (4) -- (5);
\draw[>=stealth,->] (5) -- (6);
\draw[>=stealth,->] (6) -- (7);
\draw[>=stealth,->] (7) -- (8);
\end{tikzpicture}
\end{figure}
\end{flushleft}

\paragraph{Note:}

\begin{itemize}
\item The relative order inside both the even and odd groups should remain as it was in the input.
\item The first node is considered odd, the second node even and so on \dots
\end{itemize}

\subsection{Two Pointers}
\begin{itemize}
\item 建立两个指针,分别指向odd和even节点,同时记录even节点的头。
\item 遍历链表时,先把odd节点的next指向even的next,然后odd前进到其next节点。
\item 接着把even节点的next指向当前odd的next,由于odd已经更新为even的下一个节点,因此odd的next节点就是下一个even节点。
\item 然后even前进到其next节点。以此类推
\item 最后把odd节点的next设置为开始时保存的even节点的头。
\end{itemize}

\setcounter{lstlisting}{0}
\begin{lstlisting}[style=customc, caption={Two Pointers}]
ListNode* oddEvenList( ListNode* head )
{
    if( !head )
    {
        return head;
    }

    auto odd = head;
    auto even = head->next;

    auto evenHead = even;

    while( even && even->next )
    {
        odd->next = even->next;
        odd = odd->next;
        even->next = odd->next;
        even = even->next;
    }

    odd->next = evenHead;
    return head;
}
\end{lstlisting}

\paragraph{Related Problems}
\begin{itemize}
\item \textbf{725. Split Linked List in Parts}
\end{itemize}