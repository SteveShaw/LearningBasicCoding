\section{378 --- Kth Smallest Element in a Sorted Matrix}
Given a $ n \times n $ matrix $M$ where each of the rows and columns are sorted in ascending order, find the $k$th smallest element in the matrix.
\par
Note that it is the $k$th smallest element in the sorted order, not the $k$th distinct element.

\paragraph{Example:}
\begin{flushleft}
\textbf{Input}: $k=8$ and $M$ is 
\begin{table}[H]
\begin{tabular}{ccc}
1 & 5 & 9\\
10 & 11 & 13\\
12 & 13 & 15
\end{tabular}
\end{table}
\textbf{Output}: 13
\end{flushleft}

\paragraph{Note: }
\begin{itemize}
\item You may assume $ k $ is always valid, $ 1 \leq k \leq n^2 $.
\end{itemize}

\subsection{Binary Search}
在binary search中,首先需要定义Search Space。对于数组搜索来说,有两种类型的Search Space: index和range。大部分情况下,当数组是在一个方向上是sorted,就用index作为Search Space。而当数组没有排序,而需要寻找某个特定数时,需要用range作为Search Space。这里的range指的是数组中最小和最大数之间的range。
\par
这道题的Search Space只能是用range,这是因为整个matrxi实际上是在row和column两个方向上是sorted,但row与column之间并不是,所以也就找不到一个线性的方式来建立某个数和其index的映射。
\par
采用range作为search space,
\begin{enumerate}
\item 首先以当前range的中间数作为target,在matrix中寻找不大于这个target的元素的个数$\delta$,如果$\delta$小于$k$,说明target小了,将range的左边界更新为这个中间数加1。
\item 而如果$\delta$大于或者等于$k$,说明目标数大了,将range的右边界更新为中间数。
\item 而在matrix中计算$\delta$时,最佳的方法是从$M[C-1][0]$开始,逐行逐列的搜索。假设当前搜索到了位置$(r,c)$,
\begin{itemize}
\item 如果$M[r][c]$比目标值小,。由于column $c$中,小于$r$的所有元素都比$M[r][c]$小,因此$\delta$增加$r+1$(因为$r$是从0开始的),另外,increment column $c$,即$c\gets c+1$,即向右移动。
\item 反之,则decrement $r$,即向上移动。
\end{itemize}
\end{enumerate}
可以看到上面的过程其实就是leftmost binary search,只不过Search Space是range,不是index。需要在range中找到第一个数,使得range中小于等于这个数的count为$k$。


\setcounter{lstlisting}{0}
\begin{lstlisting}[style=customc, caption={Binary Search}]
int kthSmallest( vector<vector<int>>& matrix, int k )
{
    int l = matrix[0][0];
    int r = matrix.back().back();

    //leftmost binary search
    //search space is range
    while( l < r )
    {
        int mid = l + ( r - l ) / 2;
        int count = get_numbers( matrix, mid );

        if( count < k )
        {
            l = mid + 1;
        }
        else
        {
            r = mid;
        }
    }

    return l;

}
//Helper function
//get count of numbers that are no larger than target in matrix
int get_numbers( vector<vector<int>>& matrix, int target )
{
    //matrix size is n x n
    int n = static_cast<int>( matrix[0].size() );

    int c = 0;
    int r = n - 1;
    int count = 0;

    while( ( r >= 0 ) && ( c < n ) )
    {
        if( matrix[r][c] <= target )
        {
            count += ( r + 1 );
            ++c;
        }
        else
        {
            --r;
        }
    }

    return count;
}
\end{lstlisting}

We can use iterator to simplify some code

\begin{lstlisting}[style=customc, caption={STL Iterator}]
int kthSmallest( vector<vector<int>>& matrix, int k )
{
    //the search space is [lo, hi]
    int lo = matrix[0][0];
    int hi = matrix.back().back();
    //get count of elements inside matrix
    //which is less than x
    auto less_count = [ &matrix ]( int x )
    {
        int count = 0;
        auto rp = rbegin( matrix );
        size_t col{};
        while( ( rp != matrix.rend() ) && ( col < matrix[0].size() ) )
        {
            if( ( *rp )[col] <= x )
            {
                //from first row to current row
                //the element at this column
                //are all no larger than x
                //notice, we want to get count of rows from 0 to current row
                //matrix.rend() is the row before 0, and rp is current row
                //so distance( rp, matrix.rend() ) give the number of rows
                count += ( int ) distance( rp, matrix.rend() );
                ++col;
            }
            else
            {
                ++rp;
            }
        }
        return count;
    };
    //leftmost binary search
    while( lo < hi )
    {
        auto mid = lo + ( hi - lo ) / 2;
        auto count = less_count( mid );
        if( count < k )
        {
            lo = mid + 1;
        }
        else
        {
            hi = mid;
        }
    }
    return lo;
}
\end{lstlisting}

\paragraph{Similar Problems}
\begin{itemize}
\item \textbf{373. Find K Pairs with Smallest Sums}
\item \textbf{668. Kth Smallest Number in Multiplication Table}
\item \textbf{719. Find K-th Smallest Pair Distance}
\item \textbf{786. K-th Smallest Prime Fraction}
\end{itemize}