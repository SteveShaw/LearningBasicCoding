\section{367 --- Valid Perfect Square}
Given a positive integer $ n $, write a function which returns \texttt{true} if $ n $ is a perfect square else \texttt{false}.

\paragraph{Note:} 
\begin{itemize}
\item Do not use any built-in library function such as \texttt{sqrt}.
\end{itemize}

\paragraph{Example 1:}

\begin{flushleft}
\textbf{Input}: 16
\\
\textbf{Output}: \texttt{true}
\end{flushleft}

\paragraph{Example 2:}

\begin{flushleft}
\textbf{Input}: 14
\\
\textbf{Output}: \texttt{false}
\end{flushleft}

\subsection{Newton}
我们需要查找$x^2=n$,即找到函数$f(x)=x^2-n$的root。根据Newton插值法,迭代关系如下
\[
x_{+1} = x_k - \frac{f(x_k)}{f^{'}(x_k)}
\]
将$f(x)$代入上式,可以得到
\begin{align*}
x_{k+1} &= x_k - \frac{x_{k}^2-n}{2x_k}\\
&=\frac{x_k}{2} + \frac{n}{2x_k}
\end{align*}
因此我们从初始值$n$带入上式进行递推测试,直到$x_k^2\leq n$。
\par
另外需要注意的是,由于在计算过程中都是用整数类型,为了避免上述除以2产生误差,因此将除以2放到最后进行,即按照
\[
x_{k+1} = \dfrac{x_k + \dfrac{n}{x_k}}{2}
\]

\setcounter{lstlisting}{0}
\begin{lstlisting}[style=customc, caption={Newton Interpolation}]
bool isPerfectSquare( int num )
{
    //to avoid overflow for int type
    //convert to long long
    auto x = static_cast<long long>( num );
    auto lln = x;

    //newton interpolation
    while( x * x > lln )
    {
        x = ( x + lln / x ) / 2;
    }

    return ( x * x == lln );
}
\end{lstlisting}

\subsection{Binary Search}
这个其实也可以用leftmost binary search来获得。

\begin{lstlisting}[style=customc, caption={Binary Search}]
bool isPerfectSquare( int num )
{
    long long l = 0;
    auto r = static_cast<int>( num );
    auto lln = r;

    //binary search
    while( l < r )
    {
        auto mid = ( l + r ) / 2;

        if( mid * mid < lln )
        {
            l = mid + 1;
        }
        else
        {
            r = mid;
        }
    }

    return l * l == lln;
}
\end{lstlisting}