\section{375 --- Guess Number Higher or Lower II}
We are playing the Guess Game. The game is as follows:

\begin{itemize}
\item I pick a number from 1 to $ n $. You have to guess which number I picked.

\item Every time you guess wrong, I'll tell you whether the number I picked is higher or lower.

\item However, when you guess a particular number $ x $, and you guess wrong, you pay \$$x$. You win the game when you guess the number I picked.
\end{itemize}

\paragraph{Example:}

\begin{flushleft}
$ n = 10 $, I pick 8.

First round:  You guess 5, I tell you that it's higher. You pay \$5.

Second round: You guess 7, I tell you that it's higher. You pay \$7.

Third round:  You guess 9, I tell you that it's lower. You pay \$9.

Game over. 8 is the number I picked.

You end up paying $\$5 + \$7 + \$9 = \$21$
\end{flushleft}

Given a particular $n \geq 1$, find out how much money you need to have to guarantee a win.

\subsection{MinMax Algorithm}
\begin{itemize}
\item 建立一个二维的数组$F$用于Dynamic Programming,其中$F[i][j]$表示从数字$i$到$j$之间猜中任意一个数字最少需要花费的钱数。
\item 需要遍历每一段可能的区间$[x, y]$,用$t$表示全局最小值,然后遍历$ [x,y] $的中每一个数字$k$,猜中$k$所需要花费的money $u_k = k + \max(F[x][k - 1], F[k + 1][y])$,也就是是将区间$ [x,y] $在每一个位置$k$都分为两段,然后取当前位置的花费加上左右两段中较大的花费之和。
\item 取两者之间的较大值的原因是在MinMax中,需要从最坏的情况下获得最佳值。
\item 最后取$\underbrace{\min}_{x< k < y}u_k$作为$F[x][y]$。
\item 如果$y=x+1$,很显然$F[x][y]=x$。
\end{itemize}


\setcounter{lstlisting}{0}
\begin{lstlisting}[style=customc, caption={Dynamic Programming}]
int getMoneyAmount( int n )
{
    //we use [0,n-1] to represent range [1,n]
    vector<vector<int>> F( n, vector<int>( n, INT_MAX ) );

    for( int i = 0; i < n; ++i )
    {
        //i is the correct guessed number
        //no money is needed
        F[i][i] = 0;
    }

    for( int i = 0; i < n - 1; ++i )
    {
        //we want to guess smaller number i+1
        //to put less money i+1
        //notice we use [0,n-1] to represent range [1,n]
        F[i][i + 1] = i + 1;
    }

    for( int l = 3; l <= n; ++l )
    {
        for( int x = 0; x + l - 1 < n; ++x )
        {
            int y = x + l - 1;

            //find minimum money for guessing number in range
            //[x+1][y+1]
            for( int k = x + 1; k < y; ++k )
            {
                //for each number k+1
                //the cost to guess this number
                int u = k + 1 + ( max )( F[x][k - 1], F[k + 1][y] );
                F[x][y] = ( min )( F[x][y], u );
            }
        }
    }

    return F[0][n - 1];
}
//more clearer one
int getMoneyAmount( int n )
{
    //DP array: we only care from 1 to n
    vector<vector<int>> F( n + 1, vector<int>( n + 1, INT_MAX ) );
    for( int i = 1; i < n; ++i )
    {
        //for (i, i+1), the optimal strategy
        //is place i
        F[i][i + 1] = i;
        //guess correct i don't need money
        F[i][i] = 0;
    }
    //don't forget the last one
    F[n][n] = 0;
    //three level loops to check each possible range [start,end]
    for( int l = 3; l <= n + 1; ++l )
    {
        for( int start = 1; start + l <= n + 1; ++start )
        {
            int end = start + l - 1;
            //test each number inside [start,end]
            for( int k = start + 1; k < end; ++k )
            {
                //maximum first
                int u = k + ( max )( F[start][k - 1], F[k + 1][end] );
                //minimum second
                F[start][end] = ( min )( F[start][end], u );
            }
        }
    }
    return F[1][n];
}
\end{lstlisting}

\subsection{Recursive With Memorize}
We can also use memorize with recursion to solve this problem. Essentially is the same idea with DP

\begin{lstlisting}[style=customc, caption={Memorization}]
int getMoneyAmount( int n )
{
    vector<vector<int>> memo( n + 1, vector<int>( n + 1, 0 ) );
    return dfs( 1, n, memo );
}
int dfs( int start, int end, vector<vector<int>>& memo )
{
    if( start >= end )
    {
        return 0;
    }

    if( memo[start][end] > 0 )
    {
        return memo[start][end];
    }

    int res = INT_MAX;
    //we must loop from start to end
    //not start+1 to end-1
    //otherwise, we will ignore the edge corner case
    for( int k = start; k <= end; ++k )
    {
        int l = dfs( start, k - 1, memo );
        int r = dfs( k + 1, end, memo );
        int u = k + ( max )( l, r );
        res = ( min )( res, u );
    }
    memo[start][end] = res;
    return res;
}
\end{lstlisting}

\paragraph{Similar Problems}
\begin{itemize}
\item \textbf{294. Flip Game II}
\item \textbf{374. Guess Number Higher or Lower}
\item \textbf{464. Can I Win}
\item \textbf{658. Find K Closest Elements}
\end{itemize}