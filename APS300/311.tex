\section{311 --- Sparse Matrix Multiplication}

\textbf{Medium}

Given two sparse matrices $A$ and $B$, return the result of $AB$.

You may assume that $A$'s column number is equal to $B$'s row number.

\paragraph{Example:}

\begin{flushleft}
\[
A = \begin{bmatrix}
1 & 0 & 0 \\
-1 & 0 & 3 \\
\end{bmatrix}
\]

\[
B = \begin{bmatrix}
7 & 0 & 0 \\
0 & 0 & 0 \\
0 & 0 & 1 \\
\end{bmatrix}
\]

\textbf{Output}:

\[
AB = \begin{bmatrix}
1 & 0 & 0 \\
-1 & 0 & 3 \\
\end{bmatrix} \times \begin{bmatrix}
7 & 0 & 0 \\
0 & 0 & 0 \\
0 & 0 & 1 \\
\end{bmatrix}
=\begin{bmatrix}
7 & 0 & 0 \\
-7 & 0 & 3 \\
\end{bmatrix}
\]


\end{flushleft}

\subsection{Skip Zero}
For matrix multiplication result $C$, 

\[
C[i]][j] = A[i][0]\times B[0][j] + A[i][1]\times B[1][j] + \ldots + A[i][k]\times B[k][j]
\]

Thus if $A[i][k]$ is zero, we can skip. Then we iterate over $k$th row of $B$. 

\setcounter{lstlisting}{0}
\begin{lstlisting}[style=customc, caption={Skip Zero}]
vector<vector<int>> multiply( vector<vector<int>>& A, vector<vector<int>>& B )
{
    vector<vector<int>> ans( A.size(), vector<int>( B[0].size(), 0 ) );
    for( size_t i = 0; i < A.size(); ++i )
    {
        for( size_t k = 0; k < A[i].size(); ++k )
        {
            if( A[i][k] )
            {
                for( size_t j = 0; j < B[0].size(); ++j )
                {
                    if( B[k][j] )
                    {
                        ans[i][j] += A[i][k] * B[k][j];
                    }
                }
            }
        }
    }
    return ans;
}
\end{lstlisting}

\paragraph{Related Problems}
\begin{itemize}
\item \textbf{264. Ugly Number II}
\end{itemize}