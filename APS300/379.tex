\section{379 --- Design Phone Directory}

\textbf{Medium}

Design a Phone Directory which supports the following operations:

\begin{itemize}
\item \fcc{get}: Provide a number which is not assigned to anyone.

\item \fcc{check}: Check if a number is available or not.

\item \fcc{release}: Recycle or release a number.
\end{itemize}

\paragraph{Example:}

\begin{flushleft}
\fcc{// Init a phone directory containing a total of 3 numbers: 0, 1, and 2.}

\fcc{PhoneDirectory directory = new PhoneDirectory(3);}

\fcc{// It can return any available phone number. Here we assume it returns 0.}

\fcc{directory.get();}

\fcc{// Assume it returns 1.}

\fcc{directory.get();}

\fcc{// The number 2 is available, so return true.}

\fcc{directory.check(2);}

\fcc{// It returns 2, the only number that is left.}

\fcc{directory.get();}

\fcc{// The number 2 is no longer available, so return false.}

\fcc{directory.check(2);}

\fcc{// Release number 2 back to the pool.}

\fcc{directory.release(2);}

\fcc{// Number 2 is available again, return true.}

\fcc{directory.check(2);}
\end{flushleft}

\subsection{Queue With Hash Set}
In order to achieve average $O(1)$ time complexity for each operation, we have to push each number into the queue.

\setcounter{lstlisting}{0}
\begin{lstlisting}[style=customc, caption={Queue}]
class PhoneDirectory
{
public:
    /** Initialize your data structure here
        @param maxNumbers - The maximum numbers that can be stored in the phone directory. */
    PhoneDirectory( int maxNumbers )
    {
        //add all numbers into the queue
        for( int i = 0; i < maxNumbers; ++i )
        {
            q.push( i );
        }
    }
    /** Provide a number which is not assigned to anyone.
        @return - Return an available number. Return -1 if none is available. */
    int get()
    {
        if( q.empty() )
        {
            //we have used all numbers
            return -1;
        }
        int num = q.front();
        q.pop();
        used.insert( num );
        return num;
    }
    /** Check if a number is available or not. */
    bool check( int number )
    {
        return used.find( number ) == used.end();
    }
    /** Recycle or release a number. */
    void release( int number )
    {
        if( used.find( number ) == used.end() )
        {
            //this number is not used
            //no need to recycle
            return;
        }
        q.push( number );
        used.erase( number );
    }
    queue<int> q;
    unordered_set<int> used;
};
\end{lstlisting} 