\section{330 --- Patching Array}
Given a sorted positive integer array $A$ and an integer $ n $, add/patch elements to the array such that any number in range $ [1, n] $ inclusive can be formed by the sum of some elements in the array. Return the minimum number of patches required.

\paragraph{Example 1:}

\begin{flushleft}
\textbf{Input}: $A = [1,3]$, $n = 6$
\\
\textbf{Output}: 1 
\\
\textbf{Explanation}:
\\
Combinations of $A$ are $[1]$, $[3]$, $ [1,3] $, which form possible sums of: 1, 3, 4.
\\
Now if we add/patch 2 to $ A $, the combinations are: $ [1] $, $ [2] $, $ [3] $, $ [1,3] $, $ [2,3] $, $ [1,2,3] $.
\\
Possible sums are 1, 2, 3, 4, 5, 6, which now covers the range $ [1, 6] $.
\\
So we only need 1 patch.
\end{flushleft}

\paragraph{Example 2:}

\begin{flushleft}
\textbf{Input}: $ A = [1,5,10] $, $ n = 20 $
\\
\textbf{Output}: 2
\\
\textbf{Explanation}: The two patches can be $ [2, 4] $.
\end{flushleft}

\paragraph{Example 3:}

\begin{flushleft}
\textbf{Input}: $A = [1,2,2]$, $n = 5$
\\
\textbf{Output}: 0
\end{flushleft}
\subsection{Dynamic Programming}
\begin{enumerate}
\item 定义一个变量$ x $,用来表示目前数组$A$加上已经有的patch number不能够build出的最小值,即当前build出的range为$[0\ldots x-1]$。
\item 初始化为$x$为1。
\item 由于能得到的数字范围是$[0, x-1]$,如果在$A$中有number $y\leq x$,那么我们可以把build的范围扩大到$[0\ldots x-1+y]$。
\item 而如果$y>x$,那么此时就需要添加一个patch number。这个number需要选择为能最大限度的增加能够得到的range。很显然这个数就是$x$。
\item 按照上述过程遍历完整个数组,即可得到结果所要求的patch number。
\end{enumerate}
例如,假设输入数组$A = [1, 5, 10] $, $ n=20 $。
\begin{itemize}
\item 初始化$x$为1。即最开始能够得到range为$[0]$。
\item 由于$A$中第一个数1小等于$x$,因此直接加上这个数扩展range。这时候range扩展为$[0, 1]$,$x$更新为2。
\item $A$下一个数5对于2来说比较大,因此要加上一个patch number $x$即2,扩展range到$[0, 3]$,$x$更新为4。
\item 接下来5仍然大于$x$,因此继续加上一个patch number 4,扩展range到$[0,8]$,$x$更新为9。
\item 由于这时候5小于$x$,加上这个数继续扩展range,这时候range扩展为$[0, 14]$,$x$更新为15。
\item  $A$再下一个数10小于15,加上这个扩展range为$[0, 25]$。已经能够覆盖$[0,20]$了。
\end{itemize}


\setcounter{lstlisting}{0}
\begin{lstlisting}[style=customc, caption={Dynamic Programming}]
int minPatches( vector<int>& nums, int n )
{
    //expand to long long type
    //to avoid data type overflow
    long long x = 1;
    auto ll_n = static_cast<int>( n );

    size_t i = 0;
    int ans = 0;

    while( x <= ll_n )
    {
        if( ( i < nums.size() ) && ( nums[i] <= x ) )
        {
            //extend the range to [0,x-1+nums[i]]
            x += nums[i];
            ++i;
        }
        else
        {
            //add a patch number x
            //extend the range to [0, x-1+x]
            x += x;
            ++ans;
        }
    }

    return ans;
}

\end{lstlisting}

