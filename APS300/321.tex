\section{321 --- Create Maximum Number}
Given two arrays of length $ m $ and $ n $ with digits 0--9 representing two numbers. Create the maximum number of length $ k \leq m + n $ from digits of the two. The relative order of the digits from the same array must be preserved. Return an array of the k digits.

\paragraph{Note:} 
\begin{itemize}
\item You should try to optimize your time and space complexity.
\end{itemize}

\paragraph{Example 1:}

\begin{flushleft}
\textbf{Input}:
\\
$ A  = [3,\;4,\; 6,\; 5]$
\\
$ B  = [9,\; 1,\; 2, \;5,\; 8,\;3]$
\\
$ k = 5 $
\\
\textbf{Output}: $ [9,\; 8,\; 6,\; 5,\; 3] $
\end{flushleft}

\paragraph{Example 2:}

\begin{flushleft}
\textbf{Input}:
\\
$ A = [6,\; 7] $
\\
$ B = [6,\; 0,\; 4] $
\\
$ k = 5 $
\\
\textbf{Output}: $ [6,\; 7, \;6,\; 0,\; 4] $
\end{flushleft}

\paragraph{Example 3:}

\begin{flushleft}
\textbf{Input}:
\\
$ A = [3,\; 9] $
\\
$ B = [8,\; 9] $
\\
$ k = 3 $
\\
\textbf{Output}: $ [9, \;8, \;9] $
\end{flushleft}
\subsection{Greedy Way}
\begin{enumerate}
\item 首先要确定每个数组中要取几个数,这个只能通过循环一个个测试。比如如果$A$选择取$x$个数,那么$B$就只能选择取$y=k-x$个数
\item For each iteration,对于每个数组,用贪心法得到可能组成的最大数。
\item 然后merge这两个数组得到的最大数,有点类似于merge sort的合并过程。但是需要定义如何进行两个array的比较,这里需要采用lexically comparison的方法。
\item 最后选择所有测试中的最大数。这个过程同样要基于对两个array进行lexically compare的方法。
\item 在对每个数组用贪心的方法取得组成的最大数时,可能最后产生的最大数的长度包含多于给定长度。这是因为在generate的过程中,数组的后面是否还有更大的数,需要不断的放入结果数组中,所以最后需要把多出来的数字去除掉。 \label{step0}
\item 另外在对每个数组用贪心的方法取得组成的最大数的过程中,需要用一个变量来记录该数组中需要drop掉多少个number,即数组长度减去需要的长度。如果没有这个变量来控制,很可能生成的最大数的长度会小于需要的长度。
\end{enumerate}


\setcounter{lstlisting}{0}
\begin{lstlisting}[style=customc, caption={Greedy Approach}]
vector<int> maxNumber( vector<int>& nums1, vector<int>& nums2, int k )
{
    int l1 = static_cast<int>( nums1.size() );
    int l2 = static_cast<int>( nums2.size() );

    //The count of numbers selected from nums1 array
    int sel1 = ( max )( k - l2, 0 );

    //This is the maximum count of numbers
    //can be selected from nums1 array
    int max_sel1 = ( min )( k, l1 );

    vector<int> ans( k, INT_MIN );

    vector<int> merged( k, 0 );

    vector<int> v_maxFrom1;
    vector<int> v_maxFrom2;

    v_maxFrom1.reserve( ( min )( k, l1 ) );
    v_maxFrom2.reserve( ( min )( k, l2 ) );

    for( ; sel1 <= max_sel1; ++sel1 )
    {
        process( nums1, sel1, v_maxFrom1 );
        //k-sel1 is the count of numbers selected from nums2 array
        process( nums2, k - sel1, v_maxFrom2 );

        merge( v_maxFrom1, v_maxFrom2, merged );

        //lexically compare two arrays
        if( lexical_compare( ans, 0, merged, 0 ) )
        {
            swap( ans, merged );
        }

        v_maxFrom1.clear();
        v_maxFrom2.clear();
    }

    return ans;
}

void process( const vector<int>& A, int sel, vector<int>& out )
{
    if( sel == 0 )
    {
        return;
    }

    //drop will control how many numbers
    //will be kept in out array
    int drop = static_cast<int>( A.size() ) - sel;

    for( int n : A )
    {
        while( drop && !out.empty() && ( out.back() < n ) )
        {
            out.pop_back();
            --drop;
        }

        out.push_back( n );
    }


    //since we do not increment drop
    //the final kept numbers will always no less than sel
    out.resize( sel );
}

//Use lexically compare function to compare two arrays
void merge( const vector<int>& v1, const vector<int>& v2, vector<int>& out )
{
    int l1 = static_cast< int >( v1.size() );
    int l2 = static_cast< int >( v2.size() );

    int x = 0;
    int y = 0;

    int p = 0;
    while( ( x < l1 ) || ( y < l2 ) )
    {
        if( lexical_compare( v1, x, v2, y ) )
        {
            out[p] = v2[y];
            ++p;
            ++y;
        }
        else
        {
            out[p] = v1[x];
            ++p;
            ++x;
        }
    }


}

//The routine to compare arrays lexically
bool lexical_compare( const vector<int>& A, int x, const vector<int>& B, int y )
{
    int m = static_cast< int >( A.size() );
    int n = static_cast< int >( B.size() );

    while( ( x < m ) && ( y < n ) )
    {
        if( A[x] < B[y] )
        {
            return true;
        }

        if( A[x] > B[y] )
        {
            return false;
        }

        ++x;
        ++y;
    }

    //A's length is smaller than B's length
    //and all A's numbers equal to previous m numbers in B
    return ( x == m ) && ( y < n );
}

\end{lstlisting}

We can replace with more concise STL functions

\begin{lstlisting}[style=customc, caption={STL version}]
vector<int> maxNumber( vector<int>& nums1, vector<int>& nums2, int k )
{
    int L1 = ( int )nums1.size();
    int L2 = ( int )nums2.size();
    //lambad to fill dst
    //with number = sel elements from src
    //by maintaining relative order and maximum
    auto fill_max = []( vector<int>& dst, const vector<int>& src, int sel )
    {
        int drop = ( int )src.size();
        drop -= sel;
        for( int n : src )
        {
            while( ( drop > 0 ) && ( !dst.empty() ) && ( dst.back() < n ) )
            {
                dst.pop_back();
                --drop;
            }
            dst.push_back( n );
        }
        dst.resize( sel );
    };
    //max1 and max2 are holding
    //one time maximum numbers from nums1 and nums2
    vector<int> max1;
    max1.reserve( ( min )( L1, k ) );
    vector<int> max2;
    max2.reserve( ( min )( L2, k ) );
    vector<int> ans( k, INT_MIN );
    //merged result of max1 and max2
    vector<int> merged( k, 0 );
    for( int sel1 = ( max )( 0, k - L2 ); sel1 <= ( min )( k, L1 ); ++sel1 )
    {
        if( sel1 != 0 )
        {
            fill_max( max1, nums1, sel1 );
        }
        if( k - sel1 != 0 )
        {
            fill_max( max2, nums2, k - sel1 );
        }
        //merge max1 and max2
        //each time, fetch one element from
        //the lexicographically larger one
        help( begin( max1 ), end( max1 ), begin( max2 ), end( max2 ), begin( merged ) );
        //if ans < merged, change ans to merged
        if( lexicographical_compare( begin( ans ), end( ans ), begin( merged ), end( merged ) ) )
        {
            swap( merged, ans );
        }
        max1.clear();
        max2.clear();
    }

    return ans;
}
//help function to merge [first1,last1) and [first2, last2)
//by lexicographically comparison
void help( vector<int>::iterator first1, vector<int>::iterator last1,
           vector<int>::iterator first2, vector<int>::iterator last2,
           vector<int>::iterator out )
{
    while( ( first1 != last1 ) && ( first2 != last2 ) )
    {
        if( lexicographical_compare( first1, last1, first2, last2 ) )
        {
            //[first1, last1) < [first2, last2)
            //use *first2
            *out = *first2;
            ++first2;
        }
        else
        {
            //[first1, last1) >= [first2, last2)
            //use *first1
            *out = *first1;
            ++first1;
        }
        ++out;
    }
    if( first1 != last1 )
    {
        copy( first1, last1, out );
    }
    else if( first2 != last2 )
    {
        copy( first2, last2, out );
    }
}
\end{lstlisting}