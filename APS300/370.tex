\section{370 --- Range Addition}

\textbf{Medium}

Assume you have an array of length $n$ initialized with all 0's and are given $k$ update operations.

Each operation is represented as a triplet: \fcj{[startIndex, endIndex, inc]} which increments each element of subarray \fcj{A[startIndex ... endIndex]} (\fcj{startIndex} and \fcj{endIndex} inclusive) with \fcj{inc}.

Return the modified array after all $k$ operations were executed.

\paragraph{Example:}

\begin{flushleft}
\textbf{Input}: \fcj{length = 5}, \fcj{updates = [[1,3,2],[2,4,3],[0,2,-2]]}

\textbf{Output}: \fcj{[-2,0,3,5,3]}

\textbf{Explanation}:

Initial state:

\fcj{[0,0,0,0,0]}

After applying operation \fcj{[1,3,2]}:

\fcj{[0,2,2,2,0]}

After applying operation \fcj{[2,4,3]}:

\fcj{[0,2,5,5,3]}

After applying operation \fcj{[0,2,-2]}:

\fcj{[-2,0,3,5,3]}
\end{flushleft}

\subsection{Prefix Sum}
Apply update \fcj{[start, end]} with value $n$, we can update \fcj{[start, length-1]} with value $n$ and then update \fcj{[end+1, length-1]} with value $-n$.

There is only one read query on the entire range, and it occurs at the end of all update queries. Additionally, the order of processing update queries is irrelevant.

Cumulative sums operations apply the effects of past elements to the future elements in the sequence.

\setcounter{lstlisting}{0}
\begin{lstlisting}[style=customc, caption={Prefix Sum}]
vector<int> getModifiedArray( int length, vector<vector<int>>& updates )
{
    vector<int> ans( length, 0 );
    for( const auto& update : updates )
    {
        //set arr[start] = arr[start] + n
        ans[update[0]] += update[2];
        //set arr[end+1] = arr[end+1]-n
        if( update[1] < length - 1 )
        {
            ans[update[1] + 1] -= update[2];
        }
    }
    //apply cumulative sum
    // partial_sum applies the following operation (by default) for the parameters {x[0], x[n], y[0]}:
    // y[0] = x[0]
    // y[1] = y[0] + x[1]
    // y[2] = y[1] + x[2]
    // ...  ...  ...
    // y[n] = y[n-1] + x[n]
    partial_sum( begin( ans ), end( ans ), begin( ans ) );
    return ans;
}
\end{lstlisting}

\paragraph{Similar Problems}
\begin{itemize}
\item \textbf{598. Range Addition II}
\end{itemize}