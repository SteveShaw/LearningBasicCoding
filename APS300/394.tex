\section{394 --- Decode String}
Given an encoded string $S$, return it's decoded string.
\par
The encoding rule is: $k[s]$, where the $s$ inside the square brackets is being repeated exactly $k$ times. Note that $k$ is guaranteed to be a positive integer.
\par
You may assume that the input string is always valid; No extra white spaces, square brackets are well-formed, etc.
\par
Furthermore, you may assume that the original data does not contain any digits and that digits are only for those repeat numbers, k. For example, there won't be input like $3a$ or $2[4]$.

\paragraph{Examples:}

\begin{itemize}
\item \fcj{s = "3[a]2[bc]"}, return \fcj{"aaabcbc"}.
\item \fcj{s = "3[a2[c]]"}, return \fcj{"accaccacc"}.
\item \fcj{s = "2[abc]3[cd]ef"}, return \fcj{"abcabccdcdcdef"}.
\end{itemize}


\subsection{Stack}
We need two stacks: one to push a string and another for repeated times.

\setcounter{lstlisting}{0}
\begin{lstlisting}[style=customc, caption={Iterative}]
string decodeString( string s )
{
    stack<string> ss;
    stack<int> si;
    int count = 0;
    //add empty string and count 1
    //to facilitate further processing
    ss.emplace( "" );
    si.push( 1 );
    int num = 0;
    for( auto c : s )
    {
        if( c == '[' )
        {
            //add last count
            si.emplace( num );
            //add an empty string
            ss.emplace( "" );
            num = 0;
        }
        else if( c == ']' )
        {
            auto t = ss.top();
            ss.pop();
            auto count = si.top();
            si.pop();
            for( int i = 0; i < count; ++i )
            {
                ss.top() += t;
            }
        }
        else if( ( c >= '0' ) && ( c <= '9' ) )
        {
            num = num * 10 + ( c - '0' );
        }
        else
        {
            //lowercase letter
            ss.top().push_back( c );
        }
    }
    return ss.top();
}
\end{lstlisting}

