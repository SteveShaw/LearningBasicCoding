\section{353 --- Design Snake Game}

\textbf{Medium}

Design a Snake game that is played on a device with screen size equal to width $\times$ height. 

The snake is initially positioned at the top left corner \fcj{(0,0)} with length equal to 1 unit.

You are given a list of food's positions in row-column order. When a snake eats the food, its length and the game's score both increase by 1.

Each food appears one by one on the screen. For example, the second food will not appear until the first food was eaten by the snake.

When a food does appear on the screen, it is guaranteed that it will not appear on a block occupied by the snake.

\paragraph{Example:}
\begin{flushleft}

Given \fcj{width = 3}, \fcj{height = 2}, and \fcj{food = [[1,2],[0,1]]}.

\fcj{Snake snake = new Snake(width, height, food);}

Initially the snake appears at position (0,0) and the food at (1,2).

\begin{table}[H]
\begin{tabular}{|l|l|l|}
S &  &   \\
  &  & F
\end{tabular}
\end{table}

\fcc{snake.move("R"); -> Returns 0}

\begin{table}[H]
\begin{tabular}{|l|l|l|}
 & S &   \\
  &  & F
\end{tabular}
\end{table}

\fcj{snake.move("D"); -> Returns 0}

\begin{table}[H]
\begin{tabular}{|l|l|l|}
 &  &   \\
  & S & F
\end{tabular}
\end{table}

\fcj{snake.move("R"); -> Returns 1} (Snake eats the first food and right after that, the second food appears at \fcj{(0,1)} )

\begin{table}[H]
\begin{tabular}{|l|l|l|}
 & F &   \\
  & S & S
\end{tabular}
\end{table}

\fcj{snake.move("U"); -> Returns 1}

\begin{table}[H]
\begin{tabular}{|l|l|l|}
 & F & S   \\
  &  & S
\end{tabular}
\end{table}

\fcj{snake.move("L"); -> Returns 2} (Snake eats the second food)

\begin{table}[H]
\begin{tabular}{|l|l|l|}
 & S & S  \\
  &  & S
\end{tabular}
\end{table}

\fcj{snake.move("U"); -> Returns -1} (Game over because snake collides with border)
\end{flushleft}

\subsection{Double End Queue}
We can simulate the snake with a double end queue since we need to add head and remove tail at both ends.

To check if the snake will eat itself or not, we can use a auxiliary hash map to get fast inspection.

\setcounter{lstlisting}{0}
\begin{lstlisting}[style=customc, caption={Queue}]
class SnakeGame
{
public:
    /** Initialize your data structure here.
        @param width - screen width
        @param height - screen height
        @param food - A list of food positions
        E.g food = [[1,1], [1,0]] means the first food is positioned at [1,1], the second is at [1,0]. */
    SnakeGame( int width, int height, vector<vector<int>>& food )
        : m( height )
        , n( width )
        , food_( food )
        , fi{}
    {
        //we move head of snake into q
        //at start, snake length is 1.
        q.push_front( 0 );
    }
    /** Moves the snake.
        @param direction - 'U' = Up, 'L' = Left, 'R' = Right, 'D' = Down
        @return The game's score after the move. Return -1 if game over.
        Game over when snake crosses the screen boundary or bites its body. */
    int move( string direction )
    {
        //get the head of the snake
        int head = q.front();
        //get the tail of the snake
        int tail = q.back();
        //remove tail from the snake
        q.pop_back();
        body.erase( tail );
        //get (r,c) of head
        int r = head / n;
        int c = head - n * r;
        //get (r,c) of head after next move
        switch( direction[0] )
        {
        case 'U':
            r = r - 1;
            break;
        case 'D':
            r = r + 1;
            break;
        case 'L':
            c = c - 1;
            break;
        case 'R':
            c = c + 1;
            break;
        }
        int new_head = r * n + c;
        //check if the snake touch the border
        if( ( r >= 0 ) && ( r < m ) && ( c >= 0 ) && ( c < n ) )
        {
            //snake doesn't touch border
            //check if it will eat itself
            if( body.find( new_head ) != body.end() )
            {
                return -1;
            }
        }
        else
        {
            return -1;
        }
        //add new position of snake head
        q.push_front( new_head );
        body.insert( new_head );
        //check if the snake will eat food
        if( fi < food_.size() )
        {
            if( ( r == food_[fi][0] ) && ( c == food_[fi][1] ) )
            {
                //snake eat the food
                //we keep tail, add back to the end of queue
                //and to the body
                ++fi;
                q.push_back( tail );
                body.insert( tail );
            }
        }
        //the score is the current length of snake minus 1
        return ( int )q.size() - 1;
    }
private:
    int m;
    int n;
    //food index
    size_t fi;
    vector<vector<int>>& food_;
    //body has the all the coordinates of the snake's body
    unordered_set<int> body;
    //q is used to simulate the snake movement
    deque<int> q;
};

/**
 * Your SnakeGame object will be instantiated and called as such:
 * SnakeGame* obj = new SnakeGame(width, height, food);
 * int param_1 = obj->move(direction);
 */
\end{lstlisting} 