\section{352 --- Data Stream as Disjoint Interxs}
Given a data stream input of non-negative integers $ a_1 $, $ a_2 $,$ \ldots$, $ a_n $, $ \ldots $, summarize the numbers seen so far as a list of disjoint interxs.
\par
For example, suppose the integers from the data stream are $ 1, 3, 7, 2, 6, \ldots $, then the summary will be:
\begin{table}[H]
\begin{tabular}{ccc}
$ [1, 1] $ & & \\
$ [1, 1] $ & $ [3, 3] $ & \\
$ [1, 1] $ & $ [3, 3 $] & $ [7, 7] $\\
$ [1, 3] $ & $ [7, 7] $  & \\
$ [1, 3] $ & $ [6, 7] $
\end{tabular}
\end{table}


\paragraph{Follow up:}
\begin{itemize}
\item What if there are lots of merges and the number of disjoint interxs are small compared to the data stream's size?
\end{itemize}

\subsection{Binary Search}
We make use of a tree set as the underlying data structure. 

For each new number \fcj{val}, create a new range \fcj{[val, val]}. Using \fcj{lower_bound} to locate the insert position. Then, we iterate over existing ranges by merging overlapped ones.

\setcounter{lstlisting}{0}
\begin{lstlisting}[style=customc, caption={Binary Search}]
class SummaryRanges
{
public:
    /** Initialize your data structure here. */
    SummaryRanges() {}

    void addNum( int val )
    {
        if( ranges.empty() )
        {
            ranges.emplace( val, val );
            return;
        }
        auto it = ranges.lower_bound( make_pair( val, val ) );
        //check if the left range can be merged
        if( it != ranges.begin() )
        {
            it = prev( it, 1 );
            if( it->second < val - 1 )
            {
                //left range cannot be merged with [val,val]
                //move to its right
                ++it;
            }
        }
        //merge overlapped ranges
        int first = val;
        int last = val;
        while( it != ranges.end() )
        {
            auto [start, end] = *it;
            if( ( start <= val + 1 ) && ( end >= val - 1 ) )
            {
                //[start,end] can be merged with [val, val]
                //update new range's start
                first = ( min )( first, start );
                //update new range's end
                last = ( max )( last, end );
                //erase current range, jump to its next
                it = ranges.erase( it );
            }
            else
            {
                break;
            }
        }
        //add new range [first, last]
        ranges.emplace_hint( it, make_pair( first, last ) );
    }
    vector<vector<int>> getIntervals()
    {
        vector<vector<int>> ans( ranges.size(), vector<int>( 2 ) );
        int i = 0;
        for( const auto& range : ranges )
        {
            ans[i][0] = range.first;
            ans[i][1] = range.second;
            ++i;
        }
        return ans;
    }
private:
    set<pair<int, int>> ranges;
};

/**
 * Your SummaryRanges object will be instantiated and called as such:
 * SummaryRanges* obj = new SummaryRanges();
 * obj->addNum(val);
 * vector<vector<int>> param_2 = obj->getIntervals();
 */
\end{lstlisting}

\subsection{Follow Up}
\begin{itemize}
\item 如果有大量的merge操作,而disjoint interxs相对来说较少的情况下,上述binary search的算法是最佳的。
\end{itemize}