\section{307 --- Range Sum Query: Mutable}
Given an integer array $A$, find the sum of the elements between indices $i$ and $j$ ($i \leq j$), inclusive.
\par
The \texttt{update} function modifies $A$ by updating the element at index $i$ to a given value.
\begin{lstlisting}[style=customc]
void update( int i, int val )
{
}
\end{lstlisting}

\paragraph{Example:}

\begin{lstlisting}[style=customc]
//A = [1, 3, 5]
sumRange(0, 2) // 9
update(1, 2)
sumRange(0, 2) // 8
\end{lstlisting}

\paragraph{Note:}

\begin{itemize}
\item The array is only modifiable by the \texttt{update} function.
\item You may assume the number of calls to \texttt{update} and \texttt{sumRange} function is distributed evenly.
\end{itemize}

\subsection{Bit Indexed Tree}
\begin{itemize}
\item Bit Indexed Tree 的存储方式是array.
\item BIT的每个node存放的是原有数组中某些元素的sum。
\item The size of the BIT is equal to the size of the input array。In actual implementation, the size of BIT array is $n+1$ where $n$ is the size of the input array.
\end{itemize}
如果要得到sum of the subarray $A[0\ldots x]$,算法步骤如下:
\begin{enumerate}
\item Denote the sum as $y$ and $B$ as the array for BIT,从index $i = L+1$开始进行循环,其中$L$是$ A $的长度。
\item 循环中的每一步
\begin{itemize}
\item $y\gets y+B[i]$
\item 然后Go to the parent of $B[i]$,这里$ B[i] $的parent是通过removing the last 1 bit from $ i $得到的,即$i\gets i - (i\land(-i))$。
\end{itemize}
\end{enumerate}
For example: suppose $A = [2, 1, 1, 3, 2, 3, 4, 5, 6, 7, 8, 9]$,生成的BIT如下所示
\begin{figure}[H]
\begin{tikzpicture}
[my/.style={draw, circle split, fill=red!20!}]
\node[my](0){\footnotesize {$ 0 $}\nodepart{lower}{\footnotesize \texttt{Dummy}}};
\node[my](1)[below=1cm of 0, xshift=-4cm]{\footnotesize {$ 2 $}\nodepart{lower}{\footnotesize $ 1 $}};
\node[my](2)[below=1cm of 0, xshift=-2cm]{\footnotesize {$ 3 $}\nodepart{lower}{\footnotesize $ 2 $}};
\node[my](3)[below=1cm of 0, xshift=3mm]{\footnotesize {$ 7 $}\nodepart{lower}{\footnotesize $ 4 $}};
\node[my](4)[below=1cm of 0, xshift=4.3cm]{\footnotesize {$ 21 $}\nodepart{lower}{\footnotesize $ 8 $}};
\node[my](5)[below=1cm of 2, xshift=-8mm]{\footnotesize {$ 1 $}\nodepart{lower}{\footnotesize $3$}};
\node[my](6)[below=1cm of 3, xshift=-8mm]{\footnotesize {$ 2 $}\nodepart{lower}{\footnotesize $ 5 $}};
\node[my](7)[below=1cm of 3, xshift=8mm]{\footnotesize {$ 5 $}\nodepart{lower}{\footnotesize $ 6 $}};
\node[my](8)[below=1cm of 4, xshift=-1.4cm]{\footnotesize {$ 6 $}\nodepart{lower}{\footnotesize $ 9 $}};
\node[my](9)[below=1cm of 4]{\footnotesize {$ 13 $}\nodepart{lower}{\footnotesize $ 10 $}};
\node[my](10)[below=1cm of 4, xshift=1.4cm]{\footnotesize {$ 30 $}\nodepart{lower}{\footnotesize $ 12 $}};
\node[my](11)[below=1cm of 7]{\footnotesize {$ 4 $}\nodepart{lower}{\footnotesize $ 7 $}};
\node[my](12)[below=1cm of 9]{\footnotesize {$ 8 $}\nodepart{lower}{\footnotesize $ 11 $}};
\draw[line width=1pt] (0)--(1);
\draw[line width=1pt] (0)--(2);
\draw[line width=1pt] (0)--(3);
\draw[line width=1pt] (0)--(4);
\draw[line width=1pt] (2)--(5);
\draw[line width=1pt] (3)--(6);
\draw[line width=1pt] (3)--(7);
\draw[line width=1pt] (4)--(8);
\draw[line width=1pt] (4)--(9);
\draw[line width=1pt] (4)--(10);
\draw[line width=1pt] (7)--(11);
\draw[line width=1pt] (9)--(12);
\end{tikzpicture}
\end{figure}
其中
\begin{itemize}
\item $B[0]$ 是dummy node
\item $B[y]$ is the parent of $B[x]$ if $y=x-(x\land(-x))$
\item $ B[y] $的子节点$ B[x] $保存的是$A[y\ldots x]$的sum。
\end{itemize}
对于Update操作,即update index $i$ by adding a value $t$ or $A[i]\gets A[i]+t$,算法大致步骤如下
\begin{enumerate}
\item 同样从从index $i = L+1$开始进行循环,其中$L$是$ A $的长度。
\item 在循环的每一步:
\begin{itemize}
\item Update $B[i]$ as $B[i] + t$
\item Go to parent of $ B[i] $,这里$ B[i] $的parent是通过incrementing the last 1 bit of current index $ i $得到的,即$i\gets i + (i\land(-i))$。
\end{itemize}
\end{enumerate}
\paragraph{How does Binary Indexed Tree work?}
\begin{itemize}
\item The idea is based on the fact that all positive integers can be represented as the sum of powers of 2。例如19可以表示成$19=16+2+1$。
\item BIT中的每个node保存的是sum of $n$ elements where n is a power of 2。例如,在上述示意图中,the sum of the first 12 elements 可以分为两个部分
\begin{enumerate}
\item The sum of the last 4 elements (from 9 to 12)。
\item The sum of the first 8 elements (from 1 to 8)。
\end{enumerate}
\item 而number $ n$对应的二进制数中bit 1的个数是$ \lceil\log n\rceil $。因此上述两个操作最多只需要遍历$ \lceil\log n\rceil $ nodes。
\item 构建$BIT$的时间复杂度为$ O(n\log n) $,因为需要call update operation for all $ n $ elements。
\end{itemize}
相关的算法如下:
\begin{enumerate}
\item Get sum of $A[0\ldots i]$
\item Update BIT by adding the given value $ d $ to $B[i+1]$ and all of its ancestors.
\item Contruct BIT, suppose the input array $A$ has size $ n $
\end{enumerate}
运用到本题中,要注意到\texttt{Update}中,输入的$d$其实是$A[i]$的增量。而在本题中,value是$A[i]$update之后的值,所以$d$就是$ A[i] $前后的插值。
\setcounter{lstlisting}{0}
\begin{lstlisting}[style=customc, caption={BIT}]
class NumArray
{
public:
    NumArray( vector<int> nums )
    {
        int n = static_cast<int>( nums.size() );
        vector<int> tmp( n + 1, 0 );
        swap( tmp, BIT );
        swap( A, nums );
        //Construct BIT
        for( int i = 0; i < n; ++i )
        {
            updateBIT( i, A[i] );
        }
    }
    //update BIT for the changes values in A[i]
    void updateBIT( int i, int diff )
    {
        int x = i + 1;
        int n = static_cast<int>( A.size() );
        while( x <= n )
        {
            BIT[x] += diff;
            x += ( x & ( -x ) );
        }
    }
    //val is the value that A[i] will be updated to
    //not the different
    void update( int i, int val )
    {
        int diff = val - A[i];
        //Don't forget to update A[i]
        A[i] = val;
        updateBIT( i, diff );
    }
    int getSum( int i )
    {
        int x = i + 1;
        int sum = 0;
        while( x > 0 )
        {
            sum += BIT[x];
            x -= ( x & ( -x ) );
        }
        return sum;
    }
    int sumRange( int i, int j )
    {
        int s1 = getSum( i - 1 );
        int s2 = getSum( j );
        return s2 - s1;
    }


    vector<int> BIT;

    vector<int> A;
};

\end{lstlisting}


\subsection{Segment Tree}
这是range sum最典型的数据结构,其中
\begin{itemize}
\item 叶子节点是输入数组的elements.
\item Each internal node represents some merging of the leaf nodes. The merging are different for different problems. 对于本题, merging 则代表 the sum of leaves under a node.
\item Segment Trees一般用array来代表。For each node at index $i$, the left child is at index $2\times i+1$, right child at $2\times i+2$ and the parent is at $ \lfloor (i-1)/2\rfloor $ 
\item 和Heap不同,Segment Tree不是完全二叉树,只是full binary tree,即每个node有0或者2 nodes。
\end{itemize}
具体的算法包括Construct,Query和Update。
\begin{itemize}
\item Construction:
\begin{enumerate}
\item Start the whole array $A[0\ldots L-1]$ as the first segment
\item Every time, 如果当前segment的length不为1,则将其对半划分为两个部分,然后对这两个部分继续下一层递归。
\end{enumerate}
\begin{itemize}
\item 由于构建出的tree是full binary tree with $ n $ leaves,因此总共有$ n-1 $ internal nodes。这样total nodes有$ 2\times n -1 $。
\item Tree的高度为$ \lceil log_2{n}\rceil $。总共需要分配大小为$ 2\times 2^{\lceil log_2{n}\rceil}-1 $的memory。
\end{itemize}

\item Query: Suppose we want to query the range sum of $A[l\ldots r]$, and the segment tree is $T$.  Since $T[0]$ contains the range sum of $A[0,n-1]$($n$ is the size of $A$), the recursion starts with $T[0]$ for range $[0,n-1]$.
\begin{itemize}
\item Notice: In the algorithm, we don't need $A$.
\end{itemize}

\item Update:  Suppose we are updating $A[i]$ to V. 
\begin{enumerate}
\item Starts from root of the segment tree and add $ \delta:= V-A[i] $ to all nodes which have given index $ i $ in their range.
\item If a node does not have given index $ i $ in its range, no any change to this node.
\end{enumerate}

\end{itemize}
用Segment Tree作为本题的Solution的代码如下
\begin{lstlisting}[style=customc, caption={Solution Code With Segment Tree}]
class NumArray
{
public:
    NumArray( vector<int> nums )
    {
        if( nums.empty() )
        {
            return;
        }

        swap( A, nums );


        int n = static_cast<int>( A.size() );

        //This is the height of the segment tree
        int h = static_cast< int >( log2( n ) + 1 );
        //The size of the segment tree
        int sz = ( 1 << ( h + 1 ) ) - 1;

        vector<int> tmp( sz, 0 );

        swap( tmp, st );

        //Build Segment Tree
        buildUtil( 0, n - 1, 0 );
    }

    int buildUtil( int start, int end, int ti )
    {
        if( start == end )
        {
            st[ti] = A[start];
            return st[ti];
        }

        int mid = ( start + end ) / 2;
        int l = buildUtil( start, mid, 2 * ti + 1 );
        int r = buildUtil( mid + 1, end, 2 * ti + 2 );

        st[ti] = l + r;
        return st[ti];
    }


    void updateUtil( int start, int end, int a_pos, int diff, int t_pos )
    {
        if( ( a_pos < start ) || ( a_pos > end ) )
        {
            return;
        }

        st[t_pos] += diff;

        //Very important:
        //Stop update when start = end
        if( start != end )
        {
            int mid = ( start + end ) / 2;

            updateUtil( start, mid, a_pos, diff, t_pos * 2 + 1 );
            updateUtil( mid + 1, end, a_pos, diff, t_pos * 2 + 2 );
        }

    }

    void update( int i, int val )
    {
        int n = static_cast<int>( A.size() );

        if( ( i < 0 ) || ( i > n - 1 ) )
        {
            return;
        }

        int diff = val - A[i];

        A[i] = val;

        updateUtil( 0, n - 1, i, diff, 0 );
    }

    int getSum( int start, int end, int i, int j, int ti )
    {
        //if the given range [i,j] include [start, end]
        //return the number in st[ti]
        if( ( start >= i ) && ( end <= j ) )
        {
            return st[ti];
        }

        //[i,j] does not overlap with [start, end]
        if( ( start > j ) || ( end < i ) )
        {
            //Return 0 since we cannot get
            //any value for this range
            return 0;
        }

        int mid = ( start + end ) / 2;

        int l = getSum( start, mid, i, j, ti * 2 + 1 );
        int r = getSum( mid + 1, end, i, j, ti * 2 + 2 );

        return l + r;
    }

    int sumRange( int i, int j )
    {
        int n = static_cast<int>( A.size() );

        if( ( j < 0 ) || ( i > n - 1 ) )
        {
            return -1;
        }


        return getSum( 0, n - 1, i, j, 0 );
    }

    vector<int> st;
    vector<int> A;
};

\end{lstlisting}

\subsection{Iterative Segment Tree}
上述代码中,Segment Tree的内存空间有很多浪费,实际上可以把Segment Tree的内存压缩的更多,这样就是iterative segment tree的实现方法
\begin{itemize}
\item 这种方法中,segment tree的array的大小为$2\times N$,其中$N$是输入数组的大小。

\item The leaf nodes start from index $N$ and go up to index $ 2\times N -1 $。因此,$A[i]$在segment tree array中的index为$ i+N $
\item 如果要得到某个element的parents,就要从index $ N-1 $出发,move upward。 To get the parents, we will start from index $N – 1$ and move upward. 

\item For an index $ i $, its left child is at $ 2\times i $ and right child will be at $ 2\times i + 1 $ index。 So the values at nodes $ 2\times i $ and $ 2\times i + 1 $ is combined at $i$th node to construct the tree.
\end{itemize}

\begin{lstlisting}[style=customc, caption={Iteratively Segement Tree}]
class NumArray
{
public:
    NumArray( vector<int> nums )
    {
        if( nums.empty() )
        {
            return;
        }

        swap( A, nums );

        vector<int> tmp( A.size() * 2, 0 );
        swap( tmp, T );

        //build segment tree
        build();
    }

    void build()
    {
        auto n = static_cast<int>( A.size() );

        //set leaf nodes
        for( int i = 0; i < n; ++i )
        {
            T[n + i] = A[i];
        }

        //set parent nodes
        for( int i = n - 1; i > 0; --i )
        {
            T[i] = T[i * 2] + T[i * 2 + 1];
        }
    }

    void update( int i, int val )
    {
        auto n = static_cast<int>( A.size() );

        //update leaf node
        T[i + n] = val;

        int x = i + n;

        //update parent nodes
        //since T[0] is dummy node
        //the loop runs until x=1
        while( x > 1 )
        {
            T[x / 2] = T[x] + T[x ^ 1];
            x = x / 2;
        }
    }

    int sumRange( int i, int j )
    {
        auto n = static_cast<int>( A.size() );

        int x = i + n;
        int y = j + n;

        int sum = 0;

        while( x <= y )
        {
            if( x & 1 )
            {
                //x is the right child of its parent
                sum += T[x];
                //move x to the right of its parent node
                //through x/=2 outside
                ++x;
            }

            if( ( y & 1 ) == 0 )
            {
                //y is the left child of its parent
                sum += T[y];
                //move y to the left of its parent node
                //through y/=2 outside
                --y;
            }

            x /= 2;
            y /= 2;
        }

        return sum;
    }

    vector<int> A;
    vector<int> T;
};


\end{lstlisting}