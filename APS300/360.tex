\section{360 --- Sort Transformed Array}

\textbf{Medium}

Given a sorted array of integers \fcj{nums} and integer values $a$, $b$ and $c$. Apply a quadratic function of the form $f(x) = ax^2 + bx + c$ to each element $x$ in the array.

The returned array must be in sorted order.

Expected time complexity: $O(n)$

\paragraph{Example 1:}

\begin{flushleft}
\textbf{Input}: \fcj{nums = [-4,-2,2,4]}, $a = 1$, $b = 3$, $c = 5$

\textbf{Output}: \fcj{[3,9,15,33]}
\end{flushleft}

\paragraph{Example 2:}

\begin{flushleft}
\textbf{Input}: \fcj{nums = [-4,-2,2,4]}, $ a = -1 $, $ b = 3 $, $ c = 5 $

\textbf{Output}: \fcj{[-23,-5,1,7]}
\end{flushleft}

\subsection{Two Pointers}
From math, we know 

if $a \geq 0$, $f(x)$ reaches minimum in the middle, so we fill the result array from back to front. 

if $a < 0$, $f(x)$ reaches maximum value in the middle, so we go with opposite way.

We make use of two pointers to point to front and back of input array, 

when $a\geq 0$, we compare the value of $f$ and put the greater one at current position in the result array.

Otherwise, we put the smaller one at current position in the result array.

\setcounter{lstlisting}{0}
\begin{lstlisting}[style=customc, caption={Two Pointers}]
vector<int> sortTransformedArray( vector<int>& nums, int a, int b, int c )
{
    //a >=0 : fill from back to front
    //a < 0 : fill from front to back
    size_t fill = ( a >= 0 ) ? nums.size() - 1 : 0;
    //two pointers: l and r
    size_t l{};
    size_t r = nums.size() - 1;
    auto f = [a, b, c]( int x )
    {
        return a * x * x + b * x + c;
    };
    vector<int> ans( nums.size() );
    if( a >= 0 )
    {
        while( l <= r )
        {
            int fl = f( nums[l] );
            int fr = f( nums[r] );
            //put greater one in current position
            if( fl >= fr )
            {
                ans[fill] = fl;
                ++l;
            }
            else
            {
                ans[fill] = fr;
                --r;
            }

            --fill;
        }
    }
    else
    {
        while( l <= r )
        {
            int fl = f( nums[l] );
            int fr = f( nums[r] );
            //put smaller one in current position
            if( fl <= fr )
            {
                ans[fill] = fl;
                ++l;
            }
            else
            {
                ans[fill] = fr;
                --r;
            }
            ++fill;
        }
    }
    return ans;
}
\end{lstlisting}

\paragraph{Similar Problems}
\begin{itemize}
\item \textbf{977. Squares of a Sorted Array}
\end{itemize}