\section{357. Count Numbers with Unique Digits}
Given a non-negative integer $ n $, count all numbers with unique digits, $ x $, where $0 \leq x < 10^n$.

\paragraph{Example:}

\begin{flushleft}
\textbf{Input}: 2
\\
\textbf{Output}: 91 
\\
\textbf{Explanation}: The answer should be the total numbers in the range of $0 \leq x < 100$, excluding 11,22,33,44,55,66,77,88,99
\end{flushleft}

\subsection{Permutation}
\begin{enumerate}
\item 除了个位数外,每个数字的最高位不能为零,因此最高位数字可以从1--9中任选一个,总共有9种方式。
\item 剩下的数字则可以从 0--9中 选择与最高位不同的数字中,也即从剩下的9个数字中选取$l-1$个进行全排列(l是数字的位数)。
\item 因此对于长度为$l$的数字,产生的unique digits的个数为$9\times P(9, l-1)$。
\item 对位数从1到$ n $累加即可得到总数。
\end{enumerate}

\setcounter{lstlisting}{0}
\begin{lstlisting}[style=customc, caption={Permutation}]
int countNumbersWithUniqueDigits( int n )
{
    int ans = 1;

    for( int l = 1; l <= n; ++l )
    {
        //select one number for the highest position
        //then permutation for remaining l-1
        ans += 9 * P( 9, l - 1 );
    }

    return ans;
}

//get permutation of P(m,n): select n unique elements
//from m items for permutation
int P( int m, int n )
{
    int ans = 1;

    for( int i = 0; i < n; ++i )
    {
        ans *= ( m - i );
    }

    return ans;
}
\end{lstlisting}