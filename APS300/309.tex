\section{309 --- Best Time to Buy and Sell Stock with Cooldown}
Say you have an array $A$ for which the $ i $th element is the price of a given stock on day $ i $.
\par
Design an algorithm to find the maximum profit. You may complete as many transactions as you like (ie, buy one and sell one share of the stock multiple times) with the following restrictions:
\par
You may not engage in multiple transactions at the same time (ie, you must sell the stock before you buy again).
\par
After you sell your stock, you cannot buy stock on next day. (ie, cooldown 1 day)

\paragraph{Example:}

\begin{flushleft}
\textbf{Input}: $ [1,\;2,\;3,\;0,\;2] $
\\
\textbf{Output}: 3 
\\
\textbf{Explanation}: transactions = [\texttt{buy}, \texttt{sell}, \texttt{cooldown}, \texttt{buy}, \texttt{sell}]
\end{flushleft}
\subsection{Dynamic Programming}
在day $ i $,可以有如下四种选项
\begin{enumerate}
\item 手头有一个stock,然后卖出。
\item 手头有一个stock,然后什么都不做。
\item 手头没有stock,然后买入一股。
\item 手头没有stock,然后什么也不做。
\end{enumerate}
而这四个options在day $ i-1 $ 和 day $ i $是相关联的。
\begin{enumerate}
\item Choose option 1 in day $ i $,那么在 day $ i-1 $,可以 take option 2 和 3。
\item Choose option 2 on day $ i $,那么在 day $ i-1 $,可以 take option 2 和 3。
\item Choose option 3 on day $ i $,那么在 day $ i-1 $,只能 take option 4。因为有cooldown,所以不能够选择 option 1。
\item Choose option 3 on day $ i $,那么在 day $ i-1 $,可以 take option 1 和 4。
\end{enumerate}
由于不知道哪个option会带来最大的利润,因此需要尝试所有四个选项,然后选择能够带来最大利润的那个选项。例如,假设在day $ i $,选择了option 1。那么在day $ i-1 $,就要比较option 2和3哪个可以在 day $ i $ 产生最大的利润,这就需要靠DP方法来逐步计算。
\par
最开始在day 0时,各个选项的导致的profit的初始值如下
\begin{enumerate}
\item 由于最开始没有股票,需要买入,花费$A[0]$,而在当天又卖出,得到$A[0]$,这样实际的profit为0。
\item 由于最开始没有股票,需要买入,花费$A[0]$,然后什么都不做,因此profit 为$-A[0]$。
\item 与2一样,买入股票,花费$A[0]$,profit也是$-A[0]$。
\item 由于没有transaction,因此proift为0。
\end{enumerate}
在最后一天这四个option的最大值即为所能获得的最大profit。
\setcounter{lstlisting}{0}
\begin{lstlisting}[style=customc, caption={Dynamic Programming}]
int maxProfit( vector<int>& prices )
{
    if( prices.empty() || ( prices.size() == 1 ) )
    {
        return 0;
    }

    //H[0]: have stock and sell, H[1]: have stock and do nothing
    vector<int> H( 2, 0 );
    //U[0]: no stock and buy, U[1]: no stock and do nothing
    vector<int> U( 2, 0 );

    H[1] = -prices[0];//we must buy the stock and hold
    U[0] = -prices[0];//we must buy the stock


    for( size_t i = 1; i < prices.size(); ++i )
    {
        //save last day's values
        int tmp_h0 = H[0];
        int tmp_h1 = H[1];
        int tmp_u0 = U[0];
        int tmp_u1 = U[1];

        //The iterative relationship
        H[0] = ( max )( tmp_u0 + prices[i], tmp_h1 + prices[i] ); //1<-(2,3)
        H[1] = ( max )( tmp_u0, tmp_h1 ); //2<-(2,3)

        U[0] = tmp_u1 - prices[i]; //3<-4
        U[1] = ( max )( tmp_h0, tmp_u1 ); //4<-(1,4)
    }

    //The maximum profit is
    //the maximum value of these 4 variables
    int ans = max( H[0], H[1] );
    ans = ( max )( ans, U[0] );
    ans = ( max )( ans, U[1] );

    return ans;
}
\end{lstlisting}

