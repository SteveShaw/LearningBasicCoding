\section{376 --- Wiggle Subsequence}
A sequence of numbers is called a wiggle sequence if the differences between successive numbers strictly alternate between positive and negative. The first difference (if one exists) may be either positive or negative. A sequence with fewer than two elements is trivially a wiggle sequence.
\par
For example, $ [1,7,4,9,2,5] $ is a wiggle sequence because the differences $ (6,-3,5,-7,3) $ are alternately positive and negative. In contrast, $ [1,4,7,2,5] $ and $ [1,7,4,5,5] $ are not wiggle sequences, the first because its first two differences are positive and the second because its last difference is zero.
\par
Given a sequence of integers $A$, return the length of the longest subsequence that is a wiggle sequence. A subsequence is obtained by deleting some number of elements (eventually, also zero) from the original sequence, leaving the remaining elements in their original order.

\paragraph{Example 1:}

\begin{flushleft}
\textbf{Input}: $ [1,7,4,9,2,5] $
\\
\textbf{Output}: 6
\\
\textbf{Explanation}: The entire sequence is a wiggle sequence.
\end{flushleft}

\paragraph{Example 2:}

\begin{flushleft}
\textbf{Input}: $ [1,17,5,10,13,15,10,5,16,8] $
\\
\textbf{Output}: 7
\\
\textbf{Explanation}: There are several subsequences that achieve this length. One is $ [1,17,10,13,10,16,8] $.
\end{flushleft}

\paragraph{Example 3:}

\begin{flushleft}
\textbf{Input}: $ [1,2,3,4,5,6,7,8,9] $
\\
\textbf{Output}: 2
\end{flushleft}

\paragraph{Follow up:}
\begin{itemize}
\item Can you do it in $ O(n) $ time?
\end{itemize}

\subsection{Linear Dynamic Programming}
采用Dynamic Programming,利用两个递推数组$U$ and $D$。
\begin{enumerate}
\item $U[i]$ is the length of a longest wiggle subsequence of $A[0],\ldots,A[i]$, with a \textbf{positive} difference between its last two numbers. This subsequence may or may not include $A[i]$ and there may be several such subsequences (of the same length). 称这种类型的subsequence为$U$.
\item $D[i]$ is the length of a longest wiggle subsequence of $A[0],\ldots,A[i]$, with a \textbf{negative} difference between its last two numbers. This subsequence may or may not include $A[i]$ and there may be several such subsequences (of the same length). 称这种类型的subsequence为$D$.
\end{enumerate}
然后可以得到如下的递推关系式
\begin{enumerate}
%那么可以用$A[i]$构成一个更长的subsequence $U$
\item 如果$A[i]>A[i-1]$,假设有一个类型为$D$的subsequence $S_d$ from $A[0],\ldots,A[i-1]$,其length为$D[i-1]$,而$A[x]$ is the last element of $S_d$。那么有两种情况
\begin{enumerate}
\item If $A[i] > A[x]$,显然可以把$A[i]$加入$S_d$从而构成一个类型为$U$的subsequence。
\item Otherwise, since $A[x]$ cannot be $A[i-1]$ (because $A[i-1] < A[i]\leq A[x]$), replace $A[x]$ by $A[i-1]$ gives another type $D$ subsequence $\hat{S}_d$ with same length. And then we can also add $A[i]$ to $\hat{S}_d$ to get a subsequence of type $U$.
\end{enumerate}
Both cases give $U[i]=D[i-1]+1$,而$D[i]$显然是不变的,所以$D[i] = D[i-1]$
\item 类似的,如果$A[i] < A[i-1]$,可以得到$D[i] = U[i-1]+1$,$U[i]=U[i-1]$。
\item 如果$A[i]=A[i-1]$,显然不会改变已有的type $U$和$D$ subsequences,因此$U[i]=U[i-1]$, $D[i]=D[i-1]$
\end{enumerate}

\setcounter{lstlisting}{0}
\begin{lstlisting}[style=customc, caption={Dynamic Programming}]
int wiggleMaxLength( vector<int>& nums )
{
    if( nums.empty() )
    {
        return 0;
    }

    vector<int> U( nums.size(), 1 );
    vector<int> D( nums.size(), 1 );

    for( size_t i = 1; i < nums.size(); ++i )
    {
        if( nums[i] > nums[i - 1] )
        {
            U[i] = D[i - 1] + 1;
            D[i] = D[i - 1];
        }
        else if( nums[i] < nums[i - 1] )
        {
            //D can be longer by adding nums[i]
            //to subsquence U
            D[i] = U[i - 1] + 1;
            U[i] = U[i - 1];
        }
        else
        {
            //no changes to the lengths of U and D
            U[i] = U[i - 1];
            D[i] = D[i - 1];
        }
    }

    return ( max )( U.back(), D.back() );
}
\end{lstlisting}

\subsection{O(1) Memory Dynamic Programming}
从上面的递推式可以看到,$U[i]$和$D[i]$仅仅取决于$U[i-1]$和$D[i-1]$,因此可以把数组替换为两个变量$U$和$D$。
\begin{lstlisting}[style=customc, caption={$O(1)$ Memory Dynamic Programming}]
int wiggleMaxLength( vector<int>& nums )
{
    if( nums.empty() )
    {
        return 0;
    }
    //the maximum length of type U subsequence so far
    int U = 1;
    //the maximum length of type D subsequence so far
    int D = 1;
    for( size_t i = 1; i < nums.size(); ++i )
    {
        if( nums[i] > nums[i - 1] )
        {
            //nums[i] can make a longer U subsequence
            U = D + 1;
        }
        else if( nums[i] < nums[i - 1] )
        {
            //nums[i] can make a longer D subsequence
            D = U + 1;
        }
    }
    return ( max )( U, D );
}
\end{lstlisting}