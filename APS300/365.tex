\section{365 --- Water and Jug Problem}
You are given two jugs with capacities $ x $ and $ y $ litres. There is an infinite amount of water supply available. You need to determine whether it is possible to measure exactly $ z $ litres using these two jugs.

If $ z $ liters of water is measurable, you must have $ z $ liters of water contained within one or both buckets by the end.

Operations allowed:

\begin{itemize}
\item Fill any of the jugs completely with water.
\item Empty any of the jugs.
\item Pour water from one jug into another till the other jug is  completely full or the first jug itself is empty.
\end{itemize}

\paragraph{Example 1: }

\begin{flushleft}
\textbf{Input}: $ x = 3 $, $ y = 5 $, $ z = 4 $
\\
\textbf{Output}: \texttt{True}
\end{flushleft}

\paragraph{Example 2:}

\begin{flushleft}
\textbf{Input}: $ x = 2 $, $  y = 6 $,$  z = 5 $
\\
\textbf{Output}: \texttt{False}
\end{flushleft}

\subsection{Number Theory}
The basic idea is to use the property of B\'{e}zout's identity and check if $ z $ is a multiple of greatest common divisor of $ x $ and $ y $。  B\'{e}zout's identity (also called  B\'{e}zout's lemma) is a theorem in the elementary theory of numbers:

let $ a $ and $ b $ be \textbf{nonzero} integers and let $ d $ be their greatest common divisor. Then there exist integers $ x $ and $ y $ such that $ ax+by=d $

In addition, the greatest common divisor $ d $ is the smallest positive integer that can be written as $ ax + by $

every integer of the form $ ax + by $ is a multiple of the greatest common divisor $ d $.
\begin{enumerate}
\item 注意,上述定理中$ a $和$ b $都是非零整数。
\item 因此,需要注意边界条件 $ x $,$ y $或者$ z $为零的时候。
\item 另外,题目中还提到,如果能得到$z$升水,那么最后这$z$升水是在两个jugs中的一个或者两个中。因此如果$x+y<z$,那么无论如何也不能满足这个条件。
\end{enumerate}

\setcounter{lstlisting}{0}
\begin{lstlisting}[style=customc, caption={GCD}]
bool canMeasureWater( int x, int y, int z )
{
    //one or two jugs
    //must hold z liter water
    if( x + y < z )
    {
        return false;
    }

    //edge case especially for z=0
    if( x == z || y == z || ( x + y ) == z )
    {
        return true;
    }

    //get gcd
    int d = gcd( x, y );

    return ( z % d == 0 );
}

int gcd( int a, int b )
{
    if( b == 0 )
    {
        return a;
    }

    return gcd( b, a % b );
}
\end{lstlisting}
