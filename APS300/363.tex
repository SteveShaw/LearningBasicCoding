\section{363 --- Max Sum of Rectangle No Larger Than K}
Given a non-empty 2D matrix matrix $ M $ and an integer $ k $, find the max sum of a rectangle in the matrix such that its sum is no larger than $ k $.

\paragraph{Example:}

\begin{flushleft}
\textbf{Input}: 
\begin{table}[H]
\begin{tabular}{ccc}
1 & 0 & 1\\
0 & -2 & 3
\end{tabular}
\end{table}
and $ k=2 $
\\
\textbf{Output}: 2 
\\
\textbf{Explanation}: Because the sum of rectangle 
\begin{table}[H]
\begin{tabular}{cc}
0 & 1\\
-2 & 3
\end{tabular}
\end{table}
is 2, and 2 is the max number no larger than $ k $ ($ k = 2 $).
\end{flushleft}

\paragraph{Note:}

\begin{itemize}
\item The rectangle inside the matrix must have an area larger than 0.
\item What if the number of rows is much larger than the number of columns?
\end{itemize}

\subsection{2D kadane}
\begin{enumerate}
\item 将$ M $按照行拆成多个一维数组,然后利用一维数组的累加和来找符合要求的数字。具体做法是iterate over columns。对于current column $ x $,从$x$遍历到$n-1$($ n $ is the number of columns),然后在$[x,n-1]$中的每一个column $y$,计算每一行的累加和,放入一个大小为$m$的数组$z$中($ m $ is the number of rows)。即对于current row $ r $, $z[r]=\sum\limits_{i=x}^{y}M[r][i]$。
\item After getting the sum of each row, we apply 1D kadane first to see if we can reach a maximum sum which is no larger than $k$. If the maximum sum is larger than $k$, we will go to next.
\item 然后在$z$中计算prefix sum,并寻找两个row $u$和$v$使得$z[u]-z[v]\leq k$ 并且$z[u]-z[v]$又是目前最大的difference。这样在column $y$就找到了一个具有最大sum的rectange区域,左上角位$(u, x)$,右下角index为$(v, y)$。
\item 为了能够快速定位,将$z$的prefix sum放入一个tree set $s$中,然后对于对于当前的prefix sum $t$,在$s$中binary search到$p$,使得$t\leq k+p$,即leftmost binary search。
\item 最开始$s$需要放入一个零,这样,当然prefix sum和$k$相等时,即得到以第一行开始的prefix sum。而这也是符合条件的。
\end{enumerate}

\setcounter{lstlisting}{0}
\begin{lstlisting}[style=customc, caption={2D Kadane}]
int maxSumSubmatrix( vector<vector<int>>& matrix, int k )
{
    int m = static_cast<int>( matrix.size() );
    int n = static_cast<int>( matrix[0].size() );

    int ans = INT_MIN;

    for( int c = 0; c < n; ++c )
    {
        //postfix sum
        vector<int> z( m, 0 );

        for( int x = c; x < n; ++x )
        {
            for( int r = 0; r < m; ++r )
            {
                z[r] += matrix[r][x];
            }

            set<int> s;

            s.insert( 0 );

            //prefix sum in z
            int t = 0;

            for( int y : z )
            {
                t += y;

                //find the first item such that
                //t-*it <= k
                auto it = s.lower_bound( t - k );

                if( it != s.end() )
                {
                    ans = ( max )( ans, t - *it );
                }

                s.insert( t );
            }
        }
    }

    return ans;
}
\end{lstlisting}