\section{336 --- Palindrome Pairs}
Given a list of \textbf{unique} words, find all pairs of distinct indices $ (i, j) $ in the given list, so that the concatenation of the two words, i.e. \fcj{words[i] + words[j]} is a palindrome.

\paragraph{Example 1:}

\begin{flushleft}
\textbf{Input}:\fcj{["abcd","dcba","lls","s","sssll"]}

\textbf{Output}: \fcj{[[0,1],[1,0],[3,2],[2,4]]}  

\textbf{Explanation}: The palindromes are \fcj{["dcbaabcd","abcddcba","slls","llssssll"]}
\end{flushleft}

\paragraph{Example 2:}

\begin{flushleft}
\textbf{Input}: \fcj{["bat","tab","cat"]}

\textbf{Output}: \fcj{[[0,1],[1,0]]} 

\textbf{Explanation}:The palindromes are \fcj{["battab","tabbat"]}
\end{flushleft}

\subsection{Hash Map}
Maintain a hash map which has map between each reversed word and its index.

Then, iterate original array, and for each word, split into left and right part at each index. Check if any part is a palindrome, and if another part can be found in the hash map.



\setcounter{lstlisting}{0}
\begin{lstlisting}[style=customc, caption={Hash Map}]
vector<vector<int>> palindromePairs( vector<string>& words )
{
    unordered_map<string_view, int> m;
    //to make hash map has string_view key
    //we need another array to store the reversed word
    vector<string> rev_words( words );
    int wi = 0;
    //empties save the index of word which is empty
    vector<int> empties;
    for( auto& word : rev_words )
    {
        if( word.empty() )
        {
            empties.push_back( wi++ );
            continue;
        }
        reverse( begin( word ), end( word ) );
        m.emplace( word, wi++ );
    }
    wi = 0;
    vector<vector<int>> ans;
    //helper function to check if a string s is a palindrome
    auto is_palindrome = []( string_view s )
    {
        return equal( begin( s ), begin( s ) + s.size() / 2, rbegin( s ) );
    };
    for( const auto& word : words )
    {
        auto it = m.find( word );
        if( ( it != m.end() ) && ( it->second != wi ) )
        {
            //found a pair
            ans.emplace_back( vector<int> {wi, it->second} );
        }
        //if word is palindrome and there are empty words
        //any pair of them is also a palindrome
        if( is_palindrome( word ) && !empties.empty() )
        {
            for( int j : empties )
            {
                if( j != wi )
                {
                    ans.emplace_back( vector<int> {wi, j} );
                    ans.emplace_back( vector<int> {j, wi} );
                }
            }
        }
        //check substr at each part
        string_view w = word;
        for( size_t i = 1; i < w.size(); ++i )
        {
            auto left = w.substr( 0, i );
            auto right = w.substr( i );
            it = m.find( left );
            if( ( it != m.end() ) && is_palindrome( right ) && ( it->second != wi ) )
            {
                //wi + wi[0,i] is a palindrome
                ans.emplace_back( vector<int> {wi, it->second} );
            }
            it = m.find( right );
            if( ( it != m.end() ) && is_palindrome( left ) && ( it->second != wi ) )
            {
                //wi[i,..] + wi is a palindrome
                ans.emplace_back( vector<int> {it->second, wi} );
            }
        }
        ++wi;
    }
    return ans;
}
\end{lstlisting}

\paragraph{Related Problems}
\begin{itemize}
\item \textbf{5. Longest Palindromic Substring}
\item \textbf{214. Shortest Palindrome}
\end{itemize}


