\section{338 --- Counting Bits}

\textbf{Medium}

Given a non negative integer number $n$. For every numbers $i$ in the range $0 \leq i \leq n$ calculate the number of 1's in their binary representation and return them as an array.

\paragraph{Example 1:}

\begin{flushleft}
\textbf{Input}: 2

\textbf{Output}: \fcj{[0,1,1]}
\end{flushleft}

\paragraph{Example 2:}

\begin{flushleft}
\textbf{Input}: 5

\textbf{Output}: \fcj{[0,1,1,2,1,2]}
\end{flushleft}

\paragraph{Follow up:}

\begin{itemize}
\item It is very easy to come up with a solution with run time \fcj{O(n*sizeof(int))}. But can you do it in linear time $O(n)$ /possibly in a single pass?
\item Space complexity should be $O(n)$.
\item Can you do it like a boss? Do it without using any builtin function like\fcc{ __builtin_popcount} in \fcc{c++} or in any other language.
\end{itemize}

\subsection{Dynamic Programming}

Suppose we have an integer:
\[
x = (1001011101)_2 = (605)_{10}
\]

and we already calculated and stored all the results of 0 to $x - 1$.

Then we know that $x$ is differ by rightmost bit with a number we already calculated:

\[
\hat{x} = (1001011100)_2 = (604)_{10}
\]

\setcounter{lstlisting}{0}
\begin{lstlisting}[style=customc, caption={DP}]
vector<int> countBits( int num )
{
    vector<int> F( num + 1, 0 );
    for( int i = 1; i <= num; ++i )
    {
        //find the number that is differ with
        //i by rightmost bit 1
        F[i] = F[i & ( i - 1 )] + 1;
    }
    return F;
}
\end{lstlisting}

