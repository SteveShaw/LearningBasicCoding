\section{439 --- Ternary Expression Parser}
\textcolor{red}{LOCKED} 

Given a string representing arbitrarily nested ternary expressions, calculate the result of the expression. You can always assume that the given expression is valid and only consists of digits 0--9, ?, :, $T$ and $F$ ($T$ and $F$r epresent \texttt{True} and \texttt{False} respectively).

\paragraph{Note:}

\begin{itemize}
\item The length of the given string is $\leq 10000$.
\item Each number will contain only one digit.
\item The conditional expressions group right-to-left (as usual in most languages).
\item The condition will always be either \texttt{T} or \texttt{F}. That is, the condition will never be a digit.
\item The result of the expression will always evaluate to either a digit 0--9, \texttt{T} or \texttt{F}.
\end{itemize}
 

\paragraph{Example 1:}

\begin{flushleft}
\textbf{Input}: $T?2:3$

\textbf{Output}: 2

\textbf{Explanation}: If \texttt{true}, then result is 2; otherwise result is 3.
\end{flushleft}
 

\paragraph{Example 2:}

\begin{flushleft}
\textbf{Input}: $F?1:T?4:5$

\textbf{Output}: 4

\textbf{Explanation}: The conditional expressions group right-to-left. Using parenthesis, it is read/evaluated as:

$(F ? 1 : (T ? 4 : 5)) \longrightarrow (F ? 1 : 4) \longrightarrow 4$

or

$(F ? 1 : (T ? 4 : 5)) \longrightarrow (T ? 4 : 5) \longrightarrow 4$
 
\end{flushleft}


\paragraph{Example 3:}

\begin{flushleft}
\textbf{Input}: $T?T?F:5:3$

\textbf{Output}: $F$

\textbf{Explanation}: The conditional expressions group right-to-left. Using parenthesis, it is read/evaluated as:

$(T ? (T ? F : 5) : 3) \longrightarrow (T ? F : 3) \longrightarrow F$

or

$(T ? (T ? F : 5) : 3) \longrightarrow (T ? F : 5) \longrightarrow F$
\end{flushleft}

\subsection{Stack}
\begin{itemize}
\item 将expression中每个问号的index放入一个栈中,这样栈顶的问号就是表达式中最右边的问号。
\item 然后开始处理栈,每次根据顶端的index $p$,处理$s[p-1...p+1]$段的三元表达式。然后压缩expression,因为这段表达式已经处理了。
\item 将表达式的结果放入$s[p-1]$,同时将$s[p+4...L-1]$的部分往前移动4位,即$s[i-4]=s[i]$。
\item 上述移动不会影响栈中的index位置,因为这些index都位于$p-1$之前。
\end{itemize}

\setcounter{lstlisting}{0}
\begin{lstlisting}[style=customc, caption={Stack}]
string parseTernary( string expression )
{
    stack<size_t> stk;

    //push each ?'s index into stack
    for( size_t i = 0; i < expression.size(); ++i )
    {
        if( expression[i] == '?' )
        {
            stk.push( i );
        }
    }

    while( !stk.empty() )
    {
        //evaluate expression based on current position p
        //expression[p-1] is T/F
        //expression[p+1...p+3] is the ternary expression
        auto p = stk.top();
        stk.pop();

        char x = expression[p - 1];
        char y = expression[p + 1];
        if( x == 'F' )
        {
            y = expression[p + 3];
        }

        //compress the string
        //remove from s[p...p+3]
        //set s[p-1] to the result
        expression[p - 1] = y;

        //move expression[p+4...] backward by 4 positions
        for( size_t i = p + 4; i < expression.size(); ++i )
        {
            expression[i - 4] = expression[i];
        }

        //remove unnecessary parts
        expression.resize( expression.size() - 3 );
    }

    string ans;
    ans.push_back( expression[0] );
    return ans;
}
\end{lstlisting}