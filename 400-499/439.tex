\section{439 --- Ternary Expression Parser}

\textbf{Medium}

Given a string representing arbitrarily nested ternary expressions, calculate the result of the expression. You can always assume that the given expression is valid and only consists of digits 0-9, \fcj{?}, \fcj{:}, \fcj{T} and \fcj{F} (\fcj{T} and \fcj{F} represent \fcj{True} and \fcj{False} respectively).

\paragraph{Note:}

\begin{itemize}
\item The length of the given string is $\leq 10000$.
\item Each number will contain only one digit.
\item The conditional expressions group right-to-left (as usual in most languages).
\item The condition will always be either \texttt{T} or \texttt{F}. That is, the condition will never be a digit.
\item The result of the expression will always evaluate to either a digit 0--9, \fcj{T} or \fcj{F}.
\end{itemize}
 

\paragraph{Example 1:}

\begin{flushleft}
\textbf{Input}: \fcj{"T?2:3"}

\textbf{Output}: \fcj{"2"}

\textbf{Explanation}: If \fcc{true}, then result is 2; otherwise result is 3.
\end{flushleft}
 

\paragraph{Example 2:}

\begin{flushleft}
\textbf{Input}: \fcj{"F?1:T?4:5"}

\textbf{Output}: \fcj{"4"}

\textbf{Explanation}: The conditional expressions group right-to-left. Using parenthesis, it is read/evaluated as:

\fcj{"(F ? 1 : (T ? 4 : 5))" -> "(F ? 1 : 4)" -> "4"}

or 

\fcj{"F ? 1 : (T ? 4 : 5))" -> "(T ? 4 : 5)" -> "4"}
\end{flushleft}


\paragraph{Example 3:}

\begin{flushleft}
\textbf{Input}: \fcj{"T?T?F:5:3"}

\textbf{Output}: \fcj{"F"}
\textbf{Explanation}: The conditional expressions group right-to-left. Using parenthesis, it is read/evaluated as:

\fcj{"(T ? (T ? F : 5) : 3)" -> "T ? F : 3)" -> "F"}

or

\fcj{"(T ? (T ? F : 5) : 3)" -> "(T ? F : 5)" -> "F"}
\end{flushleft}

\subsection{Stack}
\begin{itemize}
\item 将expression中每个问号的index放入一个栈中,这样栈顶的问号就是表达式中最右边的问号。
\item 然后开始处理栈,每次根据顶端的index $p$,处理$s[p-1...p+1]$段的三元表达式。然后压缩expression,因为这段表达式已经处理了。
\item 将表达式的结果放入$s[p-1]$,同时将$s[p+4...L-1]$的部分往前移动4位,即$s[i-4]=s[i]$。
\item 上述移动不会影响栈中的index位置,因为这些index都位于$p-1$之前。
\end{itemize}

\setcounter{lstlisting}{0}
\begin{lstlisting}[style=customc, caption={Stack}]
string parseTernary( string expression )
{
    stack<size_t> stk;
    //add each position of symbol ? into the stack
    size_t idx = 0;
    for( auto c : expression )
    {
        if( c == '?' )
        {
            stk.push( idx );
        }
        ++idx;
    }
    while( !stk.empty() )
    {
        auto idx = stk.top();
        stk.pop();
        //get condition
        auto c_cond = expression[idx - 1];
        //get value per c_cond
        auto c_val = expression[idx + 1];
        if( c_cond == 'F' )
        {
            c_val = expression[idx + 3];
        }
        expression[idx - 1] = c_val;
        //compress expression
        for( auto i = idx + 4; i < expression.size(); ++i )
        {
            expression[i - 4] = expression[i];
        }
        expression.resize( expression.size() - 4 );
    }
    return expression;
}
\end{lstlisting}

\subsection{Related Problems}
\begin{itemize}
\item \textbf{385. Mini Parser}
\item \textbf{722. Remove Comments}
\item \textbf{736. Parse Lisp Expression}
\end{itemize}