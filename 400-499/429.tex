\section{429 --- N-ary Tree Level Order Traversal}
Given an $n$-ary tree, return the level order traversal of its nodes' values. (ie, from left to right, level by level).

For example, given a 3-ary tree:

\begin{figure}[H]
\begin{tikzpicture}
[my/.style={draw, circle, minimum size=7mm, fill=gray!20!}]
\node[my](0) at (0,0) {1};
\node[my](1) [below=8mm of 0, xshift=-12mm] {3};
\node[my](2) [below=8mm of 0] {2};
\node[my](3) [below=8mm of 0, xshift=12mm] {4};
\node[my](4) [below=8mm of 1, xshift=-10mm] {5};
\node[my](5) [below=8mm of 1, xshift=10mm] {6};
\draw[thick] (0)--(1)--(4);
\draw[thick] (1)--(5);
\draw[thick] (0)--(2);
\draw[thick] (0)--(3);
\end{tikzpicture}
\end{figure}

We should return its level order traversal:
\begin{table}[H]
\begin{tabular}{lll}
1 & & \\
3 & 2 & 4\\
5 & 6 & 
\end{tabular}
\end{table}

\paragraph{Note:}

\begin{itemize}
\item The depth of the tree is at most 1000.
\item The total number of nodes is at most 5000.
\end{itemize}

\subsection{Queue}
典型的运用queue进行层次遍历的例子。

\setcounter{lstlisting}{0}
\begin{lstlisting}[style=customc, caption={Queue}]
vector<vector<int>> levelOrder( Node* root )
{
    if( !root )
    {
        return {};
    }

    queue<Node*> q;

    q.push( root );

    vector<vector<int>> ans;

    while( !q.empty() )
    {
        auto sz = q.size();

        //current level data
        vector<int> level;

        for( size_t i = 0; i < sz; ++i )
        {

            auto node = q.front();
            q.pop();

            level.push_back( node->val );

            for( auto child :  node->children )
            {
                if( child )
                {
                    q.push( child );
                }
            }
        }

        ans.push_back( move( level ) );
    }

    return ans;
}
\end{lstlisting}
