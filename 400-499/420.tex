\section{420 --- Strong Password Checker}
A password is considered strong if below conditions are all met:

\begin{itemize}
\item It has at least 6 characters and at most 20 characters.
\item It must contain at least one lowercase letter, at least one uppercase letter, and at least one digit.
\item It must NOT contain three repeating characters in a row (...\texttt{aaa}... is weak, but ...\texttt{aa}...\texttt{a}..." is strong, assuming other conditions are met).
\end{itemize}

Write a function \texttt{strongPasswordChecker}, that takes a string $s$ as input, and return the \textbf{MINIMUM} change required to make $ s $ a strong password. If $ s $ is already strong, return 0.
\par
\textbf{Insertion}, \textbf{deletion} or \textbf{replace} of any one character are all considered as one change.

\subsection{Greedy Algorithm}
Try to find a greedy way to use as few as possible deletions, insertions and replacements. Per the description, there are three cases to be dealt with: e need to change letters to solve 3 cases
\begin{enumerate}
\item The length of password: $\ell$
\item Missing any kind of uppercase, lowercase or digit.
\item The repeating characters.
\end{enumerate}
The required changes for the password can be divided into 3 scenarios
\begin{enumerate}
\item $\ell <  6$: 这种情况下重复字符个数的范围为$[3,5]$
\begin{itemize}
\item 如果有3个重复字符,那么增加3个字符的操作可以同时解决case 3,例如 $\texttt{aaa} \longrightarrow \texttt{a1BCaa}$;
\item 如果有4个重复字符,那么增加2个字符的操作也可以解决case 3,例如$\texttt{aaaa} \longrightarrow \texttt{aa1Baa})$;
\item 如果有5个重复字符,那么insertion和replace能同时解决case 3,例如 $\texttt{aaaaa}\longrightarrow \texttt{aa1aaB}$。
\end{itemize}
所以需要计算出minimum changes to solve the case 1 and 2.首先计算出当前密码串需要insert几个字符才能得到长度6,即$6-\ell$。由于insert也能同时解决case 2,所以当缺失的字符种类个数小于或者$6-\ell$时,insert就能解决问题。反之,如果缺失的字符种类个数大于$6-\ell$时,还需要额外的replace操作。所以$\ell < 6$的时候,所需要的最少change为$\max(6-\ell, k)$。其中$k$为缺失的字符\textbf{种类}个数。
\item $6\leq \ell \leq 20$. 这种情况下,case 1不存在,因此只需要replace。。
\item $\ell > 20$,这时候需要尽量减少deletion的次数,而最佳的替换是replace。,
\end{enumerate}
因此,需要找到方法尽可能的用replace operation。这里的trick在于: 
\begin{itemize}
\item For a sequence with $x$ repeating character, instead of changing it into a sequence of length 2 by deleting to fix case 3, we first reduce $x$ to $y$ where $y$ is the largest number that no less than $x$ and $y\bmod 3=2$. As such, there will have
\begin{enumerate}
\item When $x\bmod 3=0$, $y=x-1$
\item When $x\bmod 3=1$, $y=x-2$
\item When $x\bmod 3=2$, $y=x-3$
\end{enumerate}
\item  The reason behind this trick is that for any sequence with $3\times n+2$ repeating character, we can use \textbf{replace} $n$ characters to solve case 3.
\item Of course, if the remaining changes are less than the calculated replacement changes, we just use the minimum ones.
\end{itemize}
According the analysis above, the process for calculating the minimum changes is as below:
\begin{enumerate}
\item Get the number of \textbf{replace} changes first. This will be the sum of length of each sequence with more than 2 letters divide by 3 as we need to replace one character every 3 letters. 
\item At the same time, we get the number of deletion to change each sequence with $x$ repeating characters to $y$. 
\begin{itemize}
\item When $x\bmod 3=0$, each sequence need to delete one letter, therefore increments the number of deletions to delete one letter, denote as $d_1$. 由于每个sequence删除1个letter后就少一个replacement, so the total replacement changes will decrease by 1 for every 1 deletion.
\item When $x\bmod 3=1$, each sequence need to delete 2 letters, therefore increase the number of deletions to delete two letters by 2, denote as $d_2$. 由于每个sequence删除2个letter后就少一个replacement, so the total replacement changes will decrease by 1 for every 2 deletions.
\item When $x\bmod 3=2$, 因为删除3个就少一个需要replace操作的sequence。So, the total replacement changes will be decremented for every 3 deletions.
\end{itemize}
\item Therefore, when finish iterating over $s$, total replace changes will be changed by the following rules:
\begin{itemize}
\item For deletion of 1 characters, the number of replace changes will be decreased by $d_1$
\item For deletion of 2 characters, the number of replace changes will be decreased by $d_2 / 2$.
\item After performing the above deletions, all remaining sequences have $3\times n+2$ letters. We can apply all remaining required deletions to all these sequences. Therefore, the number of replace changes will be decreased by the remaining required number of deletions divided by 3.
\end{itemize}
\item In the end, compare the final number of replacement changes with the required changes to fix case 2. The maximum plus $\ell-20$ is the answer. 
\end{enumerate}

\setcounter{lstlisting}{0}
\begin{lstlisting}[style=customc, caption={Change to $3\times n+2$ sequence}]
int strongPasswordChecker( string s )
{
    int last_c = 0;

    int uppercase = 1;
    int lowercase = 1;
    int digit = 1;

    int count = 1;

    int num_replaces = 0;
    int num_del1 = 0;
    int num_del2 = 0;

    int L = static_cast<int>( s.size() );

    auto check_case2 = [&uppercase, &lowercase, &digit]( char c )
    {
        if( ( c >= '0' ) && ( c <= '9' ) )
        {
            //digit exist, no change is needed to insert digit
            digit = 0;
        }
        else if( ( c >= 'a' ) && ( c <= 'z' ) )
        {
            //lowercase exist, no change is needed to insert lowercase
            lowercase = 0;
        }
        else if( ( c >= 'A' ) && ( c <= 'Z' ) )
        {
            //uppercase exist, no change is needed to insert uppercase
            uppercase = 0;
        }
    };

    for( auto c : s )
    {

        check_case2( c );

        if( c == last_c )
        {
            ++count;
        }
        else
        {
            if( count >= 3 )
            {
                //need to replace each letter every 3 letters
                num_replaces += ( count / 3 );
                if( ( count % 3 ) == 0 )
                {
                    //delete one letter
                    ++num_del1;
                }
                else if( ( count % 3 ) == 1 )
                {
                    //delete two letters
                    num_del2 += 2;
                }
            }

            count = 1;
        }

        last_c = c;
    }

    //process the last character. The count may be larger than 3 but we
    //have not processed

    if( count >= 3 )
    {
        num_replaces += ( count / 3 ); //need to replace each letter every 3 letters
        if( ( count % 3 ) == 0 )
        {
            //delete one letter
            ++num_del1;
        }
        else if( ( count % 3 ) == 1 )
        {
            //delete two letters
            num_del2 += 2;
        }
        //the following is not needed
        // else
        // {
        // //delete three letters
        // num_del3 += 3;
        // }
    }

    int add_missing = uppercase + lowercase + digit;
    //length < 6
    if( L < 6 )
    {
        return ( max )( add_missing, 6 - L );
    }

    if( ( L >= 6 ) && ( L <= 20 ) )
    {
        //only replacement is needed
        return ( max )( add_missing, num_replaces );
    }

    //length > 20
    int del_needed = L - 20;

    //delete 1 letter for 3k sequence, the number of replace can be decreased by 1
    num_replaces -= ( min )( del_needed, num_del1 );

    //delete 2 letter for 3k+1 sequence, the number of replace can be decreased by 1
    int remaining_del_needed = ( max )( del_needed - num_del1, 0 );
    num_replaces -= ( min )( remaining_del_needed, num_del2 ) / 2;

    //delete 3 letter for 3k+2 sequence, the number of replace can be decreased by 1
    remaining_del_needed = ( max )( remaining_del_needed - num_del2, 0 );
    //Notice: we do not use (min)(remaining_del_needed, num_del3) /3 here
    //when remaining_del_needed < num_del3, we can only decrease by remaining_del_needed/3
    //otherwise, after we delete num_del3 /3, we can still delete more letters from
    //all remaining 3n+2 sequence(notice: all sequences remainings are 3n+2)
    num_replaces -= ( max )( remaining_del_needed, 0 ) / 3;

    //now compare number of replaces with the number of add missing letter kinds
    //we want the maximum since they can cover each other

    return del_needed + ( max )( add_missing, num_replaces );
}

\end{lstlisting}

