\section{404 --- Sum of Left Leaves}
Find the sum of all left leaves in a given binary tree.

\paragraph{Example:}
\begin{flushleft}
\textbf{Input}:
\begin{figure}[H]
\begin{tikzpicture}
[my/.style={draw, fill=gray!20, circle, minimum size=5mm}]
\node[my](0) at (0,0) {3};
\node[my](1) [below = 8mm of 0, xshift=-12mm] {9};
\node[my](2) [below = 8mm of 0, xshift=12mm] {20};
\node[my](3) [below = 8mm of 2, xshift=-9mm] {15};
\node[my](4) [below = 8mm of 2, xshift=9mm] {7};
\draw[thick] (0) -- (1);
\draw[thick] (0) -- (2);
\draw[thick] (2) -- (3);
\draw[thick] (2) -- (4);
\end{tikzpicture}
\end{figure}
\textbf{Output}: 24
\\
\textbf{Explanation}: There are two left leaves in the binary tree, with values 9 and 15 respectively. Return 24.
\end{flushleft}

\subsection{Depth First Search}
\begin{itemize}
\item We need to add boolean flag as one of parameters in the recursive function to indicate if the current node is left or right.
\item If the root has no child, we must return 0.
\end{itemize}

\setcounter{lstlisting}{0}
\begin{lstlisting}[style=customc, caption={Depth First Search}]
int sumOfLeftLeaves( TreeNode* root )
{

    int ans = 0;

    //if root has no children,
    //we must return 0
    dfs( root, false, ans );

    return ans;
}

//helper function, add is_left to indicate
//if the node is left or right
void dfs( TreeNode* node, bool is_left, int& sum )
{
    if( !node )
    {
        return;
    }

    if( !node->left && !node->right )
    {
        if( is_left )
        {
            //only add sum for left node
            sum += node->val;
        }
    }
    else
    {
        dfs( node->left, true, sum );
        dfs( node->right, false, sum );
    }
}
\end{lstlisting}


