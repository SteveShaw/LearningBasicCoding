\section{410 --- Split Array Largest Sum}
Given an array $A$ which consists of non-negative integers and an integer $ m $, you can split the array into $ m $ non-empty continuous subarrays. Write an algorithm to minimize the largest sum among these $ m $ subarrays.

\paragraph{Note:}
\begin{itemize}
\item If $ n $ is the length of array, assume the following constraints are satisfied:

\begin{enumerate}
\item    $ 1 \leq n \leq 1000$
\item    $ 1 \leq m \leq \min(50, n)$
\end{enumerate}
\end{itemize}

\paragraph{Examples:}

\begin{flushleft}
\textbf{Input}: $A = [\;7,2,5,10,8\;]$, $m = 2$
\\
\textbf{Output}: 18
\\
\textbf{Explanation}: There are four ways to split nums into two subarrays. The best way is to split it into $[\;7,2,5\;]$ and $[\;10,8\;]$, where the largest sum among the two subarrays is only 18.
\end{flushleft}

\subsection{Binary Search}
Denote $\lvert A\rvert = n$
\begin{enumerate}
\item 如果$m = n$,那么每个subarray就是$A$中的每个element,所以$A$中的最大值即为结果。而如果$m=1$,那么$\sum\limits_{i=0}^{n} A$就是结果。
\item 因此题目要求的结果必然处于上述两个值中间,可以用binary search的方法。这时候binary search找到的是count,而不是value。
\item 在binary search的过程中,按照当前的中间值对原数组进行划分,看是否存在有多于$m$个subarray,其sum小于或者等于当前的中间值。
\begin{enumerate}
\item 如果存在多于$m$个subarray,显然中间值偏小了。因此将左边界increment
\item 如果subarray的个数等于$m$,则说明这个中间值刚好能够得到$m$个subarray,但是我们还需要找到更小的值,因此这时候需要将右边界decrements
\item 而如果subarray的个数小于$m$,显然中间值偏大了,同样的,我们也需要把右边界decrements。
\end{enumerate}
\item 当binary search结束后,左边界即为所求结果。
\end{enumerate}

\setcounter{algorithm}{0}
\begin{algorithm}[H]
\caption{Binary Search}
\begin{algorithmic}[1]
\Procedure{SplitArray}{$A, L, m$}
\State $\star$ Find sum of $A$ and maximum element in $A$, denote them as $x$ and $y$ respectively
\State $\ast$ Binary search to find the minimum largest sum that can divide $A$ into $m$ subarrays
\While{$x \leq y$}
\State $z=(x+y)/2$
\State $\ast$  Check if number of subarrays with sum no larger than $z$ is larger than $m$ or not
\State $b:$= \Call{Find}{$ A, L, m, z $}
\If{ $b$ is \texttt{false}} \Comment number of subarrays is larger than $m$
\State $x\gets x+1$ \Comment This means $z$ is a little lower
\Else
\State $y\gets y-1$ \Comment This means $z$ is a little higher
\EndIf
\EndWhile
\State \Return $x$ \Comment Finally $x$ is the result
\EndProcedure
\end{algorithmic}
\end{algorithm}
Function \texttt{Find} check if number of subarrays in $A$ with sum no larger than target $T$ is larger than $m$ or not.

\begin{algorithm}[H]
\caption{Helper function to get number of subarrays whose sum are no less than $T$}
\begin{algorithmic}[1]
\Procedure{Find}{$A, L, m, T$}
\State $\delta:=1$ \Comment The count
\State $t:=0$ \Comment The sum of subarray
\For{Each number $n$ in $A$}
\State $t\gets t+n$
\If{$t> T$}
\State $\delta\gets\delta+1$ \Comment The sum before adding $n$ is no larger than $T$
\If{$\delta > m$}
\State \Return \texttt{false} \Comment Too many subarrays are found
\EndIf
\State $t\gets n$ \Comment Reset $t$ to start getting subarray sum again
\algstore{410algo}
\end{algorithmic}
\end{algorithm}
\begin{algorithm}[H]
\begin{algorithmic}[1]
\algrestore{410algo}
\EndIf
\EndFor
\State \Return \texttt{true}
\EndProcedure
\end{algorithmic}
\end{algorithm}
上述算法有两点需要注意
\begin{enumerate}
\item binary search的循环条件是$x\leq y$,而不是$x<y$,因为找到的subarray的个数小于等于$m$都会调整$x$,所以如果只有$x<y$,那么就会忽略$x=y$时的情况,这就会造成求出的largest sum会得到多于$m$个subarray,或者得到$m$个subarray,但不是最小的值。
\item Function \texttt{Find}中,$\delta$的初始值是1,因为整个array作为一个整体肯定不小于$T$,毕竟右边界是整个array的sum。
\end{enumerate}

\setcounter{lstlisting}{0}
\begin{lstlisting}[style=customc, caption={Binary Search}]
int splitArray( vector<int>& nums, int m )
{
    using ll = long long;

    ll x = 0;
    ll y = 0;

    for( int n : nums )
    {
        x = ( max )( static_cast<ll>( n ), x );
        y += n;
    }

    if( m == 1 )
    {
        return y;
    }

    //we need to loop until x > y
    //otherwise we will skip x=y
    while( x <= y )
    {
        auto z = ( x + y ) / 2;
        if( find( nums, m, z ) )
        {
            y = z - 1;
        }
        else
        {
            x = z + 1;
        }
    }

    return static_cast<int>( x );
}

int find( vector<int>& A, int m, long long T )
{
    //at first, A itself is a subarray
    //that has sum no less than T
    int delta = 1;
    long long t = 0;

    for( int n : A )
    {
        t += n;
        if( t > T )
        {
            ++delta;
            if( delta > m )
            {
                //T is a little lower
                return false;
            }

            t = n; //reset the subarray sum
        }
    }

    return true;
}
\end{lstlisting}