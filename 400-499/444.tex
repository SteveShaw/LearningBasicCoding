\section{444 ---Sequence Reconstruction}
\textcolor{red}{\large Locked}

Check whether the original sequence $A$ can be uniquely reconstructed from the sequences in $S$. The $A$ sequence is a permutation of the integers from 1 to $n$, with $1 \leq n \leq 10^4$. Reconstruction means building a shortest common supersequence of the sequences in $S$ (i.e., a shortest sequence so that all sequences in $S$ are subsequences of it). Determine whether there is only one sequence that can be reconstructed from $S$ and it is the $A$ sequence.

\paragraph{Example 1:}

\begin{flushleft}
\textbf{Input}: $A= [1,2,3]$, $S= [\,[1,\,2],\,[1,3]\,]$

\textbf{Output}: \texttt{false}

\textbf{Explanation}:

$[1,2,3]$ is not the only one sequence that can be reconstructed, because $[1,3,2]$ is also a valid sequence that can be reconstructed.
\end{flushleft}

\paragraph{Example 2:}

\begin{flushleft}
\textbf{Input}: $A= [1,2,3]$, $S= [[1,2]]$

\textbf{Output}: \texttt{false}

\textbf{Explanation}:

The reconstructed sequence can only be $[1,2]$.
\end{flushleft}


\paragraph{Example 3:}

\begin{flushleft}
\textbf{Input}: $A = [1,2,3]$, $S = [[1,2],[1,3],[2,3]]$

\textbf{Output}: \texttt{true}

\textbf{Explanation}:

The sequences $ [1,2] $, $ [1,3] $, and $ [2,3] $ can uniquely reconstruct the original sequence $ [1,2,3] $.

\end{flushleft}

\paragraph{Example 4:}

\begin{flushleft}
\textbf{Input}: $A= [4,1,5,2,6,3]$, $S = [[5,2,6,3],[4,1,5,2]]$

\textbf{Output}: \texttt{true}
\end{flushleft}

\subsection{Verify The Position}
两个要点
\begin{itemize}
\item $A$中的elements的前后顺序在$S$中必须是一致的。
\item $A$中两个相邻元素在$S$中必定也是相邻的。
\end{itemize}

因此需要判断在$S$中的每一个sequence
\begin{enumerate}
\item 前后顺序和$A$的一致。
\item 在$S$中的所有sequence能够还原出$L-1$个adjacent element pair
\end{enumerate}

\begin{itemize}
\item 对于上述第一个判断,可以用一个数组记录$A$中每个元素的位置,然后在$S$中进行比较
\item 而对于第二个判断,则需要另外一个数组对当前元素$i$在$A$中的前面一个元素是否能够在$S$中的至少一个sequence中能够得到验证。如果能够验证,则累加验证的counter,同时标明这个元素已经被验证。最后看这个counter是否等于$L-1$。
\end{itemize}

\setcounter{lstlisting}{0}
\begin{lstlisting}[style=customc, caption={Verify The Position}]
bool sequenceReconstruction( vector<int>& org, vector<vector<int>>& seqs )
{
    vector<int> pos( org.size(), -1 );

    //verify if a element's previous element is found
    vector<unsigned char> mark( org.size(), 0 );

    int L = static_cast<int>( org.size() );

    for( int i = 0; i < L; ++i )
    {
        pos[org[i] - 1] = i;
    }

    //count is the number of adjacent elements found in seqs
    int count = 0;

    for( const auto& seq : seqs )
    {
        int l = static_cast<int>( seq.size() );

        for( int i = 0; i < l; ++i )
        {
            if( ( seq[i] > L ) || ( seq[i] <= 0 ) )
            {
                //seq[i] is out of range [0,L]
                return false;
            }

            if( i > 0 )
            {
                if( pos[seq[i - 1] - 1] >= pos[seq[i] - 1] )
                {
                    //out of order in for seq[i-1] and seq[i]
                    return false;
                }
                if( mark[seq[i] - 1] == 0 )
                {
                    //we have not mark seq[i]
                    if( pos[seq[i - 1] - 1] + 1 == pos[seq[i] - 1] )
                    {
                        //mark seq[i] is verified.
                        mark[seq[i] - 1 ] = 1;
                        //found a seq[i]'s previous number in org
                        ++count;
                    }
                }
            } //end if(i>0)
        }
    }


    //only all adjacent numbers are verified
    return count == L - 1;
}
\end{lstlisting}

\subsection{Topological Sort}
We can build an ordered graph based on given \fcj{seq}. Then perform typical topological sort on the graph. The problem is asking for only one ordered output, which means in any time, there is no more than 1 element in the queue. 

We know \fcj{org} is an array of a permutation of the integers from 1 to $n$. Thus, if we find any element in given \fcj{seqs} is outside of range $[1,n]$, we can directly return \fcj{false}.

During the BFS in topological sort, we will compare the result sequence against \fcj{org}.

\begin{lstlisting}[style=customc, caption={Topological Sort}]
bool sequenceReconstruction( vector<int>& org, vector<vector<int>>& seqs )
{
    int n = ( int )( org.size() );
    vector<unordered_set<int>> g( n + 1 );
    vector<int> indeg( n + 1, 0 );
    //build the graph
    //empty_graph is used to check if seqs is empty
    bool empty_graph = true;
    for( const auto& seq : seqs )
    {
        if( !seq.empty() )
        {
            empty_graph = false;
        }
        auto len = ( long )( seq.size() );
        if( len == 1 )
        {
            if( ( seq[0] > n ) || ( seq[0] < 1 ) )
            {
                //make sure the element is inside [1,n]
                return false;
            }
        }
        for( long i = 0l; i < len - 1; ++i )
        {
            //make sure the elements in seq is inside [1,n]
            if( ( seq[i] > n ) || ( seq[i + 1] > n ) || ( seq[i] < 1 ) || ( seq[i + 1] < 1 ) )
            {
                return false;
            }
            //build adjacent nodes graph
            auto& adjs = g[seq[i]];
            auto it = adjs.find( seq[i + 1] );
            if( it == adjs.end() )
            {
                //since we may have repeat elements in different seq
                //we have to make sure for the same pair
                //the in degrees of the node only increments once
                adjs.insert( seq[i + 1] );
                indeg[seq[i + 1]] += 1;
            }
        }
    }
    //this is a empty seqs
    if( empty_graph )
    {
        return false;
    }
    //topological sort
    //starting from nodes with in degrees are zeros
    queue<int> q;
    for( int i = 1; i <= n; ++i )
    {
        if( indeg[i] == 0 )
        {
            q.push( i );
        }
    }
    //the index in org
    int org_idx = 0;
    while( !q.empty() )
    {
        //anytime, we can only have one
        //number in the queue
        if( q.size() > 1 )
        {
            return false;
        }
        auto x = q.front();
        q.pop();
        //check current element with org
        if( x != org[org_idx] )
        {
            return false;
        }
        ++org_idx;
        //check adjacent numbers
        for( int adj : g[x] )
        {
            indeg[adj] -= 1;
            if( indeg[adj] == 0 )
            {
                q.push( adj );
            }
        }
    }
    //make sure org is completely matched.
    return org_idx == ( int )( org.size() );
}
\end{lstlisting}