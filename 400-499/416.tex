\section{416 --- Partition Equal Subset Sum}
Given a non-empty array $A$ containing only positive integers, find if the array can be partitioned into two subsets such that the sum of elements in both subsets is equal.

\paragraph{Note:}

\begin{enumerate}
\item Each of the array element will not exceed 100.
\item The array size will not exceed 200.
\end{enumerate}


\paragraph{Example 1:}

\begin{flushleft}
\textbf{Input}: $[1, 5, 11, 5]$
\\
\textbf{Output}: \texttt{true}
\\
\textbf{Explanation}: The array can be partitioned as $[1, 5, 5]$ and $[11]$.
\end{flushleft}

 
\paragraph{Example 2:}

\begin{flushleft}
\textbf{Input}: $[1, 2, 3, 5]$
\\
\textbf{Output}: \texttt{false}
\\
\textbf{Explanation}: The array cannot be partitioned into equal sum subsets.
\end{flushleft}

\subsection{Top-Down Dynamic Programming}
\begin{itemize}
\item Maintain a hash map $M$, where the key is string and the value is a Boolean variable. The key is composed from two numbers: current number and target. The value will indicate if the target will be reached or not if the current number is included.
\item Apparently, the sum must be even. Start the recursion to find if there some number that can be sum up to the half of the sum.
\item In the recursive function, test each number from given start index and whenever the target is reached, the function will return \texttt{true} immediately.
\end{itemize}

\setcounter{algorithm}{0}
\begin{algorithm}[H]
\caption{Top-Down Dynamic Programming}
\begin{algorithmic}[1]
\Procedure{CanPartition}{$A, L$}
\If{$\sum\limits A$ is odd}
\State \Return \texttt{false}
\EndIf
\State $\star$ Create a hash map $M$ as memorization
\State $b:=$\Call{DFS}{$ A, L, p = 0, T = (\sum\limits A)/2, M $}
\If{$b=1$}
\State \Return \texttt{true}
\Else
\State \Return \texttt{false}
\EndIf
\EndProcedure
\end{algorithmic}
\end{algorithm}

Function \texttt{DFS} check if the target $ T $ can be obtained from in range $A[p\ldots L-1]$

\begin{algorithm}[H]
\caption{Helper function}
\begin{algorithmic}[1]
\Procedure{DFS}{$ A, L, p, T, M$}
\If{$p \geq L$ \textbf{or} $T< 0$}
\State \Return 1
\EndIf
\State $\star$ Make a string $k$ from $A[p]$ and $T$
\If{$k$ is one of keys in $M$}
\State \Return $M[k]$
\EndIf
\If{$T=0$} \Comment The target is reached
\State $M[k]:=1$
\State \Return 1
\EndIf
\State $b:=0$
\For{$i:=p$ \textbf{to} $L-1$}
\State $\ast$ Check if $T$ can be reached if $A[i]$ is included
\State $b:=$\Call{DFS}{$ A, L, i+1, T - A[i], M $}
\algstore{416algo}
\end{algorithmic}
\end{algorithm}
\begin{algorithm}[H]
\begin{algorithmic}[1]
\algrestore{416algo}
\If{$b=1$}
\State \texttt{break} \Comment We can reach $T$ by including $A[i]$
\EndIf
\EndFor
\State $M[k]:=b$ \Comment Add to memorization
\State \Return $b$
\EndProcedure
\end{algorithmic}
\end{algorithm}

\setcounter{lstlisting}{0}
\begin{lstlisting}[style=customc, caption={Top Down Dynamic Programming}]
bool canPartition( vector<int>& nums )
{
    long long sum = 0;

    for( int n : nums )
    {
        sum += n;
    }

    //only even sum can be partitioned
    if( sum & 1 )
    {
        return false;
    }

    unordered_map<string, unsigned char> memo;

    return dfs( nums, 0, sum / 2, memo );

}

bool dfs( vector<int>& nums, size_t start, long long target, unordered_map<string, unsigned char>& memo )
{
    if( ( start >= nums.size() ) || ( target < 0 ) )
    {
        //since all elements are positive numbers
        //if target is less than zero,
        //no way to reach.
        return false;
    }

    //form the key by nums[start] and target
    auto key = to_string( nums[start] ) + "#" + to_string( target );

    auto it = memo.find( key );

    if( it != memo.end() )
    {
        return it->second == 1;
    }

    if( target == 0 )
    {
        memo.emplace( key, 1 );
        return true;
    }

    bool b = false;

    for( size_t i = start; i < nums.size(); ++i )
    {
        //test if nums[i] is included, the target can be reached or not
        b = dfs( nums, i + 1,  target - nums[i], memo );

        if( b )
        {
            break;
        }
    }

    memo.emplace( key, b ? 1 : 0 );

    return b;
}

\end{lstlisting}

\subsection{0/1 knapsack problem}
\begin{itemize}
\item Maintain an array $F$. The size is equal to $\sum\limits A /2 + 1$. $F[t]$ means if we can get sum $F[t]$ by a few numbers from $A$. 
\item For each number $n$ in $A$, we have the option to choose or not choose (0/1 choice). If we choose this number, then we are leaving $t-n$ to be reached. Therefore $F[t] = F[t-n]$. Otherwise, $F[t]$ will be itself.
\item In the algorithm, we have two loops. The outer loop is iterating $A$ and the inner loop will iterate from $\sum\limits A /2$ to current number $n$ because we cannot get negative sum from positive numbers. For current target $t$, if $F[t-n]$ is \texttt{true}, then $F[t]$ will be \texttt{true}.
\item Finally, if $F[\sum\limits A /2]$ is \texttt{true}, it means we can partition $A$ into two parts with equal sum
\end{itemize}

\begin{lstlisting}[style=customc, caption={Bottom-up Dynamic Programming}]
bool canPartition( vector<int>& nums )
{
    long long sum = 0;
    for( int n : nums )
    {
        sum += n;
    }

    if( sum & 1 )
    {
        return false;
    }

    long long T = sum / 2;
    vector<unsigned char> F( T + 1, 0 );

    F[0] = 1;

    for( int n : nums )
    {
        for( auto t = T; t >= n; --t )
        {
            //if F[t-n] is true, means
            //we can get t by including n
            F[t] = ( F[t] || F[t - n] );
        }
    }

    return F[T] > 0;

}
\end{lstlisting}

