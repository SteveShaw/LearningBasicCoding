\section{456 --- 132 Pattern}
Given a sequence of $n$ integers $a_1$, $a_2$, $\ldots$, $a_n$, a 132 pattern is a subsequence $a_i$, $a_j$, $a_k$ such that $i < j < k$ and $a_i < a_k < a_j$. Design an algorithm that takes a list of $n $ numbers as input and checks whether there is a 132 pattern in the list.

Note: $n$ will be less than 15,000.

\paragraph{Example 1:}

\begin{flushleft}
\textbf{Input}: [1, 2, 3, 4]

\textbf{Output}: \texttt{False}

\textbf{Explanation}: There is no 132 pattern in the sequence.
\end{flushleft}

\paragraph{Example 2:}

\begin{flushleft}
\textbf{Input}: [3, 1, 4, 2]

\textbf{Output}: \texttt{True}

\textbf{Explanation}: There is a 132 pattern in the sequence: $[1, 4, 2]$.
\end{flushleft}

\paragraph{Example 3:}

\begin{flushleft}
\textbf{Input}: [-1, 3, 2, 0]

\textbf{Output}: \texttt{True}

\textbf{Explanation}: There are three 132 patterns in the sequence: [-1, 3, 2], [-1, 3, 0] and [-1, 2, 0].

\end{flushleft}

\subsection{Brute Froce}
The most direct approach is to test each $(i,j,k)$ to check if the corresponding numbers satisfy the 132 criteria.

To improve the efficiency, we can observe the following fact: For a number $A[j]$, if we find a smallest $A[i]$ such that $A[i] < A[j]$ and $i<j$, then for any $k> j$, the likelihood that $A[k]$ is between $A[i]$ and $A[j]$ is increased.

Thus, we can keep track of minimum number $x$ until $j$, and then check all the index $k> j$ to see if there is any $A[k]<A[j]$ and $A[k]> x$.

\setcounter{lstlisting}{0}
\begin{lstlisting}[style=customc, caption={Better Brute Force}]
bool find132pattern( vector<int>& nums )
{
    if( nums.empty() || nums.size() < 3 )
    {
        return false;
    }

    int x = nums[0];

    for( size_t j = 0; j < nums.size() - 1; ++j )
    {
        //keep track of minimum number so far
        x = ( min )( x, nums[j] );

        for( size_t k = j + 1; k < nums.size(); ++k )
        {
            //nums[k] is in [x, nums[j]]
            if( ( nums[k] < nums[j] ) && ( nums[k] > x ) )
            {
                return true;
            }
        }
    }

    return false;
}
\end{lstlisting}

\subsection{Searching Intervals}
This approach is relying on the fact: to maximize the likelihood that we can find $A[k]$, $A[i]$ and $A[j]$ must be the start and end points of a local increasing segment. 

To achieve this, we can keep track of minimum number, $x$, found after the last peak.  Whenever we encounter $A[y] < A[y-1]$, we know that $A[y-1]$ was the endpoint of the last found increasing segment. Thus, we can iterate over $k > y$ to find the 132 pattern.

But, we can bypass the iteration process if we can store all the local increasing segments found so far. Therefore, during the traversing, suppose $x$ is the start of a segment, whenever we find $A[y] < A[y-1]$, add the local increasing segment $A[x\ldots y-1]$ into the stored ranges. At the same time, we check if the current number falls in any of the ranges found so far. If so, 132 pattern is found.

If no such element is found at the end, this means we cannot find 132 pattern in $A$.

\begin{lstlisting}[style=customc, caption={Search Intervals}]
bool find132pattern( vector<int>& nums )
{

    size_t x = 0;

    //store the local increasing segments so far
    vector<array<int, 2>> segs;

    for( size_t y = 1; y < nums.size(); ++y )
    {
        if( nums[y - 1] > nums[y] ) //or nums[y - 1] >= nums[y]
        {
            //the last local increasing segment
            //ends
            if( x < y - 1 )
            {
                //exclude the case x=y-1
                //add local increasing segment
                //[x,y-1] to the stored segs
                segs.push_back( array<int, 2> {x, y - 1} );
            }

            //update next segement start as y
            x = y;
        }

        //check if nums[y] is in the ranges that in segs
        for( const auto& seg : segs )
        {
            if( ( nums[seg[0]] < nums[y] ) && ( nums[y] < nums[seg[1]] ) )
            {
                return true;
            }
        }
    }

    return false;
}
\end{lstlisting}

 

\subsection{Stack}
Suppose we want to find a 123 sequence with $n_1 < n_2 < n_3$, we just need to find $n_3$, followed by $n_2$ and $n_1$. Now if we want to find a 132 sequence with $n_1 < n_3 < n_2$, we need to switch up the order of searching. we want to first find $n_2$, followed by $n_3$, then $n_1$.

\setcounter{lstlisting}{0}
\begin{lstlisting}[style=customc, caption={Stack}]
bool find132pattern( vector<int>& nums )
{
    if( nums.empty() || ( nums.size() < 3 ) )
    {
        return false;
    }

    int L = static_cast<int>( nums.size() );

    int x_k = INT_MIN;

    stack<int> stk;

    for( int i = L - 1; i >= 0; --i )
    {
        //stk is not empty means
        //we have x_j > x_k and (j < k)
        //so if x_i < x_k is found
        //we found the pattern
        if( !stk.empty() && ( nums[i] < x_k ) )
        {
            return true;
        }

        //we put x_j in the stack
        //pop all numbers in the stack that is less than nums[i]
        //the last one will be the latest x_k < x_j
        //this is strict less than, so we use
        //stk.top() < nums[i]
        while( !stk.empty() && ( stk.top() < nums[i] ) )
        {
            //update x_k as the top of the stack
            x_k = stk.top();
            stk.pop();
        }

        //now the numbers in the stack
        //are larger than x_k
        stk.push( nums[i] );
    }

    return false;
}
\end{lstlisting}