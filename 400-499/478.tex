\section{478 --- Generate Random Point in a Circle}
Given the radius $r$ and $(x,y)$ positions of the center of a circle, write a function which generates a uniform random point in the circle.

\paragraph{Note:}

\begin{itemize}
\item input and output values are in floating-point.
\item $r$ and $(x,y)$ position of the center of the circle is passed into the class constructor.
\item a point on the circumference of the circle is considered to be in the circle.
\item Function returns a size 2 array containing $x$-position and $y$-position of the random point, in that order.

\end{itemize}

\subsection{Inverse Transform}
To generate random values according to a custom distribution by using a function that generates uniform random values between 0 and 1, we can use the inverse transform sampling method.

\begin{itemize}
\item Create the cumulative distribution function (\texttt{CDF})
\item Mirror the \texttt{CDF} along $y = x$
\item Apply the resulting function to a uniform value between 0 and 1
\end{itemize}

For example, Suppose we want to generate a random point with the following distribution:

\begin{itemize}
\item $1/5$ of the points uniformly between 1 and 2, and
\item $4/5$ of the points uniformly between 2 and 3.
\end{itemize}

The \textbf{probability density function} (PDF) in this case would look like this:
\[
P(x) = 
\begin{cases}
0  &  x < 1 \\
\frac{1}{5}  & 1\leq x<2 \\
\frac{4}{5} & 2\leq x <3 \\
0 & x\geq 3
\end{cases}
\]

\paragraph{Step 1: Create the CDF}


The \texttt{CDF} is, as the name suggests, the cumulative version of the \texttt{PDF}. Intuitively: While \texttt{PDF} $p(x)$ describes the number of random values at $x$, \texttt{CDF} $\Phi(x)$ describes the number of random values less than $x$.

Since we're working with reals, the \texttt{CDF} is expressed as the integral of the \texttt{PDF}. In this case the \texttt{CDF} would look like:

\[
\Phi(x) = 
\begin{cases}
0 & x < 1 \\
\int_{1}^{x}\frac{1}{5}dx = \frac{1}{5}(x-1) & 1\leq x < 2 \\
\frac{1}{5} + \int_{2}^{x}\frac{4}{5}dx = \frac{1}{5} + \frac{4}{5}(x-2) & 2\leq x < 3 \\
1 & x \geq 3
\end{cases}
\]


\paragraph{Step 2: Inverse the CDF}

By inversing the \texttt{CDF}, we get $\Phi^{-1}(x)$. Mathematically this boils down to swapping $x$ and $y$ and solving for $y$. For example, suppose $\Phi(x) = x^2$. Denote $y=\Phi(x)$. 

\begin{enumerate}
\item Swap $x$ and $y$, we get $x=y^2$
\item Solve $y$ by $x$, we get $y=\sqrt{x}$
\item Thus $\Phi^{-1}(x) = \sqrt{x}$
\end{enumerate}


\paragraph{Step 3: Apply the resulting function to a uniform value between 0 and 1}

With the $\Phi^{-1}(x)$ we have all we need. We now simply feed it with uniformly random values between 0 and 1 through function $f$: $\Phi^{-1}(f())$. The result is random values with the desired distribution. Taking the example in Step 2, we get the inverse CDF $\Phi^{-1}(x) = \sqrt{x}$, thus we can get the random values from the uniform random input from the function $\sqrt{f()}$.

In this problem, to generate uniformly distributed numbers within a circle of radius $R$ in the $(x,y)$ plane. At first polar coordinates seems like a great idea, and the naive solution is to pick a radius $r$ uniformly distributed in $[0, R]$, and then an angle $\theta$ uniformly distributed in $[0, 2\pi]$. But, we will end up with an excess of points near the origin $(0, 0)$.  This is wrong because if we look at a certain angle interval, say $[\theta, \theta+\Delta\theta]$, there needs to be more points generated further out (at large $r$), than close to zero. As a result, the radius $r$ must not be picked from a uniform distribution, but a \texttt{pdf} function $p(r) = 2r/R^2$.

By integral, we get the \texttt{CDF} function $\Phi(r) = \int_{-\infty}^{r}\dfrac{2r}{R^2} = \dfrac{r^2}{R^2}$. Apply the process in Step 2
\begin{enumerate}
\item Swap $r$ and $y$: $r=\dfrac{y^2}{R^2}$
\item Solve $y$ by $r$: $y=\sqrt{R^2 r}=R\sqrt{r}$
\item We generate random uniform values for $r$ and get $y$ through $y=R\sqrt(r)$ to get the uniform values of radius.
\end{enumerate}

Lastly, we generate the uniform angle $\theta$ in $[0, 2\pi]$. Then convert from polar coordinate to Cartesian coordinate, we get
\begin{align*}
x &= r\times \cos\theta\\
y &= r\times \sin\theta
\end{align*}


\setcounter{lstlisting}{0}
\begin{lstlisting}[style=customc, caption={Inverse Transform}]
class Solution
{
public:
    Solution( double radius, double x_center, double y_center )
        : m_r( radius )
        , m_x( x_center )
        , m_y( y_center )
    {
        random_device rd;  //Will be used to obtain a seed for the random number engine
        m_gen.seed( rd() ); //Standard mersenne_twister_engine seeded with rd()
    }

    vector<double> randPoint()
    {
        //generate angle
        double theta = m_dis( m_gen ) * 2 * 3.1415926;
        //generate raidus
        double r = sqrt( m_dis( m_gen ) ) * m_r;

        return {m_x + r * cos( theta ), m_y + r * sin( theta )};

    }

    mt19937 m_gen;
    uniform_real_distribution<> m_dis;

    double m_r;
    double m_x;
    double m_y;
};
\end{lstlisting}

\subsection{Rejection Sampling}
To get uniform random points in a circle with radius $r$ at center $(x,y)$, we can generate uniform random points in the square with side length $2r$ at center point$(x,y)$, keeping all of the points that are at no larger than $r$ distance from the center, and rejecting all others not meeting this criteria. This technique is called rejection sampling. Each possible location on the circle has the same probability of being generated, so the sampling of points will be uniformly distributed.

The area of the square is $(2r)^2 = 4r^2$ and the area of the circle is $\pi r \approx 3.14R$. Then, we get $\dfrac{3.14R}{4R} = \dfrac{3.14}{4} = .785$. Therefore, we will get a usable sample approximately $78.5\%$ of the time and the expected number of times that we will need to sample until we get a usable sample is $\dfrac{1}{.785} \approx 1.274 $

\begin{lstlisting}[style=customc, caption={Rejection Sampling}]
class Solution
{
public:
    Solution( double radius, double x_center, double y_center )
    {
        c_x = x_center;
        c_y = y_center;
        c_r = radius;
    }

    vector<double> randPoint()
    {
        //get the bottom left point of the square
        double x0 = c_x - c_r;
        double y0 = c_y - c_r;

        while( true )
        {
            //rand_x in range [x0, x0+2*c_r]
            double rand_x = x0 + rand_gen( seed_gen ) * 2 * c_r;

            //rand_y in range [y0, y0+2*c_r]
            double rand_y = y0 + rand_gen( seed_gen ) * 2 * c_r;

            double dx = ( rand_x - c_x );
            double dy = ( rand_y - c_y );

            //get the square distance from the center (c_x, c_y)
            double square_d = dx * dx + dy * dy;

            if( square_d <= c_r * c_r )
            {
                //this point is inside the circle
                return {rand_x, rand_y};
            }
        }

        return {};
    }

    double c_x;
    double c_y;
    double c_r;

    //c++1x random number generation
    mt19937 seed_gen{random_device{}()};
    uniform_real_distribution<double> rand_gen{0, 1};
};
\end{lstlisting}

\subsection{Inverse Transform Sampling}
Assume that we are given a circle, $\alpha$, with radius 1 that is centered at the origin, and our task is to sample uniform random points on this circle.

Lets imagine another circle, $\beta$, with radius $\dfrac{1}{2}$, which is also centered at the origin.

The circumference of $\alpha$ is twice that of $\beta$. Also, the probability of sampling a point from a sub-region in circle $\alpha$ is directly proportional to the area of the sub-region. Therefore, the probability of sampling a point on the perimeter of $\alpha$ is twice that of sampling a point on the perimeter of $\beta$.

More generally, what is implied is that the sampling probability is directly proportional to the distance from the origin, from $0$ to $r$. This can be modeled as a \textbf{probability density function} $f$, where $x$ is the distance from the origin and $f(x)$ is the relative sampling probability at $x$, and $f(x)=cx$ where $c$ is a constant.

Since the area under any probability density function curve must be $1$, and the radius of circle $\alpha$ is 1, we have $ f(x) = cx* 1/2=1 \implies c=2$. Thus $f(x)=2x$.

Using this PDF $f$, we can compute the \textbf{cumulative distribution function} (CDF), $F$, where $F(x)$ is the probability of sampling a point within a distance of $x$ from the origin.

CDF is the integral of the probability density function, when we have
\[
F(x) = \int{f(x)} = \int{2x} = x^2
\]

Now, CDF $F$ can be used to compute the inverse cumulative distribution function $F^{-1}$, which accepts uniform random value in $[0,1]$, and returns a random distance from origin in accordance with $f$.
 
\begin{align*}
F^{-1}\left(F(x)\right) &= x \\
F^{-1}\left(x^2\right) &= x\\
F^{-1}(x) &= \sqrt{x}
\end{align*}

Now, to generate a uniform random point on CC, we just need to compute a random distance $D$ from origin using $F^{-1}$  and a uniform random angle $\theta$ over the range $[0, 2\cdot \pi)$

The points will be generated as polar coordinates. To convert to cartesian coordinates, we can use the following formulas.

\begin{align*}
x &= D \cdot \cos(\theta) \\ y &= D \cdot \sin(\theta)
\end{align*}

\begin{lstlisting}[style=customc, caption={Inverse Transform Sampling}]
class Solution
{
public:
    Solution( double radius, double x_center, double y_center )
    {
        c_x = x_center;
        c_y = y_center;
        c_r = radius;
    }

    vector<double> randPoint()
    {
        //generate random distance from origin
        //by a uniform rand value x in [0,1]
        //using CDF F(x) = sqrt(x)

        double D = sqrt( rand_gen( seed_gen ) ) * c_r;

        //generate radom theta in [0,2pi]
        double theta = rand_gen( seed_gen ) * 2 * 3.1415926;

        double x = D * cos( theta );
        double y = D * sin( theta );

        return {x + c_x, y + c_y};

    }

    double c_x;
    double c_y;
    double c_r;

    mt19937 seed_gen{random_device{}()};
    uniform_real_distribution<double> rand_gen{0, 1};
};
\end{lstlisting}