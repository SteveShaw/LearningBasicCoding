\section{474 --- Ones and Zeroes}
In the computer world, use restricted resource you have to generate maximum benefit is what we always want to pursue.

For now, suppose you are a dominator of $m$ 0s and $n$ 1s respectively. On the other hand, there is an array with strings consisting of only 0s and 1s.

Now your task is to find the maximum number of strings that you can form with given $m$ 0s and $n$ 1s. Each 0 and 1 can be used at most once.

\paragraph{Note:}

\begin{itemize}
\item The given numbers of 0s and 1s will both not exceed 100
\item The size of given string array won't exceed 600.
\end{itemize} 

\paragraph{Example 1}

\begin{flushleft}
\textbf{Input}: \lstinline[language=C++, basicstyle=\small\ttfamily, keywordstyle=\bfseries\color{green!40!black}]|A = {"10", "0001", "111001", "1", "0"}|, $m = 5$, $n = 3$

\textbf{Output}: 4

\textbf{Explanation}: This are totally 4 strings can be formed by the using of 5 0s and 3 1s, which are 10, 0001, 1, 0
\end{flushleft}
 

\paragraph{Example 2:}
\begin{flushleft}

\textbf{Input}: \lstinline[language=C++, basicstyle=\small\ttfamily, keywordstyle=\bfseries\color{green!40!black}]|A = {"10", "0", "1"}|, $m = 1$, $n = 1$

\textbf{Output}: 2

\textbf{Explanation}: You could form 10, but then you'd have nothing left. Better form 0 and 1.
\end{flushleft}

\subsection{Brute Force}
In the brute force approach, we consider every subset of $A$ and count the total number of zeroes and ones in that subset. The subset with zeroes less than equal to $m$ and ones less than equal to $n$ will be considered as the valid subsets. The maximum length subset among all valid subsets will be the answer.

To improve the efficiency, we can limit the number of subsets by breaking the loop when total number of zeroes exceed $m$ or total number of ones exceed $n$. But this will reduce little computation not the complexity.

Obviously, there are $2^{\ell}$ subsets possible for the list of length $\ell$ and here we are using \lstinline[language=C++, basicstyle=\small\ttfamily, keywordstyle=\bfseries\color{green!40!black}]|int|(32 bits) for iterating every subset. So this method will not work for the list having length greater than 32.


\subsection{Backtracking}
In the brute force approach, we consider every subset iteratively. We can also make use of backtracking approach in a recursive manner. Suppose the function is \lstinline[language=C++, basicstyle=\small\ttfamily, keywordstyle=\bfseries\color{green!40!black}]|F(A, i, ones, zeroes)|. This function takes the given list of strings $A$ and gives the size of the largest subset with 1s and 0s considering the strings lying after the index \lstinline[language=C++, basicstyle=\small\ttfamily, keywordstyle=\bfseries\color{green!40!black}]|i|th index(including itself) in \lstinline[language=C++, basicstyle=\small\ttfamily, keywordstyle=\bfseries\color{green!40!black}]|A|.

For each current string, we have two options

\begin{enumerate}
\item Include the current string in the subset currently being considered and deduct the number of 0's and 1's in the current string from $m$ and $n$ respectively. We also need to increment the total number of strings considered so far by 1. 
\item Not include the current string in the current subset. In this case, we don't the count of 1s and 0s. 
\end{enumerate}

The larger value out of these two options represents the required result to be returned for the current function call.

Finally, the maximum number of subsets is given by calling \lstinline[language=C++, basicstyle=\small\ttfamily, keywordstyle=\bfseries\color{green!40!black}]|F(A, 0, m, n)|.

\setcounter{lstlisting}{0}
\begin{lstlisting}[style=customc, caption={Backtrack}]
int findMaxForm( vector<string>& strs, int m, int n )
{
    return F( strs, 0, 0, 0, m, n );
}

//backtrack recursive function
int F( vector<string>& A, size_t start, int ones, int zeros, int m, int n )
{
    if( start == A.size() )
    {
        //No string available
        return 0;
    }

    auto get_ones = []( const string & s )
    {
        int count_1s = 0;

        for( auto c : s )
        {
            count_1s += ( c - '0' );
        }

        return count_1s;
    };

    int num_ones = get_ones( A[start] );
    int num_zeros = static_cast< int >( A[start].size() ) - num_ones;

    int sel = 0;

    if( ( ( ones + num_ones ) <= n ) && ( ( zeros + num_zeros ) <= m ) )
    {
        //try option 1: put A[start] into the subset
        sel = F( A, start + 1, ones + num_ones, zeros + num_zeros, m, n ) + 1;
    }

    //try option 2: don't put A[start] into the subset
    int non_sel = F( A, start + 1, ones, zeros, m, n );

    return ( max )( sel, non_sel );
}
\end{lstlisting}

\subsection{Backtracking With Memorization}
The previous backtracking approach makes use of a lot of redundant function calls for  the same values of \lstinline[language=C++, basicstyle=\small\ttfamily, keywordstyle=\bfseries\color{green!40!black}]|(i, zeroes, ones)|. This redundancy in the recursive tree can be pruned off by making use of a 3-D memorization array, say, $M$.

$M[i][j][k]$ is used to store the result obtained for the function call \lstinline[language=C++, basicstyle=\small\ttfamily, keywordstyle=\bfseries\color{green!40!black}]|F(A, i, j, k)|. Or in other words, it stores the maximum number of subsets possible considering the strings starting from the \lstinline[language=C++, basicstyle=\small\ttfamily, keywordstyle=\bfseries\color{green!40!black}]|i|th index onwards, provided only \lstinline[language=C++, basicstyle=\small\ttfamily, keywordstyle=\bfseries\color{green!40!black}]|j| zeros and \lstinline[language=C++, basicstyle=\small\ttfamily, keywordstyle=\bfseries\color{green!40!black}]|k| ones are available to be used.

Thus, whenever $M[i][j][k]$ has a valid value, we can apply the result directly.

\begin{lstlisting}[style=customc, caption={Backtracking With Memorization}]
int findMaxForm( vector<string>& strs, int m, int n )
{
    //memorization cache
    vector<vector<vector<int>>> M( strs.size(), vector<vector<int>>( m + 1, vector<int>( n + 1, 0 ) ) );

    return F( strs, 0, 0, 0, m, n, M );
}

int F( vector<string>& A, size_t start, int ones, int zeros, int m, int n, vector<vector<vector<int>>>& M )
{
    if( start == A.size() )
    {
        return 0;
    }

    if( M[start][zeros][ones] )
    {
        //directly use cached result
        return M[start][zeros][ones];
    }

    auto get_ones = []( const string & s )
    {
        int count_1s = 0;

        for( auto c : s )
        {
            count_1s += ( c - '0' );
        }

        return count_1s;
    };

    int num_ones = get_ones( A[start] );
    int num_zeros = static_cast< int >( A[start].size() ) - num_ones;

    int sel = 0;

    if( ( ( ones + num_ones ) <= n ) && ( ( zeros + num_zeros ) <= m ) )
    {
        sel = F( A, start + 1, ones + num_ones, zeros + num_zeros, m, n, M ) + 1;
    }

    int non_sel = F( A, start + 1, ones, zeros, m, n, M );

    //save value to cache
    M[start][zeros][ones] = ( max )( sel, non_sel );
    return M[start][zeros][ones];
}
\end{lstlisting}

\subsection{0-1 knapsack problem}
For each string, we have two options: either use or not using it. This is a typical 0--1 knapsack problem. This problem can be solved by using 2-D Dynamic Programming. 

We can make use of an array, $\delta$ such that an entry $\delta[x][y]$ denotes the maximum number of strings that can be included in the subset given only $x$ zeros and $y$ ones are available.

To fill $\delta$, we traverse the given strings. Suppose current string $s$ contains $i$ zeros and $j$ ones. If we choose to put $s$ in any one possible subset, as long as adding $s$ will not exceed given $m$ zeros and $n$ ones, we have $\delta[x][y] = 1 + \delta[x-i][y-j]$.

But we can also choose not to select $s$. In summary, $\delta[x][y] = \max(1 + \delta[x-i][y-j], \delta[x][y])$.

Thus, after the complete list of strings has been traversed, $\delta[m][n]$ gives the answer.

\setcounter{lstlisting}{0}
\begin{lstlisting}[style=customc, caption={0-1 knapsack problem}]
int findMaxForm( vector<string>& strs, int m, int n )
{
    //the search space is from (0,0) to (m,n)
    vector<vector<int>> F( m + 1, vector<int>( n + 1, 0 ) );

    //for m=0 and n=0, we cannot select any string
    //from strs, therefore F[0][0]=0

    for( const auto& s :  strs )
    {
        //get 0s and 1s in s
        int ones = 0;

        for( auto c : s )
        {
            ones += ( c - '0' );
        }

        int zeros = static_cast<int>( s.size() ) - ones;

        for( int i = m; i >= zeros; --i )
        {
            for( int j = n; j >= ones; --j )
            {
                //find in the search space
                //where (i-zeros, j-ones) contains
                //the number of strings that can form with s
                F[i][j] = ( max )( F[i][j], 1 + F[i - zeros][j - ones] );
            }
        }
    }
    return F[m][n];
}
\end{lstlisting}