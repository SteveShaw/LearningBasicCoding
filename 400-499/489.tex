\section{489 --- Robot Room Cleaner}

\textbf{Hard}

Given a robot cleaner in a room modeled as a grid.

Each cell in the grid can be empty or blocked.

The robot cleaner with 4 given APIs can move forward, turn left or turn right. Each turn it made is 90 degrees.

When it tries to move into a blocked cell, its bumper sensor detects the obstacle and it stays on the current cell.

Design an algorithm to clean the entire room using only the 4 given APIs shown below.

\begin{lstlisting}[style=customc]
interface Robot {
  // returns true if next cell is open and robot moves into the cell.
  // returns false if next cell is obstacle and robot stays on the current cell.
  boolean move();

  // Robot will stay on the same cell after calling turnLeft/turnRight.
  // Each turn will be 90 degrees.
  void turnLeft();
  void turnRight();

  // Clean the current cell.
  void clean();
}
\end{lstlisting}

\paragraph{Example:}

\textbf{Input}:

\begin{flushleft}

\[
\begin{bmatrix}
1 & 1 & 1 & 1 & 1 & 0 & 1 & 1\\ 
1 & 1 & 1 & 1 & 1 & 0 & 1 & 1\\ 
1 & 0 & 1 & 1 & 1 & 1 & 1 & 1\\ 
0 & 0 & 0 & 1 & 0 & 0 & 0 & 0\\ 
1 & 1 & 1 & 1 & 1 & 1 & 1 & 1
\end{bmatrix}
\]

\fcj{row = 1},

\fcj{col = 3}


\textbf{Explanation}:

All grids in the room are marked by either 0 or 1.

0 means the cell is blocked, while 1 means the cell is accessible.
The robot initially starts at the position of \fcj{row=1}, \fcj{col=3}.

From the top left corner, its position is one row below and three columns right.
\end{flushleft}

\paragraph{Notes:}

\begin{itemize}
\item The input is only given to initialize the room and the robot's position internally. You must solve this problem \fcj{"blindfolded"}. In other words, you must control the robot using only the mentioned 4 APIs, without knowing the room layout and the initial robot's position.
\item The robot's initial position will always be in an accessible cell.
\item The initial direction of the robot will be facing up.
\item All accessible cells are connected, which means the all cells marked as 1 will be accessible by the robot.
\item Assume all four edges of the grid are all surrounded by wall.
\end{itemize}

\subsection{Spiral Backtracking}
This solution is based on the same idea as maze solving algorithm called \textbf{right-hand rule}. 

Go forward, cleaning and marking all the cells on the way as visited. At the obstacle \textbf{turn right}, again go forward, etc. \textbf{Always turn right} at the obstacles and then \textbf{go forward}. Consider already visited cells as virtual obstacles.

\begin{itemize}
\item What to do if after the right turn there is an obstacle just in front? Turn right again.
\item How to explore the alternative paths from the cell? Go back to that cell and then turn right from your last explored direction.
\item When to stop? \textbf{Stop} when you explored all possible paths, i.e. all 4 directions (up, right, down, and left) for each visited cell. 
\end{itemize}

To implement the algorithm for the backtrack function \fcj{backtrack(cell = (0, 0), direction = 0)}.

\begin{itemize}
\item Mark the cell as visited and clean it up.
\item Explore 4 directions: up, right, down, and left (the order is important since the idea is always to turn right):
\begin{enumerate}
\item Check the next cell in the chosen direction: 
\begin{itemize}
\item if it is not visited, and there is no obstacle: Move forward, explore next cells (\fcj{backtrack(new_cell, new_direction)}), and then backtrack (go back to the previous cell)
\item Otherwise, \textbf{turn} right because now there is an obstacle (or a virtual obstacle) just in front.
\end{itemize}
\end{enumerate}
\end{itemize}

\setcounter{lstlisting}{0}
\begin{lstlisting}[style=customc, caption={DFS}]
void cleanRoom( Robot& robot )
{
    unordered_set<int> seen;
    backtrack( 0, 0, 0, seen, robot );
}
void backtrack( int r, int c, int dir, unordered_set<int>& seen, Robot& robot )
{
    seen.insert( r * 16061 + c );
    robot.clean();

    int dr[4] = {-1, 0, 1, 0};
    int dc[4] = {0, 1, 0, -1};
    //check 4 directions
    for( int i = 0; i < 4; ++i )
    {
        int next_dir = ( dir + i ) % 4;
        int nr = r + dr[next_dir];
        int nc = c + dc[next_dir];
        if( seen.find( nr * 16061 + nc ) == seen.end() )
        {
            if( robot.move() )
            {
                //recursively clean the room
                backtrack( nr, nc, next_dir, seen, robot );
                //go back to previous cell
                robot.turnRight();
                robot.turnRight();
                robot.move();
                robot.turnRight();
                robot.turnRight();
            }
        }
        //turn to next direction
        robot.turnRight();
    }
}
\end{lstlisting}

\paragraph{Related Problems}
\begin{itemize}
\item \textbf{286. Walls and Gates}
\end{itemize}