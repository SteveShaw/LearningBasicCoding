\section{441 --- Arranging Coins}
You have a total of $n$ coins that you want to form in a staircase shape, where every $k$-th row must have exactly $k$ coins.

Given $n$, find the total number of full staircase rows that can be formed.

$n$ is a non-negative integer and fits within the range of a 32-bit signed integer.

\paragraph{Example 1:}

\begin{flushleft}
$n = 5$

The coins can form the following rows:

\begin{figure}[H]
\begin{tikzpicture}
[every node/.style={draw, circle,
 minimum size=6mm, fill=gray!20!},
  node distance=5mm and 5mm, 
  every join/.style={>=stealth,->},
 thick
]
{
[start chain]
\node[on chain](0) {};
}

{
[start chain]
\node[on chain, below=of 0](1) {};
\node[on chain] {};
}

{
[start chain]
\node[on chain, below=of 1] {};
\node[on chain] {};
}
\end{tikzpicture}
\end{figure}

Because the 3rd row is incomplete, we return 2.
\end{flushleft}


\paragraph{Example 2:}

\begin{flushleft}
n = 8

The coins can form the following rows:

\begin{figure}[H]
\begin{tikzpicture}
[every node/.style={draw, circle,
 minimum size=6mm, fill=gray!20!},
  node distance=5mm and 5mm, 
  every join/.style={>=stealth,->},
 thick
]
{
[start chain]
\node[on chain](0) {};
}

{
[start chain]
\node[on chain, below=of 0](1) {};
\node[on chain] {};
}

{
[start chain]
\node[on chain, below=of 1](2) {};
\node[on chain] {};
\node[on chain] {};
}

{
[start chain]
\node[on chain, below=of 2] {};
\node[on chain] {};
}
\end{tikzpicture}
\end{figure}

Because the 4th row is incomplete, we return 3.
\end{flushleft}

\subsection{Binary Search}
其实就是计算最大的整数$x$使得$x\times(x+1)/2\geq n$,可以用binary search的方式进行查找。

\setcounter{lstlisting}{0}
\begin{lstlisting}[style=customc, caption={Binary Search}]
int arrangeCoins( int n )
{
    long long l = 1;
    long long r = n;

    //left most search
    while( l < r )
    {
        long long m = ( l + r ) / 2;

        if( m * ( m + 1 ) / 2 < n )
        {
            l = m + 1;
        }
        else
        {
            r = m;
        }
    }

    //we need to check if l
    //can precisly get the to n
    auto x = ( l + 1 ) * l / 2;
    if( x == n )
    {
        return l;
    }

    return l - 1;

}
\end{lstlisting}
