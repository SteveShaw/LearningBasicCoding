\section{421 --- Maximum XOR of Two Numbers in an Array}
Given a non-empty array of numbers, $a_0, a_1, a_2, \ldots, a_{n-1}$, where $0 \leq a_i < 2^{31}$.

Find the maximum result of $a_i$ \texttt{XOR} $a_j$, where $0 \leq i, j < n$.

Could you do this in $O(n)$ runtime?

\paragraph{Example:}

\begin{flushleft}
\textbf{Input}: $[3, 10, 5, 25, 2, 8]$

\textbf{Output}: 28

Explanation: The maximum result is $5\ \texttt{XOR}\ 25 = 28$.
\end{flushleft}

\subsection{Binary Trie}
\begin{itemize}
\item 将每个number的bits放入一个trie树中,由于bit只有0和1,因此这个trie树实际上是一个二叉树。用left child代表0,right child代表1。
\item 首先遍历数组,将所有number放入trie树中。
\item 再次遍历数组,对当前的number,从高位到低位的每个bit,在trie树中,
\begin{enumerate}
\item 如果当前bit对应的节点有两个child,那么选择和bit不同的那个child,即bit 0选择right child,反之, bit 1选择left child,这样使得xor尽可能的大。
\item 如果bit对应的节点只有一个child,就选择这个child,计算得到的xor值。
\end{enumerate}
\end{itemize}

\setcounter{lstlisting}{0}
\begin{lstlisting}[style=customc, caption={Binary Trie}]
int findMaximumXOR( vector<int>& nums )
{
    trie* root = new trie;

    //fill the trie tree
    for( int n : nums )
    {
        auto t = root;
        for( int b = 31; b >= 0; --b )
        {
            if( n & ( 1 << b ) )
            {
                //bit 1 goes to right
                if( !t->right )
                {
                    t->right = new trie;
                }

                t =  t->right;
            }
            else
            {
                //bit 0 goes to left
                if( !t->left )
                {
                    t->left = new trie;
                }

                t = t->left;
            }

        }
    }

    //calculate maximum xor
    int best = 0;

    for( int n : nums )
    {
        auto t = root;

        int bxor = 0;

        for( int b = 31; b >= 0; --b )
        {
            int x = ( n & ( 1 << b ) ); //x=0..1..0

            int sel = 0; //the selected bit

            if( t->left && t->right )
            {
                if( x )
                {
                    //bit 1 --- select 0
                    t = t->left;
                    sel = 0;
                }
                else
                {
                    //bit 0 --- select 1
                    t = t->right;
                    sel = 1;
                }
            }
            else if( t->left )
            {
                //only has left child
                //select this child
                t = t->left;
                sel = 0;
            }
            else
            {
                //only has right child
                //select this child
                t = t->right;
                sel = 1;
            }

            bxor |= ( x ^ ( sel << b ) );
        }

        best = ( max )( best, bxor );
    }

    return best;
}

struct trie
{
    trie* left = nullptr;
    trie* right = nullptr;
};
\end{lstlisting}

\subsection{Hash Set}
\begin{itemize}
\item 首先按位循环,从最高位到最低位
\item 然后遍历整个数组,与trie树类似,用一个从左往右的 mask,得到当前number的bits前缀,将其存入一个hash set中。
\item 假设最大的xor值为$x$,首先将$x$的当前bit位设为1,然后用到xor的一个性质,即如果$a\ \texttt{xor}\ b = c$,那么$a=b\ \texttt{xor}\ c$。因此遍历hash set中的每个prefix,与$x$进行xor后,看结果能不能在hash set中找到,如果能找到,说明$x$是可以继续增大的,否则保留$x$为当前值。
\end{itemize}

\setcounter{lstlisting}{0}
\begin{lstlisting}[style=customc, caption={Hash Set}]
int findMaximumXOR( vector<int>& nums )
{
    int best = 0;
    int mask = 0;
    for( int b = 31; b >= 0; --b )
    {
        mask |= ( 1 << b );

        unordered_set<int> s_prefix;

        //get all bit prefix
        for( int n : nums )
        {
            s_prefix.insert( mask & n );
        }

        //to maximize xor result
        //best way is to set current bit
        //of the result to 1
        int x = best | ( 1 << b );

        //we need to check if set current bit 1 to
        //the resutl is viable
        for( int p : s_prefix )
        {
            if( s_prefix.count( p ^ x ) )
            {
                //we can find another
                //prefix y so that (p^y=x)
                //therefor, set current bit 1 to
                //the result, best, is viable
                best = x;
                break;
            }
        }
    }

    return best;
}
\end{lstlisting}
