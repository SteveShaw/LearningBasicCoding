\section{493 --- Reverse Pairs}
Given an array $A$, we call $(i, j)$ an important reverse pair if $i < j$ and $A[i] > 2\times A[j]$.

You need to return the number of important reverse pairs in the given array.

\paragraph{Example 1:}

\begin{flushleft}
\textbf{Input}: $[1,3,2,3,1]$

\textbf{Output}: 2

\end{flushleft}

\paragraph{Example 2:}

\begin{flushleft}
\textbf{Input}: $[2,4,3,5,1]$

\textbf{Output}: 3
\end{flushleft}

\paragraph{Note:}
\begin{itemize}
\item The length of the given array will not exceed 50,000.
\item All the numbers in the input array are in the range of 32-bit integer.
\end{itemize}

\subsection{Merge Sort}
We can count the number of reversed pairs during merge process.

\setcounter{lstlisting}{0}
\begin{lstlisting}[style=customc, caption={Merge Sort}]
int reversePairs( vector<int>& nums )
{
    return merge_sort( nums, 0, static_cast<9int>( nums.size() ) - 1 );
}

//merge two halves
void merge( vector<int>& A, int l, int mid, int r )
{
    int sz_l = mid - l + 1;
    int sz_r = r - mid;

    vector<int> L( A.begin() + l, A.begin() + mid + 1 );
    vector<int> R( A.begin() + mid + 1, A.begin() + r + 1 );


    int il = 0;
    int ir = 0;

    int k = 0;

    for( int k = l; k <= r; ++k )
    {
        if( ( ir >= sz_r ) || ( ( il < sz_l ) && ( L[il] < R[ir] ) ) )
        {
            A[k] = L[il];
            ++il;
        }
        else
        {
            A[k] = R[ir];
            ++ir;
        }
    }
}

//modified merge sort
int merge_sort( vector<int>& A, int l, int r )
{
    if( l >= r )
    {
        return 0;
    }

    int mid = ( l + r ) / 2;

    int count = merge_sort( A, l, mid ) + merge_sort( A, mid + 1, r );

    for( int i = l; i <= mid; ++i )
    {
        auto y = static_cast<long long>( A[i] );

        //find the first item in right half
        //that is no less than 2 * A[i]
        int x = leftmost( A, mid + 1, r + 1, y );
        //then, there are (x-(mid+1)) items
        //less than 2 * A[i]
        count += x - ( mid + 1 );
    }

    merge( A, l, mid, r );

    return count;
}

int leftmost( vector<int>& A, int l, int r, long long target )
{
    while( l < r )
    {
        int mid = ( l + r ) / 2;

        auto x = static_cast<long long>( A[mid] ) * 2;

        if( x < target )
        {
            l = mid + 1;
        }
        else
        {
            r = mid;
        }
    }

    return l;
}

\end{lstlisting}


\subsection{BIT}
\begin{itemize}
\item Create an BIT array for sorted array
\item Each item in BIT is the number of items that are larger than current element.
\end{itemize}

\setcounter{lstlisting}{0}
\begin{lstlisting}[style=customc, caption={BIT Tree}]
int reversePairs( vector<int>& nums )
{

    vector<long long> sorted_nums( nums.begin(), nums.end() );
    sort( sorted_nums.begin(), sorted_nums.end() );

    vector<int> BIT( nums.size() + 1, 0 );

    int ans = 0;

    for( size_t i = 0; i < nums.size(); ++i )
    {
        auto target = static_cast<long long>( nums[i] );

        //find the first item in sorted array
        //that is larger than nums[i]*2
        auto pos = rightmost( sorted_nums, target * 2 );

        //query how many items are larger than nums[i]
        //through BIT. Since BIT is updated when travesing
        //nums, the count is updated accordingly.
        ans += queryBIT( BIT, pos + 1 );

        //find the position of the first item in sorted array
        //that is no less than nums[i]
        pos = leftmost( sorted_nums, target );

        //update from 0 until pos+1, the count of items
        //that is no less than nums[i]
        //in sorted array
        updateBIT( BIT, pos + 1, 1 );
    }

    return ans;
}

size_t leftmost( vector<long long>& B, long long target )
{
    size_t l = 0;
    size_t r = B.size();

    while( l < r )
    {
        auto mid = ( l + r ) / 2;

        if( B[mid] < target )
        {
            l = mid + 1;
        }
        else
        {
            r = mid;
        }
    }

    return l;
}

size_t rightmost( vector<long long>& B, long long target )
{
    size_t l = 0;
    size_t r = B.size();

    while( l < r )
    {
        auto mid = ( l + r ) / 2;

        if( B[mid] <= target )
        {
            l = mid + 1;
        }
        else
        {
            r = mid;
        }
    }

    return l;
}

void updateBIT( vector<int>& B, size_t index, int val )
{
    while( index > 0 )
    {
        B[index] += val;
        index -= ( index & ( -index ) );
    }
}

int queryBIT( vector<int>& B, size_t index )
{
    int sum = 0;

    while( index < B.size() )
    {
        sum += B[index];
        index += ( index & ( -index ) );
    }

    return sum;
}
\end{lstlisting}