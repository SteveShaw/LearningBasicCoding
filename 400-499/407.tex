\section{407 --- Trapping Rain Water II}
Given an $m \times n$ matrix of positive integers representing the height of each unit cell in a 2D elevation map, compute the volume of water it is able to trap after raining.

\paragraph{Note:}

\begin{flushleft}
Both $ m $ and $ n $ are less than 110. The height of each unit cell is greater than 0 and is less than 20,000.

\end{flushleft}
 

\paragraph{Example:}

\begin{flushleft}
Given the following $3\times 6$ height map:
\begin{table}[H]
\begin{tabular}{*{6}{c}}
1 & 4 & 3 & 1 & 3 & 2 \\ 
3 & 2 & 1 & 3 & 2 & 4 \\
2 & 3 & 3 & 2 & 3 & 1
\end{tabular}
\end{table}
Return 4.
\end{flushleft}

\subsection{Priority Queue}
\begin{itemize}
\item Problem 42 是该题的2D形式,解决问题的idea来自于著名的 Liebig's law of the minimum 即 The maximum height of water for current container is determined by the lowest part of its boundary.
\item 算法需要从boundary开始,逐步推进到内部。在这个过程中,首先从bounday最低的部分开始,然后搜索其neighbors,获得其中能够得到的盛水量,然后将其neighbor cell的坐标,和该cell与当前cell的高度相比较更大的那个高度作为cell放入$q$。这个高度形成了新的boundary。
\item 本题是3D形式,因此每个格子都会有四个neighbors,除了边界的以外,因此maintain 一个priority queue $q$,将boundary中所有cell都放入$q$,然后从高度最低的部分开始进行四领域DFS。同时需要maintain 一个 2d array 来mark当前cell是否已经访问过。
\item 由于Priority Queue需要把最小高度的cell放在top,因此这是一个min prority queue。
\end{itemize}

\setcounter{lstlisting}{0}
\begin{lstlisting}[style=customc, caption={Priority Queue}]
int trapRainWater( vector<vector<int>>& heightMap )
{

    if( heightMap.empty() || heightMap[0].empty() )
    {
        return 0;
    }

    int rows = static_cast<int>( heightMap.size() );
    int cols = static_cast<int>( heightMap[0].size() );

    //we need a min priority_queue, so the comparison function
    //need return true for larger one
    auto cmp = []( const cell & c1, const cell & c2 )
    {
        return c1.h > c2.h;
    };

    priority_queue<cell, vector<cell>, decltype( cmp )> pq( cmp );

    //mark if cell at (r,c) is visited or not
    vector<vector<unsigned char>> seen( rows, vector<unsigned char>( cols, 0 ) );

    //put cells on boundaries
    for( int r = 0; r < rows; ++r )
    {
        pq.emplace( r, 0, heightMap[r][0] );
        pq.emplace( r, cols - 1, heightMap[r][cols - 1] );
        seen[r][0] = 1;
        seen[r][cols - 1] = 1;
    }
    for( int c = 1; c < cols - 1; ++c )
    {
        pq.emplace( 0, c, heightMap[0][c] );
        pq.emplace( rows - 1, c, heightMap[rows - 1][c] );
        seen[0][c] = 1;
        seen[rows - 1][c] = 1;
    }

    int dr[4] = {1, 0, -1, 0};
    int dc[4] = {0, 1, 0, -1};

    int ans = 0;
    while( !pq.empty() )
    {
        auto t = pq.top();
        pq.pop();

        for( int i = 0; i < 4; ++i )
        {
            int nr = dr[i] + t.r;
            int nc = dc[i] + t.c;

            if( ( nr < 0 ) || ( nc < 0 ) || ( nr >= rows ) || ( nc >= cols ) )
            {
                continue;
            }

            if( seen[nr][nc] )
            {
                continue;
            }

            int nh = heightMap[nr][nc];

            //only the water is contained
            //if the neighbor height is less
            //than current cell's height
            if( t.h > heightMap[nr][nc] )
            {
                //the new boundary height
                //must be the greater one of
                //current cell and its neighbor
                nh = t.h;
                ans += ( t.h - heightMap[nr][nc] );
            }

            //add the new cell into the queue
            pq.emplace( nr, nc, nh );

            //mark this cell is visited
            seen[nr][nc] = 1;
        }
    }

    return ans;

}

struct cell
{
    int r;
    int c;
    int h;

    cell( int r_, int c_, int h_ )
        : r( r_ ), c( c_ ), h( h_ )
    {}
};
\end{lstlisting}

