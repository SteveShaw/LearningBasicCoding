\section{428 --- Serialize and Deserialize N-ary Tree}
\textcolor{red}{Locked}
\\
Serialization is the process of converting a data structure or object into a sequence of bits so that it can be stored in a file or memory buffer, or transmitted across a network connection link to be reconstructed later in the same or another computer environment.
\par
Design an algorithm to \textbf{serialize} and \textbf{deserialize} an $N$-ary tree. An $N$-ary tree is a rooted tree in which each node has no more than $N$ children. There is no restriction on how your \textbf{serialization}/\textbf{deserialization} algorithm should work. You just need to ensure that an $ N $-ary tree can be serialized to a string and this string can be deserialized to the original tree structure.

For example, you may serialize the following 3-ary tree

\begin{figure}[H]
\begin{tikzpicture}
[my/.style={draw, circle, minimum size=7mm, fill=gray!20!}]
\node[my](0) at (0,0) {1};
\node[my](1) [below=8mm of 0, xshift=-12mm] {3};
\node[my](2) [below=8mm of 0] {2};
\node[my](3) [below=8mm of 0, xshift=12mm] {4};
\node[my](4) [below=8mm of 1, xshift=-10mm] {5};
\node[my](5) [below=8mm of 1, xshift=10mm] {6};
\draw[thick] (0)--(1)--(4);
\draw[thick] (1)--(5);
\draw[thick] (0)--(2);
\draw[thick] (0)--(3);
\end{tikzpicture}
\end{figure}
as [1 [3[5 6] 2 4]]. You do not necessarily need to follow this format, so please be creative and come up with different approaches yourself.

\paragraph{Note:}
\begin{itemize}
\item $ N $ is in the range of $[1, 1000]$
\item Do not use class member/global/static variables to store states. Your serialize and deserialize algorithms should be stateless.
\end{itemize}

\subsection{Preorder Traverse}
\begin{itemize}
\item Serialize: 
\begin{enumerate}
\item Fo a null node, just return.
\item Otherwise, add the node's value first, then add a space, and then the number of its children, and then add a space again.
\item If it has children, recursively traverse into each child node, do the above operation again.
\item Finally, if the generated string is not empty, we remove the last space from the string.
\end{enumerate}
\item Deserialize: We may need \texttt{istringstream} to help read the string as a input stream.
\begin{enumerate}
\item Check if we read at the end of string. If it is, return null node.
\item Read value and number of children from the stream
\item Create a new node with the value.
\item By the number of children, we add a recursively-built node into the children array of current node.
\item Finally, return the current node
\end{enumerate}
\end{itemize}

\setcounter{lstlisting}{0}
\begin{lstlisting}[style=customc, caption={Recursion}]
class Codec
{
public:
    // Encodes a tree to a single string.
    string serialize( Node* root )
    {

        string ans;

        preorder( root, ans );

        if( !ans.empty() )
        {
            //remove the last space character
            ans.pop_back();
        }

        return ans;
    }

    // Decodes your encoded data to tree.
    Node* deserialize( string data )
    {

        if( data.empty() )
        {
            return nullptr;
        }

        istringstream iss( data );

        return build( iss );
    }

    void preorder( Node* node, string& s )
    {
        if( !node )
        {
            return;
        }

        s += to_string( node->val );
        s.push_back( ' ' ); //add space to separate

        if( node->children.empty() )
        {
            s.push_back( '0' );
            s.push_back( ' ' ); //add space to separate
        }
        else
        {
            s += to_string( node->children.size() );
            s.push_back( ' ' ); //add space to separate

            for( auto node : node->children )
            {
                preorder( node, s );
            }
        }
    }

    Node* build( istringstream& iss )
    {
        if( iss.eof() )
        {
            //read at the end of string
            return nullptr;
        }

        int val;
        int len;

        //get value and number of children
        iss >> val >> len;

        Node* node = new Node;
        node->val = val;

        if( len )
        {
            node->children.reserve( len );
            for( int i = 0; i < len; ++i )
            {
                //recurisively build children
                node->children.push_back( build( iss ) );
            }
        }

        return node;
    }


};

// Your Codec object will be instantiated and called as such:
// Codec codec;
// codec.deserialize(codec.serialize(root));
\end{lstlisting}
