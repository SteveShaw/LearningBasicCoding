\section{447 --- Number of Boomerangs}
Given $n$ points in the plane that are all pairwise distinct, a \textbf{boomerang} is a tuple of points $(i, j, k)$ such that the distance between $i$ and $j$ equals the distance between $i$ and $k$ (the order of the tuple matters).

Find the number of \textbf{boomerang}s. You may assume that $n$ will be at most 500 and coordinates of points are all in the range $[-10000, 10000]$ (inclusive).

\paragraph{Example:}

\begin{flushleft}
\textbf{Input}: $[\, [0,0],[1,0],[2,0]\,]$

\textbf{Output}: 2

\textbf{Explanation}: The two boomerangs are $[\,[1,0],[0,0],[2,0]\,]$ and $[\,[1,0],[2,0],[0,0]\,]$

\end{flushleft}

\subsection{Hash Map}
\begin{itemize}
\item 对每一个点,计算其他所有点到这个点的距离,然后maintain一个hash map用来记录到这个点具有相同的距离的点的个数。
\item 在计算出所有的点后, 统计hash map中每个距离对应的点的个数,suppose it is $n$,那么这些点可以形成$n\times (n-1)/2$个pair,再和当前点构成bommerang。
因此结果要加上$n\times (n-1)/2$。
\end{itemize}

\setcounter{lstlisting}{0}
\begin{lstlisting}[style=customc, caption={Hash Map}]
int numberOfBoomerangs( vector<vector<int>>& points )
{
    int ans = 0;

    for( size_t i = 0; i < points.size(); ++i )
    {
        unordered_map<int, int> m;

        for( size_t j = 0; j < points.size(); ++j )
        {
            if( i == j )
            {
                continue;
            }
            auto& pi = points[i];
            auto& pj = points[j];

            int sq_dist = ( pi[0] - pj[0] ) * ( pi[0] - pj[0] ) + ( pi[1] - pj[1] ) * ( pi[1] - pj[1] );

            //add to the hash map
            //per the disance
            auto it = m.find( sq_dist );

            if( it == m.end() )
            {
                m.emplace( sq_dist, 1 );
            }
            else
            {
                it->second += 1;
            }
        }

        for( const auto& p : m )
        {
            int x = p.second;
            ans += x * ( x - 1 );
        }
    }

    return ans;
}
\end{lstlisting}