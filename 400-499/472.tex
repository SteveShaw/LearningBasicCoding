\section{472 --- Concatenated Words}
Given a list of words $A$ (\textbf{without duplicates}), please write a program that returns all concatenated words in the given list of words.

A concatenated word is defined as a string that is comprised entirely of at least two shorter words in the given array.

\paragraph{Example:}

\begin{flushleft}
\textbf{Input}: [\texttt{cat}, \texttt{cats}, \texttt{catsdogcats}, \texttt{dog}, \texttt{dogcatsdog}, \texttt{hippopotamuses}, \texttt{rat}, \texttt{ratcatdogcat}]

\textbf{Output}: [\texttt{catsdogcats}, \texttt{dogcatsdog}, \texttt{ratcatdogcat}]

\textbf{Explanation}: 

\texttt{catsdogcats} can be concatenated by \texttt{cats}, \texttt{dog} and \texttt{cats}; 

\texttt{dogcatsdog} can be concatenated by \texttt{dog}, \texttt{cats} and \texttt{dog}; 

\texttt{ratcatdogcat} can be concatenated by \texttt{rat}, \texttt{cat}, \texttt{dog} and \texttt{cat}.

\end{flushleft}

\paragraph{Note}:
\begin{itemize}
\item The number of elements of the given array will not exceed 10,000
\item The length sum of elements in the given array will not exceed 600,000.
\item All the input string will only include lower case letters.
\item The returned elements order does not matter.
\end{itemize}

\subsection{Build Trie Tree With Depth First Search}
We build a trie tree from all the input string. For the trie assoicated with the last letter in each string, we set the flag field to \lstinline[language=C++, basicstyle=\small\ttfamily, keywordstyle=\bfseries\color{green!40!black}]|true| to indicate this trie indicates an input string.

To check if a string is composed of the other strings, we make use of a depth first search on the built trie tree
\begin{enumerate}
\item Iterate each letter in the string.
\item If the associated trie's flag is \lstinline[language=C++, basicstyle=\small\ttfamily, keywordstyle=\bfseries\color{green!40!black}]|true|, it means we find a substring that is also in the input string. Then we recursively go to next letter and repeat the process from the root of the trie tree.
\item We need to try each letter from current index to end of the string.
\end{enumerate}

\setcounter{lstlisting}{0}
\begin{lstlisting}[style=customc, caption={DFS And Trie}]
vector<string> findAllConcatenatedWordsInADict( vector<string>& words )
{
    Trie* root = new Trie;

    //build trie
    for( const auto& w : words )
    {
        if( w.empty() )
        {
            continue;
        }

        build( root, w );
    }

    //find the candidates
    vector<string> ans;

    for( const auto& w :  words )
    {
        if( w.empty() )
        {
            continue;
        }

        if( dfs( w.c_str(), w.size(), 0, 0, root ) )
        {
            ans.emplace_back( w.c_str() );
        }
    }

    return ans;
}


private:


struct Trie
{
    unordered_map<char, Trie*> m_nexts;
    bool is_word = false;

    Trie() = default;
};

bool dfs( const char* word, size_t len, size_t start, int parts, Trie* root )
{
    auto node = root;

    for( size_t i = start; i < len; ++i )
    {
        auto it = node->m_nexts.find( word[i] );

        if( it == node->m_nexts.end() )
        {
            return false;
        }

        node = it->second;

        if( node->is_word )
        {
            //keep checking word[i+1...len-1]
            if( i == len - 1 )
            {
                //already at the end of word
                //if parts >=1, this means
                //we have already found
                //at least one substring in the words
                //now from the last matched position to i
                //there is another word can be matched
                //so the condition is parts>=1
                //not parts >1
                return parts >= 1;
            }

            //if we can find from i+1
            //we find the anwer
            if( dfs( word, len, i + 1, parts + 1, root ) )
            {
                return true;
            }
        }
    }

    return false;
}

void build( Trie* root, const string& w )
{
    auto node = root;

    for( char c : w )
    {
        auto it = node->m_nexts.find( c );
        if( it == node->m_nexts.end() )
        {
            it = node->m_nexts.emplace( c, new Trie() ).first;
        }

        node = it->second;
    }

    node->is_word = true;
}

\end{lstlisting}

\subsection{Dynamic Programming}
We can refer to (Problem 139 --- Word Break) for this approach. In summary, 

\begin{itemize}
\item Put all the input strings into a hash set.
\item Suppose $F[i]$ is the dynamic program function, which means in current string $S$, sub-string $S[0, i-1]$ can be composed of other input strings. To cover empty string case, $F$ has length $\ell+1$ where $\ell$ is the length of $S$.
\item We need two loops to check each string $S$. The outermost loop is iterate each index $i$ in $S$, and the inner one is checking index $j$ from $0$ to $i-1$ and set value of $F[i]$. 

\begin{lstlisting}[style=customc, caption={}]
for( size_t i = 0; i <= S.size(); ++i )
{
    for( size_t j = 0; j < i; ++j )
    {
        if( ( F[j] == true ) && ( HashSet.contains( S.substr( j, i - j ) ) ) )
        {
            //S[0,j] can be composed from input string
            //and S[j+1, i] is also one of input strings.
            F[i] = true;
            break;
        }
    }
}
\end{lstlisting}
\item If $F[\ell]$ is \lstinline[language=C++, basicstyle=\small\ttfamily, keywordstyle=\bfseries\color{green!40!black}]|true|, $S$ is the one meets the requirements and add it to the output list.
\item When processing each string, we need to temporarily removing it from the hash set to avoid matching itself.
\end{itemize}

\begin{lstlisting}[style=customc, caption={}]
vector<string> findAllConcatenatedWordsInADict( vector<string>& words )
{
    unordered_set<string> hash_set( begin( words ), end( words ) );

    vector<string> ans;

    for( const auto& word : words )
    {
        //F[i] indicates if word[0, i-1] can be
        //composed from input strings

        if( word.empty() )
        {
            continue;
        }

        hash_set.erase( word );
        vector<unsigned char> F( word.size() + 1, 0 );

        F[0] = 1;
        for( size_t i = 0; i <= word.size(); ++i )
        {
            //for each j between[0,i-1]
            //we use the saved result F[j]
            //and check substring word[j, i-j]

            for( size_t j = 0; j < i; ++j )
            {
                if( F[j] && hash_set.count( word.substr( j, i - j ) ) )
                {
                    //this means word[0,i-1] can be composed
                    //at least two of input strings
                    F[i] = 1;
                    break;
                }
            }
        }

        if( F[word.size()] == 1 )
        {
            //word meets the requirement
            ans.push_back( word );
        }

        hash_set.insert( word );
    }

    return ans;
}
\end{lstlisting}