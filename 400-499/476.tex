\section{476 --- Number Complement}
Given a positive integer $n$, output its complement number. The complement strategy is to flip the bits of its binary representation.

\paragraph{Note:}
\begin{itemize}
\item The given integer is guaranteed to fit within the range of a 32-bit signed integer.
\item You could assume no leading zero bit in the integer’s binary representation.
\end{itemize}

\paragraph{Example 1:}
\begin{flushleft}
\textbf{Input}: 5

\textbf{Output}: 2

\textbf{Explanation}: The binary representation of 5 is 101 (no leading zero bits), and its complement is 010. So you need to output 2.
\end{flushleft}

\paragraph{Example 2:}
\begin{flushleft}
\textbf{Input}: 1

\textbf{Output}: 0

\textbf{Explanation}: The binary representation of 1 is 1 (no leading zero bits), and its complement is 0. So you need to output 0.
\end{flushleft}

\subsection{Binary Operation}

\setcounter{lstlisting}{0}
\begin{lstlisting}[style=customc, caption={Binary Operation}]
int findComplement( int num )
{
    int x = INT_MAX;
    int mask = 1;

    //generate 1111...1
    //the highest bit 1 is same as the position
    //of highest bit 1 of num
    for( int i = 31; i >= 0; --i )
    {
        if( ( num ) & ( 1 << i ) )
        {
            break;
        }

        mask = ( x >> ( 31 - i ) );
    }

    //xor is the result
    return num ^ mask;
}
\end{lstlisting}