\section{454 --- 4Sum II}
Given four lists $A$, $B$, $C$, $D$ of integer values, compute how many tuples $(i, j, k, l)$ there are such that $A[i] + B[j] + C[k] + D[l]$ is zero.

To make problem a bit easier, all $A, B, C, D$ have same length of $N$ where $0 \leq N \leq 500$. All integers are in the range of $-2^{28}$ to $2^{28} - 1$ and the result is guaranteed to be at most $2^{31} - 1$.

\paragraph{Example:}

\begin{flushleft}
\textbf{Input}:

$A = [ 1, 2],\, B = [-2,-1],\, C = [-1, 2], \,D = [ 0, 2]$

\textbf{Output}: 2

\textbf{Explanation}:

The two tuples are:

\begin{enumerate}
\item $(0, 0, 0, 1) \longrightarrow A[0] + B[0] + C[0] + D[1] = 1 + (-2) + (-1) + 2 = 0$
\item $(1, 1, 0, 0) -> A[1] + B[1] + C[0] + D[0] = 2 + (-1) + (-1) + 0 = 0$
\end{enumerate}
\end{flushleft}

\subsection{Hash Map}

\begin{itemize}
\item Put the sum of all the element pairs from $A$ and $B$ into a hash map $M$.
\item Then the problem reduced to find the sum of two numbers from $C$ and $D$, say $x$, so that $-x$ can be found in $M$.
\end{itemize}

\setcounter{lstlisting}{0}
\begin{lstlisting}[style=customc, caption={hash map}]
int fourSumCount( vector<int>& A, vector<int>& B, vector<int>& C, vector<int>& D )
{
    unordered_map<int, int> m;

    //get the sum of A[i]+[j]
    for( size_t i = 0; i < A.size(); ++i )
    {
        for( size_t j = 0; j < B.size(); ++j )
        {
            int sum = A[i] + B[j];

            auto it = m.find( sum );

            if( it == m.end() )
            {
                m.emplace( sum, 1 );
            }
            else
            {
                it->second += 1;
            }
        }
    }
    int ans = 0;
    //find if C[k]+D[l]=-A[i]-B[j]
    for( size_t i = 0; i < C.size(); ++i )
    {
        for( size_t j = 0; j < D.size(); ++j )
        {
            int sum = C[i] + D[j];
            auto it = m.find( -sum );

            if( it != m.end() )
            {
                //each pair (i,j) will have a unique combination
                //with current (k,l)
                ans += it->second;
            }
        }
    }
    return ans;
}
\end{lstlisting}
