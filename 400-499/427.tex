\section{427 --- Construct Quad Tree}
We want to use quad trees to store an $N \times N$ boolean grid. Each cell in the grid can only be \texttt{true} or \texttt{false}. The root node represents the whole grid. For each node, it will be subdivided into four children nodes \textbf{until the values in the region it represents are all the same}.

Each node has another two boolean attributes : \texttt{isLeaf} and \texttt{val}. \texttt{isLeaf} is \texttt{true} if and only if the node is a leaf node. The \texttt{val} attribute for a leaf node contains the value of the region it represents.

Your task is to use a quad tree to represent a given grid. The following example may help you understand the problem better:

Given the 8 x 8 grid below, we want to construct the corresponding quad tree:


\subsection{Depth First Search From Top To Bottom}
\begin{itemize}
\item In this method, each time, we check if the large grid has one uniform value or not. If it is, we create a leaf node to represent this grid.
\item Otherwise, we split the grid into four parts and recursively apply the same process.
\item To speed up the checking, we may use 2D prefix sum of the grid.
\end{itemize}

\setcounter{lstlisting}{0}
\begin{lstlisting}[style=customc, caption={Build From Top To Bottom}]
Node* construct( vector<vector<int>>& grid )
{
    int N = static_cast<int>( grid.size() );

    vector<vector<int>> g_sum( N + 1, vector<int>( N + 1, 0 ) );

    //build 2d prefix sum
    //for input grid
    for( int r = 0;  r < N; ++r )
    {
        for( int c = 0; c < N; ++c )
        {
            g_sum[r + 1][c + 1] = g_sum[r + 1][c] + g_sum[r][c + 1] - g_sum[r][c] + grid[r][c];
        }
    }

    //run top to bottom build
    return dfs( grid, 0, 0, N - 1, N - 1, g_sum );
}

Node* dfs( vector<vector<int>>&grid, int top, int left, int bottom, int right, vector<vector<int>>& g_sum )
{
    if( ( top > bottom ) || ( left > right ) )
    {
        //base case
        return nullptr;
    }

    //get the sum of grid[top...bottom][left...right]
    int sum = g_sum[bottom + 1][right + 1] -
              g_sum[top][right + 1] -
              g_sum[bottom + 1][left]
              + g_sum[top][left];

    //the whole sum if grid[top...bottom][left...right] all are 1s or 0s
    int area = ( bottom - top + 1 ) * ( right - left + 1 ) * grid[top][left];

    if( sum == area )
    {
        //grid[top...bottom][left...right]
        //are all same values
        //it forms a leaf node
        Node* node = new Node;
        //must set all 4 nodes
        //to nullptr, otherwise will
        //cause exception
        node->topLeft = nullptr;
        node->topRight = nullptr;
        node->bottomLeft = nullptr;
        node->bottomRight = nullptr;

        node->val = grid[top][left] ? true : false;
        node->isLeaf = true;

        return node;
    }

    //divide grid[top...bottom][left...right] into 4 parts
    //from the middle of row and column
    int mid_r = ( top + bottom ) / 2;
    int mid_c = ( left + right ) / 2;

    Node* node = new Node;
    node->topLeft = dfs( grid, top, left, mid_r, mid_c, g_sum );
    node->topRight = dfs( grid, top, mid_c + 1, mid_r, right, g_sum );
    node->bottomLeft = dfs( grid, mid_r + 1, left, bottom, mid_c, g_sum );
    node->bottomRight = dfs( grid, mid_r + 1, mid_c + 1, bottom, right, g_sum );

    node->isLeaf = false;
    //must set node->val = false
    //otherwise, it will cause exception
    node->val = false;

    return node;
}
\end{lstlisting}

\subsection{Depth First Search From Bottom To Top}
\begin{itemize}
\item We can build from each element of the grid.
\item Then, we combine 4 connected nodes.
\item If the four nodes are all leaves and same values, we can combine them into one leaf node and delete these 4 nodes.
\item Otherwise, create a node to contains these 4 nodes.
\item Recursively apply the above process to the input grid.
\end{itemize}

\begin{lstlisting}[style=customc, caption={Build From Bottom To Top}]
Node* construct( vector<vector<int>>& grid )
{
    int N = static_cast<int>( grid.size() );
    return dfs( grid, 0, 0, N );
}

Node* dfs( vector<vector<int>>&grid, int x, int y, int size )
{
    if( size == 1 )
    {
        //this is each element in the grid
        return new Node{grid[x][y] == 1, true, nullptr, nullptr, nullptr, nullptr};
    }

    auto tl = dfs( grid, x, y, size / 2 );
    auto tr = dfs( grid, x, y + size / 2, size / 2 );
    auto bl = dfs( grid, x + size / 2, y, size / 2 );
    auto br = dfs( grid, x + size / 2, y + size / 2, size / 2 );

    if( ( tl->isLeaf && tr->isLeaf && bl->isLeaf && br->isLeaf ) )
    {
        if( ( tl->val == tr->val ) && ( bl->val == br->val ) && ( tl->val == bl->val ) )
        {
            //we can merge these 4 nodes
            //into one leaf node
            auto node = new Node;
            node->isLeaf = true;
            node->val = tl->val;

            node->topLeft = nullptr;
            node->topRight = nullptr;
            node->bottomLeft = nullptr;
            node->bottomRight = nullptr;

            //delete these nodes
            delete tl;
            delete tr;
            delete bl;
            delete br;

            return node;
        }
    }

    //create a new node to contain these 4 nodes
    return new Node{false, false, tl, tr, bl, br};
}
\end{lstlisting}