\section{465 --- Optimal Account Balancing}
A group of friends went on holiday and sometimes lent each other money. For example, Alice paid for Bill's lunch for \$10. Then later Chris gave Alice \$5 for a taxi ride. We can model each transaction as a tuple $(x, y, z)$ which means person $x$ gave person $y$ \$z. Assuming Alice, Bill, and Chris are person 0, 1, and 2 respectively (0, 1, 2 are the person's ID), the transactions can be represented as $ [[0, 1, 10], [2, 0, 5]] $.

Given a list of transactions between a group of people, $T$, return the minimum number of transactions required to settle the debt.

\paragraph{Note:}

\begin{itemize}
\item A transaction will be given as a tuple $ (x, y, z) $. Note that $x \neq y$ and $z > 0$.

\item Person's IDs may not be linear, e.g. we could have the persons 0, 1, 2 or we could also have the persons 0, 2, 6.
\end{itemize}

\paragraph{Example 1:}

\begin{flushleft}
\textbf{Input}:

$[[0,1,10], [2,0,5]]$

\textbf{Output}: 2

\textbf{Explanation}:

Person \#0 gave person \#1 \$10.

Person \#2 gave person \#0 \$5.


Two transactions are needed. One way to settle the debt is person \#1 pays person \#0 and \#2 \$5 each.
\end{flushleft}


\paragraph{Example 2:}

\begin{flushleft}
\textbf{Input}:

$ [[0,1,10], [1,0,1], [1,2,5], [2,0,5]] $

\textbf{Output}: 1

\textbf{Explanation}:
Person \#0 gave person \#1 \$10.
Person \#1 gave person \#0 \$1.
Person \#1 gave person \#2 \$5.
Person \#2 gave person \#0 \$5.

Therefore, person \#1 only need to give person \#0 \$4, and all debt is settled.
\end{flushleft}

\subsection{Depth First Search}
\begin{itemize}
\item Calculate each person's overall balance $B[i]$. $B[i]>0$ means person $i$ needs to pay $B[i]$ to other persons, while $B[i]<0$ person $i$ needs to get money from other people. 
\item We start from the first person with $B[0]$, skip all persons with $B[i]=0$ and try to find the first person that has opposite sign to $B[0]$, say, $B[j]$. We use $B[j]\gets B[j]+B[0]$ to clear $B[0]$.
\item From now on, $B[0]$ is cleared, and we recursively set other $B[k]$ to zero one by one until all $B[j]$ are zeros. At this time, update the global minimum transactions.
\end{itemize}

\setcounter{lstlisting}{0}
\begin{lstlisting}[style=customc, caption={DFS}]
int minTransfers( vector<vector<int>>& transactions )
{
    //get each person's overall balance
    unordered_map<int, int> m_bal;

    auto add_map = [&m_bal]( int id, int money )
    {
        auto it  = m_bal.find( id );
        if( it == m_bal.end() )
        {
            m_bal.emplace( id, money );
        }
        else
        {
            it->second += money;
        }
    };

    for( const auto& trans : transactions )
    {
        add_map( trans[0], -trans[2] );
        add_map( trans[1], +trans[2] );
    }

    //change to vector
    vector<int> v_bal;
    v_bal.reserve( m_bal.size() );

    for( const auto& p : m_bal )
    {
        if( p.second != 0 )
        {
            v_bal.push_back( p.second );
        }
    }

    int ans = INT_MAX;
    dfs( v_bal, 0, 0, ans );

    return ans;
}

void dfs( vector<int>& A, size_t start, int steps, int& ans )
{
    //skip person with zero balance
    while( ( start < A.size() ) && ( A[start] == 0 ) )
    {
        ++start;
    }

    if( start == A.size() )
    {
        //all persons are cleared
        //update minimum transactions
        ans = ( min )( ans, steps );
        return;
    }

    //try other persons
    auto next = start + 1;

    while( next < A.size() )
    {
        if( A[next] * A[start] < 0 )
        {
            //person next can help
            //clear person start
            int x = A[start];
            A[next] += x;

            //we seach from person (start+1)
            //increments the transactions
            dfs( A, start + 1, steps + 1, ans );

            //backtrack
            A[next] -= x;
        }
        ++next;
    }
}
\end{lstlisting}