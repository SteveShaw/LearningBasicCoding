\section{445 --- Add Two Numbers II}
You are given two non-empty linked lists representing two non-negative integers. The most significant digit comes first and each of their nodes contain a single digit. Add the two numbers and return it as a linked list.

You may assume the two numbers do not contain any leading zero, except the number 0 itself.

\paragraph{Follow up:}

\begin{itemize}
\item What if you cannot modify the input lists? In other words, reversing the lists is not allowed.
\end{itemize}

\paragraph{Example:}
\begin{flushleft}

\textbf{Input}:

\begin{figure}[H]
\begin{tikzpicture}
[every node/.style={draw, circle,
 minimum size=6mm, fill=gray!20!},
  node distance=8mm, 
  every join/.style={>=stealth,->},
 thick
]
{
[start chain]
\node[on chain](0) {7};
\node[on chain, join] {2};
\node[on chain, join] {4};
\node[on chain, join] {3};
}

{
[start chain]
\node[on chain,below = 7mm of 0] {5};
\node[on chain, join] {6};
\node[on chain, join] {4};
}

\end{tikzpicture}
\end{figure}

\textbf{Output}:

\begin{figure}[H]
\begin{tikzpicture}
[start chain, 
every node/.style={draw, circle,
 minimum size=6mm, fill=gray!20!},
  node distance=8mm, 
  every join/.style={>=stealth,->},
 thick
]
\node[on chain] {7};
\node[on chain, join] {8};
\node[on chain, join] {0};
\node[on chain, join] {7};
\end{tikzpicture}
\end{figure}

\end{flushleft}


\subsection{Record Last Non-Nine Node}
\begin{itemize}
\item 技巧是记录链表中最后一个值不是9的node的位置,然后从这个位置往后,如果有进位发生,则该位置上的值加1,而后面的则都变为零。
\end{itemize}


\setcounter{lstlisting}{0}
\begin{lstlisting}[style=customc, caption={Record Last Non-Nine Node}]
/**
 * Definition for singly-linked list.
 * struct ListNode {
 *     int val;
 *     ListNode *next;
 *     ListNode(int x) : val(x), next(NULL) {}
 * };
 */
ListNode* addTwoNumbers( ListNode* l1, ListNode* l2 )
{

    int len1 = get_length( l1 );
    int len2 = get_length( l2 );

    //always set l1 as the longer list
    if( len1 < len2 )
    {
        swap( l1, l2 );
        swap( len1, len2 );
    }

    ListNode* dummy = new ListNode( 0 );

    auto cur = dummy;

    //the last node with value < 9
    ListNode* last = cur;

    int diff = len1 - len2;

    auto x = l1;
    auto y = l2;

    //we first link the pre-parts of l1
    while( diff )
    {
        cur->next = new ListNode( x->val );

        if( x->val != 9 )
        {
            last = cur->next;
        }

        cur = cur->next;
        x = x->next;

        --diff;
    }

    //now do addition for both lists
    while( y )
    {
        int val_x = x->val;
        int val_y = y->val;

        int sum = val_x + val_y;

        if( sum >= 10 )
        {
            sum -= 10;

            //the last node with
            //value < 9
            //increments its value
            auto z = last;
            z->val += 1;

            //set values of all nodes after this node
            //in the new list to zero
            //since 9+1=10
            while( z->next )
            {
                z->next->val = 0;
                z = z->next;
            }

            //since its value set to zero
            //this will be a possible last node
            //with value < 9
            last = z;
        }

        //Add a new node to the new list
        cur->next = new ListNode( sum );
        cur = cur->next;

        if( cur->val != 9 )
        {
            //update the last node in the new list
            //with value < 9
            last = cur;
        }

        x = x->next;
        y = y->next;
    }

    if( dummy->val == 0 )
    {
        return dummy->next;
    }

    //The first node has 1
    //we need to return the whole list
    return dummy;
}

//helper function to get length of each list
int get_length( ListNode* head )
{
    auto node = head;

    int l = 0;

    while( node )
    {
        ++l;
        node = node->next;
    }

    return l;
}

\end{lstlisting}


