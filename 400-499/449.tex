\section{449 --- Serialize and Deserialize BST}
Serialization is the process of converting a data structure or object into a sequence of bits so that it can be stored in a file or memory buffer, or transmitted across a network connection link to be reconstructed later in the same or another computer environment.

Design an algorithm to serialize and deserialize a \textbf{binary search tree}. There is no restriction on how your serialization/deserialization algorithm should work. You just need to ensure that a binary search tree can be serialized to a string and this string can be deserialized to the original tree structure.

\textbf{The encoded string should be as compact as possible.}

\paragraph{Note:} 
\begin{itemize}
\item Do not use class member/global/static variables to store states. Your serialize and deserialize algorithms should be stateless.
\end{itemize}

\subsection{Postorder traversal to optimise space for the tree structure}
BST could be constructed from preorder or postorder traversal only. In brief, it's a consequence of two facts:

\begin{itemize}
\item Binary tree could be constructed from preorder/postorder and inorder traversal.
\item Inorder traversal of BST is an array sorted in the ascending order: inorder is sorted preorder sequence.
\end{itemize}
That means that BST structure is already encoded in the preorder or postorder traversal and hence they are both suitable for the compact serialization.

Serialization could be easily implemented with both strategies, in this case, preorder can be implemented more efficiently.

\setcounter{lstlisting}{0}
\begin{lstlisting}[style=customc, caption={preorder}]
class Codec
{
public:

    // Encodes a tree to a single string.
    string serialize( TreeNode* root )
    {

        string s;

        //serialize by preorder travese
        preorder( root, s );

        if( !s.empty() )
        {
            //remove last blanck
            s.pop_back();
        }


        return s;
    }

    void preorder( TreeNode* node, string& s )
    {
        if( !node )
        {
            return;
        }

        s += to_string( node->val );
        s.push_back( ' ' );


        preorder( node->left, s );
        preorder( node->right, s );

    }




    // Decodes your encoded data to tree.
    TreeNode* deserialize( string data )
    {
        size_t start = 0;
        return build( data, start, INT_MIN, INT_MAX );
    }

    //from characters to integer
    int get_val( const char*s, size_t start, size_t end )
    {
        int x = 0;
        for( size_t i = start; i < end; ++i )
        {
            x = x * 10 + ( s[i] - '0' );
        }

        return x;
    }

    //we need start as a reference type
    //because we need this parameter to be changed
    //during the recursion, so that the index in s can
    //move forward
    TreeNode* build( string& s, size_t& start, int low, int up )
    {
        if( start >= s.size() )
        {
            return nullptr;
        }

        //find index of next space
        auto pos = s.find( ' ', start );

        if( pos == string::npos )
        {
            pos = s.size();
        }

        int val = get_val( s.c_str(), start, pos );


        if( ( val < low ) || ( val > up ) )
        {
            return nullptr;
        }

        //update next search start index
        start = pos + 1;

        TreeNode* node = new TreeNode( val );
        //after left is done
        //start will be updated
        node->left = build( s, start, low, val );

        node->right = build( s, start, val, up );

        return node;
    }
};
\end{lstlisting}

