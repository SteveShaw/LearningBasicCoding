\section{460 --- LFU Cache}
Design and implement a data structure for Least Frequently Used (\texttt{LFU}) cache. It should support the following operations: \texttt{get} and \texttt{put}.

\begin{lstlisting}[style=customc]
/*
Get the value (will always be positive) of the key
if the key exists in the cache,
otherwise return -1.
*/
int get( int key );

/*
Set or insert the value if the key is not already present.
When the cache reaches its capacity, it should invalidate
the least frequently used item before inserting a new item.
For the purpose of this problem, when there is a tie
(i.e., two or more keys that have the same frequency),
the least recently used key would be evicted.
*/
void put( int key, int value );

\end{lstlisting}

\paragraph{Follow up:}

\begin{itemize}
\item Could you do both operations in O(1) time complexity?
\end{itemize}

\paragraph{Example:}

\begin{lstlisting}[style=customc]
LFUCache cache = new LFUCache( 2 /* capacity */ );

cache->put(1, 1);
cache->put(2, 2);
cache->get(1);       // returns 1
cache->put(3, 3);    // evicts key 2
cache->get(2);       // returns -1 (not found)
cache->get(3);       // returns 3
cache->put(4, 4);    // evicts key 1
cache->get(1);       // returns -1 (not found)
cache->get(3);       // returns 3
cache->get(4);       // returns 4
\end{lstlisting}

\subsection{Double Linked List And Hash Map}
注意:这里是LFU不是LRU。即如果有多个元素都是最少访问的,那么需要从中删除最近没有访问过的。

We have to make use of two lists. One list is used to store the key/value and another is used to store the frequency related key/value list. We also need a hash map to store each key and its position in the two lists.

\setcounter{lstlisting}{0}
\begin{lstlisting}[style=customc, caption={List And Hash Map}]
class LFUCache
{
public:
    LFUCache( int capacity )
    {
        m_cap = capacity;
        m_size = 0;
    }

    int get( int key )
    {
        auto p_dict = m_dict.find( key );
        if( p_dict == m_dict.end() )
        {
            return -1;
        }

        //find in m_list
        auto outer = p_dict->second.outer;
        auto inner = p_dict->second.inner;

        //save value
        int value = inner->v;

        //increments the frequency
        auto n_outer = outer;
        ++n_outer; //the next of outer;

        //erase (key, value) from outer's kv list L
        outer->L.erase( inner );

        if( ( n_outer == m_list.end() ) || ( n_outer->f != outer->f + 1 ) )
        {
            //we need to insert a new node in m_list
            n_outer = m_list.emplace( n_outer, outer->f + 1 );
            n_outer->L.emplace_back( key, value );
        }
        else
        {
            //just use n_outer's L to insert (key, value)
            n_outer->L.emplace_back( key, value );
        }

        p_dict->second.outer = n_outer;
        p_dict->second.inner = --( n_outer->L.end() );

        if( outer->L.empty() )
        {
            //L is empty
            //remove outer from m_list
            m_list.erase( outer );
        }

        return value;
    }

    void put( int key, int value )
    {
        if( m_cap <= 0 )
        {
            return;
        }


        int x = get( key );
        if( x != -1 )
        {
            //we aready update key/value
            //to new position
            m_dict[key].inner->v = value;
            return;
        }

        if( m_size == m_cap )
        {
            //remove item from the start of m_list
            auto p_kv = m_list.begin()->L.begin();
            int d_key = p_kv->k;

            //remove it from the map
            m_dict.erase( d_key );
            m_list.begin()->L.erase( p_kv );
            if( m_list.begin()->L.empty() )
            {
                m_list.erase( m_list.begin() );
            }

            --m_size;
        }

        ++m_size;

        //either m_list has no element
        //or the start frequency is not 1
        if( m_list.empty() || ( m_list.begin()->f != 1 ) )
        {
            //insert a new item at the beginning() with freq equal to 1
            m_list.emplace_front( 1 );
        }

        //add (key,value) in the m_list first node's list's end
        m_list.begin()->L.emplace_back( key, value );

        //insert (key, out iterator/inner iterator) to the map
        auto p_dict = m_dict.emplace( std::piecewise_construct, std::forward_as_tuple( key ), std::forward_as_tuple() ).first;
        p_dict->second.outer = m_list.begin();
        p_dict->second.inner = --( m_list.begin()->L.end() );
    }

private:

    //inner list node
    struct kv
    {
        int k;
        int v;

        kv() = default;

        kv( int k_, int v_ )
            : k( k_ )
            , v( v_ )
        {
        }
    };

    //outer list node
    struct node
    {
        int f;

        list<kv> L;

        node() = default;

        explicit node( int f_ )
            : f( f_ )
        {
        }
    };

    //double iterators
    struct diter
    {
        list<node>::iterator outer;
        list<kv>::iterator inner;
        diter() = default;
    };

    list<node> m_list;

    unordered_map<int, diter> m_dict;

    int m_cap;
    int m_size;
};
\end{lstlisting}