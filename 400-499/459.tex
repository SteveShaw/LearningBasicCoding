\section{459 --- Repeated Substring Pattern}
Given a non-empty string, $S$, check if it can be constructed by taking a substring of it and appending multiple copies of the substring together. You may assume the given string consists of lowercase English letters only and its length will not exceed 10000.

\paragraph{Example 1:}

\begin{flushleft}
\textbf{Input}: $ abab $

\textbf{Output}: \texttt{True}

\textbf{Explanation}: It's the substring $ ab $ twice.

\end{flushleft}

\paragraph{Example 2:}

\begin{flushleft}
\textbf{Input}: $ aba $

\textbf{Output}: \texttt{False}
\end{flushleft}


\paragraph{Example 3:}

\begin{flushleft}
\textbf{Input}: $ abcabcabcabc $

\textbf{Output}: \texttt{True}

\textbf{Explanation}: It's the substring $ abc $ four times. (And the substring $ abcabc $ twice.)

\end{flushleft}

\subsection{KMP}
\begin{itemize}
\item We build KMP table $T$ first from $S$.
\item Any entry in KMP table, $T[i]$, is the length of longest proper prefix of $S[0,i]$, which is also a suffix of $S[0,i]$.
\item Thus $T[L-1]$ ($L$ is the length of $S$) is the length of longest proper prefix of $S$. Thus, if $S$ can be built by one of its substring, say $U$, then $U$ must be the suffix of $S$ and also the proper prefix of $S$.
\item We only need to check if $T[L-1]$ is greater than zero and $L$ is the multiple of $L-T[L-1]$.  
\end{itemize}

\setcounter{lstlisting}{0}
\begin{lstlisting}[style=customc, caption={KMP}]
bool repeatedSubstringPattern( string s )
{
    //build kmp table
    vector<size_t> f( s.size(), 0 );

    for( size_t i = 1; i < s.size(); ++i )
    {
        auto t = f[i - 1];

        while( ( t > 0 ) && ( s[i] != s[t] ) )
        {
            t = f[t - 1];
        }

        if( s[i] == s[t] )
        {
            t = t + 1;
        }

        f[i] = t;
    }

    //if s is made of a multiple substring t
    //then x = (r-1) * len(t) where r is the
    //count of t in x
    //so l-x is the length of the t
    //we need to make sure x is not zero
    //and l can be devided by (l-x)

    auto x = f.back();

    auto l = s.size();

    if( ( x > 0 ) && ( l % ( l - x ) == 0 ) )
    {
        return true;
    }

    return false;
}
\end{lstlisting}

\subsection{The Trick To Find A Repeatable String}
Based on this fact:
\begin{itemize}
\item First char of input string is first char of repeated sub-string
\item Last char of input string is last char of repeated sub-string
\item Assume $T = S+S$ where $S$ is the input string.
\item Remove first and last letter from $T$.
\item If $S$ exists in $T$, we know that $S$ contains a repeatable sub-string. Suppose index $i$ in $T$ is the index where $S$ starts, then the repeatable string in $S$ has length $i+1$ and the repeat sub-string in $S$ is $S[0, i]$.
\end{itemize}