\section{466 --- Count The Repetitions}
Define $S = [s,n]$ as the string $S$ which consists of $n$ connected strings s. For example, \lstinline[language=C++, basicstyle=\small\ttfamily, keywordstyle=\bfseries\color{green!40!black}]|[abc, 3] = abcabcabc|.

On the other hand, we define that string $s_1$ can be obtained from string $s_2$ if we can remove some characters from $ s_2 $ such that it becomes $ s_1 $. For example, \lstinline[language=C++, basicstyle=\small\ttfamily, keywordstyle=\bfseries\color{green!40!black}]|"abc"| can be obtained from \lstinline[language=C++, basicstyle=\small\ttfamily, keywordstyle=\bfseries\color{green!40!black}]|"abdbec"| based on our definition, but it can not be obtained from \lstinline[language=C++, basicstyle=\small\ttfamily, keywordstyle=\bfseries\color{green!40!black}]|"acbbe"|.

You are given two non-empty strings $ s_1 $ and $ s_2 $ (each at most 100 characters long) and two integers $0 \leq n_1 \leq 10^6$ and $1 \leq n_2 \leq 10^6$. Now consider the strings $S_1$ and $S_2$, where $S_1=[s_1,n_1]$ and $S_2=[s_2,n_2]$. Find the maximum integer $M$ such that $[S_2,M]$ can be obtained from $S_1$.

\section{Example:}

\begin{flushleft}
\textbf{Input}:

$s_1=\texttt{acb}$, $n_1=4$

$s_2=\texttt{ab}$, $n_2=2$

\textbf{Return}: 2

\end{flushleft}

\subsection{Brute Force}
According to the question, we need to find $m$ such that $[S_2,m]$ is the largest subsequence (not substring) that can be found in $S_1$. Notice that $S_2$ is $s_2$ repeats for $n_2$ times and $S_1$ is $s_1$ repeats for $n_1$ times. Therefore, we can find the number of times which are how many $s_2$ repeats in $[s_1,n_1]$, denote as $x$. Then the number of times $S_2$ which is $[s_2,n_2]$ repeats in $S_1$ is $(x/n_2)$

\setcounter{algorithm}{0}
\begin{algorithm}[H]
\caption{Brute Force}
\begin{algorithmic}[1]
\Procedure{GetMaxRepetitions}{$s_1, L_1, n_1, s_2, L_2, n_2$}
\State $p := 0$ \Comment The current index in $s_2$ to be checked against $s_1$
\State $x := 0$ \Comment the number of times $s_2$ repeats in $S_1$
\For{$i:=1$ \textbf{to} $n_1$}
\For{$j:=0$ \textbf{to} $L_1-1$}
\If{$s_2[p] = s_1[j]$}
\State $p = p+1$
\EndIf
\algstore{466algo}
\end{algorithmic}
\end{algorithm}
\begin{algorithm}[H]
\begin{algorithmic}[1]
\algrestore{466algo}
\If{$p = L_2$} \Comment $s_2$ has completed one repetition 
\State $x \gets x + 1$ \Comment Repeat count increments
\State $p \gets 0$ \Comment Reset index in $s_2$ to zero
\EndIf
\EndFor
\EndFor
\State \Return $x/n_2$
\EndProcedure
\end{algorithmic}
\end{algorithm}

\subsection{Improved Scanning Strategy}
We need to scan $s_1$ for $n_1$ times. After completely scanning of $i$-th $s_1$ block, we will have

\begin{itemize}
\item The accumulative count of $s_2$ repeated so far.
\item The next position $p$ in $s_2$. We will search $s_2[p]$ first in the $(i+1)$th $s_1$ block.
\end{itemize}

For example, Suppose $s_1=\texttt{abc}$, $s_2=\texttt{bac}$, after scanning the first $s_1$ block, we have found $s_2[0]$. We will look for $s_2[1]$ in the second $s_1$ block. Therefore $p=1$. The values for $p$ is in the range $[0, \lvert s_2\rvert]$. Therefore, by the \textbf{Pigeonhole principle}, we must have two same $p$ after scanning ($\lvert s_2\rvert + 1$) $s_1$ blocks.

The following picture shows how $p$ and count of $s_2$ are updated where we have 

\begin{itemize}
\item $s_1=\texttt{abaacdbac}$ and $n_1=100$
\item $s_2=\texttt{adcbd}$ and $n_2=4$
\end{itemize}

\begin{figure}[H]
\begin{tikzpicture}
[my/.style={draw, rectangle split, rectangle split parts=2, color=blue, fill=gray!20!, minimum height=1cm}]
\matrix[every node/.style={anchor=base}, column sep=-1mm]
{
\node(a0){a};& \node{b}; & \node{a};& \node{a};& \node{c};& \node{d};& \node{b};& \node{a};& \node(t0){c}; & \node(a1){a};& \node{b}; & \node{a};& \node{a};& \node{c};& \node{d};& \node{b};& \node{a};& \node(t1){c}; 
& \node(a2){a};& \node{b}; & \node{a};& \node{a};& \node{c};& \node{d};& \node{b};& \node{a};& \node(t2){c};

& \node(a3){a};& \node{b}; & \node{a};& \node{a};& \node{c};& \node{d};& \node{b};& \node{a};& \node(t3){c};

& \node(a4){a};& \node{b}; & \node{a};& \node{a};& \node{c};& \node{d};& \node{b};& \node{a};& \node(t4){c};

\\
\node(b0){\textcolor{red}{a}}; & \node{}; & \node{}; & \node{};& \node{};& \node{\textcolor{red}{d}}; & \node{};& \node{};& \node{\textcolor{red}{c}}; & \node(b1){}; & \node{\textcolor{red}{b}}; & \node{}; & \node{};& \node{};& \node(e0){\textcolor{red}{d}}; & \node{};& \node{\textcolor{red}{a}};& \node{};\node{}; 
& \node(b2){}; & \node{}; & \node{};& \node{};& \node{}; & \node{\textcolor{red}{d}};& \node{};& \node{}; & \node{\textcolor{red}{c}}; 

& \node(b3){}; & \node{\textcolor{red}{b}}; & \node{};& \node{};& \node{}; & \node(e1){\textcolor{red}{d}};& \node{};& \node{\textcolor{red}{a}}; & \node{};

& \node(b4){}; & \node{}; & \node{};& \node{};& \node{}; & \node{\textcolor{red}{d}};& \node{};& \node{}; & \node{\textcolor{red}{c}}; \\
};

\node[my](i0) [below=1.5cm of b0.south, anchor=north west] {$p=0$\nodepart{two}$\delta=0$};
\draw[>=stealth, ->] (i0.north west) -- (b0.south);

\node[my](i1) [below=1.5cm of b1.south, anchor=north west] {$p=3$\nodepart{two}$\delta=0$};
\draw[>=stealth, ->] (i1.north west) -- (b1.south);

\node[my](i2) [below=1.5cm of b2.south, anchor=north west] {$p=1$\nodepart{two}$\delta=1$};
\draw[>=stealth, ->] (i2.north west) -- (b2.south);

\node[my](i3) [below=1.5cm of b3.south, anchor=north west] {$p=3$\nodepart{two}$\delta=1$};
\draw[>=stealth, ->] (i3.north west) -- (b3.south);

\node[my](i4) [below=1.5cm of b4.south, anchor=north west] {$p=1$\nodepart{two}$\delta=2$};
\draw[>=stealth, ->] (i4.north west) -- (b4.south);

\draw[very thick, blue!30!red!70!] ($(t0.north east) + (0.25mm, 1mm)$) -- ($(t0.south east) + (0.25mm, 0)$);
\draw[very thick, blue!30!red!70!] ($(t1.north east) + (0.25mm, 1mm)$) -- ($(t1.south east) + (0.25mm, 0)$);

\draw[very thick, blue!30!red!70!] ($(t2.north east) + (0.25mm, 1mm)$) -- ($(t2.south east) + (0.25mm, 0)$);

\draw[very thick, blue!30!red!70!] ($(t3.north east) + (0.25mm, 1mm)$) -- ($(t3.south east) + (0.25mm, 0)$);
\draw[very thick, green!30!blue!50!] ($(e0.north east) + (0.25mm, 1mm)$) -- ($(e0.south east) + (0.25mm, 0)$);
\draw[very thick, green!30!blue!50!] ($(e1.north east) + (0.25mm, 1mm)$) -- ($(e1.south east) + (0.25mm, 0)$);
\end{tikzpicture}
\end{figure}

\paragraph{Outline}

\begin{itemize}
\item Maintain two arrays (or two hash maps to improve efficiency) $a$ and $b$. Both of them have the size equal to $\lvert s_2\rvert+1$ as mentioned before.
\item $a$ records $p$ while $b$ saves the accumulative count of $s_2$ at the start of each $s_1$ block in $S_1$.
\item Loop over the $n_1$ blocks of $s_1$. At current $i$th $s_1$ block, we check if a repetition of $p$ can be found in $a$. That is, if there exits a $k$ such that $a[k] = p$ and $0\leq k\leq (i-1)$.
\item If we find a $a[k]=p$, it means the a complete pattern is across from $k+1$th to $(i)$th block (the pattern restarts after $k$ th block). Hence, the pattern's \textbf{length} is $i-k$ blocks. Take the picture above as the example, we find $p=3$ at 2nd $s_1$ block and 4rd block. The pattern spans 2 blocks (3rd and 4th block).
\item From $(k+1)$th block to $n_1-1$th block, there are $n_1-k-1$ blocks. By considering the length of one pattern, we would have $q= \lfloor (n_1-k-1)/(i-k)\rfloor$ patterns.
\item In each pattern, $s_2$ appears $b[i] - b[k]$ times. As a result, the total counts of $s_2$ in all repeated patterns are $q\times (b[i]-b[k])$.  Take the picture above as the example, there are only one $s_2$ subsequence from 3rd block to 4th block.
\item If $(n_1-k-1)$ does not divided by $i-k$, the remainder $r$ is number of blocks after the whole patterns. These $r$ blocks do not contain the pattern. The number of $s_2$ in these $r$ blocks are $\delta[n_1-1] - \delta[k+(i-k)\times q] = \delta[k+(i-k)\times q + r] - \delta[k+(i-k)\times q]$. Since the pattern repeats for $q$ times, the increments of the number of $s_2$ also repeats. Then we get $\delta[k+(i-k)\times q + r] - \delta[k+(i-k)\times q] = \delta[k+r]-\delta[k]$. 
\item We also need to know how many $s_2$ exist before the pattern starts. This is indeed is $b[k]$.
\item Edge case: If no repeating pattern is found, this implies $n_1 < \lvert s_2\rvert+1$ by the above analysis. In this case, the counts of $s_2$ in $s_1$ is $b[n_1-1]/(n_2)$.
\end{itemize}

\begin{algorithm}[H]
\caption{Improved Scanning Strategy}
\begin{algorithmic}[1]
\Procedure{GetMaxRepetitions}{$s_1$, $L_1$, $n_1$, $s_2$, $L_2$, $n_2$}
\State $\star$ Initialize an array $a$ with length $L_1+1$ to store index of $s_2$ at start of each $s_1$ block
\State $\star$ Initialize an array $b$ with length $L_1+1$ to store count of $s_2$ at start of each $s_1$ block
\State $p := 0$
\State $\delta := 0$
\For{$i:=0$ \textbf{to} $n_1-1$} \label{466Loop1}
\For{$j:=0$ \textbf{to} $L_2-1$} \label{466Loop2}
\If{$s_1[i] = s_2[j]$}
\State $p\gets p+1$
\EndIf
\algstore{466algo}
\end{algorithmic}
\end{algorithm}
\begin{algorithm}[H]
\begin{algorithmic}[1]
\algrestore{466algo}
\If{$p = L_2$}
\State $p \gets 0$
\State $\delta \gets \delta + 1$
\EndIf
\EndFor \Comment End[\ref{466Loop2}]
\State $b[i] \gets \delta$
\State $a[i] \gets p$
\For{$k:=0$ \textbf{to} $i-1$} \label{466Loop3}
\If{$p = a[k]$}
\State $\alpha := i - k$ \Comment The interval of pattern
\State $q := (n_1-k)/(\alpha)$ \Comment The repeat times of pattern
\State $r := (n_1-k) - q \times \alpha$ \Comment The blocks does not contain the pattern
\State $z := (b[i]-b[k])\times q$ \Comment The total $s_2$ in all patterns
\State $\hat{z}:=(b[k+r] - b[k])$ \Comment total $s_2$ after the patterns
\State $p:=b[k]$ \Comment total $s_2$ before the patterns.
\State $x:=(z+\hat{z}+p) / n_2$ \Comment Total $S_2$ in $S_1$
\State \Return $x$
\EndIf
\EndFor \Comment End line [\ref{466Loop3}]
\EndFor \Comment End line [\ref{466Loop1}]
\State $\ast$ If no repeating pattern is found, the whole count of $s_2$ divided by $n_2$ is the result
\State \Return $b[n_1-1]/n_2$
\EndProcedure
\end{algorithmic}
\end{algorithm}

\setcounter{lstlisting}{0}
\begin{lstlisting}[style=customc, caption={Find Repeat Pattern}]
int getMaxRepetitions( string s1, int n1, string s2, int n2 ) {

    unordered_map<int, int> p_dict; //position
    unordered_map<int, int> d_dict; //accumulate counts

    int s2_i = 0; //the index of s2 in the next s1 block
    int s2_counts = 0; //count of s2 so far

    int L1 = static_cast< int >( s1.size() );
    int L2 = static_cast< int >( s2.size() );

    for( int block = 1; block <= n1; ++block )
    {
        for( int s1_i = 0; s1_i < L1; ++s1_i )
        {
            if( s2[s2_i] == s1[s1_i] )
            {
                ++s2_i;
            }

            //one s2 has been scanned
            if( s2_i == L2 )
            {
                s2_i = 0;
                ++s2_counts;
            }
        }
        //find if this position
        //has appeared before
        auto it = p_dict.find( s2_i );
        if( it == p_dict.end() )
        {
            //not found
            //we have not found repetition pattern

            //link current scanned index in s2
            //and the block
            p_dict.emplace( s2_i, block );

            //link current block and number of s2 have been
            //found
            d_dict.emplace( block, s2_counts );
        }
        else
        {
            int start_block = it->second;

            //how many blocks that the pattern across
            //the pattern is across [start_block+1, block]
            int num_blocks = block - start_block;

            //how many patterns we can find from start_block to the end
            int num_patterns = ( n1 - start_block ) / num_blocks;

            //how many blocks that cannot support a whole pattern after
            int rem_blocks = ( n1 - start_block ) - num_blocks * num_patterns;

            // now we need to count number of s2 from the pattern

            //get how many s2 are there before the pattern start
            int num_s2_before_pattern = d_dict[start_block];

            //number of s2 in all patterns
            int num_s2_in_pattern = ( s2_counts - d_dict[start_block] ) * num_patterns;

            //number of s2 after the pattern
            //this should be d_dict[n1] - d_dict[block]
            //but right now d_dict[n1] is not available
            //however, we know that n1=start_block + num_blocks + rem_blocks;
            // block = start_block + num_blocks;
            //the increments of s2 inside these blocks will also repeat
            int num_s2_after_pattern = d_dict[start_block + rem_blocks] - d_dict[start_block];

            return ( num_s2_before_pattern + num_s2_in_pattern + num_s2_after_pattern ) / n2;
        };

    }

    //no repetition can be found
    //only returns the recorded total number of s2
    //divided by n2
    return d_dict[n1] / n2;
}
\end{lstlisting}

