\section{400 --- Nth Digit}
Find the $n$th digit of the infinite integer sequence 1, 2, 3, 4, 5, 6, 7, 8, 9, 10, 11, $\ldots$

\paragraph{Note:}
\begin{itemize}
\item $ n $ is positive and will fit within the range of a 32-bit signed integer ($n < 2^{31}$).
\end{itemize}

\paragraph{Example 1:}

\begin{flushleft}
\textbf{Input}: 3
\textbf{Output}: 3
\end{flushleft}

\paragraph{Example 2:}

\begin{flushleft}
\textbf{Input}: 11
\textbf{Output}: 0
\textbf{Explanation}:
\\
The 11th digit of the sequence 1, 2, 3, 4, 5, 6, 7, 8, 9, 10, 11, $\ldots$ is a 0, which is part of the number 10.
\end{flushleft}

\subsection{Mathematical Induction}
问题可以分为三个部分
\begin{enumerate}
\item 计算包含第$n$个digit的number $x$是几位数。
\item 找到$x$。
\item 找到包含在$x$中的第$n$个digit
\end{enumerate}
例如,假设需要找到第250个digit,
\begin{enumerate}
\item 很容易找到$x$是三位数,因为1--9有9个数字,而10--99有180个数字,$250 - 9 - 90 \times 2 = 61$。即$n$是3位数中从100开始的第61个数字。
\item 找到$x$: 由于计数是从1开始,所以$x=100+(61-1)/3=120$。
\item 最后在$x$中找到第$n$th digit: 因为$(n-1)\bmod 3=0$,所以$x$中第0个数字即为$n$th digit。即1。
\end{enumerate}

\setcounter{lstlisting}{0}
\begin{lstlisting}[style=customc, caption={Mathematical Induction}]
int findNthDigit( int n )
{
    long long base = 9;
    long long z = 1; //number of digits
    long long start = 1; // start=1,10,100,....

    auto ln = static_cast<long long>( n );

    while( ln > base * z )
    {
        ln -= base * z;
        base *= 10;
        ++z;
        start *= 10;
    }

    //get number x that contains nth digit
    auto x = start + ( ( ln - 1 ) / z );

    //get the index of nth digit in x
    //from left to right
    auto i = ( ( ln - 1 ) % z );

    //convert to string
    auto s = to_string( x );

    return s[i] - '0';

}
\end{lstlisting}
