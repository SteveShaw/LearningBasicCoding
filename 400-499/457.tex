\section{457 --- Circular Array Loop}
You are given a circular array $A$ of positive and negative integers. If a number $ k $ at an index is positive, then move forward $k$ steps. Conversely, if it's negative ($-k$), move backward $k$ steps. Since the array is circular, you may assume that the last element's next element is the first element, and the first element's previous element is the last element.

Determine if there is a loop (or a cycle) in $A$. A cycle must start and end at the same index and the cycle's length is larger than 1. Furthermore, movements in a cycle must all follow a single direction. In other words, a cycle must not consist of both forward and backward movements.

 
\paragraph{Example 1:}

\begin{flushleft}
\textbf{Input}: [2,-1,1,2,2]

\textbf{Output}: \texttt{true}

\textbf{Explanation}: There is a cycle, from index $0 \longrightarrow 2 \longrightarrow 3 \longrightarrow 0$. The cycle's length is 3.
\end{flushleft}


\paragraph{Example 2:}

\begin{flushleft}
\textbf{Input}: [-1,2]

\textbf{Output}: \texttt{false}

\textbf{Explanation}: The movement from index $1 \longrightarrow 1 \longrightarrow 1 \ldots$ is not a cycle, because the cycle's length is 1. By definition the cycle's length must be greater than 1.

\end{flushleft}


\paragraph{Example 3:}

\begin{flushleft}
\textbf{Input}: [-2,1,-1,-2,-2]

\textbf{Output}: \texttt{false}

\textbf{Explanation}: The movement from index $1 \longrightarrow 2 \longrightarrow 1 \longrightarrow \ldots$ is not a cycle, because movement from index $1 \longrightarrow 2$ is a forward movement, but movement from index $2 \longrightarrow 1$ is a backward movement. All movements in a cycle must follow a single direction.
\end{flushleft}
 

\paragraph{Note:}

\begin{itemize}
\item $-1000 \leq A[i] \leq 1000$
\item $A[i] \neq 0$
\item $1 \leq \lvert A\rvert \leq 5000$
 \end{itemize}

\paragraph{Follow up:}

\begin{itemize}
\item Could you solve it in $O(n)$ time complexity and $O(1)$ extra space complexity?
\end{itemize}

\setcounter{lstlisting}{0}
\begin{lstlisting}[style=customc, caption={Fast And Slow Pointers}]
bool circularArrayLoop( vector<int>& nums )
{

    int L = static_cast<int>( nums.size() );

    auto next = [L, &nums]( int x )
    {

        int y = x + nums[x];
        y = ( y % L );

        if( y < 0 )
        {
            y += L;
        }

        return y;
    };

    //start from each index
    //to see if there is a loop
    for( int i = 0; i < L; ++i )
    {
        if( nums[i] == 0 )
        {
            continue;
        }

        int slow = i;
        int fast = next( i );

        //fast will jump two index
        //so make sure these two index has
        //same directions as i
        while( ( nums[fast] * nums[i] > 0 ) && ( nums[next( fast )] * nums[i] > 0 ) )
        {
            if( slow == fast )
            {
                //the loop contains only element
                if( slow == next( slow ) )
                {
                    break;
                }

                //we found a loop
                return true;
            }

            //jump one index for slow
            slow = next( slow );
            //jump two index for fast
            fast = next( next( fast ) );
        }


        //we cannot find loop starting from i
        int x = i;
        int val = nums[i];
        //set the elements in the path from i to zero
        while( nums[x] * val > 0 )
        {
            //nums[x] needs to be the
            //same direction as x
            int y = next( x );
            nums[x] = 0;
            x = y;
        }
    }

    return false;
}

\end{lstlisting}