\section{436 --- Find Right Interval}
Given a set of intervals $I$, for each of the interval $i$, check if there exists an interval $j$ whose start point is bigger than or equal to the end point of the interval $i$, which can be called that $j$ is on the \texttt{right} of $ i $.

For any interval $ i $, you need to store the minimum interval $ j $'s index, which means that the interval $ j $ has the minimum start point to build the \texttt{right} relationship for interval $ i $. If the interval $ j $ doesn't exist, store $-1$ for the interval i. Finally, you need output the stored value of each interval as an array.

\paragraph{Note:}

\begin{itemize}
\item You may assume the interval's end point is always bigger than its start point.
\item You may assume none of these intervals have the same start point.
\end{itemize}
 

\paragraph{Example 1:}

\begin{flushleft}
\textbf{Input}: [ [1,2] ]

\textbf{Output}: [$-1$]

\textbf{Explanation}: There is only one interval in the collection, so it outputs $-1$.
\end{flushleft}
 

\paragraph{Example 2:}

\begin{flushleft}
\textbf{Input}: [ [3,4], [2,3], [1,2] ]

\textbf{Output}: [$-1, 0, 1$]

\textbf{Explanation}: There is no satisfied right interval for [3,4].

For [2,3], the interval [3,4] has minimum-right start point;

For [1,2], the interval [2,3] has minimum-right start point.
\end{flushleft}
 

\paragraph{Example 3:}

\begin{flushleft}
\textbf{Input}: [ [1,4], [2,3], [3,4] ]

\textbf{Output}: [$-1, 2, -1$]

\textbf{Explanation}: There is no satisfied right interval for [1,4] and [3,4].

For [2,3], the interval [3,4] has minimum-right start point.
\end{flushleft}

\subsection{Binary Search Lower Bound}

\setcounter{lstlisting}{0}
\begin{lstlisting}[style=customc, caption={Lower Bound}]
vector<int> findRightInterval( vector<vector<int>>& intervals )
{
    map<int, size_t> m;

    for( size_t i = 0; i < intervals.size(); ++i )
    {
        m.emplace( intervals[i][0], i );
    }

    vector<int> ans( intervals.size(), -1 );

    for( size_t i = 0 ; i < intervals.size(); ++i )
    {
        //find the first start point that is
        //no less than current end
        auto it = m.lower_bound( intervals[i][1] );
        if( it != m.end() )
        {
            ans[i] = it->second;
        }
    }

    return ans;

}
\end{lstlisting}