\section{464 --- Can I Win}
In the \textbf{100 game}, two players take turns adding, to a running total, any integer from 1 to 10. The player who first causes the running total to reach or exceed 100 wins.

What if we change the game so that players cannot re-use integers?

For example, two players might take turns drawing from a common pool of numbers of $[1, 15]$ without replacement until they reach a total $\geq 100$.

Given an integer $x$ which is the maximum integer that the players can choose, and another integer $y$ which is the desired total, determine if the first player to move can force a win, assuming both players play optimally.

You can always assume that $x$ will not be larger than 20 and $y$ will not be larger than 300.

\paragraph{Example}

\begin{flushleft}
\textbf{Input}: $x = 10$ , $y = 11$

\textbf{Output}: \texttt{false}

\textbf{Explanation}: No matter which integer the first player choose, the first player will lose.

The first player can choose an integer from 1 up to 10.

If the first player choose 1, the second player can only choose integers from 2 up to 10.

The second player will win by choosing 10 and get a total equal to 11, which is equal or larger than $y$.

Same with other integers chosen by the first player, the second player will always win.
\end{flushleft}

\subsection{Depth First Search}

There are two facts we can directly use: Suppose $z$ is the he total summation by adding 1 to $x$, i.e., $z=\sum\limits_{i=1}^{x} i$;

\begin{itemize}
\item If $z$ smaller than the desired total $y$, nobody can win.
\item If $z$ is equal to $y$, the first player can win only if there are odd numbers of integers, i.e. $x$ is a odd number.
\end{itemize}

When $z$ is larger than $y$, we make use of DFS approach

\begin{enumerate}
\item Since there are no larger than 20 numbers, we can use bit operation to represent current numbers that first player has selected.
\item Starting from the first player, we try each number $n$ from 1 to $x$. When $n$ is not selected by the first player
\begin{itemize}
\item If the remaining total is no larger than $n$, the first player can select $n$ to win the game.
\item Otherwise, let first player select $n$ by marking the bit $n$ as 1, and decreased the desired total by $n$. 
\item Next time is second player to choose the number, the recursive function will be called again. If the result is \lstinline[language=C++, basicstyle=\small\ttfamily, keywordstyle=\bfseries\color{green!40!black}]|false|, we know that second player lose the game. Hence, we return \lstinline[language=C++, basicstyle=\small\ttfamily, keywordstyle=\bfseries\color{green!40!black}]|true| here for first player.
\item Finally, if no any choice can result first player win, we return \lstinline[language=C++, basicstyle=\small\ttfamily, keywordstyle=\bfseries\color{green!40!black}]|false| here for first player.
\end{itemize}
\end{enumerate}


\setcounter{lstlisting}{0}
\begin{lstlisting}[style=customc, caption={Depth First Search}]
bool canIWin( int maxChoosableInteger, int desiredTotal )
{
    if( desiredTotal < 2 )
    {
        return true;
    }

    int sum = ( 1 + maxChoosableInteger ) * maxChoosableInteger / 2;

    if( sum < desiredTotal )
    {
        //no one can win
        return false;
    }

    if( sum == desiredTotal )
    {
        //only odd number first person can win
        return ( maxChoosableInteger & 1 );
    }

    unordered_map<int, unsigned char> memo;

    return dfs( maxChoosableInteger, desiredTotal, memo, 0 );
}

//the helper function to do depth first search
bool dfs( int x, int y, unordered_map<int, unsigned char>& memo, int state )
{
    auto it = memo.find( state );
    if( it != memo.end() )
    {
        return it->second == 1;
    }

    for( int i = 0; i < x; ++i )
    {
        //number i+1 is not selected
        if( !( state & ( 1 << i ) ) )
        {
            if( y <= ( i + 1 ) )
            {
                //since the first person will select (i+1)
                //if y < (i+1)
                //first person can win;
                memo.emplace( state, 1 );
                return true;
            }

            int next_state = ( state | ( 1 << i ) );

            if( !dfs( x, y - ( i + 1 ), memo, next_state ) )
            {
                memo.emplace( state, 1 );
                return true;
            }
        }
    }

    memo.emplace( state, 0 );
    return false;
}
\end{lstlisting}
