\section{413 --- Arithmetic Slices}
A sequence of number is called arithmetic if it consists of at least three elements and if the difference between any two consecutive elements is the same.
\par
For example, these are arithmetic sequence:
\par
$1, 3, 5, 7, 9$
\par
$7, 7, 7, 7$
\par
$3, -1, -5, -9$
\par
The following sequence is not arithmetic.
\par
$1, 1, 2, 5, 7$
\par
A zero-indexed array $ A $ consisting of $ N $ numbers is given. A slice of that array is any pair of integers $(P, Q)$ such that $0 \leq P < Q < N$.
\par
A slice $(P, Q)$ of array A is called arithmetic if the sequence: $A[P], A[p + 1], \ldots, A[Q - 1], A[Q]$ is arithmetic. In particular, this means that $ P + 1 < Q $.
\par
The function should return the number of arithmetic slices in the array A.

\paragraph{Example:}
\begin{flushleft}
\textbf{Input}: $A = [1, 2, 3, 4]$
\\
\textbf{Output}: 3
\\
\textbf{Explanation}: There are 3 arithmetic slices in A: $[1, 2, 3]$, $[2, 3, 4]$ and $[1, 2, 3, 4]$ itself.
\end{flushleft}

\subsection{Recursion}
\begin{itemize}
\item Suppose we have a recursive function $F(A,i)$ returns the number of arithmetic slices in the range $A[k,\ldots,i]$. But these slices are not a part of any range $A[k,\ldots, j]$ where $j<i$. 
\item In addition, $F$ updates the total number of arithmetic slices $z$ in $A$.
\item Thus, $k$ is the minimum index that $A[k,\ldots,i]$ is a valid arithmetic slice
\item Now, suppose we have the number of arithmetic slices in $A[0\ldots i-1]$ end with element $A[i-1]$is $x$.
\begin{enumerate}
\item We do not care what is the minimum index $k$ that let $A[k\ldots i-1]$ is an arithmetic slice.
\item What we are sure is that the differences between consecutive elements inside the slice are all the same.
\end{enumerate}
\item Hence, for the next element $A[i]$, if $A[i]-A[i-1]$ is same as $A[i-1]-A[i-2]$, then the number of arithmetic slices in $A[0\ldots i]$ end with element $A[i]$ is $x+1$.
These slices comes from $A[k\ldots i]$, $\ldots$, $A[i-2\ldots i]$.Therefore, the total number of arithmetic slices are increased by $x+1$
\item If $A[i]-A[i-1]$ is not equal to $A[i-1] - A[i-2]$, the number of slices end in $A[i]$ is zero. The total number of arithmetic doesn't change.
\end{itemize}

\setcounter{algorithm}{0}
\begin{algorithm}[H]
\caption{Recursion}
\begin{algorithmic}[1]
\Procedure{NumberOfArithmeticSlices}{$A, L$}
\State $z:=0$ \Comment The total number of arithmetic slices
\State \Call{F}{$A, L-1, z$}
\State \Return $z$
\EndProcedure
\end{algorithmic}
\end{algorithm}

Function \texttt{F} take the input array $A$ and returns the number of arithmetic slices in range $A[0\ldots i]$ end with $A[i]$. \texttt{F} also updates the total number of arithmetic slices $z$.
\begin{algorithm}[H]
\caption{Helper function}
\begin{algorithmic}[1]
\Procedure{F}{$A, i, z$}
\If{$i < 2$}
\State \Return 0
\EndIf
\State $x:=0$ \Comment The number of arithmetic slices ending in $A[i]$
\If{$A[i]- A[i-1] = A[i-1] - A[i-2]$}
\State $\star$ $A[i]$ can be added to any arithmetic slice end in $A[i-1]$
\State $x\gets 1 + F(A,i-1, z)$
\State $z\gets x$
\Else
\State $F(A, i-1, z)$ \Comment No change for $z$
\EndIf
\State \Return $x$
\EndProcedure
\end{algorithmic}
\end{algorithm}

\subsection{Dynamic Programming}
\begin{itemize}
\item In the recursive approach, we start with the full range $A[0\ldots n-1]$. From the algorithm, we can see that the result for $A[0\ldots i]$ only depends on the elements in the range $A[0\ldots i]$, and not on any number beyond this range. This means dynamic programming is a possible approach.
\item Suppose $F[i]$ is the number of arithmetic slices ending with element $A[i]$ in the range $A[k, i]$ (where $k$ is the minimum index that makes $A[k\ldots i]$ is a arithmetic slice).
\item For element $A[i]$, if $A[i] - A[i-1] = A[i-1] - A[i-2]$, then $A[i]$ can be added to any arithmetic slice ending with $A[i-1]$. Hence, $F[i] = 1 + F[i-1]$ as discussed above. Then the total number is updated by adding $F[i]$. (这多出来的一个sequence 是$A[i-2], A[i-1], A[i]$).
\item Since $F[i]$ only depends on $F[i-1]$, we can only use two variable to replace $F[i]$ and $F[i-1]$
\end{itemize}

\begin{algorithm}[H]
\caption{Dynamic Programming}
\begin{algorithmic}[1]
\Procedure{NumberOfArithmeticSlices}{$A, L$}
\If{$A$ is empty} \Comment Corner case
\State \Return 0
\EndIf
\State $z:=0$ \Comment The total number
\State $x:=0$ \Comment The number of arithmetic slice ending at $i$
\For{$i:=2$ \textbf{to} $ L-1 $}
\If{$A[i]- A[i-1] = A[i-1] - A[i-2]$}
\State $A[i]$ can be added to any arithmetic slice end in $A[i-1]$
\State $x\gets x+1$
\State $z\gets z+x$ \Comment Update total number
\Else
\State $x\gets 0$ \Comment Reset $x$ to zero since no any arithmetic slice ending in $A[i]$
\EndIf
\EndFor
\State \Return $ z $
\EndProcedure
\end{algorithmic}
\end{algorithm}
