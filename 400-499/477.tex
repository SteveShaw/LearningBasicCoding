\section{477 --- Total Hamming Distance}
The Hamming distance between two integers is the number of positions at which the corresponding bits are different.

Now your job is to find the total Hamming distance between all pairs of the given numbers.

\paragraph{Example:}
\begin{flushleft}

\textbf{Input}: 4, 14, 2

\textbf{Output}: 6

\textbf{Explanation}: 

In binary representation, the 4 is 0100, 14 is 1110, and 2 is 0010 (just
showing the four bits relevant in this case). So the answer will be: $2 + 2 + 2 = 6$.


\end{flushleft}
\paragraph{Note:}

\begin{itemize}
\item Elements of the given array are in the range of 0 to $10^9$
\item Length of the array will not exceed $10^4$.
\end{itemize}

\subsection{Bit Manipulation}
\begin{itemize}
\item 统计每个bit位上的0和1的个数$m$ and $n$。
\item 于是每个bit位上两两数字所能产生的不同的bit位的个数即为$m\times n$。 而因为$m+n=L$($L$是数组的长度),因此只需算出1的个数即可。
\end{itemize}


\setcounter{lstlisting}{0}
\begin{lstlisting}[style=customc, caption={Bit Manipulation}]
int totalHammingDistance( vector<int>& nums )
{
    int mask = 1;

    int ans = 0;

    int L = static_cast<int>( nums.size() );

    for( int i = 0; i < 32; ++i )
    {
        int ones = 0;

        mask = ( 1 << i );

        //get 1s at current bit
        for( int n : nums )
        {
            if( mask & n )
            {
                ++ones;
            }
        }

        //total hamming distance at current bit
        //is ones * (L-ones)
        ans += ones * ( L - ones );

    }
	
    return ans;
}

\end{lstlisting}
