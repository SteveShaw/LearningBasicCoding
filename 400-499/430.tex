\section{430 --- Flatten a Multilevel Doubly Linked List}
You are given a doubly linked list which in addition to the next and previous pointers, it could have a child pointer, which may or may not point to a separate doubly linked list. These child lists may have one or more children of their own, and so on, to produce a multilevel data structure, as shown in the example below.

\begin{figure}[H]
\begin{tikzpicture}
[my/.style={draw, rectangle, minimum size = 1cm, fill=gray!20!},
node distance=8mm and 5mm, >=stealth,->]
\node[my](0) at (0,0) {1};
\node[my](1) [right=of 0] {2};
\node[my](2) [right=of 1] {3};
\node[my](3) [right=of 2] {4};
\node[my](4) [right=of 3] {5};
\node[my](5) [right=of 4] {6};
\node[my](6) [right=of 5] {$\emptyset$};
\draw (0.25) -- (1.155);
\draw (1.195) -- (0.345);
\draw (1.25) -- (2.155);
\draw (2.195) -- (1.345);
\draw (2.25) -- (3.155);
\draw (3.195) -- (2.345);
\draw (3.25) -- (4.155);
\draw (4.195) -- (3.345);
\draw (4.25) -- (5.155);
\draw (5.195) -- (4.345);
\draw (5.25) -- (6.155);
\draw (6.195) -- (5.345);
%level2
\node[my](7) [below=of 2] {7};
\node[my](8) [right=of 7] {8};
\node[my](9) [right=of 8] {9};
\node[my](10) [right=of 9] {10};
\node[my](e1) [right=of 10] {$\emptyset$};
\draw (2.245) -- (7.115);
\draw (7.65) -- (2.295);
\draw (7.25) -- (8.155);
\draw (8.195) -- (7.345);
\draw (8.25) -- (9.155);
\draw (9.195) -- (8.345);
\draw (9.25) -- (10.155);
\draw (10.195) -- (9.345);
\draw (10.25) -- (e1.155);
\draw (e1.195) -- (10.345);
%level 3
\node[my](11) [below=of 8] {11};
\node[my](12) [right=of 11] {12};
\node[my](e2) [right=of 12] {$\emptyset$};
\draw (8.245) -- (11.115);
\draw (11.65) -- (8.295);
\draw (11.25) -- (12.155);
\draw (12.195) -- (11.345);
\draw (12.25) -- (e2.155);
\draw (e2.195) -- (12.345);
\end{tikzpicture}
\end{figure}

Flatten the list so that all the nodes appear in a single-level, doubly linked list. You are given the head of the first level of the list.

\begin{figure}[H]
\begin{tikzpicture}
[my/.style={draw, rectangle, minimum size = 8mm, fill=gray!20!},
node distance=5mm, >=stealth,->]
\node[my](0) at (0,0) {1};
\node[my](1) [right=of 0] {2};
\node[my](2) [right=of 1] {3};

\node[my](7) [right=of 2] {7};
\node[my](8) [right=of 7] {8};


\node[my](11) [right=of 8] {11};
\node[my](12) [right=of 11] {12};

\node[my](9) [right=of 12] {9};
\node[my](10) [right=of 9] {10};

\node[my](3) [right=of 10] {4};
\node[my](4) [right=of 3] {5};
\node[my](5) [right=of 4] {6};
\node[my](e) [right=of 5] {$\emptyset$};

\draw (0.25) -- (1.155);
\draw (1.195) -- (0.345);
\draw (1.25) -- (2.155);
\draw (2.195) -- (1.345);

\draw (2.25) -- (7.155);
\draw (7.195) -- (2.345);


\draw (7.25) -- (8.155);
\draw (8.195) -- (7.345);

\draw (8.25) -- (11.155);
\draw (11.195) -- (8.345);


\draw (11.25) -- (12.155);
\draw (12.195) -- (11.345);

\draw (12.25) -- (9.155);
\draw (9.195) -- (12.345);

\draw (9.25) -- (10.155);
\draw (10.195) -- (9.345);

\draw (10.25) -- (3.155);
\draw (3.195) -- (10.345);


\draw (3.25) -- (4.155);
\draw (4.195) -- (3.345);
\draw (4.25) -- (5.155);
\draw (5.195) -- (4.345);

\draw (5.25) -- (e.155);
\draw (e.195) -- (5.345);


\end{tikzpicture}
\end{figure}

\subsection{Flatter One Level Each Time}
\begin{itemize}
\item Iterate over the list. If current node has a child, we iterate over from the child until the last node of the child list.
\item Connect the pointer of the last node with current node's next node.
\item Connect the pointer of current node and it child.
\item Set current node's child as null.
\item Go to the next node of current node.
\item In this way, we flatten one whole child list at a time. Even though the child list may contain another grandchild list, we iterate from current node's next which will iterate over the child list again.
\end{itemize}

\setcounter{lstlisting}{0}
\begin{lstlisting}[style=customc, caption={Flatten one list at a time}]
/*
// Definition for a Node.
class Node {
public:
    int val;
    Node* prev;
    Node* next;
    Node* child;

    Node() {}

    Node(int _val, Node* _prev, Node* _next, Node* _child) {
        val = _val;
        prev = _prev;
        next = _next;
        child = _child;
    }
};
*/

Node* flatten( Node* head )
{
    auto node = head;
    while( node )
    {
        if( node->child )
        {
            auto next = node->next;

            //connect prev and next pointers of node and its child
            node->child->prev = node;
            node->next = node->child;
            node->child = nullptr;

            //iterate over the child list
            auto x = node->next;

            //search the last node
            while( x->next )
            {
                x = x->next;
            }

            //connect prev and next pointers of x and next node
            x->next = next;
            if( next )
            {
                next->prev = x;
            }
        }

        //forward to the next node
        //we may iterate over the child list again
        node = node->next;
    }
    return head;
}
\end{lstlisting}