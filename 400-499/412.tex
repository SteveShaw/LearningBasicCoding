\section{412 --- Fizz Buzz}
Write a program that outputs the string representation of numbers from 1 to $ n $.
\par
But for multiples of three it should output \texttt{Fizz} instead of the number and for the multiples of five output \texttt{Buzz}. For numbers which are multiples of both three and five output FizzBuzz.

\paragraph{Example:}
\begin{flushleft}
\textbf{Input}: 15
\\
\textbf{Output}: $[\;1,2,\texttt{Fizz}, 4, \texttt{Bizz}, \texttt{Fizz}, 7, 8, \texttt{Fizz}, \texttt{Buzz}, 11, \texttt{Fizz}, 13, 14, \texttt{FizzBuzz}\;]$
\end{flushleft}

\subsection{String Concatenation}
\begin{itemize}
\item Initially set current string is empty
\item If current number can be divide by 3, add \texttt{Fizz}
\item If current number can be divide by 5, add \texttt{Buzz}
\item If the string is still empty, the current number cannot be divided by 3 and 5, so set the string to current number itself
\end{itemize}

\subsection{Naive Approach}
循环四遍,
\begin{enumerate}
\item 第一遍所有的字符串设置为数字本身,
\item 第二遍每隔3个数字,设置\texttt{Fizz},
\item 第三遍每隔5个数字,设置\texttt{Buzz},
\item 第四遍每隔15个数字,设置\texttt{FizzBuzz},
\end{enumerate}

\setcounter{lstlisting}{0}
\begin{lstlisting}[style=customc, caption={Naive Approach}]
vector<string> fizzBuzz( int n )
{
    vector<string> ans;

    ans.reserve( n );

    for( int i = 1; i <= n; ++i )
    {
        ans.emplace_back( to_string( i ) );
    }

    for( int i = 3; i <= n; i += 3 )
    {
        ans[i - 1] = "Fizz";
    }

    for( int i = 5; i <= n; i += 5 )
    {
        ans[i - 1] = "Buzz";
    }

    for( int i = 15; i <= n; i += 15 )
    {
        ans[i - 1] = "FizzBuzz";
    }

    return ans;
}
\end{lstlisting}