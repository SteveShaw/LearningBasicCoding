\section{423 --- Reconstruct Original Digits from English}
Given a non-empty string containing an out-of-order English representation of digits 0--9, output the digits in ascending order.

\paragraph{Note:}
\begin{itemize}
\item Input contains only lowercase English letters.
\item Input is guaranteed to be valid and can be transformed to its original digits. That means invalid inputs such as \texttt{abc} or \texttt{zerone} are not permitted.
\item Input length is less than 50,000.
\end{itemize}

\paragraph{Example 1:}

\begin{flushleft}
\textbf{Input}: \texttt{owoztneoer}

\textbf{Output}: 012
\end{flushleft}

\paragraph{Example 2:}
\begin{flushleft}
\textbf{Input}: \texttt{fviefuro}

\textbf{Output}: 45
\end{flushleft}

\subsection{Find Unique Letters In Each Number}
We can observe the following facts:
\begin{itemize}
\item Count unique letters in each even numbers.
\begin{itemize}
\item Letter \texttt{z} is present only in \texttt{zero}.
\item Letter \texttt{w} is present only in \texttt{two}.
\item Letter \texttt{u} is present only in \texttt{four}.
\item Letter \texttt{x} is present only in \texttt{six}.
\item Letter \texttt{g} is present only in \texttt{eight}.
\end{itemize}
\item Then we count the unique letters that have in both an even number and a odd number 

\begin{itemize}
\item Letter \texttt{h} is present only in \texttt{three} and \texttt{eight}.
\item Letter \texttt{f} is present only in \texttt{five} and \texttt{four}.
\item Letter \texttt{s} is present only in \texttt{seven} and \texttt{six}.
\end{itemize}

\item Finally, we count the unique letters in \texttt{nine} and \texttt{one}, and the logic is basically the same :
\begin{itemize}
\item Letter \texttt{i} is present in \texttt{nine}, \texttt{five}, \texttt{six}, and \texttt{eight}.
\item Letter \texttt{n} is present in \texttt{one}, \texttt{seven}, and \texttt{nine}.
\end{itemize}

\end{itemize}

\setcounter{lstlisting}{0}
\begin{lstlisting}[style=customc, caption={Unique Letters Count}]
string originalDigits( string s )
{
    //get the count of each letter
    int m[26] = {0};

    for( char c :  s )
    {
        m[c - 'a'] += 1;
    }

    int count[10] = {0};

    // letter "z" is present only in "zero"
    count[0] = m['z' - 'a'];
    // letter "w" is present only in "two"
    count[2] = m['w' - 'a'];
    // letter "u" is present only in "four"
    count[4] = m['u' - 'a'];
    // letter "x" is present only in "six"
    count[6] = m['x' - 'a'];
    // letter "g" is present only in "eight"
    count[8] = m['g' - 'a'];


    // letter "h" is present only in "three" and "eight"
    count[3] = m['h' - 'a'] - count[8];
    // letter "f" is present only in "five" and "four"
    count[5] = m['f' - 'a'] - count[4];
    // letter "s" is present only in "seven" and "six"
    count[7] = m['s' - 'a'] - count[6];

    // letter "i" is present in "nine", "five", "six", and "eight"
    count[9] = m['i' - 'a'] - count[5] - count[6] - count[8];
    // letter "n" is present in "one", "nine", and "seven"
    count[1] = m['n' - 'a'] - count[7] - 2 * count[9];

    //generate output
    string ans;
	
    for( int i = 0; i < 10; ++i )
    {
        if( count[i] == 0 )
        {
            continue;
        }

        ans.append( count[i], i + '0' );
    }

    return ans;

}
\end{lstlisting}
