\section{402 --- Remove K Digits}
Given a non-negative integer $n$ represented as a string, remove $k$ digits from the number so that the new number is the smallest possible.
\par
\paragraph{Note:}

\begin{itemize}
\item     The length of $ n $ is less than 10002 and will be no less than $ k $.
\item     The given $ n $ does not contain any leading zero.

\end{itemize}

\paragraph{Example 1:}

\begin{flushleft}
\textbf{Input}: $n = 1432219$, $k = 3$
\\
\textbf{Output}: $1219$
\\
\textbf{Explanation}: Remove the three digits 4, 3, and 2 to form the new number 1219 which is the smallest.
\end{flushleft}

\paragraph{Example 2:}

\begin{flushleft}
\textbf{Input}: $n = 10200$, $k = 1$
\\
\textbf{Output}: 200
\\
\textbf{Explanation}: Remove the leading 1 and the number is 200. Note that the output must not contain leading zeroes.
\end{flushleft}

\paragraph{Example 3:}

\begin{flushleft}
\textbf{Input}: $n = 10$, $k = 2$
\\
\textbf{Output}: 0
\\
\textbf{Explanation}: Remove all the digits from the number and it is left with nothing which is 0.

\end{flushleft}

\subsection{Sliding Window}
\begin{itemize}
\item 采用sliding window的方法,maintain a double end queue: $q$
\item 从头开始遍历$ n $,将第一个字符放入结果$q$中,然后继续,如果当前digit比$q$中的最后一个digit要大,从$q$中移除最后一个digit,直至$q$为空或者在当前$q$的最后digit不大于当前digit。同时记录从$q$中remove last digit的次数 $\delta$,如果$\delta=k$,就不能再remove last digit了。
\item Remove $q$ 中可能的leading zeros。
\item 最后,$q$中包含的digit个数可能会大于$n$的长度$\ell$减去$k$。原因是$n$中可能会有很多重复的比较小的digit。只需要从$q$中返回$\ell-k$个digits即可。
\end{itemize}

\setcounter{algorithm}{0}
\begin{algorithm}[H]
\caption{Sliding Window}
\begin{algorithmic}[1]
\Procedure{RemoveKdigits}{$n, \ell, k$}
\State $\star$ Maintain a double end queue $q$
\State $\delta:=0$ \Comment The count of removing digits
\For{$i:=0$ \textbf{to} $\ell-1$}
\State $d:=n[i]$
\While{$q$ is not empty \textbf{and} $\delta < k$ \textbf{and} $q$'s last digit is less than $d$}
\State $\star$ Remove last digit from $q$
\State $\delta\gets\delta+1$ \Comment Increments the count of removing digits
\EndWhile
\State $\star$ Add $d$ to end of $q$
\EndFor
\State $\ast$ Remove leading zeros from $q$
\While{$q$ is not empty \textbf{and} $q$'s first digit is zero}
\State $\star$ Remove first digit from $q$
\EndWhile
\If{$q$ is empty}
\State \Return 0 \Comment Only zero remains in $q$
\EndIf
\State $l:=\min(\ell-k, \lvert q\rvert)$ \Comment $q$ may contain more than $\ell-k$ digits because of duplicate smaller numbers
\State $s=\emptyset$ \Comment The result number
\For{$x:=0$ \textbf{to} $l-1$}
\State $\star$ Add first digit of $q$ to $s$
\State $\star$ Remove first digit from $q$
\EndFor
\If{$s$ is empty}
\State \Return 0 \Comment All digits are removed from $n$
\Else
\State \Return $s$
\EndIf
\algstore{402algo}
\end{algorithmic}
\end{algorithm}
\begin{algorithm}[H]
\begin{algorithmic}[1]
\algrestore{402algo}
\EndProcedure
\end{algorithmic}
\end{algorithm}

\setcounter{lstlisting}{0}
\begin{lstlisting}[style=customc, caption={Sliding Window}]
string removeKdigits( string num, int k )
{
    deque<char> q;
    //the count of remove digits
    int count = 0;

    //process sliding window
    for( auto c : num )
    {
        while( !q.empty() && ( count < k ) && ( q.back() > c ) )
        {
            q.pop_back();
            ++count;
        }
        q.push_back( c );
    }

    while( !q.empty() && q.front() == '0' )
    {
        q.pop_front();
    }

    if( q.empty() )
    {
        return "0";
    }

    int ln = static_cast<int>( num.size() );
    int lq = static_cast<int>( q.size() );

    //q may contain more than ln-k digits because of duplicate smaller digits
    int total = ( min )( ln - k, lq );

    string ans;

    for( int x = 0; x < total; ++x )
    {
        ans.push_back( q.front() );
        q.pop_front();
    }

    //the result string may still be empty
    //if total is zero
    return ans.empty() ? "0" : ans;
}
\end{lstlisting}