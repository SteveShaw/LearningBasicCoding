\section{414 --- Third Maximum Number}
Given a non-empty array of integers $A$, return the third maximum number in this array. If it does not exist, return the maximum number. The time complexity must be in $O(n)$.

\paragraph{Example 1:}

\begin{flushleft}
\textbf{Input}: $[3, 2, 1]$
\\
\textbf{Output}: 1
\\
\textbf{Explanation}: The third maximum is 1.
\end{flushleft}

\paragraph{Example 2:}

\begin{flushleft}
\textbf{Input}: [1, 2]
\\
\textbf{Output}: 2
\\
\textbf{Explanation}: The third maximum does not exist, so the maximum (2) is returned instead.
\end{flushleft}

\paragraph{Example 3:}

\begin{flushleft}
\textbf{Input}: [2, 2, 3, 1]
\\
\textbf{Output}: 1
\\
\textbf{Explanation}: Note that the third maximum here means the third maximum distinct number. Both numbers with value 2 are both considered as second maximum.
\end{flushleft}

\subsection{Swap}
\begin{itemize}
    \item 用三个变量$x$, $y$, $z$来分别保存第一大,第二大,和第三大的数
    \item 然后我们遍历数组,如果遍历到的数字大于当前第一大的数$x$,那么$z\gets y$, $y\gets x$,$x$更新为当前数字。如果当前数字大于$y$,小于$x$,那么$z\gets y$,$x$更新为当前数字。如果当前数字大于$z$,小于$y$,那就只更新$z$为当前数字
    \item 初始化$x$,$y$和$z$时要用长整型long的最小值,否则当数组中有int type的最小值存在时,就会出现错误。
\end{itemize}

\setcounter{lstlisting}{0}
\begin{lstlisting}[style=customc, caption={Swap}]
int thirdMax( vector<int>& nums )
{
    long long x = nums[0];
    long long y = LLONG_MIN;
    long long z = LLONG_MIN;

    for( int n : nums )
    {
        if( n > x )
        {
            z = y;
            y = x;
            x = n;
        }
        else if( ( n > y ) && ( n < x ) ) //need to skip duplicate numbers
        {
            z = y;
            y = n;
        }
        else if( ( n > z ) && ( n < y ) ) //need to skip duplicate numbers
        {
            z = n;
        }
    }

    return z == LLONG_MIN ? x : z;
}
\end{lstlisting}