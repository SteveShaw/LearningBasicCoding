\section{446 --- Arithmetic Slices II - Subsequence}
A sequence of numbers is called \textbf{arithmetic} if it consists of at least three elements and if the difference between any two consecutive elements is the same.

For example, these are arithmetic sequences:

$1, 3, 5, 7, 9$

$7, 7, 7, 7$

$3, -1, -5, -9$

The following sequence is not arithmetic.

$1, 1, 2, 5, 7$

A zero-indexed array $A$ consisting of $N$ numbers is given. A subsequence slice of that array is any sequence of integers $(P_0, P_1, \ldots, P_k)$ such that $0 \leq P_0 < P1 < \ldots < P_k < N$.

A subsequence slice $(P_0, P_1, \ldots, P_k)$ of array $A$ is called arithmetic if the sequence $A[P_0]$, $A[P_1]$, $\ldots$, $A[P_k-1]$, $A[P_k]$ is arithmetic. In particular, this means that $k \geq 2$.

The function should return the number of arithmetic subsequence slices in the array $A$.

The input contains $N$ integers. Every integer is in the range of $-2^{31}$ and $2^{31}-1$ and $0 \leq N \leq 1000$. The output is guaranteed to be less than $2^{31}-1$.
 

\paragraph{Example:}
\begin{flushleft}
\textbf{Input}: $[2, 4, 6, 8, 10]$

\textbf{Output}: 7

\textbf{Explanation}:

All arithmetic subsequence slices are:

[2,4,6]

[4,6,8]

[6,8,10]

[2,4,6,8]

[4,6,8,10]

[2,4,6,8,10]

[2,6,10]
\end{flushleft}


\subsection{Dynamic Programming}
To determine an arithmetic sequence, we need at least two parameters: the first (or last) element of the sequence, and the common difference.

Let $F[i][d]$ denotes the number of arithmetic subsequences that ends with $A[i]$ and its common difference is $d$.

Let's try to find the state transitions based on the representation above. 

\begin{itemize}
\item Assume we want to append a new element $A[i]$ to existing arithmetic subsequences to form new subsequences. We can append $A[i]$ to an existing arithmetic subsequence, only if the difference between the sequence's last element and $A[i]$ is equal to the sequence's common difference.
\item Thus, we can define the state transitions for the element A[i] as :

\[F[i][A[i] - A[j]] \gets F[i][A[i]-A[j]] +  F[j][A[i] - A[j]] \; \forall j < i\]

This demonstrates the appending process above to form new arithmetic subsequences.
\item But here comes the problem. Initially all $F[i][d]$ are 0, how to form a new arithmetic subsequence if there are no existing subsequences before?
\item In the original definition of arithmetic subsequences, the length of the subsequence must be at least 3. This makes it hard to form new subsequences if only two indices $i$ and $j$ are given. We may need to take the subsequences of length 2 into account, i.e., For any pair $i$, $j$ ($i \neq j$), $A[i]$ and $A[j]$ can always form a \textbf{weak} arithmetic subsequence.
\item Thus we can change the state representations accordingly: $F[i][d]$ denotes the number of \textbf{weak} arithmetic subsequences that ends with $A[i]$ and its common difference is $d$, the state transitions are quite straightforward:

\[F[i][A[i] - A[j]] \gets F[i][A[i]-A[j]] +  F[j][A[i] - A[j]] + 1\; \forall j < i\]

The 1 appears here because we can form a new \textbf{weak} arithmetic subsequence for the pair $(i, j)$.
\item Now the number of all weak arithmetic subsequences is the sum of all $F[i][d]$. To get the number of arithmetic subsequences that are \textbf{not weak}, we have two ways:
\begin{enumerate}
\item Count the number of \textbf{pure weak} arithmetic subsequences directly. The pure weak arithmetic subsequences are the arithmetic subsequences of length 2, so the number of pure weak arithmetic subsequences should be equal to the number of pairs $(i, j)$, which is $\dfrac{n \times (n - 1)}{2}$
\item For $F[i][A[i] - A[j]] \gets F[i][A[i]-A[j]] +  F[j][A[i] - A[j]] + 1$, $F[j][A[i] - A[j]]$ is the number of existing weak arithmetic subsequences, while 1 is the new subsequence built with $A[i]$ and $A[j]$. When appending new elements to existing weak arithmetic subsequences, we are forming arithmetic subsequences. So $F[j][A[i] - A[j]]$ is the number of new formed arithmetic subsequences, and can be added to the answer.
\end{enumerate}
\end{itemize}
Take the following example to illustrate the process:

Suppose $A =[1, 1, 2, 3, 4, 5]$

\begin{itemize}
\item For the first element 1, there is no element in front of it, the answer remains 0.

\item For the second element 1, the element itself with the previous 1 can form a pure weak arithmetic subsequence with common difference 0: $[1, 1]$.

\item For the third element 2, it cannot be appended to the only weak arithmetic subsequence $[1, 1]$, so the answer remains 0. Similar to the second element, it can form new weak arithmetic subsequences $[1, 2]$ and $[1, 2]$.

\item For the forth element 3, if we append it to some arithmetic subsequences ending with 2, these subsequences must have a common difference of $3 - 2 = 1$. Indeed there are two: $[1, 2]$ and $[1, 2]$. So we can append 3 to the end of these subsequences, and the answer is added by 2. Similar to above, it can form new weak arithmetic subsequences $[1, 3]$, $[1, 3]$ and $[2, 3]$.

\item The other elements are the same.
\end{itemize}

\setcounter{lstlisting}{0}
\begin{lstlisting}[style=customc, caption={Dynamic Programming}]
int numberOfArithmeticSlices( vector<int>& A )
{
    using ll_t = long long;

    vector<unordered_map<long long, int>> m( A.size() );

    int ans = 0;

    for( size_t i = 1; i < A.size(); ++i )
    {
        for( size_t j = 0; j < i; ++j )
        {
            auto d = static_cast<ll_t>( A[i] ) - static_cast<ll_t>( A[j] );

            //skip the difference that is overflow integer type
            if( diff > INT_MAX || diff < INT_MIN )
            {
                continue;
            }

            int count = 0;

            //find F[j][d]
            auto it = m[j].find( diff );

            if( it != m[j].end() )
            {
                count = it->second;
            }

            //the number of arithmetic subsequence
            //ending A[i] with difference = diff
            //is same as the number of arithemtic subsequence
            //ending A[j] with difference = diff
            ans += count;

            //update F[i][d]
            it = m[i].find( diff );

            //plus one because a new arithmetic subsequence
            //starting with a[j] and ending with a[i]

            if( it == m[i].end() )
            {
                m[i].emplace( diff, count + 1 );
            }
            else
            {
                it->second += ( count + 1 );
            }
        }
    }

    return ans;
}
\end{lstlisting}
