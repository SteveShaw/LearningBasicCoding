\section{94 --- Target Sum}
You are given a list of non-negative integers, $a_1, a_2, \ldots a_n$, and a target, $S$. Now you have 2 symbols $+$ and $-$. For each integer, you should choose one from $+$ and $-$ as its new symbol.

Find out how many ways to assign symbols to make sum of integers equal to target $S$.

\paragraph{Example 1:}

\begin{flushleft}
\textbf{Input}: $A = [1, 1, 1, 1, 1]$, $S = 3$
 
\textbf{Output}: 5

\textbf{Explanation}: 

$-1+1+1+1+1 = 3$

$+1-1+1+1+1 = 3$

$+1+1-1+1+1 = 3$

$+1+1+1-1+1 = 3$

$+1+1+1+1-1 = 3$


\end{flushleft}

There are 5 ways to assign symbols to make the sum of $A$ be target 3.

\paragraph{Note:}
\begin{itemize}
\item The length of the given array is positive and will not exceed 20.
\item The sum of elements in the given array will not exceed 1000.
\item Your output answer is guaranteed to be fitted in a 32-bit integer.
\end{itemize}

\paragraph{0-1 Kapsnack}
\begin{itemize}
\item Suppose the sum of all positive integers is $S_0$ and sum of all negative integers is $S_1$
\item We have $S_0 + S_1=S$, then we have $S_0 = S- S_1 \to S_0 \times 2 = S - S_1+S_0$. Since $S_0-S_1=\sum\limits_{i=0}^{n-1}a_i$, we have $S_0\times 2 = S + \sum\limits_{i=0}^{n-1}a_i$.
\item The overall value space is $[0, T+1]$ where $T = (S+\sum\limits_{i=0}^{n-1}a_i) / 2$
\end{itemize}

\setcounter{lstlisting}{0}
\begin{lstlisting}[style=customc, caption={0-1 Kapsnack}]
int findTargetSumWays( vector<int>& nums, int S )
{

    int sum = 0;

    for( int n :  nums )
    {
        sum += n;
    }

    //we cannot have sum outside [-sum, sum]
    if( ( S >  sum ) || ( S < -sum ) )
    {
        return 0;
    }

    // S(P) + S(N) = S
    // S(P) + S(P) =  S + (S(P)-S(N)) = S + sum(A)
    int T = S + sum;

    if( T & 1 )
    {
        //if T is odd, we cannot get the
        //answer
        return 0;
    }

    T >>= 1;

    vector<int> F( T + 1, 0 );

    F[0] = 1;

    for( int n : nums )
    {
        for( int i = T; i >= n; --i )
        {
            F[i] += F[i - n];
        }
    }

    return F.back();
}

\end{lstlisting}
