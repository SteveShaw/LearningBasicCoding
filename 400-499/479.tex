\section{479 --- Largest Palindrome Product}
Find the largest palindrome made from the product of two $n$-digit numbers.

Since the result could be very large, you should return the largest palindrome mod 1337.
 

\paragraph{Example:}
\begin{flushleft}

\textbf{Input}: 2

\textbf{Output}: 987

\textbf{Explanation}: $99 \times 91 = 9009$, $9009 \bmod 1337 = 987$

\end{flushleft}
 

\paragraph{Note:}

\begin{itemize}
\item The range of $n$ is $[1,8]$.
\end{itemize}

\subsection{Try All From Largest}

We can try each number from $10 ^{n-1}$ until $10^{n}-1$. To speed the process, we can try from the largest number until smallest one.

For current number $x$
\begin{enumerate}
\item generate a palindrome number, $z$, from $x$.
\item Search in $[10^{n-1}, 10^n-1]$ to see if there exits a number $y$ such that $z$ is the multiple of $y$. If it is, we know $z$ is the answer.
\item To find $y$, we still test each number starting from the largest one, i.e. $10^n-1$, until we have $y\times y < z$. Because after that, $y < z/y$ and we just repeat the process.
\end{enumerate}

\setcounter{lstlisting}{0}
\begin{lstlisting}[style=customc, caption={Try All From Largest}]
int largestPalindrome( int n )
{
    if( n == 1 )
    {
        return 9;
    }

    //generate the maximum and minimum number
    //for n-digit number
    int r = 1; //upper bound

    for( int i = 1; i <= n; ++i )
    {
        r *= 10;
    }

    --r;

    int l = 1;//lower bound
    for( int i = 1; i < n; ++i )
    {
        l *= 10;
    }

    for( int x = r; x >= l; --x )
    {
        //form the palindrome
        auto z = palindrome( x );

        //since we test starting from the larger number
        //we should stop when y^2 >= z since after that
        //y will be smaller than z/y, it will repeat
        //so y >= z/y ==> y^2 >=z
        for( long long y = r; y * y >= z; --y )
        {
            if( ( z % y ) == 0 )
            {
                return ( z % 1337 );
            }
        }
    }

    return 0;
}

long long palindrome( int n )
{
    long long x = static_cast<long long>( n );

    while( n )
    {
        int q = n / 10;
        int r = n - q * 10;

        x = x * 10 + r;
        n = q;
    }

    return x;
}
\end{lstlisting}