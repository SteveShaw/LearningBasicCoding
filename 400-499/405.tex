\section{405 --- Convert a Number to Hexadecimal}
Given an integer, write an algorithm to convert it to hexadecimal. For negative integer, two’s complement method is used.

\paragraph{Note:}

\begin{itemize}
\item All letters in hexadecimal ($a$--$f$) must be in lowercase.
\item The hexadecimal string must not contain extra leading 0s. If the number is zero, it is represented by a single zero character; otherwise, the first character in the hexadecimal string will not be the zero character.
\item The given number is guaranteed to fit within the range of a 32-bit signed integer.
\item You must not use any method provided by the library which converts/formats the number to hex directly.
\end{itemize}

\paragraph{Example 1:}

\begin{flushleft}
\textbf{Input}: 26
\\
\textbf{Output}: \texttt{1a}
\end{flushleft}

\paragraph{Example 2:}

\begin{flushleft}
\textbf{Input}: $-1$
\\
\textbf{Output}: \texttt{ffffffff}
\end{flushleft}

\subsection{Bit Operation}
\begin{itemize}
\item Basic idea: Right shift the number 4 bits, and then mask with 0xF. 
\item To stop the shift, either the number is zero or the number has been right shifted 8 times. The latter is dealing with the case when number is $-1$. Without latter, the shift will continue forever.
\end{itemize}

\setcounter{lstlisting}{0}
\begin{lstlisting}[style=customc, caption={Bit Manipulation}]
string toHex( int num )
{
    if( num == 0 )
    {
        return "0";
    }

    string hex = "0123456789abcdef";

    int mask = 15;

    string ans;

    int shifts = 0;

    //stop when shift 8 times
    //or num is zero
    while( ( shifts < 8 ) && num )
    {
        ans.push_back( hex[num & mask] );
        num >>= 4;
        ++shifts;
    }

    reverse( ans.begin(), ans.end() );

    return ans;
}
\end{lstlisting}

