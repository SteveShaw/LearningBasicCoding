\section{803 --- Bricks Falling When Hit}

\textbf{Hard}

We have a grid of 1s and 0s; the 1s in a cell represent bricks.  A brick will not drop if and only if it is directly connected to the top of the grid, or at least one of its (4-way) adjacent bricks will not drop.

We will do some erasures sequentially. Each time we want to do the erasure at the location \fcj{(i, j)}, the brick (if it exists) on that location will disappear, and then some other bricks may drop because of that erasure.

Return an array representing the number of bricks that will drop after each erasure in sequence.

\paragraph{Example 1:}
\begin{flushleft}

\textbf{Input}: 

\fcj{grid = [[1,0,0,0],[1,1,1,0]]}

\fcj{hits = [[1,0]]}

\textbf{Output}: \fcj{[2]}

\textbf{Explanation}: 

If we erase the brick at (1, 0), the brick at (1, 1) and (1, 2) will drop. So we should return 2.

\end{flushleft}

\paragraph{Example 2:}

\begin{flushleft}
\textbf{Input}: 

\fcj{grid = [[1,0,0,0],[1,1,0,0]]}

\fcj{hits = [[1,1],[1,0]]}

\textbf{Output}: \fcj{[0,0]}

\textbf{Explanation}: 

When we erase the brick at (1, 0), the brick at (1, 1) has already disappeared due to the last move. So each erasure will cause no bricks dropping.  Note that the erased brick (1, 0) will not be counted as a dropped brick.
\end{flushleft}
 

\paragraph{Note:}

\begin{itemize}
\item The number of rows and columns in the grid will be in the range [1, 200].
\item The number of erasures will not exceed the area of the grid.
\item It is guaranteed that each erasure will be different from any other erasure, and located inside the grid.
\item An erasure may refer to a location with no brick - if it does, no bricks drop.
\end{itemize}

\subsection{Union Find}
The problem is about knowing information about the connected components of a graph as we cut vertices. In particular, we'll like to know the size of the component touching the top edge between each cut. Here, a cut refers to the erasure of a vertex.

A useful data structure for joining connected components is a disjoint set union structure. The key idea in this problem is that we can use this structure if we work in reverse: instead of looking at the graph as a series of sequential cuts, we will look at the graph after all the cuts, and reverse each cut.

\setcounter{lstlisting}{0}
\begin{lstlisting}[style=customc, caption={Union Find}]
vector<int> hitBricks( vector<vector<int>>& grid, vector<vector<int>>& hits )
{
    auto M = static_cast< int >( grid.size() );
    auto N = static_cast< int >( grid[0].size() );
    int dr[] = { 1, 0, -1, 0 };
    int dc[] = { 0, 1, 0, -1 };
    //the vectors for DSU
    vector<int> parents( grid.size() * grid[0].size() + 1, 0 );
    vector<int> ranks( grid.size() * grid[0].size() + 1, 0 );
    //the size of connected component
    vector<int> sz_comp( parents.size(), 1 );
    for( size_t i = 0ull; i < parents.size(); ++i )
    {
        parents[i] = i;
    }
    //A is the final state of grid
    //after applying all hits
    vector<vector<int>> A = grid;
    for( auto& hit : hits )
    {
        A[hit[0]][hit[1]] = 0;
    }
    //
    for( int r = 0; r < M; ++r )
    {
        for( int c = 0; c < N; ++c )
        {
            if( A[r][c] == 1 )
            {
                int x = r * N + c;
                if( r == 0 )  //root
                {
                    //all bricks that on the top
                    //are grouped under the point M * N
                    connect( x, M * N, parents, sz_comp, ranks );
                }
                if( ( r > 0 ) && ( A[r - 1][c] == 1 ) )
                {
                    connect( x, ( r - 1 ) * N + c, parents, sz_comp, ranks );
                }
                if( ( c > 0 ) && ( A[r][c - 1] == 1 ) )
                {
                    connect( x, r * N + c - 1, parents, sz_comp, ranks );
                }
            }
        }
    }
    vector<int> ans( hits.size() );
    //traverse hits from end to begin
    //we are simulatiing adding bricks back from the final state
    //thus, the result array also should be reversed
    transform( rbegin( hits ), rend( hits ), rbegin( ans ),
               //lambad function to get the bricks that dropped
               [&]( const vector<int>& hit )
    {
        int r = hit[0];
        int c = hit[1];
        //get the number of bricks that connected to roof
        //before adding brick at (r,c)
        int num_bricks_on_roof = get_bricks_on_roof( parents, sz_comp );
        if( grid[r][c] == 0 )
        {
            //this is not brick
            //nothing is dropped
            return 0;
        }
        //check adjacent bricks
        auto x = r * N + c;
        for( int dir = 0; dir < 4; ++dir )
        {
            int nr = r + dr[dir];
            int nc = c + dc[dir];
            if( ( nr >= 0 ) && ( nr < M ) && ( nc >= 0 ) && ( nc < N ) && ( A[nr][nc] == 1 ) )
            {
                //connect brick at (nr,nc)
                connect( x, nr * N + nc, parents, sz_comp, ranks );
            }
        }
        if( r == 0 )
        {
            //this is the brick on the roof
            //connect to M * N
            connect( x, M * N, parents, sz_comp, ranks );
        }
        //add brick back
        A[r][c] = 1;
        //get the number of bricks that dropped
        //when add brick (r,c) back
        //the reason we minus 1 is that we minus the brick at (r,c)
        return ( max )( 0, get_bricks_on_roof( parents, sz_comp ) - num_bricks_on_roof - 1 );

    } );

    return ans;
}
//union find routine
int find( int x, vector<int>& parents )
{
    while( x != parents[x] )
    {
        x = parents[x];
    }
    return x;
}
//union x and y
//by path compression
void connect( int x, int y, vector<int>& parents, vector<int>& sz_comp, vector<int>& ranks )
{
    auto px = find( x, parents );
    auto py = find( y, parents );

    if( px != py )
    {
        if( ranks[px] < ranks[py] )
        {
            //swap px and py
            //so that px has longer path
            swap( px, py );
        }
        else if( ranks[px] == ranks[py] )
        {
            ranks[px] += 1;
        }
        //set py's parent to px
        parents[py] = px;
        //increase the size of component rooted at px
        sz_comp[px] += sz_comp[py];
    }
}
//get the number of elements in the component that contains
//the point M*N which represents the number of bricks connected to roof
int get_bricks_on_roof( vector<int>& parents, vector<int>& sz_comp )
{
    //find the parent of M * N
    auto x = sz_comp.size() - 1;
    auto px = find( x, parents );
    //the reason we minus 1 is that M*N is a dummy brick
    return sz_comp[px] - 1;
}
\end{lstlisting}

\paragraph{Related Problem}

\begin{itemize}
\item 200. Number of Islands
\item 1101. The Earliest Moment When Everyone Become Friends
\item 1168. Optimize Water Distribution in a Village
\end{itemize}