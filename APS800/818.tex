\section{818 --- Race Car}

\textbf{Hard}

Your car starts at position 0 and speed $+1$ on an infinite number line.  (Your car can go into negative positions.)

Your car drives automatically according to a sequence of instructions A (accelerate) and R (reverse).

When you get an instruction \fcj{"A"}, your car does the following: \fcj{position += speed}, \fcj{speed *= 2}.

When you get an instruction \fcj{"R"}, your car does the following: if your speed is positive then speed is changed to $-1$ , otherwise \fcj{speed = 1}.  (Your position stays the same.)

For example, after commands \fcj{"AAR"}, your car goes to positions \fcj{0->1->3->3}, and your speed goes to \fcj{1->2->4->-1}.

Now for some target position, say the length of the shortest sequence of instructions to get there.

\paragraph{Example 1:}

\begin{flushleft}
\textbf{Input}: 

\fcj{target = 3}

\textbf{Output}: 2

\textbf{Explanation}: 

The shortest instruction sequence is \fcj{"AA"}.

Your position goes from \fcj{0->1->3}.
\end{flushleft}

\paragraph{Example 2:}

\begin{flushleft}
\textbf{Input}: 

\fcj{target = 6}

\textbf{Output}: 5

\textbf{Explanation}: 

The shortest instruction sequence is \fcj{"AAARA"}.

Your position goes from \fcj{0->1->3->7->7->6}.
\end{flushleft}
 

\paragraph{Note:}

\begin{itemize}
\item \fcj{1 <= target <= 10000}.
\end{itemize}

\subsection{Dynamic Programming}
If the target $t$ is between $[2^{k-1}, 2^k)$, we can build a dp array for other target $u$ which is less than $t$.

We may have two ways to reach $t$:

\begin{itemize}
\item Go pass $t$, stop and turn back.

If we take $k$ commands $A$, we will reach 

\[
1 + 2 + 4 + \ldots + 2 ^{k-1} = 2^ k -1
\]

and take one command $R$.

\item Go as far as possible before pass target, stop and turn back.

In this case, We take $k$ commands $A$, and one command $R$. Then we back with $i$ commands $A$ to go back.
\end{itemize}

\setcounter{lstlisting}{0}
\begin{lstlisting}[style=customc, caption={DP}]
int racecar( int target )
{
    vector<int> F( target + 1, 1000000 );
    F[0] = 0;
    F[1] = 1;
    if( target > 1 )
    {
        F[2] = 4;
    }
    //get the k such that
    //2^(k-1) <= x < 2^k
    //we use gcc builtin function __builtin_clz
    //get leading zeros of x.
    //then k = 32 - leading zeros;
    auto get_k = []( unsigned int x )
    {
        return 32 - __builtin_clz( x );
    };
    for( int t = 3; t <= target; ++t )
    {
        int k = get_k( t );
        auto lo = ( 1 << ( k - 1 ) );
        auto hi = ( 1 << k ) - 1;
        if( t == hi )
        {
            //we can reach t by taking (k-1) A commands
            //and one R command
            F[t] = k;
        }
        else
        {
            for( int i = 0; i < k - 1; ++i )
            {
                //we can start from position (t-2^(k-1) + 2^i)
                //take (k-1) commands A, one command R
                //then go back by (i) commands A, and one command R
                F[t] = ( min )( F[t - lo + ( 1 << i )] + ( k - 1 ) + i + 2, F[t] );
            }

            if( hi - t - 1 < t )
            {
                //we can take k commands A to 2^k-1
                //then back to t with number of commands F[hi-t]
                F[t] = ( min )( F[t], F[hi - t] + k + 1 );
            }
        }
    }
    return F[target];
}
\end{lstlisting}