\section{1168 --- Optimize Water Distribution in a Village}

\textbf{Hard}

There are $n$ houses in a village. We want to supply water for all the houses by building wells and laying pipes.

For each house $i$, we can either build a well inside it directly with cost \fcj{wells[i]}, or pipe in water from another well to it. The costs to lay pipes between houses are given by the array \fcj{pipes}, where each \fcj{pipes[i] = [house1, house2, cost]} represents the cost to connect \fcj{house1} and \fcj{house2} together using a pipe. Connections are bidirectional.

Find the minimum total cost to supply water to all houses.


\paragraph{Example 1:}

\begin{flushleft}
\textbf{Input}: $ n = 3 $, \fcj{wells = [1,2,2]}, \fcj{pipes = [[1,2,1],[2,3,1]]}

\textbf{Output}: 3

\textbf{Explanation}: 

The image shows the costs of connecting houses using pipes.

The best strategy is to build a well in the first house with cost 1 and connect the other houses to it with cost 2 so the total cost is 3.

\begin{figure}[H]
\begin{tikzpicture}
[every node/.style={draw, circle, fill=gray!20!, minimum size=9mm},
thick]
\node(1){1};
\node(2)[below=2cm of 1, xshift=-5mm]{2};
\node(3)[below=1.2cm of 1, xshift=3cm]{3};
\draw (1) -- (2);
\draw (2) -- (3);
\end{tikzpicture}
\end{figure}

\end{flushleft}
 

\paragraph{Constraints:}

\begin{itemize}
\item $1 \leq n \leq 10000$
\item \fcj{wells.length == n}
\item \fcj{0 <= wells[i] <= 10^5}
\item \fcj{1 <= pipes.length <= 10000}
\item \fcj{1 <= pipes[i][0], pipes[i][1] <= n}
\item \fcj{0 <= pipes[i][2] <= 10^5}
\item \fcj{pipes[i][0] != pipes[i][1]}
\end{itemize}

\subsection{Minimum Spanning Tree}
This is a typical Minimum Spanning Tree problem, except that at each node, we have the option of digging new wells.

The trick here is to view \textit{digging new wells} as \textit{laying new pipes to a hidden well 0}. Then we have a typical \textbf{MST} problem to solve.

We use Kruskal algorithm using a disjoint set. The pseudocode of this algorithm is as below

\setcounter{lstlisting}{0}
\begin{lstlisting}[style=customc, caption={Kruskal's Algorithm}]
vector<Edge> kruskal( vector<Edge> edges, int numVertices )
{
    DisjSets ds{ numVertices };
    priority_queue pq{ edges };
    vector<Edge> mst;
    while( mst.size( ) != numVertices - 1 )
    {
        Edge e = pq.pop( ); // Edge e = (u, v)
        SetType uset = ds.find( e.getu( ) );
        SetType vset = ds.find( e.getv( ) );
        if( uset != vset )
        {
            // Accept the edge
            mst.push_back( e );
            ds.union( uset, vset );
        }
    }
    return mst;
}
\end{lstlisting}

In the implementation, we first change each \fcj{wells} into a \fcj{pipe} which is the cost from each node to a dummy node 0 and add to the input array \fcj{pipes}.

The priority queue is not needed here since we can sort the pipes inplace.
\begin{lstlisting}[style=customc, caption={MST By Kruskal}]
int minCostToSupplyWater( int n, vector<int>& wells, vector<vector<int>>& pipes )
{
    int node = 1;
    //change each well into a pipe
    //which connect node to a dummy house 0
    //and then insert into pipes
    for( int well : wells )
    {
        pipes.emplace_back( vector<int> {0, node++, well} );
    }
    //sort pipes inplace per the cost
    sort( begin( pipes ), end( pipes ), []( const vector<int>& p1, const vector<int>& p2 )
    {
        return p1[2] < p2[2];
    } );
    //the vectors used for DSU
    vector<int> parents( n + 1ll );
    for( int i = 0; i <= n; ++i )
    {
        parents[i] = i;
    }
    vector<int> ranks( n + 1ll );
    int ans = 0;
    //Kruskal's algorithm
    for( const auto& edge : pipes )
    {
        int x = edge[0];
        int y = edge[1];
        auto px = find( x, parents );
        auto py = find( y, parents );
        if( px != py )
        {
            //we ignore add x and y into MST
            //since the problem doesn't need
            ans += edge[2];
            //connect x and y
            connect( px, py, parents, ranks );
        }
    }
    return ans;
}
//union find
int find( int x, vector<int>& parents )
{
    while( x != parents[x] )
    {
        x = parents[x];
    }

    return x;
}
//connect x and y with path compression
void connect( int x, int y, vector<int>& parents, vector<int>& ranks )
{
    auto px = find( x, parents );
    auto py = find( y, parents );
    if( px != py )
    {
        if( ranks[px] < ranks[py] )
        {
            std::swap( px, py );
        }
        else if( ranks[px] == ranks[py] )
        {
            ++ranks[px];
        }
        parents[py] = px;
    }
}
\end{lstlisting}