\section{849 --- Maximize Distance to Closest Person}

\textbf{Easy}

In a row of \fcj{seats}, 1 represents a person sitting in that seat, and 0 represents that the seat is empty. 

There is at least one empty seat, and at least one person sitting.

Alex wants to sit in the seat such that the distance between him and the closest person to him is maximized. 

Return that maximum distance to closest person.

\paragraph{Example 1:}

\begin{flushleft}
\textbf{Input}: $ [1,0,0,0,1,0,1] $

\textbf{Output}: 2

\textbf{Explanation}: 

If Alex sits in the second open seat (\fcj{seats[2]}), then the closest person has distance 2.

If Alex sits in any other open seat, the closest person has distance 1.

Thus, the maximum distance to the closest person is 2.
\end{flushleft}

\paragraph{Example 2:}

\begin{flushleft}
\textbf{Input}: $ [1,0,0,0] $

\textbf{Output}: 3

\textbf{Explanation}:
 
If Alex sits in the last seat, the closest person is 3 seats away.

This is the maximum distance possible, so the answer is 3.
\end{flushleft}

\paragraph{Note:}

\begin{itemize}
\item $1 \leq \texttt{seats.length} \leq 20000$
\item \fcj{seats} contains only 0s or 1s, at least one 0, and at least one 1.
\end{itemize}

\subsection{One Pass}
We maintain the index of last seated seat.

Then loop on all seats. If a seat is occupied, we count the distance from the index of last occupied seat
.
The final result will be maximum of distance at the beginning, half of distance in the middle and distance at the end.

\setcounter{lstlisting}{0}
\lstinputlisting[style=customc, caption={One Pass}]{849.cpp}