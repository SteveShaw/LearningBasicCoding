\section{828 --- Unique Letter String}

\textbf{Hard}

A character is unique in string $S$ if it occurs exactly once in it.

For example, in string \fcj{S = "LETTER"}, the only unique characters are \fcj{"L"} and \fcj{"R"}.

Let's define \fcj{UNIQ(S)} as the number of unique characters in string $S$.

For example, \fcj{UNIQ("LETTER") =  2}.

Given a string $S$ with only uppercases, calculate the sum of \fcj{UNIQ(substring)} over all non-empty substrings of $S$.

If there are two or more equal substrings at different positions in $S$, we consider them different.

Since the answer can be very large, return the answer modulo $10 ^ 9 + 7$.

 

\paragraph{Example 1:}

\begin{flushleft}
\textbf{Input}: \fcj{"ABC"}

\textbf{Output}: 10

\textbf{Explanation}: All possible substrings are: \fcj{"A"},\fcj{"B"},\fcj{"C"},\fcj{"AB"},\fcj{"BC"} and \fcj{"ABC"}.

Evey substring is composed with only unique letters.

Sum of lengths of all substring is $1 + 1 + 1 + 2 + 2 + 3 = 10$
\end{flushleft}

\paragraph{Example 2:}

\begin{flushleft}
\textbf{Input}: \fcj{"ABA"}

\textbf{Output}: 8

Explanation: The same as example 1, except \fcj{uni("ABA") = 1}.
\end{flushleft}

 

\paragraph{Note:} 

\begin{itemize}
\item $0 \leq \texttt{S.length} \leq 10000$.
\end{itemize}

\subsection{Dynamic Programming}
Take a string \fcj{"XAXAXXAX"} as the example. If we want to find out the substring that contains the second \fcj{"A"} as a unique character, we can take put parenthesis into as \fcj{"XA(XAXX)AX"}. Then the parts inside the parenthesis is one of substrings that we want to find.

What we need to do are
\begin{itemize}
\item insert \fcj{"("} somewhere between the 1st and 2nd \texttt{A}. We have \fcj{"A(XA"} and \fcj{"AX(A"}, 2 ways.
\item insert \fcj{")"} somewhere between the 2nd and 3rd \texttt{A}. We have \fcj{"A)XXA"}, \fcj{"AX)XA"} and \fcj{"AXX)A"}, 3 ways.
\end{itemize}

So there are $2 \times 3 = 6$ ways to make the 2nd \texttt{A} a unique character in a substring. In other words, there are only 6 substring, in which A is a unique letter.

We can count how many ways to to make every letter in S as a unique letter inside a substring. The final result will be the sum of these counts.


\setcounter{lstlisting}{0}
\lstinputlisting[style=customc, caption={DP}]{828.cpp}
