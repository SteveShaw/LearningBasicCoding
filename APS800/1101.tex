\section{1101 --- The Earliest Moment When Everyone Become Friends}

\textbf{Medium}

In a social group, there are $N$ people, with unique integer ids from 0 to $N-1$.

We have a list of logs, where each \fcj{logs[i] = [timestamp, id_A, id_B]} contains a non-negative integer timestamp, and the ids of two different people.

Each log represents the time in which two different people became friends.  Friendship is symmetric: if A is friends with B, then B is friends with A.

Let's say that person A is acquainted with person B if A is friends with B, or A is a friend of someone acquainted with B.

Return the earliest time for which every person became acquainted with every other person. Return $-1$ if there is no such earliest time.

 

\paragraph{Example 1:}

\begin{flushleft}
\textbf{Input}: 

\fcj{logs = } 

\fcj{[[20190101,0,1],[20190104,3,4],[20190107,2,3],}

\fcj{[20190211,1,5],[20190224,2,4],[20190301,0,3],[20190312,1,2],[20190322,4,5]]}
 
$N = 6$

\textbf{Output}: 20190301

Explanation: 

The first event occurs at \fcj{timestamp = 20190101} and after 0 and 1 become friends we have the following friendship groups \fcj{[0,1], [2], [3], [4], [5]}.

The second event occurs at \fcj{timestamp = 20190104} and after 3 and 4 become friends we have the following friendship groups \fcj{[0,1], [2], [3,4], [5]}.

The third event occurs at \fcj{timestamp = 20190107} and after 2 and 3 become friends we have the following friendship groups \fcj{[0,1], [2,3,4], [5]}.

The fourth event occurs at \fcj{timestamp = 20190211} and after 1 and 5 become friends we have the following friendship groups \fcj{[0,1,5], [2,3,4]}.

The fifth event occurs at \fcj{timestamp = 20190224} and as 2 and 4 are already friend anything happens.

The sixth event occurs at \fcj{timestamp = 20190301} and after 0 and 3 become friends we have that all become friends.
\end{flushleft}
 

\paragraph{Note:}

\begin{itemize}
\item $2 \leq N \leq 100$
\item \fcj{1 <= logs.length <= 10^4}
\item \fcj{0 <= logs[i][0] <= 10^9}
\item \fcj{0 <= logs[i][1], logs[i][2] <= N - 1}
\item It's guaranteed that all \fcj{timestamps} in \fcj{logs[i][0]} are different.
\item \fcj{logs} are not necessarily ordered by some criteria.
\item \fcj{logs[i][1] != logs[i][2]}
\end{itemize}

\subsection{Union Find}
We can apply disjoint set to this problem. When connect two nodes, we record the number of groups and finally check if the number of groups is one.

\setcounter{lstlisting}{0}
\begin{lstlisting}[style=customc, caption={DSU}]
int earliestAcq( vector<vector<int>>& logs, int N )
{
    //sort logs so that the timestamp is sorted
    sort( logs.begin(), logs.end() );
    //DSU related vectors
    vector<int> parents( N );
    vector<int> ranks( N );
    for( int i = 0; i < N; ++i )
    {
        parents[i] = i;
    }
    int ans = -1;
    //initially, each person is a group
    //so we have N groups
    int groups = N;
    for( const auto& log : logs )
    {
        auto x = log[1];
        auto y = log[2];
        auto px = find( x, parents );
        auto py = find( y, parents );
        if( px != py )
        {
            //connect these two persons
            //record the timestamp
            ans = log[0];
            connect( px, py, parents, ranks );
            //reduce groups by one
            --groups;
        }
    }
    //only one group will get the answer
    return groups == 1 ? ans : -1;
}
int find( int x, vector<int>& parents )
{
    while( x != parents[x] )
    {
        x = parents[x];
    }

    return x;
}
void connect( int x, int y, vector<int>& parents, vector<int>& ranks )
{
    auto px = find( x, parents );
    auto py = find( y, parents );
    if( px != py )
    {
        if( ranks[px] < ranks[py] )
        {
            std::swap( px, py );
        }
        else if( ranks[px] == ranks[py] )
        {
            ++ranks[px];
        }
        parents[py] = px;
    }
}
\end{lstlisting}