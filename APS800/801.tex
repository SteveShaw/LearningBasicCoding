\section{801 --- Minimum Swaps To Make Sequences Increasing}

\textbf{Medium}

We have two integer sequences A and B of the same non-zero length.

We are allowed to swap elements \fcj{A[i]} and \fcj{B[i]}.  Note that both elements are in the same index position in their respective sequences.

At the end of some number of swaps, A and B are both strictly increasing. (A sequence is strictly increasing if and only if 

\fcj{A[0] < A[1] < A[2] < ... < A[A.length - 1]}. )

Given A and B, return the minimum number of swaps to make both sequences strictly increasing.  It is guaranteed that the given input always makes it possible.

\paragraph{Example:}

\begin{flushleft}
\textbf{Input}: \fcj{A = [1,3,5,4]}, \fcj{B = [1,2,3,7]}

\textbf{Output}: 1

\textbf{Explanation}: 

Swap A[3] and B[3].  Then the sequences are:

\fcj{A = [1, 3, 5, 7]} and \fcj{B = [1, 2, 3, 4]}

which are both strictly increasing.
\end{flushleft}

\paragraph{Note:}

\begin{itemize}
\item A, B are arrays with the same length, and that length will be in the range \fcj{[1, 1000]}.

\item \fcj{A[i]}, \fcj{B[i]} are integer values in the range \fcj{[0, 2000]}.
\end{itemize}

\subsection{Dynamic Programming}
Suppose $n_1$ is the cost of making the first $i-1$ columns increasing and not swapping the $(i-1)$th column; and $s_1$ is the cost of making the first $i-1$ columns increasing and swapping the $(i-1)$th column.

Now we want candidates $n_2$ (and $s_2$), the costs of making the first $i$ columns increasing if we do not swap (or swap, respectively) the $i$th column.

\paragraph{Algorithm}

Say $a_1 = A[i-1]$, $b_1 = B[i-1]$ and $a_2 = A[i]$, $b_2 = B[i]$.

Now, if $a_1 < a_2$ and $b_1 < b_2$, it is allowed to have both of these columns unswapped, or both of these columns swapped. This leads to $n_2 = \min(n_2, n_1)$ and $s_2 = \min(s_2, s_1 + 1)$.

Another possibility is that $ a_1 < b_2 $ and $ b_1 < a_2 $. This means that it is allowed to have exactly one of these columns swapped. This leads to $n_2 = \min(n_2, s_1)$ or $s_2 = \min(s_2, n_1 + 1)$.

Note that it is important to use two \fcj{if} statements separately, because both of the above possibilities might be possible.

At the end, the optimal solution must leave the last column either natural or swapped, so we take the minimum number of swaps between the two possibilities.


\setcounter{lstlisting}{0}
\begin{lstlisting}[style=customc, caption={Dynamic Programming}]
int minSwap( vector<int>& A, vector<int>& B )
{
    //n1 = the cost of making first (i-1) columns sorted
    //but not swap column (i-1)
    int n1 = 0;
    //s1 = the cost of making first (i-1) columns sorted
    //but swap column (i-1)
    int s1 = 1;

    for( size_t i = 1; i < A.size(); ++i )
    {
        int n2 = INT_MAX;
        int s2 = INT_MAX;
        if( ( A[i - 1] < A[i] ) && ( B[i - 1] < B[i] ) )
        {
            //we can choose to not swap column (i-1) and (i)
            n2 = ( min )( n2, n1 );
            //or we can choose to swap column (i-1) and (i)
            s2 = ( min )( s2, s1 + 1 );
        }

        if( ( A[i - 1] < B[i] ) && ( B[i - 1] < A[i] ) )
        {
            //we can choose to swap column (i-1)
            //but leave column i not swapped
            n2 = ( min )( n2, s1 );
            //or we can choose to swap column i
            //but leave column (i-1) not swapped
            s2 = ( min )( s2, n1 + 1 );
        }

        n1 = n2;
        s1 = s2;
    }
    //the minimum of n1 and s1 is the answer
    return ( min )( n1, s1 );
}
\end{lstlisting}
