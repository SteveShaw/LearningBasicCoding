\section{843 --- Guess the Word}

\textbf{Hard}

We are given a word list of unique words, each word is 6 letters long, and one word in this list is chosen as \textbf{secret}.

You may call \fcj{master.guess(word)} to guess a word.  The guessed word should have type \fcj{string} and must be from the original list with 6 lowercase letters.

This function returns an \fcj{integer} type, representing the number of exact matches (value and position) of your guess to the \textbf{secret word}.  Also, if your guess is not in the given \fcj{wordlist}, it will return $-1$ instead.

For each test case, you have 10 guesses to guess the word. At the end of any number of calls, if you have made 10 or less calls to \fcj{master.guess} and at least one of these guesses was the \textbf{secret}, you pass the testcase.

Besides the example test case below, there will be 5 additional test cases, each with 100 words in the word list.  The letters of each word in those testcases were chosen independently at random from \fcj{'a'} to \fcj{'z'}, such that every word in the given word lists is unique.

\paragraph{Example 1:}

\begin{flushleft}
\textbf{Input}: \fcj{secret = "acckzz"}, \fcj{wordlist = ["acckzz","ccbazz","eiowzz","abcczz"]}

\textbf{Explanation}:

\fcj{master.guess("aaaaaa")} returns $-1$, because \fcj{"aaaaaa"} is not in wordlist.

\fcj{master.guess("acckzz")} returns 6, because \fcj{"acckzz"} is secret and has all 6 matches.

\fcj{master.guess("ccbazz")} returns 3, because \fcj{"ccbazz"} has 3 matches.

\fcj{master.guess("eiowzz")} returns 2, because \fcj{"eiowzz"} has 2 matches.

\fcj{master.guess("abcczz")} returns 4, because \fcj{"abcczz"} has 4 matches.

We made 5 calls to master.guess and one of them was the secret, so we pass the test case.
\end{flushleft}
