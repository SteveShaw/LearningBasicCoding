\section{800 --- Similar RGB Color}
In the following, every capital letter represents some hexadecimal digit from 0 to f.

The red-green-blue color \fcj{"#AABBCC"} can be written as \fcj{"#ABC"} in shorthand.  For example, \fcj{"#15c"} is shorthand for the color \fcj{"#1155cc"}.

Now, say the similarity between two colors \fcj{"#ABCDEF"} and \fcj{"#UVWXYZ"} is 

\fcj{-(AB - UV)^2 - (CD - WX)^2 - (EF - YZ)^2}.

Given the color \fcj{"#ABCDEF"}, return a 7 character color that is most similar to \fcj{#ABCDEF}, and has a shorthand (that is, it can be represented as some \fcj{"#XYZ"}

\paragraph{Example 1:}

\begin{flushleft}
\textbf{Input}: \fcj{color = "#09f166"}

\textbf{Output}: \fcj{"#11ee66"}

\textbf{Explanation}:  

The similarity is \fcj{-(0x09 - 0x11)^2 -(0xf1 - 0xee)^2 - (0x66 - 0x66)^2 = -64 -9 -0 = -73}.

This is the highest among any shorthand color.
\end{flushleft}

\paragraph{Note:}

\begin{itemize}
\item \fcj{color} is a string of length 7.
\item \fcj{color} is a valid RGB color: for \fcj{i > 0}, \fcj{color[i]} is a hexadecimal digit from 0 to f
\item Any answer which has the same (highest) similarity as the best answer will be accepted.
\item All inputs and outputs should use lowercase letters, and the output is 7 characters.
\end{itemize}

\subsection{Brute Force}
Each digit in the shorthand \fcj{"#RGB"} could be from 0 to 15. This leads to a color of 
\fcj{17 * R * (1 << 16) + 17 * G * (1 << 8) + 17 * B}. 

The reason for the 17 is because a hexadecimal value of \fcj{0x22} is equal to \fcj{2 * 16 + 2 * 1} which is \fcj{2 * (17)}. The other values for red and green work similarly, just shifted up by 8 or 16 bits.

It should be noted that the answer is always unique. Indeed, for two numbers that differ by 17, every integer is closer to one than the other. For example, with 0 and 17, 8 is closer to 0 and 9 is closer to 17 --- there is no number that is tied in closeness.

\setcounter{lstlisting}{0}
\begin{lstlisting}[style=customc, caption={Brute Force}]
string similarRGB( string color )
{
    for( size_t i = 1; i < color.size(); i += 2 )
    {
        //change to integer
        auto num = stoi( color.substr( i, 2 ), nullptr, 16 );
        //get the closet integer with form (cc)
        int x = num / 17 + ( ( num % 17 ) > 8 ? 1 : 0 );
        //change to string
        color[i] = ( x > 9 ) ? ( x - 10 + 'a' ) : ( x + '0' );
        color[i + 1] = color[i];
    }

    return color;
}
\end{lstlisting}