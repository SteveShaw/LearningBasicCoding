\section{805 --- Split Array With Same Average}

\textbf{Hard}

In a given integer array A, we must move every element of A to either list B or list C. (B and C initially start empty.)

Return \fcj{true} if and only if after such a move, it is possible that the average value of B is equal to the average value of C, and B and C are both non-empty.

\paragraph{Example:}

\begin{flushleft}
\textbf{Input}: 

\fcj{[1,2,3,4,5,6,7,8]}

\textbf{Output}: \fcj{true}

Explanation: We can split the array into \fcj{[1,4,5,8]} and \fcj{[2,3,6,7]}, and both of them have the average of 4.5.
\end{flushleft}

\paragraph{Note:}

\begin{itemize}
\item The length of A will be in the range \fcj{[1, 30]}.
\item \fcj{A[i]} will be in the range of \fcj{[0, 10000]}.
\end{itemize}

\subsection{Dynamic Programming}

Suppose $A$ has $N$ numbers, and $B$ has $K$ elements. Then $C$ will have $N-K$ numbers.

Assume the total sum of $A$ is $x$, and $B$ is $y$, we have $y/K = (x-y)/(N-K)$. Then $y\times(N-K) = K\times (x-y)$, finally we can get $y\times N = K\times x$, i.e., $K\times x /N = y$.

Since $y$ is integer, $K\times x$ must be $N$'s multiple. Since $A$ is divided into two arrays, we can let $B$ has smaller size. Thus, $K \in [1, N/2]$. Combined with above discussion, if we cannot find a $K$ in $[1,N/2]$ such that $K\times x$ is not multiple $N$, we can directly return \fcj{false}.

So, if we find the $K$, the problem will be find $K$ numbers in $A$ such that the sum of these numbers is $y$.

Assume $F[i]$ is the all possible sums when select $i$ numbers in $A$. Then, for $F[i+1]$, we add a new number. To achieve this, we need three level loops:

\begin{enumerate}
\item Level 1: each number in $A$
\item Level 2: from $N/2$ to 1 ( each $i$ )
\item Level 3: each sum in $F[i]$.
\end{enumerate}

Then we test each $K$ to find if $y$ exists in $F$.

\setcounter{lstlisting}{0}
\begin{lstlisting}[style=customc, caption={DP}]
bool splitArraySameAverage( vector<int>& A )
{
    int sum = accumulate( begin( A ), end( A ), 0 );
    auto sz = A.size();
    auto t = ( sz / 2 );
    bool possible = false;
    //check if (sum * K) is multiple of szs
    for( auto K = 1ull; K <= t; ++K )
    {
        if( ( sum * K ) % sz == 0 )
        {
            possible = true;
            break;
        }
    }
    if( !possible )
    {
        return false;
    }
    //F[i]: find all possible sums when
    //selecting i numbers from A
    //These i numbers could be duplicate
    //so F could include sum that select duplicate numbers
    //but we don't care
    //we only care if the sum can be found
    vector<unordered_set<int>> F( t + 1 );
    //to avoid (i-1) overflow
    F[0].insert( 0 );
    for( int x : A )
    {
        for( auto K = t; K >= 1ull; --K )
        {
            for( int s : F[K - 1] )
            {
                F[K].insert( s + x );
            }
        }
    }
    //test if such a K exists
    for( auto K = 1; K <= t; ++K )
    {
        if( ( sum * K ) % sz == 0 )
        {
            auto sum1 = ( sum * K ) / sz;
            if( F[K].find( sum1 ) != F[K].end() )
            {
                return true;
            }
        }
    }
    return false;
}
\end{lstlisting}