\section{825 --- Friends Of Appropriate Ages}

\textbf{Medium}

Some people will make friend requests. The list of their ages is given and \fcj{ages[i]} is the age of the $i$-th person. 

Person $A$ will NOT friend request person $B$ ($B \neq A$) if any of the following conditions are true:

\begin{itemize}
\item \fcj{age[B] <= 0.5 * age[A] + 7}
\item \fcc{age[B] > age[A]}
\item \fcj{age[B] > 100 && age[A] < 100}
\end{itemize}

Otherwise, A will friend request B.

Note that if A requests B, B does not necessarily request A.  Also, people will not friend request themselves.

How many total friend requests are made?

\paragraph{Example 1:}

\begin{flushleft}
\textbf{Input}: \fcj{[16,16]}

\textbf{Output}: 2

\textbf{Explanation}: 2 people friend request each other.

\end{flushleft}

\paragraph{Example 2:}

\begin{flushleft}
\textbf{Input}: \fcj{[16,17,18]}

\textbf{Output}: 2

\textbf{Explanation}: Friend requests are made \fcj{17 -> 16}, \fcj{18 -> 17}.
\end{flushleft}

\paragraph{Example 3:}

\begin{flushleft}
Input: \fcj{[20,30,100,110,120]}

\textbf{Output}: 3

\textbf{Explanation}: Friend requests are made \fcj{110 -> 100}, \fcj{120 -> 110}, \fcj{120 -> 100}.
\end{flushleft}

 

\paragraph{Notes:}

\begin{itemize}
\item $1 \leq \lvert \text{ages}\rvert \leq 20000$.
\item $1 \leq \text{ages}[i] \leq 120$.
\end{itemize}

\subsection{Counting}
We can count number of people who are at the same age. Since there are only 120 possible ages, this is a much faster loop.

For each pair \fcj{(ageA, countA)}, \fcj{(ageB, countB)}, if the conditions are satisfied with respect to age, then \fcj{countA * countB} pairs of people made friend requests.

If \fcj{ageA == ageB}, then we overcounted: we should have \fcj{countA * (countA - 1)} pairs of people making friend requests instead, as we cannot friend request yourself.

\setcounter{lstlisting}{0}
\lstinputlisting[style=customc, caption={Counting}]{825.cpp}




