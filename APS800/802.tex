\section{802 --- Find Eventual Safe States}

\textbf{Medium}

In a directed graph, we start at some node and every turn, walk along a directed edge of the graph. If we reach a node that is terminal (that is, it has no outgoing directed edges), we stop.

Now, say our starting node is eventually safe if and only if we must eventually walk to a terminal node. More specifically, there exists a natural number K so that for any choice of where to walk, we must have stopped at a terminal node in less than K steps.

Which nodes are eventually safe?  Return them as an array in sorted order.

The directed graph has $N$ nodes with labels 0, 1, $\ldots$, $N-1$, where $N$ is the length of graph. The graph is given in the following form: \fcj{graph[i]} is a list of labels $j$ such that $ (i, j) $ is a directed edge of the graph.

\paragraph{Example:}
\begin{flushleft}


\textbf{Input}: \fcj{graph = [[1,2],[2,3],[5],[0],[5],[],[]]}

\textbf{Output}: \fcj{[2,4,5,6]}

Here is a diagram of the above graph.

\begin{figure}[H]
\begin{tikzpicture}
[every node/.style={draw, circle, fill=gray!20!, minimum size=5mm},
thick, ->, >=stealth, node distance=9mm, start chain]
\node(0)[on chain]{0};
\node(1)[on chain]{1};
\node(2)[on chain]{2};
\node(3)[on chain]{3};
\node(4)[on chain]{4};
\node(5)[on chain]{5};
\node(6)[on chain]{6};
\draw (0) to [bend left=45] (1);
\draw (0) to [bend left=60] (2);
\draw (1) to [bend right=45] (2);
\draw (1) to [bend right=60] (3);
\draw (2) to [bend left=60] (5);
\draw (4) to [bend left=45] (5);
\draw[red] (3) to [bend right=60] (0);
\end{tikzpicture}
\end{figure}

\end{flushleft}

\paragraph{Note:}

\begin{itemize}
\item \fcj{graph} will have length at most 10000.
\item The number of edges in the \fcj{graph} will not exceed 32000.
\item Each \fcj{graph[i]} will be a sorted list of different integers, chosen within the range 

\fcj{[0, graph.length - 1]}.
\end{itemize}

\subsection{Depth-First Search}
We can apply a classic \textit{white-gray-black} DFS algorithm by marking the node \textit{gray} on entry and \textit{black} on exit. If we see a \textit{gray} node during DFS, it must be part of a cycle.

For this problem, When we visit a node, the only possibilities are that we've marked the entire subtree black (which must be eventually safe), or it has a cycle and we have only marked the members of that cycle gray. So \textit{gray} nodes are always part of a cycle, and \textit{black} nodes are always eventually safe.

\setcounter{lstlisting}{0}
\begin{lstlisting}[style=customc, caption={DP}]
vector<int> eventualSafeNodes( vector<vector<int>>& graph )
{
    vector<int> color( graph.size(), 0 );
    auto N = static_cast< int >( graph.size() );
    vector<int> ans;
    for( int i = 0ull; i < N; ++i )
    {
        if( dfs( graph, i, color ) )
        {
            ans.push_back( i );
        }
    }
    return ans;
}
//0: white, 1: gray, 2: black
bool dfs( vector<vector<int>>& G, int start, vector<int>& color )
{
    if( color[start] > 0 )
    {
        //black node is safe node
        return color[start] == 2;
    }
    //mark current node as gray
    color[start] = 1;
    for( int adj : G[start] )
    {
        if( color[adj] == 2 )
        {
            //this is already a safe node
            //no need to visit
            continue;
        }
        else if( color[adj] == 1 )
        {
            //we are in a cycle
            return false;
        }
        else
        {
            if( !dfs( G, adj, color ) )
            {
                return false;
            }
        }
    }
    //mark node as black
    color[start] = 2;
    return true;
}
\end{lstlisting}