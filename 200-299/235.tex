\section{235 --- Lowest Common Ancestor of a Binary Search Tree}
Given a binary search tree (\texttt{BST}), find the lowest common ancestor (\texttt{LCA}) of two given nodes in the \texttt{BST}.
\par
According to the definition of \texttt{LCA} on \textbf{Wikipedia}: \textit{The lowest common ancestor is defined between two nodes $p$ and $q$ as the lowest node in $T$ that has both $p$ and $q$ as descendants (where we allow \textbf{a node to be a descendant of itself}).}

\paragraph{Example 1:}

\begin{flushleft}
\textbf{Input}: 
\begin{figure}[H]
\begin{tikzpicture}
[mynode/.style={draw,circle,minimum size=5mm, fill=gray!20!}]
\node(){};
\node[mynode](1) {6};
\node[mynode](2)[below=8mm of 1, xshift=-13mm] {2};
\node[mynode](3)[below=8mm of 1, xshift=13mm] {8};
\node[mynode](4)[below=8mm of 2, xshift=-7mm] {0};
\node[mynode](5)[below=8mm of 2, xshift=7mm] {4};
\node[mynode](6)[below=8mm of 3, xshift=-7mm] {7};
\node[mynode](7)[below=8mm of 3, xshift=7mm] {9};
\node[mynode](8)[below=8mm of 5, xshift=-7mm] {3};
\node[mynode](9)[below=8mm of 5, xshift=7mm] {5};
\draw (1) -- (2);
\draw (1) -- (3);
\draw (2) -- (4);
\draw (2) -- (5);
\draw (3) -- (6);
\draw (3) -- (7);
\draw (5) -- (8);
\draw (5) -- (9);
\end{tikzpicture}
\end{figure}
$p = 2$, $q = 8$
\\
\textbf{Output}: 6
\\
\textbf{Explanation}: 
\\
The LCA of nodes 2 and 8 is 6.
\end{flushleft}

\paragraph{Example 2:}

\begin{flushleft}
\textbf{Input}:
\begin{figure}[H]
\begin{tikzpicture}
[mynode/.style={draw,circle,minimum size=5mm, fill=gray!20!}]
\node(){};
\node[mynode](1) {6};
\node[mynode](2)[below=8mm of 1, xshift=-13mm] {2};
\node[mynode](3)[below=8mm of 1, xshift=13mm] {8};
\node[mynode](4)[below=8mm of 2, xshift=-7mm] {0};
\node[mynode](5)[below=8mm of 2, xshift=7mm] {4};
\node[mynode](6)[below=8mm of 3, xshift=-7mm] {7};
\node[mynode](7)[below=8mm of 3, xshift=7mm] {9};
\node[mynode](8)[below=8mm of 5, xshift=-7mm] {3};
\node[mynode](9)[below=8mm of 5, xshift=7mm] {5};
\draw (1) -- (2);
\draw (1) -- (3);
\draw (2) -- (4);
\draw (2) -- (5);
\draw (3) -- (6);
\draw (3) -- (7);
\draw (5) -- (8);
\draw (5) -- (9);
\end{tikzpicture}
\end{figure}
$p = 2$, $q = 4$
\\
\textbf{Output}: 2
\\
\textbf{Explanation}: 
\\
The LCA of nodes 2 and 4 is 2, since a node can be a descendant of itself according to the LCA definition.
\end{flushleft}

\paragraph{Note:}

\begin{itemize}
\item All of the nodes' values will be unique.
\item $p$ and $q$ are different and both values will exist in the \texttt{BST}.
\end{itemize}
\subsection{BST Search}

借助\texttt{BST}的特性,由于题目中说明了每个节点的value都是unique的,所以
\begin{itemize}
\item 当根节点的value比$p$和$q$的value都大时,说明$p$和$q$的\texttt{LCA}在左子树,因此递归到左子树寻找。
\item 当根节点的value比$p$和$q$的value都大小时,说明$p$和$q$的\texttt{LCA}在右子树,因此递归到右子树寻找。
\item 否则,根节点的value处于$p$和$q$的value中间,因此根节点就是$p$和$q$的\texttt{LCA}。
\end{itemize}

\setcounter{lstlisting}{0}
\begin{lstlisting}[style=customc, caption={BST Properties}]
TreeNode* LCA( TreeNode* root, TreeNode* p, TreeNode* q )
{
    if( !root )
    {
        return root;
    }

    if( ( root->val > p->val ) && ( root->val > q->val ) )
    {
        //LCA in left subtree
        return LCA( root->left, p, q );
    }

    if( ( root->val < p->val ) && ( root->val < q->val ) )
    {
        //LCA in right subtree
        return LCA( root->right, p, q );
    }

    return root;
}

\end{lstlisting}