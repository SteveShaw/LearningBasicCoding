\section{241 --- Different Ways to Add Parentheses}
Given a string of numbers and operators, return all possible results from computing all the different possible ways to group numbers and operators. The valid operators are $+$, $-$ and $\times$.

\paragraph{Example 1:}
\begin{flushleft}
Input: $2-1-1$
\\
Output: $[0, 2]$
\\
Explanation: 
\\
$((2-1)-1) = 0$
\\
$(2-(1-1)) = 2$
\end{flushleft}

\paragraph{Example 2:}
\begin{flushleft}
Input: $2\times3-4\times5$
\\
Output: $[-34, -14, -10, -10, 10]$
\\
Explanation:
\\ 
$(2\times(3-(4\times5))) = -34 $
\\
$((2\times3)-(4\times5)) = -14 $
\\
$((2\times(3-4))\times5) = -10 $
\\
$(2\times((3-4)\times5)) = -10 $
\\
$(((2\times3)-4)\times5) = 10$
\end{flushleft}
\subsection{Recursion}
\begin{itemize}
\item 遍历输入数组,每次遇到操作符号时,就从这个地方分成左右两个部分。
\item 分别对左右两个部分进行递归,返回结果就是各个部分运算得到的结果数组
\item 将两个数组中的结果两两组合即得到当前该操作符的所有运算结果。
\item 如果输入字符串只包含数字,则将该数字作为输出数组中的唯一元素。
\item 可以用memorize的方法减少重复的搜索。
\end{itemize}
\setcounter{lstlisting}{0}
\begin{lstlisting}[style=customc, caption={Recursion Without Memorization}]
vector<int> diffWaysToCompute( string input )
{
    vector<int> results;

    int num = 0;

    for( size_t i = 0; i < input.size(); ++i )
    {
        auto c = input[i];

        //if input has only number, num will be the number
        if( ( c >= '0' ) && ( c <= '9' ) )
        {
            num = num * 10 + ( c - '0' );
            continue;
        }

        //we break input into two parts
        //and recursively apply function to the two parts
        vector<int> left = diffWaysToCompute( input.substr( 0, i ) );
        vector<int> right = diffWaysToCompute( input.substr( i + 1 ) );

        //calculate results from the output arrays
        calc( left, right, results, c );

        num = 0;
    }

    // input contains only number
    if( results.empty() )
    {
        results.push_back( num );
    }

    return results;
}


void calc( vector<int>& left, vector<int>& right, vector<int>& results, char oper )
{
    if( oper == '*' )
    {
        for( auto l : left )
        {
            for( auto r : right )
            {
                results.push_back( l * r );
            }
        }
    }
    else if( oper == '-' )
    {
        for( auto l : left )
        {
            for( auto r : right )
            {
                results.push_back( l - r );
            }
        }
    }
    else if( oper == '+' )
    {
        for( auto l : left )
        {
            for( auto r : right )
            {
                results.push_back( l + r );
            }
        }
    }
}
\end{lstlisting}


\begin{lstlisting}[style=customc, caption={String View}]
vector<int> diffWaysToCompute( string input )
{
    return dfs( input );
}
vector<int> dfs( string_view sv )
{
    int num = 0;
    vector<int> vec_res;
    for( size_t i = 0; i < sv.size(); ++i )
    {
        auto c = sv[i];
        if( ( c >= '0' ) && ( c <= '9' ) )
        {
            num = num * 10 + ( c - '0' );
        }
        else
        {
            //recursive on left and right
            auto lhs = dfs( sv.substr( 0, i ) );
            auto rhs = dfs( sv.substr( i + 1 ) );
            if( c == '+' )
            {
                combine( lhs, rhs, vec_res, std::plus<int>() );
            }
            else if( c == '-' )
            {
                combine( lhs, rhs, vec_res, std::minus<int>() );
            }
            else
            {
                combine( lhs, rhs, vec_res, std::multiplies<int>() );
            }
            //reset num to zero
            num = 0;
        }
    }
    //if there is only number
    //add number to the result
    if( vec_res.empty() )
    {
        vec_res.push_back( num );
    }
    return vec_res;
}
//generate all possible results
template<typename F>
void combine( vector<int>& lhs, vector<int>& rhs, vector<int>& res, F f )
{
    for( int l : lhs )
    {
        for( int r : rhs )
        {
            res.push_back( f( l, r ) );
        }
    }
}
\end{lstlisting}