\section{216 --- Combination Sum III}
Find all possible combinations of $k$ numbers that add up to a number $n$, given that only numbers from 1 to 9 can be used and each combination should be a unique set of numbers.
\paragraph{Note:}
\begin{itemize}
\item All numbers will be positive integers.
\item The solution set must not contain duplicate combinations.
\end{itemize}
\paragraph{Example 1:}
\begin{flushleft}
\textbf{Input}: $k = 3$, $n = 7$
\\
\textbf{Output}: 
\\
$[[1,2,4]]$
\end{flushleft}
\paragraph{Example 2:}
\begin{flushleft}
\textbf{Input}: $k = 3$, $n = 9$
\\
\textbf{Output}: 
\\
$[[1,2,6], [1,3,5], [2,3,4]]$
\end{flushleft}
\subsection{Backtracking Recursion}
\begin{CJK*}{UTF8}{gbsn}
\begin{itemize}
\item 递归生成array
\item 每个小于$n$的数,先加入到当前array中,然后继续递归,结束后移除这个加入的数,然后前进到下一个数
\item 为保证不出现重复数字,递归函数应当有一个参数来表示当前搜索到哪个数字了。
\end{itemize}
\end{CJK*}
\setcounter{algorithm}{0}
\begin{algorithm}[H]
\caption{Backtracking Recursive Approach}
\begin{algorithmic}[1]
\Procedure{CombinationSumIII}{$k, n$}
\State $\star$ Initialize an empty array $x$ which will contain all valid arrays meet the requirements
\State $\star$ Initialize an empty array $y$ which save the intermediate result
\State \Call{DFS}{$n,k,1,0,x,y$} \Comment Call helper function \texttt{DFS}
\State \Return $x$
\algstore{216algo}
\end{algorithmic}
\end{algorithm}
\begin{algorithm}[H]
\begin{algorithmic}[1]
\algrestore{216algo}
\EndProcedure
\end{algorithmic}
\end{algorithm}
Input parameters of function \texttt{DFS} are
\begin{enumerate}
\item $n$: The target sum
\item $k$: The target count of numbers
\item $\alpha$: The start number to search
\item $\sigma$: The current sum of $y$
\item $y$: The current array
\item $x$: The final results
\end{enumerate}
\begin{algorithm}[H]
\caption{Helper Function}
\begin{algorithmic}[1]
\Procedure{DFS}{$n,k,\alpha,\sigma,y, x$}
\State $\star$ When $k$ is zero, the target count of numbers is reached. Recursion will stop
\If{$k = 0$}
\If{$\sigma=n$} 
\State $x\gets x+y$ \Comment Add current array to result
\EndIf
\State \Return \Comment Stop recursion
\EndIf
\State $\star$ Start search from $\alpha$
\For{$i:=\alpha \to 9$}
\State $\star$ Only number less than $n$ is accepted
\If{$i < n$}
\State $y\gets y+i$ 
\State $\star$ By adding $i$ to $y$, the count of numbers to be found is reduced to $k-1$
\State $\star$ To avoid duplicate number in $y$, next search start from $\alpha+1$
\State $\star$ Current sum $\delta$ is increased by $i$
\State \Call{DFS}{$n, k-1, \alpha+1, \delta+i, y, x$}
\State $y\gets y-i$ \Comment Remove $i$ from $y$ 
\EndIf
\EndFor
\EndProcedure
\end{algorithmic}
\end{algorithm}
\setcounter{lstlisting}{0}
\begin{lstlisting}[style=customc, caption={Backtracking}]
vector<vector<int>> combinationSum3( int k, int n )
{
    vector<vector<int>> ans;
    vector<int> v;

    dfs( n, k, 0, 1, v, ans );
	
    return ans;
}
//n: the count of numbers in v
//k: the required count of numbers
//sum: current sum of numbers in v
//start: The start number that will check
//v: The current array
void dfs( int n, int k, int sum, int start, vector<int>& v, vector<vector<int>>& ans )
{
    if( k == 0 )
    {
        if( sum == n )
        {
            ans.emplace_back( v.begin(), v.end() );
        }

        return;
    }

    for( int i = start; i <= 9; ++i )
    {
        if( i < n )
        {
			//Add i to v
            v.push_back( i );
            dfs( n, k - 1, sum + i, i + 1, v, ans );
			//Remove i from v
            v.pop_back();
        }
    }
}
\end{lstlisting}