\section{264 --- Ugly Number II}
Write a program to find the $n$-th ugly number.
\par
Ugly numbers are positive numbers whose prime factors only include 2, 3, 5. 

\paragraph{Example:}

\begin{flushleft}
\textbf{Input}: $n = 10$
\\
\textbf{Output}: 12
\\
\textbf{Explanation}: 1, 2, 3, 4, 5, 6, 8, 9, 10, 12 is the sequence of the first 10 ugly numbers.
\end{flushleft}

\paragraph{Note:}  

\begin{itemize}
\item 1 is typically treated as an ugly number.
\item $n$ does not exceed 1690.
\end{itemize}
\subsection{Dynamic Programming}
\begin{itemize}
\item 设定3个计数器$x$,$y$和$z$。分别对应factor 2,3和5。
\item maintain一个size为$n$的DP数组$A$
\item 从第一个ugly number 1开始,$A[0]=1$。下一个数就是$A[0]\times 2$,$A[0]\times 3$和$A[0]\times5$中的最小值,结果是2,因此increment对应2的计数器$x$。所以$A[1]=2$
\item 接下来 $A[2] = \min(A[1]\times2, A[0]\times 3, A[0]\times 5)$。显然是$A[0]\times 3$最小,因此increment对应3的计数器$y$。
\item 以此类推,如果遇到相同的情况,比如6,那么对应的2和3的计数器都要increment。
\end{itemize}


\setcounter{lstlisting}{0}
\begin{lstlisting}[style=customc, caption={Dynamic Programming}]
int nthUglyNumber( int n )
{
    vector<int> A( n, 0 );
    A[0] = 1;
    int id2 = 0;
    int id3 = 0;
    int id5 = 0;
    for( int i = 1; i < n; ++i )
    {
        //find next ugly number
        int n2 = A[id2] * 2;
        int n3 = A[id3] * 3;
        int n5 = A[id5] * 5;
        A[i] = ( min )( n2, n3 );
        A[i] = ( min )( A[i], n5 );
        //check index increment for 2, 3, 5
        //we should check all 3 numbers
        //so no "else if" here
        if( A[i] == n2 )
        {
            ++id2;
        }
        if( A[i] == n3 )
        {
            ++id3;
        }
        if( A[i] == n5 )
        {
            ++id5;
        }
    }
    return A.back();
}
\end{lstlisting}

\paragraph{Related Problems}
\begin{itemize}
\item \textbf{279. Perfect Squares}
\item \textbf{313. Super Ugly Number}
\item \textbf{1201. Ugly Number III}
\end{itemize}