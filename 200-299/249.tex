\section{249 --- Group Shifted Strings}
Given a string $S$, we can \texttt{shift} each of its letter to its successive letter, for example: $\texttt{abc} \longrightarrow \texttt{bcd}$. We can keep \texttt{shifting} which forms the sequence:
\par
$\texttt{abc} \longrightarrow \texttt{bcd} \longrightarrow \ldots \longrightarrow \texttt{xyz}$
\par
Given a list of strings $A$ which contains only lowercase alphabets, group all strings that belong to the same shifting sequence.
\par
For example, given: 
$[\texttt{abc},\; \texttt{bcd},\; \texttt{acef},\; \texttt{xyz},\; \texttt{az},\; \texttt{ba},\; \texttt{a},\; \texttt{z}]$,
\par
Return:
\begin{table}[H]
\begin{tabular}{l}
$\texttt{abc},\;\texttt{bcd},\;\texttt{xyz}$ \\
$\texttt{az},\;\texttt{ba}$\\
\texttt{acef}\\
$\texttt{a},\;\texttt{z}$
\end{tabular}
\end{table}

\paragraph{Note:} 
\begin{itemize}
\item For the return value, each inner list's elements must follow the lexicographic order.
\end{itemize}
\subsection{Unique Distance}
注意到shift的字符串中其实每个字母和首字符的相对差都是相等的,比如\fcj{abc}和\fcj{efg}互为shift strings,对于\fcj{abc}来说,$b$和$a$的差是1,$c$和$a$的差是2,对于\texttt{efg}来说,$f$和$e$的difference是1,$g$和$e$的difference是2。利用这个特性就能很好的对字符串进行分组了。

技巧就是用每个字符串中每个字符和首字符的difference构成一个key,然后用这个key作为group string的依照。比如\fcj{abc}的每个字符和首字符的difference构成的key为$012$,然后用$012$这个key去判断哪些字符也具有这个key。

在coding时,这个key可以转换成字符串。以避免形成0开头的数字所带来的重复key的问题。

另外,由于题目要求是按照字典顺序,同时考虑到可能存在重复的字符串,因此用multiset来存放相同的字符串。

\setcounter{lstlisting}{0}
\begin{lstlisting}[style=customc, caption={Unique Distance}]
vector<vector<string>> groupStrings( vector<string>& strings )
{
 vector<vector<string>> groupStrings( vector<string>& strings )
{
    unordered_map<string, vector<string_view>> dict;
    for( const auto& s : strings )
    {
        string key;
        for( size_t i = 1; i < s.size(); ++i )
        {
            //make sure the difference between
            //current letter and the first letter
            //is positive number
            key += to_string( ( s[i] - s[0] + 26 ) % 26 );
            key.push_back( '#' );
        }
        dict[key].emplace_back( s.c_str(), s.size() );
    }
    vector<vector<string>> ans;
    for( const auto& p : dict )
    {
        ans.emplace_back( begin( p.second ), end( p.second ) );
    }
    return ans;
}
\end{lstlisting}