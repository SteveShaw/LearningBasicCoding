\section{299 --- Bulls and Cows}
You are playing the following \textbf{Bulls and Cows} game with your friend: You write down a number and ask your friend to guess what the number is. Each time your friend makes a guess, you provide a hint that indicates how many digits in said guess match your secret number exactly in both digit and position (called \textit{bulls}) and how many digits match the secret number but locate in the wrong position (called \textit{cows}). Your friend will use successive guesses and hints to eventually derive the secret number.
\par
Write a function to return a hint according to the secret number and friend's guess, use $A$ to indicate the bulls and $B$ to indicate the cows. 
\par

Please note that both secret number and friend's guess may contain duplicate digits.

\paragraph{Example 1:}

\begin{flushleft}
\textbf{Input}: secret = 1807, guess = 7810
\\
\textbf{Output}: $1A3B$
\\
\textbf{Explanation}: 1 bull and 3 cows. The bull is 8, the cows are 0, 1 and 7.
\end{flushleft}

\paragraph{Example 2:}

\begin{flushleft}
\textbf{Input}: secret = 1123, guess = 0111
\\
\textbf{Output}: $1A1B$
\\
\textbf{Explanation}: The 1st 1 in friend's guess is a bull, the 2nd or 3rd 1 is a cow.
\end{flushleft}

\paragraph{Note:} 
\begin{itemize}
\item You may assume that the secret number and your friend's guess only contain digits, and their lengths are always equal.
\end{itemize}
\subsection{Tricks}
\begin{itemize}
\item 算法思路: iterate over secret $S$ 和 guess $G$,首先可以很快得到所有bulls的(即数字和出现位置都相等)。对于cows(即数字相同,但是出现位置不同),则maintain一个array用于记录数字在$S$和$G$中的出现次数。技巧在于,对于$G$中的数字的出现次数,用负数来表示。
\item 如果$S$当前位置数字的出现次数小于0,则表示其在$G$中出现过,increments cows,同时increments对应的出现次数,
\item 如果$G$当前位置数字的出现次数大于0,则表示其在$S$中出现过,同样increments cows,但是要decrements对应的出现次数。
\end{itemize}
\setcounter{algorithm}{0}
\begin{algorithm}[H]
\caption{Counts}
\begin{algorithmic}[1]
\Procedure{GetHint}{$S, G, L$}
\State $\star$ 创建一个array $C$,长度为10,用于记录$S$和$G$中数字的出现次数
\State $\star$ $C$中所有元素初始化零,而$G$中数字的出现次数用负数表示
\State $A:=0$ \Comment Number of bulls
\State $B:=0$ \Comment Number of cows
\For{$i:=0$ \textbf{to} $L$}
\If{$S[i] = G[i]$}
\State $A\gets A+1$ \Comment Number and position are matched --- bull
\Else
\If{$C[S[i]] < 0$} 
\State $\ast$ $S[i]$在$G$中出现过,而且出现次数大于到目前为在$S$的出现次数
\State $B\gets B+1$
\EndIf
\State $C[S[i]]\gets C[S[i]] +1$ \Comment Increments $S[i]$的出现次数
\If{$C[G[i]] > 0$} 
\State $\ast$ $G[i]$在$S$中出现过,而且出现次数大于到目前为在$G$的出现次数
\State $B\gets B+1$
\EndIf
\State $C[G[i]]\gets C[G[i]] -1$ \Comment Increments $G[i]$的出现次数(因为是负数,所以减一)
\EndIf
\EndFor
\State \Return $A$的count和$B$的count组成的字符串
\EndProcedure
\end{algorithmic}
\end{algorithm}
\setcounter{lstlisting}{0}
\begin{lstlisting}[style=customc, caption={Counts}]
string getHint( string secret, string guess )
{
    vector<int> count( 10, 0 );

    int A = 0;
    int B = 0;

    for( size_t i = 0; i < secret.size(); ++i )
    {
        if( secret[i] == guess[i] )
        {
            ++A;
        }
        else
        {
            if( count[secret[i] - '0'] < 0 )
            {
                //secret[i] appeared in guess
                ++B;
            }

            //increments the count of secret[i]
            ++count[secret[i] - '0'];

            if( count[guess[i] - '0'] > 0 )
            {
                //guess[i] appeared in secret
                ++B;
            }
            //decrements the count of guess[i]
            --count[guess[i] - '0'];
        }
    }

    string ans = to_string( A );
    ans.push_back( 'A' );
    ans += to_string( B );
    ans.push_back( 'B' );

    return ans;
}
\end{lstlisting}
\section{Two Pass}
\begin{itemize}
\item 算法本质和上述方法是一样的
\item 这里需要两个array,分别记录$S$和$G$中的数字出现的次数。
\item 同样的 iterate over secret $S$ 和 guess $G$,很快得到所有bulls的(即数字和出现位置都相等)。
\item 否则的话,分别increment $S$和$G$在当前位置的数字的次数。
\item 最后,同时遍历$S$和$G$(因为两个长度都是10),然后取其中的较小值加到$B$中。
\end{itemize}
\begin{algorithm}[H]
\caption{Two Pass}
\begin{algorithmic}[1]
\Procedure{GetHint}{$S, G, L$}
\State $\star$ 创建两个数组$x$和$y$,长度都是10
\State $\star$ $x$用于记录$S$中的number的出现次数,$y$用于记录$G$中的number的出现次数
\State $A:=0$ \Comment Number of bulls
\State $B:=0$ \Comment Number of cows
\For{$i:=0$ \textbf{to} $L-1$}
\If{$S[i]=G[i]$}
\State $A\gets A+1$ \Comment Number and position are matched --- bull
\Else
\State $x[S[i]]\gets x[S[i]] +1$ \Comment Increments count of $S[i]$
\State $y[G[i]]\gets y[G[i]] +1$ \Comment Increments count of $G[i]$
\EndIf
\EndFor
\For{$k:=0$ \textbf{to} $9$}
\State $B\gets B+\min(x[k], y[k])$ \Comment The minimum of $S[k]$ and $G[k]$ is B
\EndFor
\State \Return $A$的count和$B$的count组成的字符串
\EndProcedure
\end{algorithmic}
\end{algorithm}
\begin{lstlisting}[style=customc, caption={Two Pass}]
string getHint( string secret, string guess )
{
    vector<int> s_count( 10, 0 );
    vector<int> g_count( 10, 0 );
    int A = 0;
    int B = 0;
    for( size_t i = 0; i < secret.size(); ++i )
    {
        if( secret[i] == guess[i] )
        {
            ++A;
        }
        else
        {
            ++s_count[secret[i] - '0'];
            ++g_count[guess[i] - '0'];
        }
    }

    for( int i = 0; i < 10; ++i )
    {
        B += ( min )( s_count[i], g_count[i] );
    }

    string ans = to_string( A );
    ans.push_back( 'A' );
    ans += to_string( B );
    ans.push_back( 'B' );

    return ans;
}
\end{lstlisting}