\section{284 --- Peeking Iterator}
Given an Iterator class interface with methods: \fcj{next()} and \fcj{hasNext()}, design and implement a \textit{PeekingIterator} that support the \fcj{peek()} operation -- it essentially \fcj{peek()} at the element that will be returned by the next call to \fcj{next()}.

\paragraph{Example:}

\begin{flushleft}
Assume that the iterator is initialized to the beginning of the list: \fcj{[1,2,3]}.

Call \fcj{next()} gets you 1, the first element in the list.

Now you call \fcj{peek()} and it returns 2, the next element. Calling next() after that \textbf{still} return 2. 

You call \fcj{next()} the final time and it returns 3, the last element. 

Calling \fcj{hasNext()} after that should return false.
\end{flushleft}

\paragraph{Follow up:} 
\begin{itemize}
\item How would you extend your design to be generic and work with all types, not just integer?
\end{itemize}

\subsection{Flag}
We can use a flag to indicate if next value is saved or not.

\setcounter{lstlisting}{0}
\begin{lstlisting}[style=customc,caption={Flag}]
// Below is the interface for Iterator, which is already defined for you.
// **DO NOT** modify the interface for Iterator.
class Iterator
{
    struct Data;
    Data* data;
public:
    Iterator( const vector<int>& nums );
    Iterator( const Iterator& iter );
    virtual ~Iterator();
    // Returns the next element in the iteration.
    int next();
    // Returns true if the iteration has more elements.
    bool hasNext() const;
};
class PeekingIterator : public Iterator
{
public:
    PeekingIterator( const vector<int>& nums ) : Iterator( nums )
    {
        // Initialize any member here.
        // **DO NOT** save a copy of nums and manipulate it directly.
        // You should only use the Iterator interface methods.
        flag_peek = false;
    }
    // Returns the next element in the iteration without advancing the iterator.
    int peek()
    {
        if( !flag_peek )
        {
            //if we have not called peek before
            //get next value
            next_val = Iterator::next();
            //set flag of peek
            flag_peek = true;
        }
        return next_val;
    }
    // hasNext() and next() should behave the same as in the Iterator interface.
    // Override them if needed.
    int next()
    {
        if( !flag_peek )
        {
            //if peek is not used
            //we can directly return next value
            return Iterator::next();
        }
        //clear peek flag
        flag_peek = false;
        return next_val;
    }
    bool hasNext() const
    {
        //either we have called peek
        //and get the next value
        //or use hasNext() of base class
        return flag_peek || Iterator::hasNext();
    }
    bool flag_peek;
    int next_val;
};
\end{lstlisting}

\subsection{Copy Constructor}
Since base class \fcj{Iterator} has copy constructor, we can directly build a new iterator object in peek and call its \fcj{next()} function.


\begin{lstlisting}[style=customc, caption={Copy Constructor}]
// Below is the interface for Iterator, which is already defined for you.
// **DO NOT** modify the interface for Iterator.
class Iterator
{
    struct Data;
    Data* data;
public:
    Iterator( const vector<int>& nums );
    Iterator( const Iterator& iter );
    virtual ~Iterator();
    // Returns the next element in the iteration.
    int next();
    // Returns true if the iteration has more elements.
    bool hasNext() const;
};
class PeekingIterator : public Iterator
{
public:
    PeekingIterator( const vector<int>& nums ) : Iterator( nums )
    {
        // Initialize any member here.
        // **DO NOT** save a copy of nums and manipulate it directly.
        // You should only use the Iterator interface methods.
    }
    // Returns the next element in the iteration without advancing the iterator.
    int peek()
    {
        Iterator cc( *this );
        return cc.next();
    }
    // hasNext() and next() should behave the same as in the Iterator interface.
    // Override them if needed.
    int next()
    {
        return Iterator::next();
    }
    bool hasNext() const
    {
        return Iterator::hasNext();
    }
};
\end{lstlisting}

\paragraph{Related Problems}
\begin{itemize}
\item \textbf{173. Binary Search Tree Iterator}
\item \textbf{251. Flatten 2D Vector}
\item \textbf{281. Zigzag Iterator}
\end{itemize}