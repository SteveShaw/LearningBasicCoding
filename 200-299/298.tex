\section{298 --- Binary Tree Longest Consecutive Sequence}
Given a binary tree, find the length of the longest consecutive sequence path.
\par
The path refers to any sequence of nodes from some starting node to any node in the tree along the parent-child connections. The longest consecutive path need to be from parent to child (cannot be the reverse).

\paragraph{Example:}

\begin{flushleft}

\begin{figure}[H]
\begin{tikzpicture}
[mynode/.style={draw,circle,minimum size=5mm, fill=gray!20!}]
\node(){};
\node[mynode](a) {1};
\node[mynode](b)[below=8mm of a, xshift=8mm] {3};
\node[mynode](c)[below=8mm of b, xshift=-8mm] {2};
\node[mynode](d)[below=8mm of b, xshift=8mm] {4};
\node[mynode](e)[below=8mm of d, xshift=8mm] {5};
\draw (a) -- (b);
\draw (b) -- (c);
\draw (b) -- (d);
\draw (d) -- (e);
\end{tikzpicture}
\end{figure}

Longest consecutive sequence path is 3--4--5, so return 3.

\begin{figure}[H]
\begin{tikzpicture}
[mynode/.style={draw,circle,minimum size=5mm, fill=gray!20!}]
\node(){};
\node[mynode](a) {2};
\node[mynode](b)[below=8mm of a, xshift=8mm] {3};
\node[mynode](c)[below=8mm of b, xshift=-8mm] {2};
\node[mynode](d)[below=8mm of c, xshift=-8mm] {1};
\draw (a) -- (b);
\draw (b) -- (c);
\draw (c) -- (d);
\end{tikzpicture}
\end{figure}

Longest consecutive sequence path is 2--3,not 3--2--1, so return 2.
\end{flushleft}
\subsection{Depth First Search}
The bottom-up approach is similar to a post-order traversal. We return the consecutive path length starting at current node to its parent. Then its parent can examine if its node value can be included in this consecutive path.

\setcounter{lstlisting}{0}
\begin{lstlisting}[style=customc, caption={DFS}]
int longestConsecutive( TreeNode* root )
{
    if( !root )
    {
        return 0;
    }
    int ans = 0;
    dfs( root, 1, ans );
    return ans;
}

void dfs( TreeNode *t, int len, int &ans )
{
    ans = max( ans, len );
    if( t->left )
    {
        if( t->left->val == t->val + 1 )
        {
			//len<-len+1
            dfs( t->left, len + 1, ans );
        }
        else
        {
			//reset len to 1
            dfs( t->left, 1, ans );
        }
    }

    if( t->right )
    {
        if( t->right->val == t->val + 1 )
        {
            //len<-len+1
            dfs( t->right, len + 1, ans );
        }
        else
        {
            //reset len to 1
            dfs( t->right, 1, ans );
        }
    }
}

\end{lstlisting}
\subsection{Top Down Depth-first Search}
A top down approach is similar to an in-order traversal. We use a variable \fcj{length} to store the current consecutive path length and pass it down the tree. As we traverse, we compare the current node with its parent node to determine if it is consecutive. If not, we reset the \fcj{length} to 1.
\begin{lstlisting}[style=customc, caption={DFS 2}]
int longestConsecutive( TreeNode* root )
{
    return dfs( root, nullptr, 0 );
}

int dfs( TreeNode *t, TreeNode *parent, int len )
{
    if( !t )
    {
        return len;
    }

    if( parent && ( t->val == parent->val + 1 ) )
    {
        ++len;
    }
    else
    {
        len = 1;
    }

    int l_len = dfs( t->left, t, len );
    int r_len = dfs( t->right, t, len );

    return ( max )( len, ( max )( l_len, r_len ) );
}
\end{lstlisting}