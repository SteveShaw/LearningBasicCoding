\section{228 --- Summary Ranges}
Given a sorted integer array $A$ without duplicates, return the summary of its ranges.

\paragraph{Example 1:}

\begin{flushleft}
\textbf{Input}:  $[0,1,2,4,5,7]$
\\
\textbf{Output}: $[0\to2, 4\to5, 7]$
\\
\textbf{Explanation}: $0,1,2$ form a continuous range; $4,5$ form a continuous range.
\end{flushleft}

\paragraph{Example 2:}

\begin{flushleft}
\textbf{Input}:  $[0,2,3,4,6,8,9]$
\\
\textbf{Output}: $[0,2\to4,6,8\to9]$
\\
\textbf{Explanation}: $2,3,4$ form a continuous range; $8,9$ form a continuous range.
\end{flushleft}
\subsection{Plain Algorithm}

\begin{itemize}
\item 遍历数组,然后寻找连续数的segment。
\item 需要小心处理每一个分割点。
\item 需要将\texttt{int} type升级成\texttt{long long} type以处理数据精度overflow。
\end{itemize}
\setcounter{lstlisting}{0}
\begin{lstlisting}[style=customc, caption={Plain Algorithm}]
vector<string> summaryRanges( vector<int>& nums )
{
    if( nums.empty() )
    {
        return {};
    }

    vector<string> ans;

    // Need to check if the back of nums has been
    // added into a range
    int last_end = nums.back() + 1;

    long long a = 0;
    long long b = 0;

    for( size_t i = 0; i < nums.size() - 1; ++i )
    {
        a = static_cast<long long>( nums[i + 1] );
        b = static_cast<long long>( nums[i] );
        if( a - b == 1 )
        {
            size_t j = i + 1;

            while( j < nums.size() - 1 )
            {
                a = static_cast<long long>( nums[j + 1] );
                b = static_cast<long long>( nums[j] );

                if( a - b != 1 )
                {
                    break;
                }
                ++j;
            }

            string s = to_string( nums[i] );
            s += "->";
            s += to_string( nums[j] );

            ans.emplace_back( s );
            i = j;

            //update range end
            last_end = nums[j];
        }
        else
        {
            ans.emplace_back( to_string( nums[i] ) );
            //update range end
            last_end = nums[i];
        }
    }

    if( nums.back() != last_end )
    {
        ans.emplace_back( to_string( nums.back() ) );
    }
    return ans;
}
\end{lstlisting}