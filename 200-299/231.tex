\section{231 --- Power of Two}
Given an integer $n$, write a function to determine if it is a power of two.

\paragraph{Example 1:}

\begin{flushleft}
\textbf{Input}: 1
\\
\textbf{Output}: \texttt{true} 
\\
\textbf{Explanation}: $2^0 = 1$
\end{flushleft}


\paragraph{Example 2:}

\begin{flushleft}
\textbf{Input}: 16
\\
\textbf{Output}: \texttt{true}
\\
\textbf{Explanation}: $2^4 = 16$
\end{flushleft}

\paragraph{Example 3:}
\begin{flushleft}
\textbf{Input}: 218
\\
\textbf{Output}: \texttt{false}
\end{flushleft}
\subsection{Bit Operation}
\begin{CJK*}{UTF8}{gbsn}
\begin{itemize}
\item 利用$n$\&$(n-1)$能够消除rightmost 1-bit的特性。如果是2的power,那么$n$\&$(n-1)$就为零。
\item 需要考虑当$n$是int type所能代表的最大和最小值,以及$n=0$。这三种情况都返回\texttt{false}。
\end{itemize}
\end{CJK*}
\setcounter{lstlisting}{0}
\begin{lstlisting}[style=customc, caption={Bit Operation}]
bool isPowerOfTwo( int n )
{
    if( n == INT_MAX || n == INT_MIN || n == 0 )
    {
        return false;
    }

    int x = ( ( n - 1 ) & n );

    return x == 0;
}
\end{lstlisting}