\section{213 --- House Robber II}
You are a professional robber planning to rob houses along a street. Each house has a certain amount of money stashed. All houses at this place are \textbf{arranged in a circle}. That means the first house is the neighbor of the last one. Meanwhile, adjacent houses have security system connected and it will automatically contact the police \textbf{if two adjacent houses were broken into on the same night}.
\par
Given a list of non-negative integers $A$ representing the amount of money of each house, determine the maximum amount of money you can rob tonight without alerting the police.
\paragraph{Example 1:}
\begin{flushleft}
\textbf{Input}: $[2,3,2]$
\\
\textbf{Output}: 3
\\
\textbf{Explanation}: 
\\
You cannot rob house 1 (money = 2) and then rob house 3 (money = 2), because they are adjacent houses.
\end{flushleft}
\textbf{Example 2:}
\begin{flushleft}
\textbf{Input}: $[1,2,3,1]$
\\
\textbf{Output}: 4
\\
\textbf{Explanation}: 
\\
Rob house 1 (money = 1) and then rob house 3 (money = 3). Total amount you can rob $= 1 + 3 = 4$.
\end{flushleft}
\subsection{Dynamic Programming}
\begin{CJK*}{UTF8}{gbsn}
和199相类似的题,但是由于$A[0]$和$A[L-1]$是相邻的($L$是$A$的长度),因此实际上问题其实就是求出下列两种情况的最大值
\begin{itemize}
\item 从$A[0]$到$A[L-2]$得到的最大值
\item 从$A[1]$到$A[L-1]$得到的最大值
\end{itemize}
\end{CJK*}
\setcounter{algorithm}{0}
\begin{algorithm}[H]
\caption{Dynamic Programming}
\begin{algorithmic}[1]
\Procedure{Rob}{$A, L$}
\If{$L = 0$} \Comment Empty array
\State \Return 0
\EndIf
\State $\star$ Get the maximum amount from $A[0]$ to $A[L-2]$
\State $x:=A[0]$ \Comment The maximum amount if $A[i]$ is selected
\State $y:=0$ \Comment The maximum amount if $A[i]$ is not selected
\For{$i:=0\to L-2$}
\State $x_0:=x$
\State $y_0:=y$
\State $x\gets \max(x_0, y_0+A[i])$ \Comment $A[i]$ is selected, so $A[i-1]$ cannot be selected
\State $y\gets \max(x_0, y_0)$ \Comment $A[i]$ is not selected, so $A[i-1]$ can be selected or not
\EndFor
\State $z_0:=\max(x,y)$ \Comment The maximum amount from $A[0]$ to $A[L-2]$
\If{$L=1$} \Comment Only one element in $A$
\State \Return $z_0$
\EndIf
\State $\star$ Get the maximum amount from $A[1]$ to $A[L-1]$
\State $x:=A[1]$ \Comment The maximum amount if $A[i]$ is selected
\State $y:=0$ \Comment The maximum amount if $A[i]$ is not selected
\For{$i:=1\to L-1$}
\State $x_0:=x$
\State $y_0:=y$
\State $x\gets \max(x_0, y_0+A[i])$ \Comment $A[i]$ is selected, so $A[i-1]$ cannot be selected
\State $y\gets \max(x_0, y_0)$ \Comment $A[i]$ is not selected, so $A[i-1]$ can be selected or not
\EndFor
\State $z_1:=\max(x,y)$ \Comment The maximum amount from $A[1]$ to $A[L-1]$
\State \Return $\max(z_0, z_1)$
\EndProcedure
\end{algorithmic}
\end{algorithm}
\setcounter{lstlisting}{0}
\begin{lstlisting}[style=customc, caption={Dynamic Programming}]
int rob( vector<int>& nums )
{
    if( nums.empty() )
    {
        return 0;
    }

	// Process from A[0] to A[L-2]
    int x = nums[0];
    int y = 0;

    int L = static_cast<int>( nums.size() );

    for( int i = 1; i < L - 1; ++i )
    {
        int last_x = x;
        int last_y = y;

        x = ( max )( last_x, last_y + nums[i] );
        y = ( max )( last_x, last_y );
    }

    int ans = ( max )( x, y );

	// Process from A[1] to A[L-1]
    if( nums.size() < 2 )
    {
        return ans;
    }

    x = nums[1];
    y = 0;

    for( int i = 2; i < L; ++i )
    {
        int last_x = x;
        int last_y = y;

        x = ( max )( last_x, last_y + nums[i] );
        y = ( max )( last_x, last_y );
    }

    int t = ( max )( x, y );

    ans = ( max )( ans, t );

    return ans;
}
\end{lstlisting}
