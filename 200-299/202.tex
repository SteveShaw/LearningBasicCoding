\section{202 --- Happy Number}
Write an algorithm to determine if a number is \textit{happy}.
\par
A happy number is a number defined by the following process: 
\par
Starting with any positive integer, replace the number by the sum of the squares of its digits, and repeat the process until the number equals 1 (where it will stay), or it loops endlessly in a cycle which does not include 1. Those numbers for which this process ends in 1 are \textit{happy numbers}.
\paragraph{Example: }
\begin{flushleft}

\textbf{Input}: 19
\\
\textbf{Output}: \fcj{true}
\\
\textbf{Explanation}: 
\begin{align*}
1^2 + 9^2 &= 82\\
8^2 + 2^2 &= 68\\
6^2 + 8^2 &= 100\\
1^2 + 0^2 + 0^2 &= 1\\
\end{align*}
\end{flushleft}
\subsection{Hash Set}
可以用hash set来记录所有出现过的数字,然后每出现一个新数字,在set中查找看是否存在
\setcounter{lstlisting}{0}
\begin{lstlisting}[style=customc, caption={Hash Set}]
bool isHappy( int n )
{
    unordered_set<int> s;
    s.insert( n );
    while( true )
    {
        int m = n;
        int sum = 0;
        while( m )
        {
            int q = m / 10;
            int r = m - 10 * q;
            sum += ( r * r );
            m = q;
        }
        if( sum == 1 )
        {
            return true;
        }
        if( s.find( sum ) != s.end() )
        {
            return false;
        }
        s.insert( sum );
        n = sum;
    }
    return false;
}

\end{lstlisting}
\subsection{Mathematical Induction}
Non happy number有个特点,循环的数字中必定会有4
\begin{lstlisting}[style=customc, caption={Mathematical Induction}]
bool isHappy( int n )
{
    while( n != 1 && n != 4 )
    {
        int t = 0;
        while( n )
        {
            t += ( n % 10 ) * ( n % 10 );
            n /= 10;
        }
        n = t;
    }
    return n == 1;
}
\end{lstlisting}