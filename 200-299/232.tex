\section{232 --- Implement Queue using Stacks}
Implement the following operations of a queue using stacks.
\begin{flushleft}
\begin{lstlisting}[style=customc]
push(x) //Push element x to the back of queue.
pop() //Removes the element from in front of queue.
peek() //Get the front element.
empty() //Return whether the queue is empty.
\end{lstlisting}
\end{flushleft}

\paragraph{Example:}

\begin{flushleft}
\begin{lstlisting}[style=customc]
MyQueue queue = new MyQueue();

queue.push(1);
queue.push(2);  
queue.peek();  // returns 1
queue.pop();   // returns 1
queue.empty(); // returns false
\end{lstlisting}
\end{flushleft}

\paragraph{Notes:}

\begin{itemize}
\item You must use only standard operations of a stack -- which means only \texttt{push} to \texttt{top}, \texttt{peek}/\texttt{pop} from top, \texttt{size}, and \texttt{is empty} operations are valid.
\item Depending on your language, stack may not be supported natively. You may simulate a stack by using a list or deque (double-ended queue), as long as you use only standard operations of a stack.
\item You may assume that all operations are valid (for example, no \texttt{pop} or \texttt{peek} operations will be called on an empty queue).
\end{itemize}
\subsection{Two Stacks}

\begin{itemize}
\item 因为是queue,所以新加入的number需要push到bottom of the stack.
\begin{enumerate}
\item 首先把$S_1$中的所有number都转移到辅助栈$S_2$中
\item 然后把新加入的number push到$S_2$中,这时候这个新的number就在$S_2$中的top上。
\item 再将$S_2$中的所有元素弹出,push到$S_1$。
\end{enumerate}
\item 由于上述push的操作,$S_1$总是把第一个插入的number放在top。因此直接从$S_1$弹出top即可。
\item 这种方法中,$S_2$只是用于reverse input numbers的顺序,因此除了push函数中会暂时有numbers进入,其余任何时候都是empty。
\end{itemize}

\setcounter{lstlisting}{0}
\begin{lstlisting}[style=customc,caption={Two Stacks}]
class MyQueue
{
public:
    /** Initialize your data structure here. */
    MyQueue()
    {
    }

    /** Push element x to the back of queue. */
    void push( int x )
    {
        if( stk1.empty() )
        {
            stk1.push( x );
            return;
        }

        //Pop all numbers in stk1 and push to stk2
        while( !stk1.empty() )
        {
            auto t = stk1.top();
            stk1.pop();

            stk2.push( t );
        }

        stk2.push( x );

        while( !stk2.empty() )
        {
            auto t = stk2.top();
            stk2.pop();

            stk1.push( t );
        }
    }

    /** Removes the element from in front of queue and returns that element. */
    int pop()
    {
        int t = stk1.top();
        stk1.pop();

        return t;
    }

    /** Get the front element. */
    int peek()
    {
        return stk1.top();
    }

    /** Returns whether the queue is empty. */
    bool empty()
    {
        return stk1.empty();
    }

    stack<int> stk1;
    stack<int> stk2;

    int front;
};
\end{lstlisting}
\subsection{Two Stacks With Amortized O(1)}

\begin{itemize}
\item 新加入的number $x$总是放在$S_1$的top
\item 对于pop
\begin{enumerate}
\item 由于需要remove的number是位于$S_1$的底部,因此需要 pop all elements from $S_1$, 然后 push to $S_2$. 通过这种操作,$S_1$底部的number就成为$S_2$的top。
\item 只有当$S_2$为empty时,才进行上述操作,否则直接从$S_2$ pop。这就意味着在某些时刻,$S_1$和$S_2$同时都有多于1个的number。
\end{enumerate}
\item 这种方法中$S_2$和$S_1$有可能同时不为empty,因此取出queue的front元素就不能用方法1中直接从$S_1$中获得了,需要用一个辅助变量记录第一个插入的number。这样当$S_2$不为空时,表示此时还没有进行push操作,或者已经pop掉所有以前的numbers。
\end{itemize}
\setcounter{lstlisting}{0}
\begin{lstlisting}[style=customc, caption={Two Stack With Amortized O(1)}]
class MyQueue
{
public:
    /** Initialize your data structure here. */
    MyQueue()
    {
    }

    /** Push element x to the back of queue. */
    void push( int x )
    {
        //front is the first element of the queue
        //Need to set when the first time to call push()
        if( S1.empty() )
        {
            front = x;
        }

        S1.push( x );
    }

    /** Removes the element from in front of queue and returns that element. */
    int pop()
    {
        if( S2.empty() )
        {
            while( !S1.empty() )
            {
                auto t = S1.top();
                S1.pop();
                S2.push( t );
            }
        }

        auto t = S2.top();
        S2.pop();

        return t;
    }

    /** Get the front element. */
    int peek()
    {
        //Since S2 may contain input-reversed numbers
        //Need to check if S2 is empty or not
        //If it is not empty, the front element is S2's top
        //Otherwise, means push() has not been called
        //Or Pop() has been called to pop all numbers
        //Therefore, front variable is the current front
        if( !S2.empty() )
        {
            return S2.top();
        }
        else
        {
            return front;
        }
    }

    /** Returns whether the queue is empty. */
    bool empty()
    {
        return ( S1.empty() ) && ( S2.empty() );
    }

    stack<int> S1;
    stack<int> S2;

    int front;
};
\end{lstlisting}
