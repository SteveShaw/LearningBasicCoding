\section{268. Missing Number}
Given an array $A$ containing $n$ distinct numbers taken from $0, 1, 2, \ldots, n$, find the one that is missing from the array.

\paragraph{Example 1:}

\begin{flushleft}
\textbf{Input}: $[3,0,1]$
\\
\textbf{Output}: 2
\end{flushleft}

\paragraph{Example 2:}

\begin{flushleft}
\textbf{Input}: $[9,6,4,2,3,5,7,0,1]$
\\
\textbf{Output}: 8
\end{flushleft}

\paragraph{Note:}
\begin{itemize}
\item Your algorithm should run in linear runtime complexity. \item Could you implement it using only constant extra space complexity?
\end{itemize}
\subsection{Mathematical Induction}
先求出整个数组的和,然后用$(L+1)\times L/2$得到$0,1,\ldots,L$的和,这两个之差即为missing number
\subsection{Bit Manipulation}
由于$A$中只包含了$n$个数,因此index的范围是从$0$到$n-1$,因此将每个index和其对应的数进行XOR,然后再计算整体的XOR,由于重复的数的XOR结果为零,那个missing number就是最终的XOR结果。
\setcounter{lstlisting}{0}
\begin{lstlisting}[style=customc, caption={Bit Manipulation}]
int missingNumber( vector<int>& nums )
{
    int ans = 0;
    for( int i = 0; i < nums.size(); ++i )
    {
        ans ^= ( i + 1 ) ^ nums[i];
    }
    return ans;
}
\end{lstlisting}