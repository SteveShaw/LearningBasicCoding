\section{281 --- Zigzag Iterator}
Given two 1d vectors $A$ and $B$, implement an iterator to return their elements alternately.
\par
For example, given two 1d vectors:
$A = [1, 2]$ and $B = [3, 4, 5, 6]$
\par
By calling \fcj{next} repeatedly until \fcj{hasNext} returns \fcc{false}, the order of elements returned by next should be: $[1, 3, 2, 4, 5, 6]$.
\paragraph{Follow up:}
\begin{itemize}
    \item What if you are given $k$ 1d vectors? How well can your code be extended to such cases?
\end{itemize}

\paragraph{Clarification:}
\begin{itemize}
    \item The \fcj{Zigzag} order is not clearly defined and is ambiguous for $k > 2$ cases. If \fcj{Zigzag} does not look right to you, replace \fcj{Zigzag} with \fcj{Cyclic}. For example, given the following input:
\begin{table}[H]
    \begin{tabular}{l}
        $[1,2,3]$   \\
        $[4,5,6,7]$ \\
        $[8,9]$
    \end{tabular}
\end{table}
It should return $[1,4,8,2,5,9,3,6,7]$.
\end{itemize}
\subsection{Queue}
The efficient way is to use a queue with iterator. 

In the beginning, we add each \fcc{vector}'s \fcc{begin} and \fcc{end} as a pair into the queue. In the function \fcc{next}, we get the front of the queue and extract the value, then check if the next of current iterator is at the end of current vector. If it is not, we will push the iterator and its related \fcc{end} into the queue.

\setcounter{lstlisting}{0}
\begin{lstlisting}[style=customc, caption={Queue}]
class ZigzagIterator
{
public:
    ZigzagIterator( vector<int>& v1, vector<int>& v2 )
    {
        if( !v1.empty() )
        {
            q.emplace( v1.begin(), v1.end() );
        }

        if( !v2.empty() )
        {
            q.emplace( v2.begin(), v2.end() );
        }
    }
    int next()
    {
        auto t = q.front();
        q.pop();
        int x = *( t.first );
        ++t.first;
        if( t.first != t.second )
        {
            //if current vector has not completely
            //iterated, add next one
            q.emplace( t.first, t.second );
        }
        return x;
    }
    bool hasNext()
    {
        return !q.empty();
    }

    using IterT = vector<int>::iterator;
    queue<pair<IterT, IterT>> q;
};
/**
 * ZigzagIterator i(v1, v2);
 * while (i.hasNext()) cout << i.next();
 */
\end{lstlisting}

\paragraph{Related Problems}
\begin{itemize}
\item \textbf{173. Binary Search Tree Iterator}
\item \textbf{251. Flatten 2D Vector}
\item \textbf{284. Peeking Iterator}
\item \textbf{341. Flatten Nested List Iterator}
\end{itemize}