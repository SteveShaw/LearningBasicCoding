\section{210 --- Course Schedule II}
There are a total of $n$ courses you have to take, labeled from 0 to $n-1$.

Some courses may have prerequisites, for example to take course 0 you have to first take course 1, which is expressed as a pair: $[0,1]$

Given the total number of courses $n$ and a list of prerequisite \textbf{pairs} $P$, return the ordering of courses you should take to finish all courses.

There may be multiple correct orders, you just need to return one of them. If it is impossible to finish all courses, return an empty array.

\paragraph{Example 1:}
\begin{flushleft}
\textbf{Input}: $n=2$, $P=[[1,0]]$
\\
\textbf{Output}: $[0,1]$
\\
\textbf{Explanation}: There are a total of 2 courses to take. To take course 1 you should have finished  course 0. So the correct course order is $[0,1]$.
\end{flushleft}
\paragraph{Example 2:}
\begin{flushleft}
\textbf{Input}: $n=4$, $P=[[1,0],[2,0],[3,1],[3,2]]$
\\
\textbf{Output}: $[0,1,2,3]$ or $[0,2,1,3]$
\\
\textbf{Explanation}: 
\\
There are a total of 4 courses to take. To take course 3 you should have finished both courses 1 and 2. Both courses 1 and 2 should be taken after you finished course 0. So one correct course order is [0,1,2,3]. Another correct ordering is [0,2,1,3] .
\end{flushleft}
\paragraph{Note:}
\begin{itemize}
    \item The input prerequisites is a graph represented by a list of edges, not adjacency matrices. Read more about how a graph is represented.
    \item You may assume that there are no duplicate edges in the input prerequisites.
\end{itemize}
\subsection{Analysis}
This problem is similar to problem 207, so BFS and DFS are two approaches.
\subsection{DFS}
The basic idea is similar, but with little modification. 
\begin{itemize}
    \item Add node to path when it survives the DFS process
    \item Since those nodes which depend on other nodes are added first into the path, reverse the final path to get the correct order.
\end{itemize}
\setcounter{lstlisting}{0}
\begin{lstlisting}[style=customc, caption={DFS}]
vector<int> findOrder(int numCourses, vector<pair<int, int>>& prerequisites) 
{

    vector<vector<int>> g(numCourses);

    //build the graphic data structure
    for(const auto& p: prerequisites)
    {
        g[p.second].push_back(p.first);

    }

    vector<int> color(numCourses, 0);

    vector<int> ans;

    for(int i = 0; i < numCourses; ++i)
    {

        if(color[i] == 2)
        {
            continue;
        }

        if(!dfs(g, i, color, ans))
        {
            return {};
        }
    }



    reverse(ans.begin(), ans.end());
    return ans;
}

bool dfs(vector<vector<int>>& g, int start, vector<int>& color, vector<int>& ans)
{
    if(color[start] == 2)
    {
        return true;
    }

    if(color[start] == 1)
    {
        return false;
    }

    color[start] = 1;

    for(int next: g[start])
    {
        if(!dfs(g, next, color, ans))
        {
            return false;
        }
    }

    color[start] = 2;

    ans.push_back(start);

    return true;
}
\end{lstlisting}
\subsection{Topological Sort}
The basic idea is similar
\begin{itemize}
    \item Get in-degree of each node
    \item Put those nodes with zero in-degree into the queue
    \item When get a node from the queue, add to the path, then decrements in-degree of each connected node. If a connected node has zero in-degree, push it into the queue.
    \item Check if the number of visited nodes is equal to the total number of nodes. If they are not equal, there exists a cycle in the graph. Hence, the path does not exist.
\end{itemize}
\begin{lstlisting}[style=customc, caption={BFS}]
vector<int> findOrder( int numCourses, vector<pair<int, int>>& prerequisites )
{
    vector<int> v_ins( numCourses, 0 );
    vector<vector<int>> g( numCourses );
    for( const auto& p : prerequisites )
    {
        v_ins[p.first]++;
        g[p.second].push_back( p.first );
    }
    queue<int> q;
    vector<int> ans;
    ans.reserve( numCourses );
    int num_visited = 0;
    for( int i = 0; i < numCourses; ++i )
    {
        if( v_ins[i] == 0 )
        {
            q.push( i );
        }
    }
    while( !q.empty() )
    {
        int t = q.front();
        q.pop();
        ans.push_back( t );
        ++num_visited;
        for( int next : g[t] )
        {
            if( v_ins[next] > 0 )
            {
                v_ins[next]--;

                if( v_ins[next] == 0 )
                {
                    q.push( next );
                }
            }
        }//end for
    }//end while

    if( num_visited == numCourses )
    {
        return ans;
    }
    return {};
}
\end{lstlisting}

\paragraph{Related Questions}
\begin{itemize}
\item \textbf{207. Course Schedule}
\item \textbf{269. Alien Dictionary}
\item \textbf{310. Minimum Height Trees}
\item \textbf{444. Sequence Reconstruction}
\item \textbf{630. Course Schedule III}
\end{itemize}
