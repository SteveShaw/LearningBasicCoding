\section{236 --- Lowest Common Ancestor of a Binary Tree}
Given a binary tree, find the lowest common ancestor (\texttt{LCA}) of two given nodes in the tree.
\par
According to the definition of \texttt{LCA} on \textbf{Wikipedia}: \textit{The lowest common ancestor is defined between two nodes $p$ and $q$ as the lowest node in $T$ that has both $p$ and $q$ as descendants (where we allow \textbf{a node to be a descendant of itself}).}

\paragraph{Example 1:}

\begin{flushleft}
\textbf{Input}: 
\begin{figure}[H]
\begin{tikzpicture}
[mynode/.style={draw,circle,minimum size=5mm, fill=gray!20!}]
\node(){};
\node[mynode](1) {3};
\node[mynode](2)[below=8mm of 1, xshift=-13mm] {5};
\node[mynode](3)[below=8mm of 1, xshift=13mm] {1};
\node[mynode](4)[below=8mm of 2, xshift=-7mm] {6};
\node[mynode](5)[below=8mm of 2, xshift=7mm] {2};
\node[mynode](6)[below=8mm of 3, xshift=-7mm] {0};
\node[mynode](7)[below=8mm of 3, xshift=7mm] {8};
\node[mynode](8)[below=8mm of 5, xshift=-7mm] {7};
\node[mynode](9)[below=8mm of 5, xshift=7mm] {4};
\draw (1) -- (2);
\draw (1) -- (3);
\draw (2) -- (4);
\draw (2) -- (5);
\draw (3) -- (6);
\draw (3) -- (7);
\draw (5) -- (8);
\draw (5) -- (9);
\end{tikzpicture}
\end{figure}
$p = 5$, $q = 1$
\\
\textbf{Output}: 6
\\
\textbf{Explanation}: 
\\
The LCA of nodes 5 and 1 is 3.
\end{flushleft}

\paragraph{Example 2:}

\begin{flushleft}
\textbf{Input}:
\begin{figure}[H]
\begin{tikzpicture}
[mynode/.style={draw,circle,minimum size=5mm, fill=gray!20!}]
\node(){};
\node[mynode](1) {3};
\node[mynode](2)[below=8mm of 1, xshift=-13mm] {5};
\node[mynode](3)[below=8mm of 1, xshift=13mm] {1};
\node[mynode](4)[below=8mm of 2, xshift=-7mm] {6};
\node[mynode](5)[below=8mm of 2, xshift=7mm] {2};
\node[mynode](6)[below=8mm of 3, xshift=-7mm] {0};
\node[mynode](7)[below=8mm of 3, xshift=7mm] {8};
\node[mynode](8)[below=8mm of 5, xshift=-7mm] {7};
\node[mynode](9)[below=8mm of 5, xshift=7mm] {4};
\draw (1) -- (2);
\draw (1) -- (3);
\draw (2) -- (4);
\draw (2) -- (5);
\draw (3) -- (6);
\draw (3) -- (7);
\draw (5) -- (8);
\draw (5) -- (9);
\end{tikzpicture}
\end{figure}
$p = 5$, $q = 4$
\\
\textbf{Output}: 5
\\
\textbf{Explanation}: 
\\
The LCA of nodes 5 and 4 is 5, since a node can be a descendant of itself according to the LCA definition.
\end{flushleft}

\paragraph{Note:}

\begin{itemize}
\item All of the nodes' values will be unique.
\item $p$ and $q$ are different and both values will exist in the binary tree.
\end{itemize}
\subsection{Recursion}
这道题和235不同,这里只是一般的bianry tree。
\begin{itemize}
\item 按照深度优先的方式遍历tree。
\item 当遇到$p$或者$q$时,返回一些boolean flags。这些flags会帮助确定是否在某个path上找到了$p$或者$q$。
\item \texttt{LCA}则是其对其左右子树进行递归处理都能够返回\texttt{true}的那个节点。
\item 由于\texttt{LCA}也可能是$p$或者$q$中的一个,这种情况下只会有一个子树返回\texttt{true}
\end{itemize}
算法过程如下
\begin{itemize}
\item 从root开始遍历tree。
\item 如果current node本身就是$p$或者是$q$,将一个变量$x$设为\texttt{true},然后继续在左右子树中寻找另外一个节点。
\item 如果遍历左子树的结果$l$或者遍历右子树$r$是\texttt{true},表明$p$和$q$中的一个被找到了。
\item 在遍历过程中,任何时候发现$l$, $r$和$x$这三个变量中有两个为\texttt{true}了,表明已经找到了$p$和$q$的\texttt{LCA}。
\item 具体实现时,用$l$, $r$和$x$都设为\texttt{int}。这样可以通过判断这三个的和来判定是否有两个为\texttt{true}。
\end{itemize}
\setcounter{lstlisting}{0}
\begin{lstlisting}[style=customc, caption={Recursion}]
TreeNode* LCA( TreeNode* root, TreeNode* p, TreeNode* q )
{
    TreeNode* ans;
    dfs( root, p, q, ans );

    return ans;
}

bool dfs( TreeNode* node, TreeNode* p, TreeNode* q, TreeNode*& lca )
{
    if( !node )
    {
        return false;
    }

    int x = 0;

    if( ( node == p ) || ( node == q ) )
    {
        //one of p and q is found
        x = 1;
    }

    int l = 0;
    if( dfs( node->left, p, q, lca ) )
    {
        //Found p and/or q in left tree
        l = 1;
    }

    int r = 0;
    if( dfs( node->right, p, q, lca ) )
    {
        //Found p and/or q in right tree
        r = 1;
    }

    if( l + r + x >= 2 )
    {
        //LCA is found
        lca = node;
    }

    return ( l + r + x ) > 0; //at least find one of p and q
}
\end{lstlisting}
\subsection{Another Recursion}
\begin{CJK*}{UTF8}{gbsn}
在binary tree中搜索$p$和$q$,然后从遍历路径中找到最后一个相同的节点即为\texttt{LCA}。边界情况很简单: 首先看当前结点是否为\texttt{null},若为\texttt{null}则直接返回\texttt{null},若为$p$或$q$中的任意一个,直接返回当前结点
\par
如果当前结点不等于$p$或$q$,$p$和$q$要么分别位于左右子树中,要么同时位于左子树,或者同时位于右子树
\begin{itemize}
\item 若$p$和$q$分别位于左右子树中,那么对左右子结点进行recursion,会分别返回$p$和$q$结点的位置,而当前结点正好就是$p$和$q$的最小共同父结点,直接返回当前结点即可,这就是题目中的例子1的情况。
\item 若$p$和$q$同时位于左子树,这时候调用递归函数处理right subtree会返回\texttt{null}, 而处理left subtree有两种情况,
\begin{enumerate}
\item 函数返回的是$p$和$q$中靠近根节点的那个节点,这时候返回这个节点。这就是题目中的例子2的情况。
\item 函数返回的是$p$和$q$的\texttt{LCA},就是说当前结点的左子树中的某个结点才是$p$和$q$的\texttt{LCA},也直接返回这个结果。
\end{enumerate}
\item 若$p$和$q$同时位于右子树,和上述左子树情形类似。
\end{itemize}
\end{CJK*}
\begin{lstlisting}[style=customc, caption={Recursion Based On Left And Right Subtree}]
TreeNode* LCA( TreeNode* root, TreeNode* p, TreeNode* q )
{
    if( !root || root == p || root == q )
    {
        return root;
    }

    //process left subtree
    auto l_node = LCA( root->left, p, q );
    //process right subtree
    auto r_node = LCA( root->right, p, q );

    //found the two nodes in left and right subtree respectively
    //LCA is found
    if( l_node && r_node )
    {
        return root;
    }

    //p and q only in left subtree
    if( l_node )
    {
        return l_node;
    }

    //p and q only in right subtree
    return r_node;
}
\end{lstlisting}
\subsection{Iterative using parent pointers}
\begin{CJK*}{UTF8}{gbsn}
这种方法则需要借助queue和hash map来寻找最接近$p$和$q$的common ancestor。
\par
算法步骤如下
\begin{enumerate}
\item 从根节点开始遍历tree。由于是iterative traverse,所以需要借助queue。
\item 将当前节点的parent node保存在一个hash map $M$中直到找到$p$和$q$。
\item 一旦找到$p$和$q$,将$p$的所有ancestors从$M$中加入到一个hash set $S$中。
\item 然后在$M$中遍历$q$所有的ancestors。如果当前遇到的ancestor可以在$S$中找到,这就是$p$和$q$的\texttt{LCA}。
\end{enumerate}
\end{CJK*}
\begin{lstlisting}[style=customc, caption={Iterative Searching}]
TreeNode* LCA( TreeNode* root, TreeNode* p, TreeNode* q )
{

    queue<TreeNode*> qt;
    unordered_map<TreeNode*, TreeNode*> m;

    qt.push( root );
    m.emplace( root, nullptr ); //root's parent is empty

    //iterative traverse the tree using the queue
    while( !qt.empty() )
    {
        auto it_p = m.find( p );
        auto it_q = m.find( q );

        if( ( it_p != m.end() ) && ( it_q != m.end() ) )
        {
            //p and q both are found
            break;
        }

        auto t = qt.front();
        qt.pop();

        if( t->left )
        {
            m.emplace( t->left, t );
            qt.emplace( t->left );
        }

        if( t->right )
        {
            m.emplace( t->right, t );
            qt.emplace( t->right );
        }
    }

    unordered_set<TreeNode*> A;

    // add all p's ancestors into the hash set A
    auto t = p;
    while( t )
    {
        A.insert( t );
        t = m[t];
    }

    // find the first ancestor of q that
    // can be found in A
    t = q;
    while( t )
    {
        auto it = ancestors.find( t );

        if( it != ancestors.end() )
        {
            return *it;
        }

        t = m[t];
    }

    // cannot find LCA
    return nullptr;
}
\end{lstlisting}
