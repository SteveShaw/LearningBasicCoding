\section{257 --- Binary Tree Paths}
Given a binary tree, return all root-to-leaf paths.
\par
\paragraph{Note:}
\begin{itemize}
\item  A leaf is a node with no children.
\end{itemize}

\paragraph{Example:}
\begin{flushleft}
\textbf{Input}:
\\
\begin{figure}[H]
\begin{tikzpicture}
[mynode/.style={draw,circle,minimum size=5mm, fill=gray!20!}]
\node(){};
\node[mynode](A) {1};
\node[mynode](B)[below=8mm of A, xshift=-8mm] {2};
\node[mynode](C)[below=8mm of A, xshift=8mm] {3};
\node[mynode](D)[below=8mm of B, xshift=6mm] {5};
\draw (A) -- (B);
\draw (A) -- (C);
\draw (B) -- (D);
\end{tikzpicture}
\end{figure}
\textbf{Output}: $[1\longrightarrow 2 \longrightarrow 5, 1\longrightarrow 3]$
\\
\textbf{Explanation}: All root-to-leaf paths are: $1\longrightarrow 2\longrightarrow 5$, $1\longrightarrow 3$
\end{flushleft}
\subsection{Depth First Search}
很简单的递归,当遇到leaf node时,就将生成的路径string加入到最终结果中即可。
\setcounter{lstlisting}{0}
\begin{lstlisting}[style=customc, caption={Depth First Search}]
vector<string> binaryTreePaths( TreeNode* root )
{
    vector<string> ans;
    dfs( root, "", ans );

    return ans;
}

void dfs( TreeNode* node, string cur_path, vector<string>& paths )
{
    if( !node )
    {
        return;
    }

    cur_path += to_string( node->val );
    cur_path += "->";

    if( !node->left && !node->right )
    {
        //remove "->"
        cur_path.pop_back();
        cur_path.pop_back();

        paths.emplace_back( cur_path );
        return;
    }

    dfs( node->left, cur_path, paths );
    dfs( node->right, cur_path, paths );
}
\end{lstlisting}