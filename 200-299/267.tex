\section{267 --- Palindrome Permutation II}
Given a string $s$, return all the palindromic permutations (without duplicates) of it. Return an empty list if no palindromic permutation could be form.
\paragraph{Example 1:}
\begin{flushleft}
\textbf{Input}: $s=\texttt{aabb}$
\\
\textbf{Output}: (\texttt{abba}, \texttt{baab})
\end{flushleft}

\paragraph{Example 2:}
\begin{flushleft}
\textbf{Input}: $s=\texttt{abc}$
\\
\textbf{Output}: $\emptyset$
\end{flushleft}

\paragraph{Hint:}
\begin{itemize}
\item If a palindromic permutation exists, we just need to generate the first half of the string.
\item To generate all distinct permutations of a (half of) string, use a similar approach from: \textbf{47 --- Permutations II} or \textbf{31 --- Next Permutation}.
\end{itemize}
\subsection{Next Permutation}
按照题目的提示,
\begin{itemize}
\item 判断$s$是否可以生成palindrome permutation。
\item 根据string中相同字符的出现次数,取一半构成一个base string。如果长度是odd,就把出现奇数次的字符放在生成的字符串中间位置。
\item 为了避免重复的memory操作,采用copy 字符串的方式。把base copy到生成字符串的左半边,然后reverse copy到生成字符串的右半边。
\item 检查生成的字符串是否在hash set中存在,如果存在,就中断循环。
\item 用next permuatation方法生成下一个base。
\end{itemize}
\setcounter{lstlisting}{0}
\begin{lstlisting}[style=customc, caption={Next Permutation}]
vector<string> generatePalindromes( string s )
{
    unordered_map<char, int> m;

    for( auto c : s )
    {
        auto it = m.find( c );

        if( it == m.end() )
        {
            m.emplace( c, 1 );
        }
        else
        {
            ++it->second;
        }
    }

    int odd_counts = 0;
    char odd_char = 0;

    string base;
    base.reserve( s.size() );

    //generate base i.e. half string for
    //permutation string generation
    for( const auto& p : m )
    {
        int count = p.second;

        if( count & 1 )
        {
            ++odd_counts ;
            odd_char = p.first;
        }
        base.append( count / 2, p.first );
    }


    //Check if s can form palindrome permutation
    if( s.size() & 1 )
    {
        if( odd_counts != 1 )
        {
            return {};
        }
    }
    else
    {
        if( odd_counts > 0 )
        {
            return {};
        }
    }

    unordered_set<string> ss;

    auto next = s;


    size_t righthalf_start = next.size() / 2;
    if( odd_counts != 0 )
    {
        //we set middle character as the odd count's char
        next[righthalf_start] = odd_char;
        ++righthalf_start;
    }

    while( true )
    {
        //use copy to avoid large amount of memory operation
        copy( base.begin(), base.end(), next.begin() );

        size_t start = righthalf_start;

        for( size_t i = 0; i < base.size(); ++i )
        {
            auto j = base.size() - i - 1;
            next[start] = base[j];
            ++start;
        }

        if( ss.find( next ) != ss.end() )
        {
            break;
        }

        ss.emplace( next );


        next_permutation( base );
    }

    return { ss.begin(), ss.end() };
}

//next permutation method
void next_permutation( string& s )
{
    auto L = s.size();

    auto x = L;

    for( size_t i = L - 1; i >= 1; --i )
    {
        if( s[i] > s[i - 1] )
        {
            x = i;
            break;
        }
    }

    if( x == L )
    {
        reverse( s.begin(), s.end() );
        return;
    }

    auto y = x;

    for( size_t j = x; j < L; ++j )
    {
        if( s[j] <= s[x - 1] )
        {
            break;
        }

        y = j;
    }

    swap( s[y], s[x - 1] );

    reverse( s.begin() + x, s.end() );
}

\end{lstlisting}