\section{267 --- Palindrome Permutation II}
Given a string $s$, return all the palindromic permutations (without duplicates) of it. Return an empty list if no palindromic permutation could be form.
\paragraph{Example 1:}
\begin{flushleft}
\textbf{Input}: $s=\texttt{aabb}$
\\
\textbf{Output}: (\texttt{abba}, \texttt{baab})
\end{flushleft}

\paragraph{Example 2:}
\begin{flushleft}
\textbf{Input}: $s=\texttt{abc}$
\\
\textbf{Output}: $\emptyset$
\end{flushleft}

\paragraph{Hint:}
\begin{itemize}
\item If a palindromic permutation exists, we just need to generate the first half of the string.
\item To generate all distinct permutations of a (half of) string, use a similar approach from: \textbf{47 --- Permutations II} or \textbf{31 --- Next Permutation}.
\end{itemize}
\subsection{Next Permutation}
按照题目的提示,
\begin{itemize}
\item 判断$s$是否可以生成palindrome permutation。
\item 根据string中相同字符的出现次数,取一半构成一个base string。如果长度是odd,就把出现奇数次的字符放在生成的字符串中间位置。
\item 为了避免重复的memory操作,采用copy 字符串的方式。把base copy到生成字符串的左半边,然后reverse copy到生成字符串的右半边。
\item 检查生成的字符串是否在hash set中存在,如果存在,就中断循环。
\item 用next permuatation方法生成下一个base。
\end{itemize}
\setcounter{lstlisting}{0}
\begin{lstlisting}[style=customc, caption={Next Permutation}]
vector<string> generatePalindromes( string s )
{
    unordered_map<char, int> m;
    for( auto c : s )
    {
        m[c] += 1;
    }
    string prefix;
    string mid;
    //generate prefix and middle character
    for( auto [c, count] : m )
    {
        if( count & 1 )
        {
            mid.push_back( c );
        }
        prefix.append( count / 2, c );
        if( mid.size() > 1ull )
        {
            //cannot form palindrome
            //we have two letter with odd counts
            return {};
        }
    }
    unordered_set<string> ss;
    //generate permutation of string
    dfs( prefix, mid, 0, ss );
    return { ss.begin(), ss.end() };
}
//generate permutation of prefix
void dfs( string& prefix, const string& mid, size_t start, unordered_set<string>& ss )
{
    if( start >= prefix.size() )
    {
        auto tmp( prefix );
        tmp += mid;
        tmp.append( rbegin( prefix ), rend( prefix ) );
        ss.emplace( tmp );
        return;
    }
    for( size_t i = start; i < prefix.size(); ++i )
    {
        if( ( i != start ) && ( prefix[i] == prefix[start] ) )
        {
            continue;
        }
        swap( prefix[i], prefix[start] );
        //continue recursion from start+1, not i+1
        dfs( prefix, mid, start + 1, ss );
        //backtracking
        swap( prefix[i], prefix[start] );
    }
}
\end{lstlisting}