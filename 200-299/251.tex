\section{251 --- Flatten 2D Vector}
Implement an iterator to flatten a 2d vector.
\par
For example,
\par
Given 2d vector = $[[1,2],\;[3],\;[4,5,6]]$
\par
By calling \texttt{next} repeatedly until \texttt{hasNext} returns \texttt{false}, the order of elements returned by \texttt{next} should be: 
\par

$[1,2,3,4,5,6]$.
\par
\texttt{Hint:}

\begin{itemize}
\item How many variables do you need to keep track?
\item Two variables is all you need. Try with $x$ and $y$.
\item Beware of empty rows. It could be the first few rows.
\item To write correct code, think about the invariant to maintain. What is it?
\item The invariant is $x$ and $y$ must always point to a valid point in the 2d vector. Should you maintain your invariant ahead of time or right when you need it?
\item Not sure? Think about how you would implement \texttt{hasNext}(). Which is more complex?
\item Common logic in two different places should be refactored into a common method.
\end{itemize}

\paragraph{Follow up:}
\begin{itemize}
\item As an added challenge, try to code it using only iterators in \texttt{C++} or iterators in \texttt{Java}.
\end{itemize}
\subsection{Iterator}
将$x$定义为row的iterator,再用另外一个变量$z$指向二维数组的末尾,同时定义一个整型变量$y$来指向列位置
\setcounter{lstlisting}{0}
\begin{lstlisting}[style=customc, caption={Iterator}]
class Vector2D
{
public:
    Vector2D( vector<vector<int>>& vec2d )
    {
        x = vec2d.begin();
        z = vec2d.end();
    }
	
    int next()
    {
        return ( *x )[y++];
    }
	
    bool hasNext()
    {
        if( ( x != z ) && ( y == ( *x ).size() ) )
        {
            ++x;
            y = 0;
        }
        return x != z;
    }

private:
    vector<vector<int>>::iterator x;
    vector<vector<int>>::iterator z;
    size_t y = 0;
};
\end{lstlisting}