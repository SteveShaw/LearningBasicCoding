\section{251 --- Flatten 2D Vector}
Design and implement an iterator to flatten a 2d vector. It should support the following operations: \fcj{next} and \fcj{hasNext}.

\paragraph{Example:}

\begin{flushleft}
\fcj{Vector2D iterator = new Vector2D([[1,2],[3],[4]]);}

\fcj{iterator.next(); // return 1}

\fcj{iterator.next(); // return 2}

\fcj{iterator.next(); // return 3}

\fcj{iterator.hasNext(); // return true}

\fcj{iterator.hasNext(); // return true}

\fcj{iterator.next(); // return 4}

\fcj{iterator.hasNext(); // return false}
\end{flushleft}

\paragraph{Follow up:}
\begin{itemize}
\item As an added challenge, try to code it using only iterators in \texttt{C++} or iterators in \texttt{Java}.
\end{itemize}

\subsection{Iterator}
将$x$定义为row的iterator,再用另外一个变量$z$指向二维数组的末尾,同时定义一个整型变量$y$来指向列位置
\setcounter{lstlisting}{0}
\begin{lstlisting}[style=customc, caption={Iterator}]
class Vector2D
{
public:
    Vector2D( vector<vector<int>>& v )
    {
        x = v.begin();
        endV = v.end();
        if( x != endV )
        {
            y = x->begin();
        }
    }
    int next()
    {
        if( !hasNext() )
        {
            //make sure we have next
            return -1;
        }
        int n = *y;
        ++y;
        return n;
    }
    bool hasNext()
    {
        //we need a loop here
        //to skip empty vectors
        while( ( x != endV ) && ( y == x->end() ) )
        {
            ++x;
            if( x != endV )
            {
                y = x->begin();
            }
        }
        return x != endV;
    }
private:
    vector<vector<int>>::iterator x;
    vector<vector<int>>::iterator endV;
    vector<int>::iterator y;
};
\end{lstlisting}

Another way using iterator: we define two iterator pairs: \fcj{outer} and \fcj{inner}.

\begin{itemize}
\item \fcj{outer} is used to traverse type \fcc{vector<vector<int>>}.
\item \fcj{inner} is used to traverse inner vector
\end{itemize}

\begin{lstlisting}[style=customc, caption={Two Iterators}]
class Vector2D
{
public:
    Vector2D( vector<vector<int>>& v )
    {
        auto& [out_b, out_e] = outer;
        out_b = v.begin();
        out_e = v.end();
        if( out_b != out_e )
        {
            //only when v is not empty
            //we will get the first vector's begin and end
            auto& [in_b, in_e] = inner;
            in_b = out_b->begin();
            in_e = out_b->end();
        }
    }
    int next()
    {
        //call hasNext() to move
        //inner to the first non empty
        //child vector
        hasNext();
        int x =  *inner.first;
        ++inner.first;
        return x;

    }
    bool hasNext()
    {
        if( outer.first == outer.second )
        {
            //we reached end of v
            return false;
        }
        //go to next non empty inner vector
        while( inner.first == inner.second )
        {
            ++outer.first;
            if( outer.first != outer.second )
            {
                //set inner as the first non empty inner vector
                //begin and end
                inner.first = outer.first->begin();
                inner.second = outer.first->end();
            }
            else
            {
                //we reach the end of v
                return false;
            }
        }
        return true;
    }
    using InnerIterT = vector<int>::iterator;
    using OuterIterT = vector<vector<int>>::iterator;
    pair<InnerIterT, InnerIterT> inner;
    pair<OuterIterT, OuterIterT> outer;
};
\end{lstlisting}