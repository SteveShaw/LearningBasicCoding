\section{233 --- Number of Digit One}
Given an integer $n$, count the total number of digit 1 appearing in all non-negative integers less than or equal to $n$.

\paragraph{Example:}

\begin{flushleft}
\textbf{Input}: 13
\\
\textbf{Output}: 6 
\\
\textbf{Explanation}: 
\\
Digit 1 occurred in the following numbers: 1, 10, 11, 12, 13.
\end{flushleft}
\subsection{Solve it mathematically}
逐个分析个位,十位,百位。。。每个位上的1。

\begin{itemize}
\item 对于每一位,将$n$按照当前位分成两个部分。
\item 例如$n=3141592$,当分析百位上的1时,将$n$从百位上分成$a=31415$和$b=92$两个部分。因此从100到3141199,那么百位上的1出现的次数即为$3142\times 100$。乘以100是因为这是百位1,所以每次递增个位和十位时,百位上的1就重复了100次。而这个次数用$a$代表即为$(\lfloor a/10\rfloor+1)\times 100$。
\item 分析千位上的1时,$a=3141$而$b=592$。这时候如果仅仅考虑从1000到3141999,这个位上的1的出现次数为$315\times 1000$。但是$n=3141592 < 3141999$,因此千位上的1在百位,十位和个位递增时,只重复了593次,即从$1000$到$1592$。因此实际的重复次数是$314\times 1000 + 592$,用$a$和$b$表示即为 $(\lfloor a / 10\rfloor \times 1000) + (b + 1)$ times.
\item 上述情况的区分点在于要分析的位上的数字是否大于等于2,如果大于等于2,那么从$100\ldots0$到$199\ldots9$都能包括。否则就要考虑实际的重复次数了。
\end{itemize}
\setcounter{algorithm}{0}
\begin{algorithm}[H]
\caption{Mathematical Induction}
\begin{algorithmic}[1]
\Procedure{CountDigitOne}{$n$}
\State $\star$ Initialize $a$ and $b$ as any value
\State $\star$ Initialize the counter $x$ as zero
\For{$p:=1$ \textbf{to} $n$}
\State $\ast$ Split $n$ into $a$ and $b$ based on $p$
\State $a\gets n / p$
\State $b\gets n - a\times p$ \Comment $b\gets n\bmod p$
\State $\ast$ Analyze if the number $r$ at position $p$ is no less than 2
\State $t:=a/10$ \Comment $a$ includes the number at position $p$
\State $r:=a-t\times 10$ \Comment $r$ is the number at position $p$
\If{$r < 2$} \Comment $r$ is zero or 1, so cannot count from $10\ldots0$ to $(t)19\ldots9$
\State $x\gets x+ t\times p$ \Comment The number 1 at $p$ position appears $\lfloor a/10\rfloor \times p$ times
\If{$r=1$}
\State $x\gets x+(b+1)$ \Comment Still needs to count from $1\ldots 0$ to $(t)1(b)$
\EndIf
\Else
\State $x\gets (t+1)\times p$ \Comment Number 1 at position $p$ can appear from $10\ldots0$ to $(t)19\ldots9$
\EndIf
\State $p\gets p\times 10$ \Comment Move to next position
\EndFor
\State \Return $x$
\EndProcedure
\end{algorithmic}
\end{algorithm}
\setcounter{lstlisting}{0}
\begin{lstlisting}[style=customc, caption={Mathematical Induction}]
int countDigitOne( int n )
{
    long long a = 0;
    long long b = 0;

    auto lln = static_cast<long long>( n );

    long long ans = 0;

    for( long long p = 1; p <= lln; p *= 10 )
    {
        a = n / p;
        b = n - p * a;

        //check if the number at position p
        //which is r
        //is no less than 2 or not
        auto t = a / 10;
        auto r = a - 10 * t;

        if( r < 2 )
        {
            ans += t * p;
            ans += ( r == 1 ) ? ( b + 1 ) : 0; // need to count to 1(b)
        }
        else
        {
            ans += ( t + 1 ) * p; //Count from 10...0 to (t)19...9
        }
    }

    return static_cast<int>( ans );
}
\end{lstlisting}