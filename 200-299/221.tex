\section{221 --- Maximal Square}
Given a 2D binary matrix $M$ filled with 0s and 1s, find the largest square containing only 1s and return its area.

\paragraph{Example:}

\begin{flushleft}
\textbf{Input}: 
\begin{table}[H]
\begin{tabular}{ccccc}
1 & 0 & 1 & 0 & 0 \\
1 & 0 & \textcolor{red}{1} & \textcolor{red}{1} & 1 \\
1 & 1 & \textcolor{red}{1} & \textcolor{red}{1} & 1 \\
1 & 0 & 0 & 1 & 0
\end{tabular}
\end{table}

\textbf{Output}: 4
\end{flushleft}

\subsection{2D Dynamic Programming}
We initialize another matrix $F$ with the same dimensions as the original one initialized with all zeros.

$F(i,j)$ represents the side length of the \textbf{maximum square} whose bottom right corner is the cell with index $ (i,j) $ in the original matrix.

We can get the recursive relationship as 

$F(i,j)=\min(F(i-1,j), F(i,j-1), F(i-1,j-1)) + 1$

In the meantime, record the size of the largest square found so far. 

In the implementation, we need $F$'s dimension as $(M+1, N+1)$ to cover the edge cases like $1\times N $, $M\times 1$.

\setcounter{lstlisting}{0}
\begin{lstlisting}[style=customc, caption={2D DP}]
int maximalSquare( vector<vector<char>>& matrix )
{
    if( matrix.empty() || ( matrix[0].empty() ) )
    {
        return 0;
    }
    auto M = matrix.size();
    auto N( matrix[0].size() );
    //we need (M+1) x (N+1) as F's dimension
    //to cover edge cases
    vector<vector<int>> F( M + 1, vector<int>( N + 1, 0 ) );
    auto max_side( 0 );
    for( size_t i = 1; i <= M; ++i )
    {
        for( size_t j = 1; j <= N; ++j )
        {
            if( matrix[i - 1][j - 1] == '1' )
            {
                //fill current element only found a 1
                auto x( ( min )( F[i - 1][j], F[i][j - 1] ) );
                x = ( min )( x, F[i - 1][j - 1] );
                F[i][j] = x + 1;
                max_side = ( max )( max_side, F[i][j] );
            }
        }
    }
    return max_side * max_side;
}
\end{lstlisting}

\subsection{1D DP}
Since $F(i,j)$ only depends on $F(i-1,j)$, $F(i,j-1)$ and $F(i-1,j-1)$, we can use 1D array $F$ as the DP array.

\begin{lstlisting}[style=customc, caption={1D DP}]
int maximalSquare( vector<vector<char>>& matrix )
{
    if( matrix.empty() || ( matrix[0].empty() ) )
    {
        return 0;
    }
    auto M = matrix.size();
    auto N( matrix[0].size() );
    //DP for one row
    vector<int> F( N + 1, 0 );
    auto max_side( 0 );
    //record last F[j]
    int prev = 0;
    for( size_t i = 1; i <= M; ++i )
    {
        for( size_t j = 1; j <= N; ++j )
        {
            //save current F[j] as temp
            auto temp( F[j] );
            if( matrix[i - 1][j - 1] == '1' )
            {
                //update F[i][j] as (min)(F[i-1][j-1], F[i-1][j], F[i][j-1]) +1
                //to convert to 1D
                //F[i-1][j-1] => prev
                //F[i-1][j] => F[j] before update
                //F[i][j-1] => F[j-1] after update
                auto x( ( min )( F[j], F[j - 1] ) );
                x = ( min )( x, prev );
                F[j] = x + 1;
                max_side = ( max )( max_side, F[j] );
            }
            else
            {
                F[j] = 0;
            }
            prev = temp;
        }
    }
    return max_side * max_side;
}
\end{lstlisting}

