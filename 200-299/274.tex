\section{274 --- H-Index}
Given an array of citations (each citation is a non-negative integer) of a researcher $A$, write a function to compute the researcher's \textbf{h-index}.
\par
According to the definition of \textbf{h-index} on Wikipedia:
\begin{quote}
 A scientist has index $h$ if $h$ of his/her $N$ papers have at least h citations each, and the other $N - h$ papers have no more than $h$ citations each.
\end{quote}

\paragraph{Example:}

\begin{flushleft}
\textbf{Input}: $A = [3,0,6,1,5]$
\\
\textbf{Output}: 3 
\\
\textbf{Explanation}: 
\\
$[3,0,6,1,5]$ means the researcher has 5 papers in total and each of them had received 3, 0, 6, 1, 5 citations respectively. 
\\
Since the researcher has 3 papers with at least 3 citations each and the remaining two with no more than 3 citations each, her \textbf{h-index} is 3.
\end{flushleft}
\paragraph{Note:} 
\begin{itemize}
\item If there are several possible values for $h$, the maximum one is taken as the \textbf{h-index}
\end{itemize}
\subsection{Bucket Sort}
\begin{itemize}
\item 假设有$x$篇论文的引用数都大于或者等于$x$,那么这个$x$就是H-index。
\item 假设$n$是总的论文数,将这些论文放入$n+1$个buckets中,这些bucket的编号从0到$n$。
\item 对于每一篇论文,如果其引用数与bucket的index相等,就将其放入这个bucket中。
\item 对于那些大于$n$引用数的论文,都将其放入编号为$n$的bucket中。
\item 然后从第$n$个bucket开始往前遍历,累加每个bucekt中的论文数量,当发现累加值大于当前bucket的index,那么这个index就是H-index。
\begin{enumerate}
\item 之所以要从$n$往前遍历,是因为题目要求在存在几个可能的H-index的情况下,取最大的。
\end{enumerate}
\end{itemize}
\setcounter{algorithm}{0}
\begin{algorithm}[H]
\caption{Bucket Sort}
\begin{algorithmic}[1]
\Procedure{HIndex}{$A, L$}
\State $\star$ Create an array $B$ with size $L+1$ as buckets. Each bucket is initialized to zero
\State $\ast$ 将$A$中每个number放入$B$中
\For{$i:=0$ \textbf{to} $L-1$}
\If{$A[i]> L$}
\State $B[L]\gets B[L]+1$ \Comment Put into the end of Bucket
\Else
\State $B[A[i]]\gets B[A[i]] + 1$ \Comment Put into the bucket where index equal to $A[i]$
\EndIf
\State $\ast$ 从$n$开始往前遍历buckets
\State $\delta:=0$ \Comment The total count of papers
\For{$i:=L$ \textbf{to} 0}
\State $\delta\gets\delta+B[i]$
\If{$\delta \geq i$} \Comment Total count of papers larger than the bucket index
\State \Return $i$ \Comment $i$ 就是H-index
\EndIf
\EndFor
\EndFor
\State \Return $-1$ \Comment 实际不会到达这一行
\EndProcedure
\end{algorithmic}
\end{algorithm}
\setcounter{lstlisting}{0}
\begin{lstlisting}[style=customc, caption={Bucket Sort}]
int hIndex( vector<int>& citations )
{
    vector<int> buckets( citations.size() + 1, 0 );

    int L = static_cast< int >( citations.size() );

    for( auto n : citations )
    {
        if( n > L )
        {
            ++buckets.back();
        }
        else
        {
            ++buckets[n];
        }
    }

    int total = 0;
    //iterate from $L$ to 0
    //to get largest H-index
    for( int i = L; i >= 0; --i )
    {
        total += buckets[i];
        if( total >= i )
        {
            //i is the H-index
            return i;
        }
    }

    return 0;

}
\end{lstlisting}