\section{276 --- Paint Fence}
There is a fence with $n$ posts, each post can be painted with one of the $k$ colors.
\par
You have to paint all the posts such that no more than two adjacent fence posts have the same color.
\par
Return the total number of ways you can paint the fence.
\par
\paragraph{Note:}
\begin{itemize}
\item $n$ and $k$ are non-negative integers.
\end{itemize}
\subsection{Dynamic Programming}
\begin{itemize}
\item $n=1$: 那么有$k$个方法,因为有$k$种颜色可以选择。
\item $n=2$: 分两种情况来统计,
\begin{enumerate}
\item 相邻部分没有相同的,这种情况下,第一个post有$k$个颜色可以选择,而第二个post就只能有$k-1$种颜色选择。因此有$k\times (k-1)$个方法。
\item 相邻部分具有相同颜色,这种情况下,两个post都有$k$种颜色可以选择,因此有$k\times k$个方法。
\item 总共有$k\times k+k\times(k-1) = k^2$
\end{enumerate}
\item $n=3$:同样是两种情况,\begin{enumerate}
\item 如果与第二个post颜色相同,那么显然第二个post不能选择和第一个post相同的颜色,因此这种情况下的方法即为第二个post刷与第一个post不同颜色的方法即$k\times(k-1)$。
\item 而如果与第二个post颜色不同,那么第二个post既能够和第一个post具有相同颜色,也可以与第一个post有不相同的颜色,也就是刷第二个post的总的方法即$k^2$在乘以第三个post能够选择的颜色种类即$k-1$,因为只要与第二个post不同颜色就行,与第一个post颜色相同没有关系。
\end{enumerate}
\end{itemize}
综上所述,假设有数组$x$和$y$
\begin{itemize}
\item $x[i]$表示第$i$个post与第$i-1$个post具有相同颜色的方法数
\item $y[i]$表示第$i$个post与第$i-1$个post有不相同颜色的方法数。
\end{itemize}
那么,第$i$个post的方法$F[i]$与第$i-1$个post的方法$F[i-1]$的关系如下
\begin{align*}
F[i-1] &= x[i-1] + y[i-1]\\
x[i] &= y[i-1]\\
y[i] &= F[i-1]\times(k-1)\\
F[i] &= x[i] + y[i]\\
\end{align*}
由于上述递推关系之和$i-1$有关,因此具体代码实现时,可以用两个变量来代表$x[i]$和$y[i]$

\setcounter{lstlisting}{0}
\begin{lstlisting}[style=customc, caption={Dynamic Programming}]
int numWays( int n, int k )
{
    if( n == 0 )
    {
        return 0;
    }

    int x = 0; //when current post and last post have same colors
    int y = k; //when current post and last post have different colors
    int F = x + y; //The total ways to paint current post

    for( int i = 2; i <= n; ++i )
    {
        x = y; //x[i]=y[i-1]
        y = F * ( k - 1 ); //y[i]=F*(k-1)
        F = x + y; //F[i]=x[i]+y[i]
    }
    return F;
}
\end{lstlisting}