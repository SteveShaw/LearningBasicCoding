\section{288 --- Unique Word Abbreviation}
An abbreviation of a word follows the form <first letter><number><last letter>. Below are some examples of word abbreviations:
\begin{enumerate}
\item \texttt{it} --- \texttt{it} (no abbreviation)
\item \texttt{dog} --- \texttt{d1g} (only 1 letter between \textbf{d} and \textbf{g})
\item \texttt{internationalization} --- \texttt{i18n} (There are 18 letters between the first letter \textbf{i} and the last one \textbf{n})
\item  \texttt{localization}  --- \texttt{l10n}
\end{enumerate}
Assume you have a dictionary $A$ and given a word $W$, find whether its abbreviation is unique in the dictionary. A word's abbreviation is \textbf{unique} \textbf{if no other word} from the dictionary has the same abbreviation.

\paragraph{Example: }

Given $A$ = [ \texttt{deer}, \texttt{door}, \texttt{cake}, \texttt{card} ]

\begin{lstlisting}[style=customc]
isUnique("dear") // false
isUnique("cart") // true
isUnique("cane") // false
isUnique("make") // true
\end{lstlisting}
\subsection{Hash Map}
注意题目中的\textbf{if no other word}。例如
\begin{itemize}
\item $A$ = [\texttt{dear}],
\begin{lstlisting}[style=customc]
isUnique("door") // false
\end{lstlisting}
\item $A$ = [\texttt{door}, \texttt{door}]
\begin{lstlisting}[style=customc]
isUnique("door") // true
\end{lstlisting}
\item $A$ = [\texttt{dear}, \texttt{door}]
\begin{lstlisting}[style=customc]
isUnique("door") // false
\end{lstlisting}
\end{itemize}
结合题目中的例子,unique的要求是
\begin{itemize}
\item 如果$A$中所有单词的缩写形式都没有和$W$相同的,是unqiue。
\item 如果有相同的,那么如果这个缩写形式还有除了$W$的其他单词,就不是unique,反之则仍然是unique。
\end{itemize}
因此用一个hash map建立缩写形式和其对应的单词的映射,把所有缩写形式的相同单词放到一个hash set中。在hash map中找不到缩写形式,返回\texttt{true}。如果找到了,而且对应的hash set只有一个单词,且这个单词就是$W$。那么也返回\texttt{true}。
\setcounter{lstlisting}{0}
\begin{lstlisting}[style=customc, caption={Hash Map}]
class ValidWordAbbr
{
public:
    ValidWordAbbr( vector<string> &dictionary )
    {
        for( auto a : dictionary )
        {
            string k = a[0] + to_string( a.size() - 2 ) + a.back();
            auto iter = m.find( k );
            if( iter == m.end() )
            {
                m.emplace( k, initializer_list<string> {a} );
            }
            else
            {
                iter->second.emplace( a );
            }
        }
    }
    bool isUnique( string word )
    {
        string k = word[0] + to_string( word.size() - 2 ) + word.back();
        auto iter = m.find( k );
        if( iter == m.end() )
        {
            return false;
        }
        //iter->second.size()==1
        //and iter->second only contains word
        return iter->second.count( word ) == iter->second.size();
    }
private:
    unordered_map<string, set<string>> m;
};
\end{lstlisting}
\subsection{Memory Efficient}
为了减少内存使用,可以把上述hash map的hash set换成string。当处理字典中的单词时,如果在hash map中找不到当前单词的缩写形式,将该缩写形式和当前单词建立映射,如果该缩写形式已经在hash map中,看对应的单词是不是当前单词,如果不是,就将这个缩写形式对应的值改为空字符串,这样只要字典中一个缩写形式对应一个以上的单词,最终的结果这个缩写形式对应的单词就是空字符串,如果$W$的缩写形式也是能够在hash map中找到,则直接比较$W$和这个缩写形式所对应的字符串,看是否相等。
\begin{lstlisting}[style=customc, caption={Memory Efficient}]
class ValidWordAbbr
{
public:
    ValidWordAbbr( vector<string> &dictionary )
    {
        for( auto a : dictionary )
        {
            string k = a[0] + to_string( a.size() - 2 ) + a.back();
            auto iter = m.find( k );
            if( iter == m.end() )
            {
                m.emplace( k, a );
            }
            else
            {
                //this abbr has already
                //been found in another different word
                if( iter->second != a )
                {
                    iter->second.clear();
                }
            }
        }
    }

    bool isUnique( string word )
    {
        string k = word[0] + to_string( word.size() - 2 ) + word.back();
        auto iter = m.find( k );
        if( iter == m.end() )
        {
            return true;
        }

        return iter->second == word;
    }
private:
    unordered_map<string, string> m;
};
\end{lstlisting}