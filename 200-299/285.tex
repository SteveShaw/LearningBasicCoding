\section{Inorder Successor in BST}
Given a binary search tree and a node in it, find the in-order successor of that node in the BST.
\par
The successor of a node $p$ is the node with the smallest key greater than p.val.

\paragraph{Example 1:}
\begin{flushleft}
\textbf{Input}:
\begin{figure}[H]
\begin{tikzpicture}
[mynode/.style={draw,circle,minimum size=5mm, fill=gray!20!}]
\node(){};
\node[mynode](1) {2};
\node[mynode](2) [below = 8mm of 1, xshift=-6mm] {1};
\node[mynode](3) [below = 8mm of 1, xshift=6mm] {3};
\draw (1) -- (2);
\draw (1) -- (3);
\end{tikzpicture}
\end{figure}
$p = 1$
\\
\textbf{Output}: 2
\\
\textbf{Explanation}: 
\\
1's in-order successor node is 2. Note that both $p$ and the return value is of \texttt{TreeNode} type.
\end{flushleft}

\paragraph{Example 2:}
\begin{flushleft}
\textbf{Input}:
\begin{figure}[H]
\begin{tikzpicture}
[mynode/.style={draw,circle,minimum size=5mm, fill=gray!20!}]
\node(){};
\node[mynode](1) {5};
\node[mynode](2) [below = 8mm of 1, xshift=-6mm] {3};
\node[mynode](3) [below = 8mm of 1, xshift=6mm] {6};
\node[mynode](4) [below = 8mm of 2, xshift=-6mm] {2};
\node[mynode](5) [below = 8mm of 2, xshift=6mm] {4};
\node[mynode](6) [below = 8mm of 4, xshift=-6mm] {1};
\draw (1) -- (2);
\draw (1) -- (3);
\draw (2) -- (4);
\draw (2) -- (5);
\draw (4) -- (6);
\end{tikzpicture}
\end{figure}
$p = 6$
\\
\textbf{Output}: \texttt{null}
\\
\textbf{Explanation}: 
\\
There is no in-order successor of the current node, so the answer is \texttt{null}.
\end{flushleft}
 

\paragraph{Note:}
\begin{itemize}
    \item If the given node has no in-order successor in the tree, return \texttt{null}.
    \item It's guaranteed that the values of the tree are unique.
\end{itemize}
\subsection{Iterative}
maintain一个 flag $b$,初始化为\texttt{false}。然后进行中序遍历,
\begin{itemize}
    \item 对于遍历到的节点,首先看检查$b$是否为\fcc{true}。如果是,则说明之前遍历到了$p$,那么当前节点就是$p$的succesor,
    \item 如果$b$仍为\fcc{false},则检查遍历到的节点和$p$是否相同,如果相同,则将$b$设为\fcc{true}
\end{itemize}
\setcounter{lstlisting}{0}
\begin{lstlisting}[style=customc, caption={Iterative In-order Traverse}]
TreeNode* inorderSuccessor( TreeNode* root, TreeNode* p )
{
    auto x = root;
    stack<TreeNode*> stk;

    bool bFound = false;

    //in-order traverse
    while( x || !stk.empty() )
    {
        while( x )
        {
            stk.push( x );
            x = x->left;
        }

        x = stk.top();
        stk.pop();

        if( bFound )
        {
            return x;
        }

        if( x == p )
        {
            bFound = true;
        }

        x = x->right;
    }

    return nullptr;
}
\end{lstlisting}
\subsection{Recursive In-Order Traverse}
采用递归方式的中序遍历需要两个全局变量$x$和$y$,分别用来记录父节点和后继节点。
\begin{itemize}
    \item 开始时,初始化为\texttt{null}
    \item 在递归时,对于当前节点$t$,首先检查$x$和$p$是否相同,如果相同,则将$y$赋为$t$,然后$x$ update为$t$,因此在递归过程中,$x$始终是当前访问的节点。
    \item 最后返回$y$即可。
    \item 另外,当检查到$x$和$p$相同,将$y$设为$t$后,不能立即返回,因为如果这时候返回,递归还要回到上一层,$t$又会被访问一次,而$y$就会被设为$t$的parent节点了。所以,还需要继续递归过程。
\end{itemize}
\begin{lstlisting}[style=customc, caption={Recursive In-order Traverse}]
class Solution
{
public:
    TreeNode* inorderSuccessor( TreeNode* root, TreeNode* p )
    {
        inorder( root, p );

        return y;
    }

    void inorder( TreeNode* t, TreeNode* p )
    {
        if( !t )
        {
            return;
        }

        inorder( t->left, p );

        if( x == p )
        {
            //cannot return from here
            //otherwise it will go back
            //to last step and t will be
            //visited again
            y = t;
        }

        x = t;

        inorder( t->right, p );


    }

    TreeNode* x = nullptr;
    TreeNode* y = nullptr;
};
\end{lstlisting}
\subsection{BST Property --- Iterative}
\begin{itemize}
    \item 比较当前节点$t$的value和$p$节点值的大小,如果$t$的value大,说明$p$肯定在$t$的左子树中,反之$p$在$t$的右子树中。
    \item 在整个比较跳转的过程中,在$t$的value比$p$的值大的时候,keep recording $t$。最后当最后$t$变为null退出循环的时候,最后记录的$t$就是$p$的successor。这是因为$p$的在中序遍历的successor的值肯定是比$p$大的。而上述比较跳转过程,则逐步接近到其successor。
\end{itemize}
\begin{lstlisting}[style=customc, caption={BST Property --- Iterative}]
TreeNode* inorderSuccessor( TreeNode* root, TreeNode* p )
{
    auto t = root;
    auto ans = root;

    while( t )
    {
        if( t->val <= p->val )
        {
            t = t->right;
        }
        else
        {
            //always record t when t->val > p->val
            ans = t;
            t = t->left;
        }
    }

    return ans;
}
\end{lstlisting}
\subsection{BST Property --- Recursive}
递归的方法类似,
\begin{itemize}
    \item 当根节点值小于等于$p$节点值,说明$p$的后继节点一定在右子树中,所以对right node递归,
    \item 如果根节点值大于$p$节点值,那么有可能根节点就是$p$的后继节点,或者左子树中的某个节点是$p$的后继节点,所以先对left node递归,如果返回\texttt{null},说明根节点是后继节点,如果不是\texttt{null},即为$p$的successor
\end{itemize}
\begin{lstlisting}[style=customc, caption={BST Property --- Recursive}]
TreeNode* inorderSuccessor( TreeNode* root, TreeNode* p )
{
    if( !root )
    {
        return nullptr;
    }

    if( root->val <= p->val )
    {
        //search in right subtree
        return inorderSuccessor( root->right, p );
    }
    else
    {
        //search in left subtree
        auto next = inorderSuccessor( root->left, p );
        if( !next )
        {
            //The successor must be the root
            return root;
        }

        return next;
    }
}
\end{lstlisting}

\paragraph{Related Problems}
\begin{itemize}
\item \textbf{94. Binary Tree Inorder Traversal}
\item \textbf{173. Binary Search Tree Iterator}
\item \textbf{510. Inorder Successor in BST II}
\end{itemize}