\section{254 --- Factor Combinations}
Numbers can be regarded as product of its factors. 

For example, $8 = 2 \times 2 \times 2 = 2\times 4$

Write a function that takes an integer $n$ and return all possible combinations of its factors.

\paragraph{Note: }

\begin{itemize}
\item Each combination's factors must be sorted ascending, for example: The factors of 2 and 6 is $[2, 6]$, not $[6, 2]$.
\item You may assume that $n$ is always positive.
\item Factors should be greater than 1 and less than $n$.
\end{itemize}
 

\paragraph{Examples: }
\begin{flushleft}
\textbf{input}: 1

\textbf{output}: \fcj{[]}

\textbf{input}: 37

\textbf{output}: $[]$

\textbf{input}: 12

\textbf{output}: $[2,6], \ [2, 2, 3], \ [3, 4]$

\textbf{input}: 32

\textbf{output}:

$[[2, 16],\  [2, 2, 8],\  [2, 2, 2, 4],\ [2, 2, 2, 2, 2],\   [2, 4, 4],\ [4, 8]]$
\end{flushleft}

\subsection{Backtracking}
\begin{itemize}
\item 递归函数中,由于题目要求factor是按照升序排列,因此在测试factor时,如果遇到$n$除以factor得到比$i$小的值就可以stop测试了。
\item 另外,也是由于factor是按照升序排列,因此将当前factor数组中的最后一个也是最大的一个作为递归函数的参数,这样递归函数中测试factor的循环就从这个参数开始进行测试。
\item 为了重复利用当前factor数组,在递归调用结束后,总是弹出最后一个数。
\end{itemize}

\setcounter{lstlisting}{0}
\begin{lstlisting}[style=customc, caption={Backtracking}]
vector<vector<int>> getFactors( int n )
{
    vector<int> factors;
    vector<vector<int>> ans;
    dfs( n, 2, factors, ans );
    return ans;
}
void dfs( int n, int start, vector<int>& factors, vector<vector<int>>& ans )
{
    if( n == 1 )
    {
        if( factors.size() > 1ull )
        {
            ans.emplace_back( factors.begin(), factors.end() );
        }
    }
    else
    {
        for( int i = start; i <= n; ++i )
        {
            int q = n / i;
            int r = n -  q * i;
            if( r == 0 )
            {
                //i is one factor
                factors.push_back( i );
                dfs( q, i, factors, ans );
                factors.pop_back();
            }
        }
    }
}
\end{lstlisting}
\subsection{DFS Using Only One Function}
这种方法直接用原来的\texttt{GetFactors}进行递归调用。从2遍历到$n$的平方根,
\begin{itemize}
\item 如果$i$是因子,递归调用$n/i$,结果用$x$来保存,然后新建一个包含$i$和$n/i$两个因子的序列$y$,然后将其存入结果$z$,
\item 然后再遍历之前递归$n/i$的所得到的结果,如果$i$小于等于结果中某个factor数组的第一个数字,那么将其插入该数组的首位置,然后把该数组存入结果$z$中,
\item 例如$n = 12$,那么刚开始$i = 2$,是12的一个factor,然后对$n/i=12/2=6$递归处理,得到$[2,3]$,
\begin{itemize}
\item 首先将$[2, 6]$先存入结果$z$中,
\item 然后发现$i$(此时为2)小于等于$[2, 3]$中的第一个数字2,那么将2插入首位置得到$[2, 2, 3]$加入结果$z$中,
\item 继续测试下一个factor,此时$i$变成3,还是12的factor,对$n/i=12/3=4$递归处理,得到$[2, 2]$,
\item 同样先把$[3, 4]$存入结果$z$中,然后发现$i$(此时为3)大于$[2, 2]$中的第一个数字2,这时候就忽略掉$[2,2]$。因为$[2,2,3]$在此之前肯定已经加入到结果$z$中了。
\end{itemize}
\end{itemize}
\begin{lstlisting}[style=customc, caption={Another DFS Approach}]
vector<vector<int>> getFactors( int n )
{
    vector<vector<int>> ans;
    for( int i = 2; i * i <= n; ++i )
    {
        if( ( n % i ) == 0 )
        {
            auto nextFactors = getFactors( n / i );
            ans.emplace_back( vector<int> {i, n / i} );
            for( auto& factors : nextFactors )
            {
                if( i <= factors[0] )
                {
                    //i does not include along with factors
                    //add i
                    ans.emplace_back( vector<int> {i} );
                    //add other factors
                    ans.back().insert( ans.back().end(), factors.begin(), factors.end() );
                }
            }
        }
    }
    return ans;
}
\end{lstlisting}