\section{254 --- Factor Combinations}
Numbers can be regarded as product of its factors. 
\par
For example, $8 = 2 \times 2 \times 2 = 2\times 4$
\par
Write a function that takes an integer $n$ and return all possible combinations of its factors.

\paragraph{Note: }

\begin{itemize}
\item Each combination's factors must be sorted ascending, for example: The factors of 2 and 6 is $[2, 6]$, not $[6, 2]$.
\item You may assume that $n$ is always positive.
\item Factors should be greater than 1 and less than $n$.
\end{itemize}
 

\paragraph{Examples: }
\begin{flushleft}
\textbf{input}: 1
\\
\textbf{output}: $[]$
\\
\textbf{input}: 37
\\
\textbf{output}: $[]$
\\
\textbf{input}: 12
\\
\textbf{output}: $[2,6], \ [2, 2, 3], \ [3, 4]$
\\
\textbf{input}: 32
\\
\textbf{output}:
\\
$[[2, 16],\  [2, 2, 8],\  [2, 2, 2, 4],\ [2, 2, 2, 2, 2],\   [2, 4, 4],\ [4, 8]]$
\end{flushleft}
\subsection{Depth First Search}
\begin{itemize}
\item 递归函数中,由于题目要求factor是按照升序排列,因此在测试factor时,如果遇到$n$除以factor得到比$i$小的值就可以stop测试了。
\item 另外,也是由于factor是按照升序排列,因此将当前factor数组中的最后一个也是最大的一个作为递归函数的参数,这样递归函数中测试factor的循环就从这个参数开始进行测试。
\item 为了重复利用当前factor数组,在递归调用结束后,总是弹出最后一个数。
\end{itemize}
\setcounter{algorithm}{0}
\begin{algorithm}[H]
\caption{DFS}
\begin{algorithmic}[1]
\Procedure{GetFactors}{$n$}
\State $\star$ Initialize an array $x$ to save found factors
\State $\star$ Initialize an array $y$ to save all found factors
\State \Call{DFS}{$n, 2, x, y$} \Comment 最开始进行递归函数\texttt{DFS}调用时,从factor等于2开始测试
\State \Return $y$
\EndProcedure
\end{algorithmic}
\end{algorithm}
函数\texttt{DFS}的输入参数包括
\begin{itemize}
\item $n$: 输入的整数
\item $\alpha$: 测试的起始factor
\item $x$,$y$同上
\end{itemize}
\begin{algorithm}[H]
\caption{Helper Recursive Function}
\begin{algorithmic}[1]
\Procedure{DFS}{$n, \alpha, x, y$}
\For{$i:=\alpha$ \textbf{to} $n/\alpha$} \Comment 只需测试到$n/\alpha$,再往后起始就重复了
\State $q:=n/i$
\If{$q < i$} \Comment 这时候已经不能构成升序的factor了
\State \texttt{break} \Comment 退出循环
\EndIf
\If{$n-q\times i=0$} \Comment $i$是$n$的factor
\State $\star$将$i$加入到$x$的\texttt{back}
\State $\star$将$q$加入到$x$的\texttt{back}
\State $\star$将$x$加入到$y$的\texttt{back} \Comment 找到了一个factor数组
\State $\star$将$q$从$x$中remove,因为接下来要对$q$进行递归处理
\State \Call{DFS}{$n=q$, $\alpha$=$x$的最后一个元素, $x$, $y$}
\State $\ast$ 递归结束后,将$x$的最后一个元素从$x$中remove以重复利用$x$
\EndIf
\EndFor
\EndProcedure
\end{algorithmic}
\end{algorithm}
\setcounter{lstlisting}{0}
\begin{lstlisting}[style=customc, caption={DFS}]
vector<vector<int>> getFactors( int n )
{
    vector<int> next;
    vector<vector<int>> ans;

    //factor start from 2
    dfs( n, next, 2, ans );

    return ans;
}

void dfs( int n, vector<int>& cur, int last, vector<vector<int>>& ans )
{
    //factor testing start from last
    for( int i = last; i <= n / last; ++i )
    {
        int q = n / i;

        //The factor array needs to be sorted ascending
        //therefore we can stop testing i when q < i
        if( q < i )
        {
            break;
        }

        if( n - q * i == 0 )
        {
            cur.push_back( i );
            cur.push_back( q );
            ans.emplace_back( cur.begin(), cur.end() );
            //We must remove q from cur, since next
            //we will process q
            cur.pop_back();
            dfs( q, cur, cur.back(), ans );
            cur.pop_back();
        }
    }
}
\end{lstlisting}
\subsection{DFS Using Only One Function}
这种方法直接用原来的\texttt{GetFactors}进行递归调用。从2遍历到$n$的平方根,
\begin{itemize}
\item 如果$i$是因子,递归调用$n/i$,结果用$x$来保存,然后新建一个包含$i$和$n/i$两个因子的序列$y$,然后将其存入结果$z$,
\item 然后再遍历之前递归$n/i$的所得到的结果,如果$i$小于等于结果中某个factor数组的第一个数字,那么将其插入该数组的首位置,然后把该数组存入结果$z$中,
\item 例如$n = 12$,那么刚开始$i = 2$,是12的一个factor,然后对$n/i=12/2=6$递归处理,得到$[2,3]$,
\begin{itemize}
\item 首先将$[2, 6]$先存入结果$z$中,
\item 然后发现$i$(此时为2)小于等于$[2, 3]$中的第一个数字2,那么将2插入首位置得到$[2, 2, 3]$加入结果$z$中,
\item 继续测试下一个factor,此时$i$变成3,还是12的factor,对$n/i=12/3=4$递归处理,得到$[2, 2]$,
\item 同样先把$[3, 4]$存入结果$z$中,然后发现$i$(此时为3)大于$[2, 2]$中的第一个数字2,这时候就忽略掉$[2,2]$。因为$[2,2,3]$在此之前肯定已经加入到结果$z$中了。
\end{itemize}
\end{itemize}
\begin{lstlisting}[style=customc, caption={Another DFS Approach}]
vector<vector<int>> getFactors( int n )
{
    vector<vector<int>> z;

    for( int i = 2; i * i <= n; ++i )
    {
        if( n % i == 0 ) //i is the factor of n
        {
            vector<vector<int>> x = getFactors( n / i );
            vector<int> y{ i, n / i };

            //add y to z
            z.emplace_back( y.begin(), y.end() );

            //iterate over x
            for( auto& t : x )
            {
                if( i <= t[0] )
                {
                    t.insert( t.begin(), i );
                    z.emplace_back( t.begin(), t.end() );
                }
            }
        }
    }

    return z;
}
\end{lstlisting}