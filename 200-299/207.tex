\section{207 --- Course Schedule}
There are a total of $n$ courses you have to take, labeled from 0 to $n-1$.
\par
Some courses may have prerequisites, for example to take course 0 you have to first take course 1, which is expressed as a pair: \fcj{[0,1]}
\par
Given the total number of courses $n$ and a list of prerequisite pairs $P$, is it possible for you to finish all courses?
\paragraph{Example 1:}
\begin{flushleft}
\textbf{Input}: $2$, \fcj{[[1,0]]}
\\
\textbf{Output}: \fcj{true}
\\
\textbf{Explanation}:
\\
There are a total of 2 courses to take. To take course 1 you should have finished course 0. So it is possible.
\end{flushleft}
\paragraph{Example 2:}
\begin{flushleft}
\textbf{Input}: $2$, \fcj{[[1,0],[0,1]]}
\\
\textbf{Output}: \fcj{false}
\\
\textbf{Explanation}:
\\
There are a total of 2 courses to take. To take course 1 you should have finished course 0, and to take course 0 you should also have finished course 1. So it is impossible.
\end{flushleft}
\paragraph{Note:}
\begin{itemize}
    \item The input prerequisites is a graph represented by a list of edges, not adjacency matrices. Read more about how a graph is represented.
    \item You may assume that there are no duplicate edges in the input prerequisites.
\end{itemize}
\subsection{DFS}
To detect cycle in directed graph, mark the visited node as one color number like 1. And then if there is a cycle, some nodes that has already been colored as $1$ could be meeted again. If there is no cycle, Otherwise, after the DFS is done for this starting node, mark this node as another color number like 2.
\setcounter{lstlisting}{0}
\begin{lstlisting}[style=customc]
bool canFinish( int numCourses, vector<pair<int, int>>& prerequisites )
{
    unordered_map<int, unordered_set<int>> g;

    for( const auto& p : prerequisites )
    {
        auto it = g.find( p.second );

        if( it == g.end() )
        {
            g.emplace( p.second, initializer_list<int> {p.first} );
        }
        else
        {
            it->second.insert( p.first );
        }
    }

    vector<int> color( numCourses, 0 );

    for( const auto& p : g )
    {
        int start = p.first;
        if( !dfs( g, start, color ) )
        {
            return false;
        }
    }

    return true;
}

bool dfs( unordered_map<int, unordered_set<int>>& g, int x, vector<int>& color )
{
    if( color[x] == 2 )
    {
        //visited before
        return true;
    }

    if( color[x] == 1 )
    {
        //visited but still in DFS process
        return false;
    }

    //Mark color of x as 1 which means DFS is started and in the progress
    color[x] = 1;

    for( int y : g[x] )
    {
        if( !dfs( g, y, color ) )
        {
            return false;
        }
    }

    color[x] = 2; //Mark color of x as 2. Survivied DFS search

    return true;
}
\end{lstlisting}
\subsection{BFS}
The idea is to simply use \textbf{Topological Sorting}
\begin{enumerate}
    \item Get in-degree (number of incoming edges) for each of the vertex present in the graph and initialize the count of visited nodes as 0.
    \item Pick all the vertices with in-degree as 0 and add them into a queue $Q$
    \item Pop a vertex from $Q$ and then
    \begin{itemize}
        \item increment count of visited nodes by 1. 
        \item Decrease in-degree by 1 for all its neighboring nodes.
    \item If in-degree of a neighboring nodes is reduced to zero, then add it to $Q$.
\end{itemize}
\item Repeat 3 until $Q$ is empty.
\item If count of visited nodes is not equal to the number of nodes in the graph, then there exits a cycle.
\end{enumerate}

\begin{lstlisting}[style=customc, caption={BFS}]
bool canFinish( int numCourses, vector<pair<int, int>>& prerequisites )
{

    vector<int> idg( numCourses, 0 );

    unordered_map<int, unordered_set<int>> g;

    for( const auto& p : prerequisites )
    {
        ++idg[p.first];

        auto it = g.find( p.second );
        if( it == g.end() )
        {
            g.emplace( p.second, initializer_list<int> {p.first} );
        }
        else
        {
            it->second.insert( p.first );
        }
    }

    queue<int> q;

    int num_visited = 0;

    for( int i = 0; i < numCourses; ++i )
    {
        if( idg[i] == 0 )
        {
            q.push( i );
        }
    }

    while( !q.empty() )
    {
        int node = q.front();
        q.pop();

        ++num_visited;

        auto it = g.find( node );

        if( it == g.end() )
        {
            //node does not have any node connected
            continue;
        }

        for( int t : it->second )
        {
            if( idg[t] > 0 )
            {
                --idg[t]; //Decrease the in-degreee of t

                if( idg[t] == 0 )
                {
                    q.push( t );
                }
            }
        }
    }

    return num_visited == numCourses;

}
\end{lstlisting}