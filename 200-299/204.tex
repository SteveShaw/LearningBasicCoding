\section{204 --- Count Primes}
Count the number of prime numbers less than a non-negative number, $n$.
\paragraph{Example:}

\begin{flushleft}
\textbf{Input}: 10
\\
\textbf{Output}: 4
\\
\textbf{Explanation}: 
\\
There are 4 prime numbers less than 10, they are 2, 3, 5, 7.
\end{flushleft}
\subsection{Sieve of Eratosthenes}
\begin{itemize}
\item 从2开始遍历到$\sqrt{n}$,
\item 先找到第一个质数2,然后将其所有的倍数全部标记出来,
\item 然后到下一个质数3,标记其所有倍数,
\item 依次类推,直到$\sqrt{n}$,此时数组中未被标记的数字就是质数。
\end{itemize}
maintain 一个 $n-1$ 长度的数组来记录每个数字是否被标记,长度为$n-1$的原因是题目说是小于$n$的质数个数,并不包括$n$。

\setcounter{lstlisting}{0}
\begin{lstlisting}[style=customc, caption={Sieve of Eratosthenes}]
int countPrimes( int n )
{
    if( n <= 2 )
    {
        return 0;
    }
    vector<unsigned char> A( n - 1, 1 );

    A[0] = 0; // 1 is not a prime number

    int i = 2;

    while( i * i <= n )
    {
        if( A[i - 1] == 1 )
        {
            for( int j = i * i; j < n; j += i )
            {
                A[j - 1] = 0;
            }
        }

        ++i;
    }

    int ans = accumulate( A.begin(), A.end(), 0 );

    return ans;
}
\end{lstlisting}