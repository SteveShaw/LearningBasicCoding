\section{269 --- Alien Dictionary}
There is a new alien language which uses the latin alphabet. However, the order among letters are unknown to you. You receive a list of non-empty words $W$ from the dictionary, where words are sorted lexicographically by the rules of this new language. Derive the order of letters in this language.

\paragraph{Example 1:}

\begin{flushleft}
\textbf{Input}
\begin{table}[H]
\begin{tabular}{l}
\texttt{wrt}  \\
\texttt{wrf}  \\
\texttt{er}   \\
\texttt{ett}  \\
\texttt{rftt}
\end{tabular}
\end{table}
\textbf{Output}: \texttt{wertf}
\end{flushleft}

\paragraph{Example 2:}

\begin{flushleft}
\textbf{Input}:
\begin{table}[H]
\begin{tabular}{l}
\texttt{z}  \\
\texttt{x} 
\end{tabular}
\end{table}
\textbf{Output}: \texttt{zx}
\end{flushleft}

\paragraph{Example 3:}

\begin{flushleft}
\textbf{Input}:
\begin{table}[H]
\begin{tabular}{l}
\texttt{z}  \\
\texttt{x} \\
\texttt{z} 
\end{tabular}
\end{table}
\textbf{Output}: $\emptyset$
\\
\textbf{Explanation}: The order is invalid, so return  $\emptyset$.
\end{flushleft}

\paragraph{Note:}

\begin{itemize}
\item You may assume all letters are in lowercase.
\item You may assume that if a is a prefix of b, then a must appear before b in the given dictionary.
\item If the order is invalid, return an empty string.
\item There may be multiple valid order of letters, return any one of them is fine.
\end{itemize}
\subsection{Toplogical Sort}
这其实是build a directed graph, calculate each node's in-degree and output them according to in-degree.
\begin{itemize}
\item 构建出相邻的word在相同的位置上的letter形成的edges所构成的graph。
\item 同时用一个hash set保存所有words中的letters。
\item 边界情况:例如\texttt{baac}出现在\texttt{baa}之前,这种情况下不是valid order。
\item 根据BFS为基础的topological sort, starting with nodes with zero in-degree. 
\item Finally, need to make sure there is no circle in the graph. We need to check if the generated string size is equal to the number of unique letters.
\end{itemize}

\setcounter{lstlisting}{0}
\begin{lstlisting}[style=customc, caption={BFS}]
string alienOrder( vector<string>& words )
{
    unordered_map<char, unordered_set<char>> g;

    unordered_set<char> chs;

    auto L = static_cast< int >( words.size() );

    for( int i = 0; i < L - 1; ++i )
    {
        auto& w1 = words[i];
        auto& w2 = words[i + 1];

        chs.insert( w1.begin(), w1.end() );

        auto len = ( min )( w1.size(), w2.size() );
        auto k = 0;

        for( k = 0; k < len; ++k )
        {
            if( w1[k] != w2[k] )
            {
                auto it = g.find( w1[k] );

                if( it == g.end() )
                {
                    g.emplace( w1[k], initializer_list<char> { w2[k] } );
                }
                else
                {
                    it->second.emplace( w2[k] );
                }

                break;
            }
        }

        //Invalid Order
        if( ( k == len ) && ( w1.size() > w2.size() ) )
        {
            return "";
        }
    }

    // add final word's characters into the
    chs.insert( words.back().begin(), words.back().end() );

    vector<int> v_degIn( 26, 0 );

    for( const auto& p : g )
    {
        for( auto c : p.second )
        {
            v_degIn[c - 'a']++;
        }
    }

    queue<char> q;

    string ans = "";

    for( auto c : chs )
    {
        if( v_degIn[c - 'a'] == 0 )
        {
            // add in-degree=0 nodes
            // into the queue
            q.push( c );
            ans.push_back( c );
        }
    }

    while( !q.empty() )
    {
        auto c = q.front();
        q.pop();

        for( auto next : g[c] )
        {
            v_degIn[next - 'a']--;

            if( v_degIn[next - 'a'] == 0 )
            {
                q.push( next );
                ans.push_back( next );
            }
        }
    }

    if( ans.size() == g.size() )
    {
        return ans;
    }

    //There exists a cycle
    return "";

}
\end{lstlisting}
\subsection{DFS}
如果用Depth First Search,需要改变$g$的存储结构: 用一个Boolean二维数组。如果$g[i][j]=\texttt{true}$,就表示有directed edge between $i$ and $j$。
\par
另外采用这种结构就不需要用额外的hash set去存储所有words中的letter,只需要将$g[i][i]=\texttt{true}$。
\par
DFS需要对已经访问的节点进行mark,但可以从任何一个node开始进行深度搜索。由于题目中所有的words中只包含了lowercase letter,因此可以逐个递归26个lowercase letter。
\par
在递归中
\begin{itemize}
\item 检测该字符是否包含在words中
\item 访问从该字符出发的与其相连的其他字符。检测这些字符有没有被访问过,如果访问过,表明存在cycle。输出invalid order即空序列。
\item 然后继续从这些字符进行深度搜索。任何一个深度搜索返回invalid order都直接输出空序列。
\item 如果这些字符都是valid order,将当前字符标注为已访问。同时将其标注为不在words中的letter。
\item 由于加入letter到输出序列是在每次深度搜索完成后,因此最后产生的序列其实是倒序的,返回时需要将这个序列进行reverse。
\end{itemize}
根据上述递归过程,在字典序排序后面的字符会比前面的更早处理。如果最开始进行DFS的letter是字典序排序最前面的,那么经过从这个letter开始的深度搜索一次就完成了字典序列的创建。
\setcounter{lstlisting}{0}
\begin{lstlisting}[style=customc, caption={DFS}]
string alienOrder( vector<string>& words )
{
    vector<vector<unsigned char>> g( 26, vector<unsigned char> ( 26, 0 ) );

    auto L = static_cast< int >( words.size() );

    //A helper function to add letters from
    //a string to g
    auto add_chars = [&g]( const string & w )
    {
        for( auto c : w )
        {
            g[c - 'a'][c - 'a'] = 1;
        }
    };

    for( int i = 0; i < L - 1; ++i )
    {
        auto &w1 = words[i];
        auto &w2 = words[i + 1];

        add_chars( w1 );

        auto l = ( min )( w1.size(), w2.size() );

        size_t k = 0;

        for( ; k < l; ++k )
        {
            if( w1[k] != w2[k] )
            {
                g[w1[k] - 'a'][w2[k] - 'a'] = 1;
                break;
            }
        }

        //this is invalid order
        if( ( k == l ) && ( w1.size() > w2.size() ) )
        {
            return "";
        }
    }

	//The last word also needs to be added
    add_chars( words.back() );

    vector<unsigned char> seen( 26, 0 );

    string ans;

    //test each letter by DFS
    for( char c = 'a'; c <= 'z'; ++c )
    {
        int i = c - 'a';
        if( !dfs( i, g, seen, ans ) )
        {
            return "";
        }
    }

    return string( ans.rbegin(), ans.rend() );
}

bool dfs( int x, vector<vector<unsigned char>>&g, vector<unsigned char>& seen, string &ans )
{
    if( g[x][x] == 0 )
    {
        return true;
    }
    seen[x] = 1;

    for( int i = 0; i < 26; ++i )
    {
        if( ( i == x ) || ( g[x][i] == 0 ) )
        {
            continue;
        }

        if( seen[i] == 1 )
        {
            return false;
        }

        if( !dfs( i, g, seen, ans ) )
        {
            return false;
        }
    }

    //mark this letter as
    //unseen and not included in words
    //to make sure no repeat
    seen[x] = 0;
    g[x][x] = 0;

    ans.push_back( x + 'a' );

    return true;
}
\end{lstlisting}

\paragraph{Related Questions}
\begin{itemize}
\item \textbf{210. Course Schedule II}
\end{itemize}