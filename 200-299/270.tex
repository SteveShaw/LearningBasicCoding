\section{270 --- Closest Binary Search Tree Value}
Given a non-empty binary search tree and a target value $T$, find the value in the BST that is closest to the target.

\paragraph{Note:}

\begin{itemize}
\item Given target value is a floating point.
\item You are guaranteed to have only one unique value in the BST that is closest to the target.
\end{itemize}
\subsection{BST}
由于是比较浮点值,因此用当前节点的值和目标值的差的绝对值作为close的判定依据。
\setcounter{lstlisting}{0}
\begin{lstlisting}[style=customc,caption={BST}]
int closestValue( TreeNode* root, double target )
{
    int ans = root->val;
    auto node = root;

    while( node )
    {
        //compare the abs value
        if( abs( ans - target ) >= abs( node->val - target ) )
        {
            ans = node->val;
        }
        node = target < node->val ? node->left : node->right;
    }
    return ans;
}
\end{lstlisting}

\paragraph{Related Problems}
\begin{itemize}
\item \textbf{222. Count Complete Tree Nodes}
\item \textbf{272. Closest Binary Search Tree Value II}
\item \textbf{700. Search in a Binary Search Tree}
\end{itemize}