\section{271 --- Encode and Decode Strings}
Design an algorithm to encode a list of strings $A$ to a string $S$. The encoded string is then sent over the network and is decoded back to the original list of strings.
\par
Machine 1 (\textbf{sender}) has the function:

\begin{lstlisting}[style=customc]
string encode(vector<string> strs) {
  // ... your code
  return encoded_string;
}
\end{lstlisting}

Machine 2 (\textbf{receiver}) has the function:

\begin{lstlisting}[style=customc]
vector<string> decode(string s) {
  //... your code
  return strs;
}
\end{lstlisting}
 

So Machine 1 does:

\begin{lstlisting}[style=customc]
string encoded_string = encode(strs);
\end{lstlisting}
 

and Machine 2 does:

\begin{lstlisting}[style=customc]
vector<string> strs2 = decode(encoded_string);
\end{lstlisting}
 

\texttt{strs2} in Machine 2 should be the same as \texttt{strs} in Machine 1.
\par
Implement the \texttt{encode} and \texttt{decode} methods.

\paragraph{Note:}

\begin{itemize}
\item The string may contain any possible characters out of 256 valid \texttt{ascii} characters. Your algorithm should be generalized enough to work on any possible characters.
\item Do not use class member/global/static variables to store states. Your encode and decode algorithms should be stateless.
\item Do not rely on any library method such as \texttt{eval} or \textbf{serialize} methods. You should implement your own encode/decode algorithm.
\end{itemize}
\subsection{Analysis}
\begin{itemize}
\item 如果不涉及到复杂的编码算法设计,这道题其实可以很简单的将每个字符串拼接起来,中间用一个特殊字符进行分开,同时将每个字符串的长度编码转换成字符串加入到输出string中。
\item 由于可能string中可能会包含不属于ASCII的字符,因此特殊字符可以用backslah或者字符串的末尾结束符。
\end{itemize}