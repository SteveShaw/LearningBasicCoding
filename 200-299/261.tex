\section{261 --- Graph Valid Tree}
Given $n$ nodes labeled from 0 to $n - 1$ and a list of undirected edges (each edge is a pair of nodes), write a function to check whether these edges make up a valid tree.
\paragraph{Example 1:}
\begin{flushleft}
\textbf{Input}: $n=5$
\begin{figure}[H]
\begin{tikzpicture}
[mynode/.style={draw,circle,minimum size=5mm, fill=gray!20!}]
\node(){};
\node[mynode](A) {0};
\node[mynode](B)[below=8mm of A, xshift=-12mm] {1};
\node[mynode](C)[below=8mm of A] {2};
\node[mynode](D)[below=8mm of A, xshift=12mm] {3};
\node[mynode](E)[below=8mm of B, xshift=5mm] {4};
\draw (A) -- (B);
\draw (A) -- (C);
\draw (A) -- (D);
\draw (B) -- (E);
\end{tikzpicture}
\end{figure}
\textbf{Output}: \texttt{true}
\end{flushleft}

\paragraph{Example 2:}
\begin{flushleft}
\textbf{Input}: $n=5$
\begin{figure}[H]
\begin{tikzpicture}
[mynode/.style={draw,circle,minimum size=5mm, fill=gray!20!}]
\node(){};
\node[mynode](A) {0};
\node[mynode](B)[below=8mm of A, xshift=-12mm] {1};
\node[mynode](C)[right=8mm of B] {4};
\node[mynode](D)[below=8mm of B, xshift=-10mm] {2};
\node[mynode](E)[below=8mm of B, xshift=10mm] {3};
\draw (A) -- (B);
\draw (B) -- (C);
\draw (B) -- (D);
\draw (B) -- (E);
\end{tikzpicture}
\end{figure}
\textbf{Output}: \texttt{false}
\end{flushleft}
\paragraph{Hint:}
\begin{itemize}
\item Given $n = 5$ and a graph as below:
\begin{figure}[H]
\centering
\begin{tikzpicture}
[mynode/.style={draw,circle,minimum size=5mm, fill=gray!20!}]
\node(){};
\node[mynode](A) {0};
\node[mynode](B)[below=8mm of A, xshift=-12mm] {1};
\node[mynode](C)[right=8mm of B] {2};
\node[mynode](D)[right=12mm of A] {3};
\node[mynode](E)[below=8mm of D] {4};
\draw (A) -- (B);
\draw (B) -- (C);
\draw (D) -- (E);
\end{tikzpicture}
\end{figure}
what should your return? Is this case a valid tree?
\item According to the definition of tree on Wikipedia: 
\begin{quote}
a tree is an undirected graph in which any two vertices are connected by exactly one path. 
\end{quote}
In other words, any connected graph without simple cycles is a tree.
\end{itemize}

\paragraph{Note:}
\begin{itemize}
\item you can assume that no duplicate edges will appear in edges. Since all edges are undirected, $[0, 1]$ is the same as $[1, 0]$ and thus will not appear together in edges.
\end{itemize}
\subsection{Depth First Search}
按照定义,Tree其实就是没有cycle的undirected graph。
\begin{itemize}
\item 由于是undirected,因此从edges建立邻接表时,两端节点就要建立邻接表。
\item 然后从第一个节点开始进行深度搜索,用一个颜色标注与其相邻的所有节点。
\item 如果在深度搜索过程中发现某个不是搜索开始的节点已经被标注了相同的颜色,就表示存在cycle。
\end{itemize}
\setcounter{lstlisting}{0}
\begin{lstlisting}[style=customc, caption={DFS}]
bool validTree( int n, vector<pair<int, int>>& edges )
{
    //graph based on adjacent nodes
    vector<vector<int>> g( n, vector<int>() );
    //colors
    vector<int> colors( n, false );

    //create g from edges
    //This is a undirected graph so
    //we have to do for both sides
    for( const auto& e : edges )
    {
        g[e.first].push_back( e.second );
        g[e.second].push_back( e.first );
    }

    if( !dfs( g, v, 0, -1 ) )
    {
        //There exits a cycle
        return false;
    }

    for( int color  : colors )
    {
        //The graph is disjointed
        if( color == 0 )
        {
            return false;
        }
    }
    return true;
}
bool dfs( int start, int parent, vector<vector<int>> &g, vector<int> &colors )
{
    if( colors[start] == 1 )
    {
        //We have visited start before
        //so this is a cycle
        return false;
    }

    //set start node's color to 1
    colors[start] = 1;

    for( int next : g[start] )
    {
        //only check non-parent nodes
        if( next != parent )
        {
            if( !dfs( next, start, g, v ) )
            {
                return false;
            }
        }
    }
    return true;
}

\end{lstlisting}
\subsection{Breath First Search}
\begin{itemize}
\item 需要用queue来辅助遍历,
\item 用一个hash set来记录遍历到的节点。
\item 从节点0开始,将0加入到q中,然后加入到hash set里。开始处理queue。
\item 如果遍历到一个节点,在set中没有,则加入set,如果已经存在,则返回\texttt{false}。
\item 另外在遍历节点$i$的邻接链表的时候,与$i$相连的node在处理完后要把$i$从其邻接表中删除,这样在以后的递归中,如果没有cycle,就不会再返回处理$i$了。
\end{itemize}
\begin{lstlisting}[style=customc, caption={BFS}]
bool validTree( int n, vector<pair<int, int>>& edges )
{
    vector<unordered_set<int>> g( n );

    unordered_set<int> colors;

    //build g from edges
    for( const auto& e : edges )
    {
        g[e.first].insert( e.second );
        g[e.second].insert( e.first );
    }

    queue<int> q;
    //add 0 to queue
    q.push( 0 );
    //color 0
    colors.insert( 0 );

    while( !q.empty() )
    {
        int node = q.front();
        q.pop();

        for( int next : g[node] )
        {
            //There is a cycle
            if( colors.find( next ) != colors.end() )
            {
                return false;
            }

            colors.insert( next );
            q.push( next );

            auto it = g[next].find( node );

            if( it != g[next].end() )
            {
                //remove node from next's adjacent node list
                //to avoid repeat searching
                g[next].erase( it );
            }
        }

    }

    int nodes = static_cast< int >( colors.size() );

    //only when we color all nodes
    //otherwise the graph is disjointed
    return nodes == n;

}
\end{lstlisting}
\subsection{Union}
\begin{itemize}
\item 初始化一个size为$n$的parents数组,其中$\texttt{parents}[i]=i$,即每个节点的parent就是其自身
\item 遍历edges,分别查找edges两边的node的parent,如果发现这两个node的parent相同,表明存在cycle,直接返回\texttt{false}。
\item 如果不相同,将其中一个node的parent设为另外一个node。
\item 最后遍历parents数组,看数组中每个元素的parent是不是都相同,如果有不同的,表明graph是disjointed,返回\texttt{false}。
\end{itemize}
\begin{lstlisting}[style=customc, caption={Union Find}]
bool validTree( int n, vector<pair<int, int>>& edges )
{
    vector<int> parents( n, 0 );
    for( int i = 0; i < n; ++i )
    {
        parents[i] = i;
    }

    for( const auto& e : edges )
    {
        int p1 = find_parent( parents, e.first );
        int p2 = find_parent( parents, e.second );

        //There is a cycle
        if( p1 == p2 )
        {
            return false;
        }

        parents[p2] = p1;
    }

    //check if there is more than 1 parents
    //more than 1 parents means the graph is disjointed
    int p0 = find_parent( parents, 0 );

    for( int i = 0; i < n; ++i )
    {
        int pi = find_parent( parents, i );

        if( pi != p0 )
        {
            return false;
        }
    }

    return true;
}

int find_parent( vector<int>& parents, int i )
{
    while( parents[i] != i )
    {
        i = parents[i];
    }

    return i;
}
\end{lstlisting}