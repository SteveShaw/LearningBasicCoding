\section{225 --- Implement Stack using Queues}
Implement the following operations of a stack using queues.

\begin{itemize}
\item \texttt{push(x)} -- Push element x onto stack.
\item \texttt{pop()} -- Removes the element on top of the stack.
\item \texttt{top()} -- Get the top element.
\item \texttt{empty()} -- Return whether the stack is empty.
\end{itemize}
\paragraph{Example:}
\begin{lstlisting}[style=customc]
MyStack stack = new MyStack();

stack.push(1);
stack.push(2);  
stack.top();   // returns 2
stack.pop();   // returns 2
stack.empty(); // returns false
\end{lstlisting}

\paragraph{Notes:}

\begin{itemize}
\item You must use only standard operations of a queue -- which means only \texttt{push} to back,\texttt{ peek/pop} from front, \texttt{size}, and \texttt{is empty} operations are valid.
\item Depending on your language, queue may not be supported natively. You may simulate a queue by using a list or deque (double-ended queue), as long as you use only standard operations of a queue.
\item You may assume that all operations are valid (for example, no pop or top operations will be called on an empty stack).
\end{itemize}
\subsection{One queue}
\begin{CJK*}{UTF8}{gbsn}
这题一般的思路是用两个queue,然后当需要进行pop时,就将其中一个queue中的元素弹出放入另外一个queue。这无形中增加了memory的需求。其实这题可以借助一个queue来完成,即每次加入一个元素后,检查当前queue的size是否大于1。 如果大于1,则不停的将queue的front弹出重新加入queue中,直到处理到当前加入的元素。每次这样操作后,queue中的元素排列顺序就和stack是一样的了。
\end{CJK*}
\setcounter{lstlisting}{0}
\begin{lstlisting}[style=customc, caption={Queue}]
class MyStack
{
public:
    /** Initialize your data structure here. */
    MyStack()
    {
    }

    /** Push element x onto stack. */
    void push( int x )
    {
        q.push( x );

        auto sz = q.size();

		//make sure the loop stop at current element x
        while( sz > 1 )
        {
            int t = q.front();
            q.pop();
            q.push( t );

            --sz;
        }
    }

    /** Removes the element on top of the stack and returns that element. */
    int pop()
    {
        int t = q.front();
        q.pop();

        return t;
    }

    /** Get the top element. */
    int top()
    {
        return q.front();
    }

    /** Returns whether the stack is empty. */
    bool empty()
    {
        return q.empty();
    }

    queue<int> q;
};


\end{lstlisting}