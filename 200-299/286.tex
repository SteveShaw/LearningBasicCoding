\section{285 --- Walls and Gates}
You are given a $m \times n$ 2D grid initialized with these three possible values.
\begin{itemize}
\item $-1$  ---  A wall or an obstacle.
\item $0$ --- A gate.
\item $+\infty$ --- Infinity means an empty room. We use the value $2^{31} - 1 = 2147483647$ to represent $+\infty$ as you may assume that the distance to a gate is less than 2147483647.
\end{itemize}
Fill each empty room with the distance to its nearest gate. If it is impossible to reach a gate, it should be filled with $+\infty$.
\par
For example, given the 2D grid:
\begin{table}[H]
\begin{tabular}{cccc}
$+\infty$ & -1 & 0 & $+\infty$\\
$+\infty$ & $+\infty$ & $+\infty$ & $-1$\\
$+\infty$ &  $-1$ & $+\infty$ & $-1$\\
0 & $-1$ & $+\infty$ & $+\infty$
\end{tabular}
\end{table}

After running your function, the 2D grid should be:

\begin{table}[H]
\begin{tabular}{cccc}
  3 & $-1$ &  0 &  1\\
  2 &  2 &  1 & $-1$\\
  1 & $-1$ &  2 & $-1$\\
  0 & $-1$ &  3 &  4
\end{tabular}
\end{table}
\subsection{Depth First Search}
类似于求Distance Map的问题,
\begin{itemize}
\item 首先遍历整个grid,每次遇到0,以其周围四个相邻点为起点,开始DFS遍历,并带入深度值1
\item 由于题目中的墙是$-1$,而门是$0$,因此可以直接比较当前深度值和当前位置上的值。
\item 如果遇到的值大于当前深度值,我们将位置值赋为当前深度值,并对当前点的四个相邻点开始DFS遍历,注意此时深度值需要加1
\item 这样遍历完成后,所有的位置就被正确地更新
\end{itemize}
\setcounter{lstlisting}{0}
\begin{lstlisting}[style=customc, caption={DFS}]
void wallsAndGates( vector<vector<int>>& rooms )
{
    int m = static_cast<int>( rooms.size() );
    int n = static_cast<int>( rooms[0].size() );

    for( int i = 0; i < m; ++i )
    {
        for( int j = 0; j < n; ++j )
        {
            if( rooms[i][j] == 0 )
            {
                //dist = 0
                dfs( rooms, i, j, 0 );
            }
        }
    }
}

void dfs( vector<vector<int>>& rooms, int i, int j, int dist )
{

    int m = static_cast<int>( rooms.size() );
    int n = static_cast<int>( rooms[0].size() );

    if( i < 0 || i >= m || j < 0 || j >= n || rooms[i][j] < dist )
    {
        return;
    }

    rooms[i][j] = dist;
    //recursive on four directions
    dfs( rooms, i + 1, j, dist + 1 );
    dfs( rooms, i - 1, j, dist + 1 );
    dfs( rooms, i, j + 1, dist + 1 );
    dfs( rooms, i, j - 1, dist + 1 );
}
\end{lstlisting}
\subsection{Breadth First Search}
BFS方法需要借助queue,
\begin{itemize}
\item 首先把门的位置坐标都放入queue中,
\item 对于门位置的四个相邻点,首先判断其是否在grid范围内,并且距离值是否大于上一位置的距离值加1。
\item 如果满足这些条件,将当前位置的距离值设为赋为上一位置的距离值加1,并将该位置坐标放入queue中。
\item 当queue中的元素遍历完了,所有位置的值就被正确地更新了
\end{itemize}
\begin{lstlisting}[style=customc, caption={BFS}]
void wallsAndGates( vector<vector<int>>& rooms )
{
    queue<pair<int, int>> q;

    //The four direction offsets
    vector<vector<int>> dirs{{0, -1}, {-1, 0}, {0, 1}, {1, 0}};

    int m = static_cast<int>( rooms.size() );
    int n = static_cast<int>( rooms[0].size() );

    for( int i = 0; i < m; ++i )
    {
        for( int j = 0; j < n; ++j )
        {
            if( rooms[i][j] == 0 )
            {
                //add gate's coordinate
                //into the queue
                q.emplace( i, j );
            }
        }
    }

    //BFS
    while( !q.empty() )
    {
        int i = q.front().first;
        int j = q.front().second;
        q.pop();

        for( const auto& d : dirs )
        {
            int x = i + d[0];
            int y = j + d[1];
            if( x < 0 || x >= m || y < 0 || y >= n || rooms[x][y] < rooms[i][j] + 1 )
            {
                continue;
            }

            //only update distance for (x,y) when
            //its value is no less than
            //distance of (i,j) plus 1
            rooms[x][y] = rooms[i][j] + 1;

            q.emplace( x, y );
        }
    }
}
\end{lstlisting}

