\section{272 --- Closest Binary Search Tree Value IIe}
Given a non-empty binary search tree and a target value $T$, find $k$ values in the BST that are closest to the target.

\paragraph{Note:}

\begin{itemize}
\item Given target value is a floating point.
\item You may assume $k$ is always valid, that is: $k \leq$ total nodes.
\item You are guaranteed to have only one unique set of $k$ values in the BST that are closest to the target.
\end{itemize}
 
\paragraph{Follow up:}
\begin{itemize}
\item Assume that the BST is balanced, could you solve it in less than $O(n)$ runtime (where $n =$ total nodes)?
\end{itemize}

\paragraph{Hint:}

\begin{enumerate}
\item Consider implement these two helper functions:
\begin{enumerate}
\item \texttt{getPredecessor}($N$), which returns the next smaller node to $N$.
\item \texttt{getSuccessor}($N$), which returns the next larger node to $N$.
\end{enumerate}
\item Try to assume that each node has a parent pointer, it makes the problem much easier.
\item Without parent pointer we just need to keep track of the path from the root to the current node using a stack.
\item You would need two stacks to track the path in finding \textbf{predecessor} and \textbf{successor} node separately.
\end{enumerate}
\subsection{In-order Traverse}
\begin{itemize}
\item In-order traverse on BST得到的是从小到大排序的序列。
\item 在in-order traverse中,如果当前结果序列的size小于$k$,直接把访问的节点的值放入结果序列$A$中。
\item 如果当前结果序列的size等于$k$了,则将当前访问的节点值与当前结果序列$A$的第一个值$A[0]$按照与$T$的差的绝对值进行比较。如果当前节点值与$T$的差的绝对值更小,说明当前节点值与$T$更接近,将$A[0]$从$A$中去除,将当前节点值加到$A$的末尾。
\item 比较$A[0]$而不是$A$其他的元素,是因为$A[0]$是$A$中最小的number。如果当前节点值比$A[0]$更靠近$T$,那么由于当前节点值比$A$中所有的number都要大,所以除了$A[0]$外的其他number肯定都要比$A[0]$更接近$T$。而我们需要得到$k$个更接近$T$的值,因此只需要移除掉$A[0]$。
\item 如果当前节点值与$T$的差的绝对值比$A[0]$的更大,这时候就不用再继续遍历了,因为后面的节点值与$T$的差的绝对值更大。
\end{itemize}
\setcounter{algorithm}{0}
\begin{algorithm}[H]
\caption{In-order Traverse}
\begin{algorithmic}[1]
\Procedure{ClosestKValues}{$R, T, k$}
\State $\star$ Create the result array $A$
\State \Call{InOrder}{$S=R, T, A, K=k$} \Comment Call function \texttt{InOrder}
\State \Return $A$
\EndProcedure
\end{algorithmic}
\end{algorithm}
Function \texttt{InOrder} traverse the BST tree start from node $S$.

\begin{algorithm}[H]
\caption{Helper Function For In-order Traverse}
\begin{algorithmic}[1]
\Procedure{InOrder}{$S, T, A, K$}
\If{$S$ is null node}
\State \Return
\EndIf
\State $\star$ Call \texttt{InOrder} on left subtree of $S$
\If{size of $A$ is less than $K$}
\State $\star$ 将$S$的节点值加到$A$的末尾
\Else
\If{$S$的节点值与$K$的差的绝对值小于$\lvert K-A[0]\rvert$}
\State $\star$ Remove $A[0]$ from $A$
\State $\star$ 将$S$的节点值加到$A$的末尾
\Else
\State \Return \Comment 后面遍历到的节点值会更偏离$T$
\algstore{272algo}
\end{algorithmic}
\end{algorithm}
\begin{algorithm}[H]
\begin{algorithmic}[1]
\algrestore{272algo}
\EndIf
\EndIf
\State $\star$ Call \texttt{InOrder} on right subtree of $S$
\EndProcedure
\end{algorithmic}
\end{algorithm}
\setcounter{lstlisting}{0}
\begin{lstlisting}[style=customc,caption={In-order Traverse}]
vector<int> closestKValues( TreeNode* root, double target, int k )
{
    vector<int> ans;
    ans.reserve( k );

    auto K = static_cast< size_t >( k );

    inorder( root, target, ans, K );

    return ans;
}

void inorder( TreeNode* node, double T, vector<int>& A, size_t k )
{
    if( !node )
    {
        return;
    }

    inorder( node->left, T, A, k );

    if( A.size() < k )
    {
        A.push_back( node->val );
    }
    else
    {
        auto a0 = static_cast< double >( A[0] );
        auto dv = static_cast< double >( node->val );


        if( abs( a0 - T ) > abs( dv - T ) )
        {
            //node-val is closer to T
            //Remove A[0]
            A.erase( A.begin() );
            //Add node->val
            A.push_back( node->val );
        }
        else
        {
            //We cannot find value that is closer to T
            //from now on
            return;
        }
    }

    inorder( node->right, T, A, k );
}
\end{lstlisting}
\subsection{Based On Hints}
\begin{itemize}
\item Maintian two stacks: $P$ and $S$。
\item At start, 根据BST的特性,将节点值小于$T$的节点push到$P$中,将节点值大于或者等于$T$的节点push到$S$中。
\item 然后循环$k$次,每次都比较$P$和$S$的栈顶node的value哪个更close to $T$。
\item 如果$P$的栈顶node的value更close to $T$,将这个栈顶node的value输出到结果序列$A$中并从$P$中删除该node。接下来借助函数\texttt{getPredecessor}寻找the next smaller node to this node。在BST中,这个next \textbf{smaller} node是该节点的\textbf{左子树的最右节点}如果该节点的左子树存在。在寻找的过程中,把路径上的所有节点都push到$P$中。
\item 如果$S$的栈顶node的value更close to $T$,将这个栈顶node的value输出到结果序列$A$中并从$S$中删除该node。接下来借助函数\texttt{getSuccessor}寻找the next \textbf{larger} node to this node。在BST中,这个next \textbf{larger} node是该节点的\textbf{右子树的最左节点}如果该节点的右子树存在。在寻找的过程中,把路径上的所有节点都push到$S$中。
\end{itemize}
\begin{algorithm}[H]
\caption{Find Next Larger And Next Smaller Node}
\begin{algorithmic}[1]
\Procedure{ClosestKValues}{$R, T, k$}
\State $\star$ Create two stacks $P$ and $S$
\State $\ast$ Push nodes whose values less than $T$ to $P$. Otherwise, push to $S$
\State $x:=R$
\While{$x$ is not null node}
\If{value of $x$ is less than $T$}
\State $\star$ Push $x$ to $P$
\State $\star$ Update $x$ to its left child node
\Else
\State $\star$ Push $x$ to $S$
\State $\star$ Update $x$ to its right child node
\EndIf
\EndWhile
\State $\delta:=0$ \Comment The counter of loops
\State $\star$ Create an array $A$ as the result array
\While{$\delta < k$}
\If{$S$ is empty \textbf{or} top node's value of $P$ is closer to $T$}
\State $\star$ Add $P$'s top node's value to $A$
\State $\star$ Call \texttt{GetPredecessor} to find next smaller nodes to $P$'s top node
\Else
\State $\star$ Add $S$'s top node's value to $A$
\State $\star$ Call \texttt{GetSuccessor} to find next larger nodes to $S$'s top node
\EndIf
\State $\delta\gets\delta+1$ \Comment Increments the counter of loops
\EndWhile
\State \Return $A$
\EndProcedure
\end{algorithmic}
\end{algorithm}
Function \texttt{GetPredecessor}寻找the next smaller node to $P$'s top node。同时把路径上的节点都push到$P$中。
\begin{algorithm}[H]
\caption{Get Next Smaller Node}
\begin{algorithmic}[1]
\Procedure{GetPredecessor}{$P$}
\State $\star$ Set $t$ as $P$'s top node
\State $\star$ Pop $P$'s top node
\If{$t$'s left child node is not null}
\State $\star$ Push $t$'s \textbf{left} child node to $P$
\State $\star$ Update $t$ as its left child node
\While{$t$'s \textbf{right} child node is not null}
\State $\star$ Push $t$'s \textbf{right} child node to $P$
\State $\star$ Update $t$ as its \textbf{right} child node
\EndWhile
\EndIf
\EndProcedure
\end{algorithmic}
\end{algorithm}
Function \texttt{GetSuccessor}寻找the next larger node to $S$'s top node。同时把路径上的节点都push到$S$中。

\begin{algorithm}[H]
\caption{Get Next Larger Node}
\begin{algorithmic}[1]
\Procedure{GetSuccessor}{$S$}
\State $\star$ Set $t$ as $S$'s top node
\State $\star$ Pop $S$'s top node
\If{$t$'s right child node is not null}
\State $\star$ Push $t$'s \textbf{right} child node to $S$
\State $\star$ Update $t$ as its right child node
\While{$t$'s \textbf{left} child node is not null}
\State $\star$ Push $t$'s \textbf{left} child node to $S$
\State $\star$ Update $t$ as its \textbf{left} child node
\EndWhile
\EndIf
\EndProcedure
\end{algorithmic}
\end{algorithm}
\begin{lstlisting}[style=customc, caption={Find Next Larger And Next Smaller Node}]
vector<int> closestKValues( TreeNode* root, double target, int k )
{
    stack<TreeNode*> P;
    stack<TreeNode*> S;

    auto node = root;

    while( node )
    {
        auto dv = static_cast< double >( node->val );

        if( dv < target )
        {
            P.push( node );
            node = node->right;
        }
        else
        {
            S.push( node );
            node = node->left;
        }

    }

    int loops = 0;
    vector<int> ans;
    ans.reserve( k );

    while( loops < k )
    {
        double dp = P.empty() ? 0 : abs( static_cast< double >( P.top()->val ) - target );
        double ds = S.empty() ? 0 : abs( static_cast< double >( S.top()->val ) - target );

        if( S.empty() )
        {
            ans.push_back( P.top()->val );
            getPredecessor( P );
        }
        else if( !P.empty() && ( dp < ds ) )
        {
			//P.top() is closer to T
            ans.push_back( P.top()->val );
            getPredecessor( P );
        }
        else //S is not empty or P is empty or dp >= ds
        {
            ans.push_back( S.top()->val );
            getSuccessor( S );
        }

        ++loops;
    }

    return ans;
}
//Get next smaller node to P.top()
void getPredecessor( stack<TreeNode*>& P )
{
    auto t = P.top();
    P.pop();

    if( t->left )
    {
        P.push( t->left );
        t = t->left;
        while( t->right )
        {
            P.push( t->right );
            t = t->right;
        }
    }
}
//Get next larger node to S.top()
void getSuccessor( stack<TreeNode*>& S )
{
    auto t = S.top();
    S.pop();

    if( t->right )
    {
        S.push( t->right );

        t = t->right;
        while( t->left )
        {
            S.push( t->left );
            t = t->left;
        }

    }
}
\end{lstlisting}