\section{275 --- H-Index II}
Given an array of citations \textbf{sorted in ascending order} (each citation is a non-negative integer) of a researcher, write a function to compute the researcher's h-index.
\par
According to the definition of \textbf{h-index} on Wikipedia:
\begin{quote}
 A scientist has index $h$ if $h$ of his/her $N$ papers have at least h citations each, and the other $N - h$ papers have no more than $h$ citations each.
\end{quote}

\paragraph{Example:}

\begin{flushleft}
\textbf{Input}: $A = [0,1,3,5,6]$
\\
\textbf{Output}: 3 
\\
\textbf{Explanation}: 
\\
$[0,1,3,5,6]$ means the researcher has 5 papers in total and each of them had received 0, 1, 3, 5, 6 citations respectively. 
\\
Since the researcher has 3 papers with \textbf{at least} 3 citations each and the remaining two with \textbf{no more than} 3 citations each, her \textbf{h-index} is 3.
\end{flushleft}
\paragraph{Note:} 
\begin{itemize}
\item If there are several possible values for $h$, the maximum one is taken as the \textbf{h-index}
\end{itemize}

\paragraph{Follow up:}

\begin{itemize}
\item This is a follow up problem to \textbf{274 --- H-Index}, where citations is now guaranteed to be sorted in ascending order.
\item Could you solve it in \textbf{logarithmic} time complexity?
\end{itemize}
\subsection{Binary Search}
\begin{itemize}
\item 由于数组是排序的,因此binary search找到第一个index $i$,使得$A[i]\geq L -i $,其中$L$是$A$的长度。由于$A[i]$到$A[L-1]$有$L-i$个数,因此只要这些数都大于或者等于$L-i$,就找到了符合要求的H-index。
\item 找到$i$后,$L-i$就是H-index。
\item 很显然这就是leftmost binary search
\end{itemize}
\setcounter{lstlisting}{0}
\begin{lstlisting}[style=customc, caption={Leftmost Binary Search}]
int hIndex( vector<int>& citations )
{
    if( citations.empty() )
    {
        return 0;
    }

    int L = static_cast<int>( citations.size() );

    int l = 0;
    int r = L;

    //leftmost binary search
    //find first index i where A[i]>=L-i
    while( l < r )
    {
        int mid = ( l + r ) / 2;
        if( citations[mid] < L - mid )
        {
            l = mid + 1;
        }
        else
        {
            r = mid;
        }
    }

    //L-i is the H-index
    return L - l;
}
\end{lstlisting}