\section{297 --- Serialize and Deserialize Binary Tree}
Serialization is the process of converting a data structure or object into a sequence of bits so that it can be stored in a file or memory buffer, or transmitted across a network connection link to be reconstructed later in the same or another computer environment.
\par
Design an algorithm to \textbf{serialize} and \textbf{deserialize} a binary tree. There is no restriction on how your serialization/deserialization algorithm should work. You just need to ensure that a binary tree can be serialized to a string and this string can be deserialized to the original tree structure.

\paragraph{Example: }

\begin{flushleft}
You may serialize the following tree:
\begin{figure}[H]
\begin{tikzpicture}
[mynode/.style={draw,circle,minimum size=5mm, fill=gray!20!}]
\node(){};
\node[mynode](a) {1};
\node[mynode](b)[below=8mm of a, xshift=-8mm] {2};
\node[mynode](c)[below=8mm of a, xshift=8mm] {3};
\node[mynode](d)[below=8mm of c, xshift=-6mm] {4};
\node[mynode](e)[below=8mm of c, xshift=6mm] {5};
\draw (a) -- (b);
\draw (a) -- (c);
\draw (c) -- (d);
\draw (c) -- (e);
\end{tikzpicture}
\end{figure}
as \fcj{[1,2,3,null,null,4,5]}
\end{flushleft}

\paragraph{Clarification:}
\begin{flushleft}
The above format is the same as how \textbf{LeetCode} serializes a binary tree. You do not necessarily need to follow this format, so please be creative and come up with different approaches yourself.
\end{flushleft}

\subsection{Preorder}
In serialization, we visit the tree with preorder sequence. Using \fcj{null} to represent the empty node, and separate each node with a comma.

In deserializaton, we splice the given string, and then initiate the node value, and then calls itself to construct its left and right child nodes.


