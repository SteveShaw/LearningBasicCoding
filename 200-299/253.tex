\section{253 --- Meeting Rooms II}
Given an array of meeting time intervals $A$ consisting of start and end times $[s_1,\;e_1],\;[s_2,\;e_2],\;\ldots$ $(s_i < e_i)$, find the minimum number of conference rooms required.


\paragraph{Example 1:}

\begin{flushleft}
\textbf{Input}: \fcj{[[0, 30],[5, 10],[15, 20]]}

\textbf{Output}: 2

\end{flushleft}

\paragraph{Example 2:}

\begin{flushleft}
\textbf{Input}: \fcj{[[7,10],[2,4]]}

\textbf{Output}: 1
\end{flushleft}

\subsection{Tree Map}
Iterate the input arrays, For start time, increment the value, and for end time, decrement the value. 

Then, iterate over the map, get the sum of all keys' values. The maximum value during this process will be the maximum number of overlapped intervals.

\setcounter{lstlisting}{0}
\begin{lstlisting}[style=customc, caption={Tree Map}]
int minMeetingRooms( vector<Interval>& intervals )
{
    map<int, int> m;

    for( auto& itv : intervals )
    {
        //insert start
        auto it = m.find( itv.start );
        if( it != m.end() )
        {
            ++it->second;
        }
        else
        {
            m.emplace( itv.start, 1 );
        }

        //insert end
        it = m.find( itv.end );
        if( it != m.end() )
        {
            --it->second;
        }
        else
        {
            m.emplace( itv.end, -1 );
        }
    }

    //iterate over m
    int delta = 0;
    int x = 0;
    for( const auto& p : m )
    {
        //p.second is the times of each start and end
        delta += p.second;
        x = ( max )( x, delta );
    }
    return x;
}
\end{lstlisting}

\subsection{Two Arrays}

If we separate the starting and ending time, and treat them individually, then the identity of a meeting goes away. This is fine because:

When we encounter an ending event, that means that some meeting that started earlier has ended now. We are not really concerned with which meeting has ended. All we need is that some meeting ended thus making a room available.

Using two arrays say $X$ and $Y$. $X$ is used to record sorted start times while $Y$ for end times. 

When iterating over both $X$ and $Y$. If current start time is earlier than current end time, increment the result, and jump to next start time in $X$. Otherwise, we jump to next end time in $Y$ and next start time in $X$.

\begin{lstlisting}[style=customc, caption={Two Arrays}]
int minMeetingRooms( vector<vector<int>>& intervals )
{
    vector<int> starts( intervals.size() );
    vector<int> ends( intervals.size() );
    size_t i = 0;
    for( const auto& x : intervals )
    {
        starts[i] = x[0];
        ends[i] = x[1];
        ++i;
    }
    sort( begin( starts ), end( starts ) );
    sort( begin( ends ), end( ends ) );
    size_t ix = 0ull;
    size_t iy = 0ull;
    int ans = 0;
    auto N = intervals.size();
    while( ( ix < N ) && ( iy < N ) )
    {
        if( starts[ix] < ends[iy] )
        {
            //overlapped intervals
            ++ans;
            //only jump to next in starts
            ++ix;
        }
        else
        {
            //we neet to jump to next
            //in starts and ends
            ++iy;
            ++ix;
        }
    }
    return ans;
}
\end{lstlisting}

\subsection{Priority Queue}

we can keep all the rooms in a \textbf{min heap} where the key for the min heap would be the ending time of meeting.

So, every time we want to check if any room is free or not, simply check the topmost element of the \textbf{min heap} as that would be the room that would get free the earliest out of all the other rooms currently occupied.

If the room we extracted from the top of the \textbf{heap} isn't free, then no other room is. So, we can save time here and simply allocate a new room.

After processing all the meetings, the size of the heap will tell us the number of rooms allocated. This will be the minimum number of rooms needed to accommodate all the meetings.

\begin{lstlisting}[style=customc, caption={Min Heap}]
int minMeetingRooms( vector<vector<int>>& intervals )
{
    if( intervals.empty() )
    {
        return 0;
    }
    //min heap
    priority_queue<int, vector<int>, greater<int>> pq;
    //sort intervals by start time
    sort( begin( intervals ), end( intervals ), []( const vector<int>& a1, const vector<int>& a2 )
    {
        return a1[0] < a2[0];
    } );
    pq.push( intervals[0][1] );
    for( size_t i = 1; i < intervals.size(); ++i )
    {
        if( intervals[i][0] >= pq.top() )
        {
            //we can use this room that occupied before
            pq.pop();
        }
        //either we have used previous meeting room
        //or we allocate a new one with end time as
        //intervals[i][1]
        pq.push( intervals[i][1] );
    }
    return ( int )( pq.size() );
}
\end{lstlisting}