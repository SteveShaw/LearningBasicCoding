\section{253 --- Meeting Rooms II}
Given an array of meeting time intervals $A$ consisting of start and end times $[s_1,\;e_1],\;[s_2,\;e_2],\;\ldots$ $(s_i < e_i)$, find the minimum number of conference rooms required.
\par
\begin{flushleft}
\textbf{Input}: $A =[[0, 30],\;[5, 10],\;[15, 20]]$
\\
\textbf{Output}: 2
\end{flushleft}
\subsection{Tree Map}
\begin{itemize}
\item Tree Map能够自动排序插入的数
\item 遍历$A$,同时maintain一个tree map $M$
\begin{itemize}
\item 遇到start,将其插入$M$,对应的值设为$1$,如果已经存在将其对应的value加1
。
\item 遇到end,将其拆入$M$,对应的值设为$-1$。如果这个end在$M$中存在,将其对应的value减1。
\end{itemize}
\item 然后遍历$M$,累加各个key的value,同时maintain一个计数器,如果发生overlap,那么这个计数器的值在一直增加。
\item 记录遍历过程中计数器的最大值,这个最大值就是overlap的最多的interval的个数。
\end{itemize}
\setcounter{algorithm}{0}
\begin{algorithm}[H]
\caption{Counter Of Overlapped Intervals}
\begin{algorithmic}[1]
\Procedure{MinMeetingRooms}{$A, L$}
\State $\star$ 创建一个Tree Map $M$
\For{$i:=0$ \textbf{to} L-1}
\State $\star$ 将$A[i]$的\texttt{start}插入$M$,对应的value为1,如果存在就increments其value
\State $\star$ 将$A[i]$的\texttt{end}插入$M$,对应的value为$-1$,如果存在就decrements其value
\EndFor
\State $\ast$ 初始化一个计数器$\delta$为0
\State $\ast$ 创建一个变量$x$作为记录$\delta$在遍历$M$中出现的最大值
\For{Each key $k$ in $M$}
\State $\delta\gets\delta+M[k]$
\State $x\gets\max(x, \delta)$
\EndFor
\algstore{253algo}
\end{algorithmic}
\end{algorithm}
\begin{algorithm}[H]
\begin{algorithmic}[1]
\algrestore{253algo}
\State \Return $x$
\EndProcedure
\end{algorithmic}
\end{algorithm}
\setcounter{lstlisting}{0}
\begin{lstlisting}[style=customc, caption={Tree Map}]
int minMeetingRooms( vector<Interval>& intervals )
{
    map<int, int> m;

    for( auto& itv : intervals )
    {
        //insert start
        auto it = m.find( itv.start );
        if( it != m.end() )
        {
            ++it->second;
        }
        else
        {
            m.emplace( itv.start, 1 );
        }

        //insert end
        it = m.find( itv.end );
        if( it != m.end() )
        {
            --it->second;
        }
        else
        {
            m.emplace( itv.end, -1 );
        }
    }

    //iterate over m
    int delta = 0;
    int x = 0;
    for( const auto& p : m )
    {
        //p.second is the times of each start and end
        delta += p.second;
        x = ( max )( x, delta );
    }

    return x;
}
\end{lstlisting}