\section{340 --- Longest Substring with At Most K Distinct Characters}

\textbf{Hard}

Given a string, find the length of the longest substring T that contains at most $k$ distinct characters.

\paragraph{Example 1:}

\begin{flushleft}
\textbf{Input}: \fcj{s = "eceba"}, $k = 2$

\textbf{Output}: 3

Explanation: T is \fcj{"ece"} which its length is 3.
\end{flushleft}

\paragraph{Example 2:}

\begin{flushleft}
\textbf{Input}: \fcj{s = "aa"}, $k = 1$

\textbf{Output}: 2

\textbf{Explanation}: T is \fcj{"aa"} which its length is 2.
\end{flushleft}

\subsection{Sliding Window}
The idea is to set both pointers in the position 0 and then move right pointer to the right while the window contains not more than $k$ distinct characters. 

If at some point we've got $k + 1$ distinct characters, we move left pointer to keep not more than $k + 1$ distinct characters in the window.

When the input string contains $n$ distinct characters, at each step one uses $O(k)$ time to find a minimum value in the hashmap with $k$ elements and so the overall time complexity is $O(Nk)$.

\setcounter{lstlisting}{0}
\begin{lstlisting}[style=customc, caption={Sliding Window With Hash Map}]
int lengthOfLongestSubstringKDistinct( string s, int k )
{
    if( s.empty() || ( k == 0 ) )
    {
        return 0;
    }
    size_t left = 0;
    unordered_map<char, int> fly;
    int ans = 0;
    for( size_t right = 0; right < s.size(); ++right )
    {
        fly[s[right]] += 1;
        int sz = ( int )fly.size();
        while( ( left < right ) && ( sz > k ) )
        {
            //shrink the window to make sure
            //no larger than k distinct letters
            //are there
            auto it = fly.find( s[left] );
            if( it->second == 1 )
            {
                //s[left] is completely removed
                //so erase it from fly
                fly.erase( s[left] );
            }
            else
            {
                --it->second;
            }
            sz = ( int )fly.size();
            ++left;
        }
        ans = ( max )( ans, ( int )right - ( int )left + 1 );
    }
    return ans;
}
\end{lstlisting}

