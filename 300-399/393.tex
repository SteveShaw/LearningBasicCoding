\section{393 --- UTF-8 Validation}
A character in UTF8 can be from 1 to 4 bytes long, subjected to the following rules:
\begin{enumerate}
\item For 1-byte character, the first bit is a 0, followed by its unicode code.
\item For $ n $-bytes character, the first $ n $-bits are all one's, the $ n+1 $ bit is 0, followed by $ n-1 $ bytes with most significant 2 bits being 10.
\end{enumerate}
This is how the UTF-8 encoding would work:
\begin{table}[H]
\begin{tabular}{ll}
number range (hexadecimal) &  UTF-8 octet sequence (binary) \\
\hline
0000 0000 --- 0000 007F & 0xxxxxxx \\
0000 0080 --- 0000 007F & 110xxxxx 10xxxxxx \\
0000 0800 --- 0000 FFFF & 1110xxxx 10xxxxxx 10xxxxxx \\
0001 0000 --- 0010 FFFF & 11110xxx 10xxxxxx 10xxxxxx 10xxxxxx
\end{tabular}
\end{table}
Given an array of integers $A$ representing the data, return whether it is a valid \textbf{UTF}-8 encoding.

\paragraph{Note:}
\begin{itemize}
\item The input is an array of integers. Only the least significant 8 bits of each integer is used to store the data. This means each integer represents only 1 byte of data.
\end{itemize}

\paragraph{Example 1:}

\begin{flushleft}
\textbf{Input}: $A = [197, 130, 1]$, which represents the octet sequence: 11000101 10000010 00000001.
\\
\textbf{Output}: \texttt{true}.
\\
\textbf{Explaination}: It is a valid \textbf{UTF}-8 encoding for a 2-bytes character followed by a 1-byte character.
\end{flushleft}

\paragraph{Example 2:}

\begin{flushleft}
\textbf{Input}: $A = [235, 140, 4]$, which represented the octet sequence: 11101011 10001100 00000100.
\\
\textbf{Output}: \texttt{false}.
\\
\textbf{Explaination}:
\begin{itemize}
\item The first 3 bits are all one's and the 4th bit is 0 means it is a 3-bytes character.
\item The next byte is a continuation byte which starts with 10 and that's correct.
\item But the second continuation byte does not start with 10, so it is invalid.
\end{itemize}

\end{flushleft}