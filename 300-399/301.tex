\section{301 --- Remove Invalid Parentheses}
Remove the minimum number of invalid parentheses in order to make the input string valid. Return all possible results.
\par
\textbf{Note:} The input string may contain letters other than the parentheses ( and ).

\paragraph{Example 1:}

\begin{flushleft}
\textbf{Input}: $()())()$
\\
\textbf{Output}: [$()()()$, $(())()$]
\\
\end{flushleft}

\paragraph{Example 2:}

\begin{flushleft}
\textbf{Input}: $(a)())()$
\\
\textbf{Output}: [$(a)()()$, $(a())()$]
\end{flushleft}

\paragraph{Example 3:}


\begin{flushleft}
\textbf{Input}: )(
\\
\textbf{Output}: []
\end{flushleft}
\subsection{Backtracking}
\begin{itemize}
	\item 对每一个括号,都有两个选择,加入或者不加入最终的expression。
	\item 由于每个选择都要进行测试,因此可以通过递归的方法生成。
	\item 递归的状态由当前$S$中的index $p$来确定,同时递归的状态中加入$x$和$y$分别代表已经加入expression的左括号和右括号的个数。同时,递归的状态中还包括从$S$中不加入expression的字符的counter $z$。
	\item 如果$S[p]$不是左括号或者右括号,直接加入到生成的expresssion。
	\item 如果$S[p]$是左括号或者右括号,那么如上所述,可以把$S[p]$加入或者不加入到expression中。
	\begin{enumerate}
		\item 首先选择不加入,然后继续递归,这时候$x$和$y$都不变,但是记录不加入expression的字符个数的counter $z$ increments。
		\item 然后选择加入。
		\begin{itemize}
			\item 如果$S[p]$是左括号,$x$ increments,其余状态都不变
			\item 如果$S[p]$是右括号,这时候要判断$y$和$x$的关系,如果$y>x$,即右括号个数大于左括号个数,这时候就不用继续递归了,因为这时候一定不是valid expression。否则,继续递归,$y$ increments,其余状态不变。
		\end{itemize}
	\end{enumerate}
\item 当$p$equal to $S$的length时,如果$x$和$y$相等,表示expression是valid string。但这时候需要比较$z$是否比之前获得的最少的count of characters removed,如果相等,加入最终的结果中,如果$z$还要小,那么$z$最新的global minimum count,这时候要把结果清空,然后加入当前的expression。
\item 为了避免重复的string,用一个hash set来存放expression。
\end{itemize}
\begin{algorithm}[H]
	\caption{Recursion}
	\begin{algorithmic}[1]
		\Procedure{RemoveInvalidParentheses}{$S, L$}
		\State $\star$ Create an empty hash set $A$ to store the valid expressions
		\State $\star$ Create an empty string $E$ to store current valid expression
		\State $z_0:=0$ \Comment Global minimum count of characters removed from $S$
		\State \Call{DFS}{$S, L, p = 0, x = 0, y = 0, E, A, 0, \hat{z}=z_0$}
		\EndProcedure
	\end{algorithmic}
\end{algorithm}
Function \texttt{DFS}递归生成所有valid parentheses expressions。递归的状态包括
\begin{itemize}
	\item $p$ --- Current index in $S$
	\item $x$, $y$ --- The count of left parentheses and right parentheses
	\item $E$ --- 当前生成的valid expression
	\item $A$ --- 保存所有生成的具有最小count of removed characters的valid expression的hash set
\end{itemize}
\begin{algorithm}[H]
	\caption{Recursion Helper Function}
	\begin{algorithmic}[1]
		\Function{DFS}{$S,L,p,x,y,E,A,z,\hat{z}$}
		\State $\ast$ 递归结束的处理
		\If{$p=L$}
		\If{$x=y$} \Comment 只有当前左括号个数与右括号个数相等时,才是合法的expression
		\If{$z = \hat{z}$} \Comment Removed character个数与当前的global
		 minimum相等
		\algstore{301algo}
		\end{algorithmic}
	\end{algorithm}
\begin{algorithm}[H]
	\begin{algorithmic}[1]
		\algrestore{301algo}
				 \State $\star$ 把$E$加入$A$中
		\ElsIf{$z<\hat{z}$} \Comment Removed character个数比当前的global minmum还要少
		\State $\star$ Clear $A$ 因为$A$中的所有expression比起$E$来更短
		\State $\hat{z}\gets z$ \Comment Update Global minimum count
		\State $\star$ 把$E$加入$A$中
		\EndIf
		\EndIf
		\State \Return 
		\EndIf
		
				\If{$S[p]$既不是左括号也不是右括号}
		\State $\star$ Add $S[p]$ to $E$
		\State $\ast$ 继续下一层递归
		
				\State \Call{DFS}{$S, L, p+1, x, y, E, A, z, \hat{z}$}
		\State $\star$ 上述递归结束后,把$S[p]$从$E$中移除
		
				\Else
		\State $\ast$ 首先选择不把$S[p]$加入$E$中,然后继续下一层递归
		\State \Call{DFS}{$S, L, p+1, x, y, E, A, z+1, \hat{z}$} \Comment 这时候$z$要increments
		\State $\star$ $S[p]$加入$E$中
		\If{$S[p]$是左括号}
		\State $\ast$ 继续下一层递归,这时候$E$中多了一个左括号,因此$x\gets x+1$
		\State \Call{DFS}{$S, L, p+1, x+1, y, E, A, z, \hat{z}$}
		\ElsIf{$x>y$} \Comment 右括号情况下,只有左括号的个数大于右括号个数才能加入这个右括号
		\State $\ast$ 继续下一层递归,这时候$E$中多了一个右括号,因此$y\gets y+1$
		\State \Call{DFS}{$S, L, p+1, x, y+1, E, A, z, \hat{z}$}
		\EndIf
		\State $\star$ 结束后,把$S[p]$从$E$中移除
		\EndIf
		\EndFunction
	\end{algorithmic}
\end{algorithm}
\setcounter{lstlisting}{0}
\begin{lstlisting}[style=customc, caption={Recursive Method 1}]
vector<string> removeInvalidParentheses( string s )
{

    //record the expressions generated during recursion
    unordered_set<string> A;

    string E;

    E.reserve( s.size() );

    int z0 = INT_MAX;

    DFS( s, 0, 0, 0, E, A, 0, z0 );

    return {A.begin(), A.end()};
}

void DFS( const string& S, size_t p, int x, int y, string& expr, unordered_set<string>& A, int z, int& z0 )
{
    if( p == S.size() )
    {
        if( x == y )
        {
            if( z == z0 )
            {
                //current count of removed characters
                //is equal to the global minimum characters
                A.emplace( expr );
            }
            else if( z < z0 )
            {
                //current count of removed characters
                //is less than the global minimum characters
                A.clear();
                A.emplace( expr );

                z0 = z;
            }
        }

        return;
    }

    if( ( S[p] != '(' ) && ( S[p] != ')' ) )
    {
        //Other letters
        //Just add to the expression
        expr.push_back( S[p] );
        DFS( S, p + 1, x, y, expr, A, z, z0 );
        expr.pop_back();
    }
    else
    {
        //First choosing not to add to the expression
        DFS( S, p + 1, x, y, expr, A, z + 1, z0 );

        //Then choose adding to the expression
        expr.push_back( S[p] );

        if( S[p] == '(' )
        {
            //add left parenthesis
            DFS( S, p + 1, x + 1, y, expr, A, z, z0 );
        }
        else if( y < x )
        {
            //Only add right parenthesis when there are fewer right parenthesis than left parenthesis
            //in the expression
            DFS( S, p + 1, x, y + 1, expr, A, z, z0 );
        }

        expr.pop_back();
    }
}
\end{lstlisting}
\subsection{Optimized Recursion}
\begin{itemize}
\item 上述算法中,可以进行更多的prunning。方法是在上述递归的状态中,加入额外两个状态,即number of left misplaced parentheses and number of misplaced right parenthese removed from $S$ to get a valid expression。分别用$\alpha$和$\beta$表示。
\item 为了能够得到上述的额外两个状态的初始值,需要preprocess $S$。方法如下
\begin{itemize}
\item 从左到右扫描$S$
\item 如果遇到一个左括号,这个括号可能会包含在一个valid expression中,取决于在其右边是否存在匹配的右括号。但是这时候我们并不知道,因此just increments 左括号的counter $\alpha$。
\item 如果遇到一个右括号,如果当前左括号个数$\alpha$为零,显然这个右括号不能被匹配,因此increments misplaced的右括号的counter $\beta$。如果$\alpha$不为零,这个右括号可以被之前的一个左括号匹配,而这时候misplaced left parenthese counter $\alpha$就要decrement。
\end{itemize}
\item 由于$\alpha$和$\beta$告诉我们需要remove多少parenthesis,算法需要进行修改一边利用这两个状态信息。
\begin{enumerate}
\item 当前$p$ equal to $S$的length时, 这时候expression is valid的条件是$\alpha=0$ and $\beta=0$,不再需要判定$x$和$y$是否相等了。
\item 当选择不加入当前的左括号或者右括号时,需要检查$\alpha$或者$\beta$是否为零,如果是零,表示不能再remove了,因此也就不能继续进行递归了。
\end{enumerate}
\end{itemize}
\begin{algorithm}[H]
	\caption{Recursion Method 2: More Efficient}
	\begin{algorithmic}[1]
		\Procedure{RemoveInvalidParentheses}{$S, L$}
		\State $\star$ Create an empty hash set $A$ to store the valid expressions
		\State $\star$ Create an empty string $E$ to store current valid expression
		\State $\ast$ Preprocess $S$ to get number of misplaced left and right parentheses
				\algstore{301algo}
				\end{algorithmic}
			\end{algorithm}
		\begin{algorithm}[H]
			\begin{algorithmic}[1]
				\algrestore{301algo}
		\State $\alpha:=0$ and $\beta:=0$
		\For{Each character $c$ in $S$}
		\If{$c$ is left parenthesis}
		\State $\alpha\gets\alpha+1$
		\ElsIf{$c$ is right parenthesis}
		\If{$\alpha=0$}
		\State $\beta\gets\beta+1$ \Comment 不能再找到匹配的左括号
		\Else
		\State $\alpha\gets\alpha-1$ \Comment 有匹配的左括号,decrements $\alpha$
		\EndIf
		\EndIf
		\EndFor
		\State \Call{DFS}{$S, L, p = 0, x = 0, y = 0, \alpha, \beta, E, A$}
		\EndProcedure
	\end{algorithmic}
\end{algorithm}
Function \texttt{DFS}递归生成所有valid parentheses expressions。递归的状态包括
\begin{itemize}
	\item $p$ --- Current index in $S$
	\item $x$, $y$ --- The count of left parentheses and right parentheses
	\item $\alpha$, $\beta$ --- The misplaced number of left and right parentheses
	\item $E$ --- 当前生成的valid expression
	\item $A$ --- 保存所有生成的具有最小count of removed characters的valid expression的hash set
\end{itemize}
\begin{algorithm}[H]
	\caption{Recursion Helper Function}
	\begin{algorithmic}[1]
		\Function{DFS}{$S,L,p,x,y,E,A,z,\hat{z}$}
		\State $\ast$ 递归结束的处理
		\If{$p=L$}
		\If{$\alpha=0$ \textbf{and} $\beta=0$} \Comment No more misplaced parentheses
		\State $\star$ 把$E$加入$A$中
		\EndIf
		\State \Return
		\EndIf
		\algstore{301algo}
		\end{algorithmic}
	\end{algorithm}
\begin{algorithm}[H]
	\begin{algorithmic}[1]
		\algrestore{301algo}	
				\If{$S[p]$既不是左括号也不是右括号}
		\State $\star$ Add $S[p]$ to $E$
		\State $\ast$ 继续下一层递归
		
				\State \Call{DFS}{$S, L, p+1, x, y, \alpha, \beta, E, A$}
		\State $\star$ 上述递归结束后,把$S[p]$从$E$中移除
		
				\Else
		\State $\ast$ 首先选择不把$S[p]$加入$E$中,然后继续下一层递归
		\If{$S[p]$ is left parenthesis}
		\If{$\alpha > 0$}
		\State $\ast$只有当剩下的misplaced left parenthesis 大于零
		\State $\ast$才可以忽略当前的左括号进入下一层递归
		\State  \Call{DFS}{$S, L, p+1, x, y, \alpha-1, \beta, E, A$}
		\EndIf
		\State $\star$ $S[p]$加入$E$中
		\State $\ast$ 继续下一层递归,这时候$E$中多了一个左括号,因此$x\gets x+1$
		\State \Call{DFS}{$S, L, p+1, x+1, y, \alpha, \beta, E, A$}
		\State $\star$ 结束后,把$S[p]$从$E$中移除
		\Else
				\State $\ast$只有当剩下的misplaced right parenthesis 大于零时,
				\State $\ast$ 才可以忽略当前的右括号进入下一层递归
				\State  \Call{DFS}{$S, L, p+1, x, y, \alpha, \beta-1, E, A$}
				\EndIf
				\If{$x>y$} \Comment 右括号情况下,只有左括号的个数大于右括号个数才能加入这个右括号
					\State $\star$ $S[p]$加入$E$中
					\State $\ast$ 继续下一层递归,这时候$E$中多了一个右括号,因此$y\gets y+1$
					\State \Call{DFS}{$S, L, p+1, x, y+1, \alpha, \beta, E, A$}
					\State $\star$ 结束后,把$S[p]$从$E$中移除
				\EndIf	
		\EndIf
		\EndFunction
	\end{algorithmic}
\end{algorithm}
\begin{lstlisting}[style=customc, caption={Recursive Method 1}]
vector<string> removeInvalidParentheses( string s )
{

    unordered_set<string> s_expr;

    string expr;
    expr.reserve( s.size() );

    int alpha = 0;
    int beta = 0;

    //get misplaced left and right parentheses
    for( auto c : s )
    {
        if( c == '(' )
        {
            ++alpha;
        }
        else if( c == ')' )
        {
            if( alpha == 0 )
            {
                ++beta;
            }
            else
            {
                --alpha;
            }
        }
    }

    dfs( s, 0, 0, 0, alpha, beta, expr, s_expr );

    return {s_expr.begin(), s_expr.end()};
}

void dfs( const string& S, size_t p, int x, int y, int alpha, int beta, string& expr, unordered_set<string>& ss )
{
    if( p == S.size() )
    {
        if( ( alpha == 0 ) && ( beta == 0 ) )
        {
            ss.emplace( expr );
        }

        return;
    }

    if( ( S[p] != '(' ) && ( S[p] != ')' ) )
    {
        //Normal letter
        expr.push_back( S[p] );
        dfs( S, p + 1, x, y, alpha, beta, expr, ss );
        expr.pop_back();
    }
    else
    {
        if( S[p] == '(' )
        {
            if( alpha > 0 )
            {
                //only when remained misplaced left parentheses
                //are larger than zero, go to next recursion
                dfs( S, p + 1, x, y, alpha - 1, beta, expr, ss );
            }

            expr.push_back( S[p] );
            dfs( S, p + 1, x + 1, y, alpha, beta, expr, ss );
            expr.pop_back();
        }
        else
        {
            if( beta > 0 )
            {
                //only when remained misplaced right parentheses
                //are larger than zero, go to next recursion
                dfs( S, p + 1, x, y, alpha, beta - 1, expr, ss );
            }

            if( y < x )
            {
                //only when right parentheses is fewer than
                //left parentheses
                expr.push_back( S[p] );
                dfs( S, p + 1, x, y + 1, alpha, beta, expr, ss );
                expr.pop_back();
            }
        }
    }
}
\end{lstlisting}
\subsection{Recursive Method 3: Most Efficient}
\begin{itemize}
\item 每一次递归中,首先检测到哪个位置会出现右括号多于左括号。如果找到了这个位置,就将多余的第一个右括号移除,然后从这个移除的右括号的下一个位置继续下一层递归。
\item 如果这时候检测到左括号多于右括号,那么需要将多余的第一个左括号移除,然后从这个移除的左括号的下一个位置进行递归。但这个时候需要从右到左扫描找到第一个需要移除的左括号。
\item 为了避免从左到右和从右到左使用不同的代码,可以用一个小技巧,将$S$进行翻转,这样从右到左的扫描,就变为从左到右扫描了。
\item 但是这时候有个问题是之前的左括号和右括号的位置发生了变化,为了仍然能够得到valid expression,再得到去除第一个多余的左括号后,把生成的expression再反转一遍。
\item 为了避免在上述过程中得到duplicate expression,如果不使用hash set,可以在递归状态中包含上一次删除的括号的index。这样下一层递归就从这个index开始。
\item 可以使用一个key检查示当前是从左到右删除多余的右括号,还是在翻转字符中从左到右删除多余的左括号。前者这个key是$()$,而后者这个key则是$)($。这样通过比较这个key,就知道当前是要搜索左括号还是右括号作删除,以及当前是否需要翻转$S$。
\end{itemize}
\begin{lstlisting}[style=customc, caption={Recursive Method 3: Most Efficient}]
class Solution
{
public:
    vector<string> removeInvalidParentheses( string s )
    {

        vector<string> ans;

        dfs( s, ans, 0, 0, "()" );

        return ans;
    }

    //x: Last removed position for key[0]
    //y: Last removed position for key[1]
    void dfs( string S, vector<string>& A, size_t x, size_t y, string key )
    {
        int t = 0;

        //The target is to remove additional key[1]
        for( size_t i = x; i < S.size(); ++i )
        {
            if( S[i] == key[0] )
            {
                ++t;
            }
            else if( S[i] == key[1] )
            {
                --t;
            }

            if( t >= 0 )
            {
                //number of key[0] is larger than key[1]
                continue;
            }

            if( y > i )
            {
                //Have not yet reached the last removed position for key[1]
                //No need to further since before y, parenthesis are balanced
                return;
            }

            if( S[y] == key[1] )
            {
                //S[y] is the first index of key[1] after last removed position of key[1]
                dfs( S.substr( 0, y ) + S.substr( y + 1 ), A, i, y, key );
            }

            for( size_t j = y + 1; j <= i; ++j )
            {
                //We found another equal to key[1] candidate to be removed
                //Notice: In this loop, we are searching for candidates to be removed.
                //If there is a consequtive letters equal to key[1]
                //We only need to remove the first one from left to right
                if( ( S[j] == key[1] ) && ( S[j - 1] != key[1] ) )
                {
                    dfs( S.substr( 0, j ) + S.substr( j + 1 ), A, i, j, key );
                }
            }

            return;
        }

        //candidates equal to key[0]
        //is no less than candidiates equal to key[1]
        //We need to reverse, and
        //1. removing right parthesis when now is finding additional right parenthesis
        //2. add reversed string to output when now is finding additional left parenthesis
        reverse( S.begin(), S.end() );

        if( key[0] == '(' )
        {
            //finished left to right scan
            dfs( S, A, 0, 0, ")(" );
        }
        else
        {
            //finished right to left
            A.emplace_back( S );
        }
    }
};
\end{lstlisting}