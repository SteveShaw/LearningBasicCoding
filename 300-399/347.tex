\section{347 --- Top K Frequent Elements}
Given a non-empty array of integers $ A $, return the $ k $ most frequent elements.

\paragraph{Example 1:}

\begin{flushleft}
\textbf{Input}: $ A = [1,1,1,2,2,3] $, $ k = 2 $
\\
\textbf{Output}: $  [1,2] $
\end{flushleft}

\paragraph{Example 2:}

\begin{flushleft}
\textbf{Input}: $ A = [1] $, $ k = 1 $
\\
\textbf{Output}: $ [1] $
\end{flushleft}

\paragraph{Note:}

\begin{itemize}
\item You may assume $ k $ is always valid, $ 1 \leq k \leq $ number of unique elements.
\item Your algorithm's time complexity must be better than $O(n \log n)$, where $ n $ is the array's size.
\end{itemize}

\subsection{Hash Map And Heap}
First build the hash map for each number and its count. Then we make use of a heap to sort the count. 

\setcounter{lstlisting}{0}
\begin{lstlisting}[style=customc, caption={Heap}]
vector<int> topKFrequent( vector<int>& nums, int k )
{
    unordered_map<int, int> m;
    priority_queue<pair<int, int>, vector<pair<int, int>>, greater<pair<int, int>>> q;
    vector<int> res;
    for( auto a : nums )
    {
        ++m[a];
    }
    //add to heap
    for( auto it : m )
    {
        q.push( {it.second, it.first} );
        if( ( int )q.size() > k )
        {
            q.pop();
        }
    }
    //output first k largest
    for( int i = 0; i < k; ++i )
    {
        res.push_back( q.top().second );
        q.pop();
    }
    //notice the heap is max heap, which means
    //minimum value is at the top
    //thus we need to reverse the result
    reverse( begin( res ), end( res ) );
    return res;
}
\end{lstlisting}

\paragraph{Related Problems}
\begin{itemize}
\item \textbf{192. Word Frequency}
\item \textbf{215. Kth Largest Element in an Array}
\item \textbf{451. Sort Characters By Frequency}
\item \textbf{659. Split Array into Consecutive Subsequences}
\item \textbf{692. Top K Frequent Words}
\item \textbf{973. K Closest Points to Origin}
\end{itemize}

