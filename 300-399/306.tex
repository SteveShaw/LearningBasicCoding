\section{306 --- Additive Number}
Additive number is a string whose digits can form additive sequence.
\par
A valid additive sequence should contain at least three numbers. Except for the first two numbers, each subsequent number in the sequence must be the sum of the preceding two.
\par
Given a string containing only digits $0$---$9$, write a function to determine if it's an additive number.
\par
\paragraph{Note: }
\begin{itemize}
\item Numbers in the additive sequence cannot have leading zeros, so sequence 1, 2, 03 or 1, 02, 3 is invalid.
\end{itemize}

\paragraph{Example 1:}

\begin{flushleft}
\textbf{Input}: $112358$
\\
\textbf{Output}: \texttt{true} 
\\
\textbf{Explanation}: 
\\The digits can form an additive sequence: $1$, $1$, $ 2 $, $ 3 $, $ 5 $, $ 8 $.
\\ 
$ 1 + 1 = 2 $, $ 1 + 2 = 3 $, $ 2 + 3 = 5 $, $ 3 + 5 = 8 $
\end{flushleft}

\paragraph{Example 2:}

\begin{flushleft}
\textbf{Input}: $ 199100199 $
\\
\textbf{Output}: \texttt{true}
\\ 
\textbf{Explanation}: The additive sequence is: $ 1 $, $ 99 $, $ 100 $, $ 199 $.
\\ 
$ 1 + 99 = 100 $, $ 99 + 100 = 199 $
\end{flushleft}

\paragraph{Follow up:}
\begin{itemize}
\item How would you handle overflow for very large input integers?
\end{itemize}
\subsection{Recursion}
\begin{itemize}
\item 前两个数的长度都不能能大于输入字符串$S$的长度$L$的一半。
\item 所以开始需要两重循环生成前两个数字。
\item 第一个循环的index $i$最多到$L/2-1$,这样第一个数字就是$S[0\ldots i]$,长度就限制在$L/2$。
\item 第二个循环的index $j$则从$i+1$开始,由于其长度必须也不能大于$L$的一半,即$j-(i+1)+1\leq L/2$,即$j\leq i+L/2$
\end{itemize}
\setcounter{algorithm}{0}
\begin{algorithm}[H]
\caption{Recursion}
\begin{algorithmic}[1]
\Procedure{IsAdditiveNumber}{$S, L$}
\If{$S$ is empty}
\State \Return \texttt{false}
\EndIf
\State $\ast$ Generate first two numbers
\State $\ast$ Each length of the two numbers cannot be larger than $L/2$
\For{$i:=0$ \textbf{to} $L/2-1$}
\For{$j:=i+1$ \textbf{to} $j\leq i+L/2$}
\State $\ast$ Check the first two numbers $S[0\ldots i]$ and $S[i+1\ldots j]$
\If{\Call{Check}{$S, 0, i, j$} is \texttt{true}}
\State \Return \texttt{true}
\EndIf
\EndFor
\EndFor
\State $\ast$ 如果算法运行到这里,表明不可能形成连续的additive numbers
\State \Return \texttt{false}
\EndProcedure
\end{algorithmic}
\end{algorithm}
Function \texttt{Check} 测试当前两个number $ S[p_1\ldots p_2] $ 和$ S[p_2+1\ldots p_3] $是否能够得到仍旧属于$S$的sum。
\setcounter{algorithm}{0}
\begin{algorithm}[H]
\caption{Helper Function: Check}
\begin{algorithmic}[1]
\Procedure{Check}{$S, L, p_1, p_2, p_3$}
\State $\ast$ 处理边界情况
\If{$S[p_1\ldots p_2]$ starts with zero and length is larger than 1}
\State \Return \texttt{false}
\EndIf
\If{$S[p_2+1\ldots p_3]$ starts with zero and length is larger than 1}
\State \Return \texttt{false}
\EndIf
\algstore{306algo}
\end{algorithmic}
\end{algorithm}
\begin{algorithm}[H]
\begin{algorithmic}[1]
\algrestore{306algo}
\State $\star$ Compute sum of $S[p_1\ldots p_2]$ and $S[p_2+1\ldots p_3]$ as $X$
\State $\star$ 如果$p_3$加上$X$的长度已经超过了$S$的总长度,显然这是不成功的测试,直接返回\texttt{false}
\State $\star$ 比较$X$与$S[p_3+1\ldots\ell+p_3]$,其中$\ell$是$X$的长度,看这两个string是否相等。
\State $\star$ 如果不相等,也是不成功的测试,直接返回\texttt{false}
\State $\star$ 如果上述两个string相等,并且$p_3+\ell$刚好等于$L-1$,说明$S$已经测试完毕,返回\texttt{true}。
\State $\ast$ 如果上述两个string相等,并且$p_3+\ell$小于$L-1$,继续下一层递归测试
\State \Return \Call{Check}{$S, L, p_2+1, p_3, p_3+\ell$}
\EndProcedure
\end{algorithmic}
\end{algorithm}
\setcounter{lstlisting}{0}
\begin{lstlisting}[style=customc, caption={Recursion}]
bool isAdditiveNumber( string num )
{

    if( num.empty() )
    {
        return false;
    }

    int L = static_cast< int >( num.size() );

    //The length of the first two numbers
    //cannot be larger than L/2
    for( int i = 0; i < L / 2; ++i )
    {
        for( int j = i + 1; j <= L / 2 + i; ++j )
        {
            if( check( num, 0, i, j ) )
            {
                return true;
            }
        }
    }

    return false;
}

//check if sum of s[start...e1] and s[e1+1...e2]
//can be found in s
bool check( const string& s, int start, int e1, int e2 )
{
    int L = static_cast< int >( s.size() );


    //skip the numbers that starting with zero
    //and length is larger than 1
    if( ( s[start] == '0' ) && ( e1 - start >= 1 ) )
    {
        return false;
    }


    //skip the numbers that starting with zero
    //and length is larger than 1

    if( ( s[e1 + 1] == '0' ) && ( e2 - e1 >= 2 ) )
    {
        return false;
    }


    string sum;

    findSum( s, start, e1, e2, sum );

    int Ls = static_cast< int >( sum.size() );

    //The sum will overflow the length of s
    if( e2 + Ls >= L )
    {
        return false;
    }

    //check if the sum is in s
    for( int u = e2 + 1; u <= e2 + Ls; ++u )
    {
        if( s[u] != sum[u - e2 - 1] )
        {
            return false;
        }
    }

    //The sum reaches the end of s
    //so the whole string s is scanned properly
    if( e2 + Ls == L - 1 )
    {
        return true;
    }

    //check s[e1+1...e2] and s[e2+1...e2+Ls]
    return check( s, e1 + 1, e2, e2 + Ls );
}

//get the sum for S[p...x] + S[x+1...y]
void findSum( const string& s, int p, int x, int y, string &sum )
{
    auto i = x;
    auto j = y;

    int carry = 0;

    while( ( i >= p ) && ( j >= x + 1 ) )
    {
        int a1 = s[i] - '0';
        int a2 = s[j] - '0';

        int t = a1 + a2 + carry;
        if( t >= 10 )
        {
            carry = 1;
            t -= 10;
        }
        else
        {
            carry = 0;
        }

        sum.push_back( t + '0' );

        --i;
        --j;
    }

    while( i >= p )
    {
        int t = s[i] - '0' + carry;
        if( t >= 10 )
        {
            carry = 1;
            t -= 10;
        }
        else
        {
            carry = 0;
        }

        sum.push_back( t + '0' );
        --i;
    }

    while( j >= x + 1 )
    {
        int t = s[j] - '0' + carry;
        if( t >= 10 )
        {
            carry = 1;
            t -= 10;
        }
        else
        {
            carry = 0;
        }

        sum.push_back( t + '0' );
        --j;
    }

    if( carry == 1 )
    {
        sum.push_back( '1' );
    }

    reverse( sum.begin(), sum.end() );
}

\end{lstlisting}