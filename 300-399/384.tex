\section{384 --- Shuffle an Array}
Shuffle a set of numbers without duplicates.

\paragraph{Example:}

\begin{lstlisting}[style=customc]
// Init an array with set 1, 2, and 3.
int[] nums = {1, 2, 3};
Solution solution = new Solution( nums );

// Shuffle the array [1,2,3] and return its result. Any permutation of [1,2,3] must equally likely to be returned.
solution.shuffle();

// Resets the array back to its original configuration [1,2,3].
solution.reset();

// Returns the random shuffling of array [1,2,3].
solution.shuffle();

\end{lstlisting}

\subsection{Fisher–Yates shuffle}
On each iteration of the algorithm, we generate a random integer between the current index and the last index of the array. Then, we swap the elements at the current index and the chosen index -- this simulates drawing (and removing) the element from the hat, as the next range from which we select a random index will not include the most recently processed one. One small, yet important detail is that it is possible to swap an element with itself -- otherwise, some array permutations would be more likely than others. 

算法描述如下
\setcounter{algorithm}{0}
\begin{algorithm}[H]
\caption{Fisher–Yates shuffle}
\begin{algorithmic}[1]
\Procedure{Shuffle}{$A, L$}
\For{$i:=L-1$ \textbf{to} 1}
\State $\star$ generate a random number $r$ in $[0, i]$
\State $\star$ Swap $A[i]$ and $A[j]$
\EndFor
\EndProcedure
\end{algorithmic}
\end{algorithm}

\setcounter{lstlisting}{0}
\begin{lstlisting}[style=customc, caption={Fisher–Yates shuffle}]
class Solution
{
public:
    Solution( vector<int>& nums ): X( nums.begin(), nums.end() )
    {
    }

    /** Resets the array to its original configuration and return it. */
    vector<int> reset()
    {
        return X;
    }

    /** Returns a random shuffling of the array. */
    vector<int> shuffle()
    {
        vector<int> Y( X.begin(), X.end() );

        int n = static_cast<int>( X.size() );

        for( int i = n - 1; i >= 1; --i )
        {
            //generate random number in [0, i]
            //so the size is (i+1)
            int r = ( rand() % ( i + 1 ) );
            swap( Y[r], Y[i] );
        }

        return Y;
    }

    vector<int> X;
};

\end{lstlisting}