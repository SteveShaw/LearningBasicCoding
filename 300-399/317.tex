\section{317 --- Shortest Distance from All Buildings}

\textbf{Hard}

You want to build a house on an empty land which reaches all buildings in the shortest amount of distance. You can only move up, down, left and right. You are given a 2D grid of values 0, 1 or 2, where:

\begin{itemize}
\item Each 0 marks an empty land which you can pass by freely.
\item Each 1 marks a building which you cannot pass through.
\item Each 2 marks an obstacle which you cannot pass through.
\end{itemize}

\paragraph{Example:}

\begin{flushleft}
\textbf{Input}: \fcj{[[1,0,2,0,1],[0,0,0,0,0],[0,0,1,0,0]]}


\textbf{Output}: 7 

\textbf{Explanation}: 

Given three buildings at (0,0), (0,4), (2,2), and an obstacle at (0,2), the point (1,2) is an ideal empty land to build a house, as the total travel distance of $3+3+1=7$ is minimal. So return 7.
\end{flushleft}

\paragraph{Note:}
\begin{itemize}
\item There will be at least one building. 
\item If it is not possible to build such house according to the above rules, return $-1$.
\end{itemize}

\subsection{BFS}
We do BFS starting from each building. Every time, we update the given grid by distance to each building.

To avoid using an array to mark visited cell, we decrement the cell value each time. For example, we search for cell with value of zero and change them to $-1$. Next time, we search for cell with value of $-1$ and change to $-2$. Repeat the same process until all buildings are visited.

\setcounter{lstlisting}{0}
\begin{lstlisting}[style=customc, caption={BFS}]
int shortestDistance( vector<vector<int>>& grid )
{
    int m = ( int )( grid.size() );
    int n = ( int )( grid[0].size() );
    //this sums is used to record
    //distance to each building
    //updated each time for each building
    auto sums( grid );
    //val is used to indicate visited cell
    int val = 0;
    int min_dist = INT_MAX;
    for( int r = 0; r < m; ++r )
    {
        for( int c = 0; c < n; ++c )
        {
            if( grid[r][c] == 1 )
            {
                auto dist( grid );
                min_dist = INT_MAX;
                //start BFS: update distance
                //to this building at (r,c)
                queue<int> q;
                q.push( r * n + c );

                while( !q.empty() )
                {
                    int t = q.front();
                    q.pop();
                    int x = t / n;
                    int y = t - n * x;
                    //update for 4 directions
                    update( x + 1, y, val, dist[x][y], q, grid, dist, sums, min_dist );
                    update( x - 1, y, val, dist[x][y], q, grid, dist, sums, min_dist );
                    update( x, y + 1, val, dist[x][y], q, grid, dist, sums, min_dist );
                    update( x, y - 1, val, dist[x][y], q, grid, dist, sums, min_dist );
                }
                //decrement val
                --val;
            }
        }
    }
    return min_dist == INT_MAX ? -1 : min_dist;
}
//update dist and total dist
void update( int r, int c, int val, int d, queue<int>& q,
             vector<vector<int>>& grid,
             vector<vector<int>>& dist,
             vector<vector<int>>& total,
             int &ans )
{
    int m = ( int )( grid.size() );
    int n = ( int )( grid[0].size() );
    if( ( r >= 0 ) && ( r < m ) && ( c >= 0 ) && ( c < n ) && ( grid[r][c] == val ) )
    {
        //decrements grid[r][c] to mark it as not visited
        grid[r][c] -= 1;
        //update distance to the building
        dist[r][c] = d + 1;
        //accumulate the distance
        total[r][c] += ( dist[r][c] - 1 );
        //update the minimum so far
        ans = ( min )( ans, total[r][c] );
        //add this cell to the queue
        q.push( r * n + c );
    }
}
\end{lstlisting}

\paragraph{Related Problems}
\begin{itemize}
\item \textbf{286. Walls and Gates}
\item \textbf{296. Best Meeting Point}
\item \textbf{1162. As Far from Land as Possible}
\end{itemize}