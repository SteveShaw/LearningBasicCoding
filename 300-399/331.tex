\section{331 --- Verify Preorder Serialization of a Binary Tree}
One way to serialize a binary tree is to use pre-order traversal. When we encounter a non-null node, we record the node's value. If it is a null node, we record using a sentinel value such as $\star$.
\begin{figure}[H]
\begin{tikzpicture}
[my/.style={draw, circle, fill=gray!20!, minimum size=5mm}]
\node[my](1) at (0,0) {9};
\node[my](2)[below=8mm of 1, xshift=-12mm] {\textcolor{red}{3}};
\node[my](3)[below=8mm of 1, xshift=13mm] {2};
\node[my](4)[below=8mm of 2, xshift=-8mm] {\textcolor{red}{4}};
\node[my](5)[below=8mm of 2, xshift=8mm] {\textcolor{red}{1}};
\node[my](6)[below=8mm of 3, xshift=-8mm] {$\star$};
\node[my](7)[below=8mm of 3, xshift=8mm] {6};
\node[my](8)[below=8mm of 4, xshift=-4mm] {$\star$};
\node[my](9)[below=8mm of 4, xshift=4mm] {$\star$};
\node[my](10)[below=8mm of 5, xshift=-4mm] {$\star$};
\node[my](11)[below=8mm of 5, xshift=4mm] {$\star$};
\node[my](12)[below=8mm of 7, xshift=-4mm] {$\star$};
\node[my](13)[below=8mm of 7, xshift=4mm] {$\star$};
\draw[thick] (1) -- (2);
\draw[thick] (1) -- (3);
\draw[thick] (2) -- (4);
\draw[thick] (2) -- (5);
\draw[thick] (3) -- (6);
\draw[thick] (3) -- (7);
\draw[thick] (4) -- (8);
\draw[thick] (4) -- (9);
\draw[thick] (5) -- (10);
\draw[thick] (5) -- (11);
\draw[thick] (7) -- (12);
\draw[thick] (7) -- (13);
\end{tikzpicture}
\end{figure}
For example, the above binary tree can be serialized to the string $9,3,4,\star,\star,1,\star,\star,2,\star,6,\star,\star$, where $\star$ represents a null node.
\par
Given a string of comma separated values, verify whether it is a correct preorder traversal serialization of a binary tree. Find an algorithm without reconstructing the tree.
\par
Each comma separated value in the string must be either an integer or a character $\star$ representing null pointer.
\par
You may assume that the input format is always valid, for example it could never contain two consecutive commas such as $1,,3$.

\paragraph{Example 1:}
\begin{flushleft}
\textbf{Input}: \fcj{"9,3,4,#,#,1,#,#,2,#,6,#,#"}

\textbf{Output}: \fcc{true}
\end{flushleft}

\paragraph{Example 2:}
\begin{flushleft}
\textbf{Input}: \fcj{"1,#"}

\textbf{Output}: \fcc{false}
\end{flushleft}

\paragraph{Example 3:}
\begin{flushleft}
\textbf{Input}: \fcj{"9,#,#,1"}

\textbf{Output}: \fcc{false}
\end{flushleft}

\subsection{Iteration}
Binary tree could be considered as a number of slots to fulfill. 

At the start there is just one slot available for a number or null node. Both number and null node take one slot to be placed. 

For the \fcj{null} node the story ends up here, whereas the number will add into the tree two slots for the child nodes. Each child node could be, again, a number or a \fcj{null}.

The idea is straightforward : take the nodes one by one from \textit{preorder} traversal, and compute the number of available slots. If at the end all available slots are used up, the \textit{preorder} traversal represents the valid serialization.

\setcounter{lstlisting}{0}
\begin{lstlisting}[style=customc, caption={Iteration}]
bool isValidSerialization( string preorder )
{
    // number of available slots
    int slots = 1;
    for( size_t i = 0; i < preorder.size(); ++i )
    {
        if( preorder[i] == ',' )
        {
            // one node takes one slot
            --slots;
            // no more slots available
            if( slots < 0 )
            {
                return false;
            }
            // non-empty node creates two children slots
            if( ( i > 0 ) && ( preorder[i - 1] != '#' ) )
            {
                slots += 2;
            }
        }
    }
    //for the last node
    //since there is no comma, we only add or subtract one
    if( preorder.back() == '#' )
    {
        slots -= 1;
    }
    else
    {
        ++slots;
    }
    return slots == 0;
}
\end{lstlisting}