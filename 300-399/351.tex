\section{351 --- Android Unlock Patterns}

\textbf{Medium}

Given an Android $3\times 3$ key lock screen and two integers $m$ and $n$, where $1 \leq m \leq n \leq 9$, count the total number of unlock patterns of the Android lock screen, which consist of minimum of $m$ keys and maximum $n$ keys.

 

\paragraph{Rules for a valid pattern:}

\begin{itemize}
\item Each pattern must connect at least $m$ keys and at most $n$ keys.
\item All the keys must be distinct.
\item If the line connecting two consecutive keys in the pattern passes through any other keys, the other keys must have previously selected in the pattern. No jumps through non selected key is allowed.
\item The order of keys used matters.
\end{itemize}

\paragraph{Explanation:}

\[
\begin{bmatrix}
1 & 2 & 3\\
4 & 5 & 6\\
7 & 8 & 9
\end{bmatrix}
\]

Invalid move: 4 -- 1 -- 3 -- 6

Line 1 -- 3 passes through key 2 which had not been selected in the pattern.

Invalid move: 4 -- 1 -- 9 -- 2

Line 1 -- 9 passes through key 5 which had not been selected in the pattern.

Valid move: 2 -- 4 -- 1 -- 3 -- 6

Line 1 -- 3 is valid because it passes through key 2, which had been selected in the pattern

Valid move: 6 -- 5 -- 4 -- 1 -- 9 -- 2

Line 1 -- 9 is valid because it passes through key 5, which had been selected in the pattern.

\paragraph{Example:}

\begin{flushleft}
\textbf{Input}: $ m = 1 $, $ n = 1 $

\textbf{Output}: 9
\end{flushleft}

\subsection{Backtracking}
We can use a 2D array to record the key between two keys. For example, key 2 is between 1 and 3 horizontally. 

Next, for each number from 1 to 9, and we try possible jumps until $n$.

\setcounter{lstlisting}{0}
\begin{lstlisting}[style=customc, caption={Backtracking}]
int numberOfPatterns( int m, int n )
{
    vector<unsigned char> visited( 10, 0 );
    vector<vector<unsigned char>> jumps( 10, vector<unsigned char>( 10, 0 ) );
    //horizon
    jumps[1][3] = 2;
    jumps[3][1] = 2;
    jumps[4][6] = 5;
    jumps[6][4] = 5;
    jumps[7][9] = 8;
    jumps[9][7] = 8;
    //vertical
    jumps[1][7] = 4;
    jumps[7][1] = 4;
    jumps[2][8] = 5;
    jumps[8][2] = 5;
    jumps[3][9] = 6;
    jumps[9][3] = 6;
    //diag
    jumps[1][9] = 5;
    jumps[9][1] = 5;
    //anti-diag
    jumps[3][7] = 5;
    jumps[7][3] = 5;
    int ans = 0;
    for( int i = m; i <= n; ++i )
    {
        ans += dfs( visited, jumps, 1, i - 1 ) * 4;
        ans += dfs( visited, jumps, 2, i - 1 ) * 4;
        ans += dfs( visited, jumps, 5, i - 1 );
    }
    return ans;
}
//backtracking
int dfs( vector<unsigned char>& visited, vector<vector<unsigned char>>& jumps, int start, int remains )
{
    if( remains < 0 )
    {
        return 0;
    }
    if( remains == 0 )
    {
        return 1;
    }
    visited[start] = 1;
    int keys = 0;
    for( int next = 1; next <= 9; ++next )
    {
        if( ( visited[next] == 0 ) && ( ( jumps[start][next] == 0 ) || ( visited[jumps[start][next]] == 1 ) ) )
        {
            //we can jump to next
            keys += dfs( visited, jumps, next, remains - 1 );
        }
    }
    //backtracking
    visited[start] = 0;
    return keys;
}
\end{lstlisting}