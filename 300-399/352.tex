\section{352 --- Data Stream as Disjoint Interxs}
Given a data stream input of non-negative integers $ a_1 $, $ a_2 $,$ \ldots$, $ a_n $, $ \ldots $, summarize the numbers seen so far as a list of disjoint interxs.
\par
For example, suppose the integers from the data stream are $ 1, 3, 7, 2, 6, \ldots $, then the summary will be:
\begin{tabular}{ccc}
$ [1, 1] $ & & \\
$ [1, 1] $ & $ [3, 3] $ & \\
$ [1, 1] $ & $ [3, 3 $] & $ [7, 7] $\\
$ [1, 3] $ & $ [7, 7] $  & \\
$ [1, 3] $ & $ [6, 7] $
\end{tabular}


\paragraph{Follow up:}
\begin{itemize}
\item What if there are lots of merges and the number of disjoint interxs are small compared to the data stream's size?
\end{itemize}
\subsection{Binary Search}
\begin{itemize}
\item 和merge range的题目相同。
\item 用一个tree map作为数据结构,其中key是range的start,而value则是range的end。
\item 对每次加入的value $x$,用upper bound (rightmost binary search)找到第一个start大于$ x $的range。总共有如下三种情况
\begin{enumerate}
\item $x$比当前map中的最后一个range的start大,这时候比较$x$和最后一个range的end,如果$x$刚好等于这个end加1,那么将这个$x$ merge到最后一个range,同时修改这个range的end为$x$。而如果$x$小于这个range的end,说明$x$落在这最后一个range内,不需要做处理。如果$x$不能和最后一个range merge,创建一个新的range $[x,x]$。
\item $x$小于当前map的第一个range的start,如果$x$和第一个range的start相差1,将$x$ merge到第一个range,同时修改第一个range的start为$x$。否则,$ x $不能和第一个range merge
,创建一个新的range $[x,x]$。
\item $x$落在当前map的两个range中间。这时候需要比较$x$与右边的range的start以及$x$与左边的range的end。
\begin{itemize}
\item 如果$x$和右边range的start相差1,需要把$x$和右边range进行merge,即将右边range的start修改为$x$。
\item 如果$x$小于或者等于左边range的end,那么说明$x$落在左边range中,需要把$x$ merge到左边的range中。
\item 如果$x$和左边range的end相差1,也需要把$x$和左边range进行merge,即将左边range的end修改为$x$。
\item 如果$x$即能够和右边range merge,又能够和左边range merge,说明右边和左边的range都要merge到一起。需要把左边和右边的range从map中删除,代替以一个新的range,这个新的range将包含之前左边和右边的range。
\end{itemize}
\end{enumerate}
\end{itemize}
下面的算法是function \texttt{addNum}的算法描述。
\setcounter{algorithm}{0}
\begin{algorithm}[H]
\caption{Binary Search}
\begin{algorithmic}[1]
\Procedure{AddNum}{$x$}
\State $\star$ tree map $M$ is the member of class
\If{$M$ is empty}
\State $M[x]:=x$ \Comment 插入range $ [x,x] $
\State \Return
\EndIf
\State $\star$ 在$M$中找到第一个range $r$,其start $r_0$大于$ x $
\If{$x$ is larger than the start $ r_0 $ of last range $r$ in $M$}
\State $\ast$ 比较$x$与$ r $的end $r_1$
\If{$x-r_1=1$ }
\State $r_1\gets x$ \Comment Modify the end of $r$ as $x$
\ElsIf{$x-r_1>1$} \Comment $x$ does not touch $r$
\State $M[x]:=x$ \Comment 插入range $ [x,x] $
\EndIf
\State \Return
\EndIf
\If{$x$ is smaller than the start $ r_0 $ of first range $r$ in $M$}
\State $\ast$ 比较$x$与$ r $的start $r_0$
\If{$r_0-x=1$}
\State $r_0\gets x$ \Comment Modify the start of $r$ as $x$
\Else \Comment $x$ does not touch $r$
\State $M[x]:=x$ \Comment 插入range $ [x,x] $
\EndIf
\algstore{352algo}
\end{algorithmic}
\end{algorithm}
\begin{algorithm}[H]
\begin{algorithmic}[1]
\algrestore{352algo}
\State \Return
\EndIf
\State $\ast$ $r$既不是第一个也不是最后一个range。
\State $\star$ Get the previous range $l$ in $M$ next to $r$
\State $\alpha:=x$ \Comment The start of new range
\State $\beta:=x$ \Comment The end of new range
\If{$r_0-x=1$} \Comment $x$ touch $r$
\State $\beta \gets r_1$ \Comment Update the end of new range
\State $\ast$ 将$r$从$M$中删除
\EndIf
\If{$x-l_1=1$ \textbf{or} $x_1\leq l_1$} \Comment $x$ can be merged with $l$
\State $\alpha\gets \min(\alpha, l_0)$ \Comment The start of new range is the smaller of $l$'s start $l_0$ and $\alpha$
\State $\beta\gets \max(\beta, l_1)$ \Comment The end of new range is the bigger of $l$'s end $l_1$ and $\beta$
\State $\ast$ 将$l$从$M$中删除
\EndIf
\State $M[\alpha]:=\beta$ \Comment 插入range $ [\alpha,\beta] $
\EndProcedure
\end{algorithmic}
\end{algorithm}

\setcounter{lstlisting}{0}
\begin{lstlisting}[style=customc, caption={Binary Search}]
class SummaryRanges
{
public:
    /** Initialize your data structure here. */
    SummaryRanges()
    {

    }

    void addNum( int val )
    {

        if( m_itv.empty() )
        {
            m_itv.emplace( val, val );
            return;
        }

        //auto it = m_itv.find(
        auto it = m_itv.upper_bound( val );

        if( it == m_itv.end() )
        {
            --it;
            if( val == it->second + 1 )
            {
                //merge
                it->second = val;
            }
            else if( val > it->second + 1 )
            {
                //not touched, create a new interval
                m_itv.emplace( val, val );
            }
            //in other cases val is inside the range [it->first, it->second]

            return;
        }

        if( it == m_itv.begin() )
        {
            if( it->first - val == 1 )
            {
                //merge
                int b = val;
                int e = it->second;

                m_itv.erase( it );
                m_itv.emplace( b, e );

            }
            else
            {
                //not touched, create a new interval
                m_itv.emplace( val, val );
            }

            return;
        }


        auto prev = it;
        --prev;

        int b = val;
        int e = val;

        bool b_del_right = false;

        if( it->first - val == 1 )
        {
            //merge: we must create a new interval
            b = val;
            e = it->second;

            b_del_right = true;
        }

        bool b_del_left = false;
        if( ( val <= prev->second ) || ( val - prev->second == 1 ) )
        {
            b = ( min )( prev->first, b );
            e = ( max )( prev->second, e );
            b_del_left = true;
        }

        if( b_del_right )
        {
            m_itv.erase( it );
        }

        if( b_del_left )
        {
            m_itv.erase( prev );
        }

        //insert new range [b,e]
        m_itv.emplace( b, e );
    }

    vector<vector<int>> getIntervals()
    {
        vector<vector<int>> ans;
        ans.reserve( m_itv.size() );

        for( const auto&p : m_itv )
        {
            ans.emplace_back( initializer_list<int> {p.first, p.second} );
        }

        return ans;
    }

    map<int, int> m_itv;
};
\end{lstlisting}

\subsection{Follow Up}
\begin{itemize}
\item 如果有大量的merge操作,而disjoint interxs相对来说较少的情况下,上述binary search的算法是最佳的。
\end{itemize}