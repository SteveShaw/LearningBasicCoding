\section{390 --- Elimination Game}
There is a list of sorted integers from 1 to $ n $. Starting from left to right, remove the first number and every other number afterward until you reach the end of the list.
\par
Repeat the previous step again, but this time from right to left, remove the right most number and every other number from the remaining numbers.
\par
We keep repeating the steps again, alternating left to right and right to left, until a single number remains.
\par
Find the last number that remains starting with a list of length $ n $.

\paragraph{Example:}

\begin{flushleft}
\textbf{Input}: $n = 9$,
\begin{table}[H]
\begin{tabular}{*{9}{l}}
1&  2&  3&  4&  5&  6&  7&  8&  9\\
2 & 4& 6& 8 & & & & & \\
2 & 6 &  &  &  &  &  &  & \\
6  &  &  &  &  &  &  &  & 
\end{tabular}
\end{table}
\textbf{Output}: 6
\end{flushleft}

\subsection{Recursion}
\begin{itemize}
\item 第一轮过后,所有剩下的数字都是偶数。除以2后,得到一个新的sequence,从1到$n/2$。
\item 对这个新的sequence,其第一轮是从右到左。返回的结果应该是调用递归的结果\texttt{lastRemaining}($n / 2$)在数组1到$n/2$之间的镜像。何为镜像,比如1, 2, 3, 4这个数字,2的镜像就是3, 1的镜像是4。
\item 因此将这个新的sequence消除剩下的数字乘以2即为原数组进行第二轮消除的结果。依此类推。
\end{itemize}

\setcounter{lstlisting}{0}
\begin{lstlisting}[style=customc, caption={Recursion}]
int lastRemaining( int n )
{
    if( n == 1 )
    {
        return 1;
    }

    return 2 * ( 1 + n / 2 - lastRemaining( n / 2 ) );
}
\end{lstlisting}

下面的是比较清晰的写法
\begin{lstlisting}[style=customc, caption={Recursion Break Down}]
int lastRemaining( int n )
{
    return fromLeft( n );
}

int fromLeft( int n )
{
    if( n <= 2 )
    {
        return n;
    }

    //from left, the first element is even number
    //after elimination
    return 2 * fromRight( n / 2 );
}

int fromRight( int n )
{
    if( n <= 2 )
    {
        return 1;
    }

    if( n & 1 )
    {
        //when n is odd number
        //the first element after elimination is even number
        return 2 * fromLeft( n / 2 );
    }

    //when n is even number
    //the first element after elimination is odd number
    return 2 * fromLeft( n / 2 ) - 1;
}
\end{lstlisting}

\subsection{Iterative}
\begin{itemize}
\item The idea is to update and record first item after elimination in each turn. when the total number becomes 1, head is the only number left.
\item The first item will be updated 
\begin{enumerate}
\item If current round is starting from left.
\item If current round is starting from right and current count of remaining numbers is odd number. Why? For example, suppose current sequence is $[2, 4, 6, 8, 10]$, and now eliminating numbers from right. As a result, we will remove 10, 6 and 2. The first item is changed from 2 to 4. Now, suppose current sequence is $[2, 4, 6, 8, 10, 12]$. After elimination, the remaining numbers are $[2,6,10]$. The first item is still 2.
\end{enumerate}
\item We also maintain a variable $\delta$ as the elimination step. $\delta$ will be double each time since we eliminate every other number. Initially, $\delta=1$.
\end{itemize}


\begin{algorithm}[H]
\caption{Iterative}
\begin{algorithmic}[1]
\Procedure{LastRemaining}{$n$}
\State $x:=1$ \Comment The first number after each round
\State $\delta:=1$ \Comment The step by which the first number will move
\State $b:=0$ \Comment Indicate if current round is from left (0=left, 1=right)
\algstore{390algo}
\end{algorithmic}
\end{algorithm}
\begin{algorithm}[H]
\begin{algorithmic}[1]
\algrestore{390algo}
\While{$n>1$}
\If{$b=0$ \textbf{or} $n\bmod 2=1$}
\State $x\gets x+\delta$ \Comment The first number will be changed
\EndIf
\State $\delta\gets \delta * 2$ \Comment Double the step
\State $y\gets y/2$ \Comment Half the count of remaining numbers
\State $b=1-b$ \Comment Change to another direction
\EndWhile
\State \Return $x$
\EndProcedure
\end{algorithmic}
\end{algorithm}

\begin{lstlisting}[style=customc, caption={Iterative}]
int lastRemaining( int n )
{
    // The first number after each round
    int x = 1;
    // the jump that x will take
    int delta = 1;

    int y = n;

    bool from_left = true;

    while( y > 1 )
    {
        if( from_left || ( y & 1 ) )
        {
            //The first number is changed
            x += delta;
        }

        //double the jump
        delta *= 2;
        //change to another direction
        from_left = !from_left;
        //The count of remainings become half
        y /= 2;
    }

    return x;
}
\end{lstlisting}