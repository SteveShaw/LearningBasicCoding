\section{339 --- Nested List Weight Sum}

\textbf{Easy}

Given a nested list of integers, return the sum of all integers in the list weighted by their depth.

Each element is either an integer, or a list -- whose elements may also be integers or other lists.

\paragraph{Example 1:}

\begin{flushleft}
\textbf{Input}: \fcj{[[1,1],2,[1,1]]}

\textbf{Output}: 10 

\textbf{Explanation}: Four 1's at depth 2, one 2 at depth 1.
\end{flushleft}

\paragraph{Example 2:}

\begin{flushleft}
\textbf{Input}: \fcj{[1,[4,[6]]]}

\textbf{Output}: 27 

\textbf{Explanation}: 

One 1 at depth 1, one 4 at depth 2, and one 6 at depth 3; $1 + 4\times 2 + 6\times 3 = 27$.
\end{flushleft}

\subsection{DFS}
Because the input is nested, it is natural to think about the problem in a recursive way. 

We go through the list of nested integers one by one, keeping track of the current depth $d$. 

If a nested integer is an integer $n$, we calculate its sum as $n\times d$. 

If the nested integer is a list, we calculate the sum of this list recursively using the same process but with depth $d+1$.

\setcounter{lstlisting}{0}
\begin{lstlisting}[style=customc, caption={DFS}]
int depthSum( vector<NestedInteger>& nestedList )
{
    int sum = 0;
    for( const auto& ni : nestedList )
    {
        sum += dfs( ni, 1 );
    }
    return sum;
}
//get the sum recurisvely with level
int dfs( const NestedInteger& ni, int level )
{
    if( ni.isInteger() )
    {
        return ni.getInteger() * level;
    }
    //for list, recurisvely get the sum
    int sum = 0;
    for( const auto& child : ni.getList() )
    {
        sum += dfs( child, level + 1 );
    }
    return sum;
}
\end{lstlisting}

\subsection{BFS}
This problem can also be solved using BFS with level traverse

\begin{lstlisting}[style=customc, caption={BFS}]
int depthSum( vector<NestedInteger>& nestedList )
{
    //using reference_wrapper to avoid copy object
    queue<reference_wrapper<const NestedInteger>> q;
    int ans = 0;
    for( const auto& ni : nestedList )
    {
        q.push( cref( ni ) );
        int level = 1;
        while( !q.empty() )
        {
            auto sz = q.size();
            for( size_t i = 0; i < sz; ++i )
            {
                auto e = q.front();
                q.pop();
                if( e.get().isInteger() )
                {
                    ans += e.get().getInteger() * level;
                }
                else
                {
                    //if it is not integer
                    //push into the queue
                    for( const auto& next : e.get().getList() )
                    {
                        q.push( cref( next ) );
                    }
                }
            }
            ++level;
        }
    }
    return ans;
}
\end{lstlisting}

\paragraph{Related Problems}
\begin{itemize}
\item \textbf{364. Nested List Weight Sum II}
\item \textbf{565. Array Nesting}
\item \textbf{690. Employee Importance}
\end{itemize}