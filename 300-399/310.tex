\section{310 --- Minimum Height Trees}
For an undirected graph with tree characteristics, we can choose any node as the root. The result graph is then a rooted tree. Among all possible rooted trees, those with minimum height are called minimum height trees (MHTs). Given such a graph, write a function to find all the MHTs and return a list of their root labels.

\paragraph{Format}
\par
The graph contains $ n $ nodes which are labeled from 0 to $  n - 1 $. You will be given the number $ n $ and a list of undirected edges (each edge is a pair of labels).
\par
You can assume that no duplicate edges will appear in edges. Since all edges are undirected, $ [0, 1] $ is the same as $ [1, 0] $ and thus will not appear together in edges.

\paragraph{Example 1}:

\begin{flushleft}
\textbf{Input}: $ n = 4 $
\begin{figure}[H]
\begin{tikzpicture}
[my/.style={draw, circle, fill=gray!20!, minimum size=5mm}]
\node[my] (1) at (0,0) {0};
\node[my] (2) [below=8mm of 1] {1};
\node[my] (3) [below=8mm of 2, xshift=-8mm] {2};
\node[my] (4) [below=8mm of 2, xshift=8mm] {3};
\draw[line width=0.7pt] (1) -- (2);
\draw[line width=0.7pt] (2) -- (3);
\draw[line width=0.7pt] (2) -- (4);
\end{tikzpicture}
\end{figure}
\textbf{Output}: $ [1] $
\end{flushleft}

\paragraph{Example 2 :}

\begin{flushleft}
\textbf{Input}: $ n = 6 $
\begin{figure}[H]
\begin{tikzpicture}
[my/.style={draw, circle, fill=gray!20!, minimum size=5mm}]
\node[my] (1) at (0,0) {3};
\node[my] (2) [above=8mm of 1] {1};
\node[my] (3) [above=8mm of 1, xshift=-1cm] {0};
\node[my] (4) [above=8mm of 1, xshift=1cm] {2};
\node[my] (5) [below=8mm of 1] {4};
\node[my] (6) [below=8mm of 5] {5};
\draw[line width=0.7pt] (1) -- (2);
\draw[line width=0.7pt] (1) -- (3);
\draw[line width=0.7pt] (1) -- (4);
\draw[line width=0.7pt] (1) -- (5);
\draw[line width=0.7pt] (5) -- (6);
\end{tikzpicture}
\end{figure}
\textbf{Output}: $ [3, 4] $
\end{flushleft}

\paragraph{Note:}

\begin{itemize}
\item According to the definition of tree on Wikipedia: \texttt{a tree is an undirected graph in which any two vertices are connected by \textit{exactly} one path. In other words, any connected graph without simple cycles is a tree}.
\item The height of a rooted tree is the number of edges on the longest downward path between the root and a leaf.
\end{itemize}
\subsection{Breadth First Search}
\begin{itemize}
\item 首先建立adjacent list的graph,然后创建一个queue,将所有只有degree为1即只有一个edge的节点放入队列中
\item 然后依次从queue中弹出节点$x$,将该节点从与之相连的节点$y$的adjacent list中删除。如果$y$的degree也变为了1,将$y$放入queue中。
\item 最后如果queue中的节点小于等于2,那么当前queue中的节点就是符合要求的root nodes。
\end{itemize}
\setcounter{lstlisting}{0}
\begin{lstlisting}[style=customc, caption={Breadth First Search}]
vector<int> findMinHeightTrees( int n, vector<pair<int, int>>& edges )
{
    if( n == 0 )
    {
        return {};
    }

    if( n == 1 )
    {
        return  {0};
    }

    //create adjacent list structure
    vector<unordered_set<int>> G( n );

    //This is an unordered graph
    for( const auto& edge : edges )
    {
        G[edge.first].insert( edge.second );
        G[edge.second].insert( edge.first );
    }

    queue<int> q;

    for( int i = 0; i < n; ++i )
    {
        if( G[i].size() == 1 )
        {
            //add all nodes with only edge
            q.emplace( i );
        }
    }

    vector<int> ans;

    //loop until there are no larger than 2 nodes in the queue
    while( n > 2 )
    {
        int q_size = static_cast< int >( q.size() );
        n -= q_size;

        for( int i = 0; i < q_size; ++i )
        {
            auto t = q.front();
            q.pop();

            //remove t from t's adjacent node lists
            int adj = *( G[t].begin() );
            G[adj].erase( t );

            if( G[adj].size() == 1 )
            {
                //t's adjacent also has degree 1 now
                q.push( adj );
            }
        }
    }

    //now nodes in queue are the
    //answer
    while( !q.empty() )
    {
        ans.push_back( q.front() );
        q.pop();
    }

    return ans;

}

\end{lstlisting}
