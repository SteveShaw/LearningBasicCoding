\section{315 --- Count of Smaller Numbers After Self}
You are given an integer array $ A $ and you have to return a new counts array $ Z $. The counts array $ Z $ has the property where $ Z[i] $ is the number of smaller elements to the right of $ A[i] $.

\paragraph{Example:}

\begin{flushleft}
\textbf{Input}: $ [5,\;2,\;6,\;1] $
\\
\textbf{Output}: $ [2,\;1,\;1,\;0] $ 
\\
\textbf{Explanation}:
\\
To the right of 5 there are 2 smaller elements (2 and 1).
\\
To the right of 2 there is only 1 smaller element (1).
\\
To the right of 6 there is 1 smaller element (1).
\\
To the right of 1 there is 0 smaller element.
\end{flushleft}
\subsection{Binary Sort}
从$ A[L-1] $即$ A $的最后一个number开始,用leftmost binary search的方法插入到一个新的数组,这样新数组就是有序的,那么此时该数字在新数组中的坐标就是原数组中其右边所有较小数字的个数。之所以用leftmost,是因为要考虑到重复数字的情况。如果用\texttt{STL}的map,计算距离的时间复杂度是$ O(n) $,而用数组的时间复杂度为$ O(1) $。

\setcounter{lstlisting}{0}
\begin{lstlisting}[style=customc, caption={Binary Search And Sorting}]
vector<int> countSmaller( vector<int>& nums )
{
    //process edge scenarios
    if( nums.empty() )
    {
        return {};
    }

    if( nums.size() == 1 )
    {
        return {0};
    }

    vector<int> A; //Sorted nums
    A.reserve( nums.size() );
    A.push_back( nums.back() );

    vector<int> ans( nums.size(), 0 );

    auto L = static_cast< int >( nums.size() );

    for( int i = L - 2; i >= 0; --i )
    {
        int l = 0;
        int curSize = static_cast< int >( A.size() );
        int r = curSize;
        //leftmost binary search
        while( l < r )
        {
            int mid = ( l + r ) / 2;

            if( A[mid] < nums[i] )
            {
                l = mid + 1;
            }
            else
            {
                r = mid;
            }
        }

        ans[i] = l;
        //l is the position in A for nums[i]
        if( l == curSize )
        {
            //nums[i] is the largest, then push back
            A.push_back( nums[i] );
        }
        else
        {
            //otherwise, insert into the correct position
            A.insert( A.begin() + l, nums[i] );
        }
    }
    return ans;
}
\end{lstlisting}

\subsection{Merge Sort}
Merge Sort is suitable for problems which look for some pairs $i$, $j$
such that $i < j$, and $A[i]$, $A[j]$ satisfy some constraints.

We can find such pairs during merge sort. 

In each recursion, before we merge two sorted subarrays, the $i$ is in the left sorted subarray, and the $j$ is in the right sorted subarray. So, we can just go through both sorted subarray to find the valid $i$ and $j$ pairs. As long as this step is $O(n)$, the total time complexity would be $O(n \log n)$.

Such problems do not care about the order of js as long as $j > i$. The $j$ could be possibly anywhere after $i$ as long as $A[j]$ and $A[i]$ satisfy the given constraint. 

Balanced Binary Search Tree would work well for this purpose, but it requires extra code to build the BST and also keep it balanced to avoid $O(N^2)$ worst time complexity. For problems like these, Segment Tree and Binary Indexed Tree are also good choices and give $O(n \log n)$ time complexity.

C++ provides built-ins for merge sort including:

\begin{itemize}
\item \fcj{merge(l1.begin(), l1.end(), l2.begin(), l2.end(), result.begin());} which stores the merged array in \fcj{result}
\item \fcj{inplace_merge(l.begin(), l.middle, l.end())} where array \fcj{[begin, middle)} is merged with array \fcj{[middle, end)}.
\end{itemize}


\begin{lstlisting}[style=customc, caption={Merge Sort}]
vector<int> countSmaller( vector<int>& nums )
{
    if( nums.empty() )
    {
        return {};
    }
    //put value and its index into one array
    vector<pair<int, int>> A( nums.size() );
    int L = ( int )nums.size();
    for( int i = 0; i < L; ++i )
    {
        A[i].first = nums[i];
        A[i].second = i;
    }
    vector<int> ans( nums.size() );
    //do merge sort while getting the answer
    merge_sort( begin( A ), end( A ), ans );
    return ans;
}
void merge_sort( vector<pair<int, int>>::iterator beg, vector<pair<int, int>>::iterator end, vector<int>& ans )
{
    auto dist = distance( beg, end );
    if( dist <= 1 )
    {
        return;
    }

    auto mid = next( beg, dist / 2 );
    merge_sort( beg, mid, ans );
    merge_sort( mid, end, ans );
    //now [beg,mid) and [mid,end) are sorted
    //we then can compare left and right
    for( auto i = beg, j = mid; i != mid; ++i )
    {
        auto [l, l_idx] = ( *i );
        while( j != end )
        {
            auto [r, r_idx] = ( *j );
            if( l > r )
            {
                //smaller numbers in the right hand side
                ++j;
            }
            else
            {
                break;
            }
        }
        ans[l_idx] += ( int )distance( mid, j );
    }
    //do inplace merge
    inplace_merge( beg, mid, end );
}
\end{lstlisting}

\subsection{Binary Index Tree}
We use BIT to store the index of each \fcj{nums[i]} in the sorted array. One way is to make use of a hash map.

\begin{lstlisting}[style=customc, caption={BIT}]
vector<int> countSmaller( vector<int>& nums )
{

    vector<int> A( nums );
    sort( begin( A ), end( A ) );
    int L = ( int )nums.size();
    //use m to record index of each element
    //in the sorted array
    unordered_map<int, int> m;
    for( int i = 0; i < L; ++i )
    {
        m[A[i]] = i + 1;
    }
    //BIT array
    vector<int> BIT( L + 1, 0 );
    for( int i = L - 1; i >= 0; --i )
    {
        //get the index of nums[i]
        //in the sorted array
        int t = m[nums[i]];
        A[i] = query( t - 1, BIT );
        update( t, 1, BIT );
    }
    return A;
}
//get the number of index that
//is larger than index
int query( int index, vector<int>& BIT )
{
    int j = index + 1;
    int res = 0;
    while( j > 0 )
    {
        res += BIT[j];
        j -= ( j & ( -j ) );
    }
    return res;
}
//update index saved in BIT
void update( int index, int diff, vector<int>& BIT )
{
    int j = index + 1;
    int L = ( int )BIT.size();
    while( j < L )
    {
        BIT[j] += diff;
        j += ( j & ( -j ) );
    }
}
\end{lstlisting}

\paragraph{Related Problems}
\begin{itemize}
\item \textbf{327. Count of Range Sum}
\item \textbf{406. Queue Reconstruction by Height}
\item \textbf{493. Reverse Pairs}
\end{itemize}