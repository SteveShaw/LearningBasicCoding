\section{381 --- Insert Delete GetRandom O(1) - Duplicates allowed}
Design a data structure that supports all following operations in average $ O(1) $ time.

\paragraph{Note:} 
\begin{itemize}
\item Duplicate elements are allowed.
\end{itemize}


\begin{lstlisting}[style=customc]
insert(val): //Inserts an item val to the collection.
remove(val): //Removes an item val from the collection if present.
getRandom: //Returns a random element from current collection of elements. The probability of each element being returned is linearly related to the number of same value the collection contains.
\end{lstlisting}

\paragraph{Example:}

\begin{lstlisting}[style=customc]
// Init an empty collection.
RandomizedCollection collection = new RandomizedCollection();

// Inserts 1 to the collection. Returns true as the collection did not contain 1.
collection.insert( 1 );

// Inserts another 1 to the collection. Returns false as the collection contained 1. Collection now contains [1,1].
collection.insert( 1 );

// Inserts 2 to the collection, returns true. Collection now contains [1,1,2].
collection.insert( 2 );

// getRandom should return 1 with the probability 2/3, and returns 2 with the probability 1/3.
collection.getRandom();

// Removes 1 from the collection, returns true. Collection now contains [1,2].
collection.remove( 1 );

// getRandom should return 1 and 2 both equally likely.
collection.getRandom();
\end{lstlisting}

\subsection{Hash Map}
\begin{itemize}
\item Similar to problem 380. To deal with duplicate, we change the value type of hash map to hash set.
\item When remove a value $v$
\begin{enumerate}
\item Find if one of $v$ is at the end of the array $X$. If it is, remove the end index from $v$'s positions.
\item Otherwise, put the last element of $X$ into a position of $v$. We choose the first element in $v$'s hash set as this position, and then remove the end index for last element's positions after inserting the new position.
\end{enumerate}
\end{itemize}

\setcounter{lstlisting}{0}
\begin{lstlisting}[style=customc, caption={Similar to 380}]
class RandomizedCollection
{
public:
    /** Initialize your data structure here. */
    RandomizedCollection()
    {
    }

    /** Inserts a value to the collection. Returns true if the collection did not already contain the specified element. */
    bool insert( int val )
    {
        X.push_back( val );

        auto it = M.find( val );

        if( it == M.end() )
        {
            //insert new val
            M.emplace( val, initializer_list<size_t> {X.size() - 1} );
            return true;
        }

        //val exists, just insert to its positions
        it->second.insert( X.size() - 1 );

        return false;
    }

    /** Removes a value from the collection. Returns true if the collection contained the specified element. */
    bool remove( int val )
    {
        auto it_val = M.find( val );
        if( it_val == M.end() )
        {
            return false;
        }

        //find if val has position equal to last index of X
        auto it_val_idx = it_val->second.find( X.size() - 1 );
        if( it_val_idx == it_val->second.end() )
        {
            //find the last element of X in M
            int last = X.back();
            auto it_last = M.find( last );
            size_t i_val = *( it_val->second.begin() );

            //put last at index i_val
            X[i_val] = last;
            //add the new position into the positions
            it_last->second.insert( i_val );

            //remove the end index from last's positions
            it_last->second.erase( X.size() - 1 );

            //remove begin element from val's positions
            it_val->second.erase( it_val->second.begin() );
        }
        else
        {
            //we just only remove the end index from val's positions
            it_val->second.erase( it_val_idx );
        }

        if( it_val->second.empty() )
        {
            //val does not exist in X
            //remove from M
            M.erase( it_val );
        }


        X.pop_back();

        return true;
    }

    /** Get a random element from the collection. */
    int getRandom()
    {
        return X[rand() % X.size()];
    }

    unordered_map<int, unordered_set<size_t>> M;
    vector<int> X;
};

\end{lstlisting}

