\section{346 --- Moving Average from Data Stream}

\textbf{Easy}

Given a stream of integers and a window size, calculate the moving average of all integers in the sliding window.

\paragraph{Example:}

\begin{flushleft}
\fcj{MovingAverage m = new MovingAverage(3);}

\fcj{m.next(1) = 1}

\fcj{m.next(10) = (1 + 10) / 2}

\fcj{m.next(3) = (1 + 10 + 3) / 3}

\fcj{m.next(5) = (10 + 3 + 5) / 3}
\end{flushleft}

\subsection{Circular Queue}
We could easily implement a circular queue with a fixed-size array. 

The key to the implementation is the correlation between the index of \textbf{head} and \textbf{tail} elements, is:

\fcj{tail = (head+1) \% size}

In other words, the \textbf{tail} element is right next to the \textbf{head} element. Once we move the \textbf{head} forward, we would overwrite the previous \textbf{tail} element.

\setcounter{lstlisting}{0}
\begin{lstlisting}[style=customc, caption={Circular Queue}]
class MovingAverage
{
public:
    /** Initialize your data structure here. */
    MovingAverage( int size )
    {
        vector<int> tmp( size );
        swap( q, tmp );
        head = 0;
        sum = 0;
        count = 0;
    }
    double next( int val )
    {
        ++count;
        //get tail
        auto tail = ( head + 1 ) % q.size();
        //remove tail from the queue
        sum = sum - q[tail] + val;
        //move head to next slot
        head = tail;
        //write val to this slot
        q[head] = val;
        return ( double )sum / ( double )( ( min )( count, q.size() ) );
    }
    vector<int> q;
    size_t head;
    int sum;
    size_t count;
};
\end{lstlisting}