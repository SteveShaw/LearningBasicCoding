\section{397 --- Integer Replacement}
Given a positive integer $ n $ and you can do operations as follow:

\begin{itemize}
\item If $ n $ is even, replace $ n $ with $ n/2 $.
\item If $ n $ is odd, you can replace $ n $ with either $ n + 1 $ or $ n - 1 $.
\end{itemize}

What is the minimum number of replacements needed for $ n $ to become 1?

\paragraph{Example 1:}

\begin{flushleft}
\textbf{Input}: 8
\\
\textbf{Output}:3
\\
\textbf{Explanation}: $8 \rightarrow 4 \rightarrow 2 \rightarrow 1$
\end{flushleft}

\paragraph{Example 2:}

\begin{flushleft}
\textbf{Input}: 7
\\
\textbf{Output}: 4
\\
\textbf{Explanation}: $7 \rightarrow 8 \rightarrow 4 \rightarrow 2 \rightarrow 1$ or 
$7 \rightarrow 6 \rightarrow 3 \rightarrow 2 \rightarrow 1$
\end{flushleft}

\subsection{Dynamic Programming}
\begin{itemize}
\item Best to use top down dynamic programming with memorization.
\item Suppose $F[n]$ is the minimum number of replacements needed for $n$ to become 1, then 
\begin{enumerate}
\item For an odd number, either we change $n$ to $n-1$ or to $n+1$ then divide by 2. There are two replacements before go to $(n+1)/2$ or $(n-1)/2$, so $F[n]=2+\min(F[(n-1)/2], F[(n+1)/2])$. 
\item For an even number, we divide $n$ by 2, so $F[n] = 1 + F[n/2]$
\end{enumerate}
\end{itemize}

\setcounter{lstlisting}{0}
\begin{lstlisting}[style=customc, caption={Top Down Dynamic Programming}]
int integerReplacement( int n )
{
    //memo
    unordered_map<long long, int> memo;

    return dfs( static_cast<long long>( n ), memo );
}

int dfs( long long n, unordered_map<long long, int>& memo )
{
    if( n == 1 )
    {
        return 0;
    }

    auto it = memo.find( n );

    if( it != memo.end() )
    {
        return it->second;
    }

    if( n & 1 )
    {
        int x = dfs( ( n + 1 ) / 2, memo );
        int y = dfs( ( n - 1 ) / 2, memo );

        //two replacements before go to (n+1)/2 or (n-1)/2
        int z = ( min )( x, y ) + 2;
        memo.emplace( n, z );

        return z;
    }

    int z = 1 + dfs( n / 2, memo );
    memo.emplace( n, z );
    return z;
}
\end{lstlisting}