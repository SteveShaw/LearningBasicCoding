\section{327 --- Count of Range Sum}
Given an integer array \fcj{nums}, return the number of range sums that lie in \fcj{[lower, upper]} inclusive.

Range \fcj{sum S(i, j)} is defined as the sum of the elements in \fcj{nums} between indices $i$ and $j$ ($i \leq j$), inclusive.

\paragraph{Note:}
\begin{itemize}
\item A naive algorithm of $ O(n^2) $ is trivial. You MUST do better than that.
\end{itemize}

\paragraph{Example:}

\begin{flushleft}
\textbf{Input}: \fcj{nums = [-2,5,-1]}, \fcj{lower = -2}, \fcj{upper = 2},
\\
\textbf{Output}: 3 
\\
\textbf{Explanation}: 

The three ranges are : \fcj{[0,0]}, \fcj{[2,2]}, \fcj{[0,2]} and their respective sums are: $-2$, $-1$, $2$.

\end{flushleft}
\subsection{Prefix Sum}
\begin{itemize}
\item 获得$A$的prefix sum array $Z$。
\item 对于当前的prefix sum $Z[j]$,要找到某个$Z[i]$,使得$l\leq Z[j]-Z[i]\leq u$。即$Z[j]-u\leq Z[i]\leq Z[j]+l$。这个分别通过leftmost and rightmost binary search可以快速定位。
\item 由于prefix sum可能会相同,因此tree map的value是一个array,用来存放对应的index。
\item 最后统计有多少个index满足上述要求。
\item 为了防止数据类型越界,map和prefix sum array的数据类型都用\texttt{long long}。
\end{itemize}
\setcounter{lstlisting}{0}
\begin{lstlisting}[style=customc, caption={Tree Map And Binary Search}]
int countRangeSum( vector<int>& nums, int lower, int upper )
{
    vector<long long> Z( nums.size() + 1, 0 );

    map<long long, vector<int>> m;
    m.emplace( 0, initializer_list<int> {-1} );

    int ans = 0;

    int L = static_cast< int >( nums.size() );

    for( int i = 1; i <= L; ++i )
    {
        Z[i] = Z[i - 1] + nums[i - 1];

        //find the first key in m
        //which is equal or larger than Z[i]- upper
        auto it_low = m.lower_bound( Z[i] - upper );

        if( it_low != m.end() )
        {
            //If this key exists
            //then search for the first key in m
            //which is larger than Z[i]-lower
            auto it_high = m.upper_bound( Z[i] - lower );

            if( it_high != m.begin() )
            {
                //The numbers between *it_low and *it_hight
                //are all candidates. Then accumulate their
                //counts, i.e. the size of the array
                for( auto it = it_low; it != it_high; ++it )
                {
                    ans += static_cast< int >( it->second.size() );
                }
            }
        }

        //Put Z[i] into m
        auto it = m.find( Z[i] );
        if( it == m.end() )
        {
            m.emplace( Z[i], std::initializer_list<int> {i} );
        }
        else
        {
            it->second.push_back( i );
        }
    }

    return ans;
}

//using multiset
int countRangeSum( vector<int>& nums, int lower, int upper )
{
    long long sum = 0;
    multiset<long long> m;
    m.insert( 0 );
    int ans = 0;
    for( int n : nums )
    {
        sum += n;
        //find how many sums
        //such that sum -s <=upper
        //and sum - s > lower;
        //i.e.  sum - upper <= s < sum-lower
        auto hi = m.upper_bound( sum - lower );
        auto lo = m.lower_bound( sum - upper );
        ans += ( int )distance( lo, hi );
        //add sum to m
        m.insert( sum );
    }
    return ans;
}
\end{lstlisting}
\subsection{Merge Sort}
\begin{itemize}
\item 仍然是计算prefix sum。
\item 假设当前的left part和right part都已经sorted。当遍历left part时,对于每个index $i$,需要在right part中找出两个index $k$和$j$,使得
\begin{enumerate}
\item $ Z[j] $ is the first such that $Z[j] - Z[i] > u$
\item $ Z[k] $ is the first such that $ Z[k] - Z[i] \geq l $
\end{enumerate}
这样满足difference为$ [l,u] $之间的区间的个数就是$j-k$。
\item 另外由于这是merge sort,需要另外一个index $\alpha$,用来copy所有满足$Z[\alpha]<Z[i]$的prefix sum到一个cache中以便完成排序的过程。
\item 同样为了避免数据类型越界,prefix sum的类型也是\texttt{long long}。
\end{itemize}
\begin{lstlisting}[style=customc, caption={Merge Sort}]
int countRangeSum( vector<int>& nums, int lower, int upper )
{
    //get prefix sum
    vector<long> sums( nums.size() + 1, 0 );
    for( int i = 0; i < nums.size(); ++i )
    {
        sums[i + 1] = sums[i] + nums[i];
    }
    //call merge sort function
    return countAndMergeSort( sums, 0, sums.size(), lower, upper );
}
int countAndMergeSort( vector<long> &sums, int start, int end, int lower, int upper )
{
    if( end - start <= 1 ) return 0;

    int mid = start + ( end - start ) / 2;
    //split sums into [start,mid-1] and [mid, end]
    int cnt = countAndMergeSort( sums, start, mid, lower, upper ) + countAndMergeSort( sums, mid, end, lower, upper );

    int j = mid, k = mid, t = mid;

    //cache will have the sorted sums elements between start and end
    vector<int> cache( end - start, 0 );

    for( int i = start, r = 0; i < mid; ++i, ++r )
    {
        //find the first sums[k] in right part
        //so that sums[k] - sum[i] >= lower
        while( ( k < end ) && ( sums[k] - sums[i] < lower ) )
        {
            ++k;
        }

        //find the first sums[j] in right part
        //so that sums[j] - sum[i] < upper
        while( ( j < end ) && ( sums[j] - sums[i] <= upper ) )
        {
            ++j;
        }

        //Put all numbers in the right half that less than sums[i]
        //into cache
        while( ( t < end ) && ( sums[t] < sums[i] ) )
        {
            cache[r++] = sums[t++];
        }
        cache[r] = sums[i];

        //The indices between [k,j] are
        //making the range sum fall between [lower, upper]
        cnt += j - k;
    }
    copy( cache.begin(), cache.begin() + t - start, sums.begin() + start );
    return cnt;
}
\end{lstlisting}

We can use STL function \fcc{inplace_merge} to simplify the code

\begin{lstlisting}[style=customc, caption={STL Merge Sort}]
using ll_t = long long;
int countRangeSum( vector<int>& nums, int lower, int upper )
{
    vector<ll_t> psums( nums.size() + 1, 0 );
    for( size_t i = 0; i < nums.size(); ++i )
    {
        psums[i + 1] = nums[i] + psums[i];
    }
    return merge_sort( begin( psums ), end( psums ), lower, upper );
}
//merge sort
int merge_sort( vector<ll_t>::iterator beg, vector<ll_t>::iterator end, int lo, int hi )
{
    auto dist = distance( beg, end );
    if( dist <= 1 )
    {
        return 0;
    }
    auto mid = next( beg, dist / 2 );
    int count = merge_sort( beg, mid, lo, hi ) + merge_sort( mid, end, lo, hi );
    //we need two iterators
    //one records the beginning of number x that x - *i >=lo
    //the other records the beginning of number y that y -*i >=hi
    for( auto i = beg, j1 = mid, j2 = mid; i != mid; ++i )
    {
        while( ( j1 != end ) && ( *j1 - *i < lo ) )
        {
            ++j1;
        }
        while( ( j2 != end ) && ( *j2 - *i <= hi ) )
        {
            ++j2;
        }
        count += ( int )distance( j1, j2 );
    }
    inplace_merge( beg, mid, end );
    return count;
}
\end{lstlisting}


\paragraph{Related Problems}
\begin{itemize}
\item \textbf{315. Count of Smaller Numbers After Self}
\item \textbf{493. Reverse Pairs}
\end{itemize}