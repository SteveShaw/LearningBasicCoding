\section{354 --- Russian Doll Envelopes}
You have a number of envelopes $A$ with widths and heights given as a pair of integers $(w, h)$. One envelope can fit into another if and only if both the width and height of one envelope is greater than the width and height of the other envelope.

What is the maximum number of envelopes can you Russian doll? (put one inside other)

\paragraph{Note:}
\begin{itemize}
\item Rotation is not allowed.
\end{itemize}

\paragraph{Example:}

\begin{flushleft}
\textbf{Input}: $\left[\;[5,4],\ [6,4],\ [6,7],\ [2,3]\;\right]$
\\
\textbf{Output}: 3 
\\
\textbf{Explanation}: The maximum number of envelopes you can Russian doll is 3 ($[2,3] \longrightarrow [5,4] \longrightarrow [6,7]$).
\end{flushleft}

\subsection{Sorting}
这是longest increasing sequence的二维扩展。
\begin{itemize}
\item 对输入的数组进行排序,首先按照width从小打到大排序,如果width相等,则按照height从大到小排序。由于width是从小到大的,所以实际上是寻找height的LIS。
\item 之所以要在width相同的时候要把大的height放前面,用一个例子来说明: 例如两个doll分别为$[3, 4]$和$[3, 3]$,如果把$[3,3]$放在$[3,4]$前,那么在计算LIS时,会把$[3,3]$也计算进去,但是$[3,4]$不能包含$[3,3]$。
\end{itemize}

\setcounter{lstlisting}{0}
\begin{lstlisting}[style=customc, caption={LIS}]
int maxEnvelopes( vector<pair<int, int>>& envelopes )
{
    //sort envelopes according to ascending width
    //if widths are equal, put larger height first
    sort( envelopes.begin(), envelopes.end(), []( const pair<int, int>& p1, const pair<int, int>& p2 )
    {
        if( p1.first < p2.first )
        {
            return true;
        }

        if( p1.first == p2.first )
        {
            return p1.second > p2.second;
        }

        return false;
    } );

    vector<int> F;

    //find length of LIS
    for( const auto& evl : envelopes )
    {
        if( F.empty() )
        {
            F.push_back( evl.second );
            continue;
        }

        auto x = leftmost( F, evl.second );

        if( x == F.size() )
        {
            F.push_back( evl.second );
        }
        else
        {
            F[x] = evl.second;
        }
    }

    return static_cast<int>( F.size() );
}

size_t leftmost( vector<int>& F, int h )
{
    size_t l = 0;
    auto r = F.size();

    while( l < r )
    {
        auto mid = ( l + r ) / 2;

        if( F[mid] < h )
        {
            l = mid + 1;
        }
        else
        {
            r = mid;
        }
    }

    return l;
}

\end{lstlisting}