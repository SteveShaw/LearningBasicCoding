\section{343 --- Integer Break}
Given a positive integer $ n $, break it into the sum of at least two positive integers and maximize the product of those integers. Return the maximum product you can get.

\paragraph{Example 1:}

\begin{flushleft}
\textbf{Input}: 2
\\
\textbf{Output}: 1
\\
\textbf{Explanation}: $2 = 1 + 1$, $1 \times 1 = 1$.
\end{flushleft}

\paragraph{Example 2:}

\begin{flushleft}
\textbf{Input}: 10
\\
\textbf{Output}: 36
\\
\textbf{Explanation}: $10 = 3 + 3 + 4$, $3 \times 3 \times 4 = 36$.
\end{flushleft}

\paragraph{Note:} 
\begin{itemize}
\item You may assume that $ n $ is not less than 2 and not larger than 58.
\end{itemize}

\subsection{Dynamic Programming}
\begin{itemize}
\item 假设$ F(n) $为整数$n$分解后能得到的最大product,我们需要从1到$ n-1 $逐个进行分解测试,
\item 如果当前测试的数为$ i $,这时候$ n $ 分解成 $ i $和$ n-i $两个数字,那么有两种情况
\begin{enumerate}
\item $ n-i $不再进行分解,这时候的product为$i \times (n-i)$
\item $n-i$ 继续分解,所能得到的最大product为 $ F(n-i) $, 而$ n $能得到的product为 $ i\times F[n-i] $
\item 取这两者的最大值与当前$ F(n) $进行比较,将$ F(n) $ update为较大值。
\end{enumerate}

\begin{algorithm}[H]
\caption{Dynamic Programming}
\begin{algorithmic}[1]
\Procedure{IntegerBreak}{$ n $}
\State $\star$ Initialize an array $F$ with size $n+1$ with all elements set to 1
\State $\ast$ Following is the dynamic programming process
\For{$ i:=3 $ \textbf{to} $ n $}
\For{$ j:=1 $ \textbf{to} $i-1$}
\algstore{343algo}
\end{algorithmic}
\end{algorithm}
\begin{algorithm}[H]
\begin{algorithmic}[1]
\algrestore{343algo}
\State $F[i]\gets \max(F[i], i\times (n-i))$ \Comment Treat $n-i$ as one number
\State $F[i]\gets \max(F[i], i\times F[n-1])$ \Comment Get maximum product by dissecting $n-i$
\EndFor
\EndFor
\State \Return $F[n]$
\EndProcedure
\end{algorithmic}
\end{algorithm}
\end{itemize}

\setcounter{lstlisting}{0}
\begin{lstlisting}[style=customc, caption={Dynamic Programming}]
int integerBreak( int n )
{
    vector<int> F( n + 1, 1 );

    for( int i = 3; i <= n; ++i )
    {
        for( int j = 1; j < i; ++j )
        {
            F[i] = ( max )( F[i], ( i - j ) * j ); // (i-j) is one number
            F[i] = ( max )( F[i], j * F[i - j] ); // dissecting (i-j)
        }
    }

    return F[n];
}

\end{lstlisting}