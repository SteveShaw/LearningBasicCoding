\section{380 --- Insert Delete GetRandom O(1)}
Design a data structure that supports all following operations in average O(1) time.

\begin{lstlisting}[style=customc]
insert(val) //Inserts an item val to the set if not already present.
remove(val) //Removes an item val from the set if present.
getRandom // Returns a random element from current set of elements. Each element must have the same probability of being returned.
\end{lstlisting}

\paragraph{Example:}

\begin{lstlisting}[style=customc]
// Init an empty set.
RandomizedSet randomSet = new RandomizedSet();

// Inserts 1 to the set. Returns true as 1 was inserted successfully.
randomSet.insert(1);

// Returns false as 2 does not exist in the set.
randomSet.remove(2);

// Inserts 2 to the set, returns true. Set now contains [1,2].
randomSet.insert(2);

// getRandom should return either 1 or 2 randomly.
randomSet.getRandom();

// Removes 1 from the set, returns true. Set now contains [2].
randomSet.remove(1);

// 2 was already in the set, so return false.
randomSet.insert(2);

// Since 2 is the only number in the set, getRandom always return 2.
randomSet.getRandom();
\end{lstlisting}

\subsection{Hash Map}
\begin{itemize}
\item Maintain one array $X$ and one hash map $M$.
\item Use $X$ to store numbers and $M$ to map the number and its index in $X$
\item \texttt{Insert} operation: 
\begin{enumerate}
\item Check if the number exists in $M$. If Yes, then return \texttt{false}。Otherwise, we insert the number into the end of $X$, and insert to $M$ with its index in $X$.
\end{enumerate}
\item \texttt{Delete} operation:
\begin{enumerate}
\item To delete the number from the array $X$ in $O(1)$ time, swap the number with the last number. 
\item Since the number's index is changed, we also need to modify its mapping in $M$.
\item Then, we just only delete the last number in $X$.
\end{enumerate}
\item \texttt{getRandom} operation: Generate a random integer $i$ between $[0, \lvert X\rvert-1]$, and returns the number at index $i$.
\end{itemize}

\setcounter{lstlisting}{0}
\begin{lstlisting}[style=customc, caption={Hash Map}]
class RandomizedSet
{
public:
    /** Initialize your data structure here. */
    RandomizedSet()
    {
    }

    /** Inserts a value to the set. Returns true if the set did not already contain the specified element. */
    bool insert( int val )
    {
        auto it = M.find( val );
        if( it != M.end() )
        {
            return false;
        }

        X.push_back( val );
        M.emplace( val, X.size() - 1 );

        return true;
    }

    /** Removes a value from the set. Returns true if the set contained the specified element. */
    bool remove( int val )
    {
        auto it = M.find( val );
        if( it == M.end() )
        {
            return false;
        }

        auto pos = it->second;
        if( pos != X.size() - 1 )
        {
            //Copy the last element
            //to the position of val
            X[pos] = X.back();
            auto it_back = M.find( X.back() );
            it_back->second = pos;
        }

        //remove the last element
        X.pop_back();
        M.erase( it );

        return true;
    }

    /** Get a random element from the set. */
    int getRandom()
    {
        return X[rand() % X.size()];
    }

    vector<int> X;
    unordered_map<int, size_t> M;
};
\end{lstlisting}