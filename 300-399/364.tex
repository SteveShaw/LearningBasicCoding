\section{364 --- Nested List Weight Sum II}

\textbf{Medium}

Given a nested list of integers, return the sum of all integers in the list weighted by their depth.

Each element is either an integer, or a list -- whose elements may also be integers or other lists.

Different from the previous question where weight is increasing from root to leaf, now the weight is defined from bottom up. i.e., the leaf level integers have weight 1, and the root level integers have the largest weight.

\paragraph{Example 1:}

\begin{flushleft}
\textbf{Input}: \fcj{[[1,1],2,[1,1]]}

\textbf{Output}: 8 

Explanation: Four 1's at depth 1, one 2 at depth 2.
\end{flushleft}

\paragraph{Example 2:}

\begin{flushleft}
\textbf{Input}: \fcj{[1,[4,[6]]]}

\textbf{Output}: 17 

\textbf{Explanation}: One 1 at depth 3, one 4 at depth 2, and one 6 at depth 1; \fcj{1*3 + 4*2 + 6*1 = 17}.
\end{flushleft}

\subsection{DFS}
In \textbf{Nested List Weight Sum I} we can easily calculate the weighted sum because depth increases from root to leaf.

Here we can use the same idea to calculate the reverse sum by using the weighted sum.

Suppose we have a nested list, for example, \fcj{[a, [b], [[c]]]}. Then the value of weighted sum and reverse weighted sum will be

\fcj{weightedSum = 1a + 2b + 3c}

\fcj{reverseSum = 3a + 2b + 1c}

adding the two we get

\fcj{totalSum = 4*(a+b+c)}

Thus we can easily conclude that

\fcj{reverseSum = (maxDepth+1)* (a+b+c) - weightedSum}

Then we can use similar DFS solution to solve this problem.

\setcounter{lstlisting}{0}
\begin{lstlisting}[style=customc, caption={DFS}]
int depthSumInverse( vector<NestedInteger>& nestedList )
{
    int unweight_sum = 0;
    int weight_sum = 0;
    int max_depth = 1;
    for( const auto& ni : nestedList )
    {
        dfs( ni, 1, max_depth, unweight_sum, weight_sum );
    }
    //use the formula
    return unweight_sum * ( max_depth + 1 ) - weight_sum;
}
void dfs( const NestedInteger& ni, int depth, int& max_depth, int& sum, int& w_sum )
{
    //update maximum depth
    max_depth = ( max )( max_depth, depth );
    if( ni.isInteger() )
    {
        //update flat and weighted sum
        sum += ni.getInteger();
        w_sum += depth * ni.getInteger();
    }
    else
    {
        for( const auto& ni : ni.getList() )
        {
            //recursively to update flat and weighted sum
            dfs( ni, depth + 1, max_depth, sum, w_sum );
        }
    }
}
\end{lstlisting}

\paragraph{Related Problems}
\begin{itemize}
\item \textbf{339. Nested List Weight Sum}
\item \textbf{565. Array Nesting}
\end{itemize}
