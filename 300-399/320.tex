\section{320 --- Generalized Abbreviation}

\textbf{Medium}

Write a function to generate the generalized abbreviations of a word. 

Note: The order of the output does not matter.

\textbf{Example:}

\begin{flushleft}
\textbf{Input}: \fcj{"word"}

\textbf{Output}:

\begin{lstlisting}[style=customc]
["word", "1ord", "w1rd", "wo1d", "wor1", "2rd", "w2d", "wo2", "1o1d", "1or1", "w1r1", "1o2", "2r1", "3d", "w3", "4"]
\end{lstlisting}
\end{flushleft}

\subsection{Backtracking}

In this problem, the partial candidates in backtracking are incomplete abbreviations that can be extended by one of the two choices:

\begin{enumerate}
\item keep the next character;
\item abbreviate the next character.
\end{enumerate}

\setcounter{lstlisting}{0}
\begin{lstlisting}[style=customc, caption={Backtracking}]
vector<string> generateAbbreviations( string word )
{

    vector<string> ans;
    dfs( word, 0, 0, "", ans );
    return ans;
}
//k: the count of consecutive abbreviated characters
//start: the current index in w
void dfs( string_view w, size_t start, int k, string cur, vector<string>& ans )
{
    if( start == w.size() )
    {
        if( k > 0 )
        {
            //append last k if it is nonzero
            cur += to_string( k );
        }
        ans.emplace_back( cur );
        return;
    }
    //the branch that w[i] is abbreviated
    dfs( w, start + 1, k + 1, cur, ans );
    if( k > 0 )
    {
        cur += to_string( k );
    }
    cur.push_back( w[start] );
    // the branch that w[i] is kept
    // in this case, k is zero
    dfs( w, start + 1, 0, cur, ans );
}
\end{lstlisting}

\subsection{Bit Operation}
If we use 0 to represent a character that is not abbreviated and 1 to represent one that is. Then each abbreviation is mapped to an nn bit binary number and vice versa.

\begin{lstlisting}[style=customc, caption={Bit}]
vector<string> generateAbbreviations( string word )
{
    auto l = word.size();
    auto x = 1ull << l;
    vector<string> ans;
    for( auto i = 0ull; i < x; ++i )
    {
        ans.emplace_back( help( word, i ) );
    }
    return ans;
}
string help( string_view word, size_t t )
{
    int k = 0;
    auto l = word.size();

    string s;
    for( size_t i = 0; i < l; ++i )
    {
        if( ( t & 1 ) == 0 )
        {
            if( k )
            {
                //add abbr to s
                s += to_string( k );
                //reset k
                k = 0;
            }
            s.push_back( word[i] );
        }
        else
        {
            ++k;
        }

        t >>= 1;
    }

    if( k )
    {
        s += to_string( k );
    }

    return s;
}
\end{lstlisting}

\paragraph{Related Problems}
\begin{itemize}
\item \textbf{78. Subsets}
\item \textbf{288. Unique Word Abbreviation}
\item \textbf{411. Minimum Unique Word Abbreviation}
\end{itemize}