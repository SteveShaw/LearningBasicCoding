\section{398 --- Random Pick Index}
Given an array of integers with possible duplicates, randomly output the index of a given target number. You can assume that the given target number must exist in the array.

\paragraph{Note:}
\begin{itemize}
\item The array size can be very large. Solution that uses too much extra space will not pass the judge.
\end{itemize}

\paragraph{Example:}


\begin{lstlisting}[style=customc]
//Example:
int[] nums = new int[] {1,2,3,3,3};
Solution solution = new Solution(nums);

// pick(3) should return either index 2, 3, or 4 randomly. 
//Each index should have equal probability of returning.
solution.pick(3);

// pick(1) should return 0. Since in the array only nums[0] is equal to 1.
solution.pick(1);
\end{lstlisting}

\subsection{Reservoir Sampling}
\begin{itemize}
\item 仍旧是用size为1的reservoir sampling
\item Maintain two variables: $\delta$ and $x$.
\item 然后遍历数组,如果当前数不等于目标值,直接skip。
\item 否则,$\delta$ increments,然后generate a random number $r$ between $[0, \delta-1]$。如果$r=0$,将$x$ update为当前的index。
\item 遍历完成后,返回$x$。
\end{itemize}

\setcounter{lstlisting}{0}
\begin{lstlisting}[style=customc, caption={Reservoir Sampling}]
int pick( int target, vector<int>& A )
{
    int count = 0;
    int index = 0;
    int x = -1;

    for( int n : A )
    {
        if( n == target )
        {
            //reservoir sampling
            ++count;

            //generate a random number between [0, count]
            int r = ( rand() % count );

            if( r == 0 )
            {
                x = index;
            }
        }

        ++index;
    }

    return x;
}
\end{lstlisting}

