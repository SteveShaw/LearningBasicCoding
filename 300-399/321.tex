\section{321 --- Create Maximum Number}
Given two arrays of length $ m $ and $ n $ with digits 0--9 representing two numbers. Create the maximum number of length $ k \leq m + n $ from digits of the two. The relative order of the digits from the same array must be preserved. Return an array of the k digits.

\paragraph{Note:} 
\begin{itemize}
\item You should try to optimize your time and space complexity.
\end{itemize}

\paragraph{Example 1:}

\begin{flushleft}
\textbf{Input}:
\\
$ A  = [3,\;4,\; 6,\; 5]$
\\
$ B  = [9,\; 1,\; 2, \;5,\; 8,\;3]$
\\
$ k = 5 $
\\
\textbf{Output}: $ [9,\; 8,\; 6,\; 5,\; 3] $
\end{flushleft}

\paragraph{Example 2:}

\begin{flushleft}
\textbf{Input}:
\\
$ A = [6,\; 7] $
\\
$ B = [6,\; 0,\; 4] $
\\
$ k = 5 $
\\
\textbf{Output}: $ [6,\; 7, \;6,\; 0,\; 4] $
\end{flushleft}

\paragraph{Example 3:}

\begin{flushleft}
\textbf{Input}:
\\
$ A = [3,\; 9] $
\\
$ B = [8,\; 9] $
\\
$ k = 3 $
\\
\textbf{Output}: $ [9, \;8, \;9] $
\end{flushleft}
\subsection{Greedy Way}
\begin{enumerate}
\item 首先要确定每个数组中要取几个数,这个只能通过循环一个个测试。比如如果$A$选择取$x$个数,那么$B$就只能选择取$y=k-x$个数
\item For each iteration,对于每个数组,用贪心法得到可能组成的最大数。
\item 然后merge这两个数组得到的最大数,有点类似于merge sort的合并过程。但是需要定义如何进行两个array的比较,这里需要采用lexically comparison的方法。
\item 最后选择所有测试中的最大数。这个过程同样要基于对两个array进行lexically compare的方法。
\item 在对每个数组用贪心的方法取得组成的最大数时,可能最后产生的最大数的长度包含多于给定长度。这是因为在generate的过程中,数组的后面是否还有更大的数,需要不断的放入结果数组中,所以最后需要把多出来的数字去除掉。 \label{step0}
\item 另外在对每个数组用贪心的方法取得组成的最大数的过程中,需要用一个变量来记录该数组中需要drop掉多少个number,即数组长度减去需要的长度。如果没有这个变量来控制,很可能生成的最大数的长度会小于需要的长度。
\end{enumerate}
\setcounter{algorithm}{0}
\begin{algorithm}[H]
\caption{Greedy Way}
\begin{algorithmic}[1]
\Procedure{MaxNumber}{$A, m, B, n, k$}
\State $\ast$ Suppose $x$ is the count of numbers selected from $A$.
\State $\ast$ The minimum for $ x $ is $ \max(k-n, 0) $ and the maximum is $ \min(k, m) $
\State $N:=\emptyset$ \Comment The generated result 
\For{$x:=\max(k-n, 0)$ \textbf{to} $ \min(k, m) $}
\State $ a := $ \Call{Process}{$ A, m, x $} \Comment Generate maximum number array $ a $ from $A$
\State $ b := $ \Call{Process}{$ B, n, k-x $} \Comment Generate maximum number array $ b $ from $B$
\State $ Z:= $ \Call{Merge}{$ a, x, b, k-x $} \Comment Merge $a$ and $b$ based on lexical order
\If{\Call{Compare}{$ N,k,0, Z,k,0 $} is \texttt{true}}
\State $N\gets Z$ \Comment $ N $ is lexically smaller than $ Z $。
\EndIf
\EndFor
\State \Return $N$
\EndProcedure
\end{algorithmic}
\end{algorithm}
Function \texttt{Process} 则是对于任意一个大小为$ L $的输入数组$ T $,generate长度为$ n $的最大数字数组。
\begin{algorithm}[H]
\caption{Generate Maximum Number Array with size $ n $}
\begin{algorithmic}[1]
\Function{Process}{$T, L, n$}
\State $R:=\emptyset$ \Comment The result maximum number array
\State $\delta:=L-n$ \Comment The count of numbers to be dropped from the result maximum number array
\For{Each number $ t $ in $ T $}
\State $\ast$ If $\delta$ is zero, we cannot drop anymore because $R$ needs $n$ digits
\While{$\delta > 0$ \textbf{and} $ R\neq \emptyset $ \textbf{and} $ R[-1] < t $ }
\State $\star$ Drop the last element $R[-1]$ from $R$
\State $\delta\gets\delta-1$
\EndWhile
\State $\star$ Add $n$ to $R$
\EndFor
\State $\star$ Because of drop, $R$ may have more than $n$ digits
\State $\star$ Only keep the first $n$ digits in $R$
\State \Return $ R $
\EndFunction
\end{algorithmic}
\end{algorithm}
Function \texttt{Merge} 则按照lexical order将两个array $ A $和$ B $进行merge,以得到这两个array所能做成的最大number array $M$。
\begin{algorithm}[H]
\caption{Merge two arrays based on lexical order to get maximum number array}
\begin{algorithmic}[1]
\Procedure{Merge}{$A, m, B, n$}
\State $x:=0$ \Comment Index for $A$
\State $y:=0$ \Comment Index for $B$
\State $M:=\emptyset$ \Comment The result string
\State $p:=0$ \Comment Index for $M$
\While{$x<m$ \textbf{or} $ y < n $}
\If{\Call{Compare}{$ A, m, x, B, n, y $} is \texttt{true}}
\State $M[p]:=B[y]$ \Comment $A$ is lexically smaller than $B$, so add $B[y]$ to $M$
\Else
\State $M[p]:=A[x]$ \Comment $A$ is lexically equal or larger than $B$, so add $A[x]$ to $M$
\EndIf
\State $p\gets p+1$ \Comment Move to next slot in $M$
\EndWhile
\algstore{321algo}
\end{algorithmic}
\end{algorithm}
\begin{algorithm}[H]
\begin{algorithmic}[1]
\algrestore{321algo}
\State \Return $M$
\EndProcedure
\end{algorithmic}
\end{algorithm}
最后是Function \texttt{Compare},对两个数组$A$和$B$进行lexically comparison,其中$A$ starts with index $x$ and $ B $ starts with $y$。 如果返回\texttt{true},则表示$A[x\ldots m-1]$ is less than $B[y\ldots n-1]$。
\begin{algorithm}[H]
\caption{Lexically Compare Two Arrays With Different Start Index}
\begin{algorithmic}[1]
\Procedure{Compare}{$A,m,x,B,n,y$}
\While{$x<m$ \textbf{and} $ y<n $}
\If{$A[x] < B[y]$}
\State \Return \texttt{true} \Comment $A$ is smaller than B
\ElsIf{$A[x] > B[x]$}
\State \Return \texttt{false} \Comment $A$ is larger than B
\EndIf
\EndWhile
\State $\star$ 如果算法能到这里,说明 $A[x\ldots m-1]=B[y\ldots n-1]$
\State $\ast$ 需要比较$A$和$B$的长度,
\If{$x=m$ \textbf{and} $ y < n $}
\State \Return \texttt{true} \Comment $A$ is shorter than $B$, so $A[x\ldots m-1]<B[y\ldots n-1]$
\EndIf
\State \Return \texttt{false} \Comment Otherwise, $A[x\ldots m-1]\geq B[y\ldots n-1]$
\EndProcedure
\end{algorithmic}
\end{algorithm}
\setcounter{lstlisting}{0}
\begin{lstlisting}[style=customc, caption={Greedy Approach}]
vector<int> maxNumber( vector<int>& nums1, vector<int>& nums2, int k )
{
    int l1 = static_cast<int>( nums1.size() );
    int l2 = static_cast<int>( nums2.size() );

    //The count of numbers selected from nums1 array
    int sel1 = ( max )( k - l2, 0 );

    //This is the maximum count of numbers
    //can be selected from nums1 array
    int max_sel1 = ( min )( k, l1 );

    vector<int> ans( k, INT_MIN );

    vector<int> merged( k, 0 );

    vector<int> v_maxFrom1;
    vector<int> v_maxFrom2;

    v_maxFrom1.reserve( ( min )( k, l1 ) );
    v_maxFrom2.reserve( ( min )( k, l2 ) );

    for( ; sel1 <= max_sel1; ++sel1 )
    {
        process( nums1, sel1, v_maxFrom1 );
        //k-sel1 is the count of numbers selected from nums2 array
        process( nums2, k - sel1, v_maxFrom2 );

        merge( v_maxFrom1, v_maxFrom2, merged );

        //lexically compare two arrays
        if( lexical_compare( ans, 0, merged, 0 ) )
        {
            swap( ans, merged );
        }

        v_maxFrom1.clear();
        v_maxFrom2.clear();
    }

    return ans;
}

void process( const vector<int>& A, int sel, vector<int>& out )
{
    if( sel == 0 )
    {
        return;
    }

    //drop will control how many numbers
    //will be kept in out array
    int drop = static_cast<int>( A.size() ) - sel;

    for( int n : A )
    {
        while( drop && !out.empty() && ( out.back() < n ) )
        {
            out.pop_back();
            --drop;
        }

        out.push_back( n );
    }


    //since we do not increment drop
    //the final kept numbers will always no less than sel
    out.resize( sel );
}

//Use lexically compare function to compare two arrays
void merge( const vector<int>& v1, const vector<int>& v2, vector<int>& out )
{
    int l1 = static_cast< int >( v1.size() );
    int l2 = static_cast< int >( v2.size() );

    int x = 0;
    int y = 0;

    int p = 0;
    while( ( x < l1 ) || ( y < l2 ) )
    {
        if( lexical_compare( v1, x, v2, y ) )
        {
            out[p] = v2[y];
            ++p;
            ++y;
        }
        else
        {
            out[p] = v1[x];
            ++p;
            ++x;
        }
    }


}

//The routine to compare arrays lexically
bool lexical_compare( const vector<int>& A, int x, const vector<int>& B, int y )
{
    int m = static_cast< int >( A.size() );
    int n = static_cast< int >( B.size() );

    while( ( x < m ) && ( y < n ) )
    {
        if( A[x] < B[y] )
        {
            return true;
        }

        if( A[x] > B[y] )
        {
            return false;
        }

        ++x;
        ++y;
    }

    //A's length is smaller than B's length
    //and all A's numbers equal to previous m numbers in B
    return ( x == m ) && ( y < n );
}
};
\end{lstlisting}