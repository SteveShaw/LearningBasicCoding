\section{386 --- Lexicographical Numbers}
Given an integer $n$, return 1 --- $n$ in lexicographical order.
\par
For example, given 13, return: $[1,\;10,\;11,\;12,\;13,\;2,\;3,\;4,\;5,\;6,\;7,\;8,\;9]$.
\par
Please optimize your algorithm to use less time and space. The input size may be as large as 5,000,000

\subsection{DFS}
Just repeatedly try from 1 to 9, $1 \to 10 \to 100$ first, and then plus 1 to the deepest number. Take 13 as example:
\begin{itemize}
\item $1\to 10 \to 100$, but 100 is larger than 13, then
\item $11\to 110$, but 110 is larger than 13, then
\item $12\to 120$, but 120 is larger than 13, then
\item $13\to 130$, but 130 is larger than 13, then
\item $2 \to 20$, but 20 is larger than 13, then
\item $3 \to 30$, but 30 is larger than 13, then
\item $\ldots$
\item $9\to 90$, then we stop
\end{itemize}

\setcounter{algorithm}{0}
\begin{algorithm}[H]
\caption{DFS}
\begin{algorithmic}[1]
\Procedure{LexicalOrder}{$n$}
\State $\star$ Create an array $A$ to store the results
\State \Call{DFS}{$1, n, A$} \Comment Call helper function
\State \Return $A$
\EndProcedure
\end{algorithmic}
\end{algorithm}

Function \texttt{DFS} generate the sequence with lexical order from number $x$ to $n$ and save the results to $A$

\begin{algorithm}[H]
\caption{Helper Function}
\begin{algorithmic}[1]
\Function{DFS}{$x, n, A$}
\If{$x > n$}
\State \Return
\EndIf
\algstore{386algo}
\end{algorithmic}
\end{algorithm}
\begin{algorithm}[H]
\begin{algorithmic}[1]
\algrestore{386algo}
\State $A\gets A+n$ \Comment Add current number to $A$
\State \Call{DFS}{$x\times 10, n, A$} \Comment try ten times,i.e. add zero
\If{$x\bmod 10 \neq 9$} \Comment $x$ is not end with number 9
\State \Call{DFS}{$x+1, n, A$} \Comment try plus 1
\EndIf
\EndFunction
\end{algorithmic}
\end{algorithm}

\setcounter{lstlisting}{0}
\begin{lstlisting}[style=customc, caption={DFS}]
vector<int> lexicalOrder( int n )
{
    vector<int> ans;

    dfs( 1, n, ans );

    return ans;

}
//helper function to generate
//lexical order sequence from x to n
void dfs( int x, int n, vector<int>& ans )
{
    if( x > n )
    {
        return;
    }

    ans.push_back( x );

    //try to add zeros to x
    dfs( x * 10, n, ans );

    if( ( x % 10 ) != 9 )
    {
        //try next number to x
        dfs( x + 1, n, ans );
    }
}
\end{lstlisting}