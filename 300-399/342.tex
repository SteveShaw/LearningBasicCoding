\section{342 --- Power of Four}

\textbf{Easy}

Given an integer (signed 32 bits), write a function to check whether it is a power of 4.

\paragraph{Example 1:}

\begin{flushleft}
\textbf{Input}: 16

\textbf{Output}: \fcc{true}
\end{flushleft}

\paragraph{Example 2:}

\begin{flushleft}
\textbf{Input}: 5

\textbf{Output}: \fcc{false}
\end{flushleft}

\paragraph{Follow up:}
\begin{itemize}
\item Could you solve it without loops/recursion?
\end{itemize}

\subsection{Bit Operation}
We know how to check if a number is a power of 2.

Now the problem is to distinguish between even powers of two (when $x$ is a power of four) and odd powers of two (when $x$ is not a power of four). 

In binary representation both cases are single 1-bit followed by zeros.

What is the difference? In the first case (power of four), 1-bit is at even position: bit 0, bit 2, bit 4, etc. In the second case, at odd position.

Hence power of four would make a zero in a bitwise AND with number $(101010\ldots 10)_2$

How long should be $(101010\ldots 10)_2$ if $x$ is a signed integer? Answer is 32 bits. 

It's common to use hexadecimal representation: $(101010\ldots 10)_2 = (aaaaaaaa)_{16}$

\setcounter{lstlisting}{0}
\begin{lstlisting}[style=customc, caption={Bit Operation}]
bool isPowerOfFour( int num )
{
    if( ( num > 0 ) && ( ( num & ( num - 1 ) ) == 0 ) )
    {
        //num is power of 2
        //check if it is even power of 2
        return ( num & 0xaaaaaaaa ) == 0;
    }
    return false;
}
\end{lstlisting}

\paragraph{Related Problems}
\begin{itemize}
\item \textbf{231. Power of Two}
\item \textbf{326. Power of Three}
\end{itemize}