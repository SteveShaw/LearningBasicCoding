\section{673 --- Number of Longest Increasing Subsequence}
Given an unsorted array of integers, $A$,  find the number of longest increasing subsequence.

\paragraph{Example 1:}
\begin{flushleft}


\textbf{Input}: \lstinline[language=C++, basicstyle=\small\ttfamily, keywordstyle=\bfseries\color{green!40!black}]|[1,3,5,4,7]|

\textbf{Output}: 2

\textbf{Explanation}: The two longest increasing subsequence are \lstinline[language=C++, basicstyle=\small\ttfamily, keywordstyle=\bfseries\color{green!40!black}]|[1, 3, 4, 7]| and \lstinline[language=C++, basicstyle=\small\ttfamily, keywordstyle=\bfseries\color{green!40!black}]|[1, 3, 5, 7]|.
\end{flushleft}

\paragraph{Example 2:}
\begin{flushleft}


\textbf{Input}: \lstinline[language=C++, basicstyle=\small\ttfamily, keywordstyle=\bfseries\color{green!40!black}]|[2,2,2,2,2]|

\textbf{Output}: 5

\textbf{Explanation}: 

The length of longest continuous increasing subsequence is 1, and there are 5 subsequences' length is 1, so output 5.
\end{flushleft}

\paragraph{Note:} 
\begin{itemize}
\item Length of the given array will be not exceed 2000 and the answer is guaranteed to be fit in 32-bit \lstinline[language=C++, basicstyle=\small\ttfamily, keywordstyle=\bfseries\color{green!40!black}]|signed int|.
\end{itemize}

\subsection{Dynamic Progamming}


\subsection{Segment Tree}
The segment tree algorithm include three methods: \textbf{build}, \textbf{query} and \textbf{update}.

\begin{enumerate}
\item \textbf{build}: Build the tree from the original data
\setcounter{lstlisting}{0}
\begin{lstlisting}[style=customc, caption={Build}]
// call this method as buildSegTree(arr, 0, 0, n-1);
// Here A[] is input array and n is its size.

void buildSegTree( vector<int>& A, int treeIndex, int lo, int hi )
{
    if( lo == hi )
    {
        // leaf node. store value in node.
        tree[treeIndex] = A[lo];
        return;
    }

    int mid = lo + ( hi - lo ) / 2; // recurse deeper for children.

    //build left tree
    buildSegTree( A, 2 * treeIndex + 1, lo, mid );

    //build right tree
    buildSegTree( A, 2 * treeIndex + 2, mid + 1, hi );

    // merge build results
    tree[treeIndex] = merge( tree[2 * treeIndex + 1], tree[2 * treeIndex + 2] );
}
\end{lstlisting}
The method builds the entire tree in a bottom up fashion. When the condition \lstinline[language=C++, basicstyle=\small\ttfamily, keywordstyle=\bfseries\color{green!40!black}]|lo = hi| is satisfied, we are left with a range comprising of just a single element (which happens to be \lstinline[language=C++, basicstyle=\small\ttfamily, keywordstyle=\bfseries\color{green!40!black}]|A[lo]|). This constitutes a leaf of the tree. The rest of the nodes are built by merging the results of their two children. \lstinline[language=C++, basicstyle=\small\ttfamily, keywordstyle=\bfseries\color{green!40!black}]|treeIndex| is the index of the current node of the segment tree which is being processed.
\item \textbf{query}: Read/Query on an interval or segment of the data.

\begin{lstlisting}[style=customc, caption={Query/Read}]
// call this method as query(tree, 0, 0, n-1, i, j);
// Here [i,j] is the range/interval we are querying.
// This method relies on "null" nodes being equivalent to storing zero.
int query( vector<int>& tree, int treeIndex, int lo, int hi, int i, int j )
{
    // query for arr[i..j]

    if( lo > j || hi < i )
    {
        // segment completely outside range
        return 0; // represents a null node
    }

    if( i <= lo && j >= hi )
    {
        // segment completely inside range
        return tree[treeIndex];
    }

    // partial overlap of current segment and queried range.
    // Recurse deeper.
    int mid = lo + ( hi - lo ) / 2;

    if( i > mid )
    {
        return query( tree, 2 * treeIndex + 2, mid + 1, hi, i, j );
    }
    else if( j <= mid )
    {
        return query( tree, 2 * treeIndex + 1, lo, mid, i, j );
    }

    int leftQuery = query( tree, 2 * treeIndex + 1, lo, mid, i, mid );

    int rightQuery = query( tree, 2 * treeIndex + 2, mid + 1, hi, mid + 1, j );

    // merge query results
    return merge( leftQuery, rightQuery );
}
\end{lstlisting}
The method returns a result when the queried range matches exactly with the range represented by a current node. Else it digs deeper into the tree to find nodes which match a portion of the node exactly.

\item \textbf{update}: Update the value of an element.

\begin{lstlisting}[style=customc, caption={Update}]
// call this method as update(tree, 0, 0, n-1, i, val);
// Here you want to update the value at index i with value val.
void update( vector<int>& tree, int treeIndex, int lo, int hi, int arrIndex, int val )
{
    if( lo == hi )
    {
        // leaf node. update element.
        tree[treeIndex] = val;
        return;
    }

    // recurse deeper for appropriate child
    int mid = lo + ( hi - lo ) / 2;

    if( arrIndex > mid )
    {
        update( tree, 2 * treeIndex + 2, mid + 1, hi, arrIndex, val );
    }
    else if( arrIndex <= mid )
    {
        update( 2 * treeIndex + 1, lo, mid, arrIndex, val );
    }

    // merge updates
    tree[treeIndex] = merge( tree[2 * treeIndex + 1], tree[2 * treeIndex + 2] );
}
\end{lstlisting}
This is similar to \textbf{build}. We update the value of the leaf node of our tree which corresponds to the updated element. Later the changes are propagated through the upper levels of the tree straight to the root.
\end{enumerate}

