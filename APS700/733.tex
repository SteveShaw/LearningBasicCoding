\section{733 --- Flood Fill}
An image is represented by a 2-D array of integers, each integer representing the pixel value of the image (from 0 to 65535).

Given a coordinate \fcj{(sr, sc)} representing the starting pixel (row and column) of the flood fill, and a pixel value \fcj{newColor}, \fcj{"flood fill"} the image.

To perform a \fcj{"flood fill"}, consider the starting pixel, plus any pixels connected 4-direction to the starting pixel of the same color as the starting pixel, plus any pixels connected 4-direction to those pixels (also with the same color as the starting pixel), and so on. Replace the color of all of the aforementioned pixels with the \fcj{newColor}.

At the end, return the modified image.

\paragraph{Example 1:}

\textbf{Input}:
 
\fcj{image = [[1,1,1],[1,1,0],[1,0,1]]}

\fcj{sr = 1}, \fcj{sc = 1}, \fcj{newColor = 2}

\textbf{Output}: \fcj{[[2,2,2],[2,2,0],[2,0,1]]}

\textbf{Explanation}:
 
From the center of the image (with position \fcj{(sr, sc) = (1, 1)}), all pixels connected 

by a path of the same color as the starting pixel are colored with the new color.

Note the bottom corner is not colored 2, because it is not 4-direction connected to the starting pixel.

\paragraph{Note:}
\begin{itemize}
\item The length of \fcj{image} and \fcj{image[0]} will be in the range \fcj{[1, 50]}.

\item The given starting pixel will satisfy \fcj{0 <= sr < image.length} and \fcj{0 <= sc < image[0].length}.

\item The value of each color in \fcj{image[i][j]} and \fcj{newColor} will be an integer in \fcj{[0, 65535]}.

\end{itemize}

\subsection{Depth First Search}
This is an easy problem with Depth First Search.

Note: we only do flood filling when \fcj{image[sr][sc]!=newColor}.

\setcounter{lstlisting}{0}
\begin{lstlisting}[style=customc, caption={Depth First Search}]
vector<vector<int>> floodFill( vector<vector<int>>& image, int sr, int sc, int newColor )
{
    int M = static_cast<int>( image.size() );
    int N = static_cast<int>( image[0].size() );
    if( image[sr][sc] == newColor )
    {
        //if the pixel value at (sr,sc) is equal to newColor
        //no need to do flood fill
        return image;
    }
    flood( image, sr, sc, newColor, image[sr][sc] );
    return image;
}
//DFS helper function
void flood( vector<vector<int>>& G, int r, int c, int color, int gray )
{
    int dr[] = {-1, 0, 1, 0};
    int dc[] = {0, -1, 0, 1};
    int M = static_cast<int>( G.size() );
    int N = static_cast<int>( G[0].size() );
    G[r][c] = color;
    for( int i = 0; i < 4; ++i )
    {
        int nr = r + dr[i];
        int nc = c + dc[i];

        if( ( nr >= 0 ) && ( nr < M ) && ( nc >= 0 ) && ( nc < N ) )
        {
            if( G[nr][nc] == gray )
            {
                flood( G, nr, nc, color, gray );
            }
        }
    }
}
\end{lstlisting}