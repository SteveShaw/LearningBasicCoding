\section{775 --- Global and Local Inversions}
We have some permutation $A$ of \fcj{[0, 1, ..., N - 1]}, where $N$ is the length of $A$.

The number of (global) inversions is the number of $i < j$ with $0 \leq i < j < N$ and $A[i] > A[j]$.

The number of local inversions is the number of $i$ with $0 \leq i < N$ and $A[i] > A[i+1]$.

Return \fcj{true} if and only if the number of global inversions is equal to the number of local inversions.

\paragraph{Example 1:}

\begin{flushleft}
\textbf{Input}: \fcj{A = [1,0,2]}

\textbf{Output}: \fcj{true}

\textbf{Explanation}: There is 1 global inversion, and 1 local inversion.
\end{flushleft}

\paragraph{Example 2:}
\begin{flushleft}

\textbf{Input}: \fcj{A = [1,2,0]}

\textbf{Output}: \fcj{false}

\textbf{Explanation}: There are 2 global inversions, and 1 local inversion.

\end{flushleft}

\paragraph{Note:}

\begin{itemize}
\item $A$ will be a permutation of \fcj{[0, 1, ..., A.length - 1]}.
\item $A$ will have length in range \fcj{[1, 5000]}.
\item The time limit for this problem has been reduced.
\end{itemize}

\subsection{One Pass}
The number of global inversions is equal to the number of local inversions means we only have local inversions.

Therefore, an \textit{ideal} permutation is the one without global inversions. In this case, if zero is at index 2 or greater, $A[0] > A[2] =0$ which is a global inversion. Thus zero can only be in index 0 or 1. 

If $A[1] = 0$, we must have $A[0]=1$ otherwise $A[0] > A[j]=1$ where $j\geq 2$ and we get a global inversion.

Thus if $A$ is an \textit{ideal} permutation array, we must have 

\begin{itemize}
\item $A[0]=0$ and $A[1,N-1]$ is also ideal permutation.
\item $A[1]=0$ and $A[2,N-1]$ is also ideal permutation.
\end{itemize}

Then the condition for $A$ to be an ideal permutation array is 

$\lvert A[i] - i\rvert \leq 1 $.

\setcounter{lstlisting}{0}
\begin{lstlisting}[style=customc, caption={One Pass}]
bool isIdealPermutation( vector<int>& A )
{
    auto L( static_cast<int>( A.size() ) );
    for( int i = 0; i < L; ++i )
    {
        if( abs( A[i] - i ) > 1 )
        {
            //global inversion exists
            return false;
        }
    }
    return true;
}
\end{lstlisting}

\subsection{Keep Minimum}
If we use a brute force approach, for each index $i$, we will check if there is a index $j\geq i+2$ such that $A[i]>A[j]$. This is same as checking $A[i] > \min(A[i+2], \ldots, A[N-1]$. We can record minimum value so far from right to left to speed this checking.

\begin{lstlisting}[style=customc, caption={Keep Minimum}]
bool isIdealPermutation( vector<int>& A )
{
    auto N( static_cast<int>( A.size() ) );
    auto cur_min( N );
    for( int i = N - 1; i >= 2; --i )
    {
        cur_min = ( min )( cur_min, A[i] );
        if( A[i - 2] > cur_min )
        {
            //global inversion exists
            return false;
        }
    }
    return true;
}
\end{lstlisting}