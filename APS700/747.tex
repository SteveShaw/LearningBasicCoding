\section{747 --- Largest Number At Least Twice of Others}
In a given integer array \fcj{nums}, there is always exactly one largest element.

Find whether the largest element in the array is at least twice as much as every other number in the array.

If it is, return the \textbf{index} of the largest element, otherwise return $-1$.

\paragraph{Example 1:}

\begin{flushleft}
\textbf{Input}: \fcj{nums = [3, 6, 1, 0]}

\textbf{Output}: 1

\textbf{Explanation}: 6 is the largest integer, and for every other number in the array $x$,
6 is more than twice as big as $x$.  The index of value 6 is 1, so we return 1.

\end{flushleft}
 

\paragraph{Example 2:}

\begin{flushleft}
\textbf{Input}: \fcj{nums = [1, 2, 3, 4]}

\textbf{Output}: $-1$

Explanation: 4 isn't at least as big as twice the value of 3, so we return -1.

\end{flushleft}
 

\paragraph{Note:}

\begin{itemize}
\item \fcj{nums} will have a length in the range \fcj{[1, 50]}.
\item Every \fcj{nums[i]} will be an integer in the range \fcj{[0, 99]}.
\end{itemize}

\subsection{Find Largest And Second Largest Elements}
We can find the largest and second largest elements in \fcj{nums}, and compare them to get the answer.

\setcounter{lstlisting}{0}
\begin{lstlisting}[style=customc, caption={Scan}]
int dominantIndex( vector<int>& nums )
{
    //corner case
    if( nums.size() == 1 )
    {
        return 0;
    }
    //find largest and second largest elements
    int max1 = nums[0];
    int max2 = -1;
    int index = 0;
    int N = static_cast<int>( nums.size() );
    for( int i = 1; i < N; ++i )
    {
        if( nums[i] > max1 )
        {
            index = i;
            max2 = max1;
            max1 = nums[i];
        }
        else if( nums[i] > max2 )
        {
            max2 = nums[i];
        }
    }
    return max1 >= 2 * max2  ? index : -1;
}
\end{lstlisting}