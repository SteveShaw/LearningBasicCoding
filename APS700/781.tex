\section{781 --- Rabbits in Forest}
In a forest, each rabbit has some color. Some subset of rabbits (possibly all of them) tell you how many other rabbits have the same color as them. Those \fcj{answers} are placed in an array.

Return the minimum number of rabbits that could be in the forest.

\paragraph{Examples:}
\begin{flushleft}


\textbf{Input}: \fcj{answers = [1, 1, 2]}

\textbf{Output}: 5

\textbf{Explanation}:

The two rabbits that answered \fcj{"1"} could both be the same color, say red.

The rabbit than answered \fcj{"2"} can't be red or the answers would be inconsistent.

Say the rabbit that answered \fcj{"2"} was blue.

Then there should be 2 other blue rabbits in the forest that didn't answer into the array.

The smallest possible number of rabbits in the forest is therefore 5: 3 that answered plus 2 that didn't.

\textbf{Input}: \fcj{answers = [10, 10, 10]}

\textbf{Output}: 11

\textbf{Input}: \fcj{answers = []}

\textbf{Output}: 0
\end{flushleft}

\paragraph{Note:}

\begin{enumerate}
\item \fcj{answers} will have length at most 1000.
\item Each \fcj{answers[i]} will be an integer in the range \fcj{[0, 999]}.
\end{enumerate}

\subsection{Count}
If $ x+1 $ rabbits have same color, then we get $ x+1 $ rabbits who all answer $ x $.

now if $n$ rabbits answer $x$.

\begin{itemize}
\item If $n$ is the multiple of $x+1$, we need $n / (x + 1)$ groups of $x + 1$ rabbits.
\item Otherwise, we need $n / (x + 1) + 1$ groups of $x + 1$ rabbits.
\end{itemize}

The number of groups is \fcj{math.ceil(n / (x + 1))} which is equal to $(n+x)/(x+1)$. 

\setcounter{lstlisting}{0}
\begin{lstlisting}[style=customc, caption={Count}]
int numRabbits( vector<int>& answers )
{
    unordered_map<int, int> dict;
    for( int n : answers )
    {
        dict[n] += 1;
    }
    int ans = 0;
    for( auto[x, n] : dict )
    {
        //if n % (x+1) =0, we need n / (x+1) groups
        //otherwise, we need n / (x+1) + 1 groups
        //each group has x+1 rabbits
        //(n+x)/(x+1) covers two cases
        ans += ( x + n ) / ( x + 1 ) * ( x + 1 );
    }
    return ans;
}
\end{lstlisting}


