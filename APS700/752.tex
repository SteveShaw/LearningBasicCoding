\section{752 --- Open the Lock}
You have a lock in front of you with 4 circular wheels. Each wheel has 10 slots: \fcj{'0'}, \fcj{'1'}, \fcj{'2'}, \fcj{'3'}, \fcj{'4'}, \fcj{'5'}, \fcj{'6'}, \fcj{'7'}, \fcj{'8'}, \fcj{'9'}. The wheels can rotate freely and wrap around: for example we can turn \fcj{'9'} to be \fcj{'0'}, or \fcj{'0'} to be \fcj{'9'}. Each move consists of turning one wheel one slot.

The lock initially starts at \fcj{'0000'}, a string representing the state of the 4 wheels.

You are given a list of \fcj{deadends} dead ends, meaning if the lock displays any of these codes, the wheels of the lock will stop turning and you will be unable to open it.

Given a \fcj{target} representing the value of the wheels that will unlock the lock, return the minimum total number of turns required to open the lock, or $-1$ if it is impossible.

\paragraph{Example 1:}
\begin{flushleft}


\textbf{Input}: 

\fcj{deadends = ["0201","0101","0102","1212","2002"]}, \fcj{target = "0202"}

\textbf{Output}: 6

\textbf{Explanation}:

A sequence of valid moves would be 

\fcj{"0000" -> "1000" -> "1100" -> "1200" -> "1201" -> "1202" -> "0202"}.

Note that a sequence like \fcj{"0000" -> "0001" -> "0002" -> "0102" -> "0202"} would be invalid, because the wheels of the lock become stuck after the display becomes the dead end \fcj{"0102"}.

\end{flushleft}

\paragraph{Example 2:}

\begin{flushleft}
\textbf{Input}: 

\fcj{deadends = ["8888"]}, \fcj{target = "0009"}

\textbf{Output}: 1

\textbf{Explanation}:

We can turn the last wheel in reverse to move from \fcj{"0000" -> "0009"}.
\end{flushleft}

\paragraph{Example 3:}

\begin{flushleft}
\textbf{Input}: 

\fcj{deadends = ["8887","8889","8878","8898","8788","8988","7888","9888"]}, \fcj{target = "8888"}

\textbf{Output}: $-1$

\textbf{Explanation}:

We can't reach the target without getting stuck.
\end{flushleft}

\paragraph{Example 4:}

\begin{flushleft}
\textbf{Input}: 

\fcj{deadends = ["0000"]}, \fcj{target = "8888"}

\textbf{Output}: $-1$

\end{flushleft}

\paragraph{Note:}

\begin{itemize}
\item The length of \fcj{deadends} will be in the range \fcj{[1, 500]}.
\item \fcj{target} will not be in the list \fcj{deadends}.
\item Every string in \fcj{deadends} and the string \fcj{target} will be a string of 4 digits from the 10,000 possibilities \fcj{'0000'} to \fcj{'9999'}.

\end{itemize}

\subsection{BFS}
The problem is asking for the minimum steps, thus \textit{BFS} can ge the answer quickly. \textit{DFS} would have to traverse all depths to get the answer.

Since the problem has confined the string has four digits, thus we can transform between \fcj{string} and \fcj{integer} to get the memory consumption lower.
\setcounter{lstlisting}{0}
\begin{lstlisting}[style=customc, caption={BFS}]
int openLock( vector<string>& deadends, string target )
{
    //routine to convert to integer
    auto to_int = []( const char* pstr )
    {
        int res = 0;
        for( int i = 0; i < 4; ++i )
        {
            res = res * 10 + ( pstr[i] - '0' );
        }
        return res;
    };
    //record seen integers
    unordered_set<int> dict;
    for( const auto& deadend : deadends )
    {
        int x = to_int( deadend.c_str() );
        if( x == 0 )
        {
            return -1;
        }
        dict.insert( to_int( deadend.c_str() ) );
    }
    queue<int> q;
    q.push( 0 );
    //convert to string with four digits
    auto to_str = []( int i )
    {
        string s = "0000";
        int si = 0;
        while( i )
        {
            int q = i / 10;
            int r = i - 10 * q;
            s[si++] = '0' + r;
            i = q;
        }
        reverse( begin( s ), end( s ) );
        return s;
    };
    int level = 0;
    while( !q.empty() )
    {
        auto sz = q.size();
        for( size_t i = 0; i < sz; ++i )
        {
            auto n = q.front();
            q.pop();
            auto s = to_str( n );
            if( s == target )
            {
                //found target
                //level is the minimum steps
                return level;
            }
            //change each letter in s
            for( int i = 0; i < 4; ++i )
            {
                //increment
                int ci = s[i] - '0';
                int next = ( ci + 1 ) % 10;
                s[i] = '0' + next;
                n = to_int( s.c_str() );
                if( dict.find( n ) == dict.end() )
                {
                    //we have to add now
                    //avoid adding same into the
                    //dict
                    dict.insert( n );
                    q.push( n );
                }
                //decrement
                next = ( ci + 9 ) % 10;
                s[i] = '0' + next;
                n = to_int( s.c_str() );
                if( dict.find( n ) == dict.end() )
                {
                    dict.insert( n );
                    q.push( n );
                }
                //restore s[i]
                s[i] = '0' + ci;
            }
        }
        //increments the steps
        ++level;
    }
    //cannot reach target
    return -1;
}
\end{lstlisting}