\section{785 --- Is Graph Bipartite?}
Given an undirected \fcj{graph}, return \fcj{true} if and only if it is bipartite.

Recall that a graph is \textbf{bipartite} if we can split it's set of nodes into two independent subsets A and B such that every edge in the graph has one node in A and another node in B.

The graph is given in the following form: \fcj{ graph[i]} is a list of indexes $j$ for which the edge between nodes $i$ and $j$ exists. Each node is an integer between 0 and \fcj{graph.length - 1}.  There are no self edges or parallel edges: \fcj{graph[i]} does not contain $i$, and it doesn't contain any element twice.

\paragraph{Example 1:}
\begin{flushleft}
\textbf{Input}: \fcj{[[1,3], [0,2], [1,3], [0,2]]}

\textbf{Output}: true

\textbf{Explanation}: 

The graph looks like this:
\begin{figure}[H]
\begin{tikzpicture}
[every node/.style={draw, circle, fill=gray!20!, minimum size=5mm},
thick]
\node(0){0};
\node(1)[right=1cm of 0]{1};
\node(2)[below=1cm of 1]{2};
\node(3)[below=1cm of 0]{3};
\draw (0) -- (1) -- (2) -- (3) -- (0);
\end{tikzpicture}
\end{figure}
%0----1
%|    |
%|    |
%3----2
We can divide the vertices into two groups: \fcj{(0, 2)} and \fcj{(1, 3)}.


\end{flushleft}

\paragraph{Example 2:}

\begin{flushleft}
\textbf{Input}: \fcj{[[1,2,3], [0,2], [0,1,3], [0,2]]}

\textbf{Output}: \fcj{false}

\textbf{Explanation}: 

The graph looks like this:

\begin{figure}[H]
\begin{tikzpicture}
[every node/.style={draw, circle, fill=gray!20!, minimum size=5mm},
thick]
\node(0){0};
\node(1)[right=1cm of 0]{1};
\node(2)[below=1cm of 1]{2};
\node(3)[below=1cm of 0]{3};
\draw (0) -- (1) -- (2) -- (3) -- (0);
\draw (0) -- (2);
\end{tikzpicture}
\end{figure}

%0----1
%| \  |
%|  \ |
%3----2
We cannot find a way to divide the set of nodes into two independent subsets.


\end{flushleft}

\paragraph{Note:}

\begin{itemize}
\item \fcj{graph} will have length in range \fcj{[1, 100]}.
\item \fcj{graph[i]} will contain integers in range \fcj{[0, graph.length - 1]}.
\item \fcj{graph[i]} will not contain $i$ or duplicate values.
\item The graph is undirected: if any element $j$ is in \fcj{graph[i]}, then $i$ will be in \fcj{graph[j]}.
\end{itemize}

\subsection{Coloring Node}
For each uncolored node, we start the coloring process by doing a \textbf{DFS} on that node. Every neighbor gets colored the opposite color from the current node. If we find a neighbor colored the same color as the current node, then our coloring was impossible.


\setcounter{lstlisting}{0}
\begin{lstlisting}[style=customc, caption={DFS}]
bool isBipartite( vector<vector<int>>& graph )
{
    auto N( static_cast< int >( graph.size() ) );
    vector<int> colors( N, -1 );
    for( int i = 0; i < N; ++i )
    {
        if( colors[i] == -1 )
        {
            //node i is not colored
            //color i with color 1
            colors[i] = 1;
            //start dfs
            stack<int> stk;
            stk.push( i );
            while( !stk.empty() )
            {
                int node = stk.top();
                stk.pop();
                //visit adjacent nodes
                for( int adj : graph[node] )
                {
                    if( colors[adj] == -1 )
                    {
                        //this adjacent node is not colored
                        stk.push( adj );
                        //color it as inverse color of node
                        colors[adj] = 1 - colors[node];
                    }
                    else if( colors[adj] == colors[node] )
                    {
                        //the adjacent node has same color with node
                        //we cannot bipartite the graph
                        return false;
                    }
                }//for(adj)
            }//while(stk)
        }//if(color)
    }//for(i)
    return true;
}
\end{lstlisting}