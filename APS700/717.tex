\section{717 --- 1-bit and 2-bit Characters}
We have two special characters. The first character can be represented by one bit \fcj{0}. The second character can be represented by two bits (\fcj{10} or \fcj{11}).

Now given a string represented by several bits. Return whether the last character must be a one-bit character or not. The given string will always end with a zero.

\paragraph{Example 1:}

\begin{flushleft}
\textbf{Input}: \fcj{bits = [1, 0, 0]}

\textbf{Output}: \fcj{True}

\textbf{Explanation}: 

The only way to decode it is two-bit character and one-bit character. So the last character is one-bit character.
\end{flushleft}

\paragraph{Example 2:}
\begin{flushleft}


\textbf{Input}: \fcj{bits = [1, 1, 1, 0]}

\textbf{Output}: \fcj{False}

\textbf{Explanation}: 

The only way to decode it is two-bit character and two-bit character. So the last character is \textbf{NOT} one-bit character.

\end{flushleft}


\paragraph{Note:}

\begin{itemize}
\item \fcj{1 <= len(bits) <= 1000.}
\item \fcj{bits[i]} is always 0 or 1.
\end{itemize}

\subsection{Increment Index}
When reading from the index $i$, 

\begin{itemize}
\item if \fcj{bits[i] == 0}, the next character must have 1 bit; 
\item else \fcj{if bits[i] == 1}, the next character must have 2 bits.
\end{itemize} 

Thus, We increment read-index $i$ to the start of the next character appropriately. At the end, if $i$ is at \fcj{bits.length - 1}, the last character must have a size of 1 bit.

\setcounter{lstlisting}{0}
\begin{lstlisting}[style=customc, caption={Increment Index}]
bool isOneBitCharacter( vector<int>& bits )
{
    int n_bits = static_cast<int>( bits.size() );

    int i = 0;

    for( ; i < n_bits - 1; )
    {
        if( bits[i] == 0 )
        {
            //forward one bit
            ++i;
        }
        else
        {
            //forward two bits
            i += 2;
        }
    }
    //check if we finally at
    //last bit
    return i == ( n_bits - 1 );
}
\end{lstlisting}