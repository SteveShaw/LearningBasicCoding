\section{741 --- Cherry Pickup}
In a $N \times N$ \fcj{grid} representing a field of cherries, each cell is one of three possible integers.

\begin{itemize}
\item 0 means the cell is empty, so you can pass through;
\item 1 means the cell contains a cherry, that you can pick up and pass through;
\item $-1$ means the cell contains a thorn that blocks your way.
\end{itemize}

Your task is to collect maximum number of cherries possible by following the rules below:
\begin{itemize}
\item Starting at the position (0, 0) and reaching $(N-1, N-1)$ by moving right or down through valid path cells (cells with value 0 or 1);
\item After reaching $(N-1, N-1)$, returning to (0, 0) by moving left or up through valid path cells;
\item When passing through a path cell containing a cherry, you pick it up and the cell becomes an empty cell (0); 
\item If there is no valid path between (0, 0) and $(N-1, N-1)$, then no cherries can be collected.
\end{itemize} 


\paragraph{Example 1:}
\begin{flushleft}


\textbf{Input}: 

\fcj{grid}:

\[
\begin{bmatrix}
0 & 1 & -1 \\
1 & 0 & -1 \\
1 & 1 &  1 
\end{bmatrix}
\]

\textbf{Output}: 5

\textbf{Explanation}: 

The player started at $ (0, 0) $ and went down, down, right right to reach $ (2, 2) $.

4 cherries were picked up during this single trip, and the matrix becomes \fcj{[[0,1,-1],[0,0,-1],[0,0,0]]}.

Then, the player went left, up, up, left to return home, picking up one more cherry.

The total number of cherries picked up is 5, and this is the maximum possible.
\end{flushleft}

\paragraph{Note:}

\begin{itemize}
\item \fcj{grid} is an $N$ by $N$ 2D array, with $1 \leq N \leq 50$.
\item Each \fcj{grid[i][j]} is an integer in the set \fcj{[-1, 0, 1]}.
\item It is guaranteed that \fcj{grid[0][0]} and \fcj{grid[N-1][N-1]} are not $-1$.
\end{itemize}

\subsection{Dynamic Programming Top Down}
First of all, greedy method does not work here. The reason is that maximizing the first one-way does \textbf{NOT} guarantee that we can get overall maximum number of cherries for round-trip. In the first one-way picking, we are following only the shortest path. Therefore, we may neglect some positions that would give more cherries when going back.
\par
For example, in the following grid
\begin{figure}[H]
\centering
\begin{tikzpicture}
\matrix (m) [matrix of nodes]
{
0 & 1 & -1 \\
1 & 1 & -1 \\
1 & 1 & 0 \\
};
\draw [color=red](m-1-1.north west)--(m-2-1.south west)--(m-2-2.south west)--(m-3-2.south west)--(m-3-3.south east)--(m-3-3.north east)--(m-3-2.north east)--(m-2-2.north east)--(m-1-1.south east)--(m-1-1.north east)--cycle;
\end{tikzpicture}
\end{figure}
If the first round one-way shortest path goes along the red path for getting maximum 3 cherries, it splits the remaining 2 cherries into opposite sides of the diagonal which is impossible for round trip to pick up all cherries. 

Therefore, we have to plan the round trip altogether. Note that the path direction is irrelevant to the problem and we are forced to move only on shortest paths.

We can imagine two persons $p_1$ and $p_2$ are at $(0,0)$ during start. Then they both move by taking different path to $(N-1, N-1)$. After $t$ steps, $p_1$ is in position $(r_1,c_1)$ and $p_2$ in $(r_2, c_2)$. Both positions will have $r_1+c_1=r_2+c_2=t$. Then, $r_2 = r_1 + c_1 - c_2$, which means $r_1,c_1,c_2$ uniquely determine $p_1, p_2$ who have walked the same $t = r_1+c_1$ number of steps. This is suitable for dynamic programming quite nicely.

Suppose $F[r_1][c_1][c_2]$ be the maximum number of cherries obtained by two persons $p_1$ and $p_2$ who are starting at $(r_1, c_1)$ and $(r_2, c_2)$ respectively, and are walking towards $(N-1, N-1)$ by picking up cherries, where $r_2 = r_1+c_1-c_2$.
\par
If $G[r_1][c_1] \neq -1$ and $G[r_2][c_2] \neq -1$, then $F[r_1][c_1][c_2]$ is $(G[r_1][c_1] + G[r_2][c_2])$, plus the maximum of $F[r_1+1][c_1][c_2]$, $F[r_1][c_1+1][c_2]$, $F[r_1+1][c_1][c_2+1]$ and $F[r_1][c_1+1][c_2+1]$. We should also be careful to not double count in case when $p_1$ and $p_2$ are in the same position, i.e., $(r_1, c_1) = (r_2, c_2)$.
\par
Why do we need to plus the 
\[
\max(F[r_1+1][c_1][c_2], F[r_1][c_1+1][c_2], F[r_1+1][c_1][c_2+1], F[r_1][c_1+1][c_2+1])
\]
The reason is these four values corresponds to the 4 possibilities for $p_1$ and $p_2$ moving down and right:
\begin{enumerate}
\item $p_1$ go \textbf{down} and $p_2$ go \textbf{down}: $F[r_1+1][c_1][c_2]$ since $r_2 \gets r_1 + 1 +c_1 -c_2 = (r_1+c_1-c_2) + 1 = r_2+1$
\item $p_1$ go \textbf{right} and $p_2$ go \textbf{down}: $F[r_1][c_1+1][c_2]$ since $r_2 \gets r_1 + c_1 + 1 -c_2 = (r_1+c_1-c_2) + 1 = r_2+1$
\item $p_1$ go \textbf{down} and $p_2$ go \textbf{right}: $F[r_1+1][c_1][c_2+1]$ since $r_2$ is unchanged because $r_2= r_1 + c_1 + 1 -c_2 - 1 = (r_1+c_1-c_2) = r_2$
\item $p_1$ go \textbf{right} and $p_2$ go \textbf{right}: $F[r_1][c_1+1][c_2+1]$ since $r_2$ is unchanged because $r_2= r_1 + c_1 + 1 -c_2 - 1 = (r_1+c_1-c_2) = r_2$
\end{enumerate}

\setcounter{lstlisting}{0}
\begin{lstlisting}[style=customc, caption={Dynamic Programming Top Down}]
int cherryPickup( vector<vector<int>>& grid )
{
    auto N = grid.size();
    vector<vector<vector<int>>> F( N, vector<vector<int>>( N, vector<int>( N, -1 ) ) );
    return ( max )( 0, dfs( grid, F, 0, 0, 0 ) );
}
int dfs( vector<vector<int>>& G, vector<vector<vector<int>>>& memo, size_t r1, size_t c1, size_t c2 )
{
    //r1+c1=r2+c2
    auto r2 = r1 + c1 - c2;
    if( !isValid( G, r1, c1, r2, c2 ) )
    {
        //invalid cell, just return a very small number
        return -99999999;
    }
    //we have value from memo
    if( memo[r1][c1][c2] != -1 )
    {
        return memo[r1][c1][c2];
    }
    auto N = G.size();
    //if person 1 is in the destination cell
    if( ( r1 == N - 1 ) && ( c1 == N - 1 ) )
    {
        return G[N - 1][N - 1];
    }
    //the cherries at (r1,c1)
    int pick = G[r1][c1];

    if( c1 != c2 )
    {
        //(r1,c1) and (r2,c2) are not same cell
        pick += G[r2][c2];
    }
    //get cherries picked from four possible movements
    //combinaton of two players
    //pickFromRR: both persons move right
    int pickFromRR = dfs( G, memo, r1 + 1, c1, c2 );
    //person 1 move right, person 2 move down
    int pickFromRD = dfs( G, memo, r1 + 1, c1, c2 + 1 );
    //person 1 move down, person 2 move right
    int pickFromDR = dfs( G, memo, r1, c1 + 1, c2 );
    //person 1 move down, person 2 move down
    int pickFromDD = dfs( G, memo, r1, c1 + 1, c2 + 1 );
    //get the maximum one from these four movement combinatons
    int pickFromNextMove = ( max )( pickFromRR, pickFromRD );
    pickFromNextMove = ( max )( pickFromDR, pickFromNextMove );
    pickFromNextMove = ( max )( pickFromDD, pickFromNextMove );
    //add to current picked cherries
    pick += pickFromNextMove;
    //save to the memo array
    memo[r1][c1][c2] = pick;
    //return the result
    return pick;
}
//helper function to check if (r1,c1) and (r2,c2) are valid cell or
//they are not throns (-1)
bool isValid( vector<vector<int>>& G, size_t r1, size_t c1, size_t r2, size_t c2 )
{
    auto N = G.size();
    if( ( r1 >= N ) || ( r2 >= N ) || ( c1 >= N ) || ( c2 >= N ) )
    {
        return false;
    }
    if( ( G[r1][c1] == -1 ) || ( G[r2][c2] == -1 ) )
    {
        return false;
    }
    return true;
}
\end{lstlisting}

\subsection{Dynamic Programming Bottom Up}
Like in Top Down approach, Say $r_1 + c_1 = t$ is the $t$-th layer. Since the recursion only references the next layer, we only need to keep two layers in memory at a time.

At time $t$, let $F[r_1][r_2]$ be the most cherries that we can pick up for two people going from $(0, 0)$ to $(r_1, c_1)$ and $(0, 0)$ to $(r_2, c_2)$, where $c_1 = t-r_1$, $c_2 = t-r_2$. 

Since we are evaluating time $t$ from 0 to $N-1+N-1=2N-2$, we will fill $F$ starting from cell $(0,0)$. At time 0, $F[0][0]=G[0][0]$. Next, we loop $t$ from 1 to $2N-2$. Then we will get maximum pickups from four possible positions combinations of the two persons
\begin{itemize}
\item left--left: $p_1$ at $(r_1,c_1-1)$ and $p_2$ at $(r_2, c_2-1)$ which is $F[r_1][r_2]$
\item left--up: $p_1$ at $(r_1,c_1-1)$ and $p_2$ at $(r_2-1, c_2)$ which is $F[r_1][r_2-1]$
\item up--left: $p_1$ at $(r_1-1,c_1)$ and $p_2$ at $(r_2, c_2-1)$ which is $F[r_1-1][r_2]$
\item up--up: $p_1$ at $(r_1-1,c_1)$ and $p_2$ at $(r_2-1, c_2)$ which is $F[r_1-1][r_2-1]$
\end{itemize}

We will use another 2D array as the evaluation result at time $t$ and overwrite to $F$ before moving to next time $t+1$.

\begin{lstlisting}[style=customc, caption={DP Bottom Up}]
int cherryPickup( vector<vector<int>>& grid )
{
    auto N = grid.size();
    //array for DP
    //F[r1][r2] means the maximum cherries picked up
    //when person 1 at (r1,c1) and person 2 at (r2,c2)
    vector<vector<int>> F( N, vector<int>( N, -1 ) );
    //since we iterating time from 0 to 2*N-2
    //we will fill F from (0,0)
    F[0][0] = grid[0][0];
    //The next dp array after evaluating current dp array
    //at time t
    vector<vector<int>> next( N, vector<int>( N, -1 ) );
    for( size_t t = 1; t <= 2 * N - 2; ++t )
    {
        //the range for r1 and r2 could be
        //at time t
        size_t start = 0;
        size_t end = t;
        if( t > N - 1 )
        {
            start = t - ( N - 1 );
            end = N - 1;
        }
        for( size_t r1 = start; r1 <= end; ++r1 )
        {
            for( size_t r2 = start; r2 <= end; ++r2 )
            {
                //get c1 and c2 for two persons
                auto c1 = t - r1;
                auto c2 = t - r2;
                //get cherry amount at (r1,c1) and (r2,c2)
                auto pick1 = grid[r1][c1];
                auto pick2 = grid[r2][c2];
                //reset dp value at these two positions to -1
                next[r1][r2] = -1;
                if( ( pick1 == -1 ) || ( pick2 == -1 ) )
                {
                    //if any person cannot get to the cell, skip
                    continue;
                }
                int cherry = pick1;
                if( r1 != r2 )
                {
                    //two persons are at different cells
                    cherry += pick2;
                }

                //get max picked cherries from last move
                //which could be
                //up-up: (r1-1, c1) and (r2-1, c2)
                //left-up: (r1, c1-1) and (r2-1, c2)
                //up-left: (r1-1, c1) and (r2, c2-1)
                //left-left: (r1, c1-1) and (r2, c2-1)
                //notice F is defined for r1 and r2
                //c1 and c2 are determined from r1 and r2
                int last_pick = F[r1][r2]; //left-left
                if( r1 > 0 )
                {
                    last_pick = ( max )( last_pick, F[r1 - 1][r2] ); // up-left

                }
                if( r2 > 0 )
                {
                    last_pick = ( max )( last_pick, F[r1][r2 - 1] ); // left-up

                }
                if( ( r1 > 0 ) && ( r2 > 0 ) )
                {
                    last_pick = ( max )( last_pick, F[r1 - 1][r2 - 1] ); // up-up
                }
                if( last_pick != -1 )
                {
                    //we can go through last cell to current cell
                    next[r1][r2] = ( max )( next[r1][r2], last_pick + cherry );
                }
            }//end(r2)
        }//end(r1)
        //overwrite F by evaluated dp for current t
        swap( F, next );
    } //end(t)
    return ( max )( 0, F[N - 1][N - 1] );
}
\end{lstlisting}

