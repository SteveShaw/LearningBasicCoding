\section{739 --- Daily Temperatures}
Given a list of daily temperatures $T$, return a list such that, for each day in the input, tells you how many days you would have to wait until a warmer temperature. If there is no future day for which this is possible, put 0 instead.

For example, given the list of temperatures 

\begin{flushleft}
$T$: \fcj{[73, 74, 75, 71, 69, 72, 76, 73]}, 

your output should be 

\fcj{[1, 1, 4, 2, 1, 1, 0, 0]}.
\end{flushleft}

\paragraph{Note:} 

\begin{itemize}
\item The length of temperatures will be in the range \fcj{[1, 30000]}. 
\item Each temperature will be an integer in the range \fcj{[30, 100]}.
\end{itemize} 

\subsection{Stack}
In this approach, we reversely iterate over $T$. We also make use of a stack $S$ to record indexes. 

During the traversing
\begin{itemize}
\item As long as current temperature $t$ is no less than the temperature at top index of $S$, we will keep popping index from $S$. For each temperature at popped index, $t$ is the first one on its right side, thus, the distance to the index of $t$ is the number of days to get warmer.
\item Then, push current index into $S$.
\end{itemize}

\setcounter{lstlisting}{0}
\begin{lstlisting}[style=customc, caption={Stack}]
vector<int> dailyTemperatures( vector<int>& T )
{
    //base case: empty array
    if( T.empty() )
    {
        return {};
    }
    stack<int> stk;
    vector<int> ans;
    ans.reserve( T.size() );
    int n = static_cast<int>( T.size() );
    //The last temperature has no any day associated
    //thus is zero
    ans.push_back( 0 );
    //push the last index into the stack
    stk.push( n - 1 );
    //reversely traversing T
    for( int i = n - 2; i >= 0; --i )
    {
        while( !stk.empty() && ( T[stk.top()] <= T[i] ) )
        {
            //now T[i] is the first temperature
            //on the right side of T[stk.top()]
            //which is larger than T[stk.top()]
            stk.pop();
        }
        if( stk.empty() )
        {
            //no any temperature is larger than T[i] on
            //its right side
            ans.push_back( 0 );
        }
        else
        {
            //otherwise
            //the difference of indexes is the
            //answer: the number of days
            ans.push_back( stk.top() - i );
        }
        //push current index of temperature into the stack
        stk.push( i );
    }
    //reverse the array since it added from end
    reverse( begin( ans ), end( ans ) );
    return ans;
}
\end{lstlisting}

\subsection{Constant Memory}
Actually, we can make use of result array to solve this problem. Thus we don't need the auxiliary stack. 

We still reversely traverse $T$. Suppose we are at index $i$ and we have filled the result array, say $R$, from $i+1$ to $n-1$ ($n$ is the length of $T$). 

Now to find the most recent day after $i$ which is warmer than $T[i]$, we start searching from $i+1$. If $T[k] \leq T[i]$ for a $k >i$, we don't just move to $k+1$, we can use the result of $R[k]$ to move $k$. Since $T[k]\leq T[i]$, the most recent day that could be warmer than $T[i]$ will be at index $k+R[k]$. After updating $k$ to $k+R[k]$, we repeat the above process. 

If $R[k]=0$, this means $T[k]$ is the highest temperature from $k$. Since $T[k]\leq T[i]$, it is obvious that $R[i]=0$.

\setcounter{lstlisting}{0}
\begin{lstlisting}[style=customc, caption={Constant Memory}]
vector<int> dailyTemperatures( vector<int>& T )
{
    vector<int> ans( T.size(), 0 );
    int n = static_cast<int>( T.size() );
    for( int i = n - 2; i >= 0; --i )
    {
        //search the most recent day after i
        //which is warmer than i
        int j = i + 1;
        while( ( j < n ) && ( T[j] <= T[i] ) )
        {
            if( ans[j] > 0 )
            {
                //we move to j+ans[j]
                //since ans[j] is the most recent day
                //warmer than j
                j = j + ans[j];
            }
            else
            {
                //j has the highest temperature
                //from j
                //thus we break the loop
                j = n;
            }
        }

        if( j == n )
        {
            //since T[j] <=T[i]
            //thus T[i] is the highest temperature
            //from i
            ans[i] = 0;
        }
        else
        {
            //the days will be j-i
            ans[i] = j - i;
        }
    }
    return ans;
}
\end{lstlisting} 

