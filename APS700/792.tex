\section{792 --- Number of Matching Subsequences}
Given string $S$ and a dictionary of words \fcj{words}, find the number of \fcj{words[i]} that is a subsequence of $S$.

\paragraph{Example:}

\begin{flushleft}
\textbf{Input}: 

\fcj{S = "abcde"}

\fcj{words = ["a", "bb", "acd", "ace"]}

\textbf{Output}: 3

\textbf{Explanation}: There are three words in words that are a subsequence of S: \fcj{"a"}, \fcj{"acd"}, \fcj{"ace"}.
\end{flushleft}

\paragraph{Note:}

\begin{itemize}
\item All words in \fcj{words} and $S$ will only consists of lowercase letters.
\item The length of $S$ will be in the range of \fcj{[1, 50000]}.
\item The length of \fcj{words} will be in the range of \fcj{[1, 5000]}.
\item The length of \fcj{words[i]} will be in the range of \fcj{[1, 50]}.
\end{itemize}

\subsection{Next Letter Position}
Since the length of $S$ is large, we need to only iterate over it for once.

We can put words into buckets by starting character. Take the given example, we put each word in \fcj{words} into two buckets which are

\fcj{'a': (0,1), (2,1), {3,1}}

\fcj{'b': (1,1)}

Each bucket has pairs associated with each letter. The first number is the index of this word in \fcj{words} and the 2nd one is the position of next letter in this word.

Then we iterate over $S$. The first letter is \fcj{'a'}. Thus, we process the bucket for letter \fcj{'a'}, the three pairs are processed as below

\begin{enumerate}
\item \fcj{(0,1)}: next letter's position is 1 and word \fcj{"a"} length is 1. Thus, this is the subsequence of $S$.
\item \fcj{(2,1)}: The word is \fcj{"acd"} and the next letter is \fcj{'c'}. We create a new bucket which is \fcj{'c'} and the first associated pair will be \fcj{(2,2)} because \fcj{"acd"} has index 2 and next letter of \fcj{'c'} has index 2 in \fcj{"acd"}.
\item \fcj{(3,1)}: The word is \fcj{"ace"}, and the next letter is \fcj{'c'}. We have already created bucket for \fcj{'c'}. This time we add a new pair \fcj{3,2} because \fcj{"ace"} has index 3 and next letter of \fcj{'c'} has index 2 in \fcj{"acd"}.
\end{enumerate}

Thus, after processing bucket of \fcj{'a'}, we have 

\fcj{'b': (1,1)}

\fcj{'c': (2,2), (3,2)}

The next letter in $S$ is \fcj{'b'}. Similarly, after processing for bucket of \fcj{'b'}, we have

\fcj{'b': (1,2)}

\fcj{'c': (2,2), (3,2)}

Then process bucket of \fcj{'c'}, the processing results will be 

\fcj{'b': (1,2)}

\fcj{'d': (3,3)}

\fcj{'e': (2,3)}

Next letter in $S$ is \fcj{'d'}. When processing bucket of \fcj{'d'}, we found that the next letter's position is 3 which is the length of \fcj{"acd"}. Thus, this word is a subsequence of $S$. Now, we have two buckets

\fcj{'b': (1,2)}

\fcj{'e': (2,3)}

The last letter is \fcj{'e'} and also this reach the end of \fcj{'ace'}. Finally, we have only one bucket left

\fcj{'b': (1,2)}

Thus the word at index 1 of \fcj{words}, i.e., \fcj{"bb"} is not subsequence of $S$.

\setcounter{lstlisting}{0}
\begin{lstlisting}[style=customc, caption={Next Letter Position}]
int numMatchingSubseq( string S, vector<string>& words )
{
    vector<vector<pair<size_t, size_t>>> buckets( 26 );
    auto wi( 0ull );
    for( const auto& word : words )
    {
        //create bucket for first letter
        //which contains a few pairs (word_index, next letter index)
        //Since this is the first letter in word
        //next letter index is 1
        buckets[word[0] - 'a'].emplace_back( wi, 1 );
        ++wi;
    }
    auto ans( 0 );
    for( auto c : S )
    {
        decltype( buckets )::value_type vec_pairs;
        swap( vec_pairs, buckets[c - 'a'] );
        for( auto [wi, ci] : vec_pairs )
        {
            if( ci == words[wi].size() )
            {
                //we reach the end of words[wi]
                //this is the subsequence of S
                ++ans;
            }
            else
            {
                const auto& word( words[wi] );
                //create another bucket
                //with pair (wi, ci+1).
                //ci+1 is next letter index
                buckets[word[ci] - 'a'].emplace_back( wi, ci + 1 );
            }
        }
    }
    return ans;
}
\end{lstlisting}