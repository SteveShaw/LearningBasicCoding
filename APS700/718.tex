\section{718 --- Maximum Length of Repeated Subarray}
Given two integer arrays $A$ and $B$, return the maximum length of an subarray that appears in both arrays.

\paragraph{Example 1:}

\textbf{Input}:

\begin{flushleft}
$A$: \fcj{[1,2,3,2,1]}

$B$: \fcj{[3,2,1,4,7]}

\textbf{Output}: 3

\textbf{Explanation}: 

The repeated subarray with maximum length is \fcj{[3, 2, 1]}.

\end{flushleft}
 

\paragraph{Note:}

\begin{enumerate}
\item $1 \leq \lvert A\rvert, \lvert B\rvert \leq 1000$
\item $0 \leq A[i], B[i] < 100$
\end{enumerate}

\subsection{Dynamic Programming}

Suppose $F(i, j)$ is the length of common subarray ending at $A[i-1]$ and $B[j-1]$. The recursive relationship is simple: Only when $A[i] = B[j]$, $F(i, j) = F(i-1, j-1) +1 $. Otherwise, $F(i, j) = 0$

The reason is: if $A[i] = B[j] $ and $A[i+1]=B[j+1]$, the common length increments. Otherwise, this relationship is broken.

All we need to do, is record the continue increment common subarray length and update the maximum length accordingly.

\setcounter{lstlisting}{0}
\begin{lstlisting}[style=customc, caption={DP}]
int findLength( vector<int>& A, vector<int>& B )
{
    //note F[0][0]=0 because empty array length is zero
    vector<vector<int>> F( A.size() + 1, vector<int>( B.size() + 1, 0 ) );

    int ans = 0;

    for( size_t i = 1; i <= A.size(); ++i )
    {
        for( size_t j = 1; j <= B.size(); ++j )
        {
            if( A[i - 1] == B[j - 1] )
            {
                //increment the common length
                //ending at A[i-1] and B[j-1]
                F[i][j] = 1 + F[i - 1][j - 1];
                ans = ( max )( ans, F[i][j] );
            }
            else
            {
                //no common subarray
                //ending at A[i-1] and B[j-1]
                F[i][j] = 0;
            }
        }
    }
    return ans;
}
\end{lstlisting}