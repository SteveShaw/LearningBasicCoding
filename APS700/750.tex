\section{750 --- Number Of Corner Rectangles}
Given a grid where each entry is only 0 or 1, find the number of corner rectangles.

A \textit{corner rectangle} is 4 distinct 1s on the grid that form an axis-aligned rectangle. Note that only the corners need to have the value 1. Also, all four 1s used must be distinct.
 

\paragraph{Example 1:}

\begin{flushleft}
\textbf{Input}: 

\fcj{grid}:

\[
\begin{bmatrix}
1 & 0 & 0 & 1 & 0 \\
0 & 0 & 1 & 0 & 1 \\
0 & 0 & 0 & 1 & 0 \\
1 & 0 & 1 & 0 & 1
\end{bmatrix}
\] 


\textbf{Output}: 1

\textbf{Explanation}: 

There is only one corner rectangle, with corners \fcj{grid[1][2]}, \fcj{grid[1][4]}, \fcj{grid[3][2]}, \fcj{grid[3][4]}.



\end{flushleft} 

\paragraph{Example 2:}
\begin{flushleft}


\textbf{Input}: 

\fcj{grid}:

\[
\begin{bmatrix}
1 & 1 & 1\\
1 & 1 & 1\\
1 & 1 & 1
\end{bmatrix}
\] 

\textbf{Output}: 9

\textbf{Explanation}: 

There are four $2\times2$ rectangles, four $2\times 3$ and $3\times 2$ rectangles, and one $3\times3$ rectangle.

\end{flushleft} 

\paragraph{Example 3:}

\begin{flushleft}
\textbf{Input}: \fcj{grid = [[1, 1, 1, 1]]}

\textbf{Output}: 0

\textbf{Explanation}: Rectangles must have four distinct corners.

\end{flushleft}
 

\paragraph{Note:}

\begin{itemize}
\item The number of rows and columns of grid will each be in the range \fcj{[1, 200]}.
\item Each \fcj{grid[i][j]} will be either 0 or 1.
\item The number of 1s in the grid will be at most 6000.

\end{itemize}

\subsection{Count Corners}
We ask the question: for each additional row, how many more rectangles are added?

For each pair of 1s in the new row (say at \fcj{new_row[i]} and \fcj{new_row[j]}), we could create more rectangles where that pair forms the base. The number of new rectangles is the number of times some previous row had \fcj{row[i] = row[j] = 1}.

We maintain a count \fcj{count[i, j]}, the number of times we saw \fcj{row[i] = row[j] = 1}. When we process a new row, for every pair \fcj{new_row[i] = new_row[j] = 1}, we add \fcj{count[i, j]} to the answer, then we increment \fcj{count[i, j]}.

\setcounter{lstlisting}{0}
\begin{lstlisting}[style=customc, caption={Count Corners}]
int countCornerRectangles( vector<vector<int>>& grid )
{
    auto N = grid[0].size();
    //count[i,j] is the number of rectangles
    //for a row[i]=row[j]=1
    unordered_map<int, int> count;
    int ans = 0;
    for( const auto& row : grid )
    {
        for( size_t c1 = 0; c1 < N - 1; ++c1 )
        {
            if( row[c1] == 1 )
            {
                for( size_t c2 = c1 + 1; c2 < N; ++c2 )
                {
                    if( row[c2] == 1 )
                    {
                        //get the count of from count[c1, c2]
                        int key = c1 * N + c2;
                        auto it = count.find( key );
                        if( it == count.end() )
                        {
                            //we do not have old_row such that
                            //old_row[c1]=old_row[c2]=1
                            //we add a potential rectangle
                            count.emplace( key, 1 );
                        }
                        else
                        {
                            //we have old_row where
                            //old_row[c1]=old_row[c2]=1
                            //add the number of formed rectanges
                            //to the answer
                            ans += it->second;
                            //increments the formed rectanges at these
                            //two columns because we have a new rectangle
                            //formed with row[c1] and row[c2]
                            ++it->second;
                        }
                    }//end if(row2)
                }//end for(c2)
            }//end if(row1)
        } //end for(c1)
    }//end for (rows)
    return ans;
}
\end{lstlisting}

\subsection{Record Ones Position}
We actually don't need the \fcj{count}. If we found two rows have ones in the same column, we increments the count, say $x$. After scanning a new row, the number of formed rectangles added will be $x\times (x-1)/2$. To speed the process, we can use an array to record the columns 1s in the base row.

\begin{lstlisting}[style=customc, caption={Optimized}]
int countCornerRectangles( vector<vector<int>>& grid )
{
    auto M = grid.size();
    auto N = grid[0].size();
    int ans = 0;
    vector<size_t> ones;
    ones.reserve( N );
    for( size_t r1 = 0;  r1 < M - 1; ++r1 )
    {
        ones.clear();
        //record columns of 1s in row r1
        for( size_t c = 0; c < N; ++c )
        {
            if( grid[r1][c] )
            {
                ones.push_back( c );
            }
        }
        //for any new row r2 after r1
        //check the same columns in ones
        for( size_t r2 = r1 + 1; r2 < M; ++r2 )
        {
            int count = 0;
            for( auto p : ones )
            {
                if( grid[r2][p] )
                {
                    //increment the counter
                    ++count;
                }
            }
            //the new added rectangles will be
            //(count-1)*count/2;
            ans += ( count - 1 ) * count / 2;
        }
    }
    return ans;
}
\end{lstlisting}