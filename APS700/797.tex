\section{797 --- All Paths From Source to Target}

\textbf{Medium}

Given a directed, acyclic graph of $ N $ nodes. Find all possible paths from node 0 to node $ N-1 $, and return them in any order.

The graph is given as follows:  the nodes are 0, 1, $\ldots$, \fcj{graph.length - 1}.  \fcj{graph[i]} is a list of all nodes $j$ for which the edge $(i, j)$ exists.

\paragraph{Example:}
\begin{flushleft}


\textbf{Input}: \fcj{[[1,2], [3], [3], []]} 

\textbf{Output}: \fcj{[[0,1,3],[0,2,3]]} 

\textbf{Explanation}: The graph looks like this:

\begin{figure}[H]
\begin{tikzpicture}
[every node/.style={draw, circle, fill=gray!20!, minimum size=1cm},
%level 1/.style={sibling distance=25mm},
%level 2/.style={sibling distance=15mm},
thick, >=stealth, ->]
\node(0){0};
\node(1)[right=3cm of 0]{1};
\node(2)[below=3cm of 0]{2};
\node(3)[right=3cm of 2]{3};
\draw (0) -- (1);
\draw (0) -- (2);
\draw (2) -- (3);
\draw (1) -- (3);
\end{tikzpicture}
\end{figure}

There are two paths: \fcj{0 -> 1 -> 3} and \fcj{0 -> 2 -> 3}.
\end{flushleft}

\paragraph{Note:}

\begin{itemize}
\item The number of nodes in the graph will be in the range \fcj{[2, 15]}.
\item You can print different paths in any order, but you should keep the order of nodes inside one path.
\end{itemize}

\subsection{DFS}
This is a typical \textbf{DFS} algorithm.

\setcounter{lstlisting}{0}
\begin{lstlisting}[style=customc, caption={DFS}]
vector<vector<int>> allPathsSourceTarget( vector<vector<int>>& graph )
{
    vector<vector<int>> ans;
    vector<int> path{0};
    auto len_g( static_cast< int >( graph.size() ) );
    dfs( 0, len_g - 1, graph, path, ans );
    return ans;
}
void dfs( int start, int dst, vector<vector<int>>& G, vector<int>& path, vector<vector<int>>& ans )
{
    if( start == dst )
    {
        ans.emplace_back( begin( path ), end( path ) );
        return;
    }
    for( int next : G[start] )
    {
        path.push_back( next );
        dfs( next, dst, G, path, ans );
        path.pop_back();
    }
}
\end{lstlisting}

