\section{730 --- Count Different Palindromic Subsequences}
Given a string $S$, find the number of different non-empty palindromic sub-sequences in S, and return that number modulo $10^9 + 7$.

A sub-sequence of a string S is obtained by deleting 0 or more characters from S.

A sequence is palindromic if it is equal to the sequence reversed.

Two sequences $A_1$, $A_2$, $\ldots$ and $B_1$, $B_2$, $\ldots$ are different if there is some $i$ for which $A_i \neq B_i$.

\paragraph{Example 1:}
\begin{flushleft}


\textbf{Input}:
 
$S$: \fcj{'bccb'}

\textbf{Output}: 6

\textbf{Explanation}:
 
The 6 different non-empty palindromic subsequences are \fcj{'b', 'c', 'bb', 'cc', 'bcb', 'bccb'}.

Note that \fcj{'bcb'} is counted only once, even though it occurs twice.

\end{flushleft}

\paragraph{Example 2:}
\begin{flushleft}


\textbf{Input}:
 
$S$: \fcj{'abcdabcdabcdabcdabcdabcdabcdabcddcbadcbadcbadcbadcbadcbadcbadcba'}

\textbf{Output}: \textbf{104860361}

\textbf{Explanation}:
 
There are \textbf{3104860382} different non-empty palindromic subsequences, which is $104860361$ modulo $10^9 + 7$.
\end{flushleft}

\paragraph{Note:}

\begin{itemize}
\item The length of $S$ will be in the range \fcj{[1, 1000]}.
\item Each character $S[i]$ will be in the set \fcj{['a', 'b', 'c', 'd']}.
\end{itemize}

\subsection{Dynamic Programming By Three Dimensional Array}
Suppse $F(x,i,j)$ be the answer for the substring \fcj{S[i,j]} where \fcj{S[i] = S[j] = 'a' + x}. Since we only have 4 characters a, b, c and d, $0 \leq x < 4$. The recursive formula goes as follows:

\begin{itemize}
\item If \fcj{S[i] != 'a'+x}, $F(x,i,j) = F(x,i+1,j)$, note that here we don't include the first character \fcj{S[i]} due to our definition of $F(x,i,j)$.

\item If \fcj{S[j] != 'a'+x}, $F(x,i,j) = F(x, i, j-1)$, not including the last letter \fcj{S[j]}.

\item If \fcj{S[i] == S[j] == 'a'+x}, $F(x,i,j) = 2 + \sum\limits_{y=0}^{3}F(y, i+1,j-1)$. When the first and last characters are the same, we need to count all the distinct palindromes (for each of a, b, c and d) within the sub-window \fcj{S[i+1][j-1]} plus the 2 palindromes contributed by the first and last characters.

\end{itemize}

The final answer would be $\bmod(\sum\limits_{y=0}^{3}F(y,0,n-1), 1000000007)$ where $n$ is the length of the input string $S$. 

\setcounter{lstlisting}{0}
\begin{lstlisting}[style=customc, caption={3D DP}]
int countPalindromicSubsequences( string S )
{
    using v2d_t = vector<vector<int>>;

    vector<v2d_t> F( 4 );

    for( int i = 0; i < 4; ++i )
    {
        v2d_t tmp( S.size(), vector<int>( S.size(), 0 ) );
        F[i].swap( tmp );
    }

    //length = 1
    for( size_t i = 0; i < S.size(); ++i )
    {
        int x = S[i] - 'a';
        F[x][i][i] = 1;
    }

    //length = 2
    for( size_t i = 0; i < S.size() - 1; ++i )
    {
        int x = S[i] - 'a';
        int y = S[i + 1] - 'a';

        if( x == y )
        {
            //('aa') will have ('a', 'aa')
            //we only keep unique string
            F[x][i][i + 1] = 2;
        }
        else
        {
            //need to update F[x][i][i+1]
            //and F[y][i][i+1]
            F[x][i][i + 1] = F[x][i][i];
            F[y][i][i + 1] = F[y][i + 1][i + 1];
        }
    }

    for( size_t l = 3; l <= S.size(); ++l )
    {
        for( size_t i = 0; i + l <= S.size(); ++i )
        {
            auto j = i + l - 1;

            int x = S[i] - 'a';
            int y = S[j] - 'a';

            //we need to check and update (a), (b), (c) and (d)
            for( int ch = 0; ch < 4; ++ch )
            {
                if( x != ch )
                {
                    F[ch][i][j] = F[ch][i + 1][j];
                }
                else if( y != ch )
                {
                    F[ch][i][j] = F[ch][i][j - 1];
                }
                else
                {
                    //we want to find number of subsequence
                    //so x=y bring two palindrom subsequences
                    //S[i]/S[j] and (S[i],S[j])
                    F[ch][i][j] = 2;

                    //We have to add all of the valid palindromes
                    //between S[i+1,j-1]
                    for( int k = 0; k < 4; ++k )
                    {
                        F[ch][i][j] = bmod( F[ch][i][j] + F[k][i + 1][j - 1] );
                    }
                }

            }
        }//for(i)
    } //for(l)

    int ans = 0;

    for( int k = 0; k < 4; ++k )
    {
        ans = bmod( ans + F[k][0][S.size() - 1] );
    }

    return ans;
}

int bmod( int x )
{
    return x % 1000000007;
}
\end{lstlisting}

\subsection{Dynamic Programming By Two Dimensional Array}
Almost every palindrome is going to take one of these four forms: $a\ldots a$, $b\ldots b$, $c\ldots c$, or $d\ldots d$, where $\ldots$ represents a palindrome of zero or more characters. (The only other palindromes are a, b, c, d, and the empty string.)

Let's try to count palindromes of the form $a\ldots a$ (the other types are similar). Suppose $F(i, j)$ be the number of palindromes (including the empty string) in the string $T = S[i, j]$. To count palindromes in $T$ of the form $a\ldots a$, we will need to know the first and last occurrence of \fcj{'a'} in this string. We can make use of another 2D array $N$ whose entry $N(i,0)$ will be the next occurrence of \fcc{'a'} in \fcj{S[i:]}, $N(i,1)$ will be the next occurrence of \fcj{'b'} in \fcj{S[i:]}, and so on.

Also, we will need to know the number of unique letters in $T$ to count the single letter palindromes. We can use the information from $N$ to deduce it: if $N(i,0)$ is $\in [i, j]$, then \fcj{'a'} occurs in $T$, and so on.

There are many duplicate states $F(i,j)$ do not need to be computed, we could make use of a top-down variation of dynamic programming.