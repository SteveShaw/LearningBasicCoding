\section{799 --- Champagne Tower}

\textbf{Medium}

We stack glasses in a pyramid, where the first row has 1 glass, the second row has 2 glasses, and so on until the 100th row.  Each glass holds one cup (250ml) of champagne.

Then, some champagne is poured in the first glass at the top.  When the top most glass is full, any excess liquid poured will fall equally to the glass immediately to the left and right of it.  When those glasses become full, any excess champagne will fall equally to the left and right of those glasses, and so on.  (A glass at the bottom row has it's excess champagne fall on the floor.)

For example, after one cup of champagne is poured, the top most glass is full.  After two cups of champagne are poured, the two glasses on the second row are half full.  After three cups of champagne are poured, those two cups become full --- there are 3 full glasses total now.  After four cups of champagne are poured, the third row has the middle glass half full, and the two outside glasses are a quarter full.


Now after pouring some non-negative integer cups of champagne, return how full the $ j $-th glass in the $ i $-th row is (both $ i $ and $ j $ are 0 indexed.)

\paragraph{Example 1:}

\begin{flushleft}
\textbf{Input}: \fcj{poured = 1}, \fcj{query_glass = 1}, \fcj{query_row = 1}

\textbf{Output}: 0.0

\textbf{Explanation}: 

We poured 1 cup of champange to the top glass of the tower (which is indexed as (0, 0)). There will be no excess liquid so all the glasses under the top glass will remain empty.
\end{flushleft}

\paragraph{Example 2:}

\begin{flushleft}
\textbf{Input}: \fcj{poured = 2}, \fcj{query_glass = 1}, \fcj{query_row = 1}

\textbf{Output}: 0.5

\textbf{Explanation}: 

We poured 2 cups of champange to the top glass of the tower (which is indexed as \fcj{(0, 0)}). There is one cup of excess liquid. The glass indexed as \fcj{(1, 0)} and the glass indexed as \fcj{(1, 1)} will share the excess liquid equally, and each will get half cup of champange.
\end{flushleft}
 

\paragraph{Note:}

\begin{flushleft}
\item \fcj{poured} will be in the range of $ [0, 10 ^ 9] $.
\item \fcj{query_glass} and \fcj{query_row} will be in the range of \fcj{[0, 99]}.
\end{flushleft}

\subsection{Dynamic Programming}
Instead of keeping track of how much champagne should end up in a glass, keep track of the total amount of champagne that flows through a glass. For example, if \fcj{poured = 10} cups are poured at the top, the total flow-through of the top glass is 10; the total flow-through of each glass in the second row is 4.5, and so on.

We use $F[i][j]$ as the amount of champagne at $i$-th row and $j$-th glass. At start $F[0][0]$ is the total poured champagne. Then we split the excess amount of champagne into next level's glasses.

Since we have to compute the next row after \fcj{query_row}, the size of the $F$ is will be \fcj{(query_row+2, query_row+2)}

\setcounter{lstlisting}{0}
\begin{lstlisting}[style=customc, caption={Simulation}]
double champagneTower( int poured, int query_row, int query_glass )
{
    vector<vector<double>> F( query_row + 2, vector<double>( query_row + 2, 0. ) );
    //at start, we have poured amount of champagne at the top
    F[0][0] = poured;
    for( int i = 0; i <= query_row; ++i )
    {
        //simulate water flow from 0 to i
        for( auto j( 0 ); j <= i; ++j )
        {
            if( F[i][j] > 1 )
            {
                //(i+1,j) and (i+1,j+1) are the
                //glasses that champagne will flow from (i,j)
                F[i + 1][j] += ( F[i][j] - 1 ) / 2.0;
                F[i + 1][j + 1] += ( F[i][j] - 1 ) / 2.0;
            }
        }
    }
    return ( min )( 1.0, F[query_row][query_glass] );
}
\end{lstlisting}