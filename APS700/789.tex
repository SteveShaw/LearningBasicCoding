\section{789 --- Escape The Ghosts}
You are playing a simplified Pacman game. 

You start at the point \fcj{(0, 0)}, and your destination is \fcj{(target[0], target[1])}. There are several ghosts on the map, the $i$--th ghost starts at \fcj{(ghosts[i][0], ghosts[i][1])}.

Each turn, you and all ghosts simultaneously *may* move in one of 4 cardinal directions: north, east, west, or south, going from the previous point to a new point 1 unit of distance away.

You escape if and only if you can reach the target before any ghost reaches you (for any given moves the ghosts may take.)  If you reach any square (including the target) at the same time as a ghost, it doesn't count as an escape.

Return \fcj{true} if and only if it is possible to escape.

\paragraph{Example 1:}

\begin{flushleft}
\textbf{Input}: 

\fcj{ghosts = [[1, 0], [0, 3]]}

\fcj{target = [0, 1]}

\textbf{Output}: \fcj{true}

\textbf{Explanation}: 

You can directly reach the destination \fcj{(0, 1)} at time 1, while the ghosts located at \fcj{(1, 0)} or \fcj{(0, 3)} have no way to catch up with you.

\end{flushleft}

\paragraph{Example 2:}

\begin{flushleft}

\textbf{Input}: 

\fcj{ghosts = [[1, 0]]}

\fcj{target = [2, 0]}

\textbf{Output}: \fcj{false}

\textbf{Explanation}: 

You need to reach the destination (2, 0), but the ghost at (1, 0) lies between you and the destination.

\end{flushleft}

\paragraph{Example 3:}

\begin{flushleft}
\textbf{Input}: 

\fcj{ghosts = [[2, 0]]}

\fcj{target = [1, 0]}

\textbf{Output}: \fcj{false}

\textbf{Explanation}: 

The ghost can reach the target at the same time as you.

\end{flushleft}

\paragraph{Note:}

\begin{itemize}
\item All points have coordinates with absolute value \fcj{<= 10000}.
\item The number of ghosts will not exceed 100.
\end{itemize}

\subsection{Taxicab Distance}
The taxicab distance is the number of moves required to get from point A to point B in our grid. It is calculated as 

\fcj{dist(A, B) = abs(A.x - B.x) + abs(A.y - B.y)}.

Let's say we start at $S$, the ghost starts at $G$, the target is $T$, and the ghost first meets us at $X$. This implies 

\fcj{dist(G, X) <= dist(S, X)}, 

as the ghost must reach X before or at the time that we do.

Now, if the ghost travels from $G$ to $X$ and then to $T$, it will reach T at time 

\fcj{dist(G, T) <= dist(G, X) + dist(X, T) <= dist(S, X) + dist(X, T)}.

The first inequality is because of the triangle inequality that all distance metrics satisfy.

The above shows that it is at least as good for the ghost to just travel directly to the target: if it could reach us beforehand (at $X$), it will also reach us if it goes to $X$ then to $T$, and then it would reach us if it just went directly to $T$.

Also, if the ghost goes directly to the target, then a necessary condition is clearly that we get to the target before the ghost.

Once we can make the assumption that all parties just go directly to the target in the shortest time possible, the problem is greatly simplified.

\setcounter{lstlisting}{0}
\begin{lstlisting}[style=customc, caption={Taxicab Distance}]
bool escapeGhosts( vector<vector<int>>& ghosts, vector<int>& target )
{
    auto taxicab_dist = [&target]( const vector<int>& P )
    {

        return abs( P[0] - target[0] ) + abs( P[1] - target[1] );
    };
    //the taxicab distance from source to target
    int dist = abs( target[0] ) + abs( target[1] );
    for( const auto& ghost : ghosts )
    {
        if( taxicab_dist( ghost ) <= dist )
        {
            //this ghost cannot be avoid
            return false;
        }
    }
    return true;
}
\end{lstlisting}