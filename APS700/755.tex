\section{755 --- Pour Water}
We are given an elevation map, \fcj{heights[i]} representing the height of the terrain at that index. The width at each index is 1. After $ V $ units of water fall at index $K$, how much water is at each index?

Water first drops at index $K$ and rests on top of the highest terrain or water at that index. Then, it flows according to the following rules:

\begin{itemize}
\item If the droplet would eventually fall by moving left, then move left.
\item Otherwise, if the droplet would eventually fall by moving right, then move right.
\item Otherwise, rise at it's current position.
\end{itemize}

Here, \textit{eventually fall} means that the droplet will eventually be at a lower level if it moves in that direction. Also, \textit{level} means the height of the terrain plus any water in that column.

We can assume there's infinitely high terrain on the two sides out of bounds of the array. Also, there could not be partial water being spread out evenly on more than 1 grid block --g- each unit of water has to be in exactly one block.

\paragraph{Example 1:}
\begin{flushleft}
\textbf{Input}: 

\fcj{heights = [2,1,1,2,1,2,2]}, $V = 4$, $K = 3$

\textbf{Output}: \fcj{[2,2,2,3,2,2,2]}

\textbf{Explanation}:
%#       #
%#       #
%##  # ###
%#########
% 0123456    <- index
%
%The first drop of water lands at index K = 3:
%
%#       #
%#   w   #
%##  # ###
%#########
% 0123456    
%
%When moving left or right, the water can only move to the same level or a lower level.
%(By level, we mean the total height of the terrain plus any water in that column.)
%Since moving left will eventually make it fall, it moves left.
%(A droplet "made to fall" means go to a lower height than it was at previously.)
%
%#       #
%#       #
%## w# ###
%#########
% 0123456    
%
%Since moving left will not make it fall, it stays in place.  The next droplet falls:
%
%#       #
%#   w   #
%## w# ###
%#########
% 0123456  
%
%Since the new droplet moving left will eventually make it fall, it moves left.
%Notice that the droplet still preferred to move left,
%even though it could move right (and moving right makes it fall quicker.)
%
%#       #
%#  w    #
%## w# ###
%#########
% 0123456  
%
%#       #
%#       #
%##ww# ###
%#########
% 0123456  
%
%After those steps, the third droplet falls.
%Since moving left would not eventually make it fall, it tries to move right.
%Since moving right would eventually make it fall, it moves right.
%
%#       #
%#   w   #
%##ww# ###
%#########
% 0123456  
%
%#       #
%#       #
%##ww#w###
%#########
% 0123456  
%
%Finally, the fourth droplet falls.
%Since moving left would not eventually make it fall, it tries to move right.
%Since moving right would not eventually make it fall, it stays in place:
%
%#       #
%#   w   #
%##ww#w###
%#########
% 0123456  
%
%The final answer is [2,2,2,3,2,2,2]:
%
%    #    
% ####### 
% ####### 
% 0123456 


\end{flushleft}

\paragraph{Example 2:}

\begin{flushleft}
\textbf{Input}: 

\fcj{heights = [1,2,3,4]}, $V = 2$, $K = 2$

\textbf{Output}: \fcj{[2,3,3,4]}

\textbf{Explanation}:

The last droplet settles at index 1, since moving further left would not cause it to eventually fall to a lower height.

\end{flushleft}


\paragraph{Example 3:}
\begin{flushleft}


\textbf{Input}: 

\fcj{heights = [3,1,3]}, $V = 5$, $K = 1$

\textbf{Output}: \fcj{[4,4,4]}
\end{flushleft}
\paragraph{Note:}

\begin{itemize}
\item \fcj{heights} will have length in \fcj{[1, 100]} and contain integers in \fcj{[0, 99]}.
\item $V$ will be in range \fcj{[0, 2000]}.
\item $K$ will be in range \fcj{[0, heights.length - 1]}.
\end{itemize}

\subsection{Simulation}
When a droplet falls, we check if it is possible for it to fall left, i.e., if \fcj{heights[i - 1] <= heights[i]}. If the condition is met, the water will fall to a \textit{candidate} block in that direction. We keep track of every time we actually fall at index $x$. If we \fcj{"eventually fall"} ($x\neq K$) as described in the problem statement, the unit amount of water will stay there. 

Otherwise, we perform a similar check by going right.

If either left and right cannot drop the water, it will stay in $K$.

\setcounter{lstlisting}{0}
\begin{lstlisting}[style=customc, caption={Simulation}]
vector<int> pourWater( vector<int>& heights, int V, int K )
{
    auto L = static_cast<int>( heights.size() );
    for( int i = 0; i < V; ++i )
    {
        if( ( K > 0 ) && ( heights[K - 1] <= heights[K] ) )
        {
            int fill_p = K;
            //move left
            int l = K;
            //the water can flow left
            //if current height is no larger than its right.
            while( ( l > 0 ) && ( heights[l - 1] <= heights[l] ) )
            {
                if( heights[l - 1] < heights[l] )
                {
                    //the water can only fall into
                    //current index when the height is less than its right
                    fill_p = l - 1;
                    //we cannot break here because maybe some lower block
                    //lie left of current block
                }
                --l;
            }
            if( heights[fill_p] < heights[K] )
            {
                //we found a place that the water
                //eventually fall
                ++heights[fill_p];
                continue;
            }
        }
        if( ( K < L - 1 ) && ( heights[K + 1] <= heights[K] ) )
        {
            //similar check for moving right
            int fill_p = K;
            int r = K;
            //move right
            while( ( r < L - 1 ) && ( heights[r] >= heights[r + 1] ) )
            {
                if( heights[r] > heights[r + 1] )
                {
                    fill_p = r + 1;
                    //similarly, we cannot break here
                    //because some lower block may lie right of current block
                }
                ++r;
            }
            if( heights[fill_p] < heights[K] )
            {
                ++heights[fill_p];
                continue;
            }
        }
        //neither left nor right
        //water will fall
        //stay at K
        ++heights[K];
    }
    return heights;
}
\end{lstlisting}