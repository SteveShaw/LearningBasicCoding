\section{723. Candy Crush}
This question is about implementing a basic elimination algorithm for Candy Crush.

Given a 2D integer array \fcj{board} representing the grid of candy, different positive integers\fcj{ board[i][j]} represent different types of candies. A value of \fcj{board[i][j] = 0} represents that the cell at position $(i, j)$ is empty. The given board represents the state of the game following the player's move. Now, you need to restore the board to \textbf{a stable state} by crushing candies according to the following rules:

\begin{itemize}
\item If three or more candies of the same type are adjacent vertically or horizontally, \textbf{crush} them all at the same time -- these positions become empty.
\item After crushing all candies simultaneously, if an empty space on the board has candies on top of itself, then these candies will drop until they hit a candy or bottom at the same time. (No new candies will drop outside the top boundary.)
\item After the above steps, there may exist more candies that can be crushed. If so, you need to repeat the above steps.
\item If there does not exist more candies that can be crushed (ie. the board is stable), then return the current board.
\end{itemize}

You need to perform the above rules until the board becomes stable, then return the current board.

\paragraph{Example:}
\begin{flushleft}


\textbf{Input}:

\fcj{board}:

\[
\begin{bmatrix}
110 & 5 & 112 & 113 & 114\\
210 & 211 & 5 & 213 & 214\\
310 & 311 & 3 & 313 & 314\\
410 & 411 & 412 & 5 & 414\\
5 & 1 & 512 & 3 & 3\\
610 & 4 & 1 & 613 & 614\\
710 & 1 & 2 & 713 & 714\\
810 & 1 & 2 & 1 & 1\\
1 & 1 & 2 & 2 & 2\\
4 & 1 & 4 & 4 & 1014
\end{bmatrix}
\]

\textbf{Output}:
\[
\begin{bmatrix}
0 & 0 & 0 & 0 & 0\\
0 & 0 & 0 & 0 & 0\\
0 & 0 & 0 & 0 & 0\\
110 & 0 & 0 & 0 & 114\\
210 & 0 & 0 & 0 & 214\\
310 & 0 & 0 & 113 & 314\\
410 & 0 & 0 & 213 & 414\\
610 & 211 & 112 & 313 & 614\\
710 & 311 & 412 & 613 & 714\\
810 & 411 & 512 & 713 & 1014
\end{bmatrix}
\]

\textbf{Explanation}: 

\end{flushleft}
 

\paragraph{Note:}

\begin{itemize}
\item The length of \fcj{board} will be in the range $[3, 50]$.
\item The length of \fcj{board[i]} will be in the range $[3, 50]$.
\item Each \fcj{board[i][j]} will initially start as an integer in the range $[1, 2000]$.

\end{itemize}

\subsection{Simulation}
The algorithm consists of two major steps: a crush step, and a gravity step. 

\paragraph{Crush Step}
When crushing, one difficulty is that we might accidentally crush candy that is part of another row. For example, if the board is:

\[
\begin{bmatrix}
1 & 2 & 3\\
1 & 4 & 5\\
1 & 1 & 1
\end{bmatrix}
\]

and we crush the vertical row of 1s early, we may not see there was also a horizontal row (the 3rd row).

To remedy this, we should flag candy that should be crushed first. We could use an boolean array, or we could mark it directly on the board by making the entry negative. 

For each row or column, we could use a sliding window to find contiguous parts of the same character. If any of these parts have length 3 or more, we should flag them.

Alternatively, for each line, we could look at each width-3 slice of the line: if they are all the same, then we should flag those 3.

After, we can crush the candy by setting all flagged board cells to zero.

\paragraph{Gravity Step}
For each column, we want all the candy to go to the bottom. One way is to iterate through and keep a stack of the \textbf{not yet crushed} candy, popping and setting as we iterate through the column in reverse order.

Alternatively, we could use a sliding window, maintaining a read and write pointer. As the read pointer iterates through the column in reverse order, when the read pointer sees candy, the write pointer will write it down and move one place. After end of iteration, the write pointer will write zeros to the remainder of the column.

\setcounter{lstlisting}{0}
\begin{lstlisting}[style=customc, caption={Simulation}]
vector<vector<int>> candyCrush( vector<vector<int>>& board )
{
    auto M = board.size();
    auto N = board[0].size();

    bool bCrush = false;
    //check each column if there are
    //three same candies are put together
    for( size_t r = 0; r < M; ++r )
    {
        for( size_t c = 0; c < N - 2; ++c )
        {
            int candy = abs( board[r][c] );
            int candy2 = abs( board[r][c + 1] );
            int candy3 = abs( board[r][c + 2] );

            if( ( candy > 0 ) && ( candy == candy2 ) && ( candy2 == candy3 ) )
            {
                bCrush = true;

                board[r][c] = -candy;
                board[r][c + 1] = -candy;
                board[r][c + 2] = -candy;
            }
        }
    }
    //check each row if there are
    //three same canides are put together
    for( size_t c = 0; c < N; ++c )
    {
        for( size_t r = 0; r < M - 2; ++r )
        {
            int candy = abs( board[r][c] );
            int candy2 = abs( board[r + 1][c] );
            int candy3 = abs( board[r + 2][c] );

            if( ( candy > 0 ) && ( candy == candy2 ) && ( candy2 == candy3 ) )
            {
                bCrush = true;
                board[r][c] = -candy;
                board[r + 1][c] = -candy;
                board[r + 2][c] = -candy;
            }
        }
    }
    //simulate gravity effect
    for( size_t c = 0; c < N; ++c )
    {
        size_t writeR = M;

        for( size_t readR = M; readR >= 1; --readR )
        {
            if( board[readR - 1][c] > 0 )
            {
                board[writeR - 1][c] = board[readR - 1][c];
                --writeR;
            }
        }

        while( writeR >= 1 )
        {
            board[writeR - 1][c] = 0;
            --writeR;
        }
    }
    //if still need crush
    //we crush the whole board again
    return bCrush ? candyCrush( board ) : board;
}
\end{lstlisting}

