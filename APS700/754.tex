\section{754 --- Reach a Number}
You are standing at position 0 on an infinite number line. There is a goal at position \fcj{target}.

On each move, you can either go left or right. During the $n$--th move (starting from 1), you take $n$ steps.

Return the minimum number of steps required to reach the destination.

Example 1:

\begin{flushleft}
\textbf{Input}: \fcj{target = 3}

\textbf{Output}: 2

\textbf{Explanation}:

On the first move we step from 0 to 1.

On the second step we step from 1 to 3.
\end{flushleft}

\paragraph{Example 2:}

\begin{flushleft}
\textbf{Input}: \fcj{target = 2}

\textbf{Output}: 3

\textbf{Explanation}:

On the first move we step from 0 to 1.

On the second move we step  from 1 to $-1$.

On the third move we step from $-1$ to 2.
\end{flushleft}

\paragraph{Note:}

\begin{itemize}
\item \fcj{target} will be a non-zero integer in the range $[-10^9, 10^9]$.
\end{itemize}

\subsection{Mathematical}
The problem is equivalent to put \fcj{+} and \fcj{-} signs on the numbers 1, 2, 3, $\ldots$, $k$ so that the sum is \fcj{target}.

When \fcj{target < 0}, we can switch the signs of all the numbers to make sure the sum is \fcj{-target}. Thus, the answer for \fcj{target} is the same as \fcj{-target}, and so without loss of generality, we can consider only \fcj{target > 0}.

Suppose $k$ is the smallest number with \fcj{S = 1 + 2 + ... + k >= target}. If \fcj{S == target}, the answer is clearly $k$.

If \fcj{S > target}, we need to change some number signs. If \fcj{delta = S - target} is even, we can always find a subset of \fcj{[1, 2, ..., k]} equal to \fcj{delta / 2} and switch the signs. Thus $S$ will change to \fcj{S - delta/2 - delta/2}, i.e. \fcj{S - delta = target}. so the answer is $k$. (This depends on \fcj{q = delta / 2} being at most $S$.) 

The proof is simple: either $q \leq k$ and we choose it, or we choose $k$ in our subset and try to solve the same instance of the problem for $q = q - k$ and the set \fcj{[1, 2, ..., k-1]}

Otherwise, if \fcj{delta} is \textbf{odd}, we can't do it, as every sign change from positive to negative changes the sum by an \textit{even} number. So let's consider a candidate answer of $k+1$, which change \fcj{delta} to \fcj{delta + k + 1}. If $k$ is odd, then \fcj{delta} is still be odd and we can have an answer of $k+2$. Otherwise, \fcj{delta} will be \textbf{even}, and we will have an answer of $k+1$.

Thus, we subtract \fcj{++k} from \fcj{target} until it goes non-positive. Then $k$ will be as described, and \fcj{target} will be \fcj{delta} as described. We can output the four cases above: 

\begin{enumerate}
\item if \fcj{delta} is \textit{even} then the answer is $k$, 

\item if \fcj{delta} is \textit{odd} then the answer is $k+1$ or $k+2$ depending on $k$ is \textit{even} or \textit{odd}.
\end{enumerate}

\setcounter{lstlisting}{0}
\begin{lstlisting}[style=customc, caption={Mathematical}]
int reachNumber( int target )
{
    int k = 0;
    int S = 0;
    //change target to positive number
    if( target < 0 )
    {
        target = - target;
    }
    //find smallest k such that
    //S=1+2+...+k >= target
    while( S < target )
    {
        ++k;
        S += k;
    }
    int delta = S - target;
    if( delta & 1 )
    {
        //odd
        if( k & 1 )
        {
            //delta+k+2 is even
            return k + 2;
        }
        //delta+k+1 is even
        return k + 1;
    }
    //we can find subset in [1,2,..,k]
    //whose sum is delta/2
    //and change their sign so that the sum
    //is -delta/2. so S is changed to S-delta/2-delta/2
    //=S-delta=target
    return k;
}
\end{lstlisting}