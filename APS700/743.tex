\section{743 --- Network Delay Time}
There are $N$ network nodes, labeled 1 to $N$.

Given \fcj{times}, a list of travel times as directed edges\fcj{ times[i] = (u, v, w)}, where $u$ is the source node, $v$ is the target node, and $w$ is the time it takes for a signal to travel from source to target.

Now, we send a signal from a certain node $K$. How long will it take for all nodes to receive the signal? If it is impossible, return $-1$.

\paragraph{Example 1:}

\begin{flushleft}

\begin{figure}[H]
\begin{tikzpicture}
[every node/.style={draw, circle, fill=gray!20!, minimum size=5mm}, >=stealth, ->, thick, auto]
\node(0){2};
\node(1)[below = 5mm of 0, xshift=-25mm]{1};
\node(2)[below = 18mm of 0, xshift=8mm]{3};
\node(3)[below = 5mm of 2, xshift=-25mm]{4};
\draw (0) to node [draw=none,fill=none,swap] {1} (1);
\draw (0) to node [draw=none,fill=none]{1} (2);
\draw (2) to node [draw=none,fill=none]{1} (3);

\end{tikzpicture}
\end{figure}

\textbf{Input}: 

\fcj{times = [[2,1,1],[2,3,1],[3,4,1]]}, $N = 4$, $K = 2$

\textbf{Output}: 2
\end{flushleft}
 

\paragraph{Note:}

\begin{itemize}
\item $N$ will be in the range \fcj{[1, 100]}.
\item $K$ will be in the range $[1, N]$.
\item The length of times will be in the range \fcj{[1, 6000]}.
\item All edges \fcj{times[i] = (u, v, w)} will have $1 \leq u, v \leq N$ and $ 0 \leq w \leq 100$.
\end{itemize}

\subsection{Dijkstra's Algorithm}
The basic idea is that we find the shorted path from $K$ to each node. At the end, the maximum weight of all the nodes will be the answer. Since Dijkstra's algorithm will get shorted path for each node, we only need run once to get the result.

\setcounter{lstlisting}{0}
\begin{lstlisting}[style=customc, caption={Shortest Path}]
int networkDelayTime( vector<vector<int>>& times, int N, int K )
{
    //build adjacent graph from the given edges
    vector<vector<pair<int, int>>> g( N );
    for( const auto& t : times )
    {
        g[t[0] - 1].emplace_back( t[1] - 1, t[2] );
    }
    //delays record each node's shorted path
    //from K
    vector<int> delays( N, -1 );
    //the time to K is zero
    delays[K - 1] = 0;
    //the comparator for the priority queue
    auto cmp = [&delays]( int n1, int n2 )
    {
        return delays[n1] > delays[n2];
    };
    //maximum priority queue (minmum is at the top)
    priority_queue<int, vector<int>, decltype( cmp )> pq( cmp );
    //add start at the beginning
    pq.emplace( K - 1 );
    while( !pq.empty() )
    {
        auto t = pq.top();
        pq.pop();
        if( g[t].empty() )
        {
            continue;
        }
        //find adjacent nodes
        for( const auto& adj : g[t] )
        {
            int w = get<1>( adj );
            int node = get<0>( adj );

            if( delays[node] < 0 )
            {
                //this node has not been visisted
                delays[node] = delays[t] + w;
                pq.emplace( node );
            }
            else if( delays[node]  > delays[t] + w )
            {
                //got shorter path
                delays[node] = delays[t] + w;
                pq.emplace( node );
            }
        }
    }

    //the maximum delay of all the nodes is the answer
    //but we have to check if some nodes are not reached
    //For unreached nodes, their delay is -1
    auto t = minmax_element( begin( delays ), end( delays ) );
    if( *get<0>( t ) < 0 )
    {
        return -1;
    }
    return *get<1>( t );
}
\end{lstlisting}
