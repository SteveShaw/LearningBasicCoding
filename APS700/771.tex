\section{771. Jewels and Stones}
You're given strings \fcj{J} representing the types of stones that are jewels, and \fcj{S} representing the stones you have.  Each character in \fcj{S} is a type of stone you have.  You want to know how many of the stones you have are also jewels.

The letters in \fcj{J} are guaranteed distinct, and all characters in \fcj{J} and \fcj{S} are letters. Letters are case sensitive, so \fcj{"a"} is considered a different type of stone from \fcj{"A"}.

\paragraph{Example 1:}

\begin{flushleft}
\textbf{Input}: \fcj{J = "aA"}, \fcj{S = "aAAbbbb"}

\textbf{Output}: 3
\end{flushleft}

\paragraph{Example 2:}

\begin{flushleft}
\textbf{Input}: \fcj{J = "z"}, \fcj{S = "ZZ"}

\textbf{Output}: 0
\end{flushleft}

\paragraph{Note:}

\begin{itemize}
\item \fcj{S} and \fcj{J} will consist of letters and have length at most 50.

\item The characters in \fcj{J} are distinct.
\end{itemize}

\subsection{Hash Set}
\textbf{Easy} Problem. Simulate hash set by an array

\setcounter{lstlisting}{0}
\begin{lstlisting}[style=customc, caption={Hash Set}]
int numJewelsInStones( string J, string S )
{
    array<int, 52> m;
    m.fill( 0 );
    //1. add letters in J to a hash set
    for( auto c : J )
    {
        if( ( c <= 'Z' ) && ( c >= 'A' ) )
        {
            m[c - 'A'] = 1;
        }
        else
        {
            m[c - 'a' + 26] = 1;
        }
    }
    //2. check each letter in S against the hash set
    int ans = 0;
    for( auto c : S )
    {
        if( ( c <= 'Z' ) && ( c >= 'A' ) )
        {
            ans += m[c - 'A'];
        }
        else
        {
            ans += m[c - 'a' + 26];
        }
    }
    return ans;
}
\end{lstlisting}