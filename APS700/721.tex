\section{721 --- Accounts Merge}
Given a list \fcj{accounts}, each element \fcj{accounts[i]} is a list of strings, where the first element \fcj{accounts[i][0]} is a name, and the rest of the elements are emails representing emails of the account.

Now, we would like to merge these accounts. Two accounts definitely belong to the same person if there is some email that is common to both accounts. Note that even if two accounts have the same name, they may belong to different people as people could have the same name. A person can have any number of accounts initially, but all of their accounts definitely have the same name.

After merging the accounts, return the accounts in the following format: the first element of each account is the name, and the rest of the elements are emails \textbf{in sorted order}. The accounts themselves can be returned in any order.

\paragraph{Example 1:}

\begin{flushleft}

\textbf{Input}: 

\begin{lstlisting}[style=customc]
accounts = [
["John", "johnsmith@mail.com", "john00@mail.com"],
["John", "johnnybravo@mail.com"], 
["John", "johnsmith@mail.com", "john_newyork@mail.com"], 
["Mary", "mary@mail.com"]
]
\end{lstlisting}

\textbf{Output}: 

\begin{lstlisting}[style=customc]
["John", 'john00@mail.com', 'john_newyork@mail.com', 'johnsmith@mail.com'],
["John", "johnnybravo@mail.com"],
["Mary", "mary@mail.com"]
\end{lstlisting}


\textbf{Explanation}: 

The first and third John's are the same person as they have the common email \fcj{"johnsmith@mail.com"}.

The second John and Mary are different people as none of their email addresses are used by other accounts.

We could return these lists in any order, for example the answer 

\begin{lstlisting}[style=customc]
[
['Mary', 'mary@mail.com'], 
['John', 'johnnybravo@mail.com'], 
['John', 'john00@mail.com', 'john_newyork@mail.com','johnsmith@mail.com']
]
\end{lstlisting}

would still be accepted.



\end{flushleft}

\paragraph{Note:}

\begin{itemize}
\item The length of \fcj{accounts} will be in the range \fcj{[1, 1000]}.
\item The length of \fcj{accounts[i]} will be in the range \fcj{[1, 10]}.
\item The length of \fcj{accounts[i][j]} will be in the range \fcj{[1, 30]}.
\end{itemize}

\subsection{Union Find}
This problem is actually finding connected components in a graph. We can use DFS or \textbf{Union Find}.

To process the problem easily, we map each email to an integer index. 

The algorithm steps include
\begin{enumerate}
\item Map each email to its user.
\item Map each email to a unique ID.
\item For each user's email group, we use the first email's parent as the parent of the whole group and relink the parents.
\item Go over each email, insert emails with same parent into a group.
\end{enumerate}

\setcounter{lstlisting}{0}
\begin{lstlisting}[style=customc, caption={Union Find}]
vector<vector<string>> accountsMerge( vector<vector<string>>& accounts )
{
    int id = 0;
    //map each email to a unique id
    unordered_map<string_view, int> emailId;
    //map each email to its user
    unordered_map<string_view, string_view> emailUser;
    //map email to id
    for( const auto& A : accounts )
    {
        string_view user( A[0] );
        if( emailId.find( user ) == emailId.end() )
        {
            emailId.emplace( user, id++ );
        }

        for( size_t i = 1; i < A.size(); ++i )
        {
            string_view email( A[i] );
            if( emailId.find( email ) == emailId.end() )
            {
                emailId.emplace( email. id++ );
            }
        }
    }
    //create parents array
    vector<int> x_parents( id, 0 );

    for( int i = 0; i < id; ++i )
    {
        x_parents[i] = i;
    }

    //go over input accounts again
    //to relink the parents
    for( const auto& A : accounts )
    {
        string_view leader( A[1] );
        //we use the first email's parent
        //as the whole group emails' parent
        int gp = find( emailId[leader], x_parents );

        for( size_t i = 1; i < A.size(); ++i )
        {
            string_view follower( A[i] );
            //find this email's parent
            int p = find( emailId[follower], x_parents );
            //set this email's parent to the comment parent
            x_parents[p] = gp;
        }
    }
    //group same parents' emails togther
    //using set because it requires the emails are sorted
    unordered_map<int, set<string_view>> emailGroup;
    //iterate over each email
    for( const auto& eu : emailUser )
    {
        //get its parent
        int parent = find( emailId[eu.first], x_parents );
        //put into one group under the same parent
        emailGroup[parent].emplace( eu.first );
    }
    //output
    vector<vector<string>> ans;

    for( const auto& eg : emailGroup )
    {
        //get the user name of the group
        //from one of emails
        auto it = eg.second.begin();
        auto user = emailUser[*it];

        vector<string> A;
        A.emplace_back( user );

        //then add emails to A
        for( ; it != eg.second.end(); ++it )
        {
            A.emplace_back( *it );
        }
        ans.push_back( move( A ) );
    }
    return ans;
}
//find parent of x
int find( int x, vector<int>& parents )
{
    while( x != parents[x] )
    {
        x = parents[x];
    }

    return x;
}
\end{lstlisting}