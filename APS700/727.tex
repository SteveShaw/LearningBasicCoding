\section{727 --- Minimum Window Subsequence}
Given strings $ S $ and $ T $, find the minimum (contiguous) \textbf{substring} $ W $ of $ S $, so that $ T $ is a \textbf{subsequence} of $ W $.

If there is no such window in $ S $ that covers all characters in $ T $, return the empty string \fcj{""}. If there are multiple such minimum-length windows, return the one with the left-most starting index.

\paragraph{Example 1:}

\begin{flushleft}
\textbf{Input}: $S$: \fcj{"abcdebdde"}, $T$: \fcj{"bde"}

\textbf{Output}: \fcj{"bcde"}

\textbf{Explanation}:
 
\fcj{"bcde"} is the answer because it occurs before \fcj{"bdde"} which has the same length.

\fcj{"deb"} is not a smaller window because the elements of $ T $ in the window must occur in order.
\end{flushleft}
 

\paragraph{Note:}

\begin{itemize}
\item All the strings in the input will only contain lowercase letters.
\item The length of S will be in the range \fcj{[1, 20000]}.
\item The length of T will be in the range \fcj{[1, 100]}.

\end{itemize}
\subsection{Two Indexes Sliding Window}
We can make use of a sliding window approach framework, but something is tricky here.

This is: when finding a sliding window that contains $T$, we need to trace back from the right edge of the sliding window back to see if we can shrink the window.

And then, we need to start the right edge of sliding window from the next index of current left edge.

\setcounter{lstlisting}{0}
\begin{lstlisting}[style=customc, caption={Sliding Window}]
string minWindow( string S, string T )
{
    string_view sv( S.c_str(), S.size() );

    size_t ti = 0;

    size_t left = 0;

    string_view best;

    size_t best_len = S.size();

    for( size_t right = 0; right < S.size(); )
    {
        if( T[ti] != sv[right] )
        {
            //only increment index
            //in S
            ++right;
            continue;
        }
        //increment index
        //in T
        ++ti;

        if( ti == T.size() )
        {
            //T is matched completely
            //we need to backtrace from current s[right] and t[ti]
            auto x = right + 1;
            auto y = T.size();

            //since x and y are size_t type
            //we need to not use y>=0 or x >= left
            //[important]: the process of trace back
            while( ( x > left ) && ( y > 0 ) )
            {
                if( sv[x - 1] == T[y - 1] )
                {
                    --y;
                }

                --x;
            }
            //now left is the left edge
            //of shrinked window
            left = x;
            //update minimum length
            //and get the related string
            if( right - left + 1 < best_len )
            {
                best = sv.substr( left, right - left + 1 );
                best_len = right - left + 1;
            }
            //reset index in T to zero
            ti = 0;
            //[important]: set right to the next index of left
            //since we may find another smaller window
            //starting from left+1
            right = left + 1;
        }
        else
        {
            //T has not been matched completely
            //continue incrementing index in S
            ++right;
        }
    }
    return { best.data(), best.size() };
}
\end{lstlisting}