\section{714 --- Best Time to Buy and Sell Stock with Transaction Fee}
Your are given an array of integers \fcj{prices}, for which the $i$-th element is the price of a given stock on day $i$; and a non-negative integer \fcj{fee} representing a transaction fee.

You may complete as many transactions as you like, but you need to pay the transaction fee for each transaction. You may not buy more than 1 share of a stock at a time (ie. you must sell the stock share before you buy again.)

Return the maximum profit you can make.

\paragraph{Example 1:}

\begin{flushleft}
\textbf{Input}: \fcj{prices = [1, 3, 2, 8, 4, 9], fee = 2}

\textbf{Output}: 8

\textbf{Explanation}: The maximum profit can be achieved by:

\begin{itemize}
\item Buying at \fcj{prices[0] = 1}

\item Selling at \fcj{prices[3] = 8}

\item Buying at \fcj{prices[4] = 4}

\item Selling at \fcj{prices[5] = 9}
\end{itemize}

The total profit is $((8 - 1) - 2) + ((9 - 4) - 2) = 8$.

\end{flushleft}

\paragraph{Note:}

\begin{itemize}
\item \fcj{0 < prices.length <= 50000}.
\item \fcj{0 < prices[i] < 50000}.
\item \fcj{0 <= fee < 50000}.
\end{itemize}

\subsection{Dynamic Programming}
At the end of the $i$-th day, we maintain \fcj{cash}, the maximum profit we could have if we did not have a share of stock, and \fcj{hold}, the maximum profit we could have if we owned a share of stock.

Per the above definition, \fcj{cash = 0} and \fcj{hold = -prices[0]} at day 0.

We \textbf{only} apply transaction fee either for selling or buying, \textbf{not} for both. We choose apply to selling.

Thus, to transition from the $i$-th day to the $ (i+1) $-th day, we either \begin{itemize}
\item sell our stock: \fcj{cash = max(cash, hold + prices[i] - fee)} \item or buy a stock: \fcj{hold = max(hold, cash - prices[i])}
\end{itemize}. 
At the end, we want to return \fcj{cash}. 

\setcounter{lstlisting}{0}
\begin{lstlisting}[style=customc, caption={DP}]
int maxProfit( vector<int>& prices, int fee )
{
    //at the end of first day (0th day)
    //the maximum profit we could have if
    //we did not have a share: cash = 0
    int cash = 0;
    //the maximum profit we could have if
    //we have a share: hold = -prices[0]
    //we only apply fee to selling stock
    int hold = -prices[0];

    for( size_t i = 1; i < prices.size(); ++i )
    {
        //at the end of i-th day
        //if we have a a share, we need to buy
        //if we don't have a share at the end of (i-1)th day
        //thus, this will be (cash-prices[i])
        hold = ( max )( cash - prices[i], hold );
        //if we don't have a share, we need to
        //sell if we have a share at the end of (i-1)th day
        //this will be (hold+prices[i]-fee). We apply
        //fee to the selling
        cash = ( max )( hold + prices[i] - fee, cash );
    }

    return cash;
}
\end{lstlisting}

