\section{757 --- Set Intersection Size At Least Two}
An integer interval \fcj{[a, b]} (for integers $a < b$) is a set of all consecutive integers from $a$ to $b$, including $a$ and $b$.

Find the minimum size of a set $S$ such that for every integer interval $A$ in intervals, the intersection of $S$ with $A$ has size at least 2.

\paragraph{Example 1:}

\begin{flushleft}
\textbf{Input}: \fcj{intervals = [[1, 3], [1, 4], [2, 5], [3, 5]]}

\textbf{Output}: 3

\textbf{Explanation}:

Consider the set \fcj{S = {2, 3, 4}}.  For each interval, there are at least 2 elements from S in the interval.

Also, there isn't a smaller size set that fulfills the above condition.

Thus, we output the size of this set, which is 3.
\end{flushleft}

\paragraph{Example 2:}

\begin{flushleft}
\textbf{Input}: \fcj{intervals = [[1, 2], [2, 3], [2, 4], [4, 5]]}

\textbf{Output}: 5

\textbf{Explanation}:

An example of a minimum sized set is \fcj{[1, 2, 3, 4, 5]}.
\end{flushleft}

\paragraph{Note:}

\begin{itemize}
\item \fcj{intervals} will have length in range \fcj{[1, 3000]}.
\item \fcj{intervals[i]} will have length 2, representing some integer interval.
\item \fcj{intervals[i][j]} will be an integer in $ [0, 10^8] $.
\end{itemize}

\subsection{Greedy}
For greedy algorithm to work, we want to work with smaller intervals first. This is obvious: if interval $b$ completely contains interval $a$ and our set $S$ has two same numbers as in $a$, there must be two same numbers in $b$.

To be more efficient, we need the intervals to be sorted. Since we want to process shorter interval first, we will sort the intervals per the ending position by ascending order. If two intervals have same ending positions, the one with larger starting position is put in front. 

Then, there are three cases regarding the relationship of current interval and the set $S$ 
\begin{enumerate}
\item No intersection. We will fetch two numbers from current interval to add to $S$. To minimize the size of $S$, we shall choose the two maximum numbers of current interval. By this way, the probability that $S$ has intersection with intervals after is high.
\item Intersection with one number. This number must be the starting one of current interval. We need to get another number to add to $S$. By the similar analysis as before, this number to be added to $S$ must be the maximum number of current interval (or the ending one of current interval).
\item Intersection with two or more than two numbers. In this case, just skip current interval.
\end{enumerate}

In the implementation, we maintain an array $A$ to represent $S$. At start, we put two $-1$s into $A$. During traversing the sorted intervals, 

\begin{itemize}
\item If current one's starting number is less than the second to last in $A$, this is the case 3.
\item If current one's starting number is larger than the last number in $A$, this is the case 1.
\item Otherwise, this is the case 2.
\end{itemize}

Finally, we subtract 2 from the length of $A$ to remove the effect of two $-1$s.

\setcounter{lstlisting}{0}
\begin{lstlisting}[style=customc, caption={Greedy}]
int intersectionSizeTwo( vector<vector<int>>& intervals )
{
    //sort intervals with ending number by ascending order
    //and larger starting number when ending numbers are equal
    sort( begin( intervals ), end( intervals ), []( const vector<int>& itv1, const vector<int>& itv2 )
    {
        if( itv1[1] < itv2[1] )
        {
            return true;
        }
        if( itv1[1] == itv2[1] )
        {
            return itv1[0] > itv2[0];
        }

        return false;
    } );
    //the minimum size set
    vector<int> S{-1, -1};
    for( const auto& i : intervals )
    {
        if( i[0] <= S[S.size() - 2] )
        {
            //intersection with two or more numbers
            //skip
        }
        else if( i[0] > S.back() )
        {
            //no intersection
            //add maximum two numbers of current interval
            S.push_back( i[1] - 1 );
            S.push_back( i[1] );
        }
        else
        {
            //only one number in intersection
            S.push_back( i[1] );
        }
    }
    //substract 2 which is the effect of two -1s
    return static_cast<int>( S.size() ) - 2;
}
\end{lstlisting}

\subsection{Constant Space}
Notice in the last approach, we only need the maximum and the second to maximum numbers. Thus, we can reduce the memory consumption to $O(1)$.

\begin{lstlisting}[style=customc, caption={Constant Space}]
int intersectionSizeTwo( vector<vector<int>>& intervals )
{
    sort( begin( intervals ), end( intervals ), []( const vector<int>& itv1, const vector<int>& itv2 )
    {
        if( itv1[1] < itv2[1] )
        {
            return true;
        }
        if( itv1[1] == itv2[1] )
        {
            return itv1[0] > itv2[0];
        }

        return false;
    } );
    //the maximum number in the set S
    int max1 = -1;
    //the second to maximum number in the set S
    int max2 = -1;
    //the size of the set S
    int best_size = 0;
    for( const auto& i : intervals )
    {
        if( i[0] <= max2 )
        {
            //intersection with >=2 numbers
            continue;
        }
        else if( i[0] > max1 )
        {
            //no intersection
            //add maximum 2 numbers of interval i
            //to set S
            best_size += 2;
            max1 = i[1];
            max2 = max1 - 1;
        }
        else
        {
            //intersection with 1 number
            //which is i[0]
            //add i[1] to S
            ++best_size;
            max2 = max1;
            max1 = i[1];
        }
    }
    return best_size;
}
\end{lstlisting}