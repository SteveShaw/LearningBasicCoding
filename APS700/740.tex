\section{740 --- Delete and Earn}
Given an array of integers, \fcj{nums}, you can perform operations on the array.

In each operation, you pick any \fcj{nums[i]} and delete it to earn \fcj{nums[i]} points. After, you must delete every element equal to \fcj{nums[i] - 1} or \fcj{nums[i] + 1}.

You start with 0 points. Return the maximum number of points you can earn by applying such operations.

\paragraph{Example 1:}
\begin{flushleft}


\textbf{Input}: \fcj{nums = [3, 4, 2]}

\textbf{Output}: 6

\textbf{Explanation}:
 
Delete 4 to earn 4 points, consequently 3 is also deleted.

Then, delete 2 to earn 2 points. 6 total points are earned.
\end{flushleft}
 

\paragraph{Example 2:}
\begin{flushleft}


\textbf{Input}: \fcj{nums = [2, 2, 3, 3, 3, 4]}

\textbf{Output}: 9

\textbf{Explanation}:
 
Delete 3 to earn 3 points, deleting both 2's and the 4.

Then, delete 3 again to earn 3 points, and 3 again to earn 3 points.

9 total points are earned.
\end{flushleft}
 

\paragraph{Note:}

\begin{itemize}
\item The length of \fcj{nums} is at most 20000.
\item Each element \fcj{nums[i]} is an integer in the range \fcj{[1, 10000]}.
\end{itemize}

\subsection{Dynamic Programming}
To facilitate processing, we put all the elements into a tree map. Suppose before processing current element $x$, we have obtained the points by deleting and not deleting the largest element so far, $y$, which are $D$ (deleting) and $K$ (keep) respectively.

Thus, for each unique value $x$ of \fcj{nums} in increasing order

\begin{itemize}
\item If $x$ is adjacent to the previously largest value $y$, the points that we can obtain by deleting $x$ will be $K+x$(i.e., the points that not deleting $y$ plus the points of $x$). If we don't delete $x$, the points we can obtain will be $\max(D, K)$ (we don't get point if not deleting $x$)
\item Similarly, if $x$ is not adjacent to $y$, then the points we can obtain by deleting $x$ will be $\max(D, K)+x$, and the points by not deleting $x$ be $\max(D, K)$.

\end{itemize}
At the end, the best answer may or may not use the largest value in \fcj{nums}, so we return $\max(D, K)$.

\setcounter{lstlisting}{0}
\begin{lstlisting}[style=customc, caption={Dynamic Programming}]
int deleteAndEarn( vector<int>& nums )
{
    if( nums.empty() )
    {
        return 0;
    }
    map<int, int> dict;
    for( int n : nums )
    {
        dict[n] += 1;
    }
    auto it = dict.begin();
    //the points obtained by earning the point from current one
    int D = get<0>( *it ) * get<1>( *it );
    //the points obtained by not earning the point from current one
    int K = 0;
    //the last largest value
    int y = get<0>( *it );
    ++it;
    for( ; it != dict.end(); ++it )
    {
        int x = get<0>( *it );
        int last_D = D;
        int last_K = K;
        if( x == y + 1 )
        {
            //if we want to earn the points from x,
            //y must be deleted without earning the point
            D = x * get<1>( *it ) + last_K;
            //if we don't want to earn the points from x,
            //we can either choose obtaining from y or not from y
            K = ( max )( last_K, last_D );
        }
        else
        {
            //if we want to earn the points from x,
            //y don't need to be deleted
            D =  x * get<1>( *it ) + ( max )( last_K, last_D );
            //if we don't want to earn the points from x,
            //we can either choose obtaining from y or not from y
            K = ( max )( last_K, last_D );
        }
        //update largest element so far
        y = x;
    }
    return ( max )( D, K );
}
\end{lstlisting}