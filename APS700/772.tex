\section{772 --- Basic Calculator III}
Implement a basic calculator to evaluate a simple expression string.

The expression string may contain open \fcj{(} and closing parentheses \fcj{)}, the plus \fcj{+} or minus sign \fcj{-}, non-negative integers and empty spaces .

The integer division should truncate toward zero.

You may assume that the given expression is always valid. All intermediate results will be in the range of \fcj{[-2147483648, 2147483647]}.

Some examples:

\fcj{"1 + 1" = 2}

\fcj{" 6-4 / 2 " = 4}

\fcj{"2*(5+5*2)/3+(6/2+8)" = 21}

\fcj{"(2+6* 3+5- (3*14/7+2)*5)+3"=-12}

\subsection{Brute Force}
To process the parenthesis, we maintain a variable \fcj{balance}. Whenever we find a \fcj{"("}, start the process by increment \fcj{balance} for \fcj{"("}, and decrement for \fcj{")"}. When \fcj{balance} becomes zero, we recursively process the substring between \fcj{"("} and \fcj{")"}.

\setcounter{lstlisting}{0}
\begin{lstlisting}[style=customc, caption={Brute Force}]
int calculate( string s )
{
    return evaluate( s );
}
//helper function to evaluate the exprssion
int evaluate( string_view expr )
{
    auto L = expr.size();
    long long parse_num = 0;
    long long cur_res = 0;
    long long res = 0;
    //trick: initialize op as +
    //can simply the expression evaluation
    char op = '+';
    for( size_t i = 0; i < L; ++i )
    {
        auto c = expr[i];
        if( ( c >= '0' ) && ( c <= '9' ) )
        {
            //string to integer
            parse_num = parse_num * 10 + ( c - '0' );
        }
        else if( c == '(' )
        {
            //using balance to find
            //substring between a pair of
            //parenthesis
            auto j = i + 1;
            int balance = 1;
            for( ; j < L; ++j )
            {
                if( expr[j] == '(' )
                {
                    ++balance;
                }
                else if( expr[j] == ')' )
                {
                    --balance;
                }

                if( balance == 0 )
                {
                    break;
                }
            }
            parse_num = evaluate( expr.substr( i + 1, j - i - 1 ) );
            i = j;
        }
        //we cannot use "else if" here
        //since we have to use parse_num as early as possible
        //3. process the operator and number so far
        if( ( c == '+' ) || ( c == '-' ) || ( c == '*' ) || ( c == '/' ) || ( i == L - 1 ) )
        {
            //check last operator
            //and perform the computation
            switch( op )
            {
            case '+':
                cur_res += parse_num;
                break;
            case '-':
                cur_res -= parse_num;
                break;
            case '*':
                cur_res *= parse_num;
                break;
            case '/':
                cur_res /= parse_num;
                break;
            }
            if( ( c == '+' ) || ( c == '-' ) || ( i == L - 1 ) )
            {
                //if current operator is +/-
                //we accumulate the result of cur_res
                //also for i=L-1
                res += cur_res;
                //reset the result obtained so far
                cur_res = 0;
            }
            //update last operator
            op = c;
            //reset parsed number
            parse_num = 0;
        }
    }
    return res;
}
\end{lstlisting}