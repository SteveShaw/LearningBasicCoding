\section{722. Remove Comments}
Given a \fcc{C++} program, remove comments from it. The program source is an array where \fcc{source[i]} is the $i$--th line of the source code. This represents the result of splitting the original source code string by the newline character \fcc{\n}.

In \fcc{C++}, there are two types of comments, line comments, and block comments.

The string \fcc{//} denotes a line comment, which represents that it and rest of the characters to the right of it in the same line should be ignored.

The string \fcc{/*} denotes a block comment, which represents that all characters until the next occurrence of \fcc{*/} should be ignored. (Here, occurrences happen in reading order: line by line from left to right.) To be clear, the string \fcc{/*/} does not yet end the block comment, as the ending would be overlapping the beginning.

The first effective comment takes precedence over others: if the string \fcc{//} occurs in a block comment, it is ignored. Similarly, if the string \fcc{/*} occurs in a line or block comment, it is also ignored.

If a certain line of code is empty after removing comments, you must not output that line: each string in the answer list will be non-empty.

There will be no control characters, single quote, or double quote characters. For example, source = "string s = "/* Not a comment. */";" will not be a test case. (Also, nothing else such as defines or macros will interfere with the comments.)

It is guaranteed that every open block comment will eventually be closed, so \fcc{/*} outside of a line or block comment always starts a new comment.

Finally, implicit newline characters can be deleted by block comments. Please see the examples below for details.

After removing the comments from the source code, return the source code in the same format.

\paragraph{Example 1:}
\begin{flushleft}


\textbf{Input}: 

\begin{lstlisting}[style=customc]
source = [
"/*Test program */", 
"int main()", 
"{ ", 
"  // variable declaration ", 
"int a, b, c;",
 "/* This is a test", 
 "   multiline  ", 
 "   comment for ", 
 "   testing */", 
 "a = b + c;", 
 "}" ]
\end{lstlisting}

The line by line code is visualized as below:


\textbf{Output}: 

[\fcc{"int main()","{ ","  ","int a, b, c;","a = b + c;","}"]}

The line by line code is visualized as below:

\begin{lstlisting}[style=customc]
/*Test program */
int main()
{
    // variable declaration
    int a, b, c;
    /* This is a test
       multiline
       comment for
       testing */
    a = b + c;
}
\end{lstlisting}

\textbf{Explanation}:
 
The string \fcc{/*} denotes a block comment, including line 1 and lines 6-9. The string \fcc{//} denotes line 4 as comments.
\end{flushleft}

\paragraph{Example 2:}
\begin{flushleft}



\textbf{Input}: 

\fcc{source = ["a/*comment", "line", "more_comment*/b"]}

\textbf{Output}: \fcc{["ab"]}

\textbf{Explanation}: 

The original source string is \fcc{"a/*comment\nline\nmore_comment*/b"}, where we have bolded the newline characters.  After deletion, the implicit newline characters are deleted, leaving the string "ab", which when delimited by newline characters becomes \fcc{["ab"}].
\end{flushleft}

\paragraph{Note:}
\begin{itemize}
\item The length of source is in the range \fcc{[1, 100]}.
\item The length of source[i] is in the range \fcc{[0, 80]}.
\item Every open block comment is eventually closed.
\item There are no single-quote, double-quote, or control characters in the source code.
\end{itemize}

\subsection{Parse}
We need to parse the source line by line. Our state is that we either are in a block comment or not.

\begin{itemize}
\item If we start a block comment and we aren't in a block, then we will skip over the next two characters and change our state to be \textbf{in a block}.
\item If we end a block comment and we are in a block, then we will skip over the next two characters and change our state to be not in a block.
\item If we start a line comment and we aren't in a block, then we will ignore the rest of the line.
\item If we aren't in a block comment (and it wasn't the start of a comment), we will record the character we are at.
\item At the end of each line, if we aren't in a block, we will record the line.
\end{itemize}

\setcounter{lstlisting}{0}
\begin{lstlisting}[style=customc, caption={Parse}]
vector<string> removeComments( vector<string>& source )
{
    vector<string> ans;

    bool inBlockComment = false;

    vector<vector<string_view>> codeSegs;
    for( const auto& s : source )
    {
        string_view svCode( s.c_str(), s.size() );

        if( !inBlockComment )
        {
            //we not in a block
            //create a new array
            //to save possible code segs
            //for current line
            codeSegs.push_back( vector<string_view> {} );
        }

        while( !svCode.empty() )
        {
            if( !inBlockComment )
            {
                auto lineStart = svCode.find( "//" );
                auto blockStart = svCode.find( "/*" );

                if( lineStart < blockStart )
                {
                    //line comment take effect
                    //remaining parts are ignored
                    codeSegs.back().push_back( svCode.substr( 0, lineStart ) );
                    break;
                }

                if( blockStart < svCode.size() )
                {
                    //there is a block comment start
                    codeSegs.back().push_back( svCode.substr( 0, blockStart ) );
                    inBlockComment = true;
                    //shrink the string view
                    //ignore the part before the comment start
                    svCode.remove_prefix( blockStart + 2 );
                }
                else
                {
                    //the remaing parts of current line are all codes
                    codeSegs.back().push_back( svCode );
                    break;
                }
            }
            else
            {
                //we will find end of block comment
                auto blockEnd = svCode.find( "*/" );
                if( blockEnd < svCode.size() )
                {
                    //end block comment
                    inBlockComment = false;
                    //but we still need to stay in current line
                    //remove the parts before the comment end
                    svCode.remove_prefix( blockEnd + 2 );
                }
                else
                {
                    //whole line can be ignored
                    break;
                }
            }//end else
        } //end while(!svCode.empty)

        if( !inBlockComment )
        {
            string s;

            for( auto seg : codeSegs.back() )
            {
                s += seg;
            }

            codeSegs.pop_back();

            if( !s.empty() )
            {
                ans.push_back( move( s ) );
            }

        }

    }//end for(s:source)

    return ans;
}
\end{lstlisting}

\subsection{Related Problems}
\begin{itemize}
\item \textbf{385. Mini Parser}
\item \textbf{439. Ternary Expression Parser}
\end{itemize}