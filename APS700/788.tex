\section{788 --- Rotated Digits}
$X$ is a good number if after rotating each digit individually by 180 degrees, we get a valid number that is different from $X$.  Each digit must be rotated - we cannot choose to leave it alone.

A number is valid if each digit remains a digit after rotation. 0, 1, and 8 rotate to themselves; 2 and 5 rotate to each other; 6 and 9 rotate to each other, and the rest of the numbers do not rotate to any other number and become invalid.

Now given a positive number $N$, how many numbers $X$ from $1$ to $N$ are good?

\paragraph{Example:}

\begin{flushleft}
\textbf{Input}: 10

\textbf{Output}: 4

\textbf{Explanation}: 

There are four good numbers in the range \fcj{[1, 10]} : 2, 5, 6, 9.

Note that 1 and 10 are not good numbers, since they remain unchanged after rotating.
\end{flushleft}

\paragraph{Note:}

\begin{itemize}
\item $N$  will be in range \fcj{[1, 10000]}.
\end{itemize}

\subsection{Dynamic Programming}
Let $F$ is the DP array with size $N+1$. Each element in $F$ has three states:

\begin{itemize}
\item 0 --- invalid after rotation.
\item 1 --- same as before after rotation
\item 2 --- different number after rotation.
\end{itemize}

For numbers less than 10, we will have 

\begin{itemize}
\item $F[3]=F[4]=F[7]=0$
\item $F[0]=F[1]=F[8]=1$
\item $F[2]=F[5]=F[6]=F[9]=2$
\end{itemize}

Then for each number $x\geq 10$, $F[x]$ can be built from $F[x/10]$ and $F[\bmod(x,10)]$.

When $F[x/10]=1$ and $F[\bmod(x, 10)]=1$, we know that $F[x]=1$ (same as before).

When  one of them is 2, and another one is equal or larger than 1, we know $F[x]=2$ (valid after rotation). 

In other cases, $F[x]=0$.

\setcounter{lstlisting}{0}
\begin{lstlisting}[style=customc, caption={DP}]
int rotatedDigits( int N )
{
    vector<unsigned char> F( N + 1, 0 );
    int ans = 0;
    for( int i = 0; i <= ( min )( 9, N ); ++i )
    {
        if( ( i == 1 ) || ( i == 0 ) || ( i == 8 ) )
        {
            F[i] = 1;
        }
        else if( ( i == 2 ) || ( i == 5 ) || ( i == 6 ) || ( i == 9 ) )
        {
            F[i] = 2;
            ++ans;
        }
    }
    if( N < 10 )
    {
        return ans;
    }
    for( int i = 10; i <= N; ++i )
    {
        int q = i / 10;
        int r = i - q * 10;

        if( ( F[q] == 1 ) && ( F[r] == 1 ) )
        {
            //remain the same
            F[i] = 1;
        }
        else if( ( F[q] >= 1 ) && ( F[r] >= 1 ) )
        {
            //valid after rotate
            F[i] = 2;
            ++ans;
        }
    }
    return ans;
}
\end{lstlisting}