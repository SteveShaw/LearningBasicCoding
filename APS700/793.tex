\section{793 --- Preimage Size of Factorial Zeroes Function}

\textbf{Hard}

Let $ f(x) $ be the number of zeroes at the end of $ x! $. (Recall that $ x! = 1 \times 2 \times 3 \times \ldots \times x $, and by convention, $0! = 1$.)

For example, $ f(3) = 0 $ because $ 3! = 6 $ has no zeroes at the end, while $ f(11) = 2 $ because $ 11! = 39916800 $ has 2 zeroes at the end. Given $ K $, find how many non-negative integers $ x $ have the property that $ f(x) = K $.

\paragraph{Example 1:}

\begin{flushleft}
\textbf{Input}: K = 0

\textbf{Output}: 5

\textbf{Explanation}: 0!, 1!, 2!, 3!, and 4! end with K = 0 zeroes.
\end{flushleft}

\paragraph{Example 2:}

\begin{flushleft}
\textbf{Input}: $ K = 5 $

\textbf{Output}: 0

\textbf{Explanation}: There is no $ x $ such that $ x! $ ends in $ K = 5 $ zeroes.
\end{flushleft}

\paragraph{Note:}

\begin{itemize}
\item $ K $ will be an integer in the range $ [0, 10^9] $.
\end{itemize}

\subsection{Binary Search}

Denote $\zeta(x)$ be the number of zeroes at the end of $x!$. If the prime factorization of $ x! $ is $ (2^a * 5^b * \cdots )$, the number of times that 10 divides this is $\min(a, b)$ which is $b$ (number of 5 multiples are less than number of 2 multiples).

Thus, $\zeta(x)$ is the number of times 5 divides $ x! $, which is equal to 

$ \lfloor \frac{x}{5^1} \rfloor + \lfloor \frac{x}{5^2} \rfloor + \lfloor \frac{x}{5^3} \rfloor + \lfloor \frac{x}{5^4} \rfloor + \cdots $

Indeed, $\zeta$ is a monotone increasing function, so we can for both the \textbf{largest} and \textbf{smallest} value $ x $ such that $ \zeta(x) = K $. However, since $ \zeta(5a-1) < \zeta(5a) = \zeta(5a+1) = \zeta(5a+2) = \zeta(5a+3) = \zeta(5a+4) < \zeta(5a+5) $, if it is possible for $ \zeta(x) $ to equal $ K $ for some $ x $, return 5, else the answer is 0.

\setcounter{lstlisting}{0}
\begin{lstlisting}[style=customc, caption={Binary Search}]
int preimageSizeFZF( int K )
{
    //notice: we need long long data type
    //to avoid overflow
    long long lo( K );
    long long hi( lo * 10 + 1 );
    //standard binary search
    while( lo <= hi )
    {
        auto mid( lo + ( hi - lo ) / 2 );
        auto zeros( trail_zeros( mid ) );
        if( zeros == K )
        {
            //only five number can have K
            //zeros
            return 5;
        }
        if( zeros < K )
        {
            lo = mid + 1;
        }
        else
        {
            hi = mid - 1;
        }
    }
    return 0;
}
//helper function to get number of trailing zeros
//same as problem 173
int trail_zeros( long long n )
{
    int zeros = 0;
    while( n )
    {
        zeros += n / 5;
        n = n / 5;
    }
    return zeros;
}
\end{lstlisting}