\section{759 --- Employee Free Time}
We are given a list \fcj{schedule} of employees, which represents the working time for each employee.

Each employee has a list of non-overlapping \fcj{Intervals}, and these intervals are in sorted order.

Return the list of finite intervals representing \textbf{common}, \textbf{positive-length free time} for \textit{all} employees, also in sorted order.

\paragraph{Example 1:}

\begin{flushleft}
\textbf{Input}: \fcj{schedule = [[[1,2],[5,6]],[[1,3]],[[4,10]]]}

\textbf{Output}: \fcj{[[3,4]]}

\textbf{Explanation}:

There are a total of three employees, and all common

free time intervals would be $[-\infty, 1]$, $[3, 4]$, $[10, \infty]$.

We discard any intervals that contain inf as they aren't finite.
\end{flushleft}

 

\paragraph{Example 2:}

\begin{flushleft}
\textbf{Input}: \fcj{schedule = [[[1,3],[6,7]],[[2,4]],[[2,5],[9,12]]]}

\textbf{Output}: \fcj{[[5,6],[7,9]]}
\end{flushleft}

 
Even though we are representing \fcj{Intervals} in the form \fcj{[x, y]}, the objects inside are \fcj{Intervals}, not lists or arrays. 

For example, \fcj{schedule[0][0].start = 1}, \fcj{schedule[0][0].end = 2}, and \fcj{schedule[0][0][0]} is not defined.


Also, we wouldn't include intervals like \fcj{[5, 5]} in our answer, as they have zero length.

\paragraph{Note:}

\begin{itemize}
\item \fcj{schedule} and \fcj{schedule[i]} are lists with lengths in range \fcj{[1, 50]}.
\item \fcj{0 <= schedule[i].start < schedule[i].end <=} $10^8$.
\end{itemize}

\subsection{Line Sweep}
Sort the input intervals per the starting time, and find overlapped intervals

\setcounter{lstlisting}{0}
\begin{lstlisting}[style=customc, caption={Line Sweep}]

vector<Interval*> employeeFreeTime( vector<vector<Interval*>> schedule )
{
    //group the intervals together
    vector<Interval*> schs;
    for( const auto& itvs : schedule )
    {
        schs.insert( schs.end(), begin( itvs ), end( itvs ) );
    }
    //sort the intervals per the start time first
    //end the end time when two have equal start times
    sort( begin( schs ), end( schs ), []( Interval * p1, Interval * p2 )
    {

        if( p1->start < p2->start )
        {
            return true;
        }
        if( p1->start == p2->start )
        {
            return p1->end < p2->end;
        }
        return false;
    } );
    vector<Interval*> ans;
    //find overlapped intervals
    auto overlap = schs[0];
    for( size_t i = 1; i < schs.size(); ++i )
    {
        if( overlap->end >= schs[i]->start )
        {
            //overlap with current merged interval
            overlap->end = ( max )( overlap->end, schs[i]->end );
        }
        else
        {
            //we found a non-overlap interval
            //add to the output
            ans.emplace_back( new Interval( overlap->end, schs[i]->start ) );
            overlap = schs[i];
        }
    }
    return ans;
}
\end{lstlisting}

\subsection{Boundary Count}
Similar to problem \textbf{731 --- My Calendar II}, When booking a new interval \fcj{[start, end]}, use a tree map \fcj{delta} to count as  \fcj{delta[start]++} and \fcj{delta[end]--}.

When processing the values of \fcj{delta} in sorted order of their keys, the running sum active is the number of overlapped intervals at that time. If the sum is zero, we are at the end of a group of overlapped intervals, or the start of free time range, say $x$. 

We add \fcj{[x,-1]} into the output list where $-1$ means we have not updated the end of free time range. This end will be updated by the next interval's start time.

Finally, we may add an interval which has $-1$ as the end time. We will remove it at the end.

\begin{lstlisting}[style=customc, caption={Boundry Count}]
vector<Interval*> employeeFreeTime( vector<vector<Interval*>> schedule )
{
    //used for booking intervals
    map<int, int> events;
    for( const auto& itvs : schedule )
    {
        for( auto p : itvs )
        {
            //start: increment count
            events[p->start] += 1;
            //end: decrement count
            events[p->end] -= 1;
        }
    }
    //go through the boundry counts
    int count = 0;
    vector<Interval*> ans;
    for( const auto& evt : events )
    {
        count += get<1>( evt );
        if( count == 0 )
        {
            //we are at the end of a group of overlapped intervals
            //or the start of a free time range
            ans.push_back( new Interval( get<0>( evt ), -1 ) );
        }
        else
        {
            if( !ans.empty() && ( ans.back()->end < 0 ) )
            {
                //update the last added free time range end
                //which is current interval's start time
                ans.back()->end = get<0>( evt );
            }
        }
    }
    //finally, we will remove last one we added
    //because this one has -1 as the end time.
    if( !ans.empty() )
    {
        ans.pop_back();
    }
    return ans;
}
\end{lstlisting}