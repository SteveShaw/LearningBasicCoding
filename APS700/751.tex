\section{751 --- IP to CIDR}
Given a start IP address \fcj{ip} and a number of ips we need to cover $n$, return a representation of the range as a list (of smallest possible length) of CIDR blocks.

A CIDR block is a string consisting of an IP, followed by a slash, and then the prefix length. For example: "123.45.67.89/20". That prefix length "20" represents the number of common prefix bits in the specified range.

\paragraph{Example 1:}
\begin{flushleft}

\textbf{Input}: \fcj{ip = "255.0.0.7"}, \fcj{n = 10}

\textbf{Output}: \fcj{["255.0.0.7/32","255.0.0.8/29","255.0.0.16/32"]}

\textbf{Explanation}:

The initial ip address, when converted to binary, looks like this (spaces added for clarity):

\fcj{255.0.0.7 -> 11111111 00000000 00000000 00000111}

The address \fcj{"255.0.0.7/32"} specifies all addresses with a common prefix of 32 bits to the given address, i.e. just this one address.

The address \fcj{"255.0.0.8/29"} specifies all addresses with a common prefix of 29 bits to the given address:

\fcj{255.0.0.8 -> 11111111 00000000 00000000 00001000}

Addresses with common prefix of 29 bits are:

\fcj{11111111 00000000 00000000 00001000}

\fcj{11111111 00000000 00000000 00001001}

\fcj{11111111 00000000 00000000 00001010}

\fcj{11111111 00000000 00000000 00001011}

\fcj{11111111 00000000 00000000 00001100}

\fcj{11111111 00000000 00000000 00001101}

\fcj{11111111 00000000 00000000 00001110}

\fcj{11111111 00000000 00000000 00001111}


The address \fcj{"255.0.0.16/32"} specifies all addresses with a common prefix of 32 bits to the given address, i.e. just 

\fcj{11111111 00000000 00000000 00010000.}

In total, the answer specifies the range of 10 ips starting with the address 255.0.0.7 .

There were other representations, such as:
["255.0.0.7/32","255.0.0.8/30", "255.0.0.12/30", "255.0.0.16/32"],
but our answer was the shortest possible.

Also note that a representation beginning with say, "255.0.0.7/30" would be incorrect,
because it includes addresses like 255.0.0.4 = 11111111 00000000 00000000 00000100 
that are outside the specified range.

\end{flushleft}
Note:

    ip will be a valid IPv4 address.
    Every implied address ip + x (for x < n) will be a valid IPv4 address.
    n will be an integer in the range [1, 1000].
