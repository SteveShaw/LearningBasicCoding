\section{751 --- IP to CIDR}
Given a start IP address \fcj{ip} and a number of ips we need to cover $n$, return a representation of the range as a list (of smallest possible length) of CIDR blocks.

A CIDR block is a string consisting of an IP, followed by a slash, and then the prefix length. For example: \fcj{"123.45.67.89/20"}. That prefix length \fcj{"20"} represents the number of common prefix bits in the specified range.

\paragraph{Example 1:}
\begin{flushleft}

\textbf{Input}: \fcj{ip = "255.0.0.7"}, \fcj{n = 10}

\textbf{Output}: \fcj{["255.0.0.7/32","255.0.0.8/29","255.0.0.16/32"]}

\textbf{Explanation}:

The initial ip address, when converted to binary, looks like this (spaces added for clarity):

\fcj{255.0.0.7 -> 11111111 00000000 00000000 00000111}

The address \fcj{"255.0.0.7/32"} specifies all addresses with a common prefix of 32 bits to the given address, i.e. just this one address.

The address \fcj{"255.0.0.8/29"} specifies all addresses with a common prefix of 29 bits to the given address:

\fcj{255.0.0.8 -> 11111111 00000000 00000000 00001000}

Addresses with common prefix of 29 bits are:

\fcj{11111111 00000000 00000000 00001000}

\fcj{11111111 00000000 00000000 00001001}

\fcj{11111111 00000000 00000000 00001010}

\fcj{11111111 00000000 00000000 00001011}

\fcj{11111111 00000000 00000000 00001100}

\fcj{11111111 00000000 00000000 00001101}

\fcj{11111111 00000000 00000000 00001110}

\fcj{11111111 00000000 00000000 00001111}


The address \fcj{"255.0.0.16/32"} specifies all addresses with a common prefix of 32 bits to the given address, i.e. just 

\fcj{11111111 00000000 00000000 00010000.}

In total, the answer specifies the range of 10 ips starting with the address 255.0.0.7 .

There were other representations, such as:

\fcj{["255.0.0.7/32","255.0.0.8/30", "255.0.0.12/30", "255.0.0.16/32"]},

but our answer was the \textbf{shortest} possible.

Also note that a representation beginning with say, \fcj{"255.0.0.7/30"} would be incorrect,
because it includes addresses like 255.0.0.4 = 11111111 00000000 00000000 00000100 
that are outside the specified range.

\end{flushleft}



\paragraph{Note:}

\begin{enumerate}
\item \fcj{ip} will be a valid IPv4 address.
\item Every implied address \fcj{ip + x} (for $x < n$ ) will be a valid IPv4 address.
\item $n$ will be an integer in the range \fcj{[1, 1000]}.
\end{enumerate}

\subsection{Greedy}

If we know the lowest bit 1, we can know how many addresses a CIDR can cover. The way to find the lowest bit 1 is by using \fcc{x & -x}.

After knowing number of covered ip addresses, we have to take $n$ into consideration. 

\setcounter{lstlisting}{0}
\begin{lstlisting}[style=customc, caption={Greedy}]
vector<string> ipToCIDR( string ip, int n )
{
    //help routint to convert string to integer
    auto to_int = []( string_view s )
    {
        int n = 0;
        for( auto c : s )
        {
            n = n * 10 + ( c - '0' );
        }
        return n;
    };
    //convert ip to integer
    string_view vip( ip );
    size_t start = 0;
    auto p = vip.find( '.', start );
    long long x = 0;
    while( p != string::npos )
    {
        auto sub = vip.substr( start, p - start );
        x = 256 * x + to_int( sub );
        start = p + 1;
        p = vip.find( '.', start );
    }
    x = 256 * x + to_int( vip.substr( start ) );
    //find ranges of ips
    vector<string> ans;
    auto lln = static_cast< long long >( n );
    while( lln > 0 )
    {
        //find how many ips can cover by x
        auto covers = ( x & ( -x ) );
        //make sure covers are less than given n
        while( covers > lln )
        {
            covers >>= 1;
        }
        //change to CIDR
        ans.push_back( toCIDR( x, covers ) );
        x += covers;
        lln -= covers;
    }

    return ans;
}
//get maximum exponent pow so that 2^pow <= covers
long long max_pow2( long long covers )
{
    if( covers == 0 )
    {
        return 1;
    }

    long long pow = 0;

    while( covers > 0 )
    {
        covers >>= 1;
        ++pow;
    }
    return pow - 1;
}
//output to CIDR
string toCIDR( long long x, long long covers )
{
    //change x to ip address
    string ip;
    for( int i = 1; i <= 4; ++i )
    {
        //output each segment
        int y = ( ( x >> ( 32 - i * 8 ) ) & 255 );
        ip += to_string( y );
        ip.push_back( '.' );
    }
    ip.back() = '/';
    //add covers
    auto pow = max_pow2( covers );
    //32-pow is the common prefix length
    ip += to_string( 32LL - pow );
    return ip;
}
\end{lstlisting}
