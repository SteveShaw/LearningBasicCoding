\section{774 --- Minimize Max Distance to Gas Station}
On a horizontal number line, we have gas stations at positions \fcj{stations[0]}, \fcj{stations[1]}, $\ldots$, \fcj{stations[N-1]}, where \fcj{N = stations.length}.

Now, we add $K$ more gas stations so that \textbf{D}, the maximum distance between adjacent gas stations, is minimized.

Return the smallest possible value of \textbf{D}.

\paragraph{Example:}

\begin{flushleft}
\textbf{Input}: 

\fcj{stations = [1, 2, 3, 4, 5, 6, 7, 8, 9, 10]}

$K = 9$

\textbf{Output}: \fcj{0.500000}
\end{flushleft}

\paragraph{Note:}

\begin{itemize}
\item \fcj{stations.length} will be an integer in range \fcj{[10, 2000]}.
\item \fcj{stations[i]} will be an integer in range $[0, 10^8]$.
\item $K$ will be an integer in range $[1, 10^6]$.
\item Answers within $10^{-6}$ of the true value will be accepted as correct.
\end{itemize}

\subsection{Dynamic Programming [MLE]}
Suppose \fcj{F[n][y]} be the answer by adding \fcj{y} more gas stations to the first \fcj{n} intervals of stations, where the $i$--th interval is \fcj{dist[i] = stations[i+1] - stations[i]}. 

To get \fcj{F[n+1][y]}, we can test by putting $x$ gas stations in the $(n+1)$--th interval to get a best distance as \fcj{dist[n+1]/(x+1)}. The the best distance for the rest of the intervals is \fcj{dist[n][y-x]}. Then, \fcj{F[n+1][t]} is the minimum when we try every possible $x$ from 0 to $y$.

Finally, The last element of \fcj{F} is the answer.

\setcounter{lstlisting}{0}
\begin{lstlisting}[style=customc, caption={DP[MLE]}]
double minmaxGasDist( vector<int>& stations, int K )
{
    auto L( stations.size() );
    //get distance between two consecutive stations
    vector<double> dists( L - 1 );
    for( size_t i = 1; i < stations.size(); ++i )
    {
        dists[i - 1] = static_cast< double >( stations[i] ) - static_cast< double >( stations[i - 1] );
    }
    //start DP
    vector<vector<double>> F( L - 1, vector<double>( K + 1 ) );
    //fill the corner case
    //add y more stations at the start
    //it only changes the first interval
    for( int y = 0; y <= K; ++y )
    {
        F[0][y] = dists[0] / ( y + 1 );
    }
    //fill the DP
    for( auto n = 1U; n < L - 1; ++n )
    {
        for( auto y = 0; y <= K; ++y )
        {
            double best = 999'999'999;
            //test each possible station can be added
            //to nth-interval
            for( auto x = 0; x <= y; ++x )
            {
                //get the local maximum
                //when adding x more stations to n-th interval
                //the maximum for n-th interval is dists[n] / (x+1)
                //the maximum for previous n-1 intervals are F[n-1][y-x]
                auto local_best = ( max )( dists[n] / static_cast<double>( x + 1 ), F[n - 1][y - x] );
                //get current gloal minimum
                best = ( min )( best,  local_best );
            }
            //set to F[n][y]
            F[n][y] = best;
        }
    }
    return F.back().back();
}
\end{lstlisting}

\subsection{Greedy}
We can add a gas station to the current largest interval until we add total of $K$. We maintain an array \fcj{counts} to keep track of number of parts for an interval would have when adding one more gas station. Initially, each interval has only one part.

\subsection{Binary Search}
We can search the target distance in $[0, 10^8]$. For each target distance, we calculate how many more gas stations we need to. If the number of required stations is larger than $K$, we will enlarge the target distance (i.e., search in the upper half), otherwise, search in the lower half.

\begin{lstlisting}[style=customc, caption={Binary Search}]
double minmaxGasDist( vector<int>& stations, int K )
{
    auto l( 0.0 );
    auto r( 100'000'00. );
    auto t( 1. / 1000'000. );
    while( r - l > t )
    {
        auto dist = ( r + l ) / 2.;
        auto num_more = needed_stations( stations, dist, K );
        if( num_more <= K )
        {
            //we want to find a bigger one
            r = dist;
        }
        else
        {
            //we have to enlarge the distance
            l = dist;
        }
    }
    return l;
}
//helper function to get how many more stations to be added
int needed_stations( vector<int>& stations, double D, int K )
{
    auto needed( 0 );

    for( auto i = 1ull; i < stations.size(); ++i )
    {
        //we want to get ( stations[i] - stations[i - 1] ) / D + 1 as the
        //number stations between this two positions
        //and then minus 1 is the number of needed stations
        needed += static_cast< int >( ( stations[i] - stations[i - 1] ) / D );
    }
    return needed;
}
\end{lstlisting}