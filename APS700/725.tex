\section{725 --- Split Linked List in Parts}
Given a (singly) linked list with head node \fcj{root}, write a function to split the linked list into $k$ consecutive linked list \fcj{"parts"}.

The length of each part should be as equal as possible: no two parts should have a size differing by more than 1. This may lead to some parts being null.

The parts should be in order of occurrence in the input list, and parts occurring earlier should always have a size greater than or equal parts occurring later.

Return a List of \fcj{ListNode} to represent the linked list parts that are formed.

\paragraph{Example 1} 
\begin{flushleft}
\textbf{Input}:

\begin{figure}[H]
\begin{tikzpicture}
[start chain, every node/.style={draw, circle,minimum size=6mm, fill=gray!20!},
  node distance=8mm, every join/.style={>=stealth,->},
 thick]
\node [draw,on chain,join] {1};
\node [draw,on chain,join] {2};
\node [draw,on chain,join] {3};
\node [draw,on chain,join] {4};
\end{tikzpicture}

$k = 5$

\textbf{Output}:

5 equal parts: \fcj{[[1], [2], [3], [4], null]}

\end{figure}
\end{flushleft}

\paragraph{Example 2:}
\begin{flushleft}


\textbf{Input}:

\fcj{root = [1, 2, 3]}, $k = 5$

\textbf{Output}: \fcj{[[1],[2],[3],[],[]]}

\textbf{Explanation}:

The input and each element of the output are \fcj{ListNodes}, not arrays.

For example, the input root has 

\fcj{root.val = 1}, 

\fcj{root.next.val = 2}, 

\fcj{root.next.next.val = 3}, and 

\fcj{root.next.next.next = null}.

The first element \fcj{output[0]} has \fcj{output[0].val = 1}, \fcj{output[0].next = null}.

The last element \fcj{output[4] is null}, but it's string representation as a ListNode is [].

\end{flushleft}

\paragraph{Example 2:}

\begin{flushleft}

\textbf{Input}: 


\fcj{root = [1, 2, 3, 4, 5, 6, 7, 8, 9, 10]}, $k = 3$

\textbf{Output}: \fcj{[[1, 2, 3, 4], [5, 6, 7], [8, 9, 10]]}

\textbf{Explanation}:

The input has been split into consecutive parts with size difference at most 1, and earlier parts are a larger size than the later parts.

\end{flushleft}

\paragraph{Note:}

\begin{itemize}
\item The length of \fcj{root} will be in the range \fcj{[0, 1000]}.
\item Each value of a node in the input will be an integer in the range \fcj{[0, 999]}.
\item $k$ will be an integer in the range $[1, 50]$.
\end{itemize}

\subsection{Split Input List}
If there are $N$ nodes in the linked list root, then there are $N / k$ items in each part, plus the first $\bmod(N, k)$ parts have an extra item. We can count $N$ with a simple loop.

Now for each part, we will try to set the first $r$ parts has length $N/k + 1$ and remaining with $N/k$. We can directly split the input list.

\setcounter{lstlisting}{0}
\begin{lstlisting}[style=customc, caption={Split Input List}]
vector<ListNode*> splitListToParts( ListNode* root, int k )
{
    auto node = root;
    //get total nodes in the list
    int len = 0;
    while( node )
    {
        ++len;
        node = node->next;
    }

    //q will be number of nodes each part at least have
    int q = len / k;
    //first r parts will have (q+1) nodes
    int r = len - k * q;

    auto start = root;

    vector<ListNode*> ans;
    ans.reserve( k );

    while( k )
    {
        if( start == nullptr )
        {
            break;
        }
        auto partEnd = start;

        int count = q;
        //set first (r) parts
        //have (q+1) nodes
        if( r > 0 )
        {
            ++count;
            --r;
        }
        //the last node of this part
        auto last = partEnd;
        //split the part
        //from start to (start+count-1)
        for( int x = 0; x < count; ++x )
        {
            if( !partEnd )
            {
                //make sure we are not
                //at the end of the list
                break;
            }
            last = partEnd;
            partEnd = partEnd->next;
        }
        //set the part's end node
        //to null pointer
        last->next = nullptr;
        ans.push_back( start );
        //the start of next part
        start = partEnd;
        --k;
    }

    //we have consumed all nodes
    //now add null list
    while( k )
    {
        ans.push_back( nullptr );
        --k;
    }
    return ans;
}
\end{lstlisting}
