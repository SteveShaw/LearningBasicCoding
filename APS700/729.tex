\section{729 --- My Calendar I}
Implement a \fcj{MyCalendar} class to store your events. A new event can be added if adding the event will not cause a double booking.

Your class will have the method, \fcj{book(int start, int end)}. Formally, this represents a booking on the half open interval \fcj{[start, end)}, the range of real numbers $x$ such that \fcj{start <= x < end}.

A double booking happens when two events have some non-empty intersection (ie., there is some time that is common to both events.)

For each call to the method \fcj{MyCalendar.book}, return \fcj{true} if the event can be added to the calendar successfully without causing a double booking. Otherwise, return \fcj{false} and do not add the event to the calendar.

Your class will be called like this: 

\fcj{MyCalendar cal = new MyCalendar(); MyCalendar.book(start, end)}

\paragraph{Example 1:}
\begin{flushleft}


\fcj{MyCalendar();}

\fcj{MyCalendar.book(10, 20); // returns true}

\fcj{MyCalendar.book(15, 25); // returns false}

\fcj{MyCalendar.book(20, 30); // returns true}


\textbf{Explanation}:
 
The first event can be booked.  The second can't because time 15 is already booked by another event.

The third event can be booked, as the first event takes every time less than 20, but not including 20.

\end{flushleft}

\paragraph{Note:}

\begin{itemize}
\item The number of calls to \fcj{MyCalendar.book} per test case will be at most 1000.
\item In calls to \fcj{MyCalendar.book(start, end)}, start and end are integers in the range $[0, 10^9]$.
\end{itemize}

\subsection{Binary Search}
If we maintained all events in sorted order, we could check whether an event could be booked in $O(\log N)$ time (where $N$ is the number of events already booked) by binary searching for where the event should be placed. We would also have to insert the event in our sorted structure.

\setcounter{lstlisting}{0}
\begin{lstlisting}[style=customc, caption={Binary Search}]
class MyCalendar
{
public:
    MyCalendar() {}

    bool book( int start, int end )
    {
        if( x_events.empty() )
        {
            x_events.emplace( start, end );
            return true;
        }
        //find the first event
        //lies right of [start,end]
        auto x_low = x_events.lower_bound( start );

        if( x_low != x_events.end() )
        {
            if( x_low->first < end )
            {
                //overlap
                return false;
            }
        }

        if( x_low != x_events.begin() )
        {
            //check the last event
            //lies left of [start,end]
            --x_low;

            if( start < x_low->second )
            {
                //overlap
                return false;
            }
        }

        x_events.emplace( start, end );
        return true;
    }

    map<int, int> x_events;
};
\end{lstlisting}