\section{795 --- Number of Subarrays with Bounded Maximum}

\textbf{Medium}

We are given an array $A$ of positive integers, and two positive integers $L$ and $R$ ($L \leq R$).

Return the number of (contiguous, non-empty) subarrays such that the value of the maximum array element in that subarray is at least $L$ and at most $R$.

\paragraph{Example :}
\begin{flushleft}


\textbf{Input}: 

\fcj{A = [2, 1, 4, 3]}

$ L = 2 $

$ R = 3 $

\textbf{Output}: 3

\textbf{Explanation}: 

There are three subarrays that meet the requirements: \fcj{[2]}, \fcj{[2, 1]}, \fcj{[3]}.
\end{flushleft}

\paragraph{Note:}

\begin{enumerate}
\item $L$, $R$  and \fcj{A[i]} will be an integer in the range $[0, 10^9]$.
\item The length of $A$ will be in the range of \fcj{[1, 50000]}.
\end{enumerate}

\subsection{Counting}
As we only care about whether each element is less than, between, or greater than the interval $ [L, R] $, we can mark each element is either $a$ if it is less than $L$; $b$ if it is between $L$ and $R$; or $c$ if it is greater than $R$.

Then, we want the number of subarrays with no $c$ and at least one $b$. $c$ split the array into groups of arrays with only $a$ and $b$. For example, if the given array is \fcj{[a, a, b, c, c, b, a, c, a]}, then the answer is just the answer for \fcj{[a, a, b]} plus the answer for \fcj{[b, a]} plus the answer for \fcj{[a]}.

For each such group (arrays of only $a$ or $b$), to count the number of subarrays that have at least one $b$, let's count all the subarrays in the group, minus the subarrays that only have $a$.

For example, 

\fcj{[a, a, b]} has 6 subarrays, 3 of which are $a$-only, which adds 3 to the answer; 

\fcj{[b, a]} has 3 subarrays, 1 of which is $a$-only, which adds 2 to the answer; 

and \fcj{[0]} has 1 subarray, 1 of which is $a$-only, which adds 0 to the answer. 

The final answer to the previous original example of \fcj{A = [0, 0, 1, 2, 2, 1, 0, 2, 0]} is $ 3 + 2 + 0 = 5 $.

\setcounter{lstlisting}{0}
\begin{lstlisting}[style=customc, caption={Count}]
int numSubarrayBoundedMax( vector<int>& A, int L, int R )
{
    //lambda function to find
    //number of subarray with value no larger than limit
    auto count = [ &A ]( int limit )
    {
        int res = 0;
        int consec = 0;
        for( int n : A )
        {
            if( n <= limit )
            {
                ++consec;
            }
            else
            {
                consec = 0;
            }

            res += consec;
        }

        return res;
    };
    //remove the number of subarrays with value less than L
    return count( R ) - count( L - 1 );
}
\end{lstlisting}