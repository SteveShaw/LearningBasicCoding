\section{794 --- Valid Tic-Tac-Toe State}

\textbf{Medium}

A Tic-Tac-Toe board is given as a string array \fcj{board}. Return \fcj{true} if and only if it is possible to reach this board position during the course of a valid tic-tac-toe game.

The board is a $ 3 \times 3 $ array, and consists of characters \fcj{" "}, \fcj{"X"}, and \fcj{"O"}.  The \fcj{" "} character represents an empty square.

Here are the rules of Tic-Tac-Toe:

\begin{itemize}
\item Players take turns placing characters into empty squares (\fcj{" "}).

\item The first player always places \fcj{"X"} characters, while the second player always places \fcj{"O"} characters.

\item \fcj{"X"} and \fcj{"O"} characters are always placed into empty squares, never filled ones.

\item The game ends when there are 3 of the same (non-empty) character filling any row, column, or diagonal.

\item The game also ends if all squares are non-empty.

\item No more moves can be played if the game is over.
\end{itemize}

\paragraph{Example 1:}

\begin{flushleft}
\textbf{Input}: \fcj{board = ["O  ", "   ", "   "]}

\textbf{Output}: \fcj{false}

\textbf{Explanation}: The first player always plays \fcj{"X"}.
\end{flushleft}

\paragraph{Example 2:}

\begin{flushleft}
\textbf{Input}: \fcj{board = ["XOX", " X ", "   "]}

\textbf{Output}: \fcj{false}

\textbf{Explanation}: Players take turns making moves.
\end{flushleft}

\paragraph{Example 3:}

\begin{flushleft}
\textbf{Input}: \fcj{board = ["XXX", "   ", "OOO"]}

\textbf{Output}: \fcj{false}
\end{flushleft}

\paragraph{Example 4:}

\begin{flushleft}
\textbf{Input}: \fcj{board = ["XOX", "O O", "XOX"]}

\textbf{Output}: \fcj{true}
\end{flushleft}

\paragraph{Note:}

\begin{itemize}
\item board is a length-3 array of strings, where each string \fcj{board[i]} has length 3.
\item Each \fcj{board[i][j]} is a character in the set \fcj{[" ", "X", "O"]}.
\end{itemize}

\subsection{Ad Hoc}
Since players take turns, the number of \fcj{'X'} must be equal to or one greater than the number of \fcj{'O'}.

\begin{enumerate}
\item If the first player wins, the number of \fcj{'X'} is one more than the number of \fcj{'O'}.

\item If the second player wins, the number of \fcj{'X'}s is equal to the number of \fcj{'O'}.

\item The board can't simultaneously have three \fcj{'X'} and three \fcj{'O'} in a row, and once one player has won (on their move), there are no subsequent moves.
\end{enumerate}


It turns out these conditions are also sufficient to check the validity of all boards. 

For the first two cases, count the number of \fcj{'X'} and \fcj{'O'}.

As to the 3rd case, use a helper function \fcj{win(player)} to check if that \fcj{player} has won. This checks the 8 horizontal, vertical, or diagonal lines of the board to find if there are three in a row equal to that player.


\setcounter{lstlisting}{0}
\begin{lstlisting}[style=customc, caption={Ad Hoc}]
bool validTicTacToe( vector<string>& board )
{
    auto count_X( 0 );
    auto count_O( 0 );
    for( const auto& row : board )
    {
        for( auto c : row )
        {
            if( c == 'X' )
            {
                count_X += 1;
            }
            else if( c == 'O' )
            {
                count_O += 1;
            }
        }
    }
    //num(x)==num(o) or num(x)==num(o) + 1
    if( ( count_X != count_O ) && ( count_X != count_O + 1 ) )
    {
        return false;
    }
    //if X wins, num(x)=num(o)+1 because X play first
    if( can_win( board, 'X' ) && ( count_X != count_O + 1 ) )
    {
        return false;
    }
    //if O wins, num(x)=num(o) because X play first
    if( can_win( board, 'O' ) && ( count_X != count_O ) )
    {
        return false;
    }
    return true;
}
//check if player wins
//any row, column, diagnaol or anti-diagnaol
//have same kind of player
bool can_win( const vector<string>& b, char player )
{
    for( int i = 0; i < 3; ++i )
    {
        //horizon
        if( ( b[i][0] == player ) && ( b[i][1] == player ) && ( b[i][2] == player ) )
        {
            return true;
        }

        //vertical
        if( ( b[0][i] == player ) && ( b[1][i] == player ) && ( b[2][i] == player ) )
        {
            return true;
        }
    }

    //diagnaol
    if( ( b[0][0] == player ) && ( b[1][1] == player ) && ( b[2][2] == player ) )
    {
        return true;
    }
    //aniti-diagnaol
    if( ( b[2][0] == player ) && ( b[1][1] == player ) && ( b[0][2] == player ) )
    {
        return true;
    }
    return false;
}
\end{lstlisting}