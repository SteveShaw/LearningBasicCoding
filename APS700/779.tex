\section{779 --- K-th Symbol in Grammar}
On the first row, we write a 0. Now in every subsequent row, we look at the previous row and replace each occurrence of \fcj{0} with \fcj{01}, and each occurrence of 1 with \fcj{10}.

Given row $N$ and index $K$, return the $K$--th indexed symbol in row $N$. (The values of $K$ are 1-indexed.)

\paragraph{Examples:}
\begin{flushleft}
\textbf{Input}: $N = 1$, $K = 1$

\textbf{Output}: 0

\textbf{Input}: $N = 2$, $K = 1$

\textbf{Output}: 0

\textbf{Input}: $N = 2$, $K = 2$

\textbf{Output}: 1

\textbf{Input}: $N = 4$, $K = 5$

\textbf{Output}: 1


\textbf{Explanation}:

row 1: \fcj{0}

row 2: \fcj{01}

row 3: \fcj{0110}

row 4: \fcj{01101001}

\end{flushleft}

\paragraph{Note:}

\begin{enumerate}
\item $N$ will be an integer in the range \fcj{[1, 30]}.
\item $K$ will be an integer in the range $[1, 2^{N-1}]$.
\end{enumerate}

\subsection{Recursion}
We can notice a pattern by observing a few rows: the right half is always the left half \textit{flipped}.

Therefore, if $K$ is in the right half, we could set $K\gets K - 2 ^{N-2}$ to be in the first half, and flip the final answer.

\setcounter{lstlisting}{0}
\begin{lstlisting}[style=customc, caption={Recursion}]
int kthGrammar( int N, int K )
{
    if( N == 1 )
    {
        return 0;
    }
    if( K <= ( 1 << ( N - 2 ) ) )
    {
        //K is in the left half
        //we can find in level N-1
        return kthGrammar( N - 1, K );
    }
    //K is in the right half
    //we can find in level N-1 for index K - ( 1 << ( N - 2 ) )
    //and flip the result
    return ( kthGrammar( N - 1, K - ( 1 << ( N - 2 ) ) ) ) ^ 1;
}
\end{lstlisting}

\subsection{Binary Count}
We know that right half of every row is the left half flipped.

When $K$ is written in binary (indexing from zero),  indexes of the right half of a row are ones with the first bit set to 1.

This means the number of times we flip the final answer is just the number of 1s in the binary representation of $K-1$.

\setcounter{lstlisting}{0}
\begin{lstlisting}[style=customc, caption={Bit Counts}]
int kthGrammar( int N, int K )
{
    int n = 0;
    --K;
    //the result is 1 flip
    //by the times which equal to
    //the bit 1s in K
    while( K )
    {
        K = ( K & ( K - 1 ) );
        ++n;
    }
    return n & 1;
}
\end{lstlisting}