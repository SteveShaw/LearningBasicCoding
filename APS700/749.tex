\section{749 --- Contain Virus}
A virus is spreading rapidly, and your task is to quarantine the infected area by installing walls.

The world is modeled as a 2-D array of cells, where 0 represents uninfected cells, and 1 represents cells contaminated with the virus. A wall (and only one wall) can be installed between any two \textbf{4-directionally adjacent cells}, on the shared boundary.

Every night, the virus spreads to all neighboring cells in all four directions unless blocked by a wall. Resources are limited. Each day, you can install walls around only one region -- the affected area (continuous block of infected cells) that threatens the most uninfected cells the following night. There will never be a tie.

Can you save the day? If so, what is the number of walls required? If not, and the world becomes fully infected, return the number of walls used.

\paragraph{Example 1:}

\begin{flushleft}
\textbf{Input}: 

\fcj{grid}:
\[
\begin{bmatrix}
0 & 1 & 0 & 0 & 0 & 0 & 0 & 1\\
0 & 1 & 0 & 0 & 0 & 0 & 0 & 1\\
0 & 0 & 0 & 0 & 0 & 0 & 0 & 1\\
0 & 0 & 0 & 0 & 0 & 0 & 0 & 0
\end{bmatrix}
\]

\textbf{Output}: 10


\textbf{Explanation}:

There are 2 contaminated regions.

On the first day, add 5 walls to quarantine the viral region on the left. The board after the virus spreads is:

\[
\begin{bmatrix}
0 & 1 & 0 & 0 & 0 & 0 & 1 & 1\\
0 & 1 & 0 & 0 & 0 & 0 & 1 & 1\\
0 & 0 & 0 & 0 & 0 & 0 & 1 & 1\\
0 & 0 & 0 & 0 & 0 & 0 & 0 & 1
\end{bmatrix}
\]

On the second day, add 5 walls to quarantine the viral region on the right. The virus is fully contained.
\end{flushleft}

\paragraph{Example 2:}

\begin{flushleft}
\textbf{Input}: 

\fcj{grid}:

\[
\begin{bmatrix}
1 & 1 & 1\\
1 & 0 & 1\\
1 & 1 & 1
\end{bmatrix}
\]

\textbf{Output}: 4

\textbf{Explanation}: 

Even though there is only one cell saved, there are 4 walls built.

Notice that walls are only built on the shared boundary of two different cells.
\end{flushleft}

\paragraph{Example 3:}

\begin{flushleft}
\textbf{Input}: 

\fcj{grid}:

\[
\begin{bmatrix}
1 & 1 & 1 & 0 & 0 & 0 & 0 & 0 & 0\\ 
1 & 0 & 1 & 0 & 1 & 1 & 1 & 1 & 1\\
1 & 1 & 1 & 0 & 0 & 0 & 0 & 0 & 0
\end{bmatrix}
\]

\textbf{Output}: 13

\textbf{Explanation}: 

The region on the left only builds two new walls.
\end{flushleft}

\paragraph{Note:}

\begin{itemize}
\item The number of rows and columns of \fcj{grid} will each be in the range \fcj{[1, 50]}.
\item Each \fcj{grid[i][j]} will be either 0 or 1.
\item Throughout the described process, there is always a contiguous viral region that will infect strictly more uncontaminated squares in the next round.
\end{itemize}