\section{716 --- Max Stack}
Design a max stack that supports \fcj{push}, \fcj{pop}, \fcj{top}, \fcj{peekMax} and \fcj{popMax}.

\begin{itemize}
\item \fcj{push(x)} -- Push element x onto stack.
\item \fcj{pop()} -- Remove the element on top of the stack and return it.
\item \fcj{top()} -- Get the element on the top.
\item \fcj{peekMax()} -- Retrieve the maximum element in the stack.
\item \fcj{popMax()} -- Retrieve the maximum element in the stack, and remove it. If you find more than one maximum elements, only remove the top-most one.
\end{itemize}

\paragraph{Example 1:}
\begin{flushleft}

\fcj{MaxStack stack = new MaxStack();}

\fcj{stack.push(5); }

\fcj{stack.push(1);}

\fcj{stack.push(5);}

\fcj{stack.top(); -> 5 }

\fcj{stack.popMax(); -> 5}

\fcj{stack.top(); -> 1}

\fcj{stack.peekMax(); -> 5}

\fcj{stack.pop(); -> 1}

\fcj{stack.top(); -> 5}
\end{flushleft}

\paragraph{Note:}
\begin{enumerate}
\item $-10^7 \leq x \leq 10^7$
\item Number of operations won't exceed 10000.
\item The last four operations won't be called when stack is empty.
\end{enumerate}

\subsection{Two Stacks}

A regular stack already supports the first 3 operations, so we focus on the last two.

For \fcj{peekMax}, we make use of another stack to track the largest value we've seen on the side. For example if we add \fcj{[2, 1, 5, 3, 9]}, we'll remember \fcj{[2, 2, 5, 5, 9]}. 

For \fcj{popMax}, we know what the current maximum (\fcj{peekMax}) is. We can use a buffer to save all numbers pop from the regular stack until we find that maximum, then push the popped elements int buffer back on the stack.

In the implementation, we use \fcc{vector} to replace \fcc{stack}.

\setcounter{lstlisting}{0}
\begin{lstlisting}[style=customc, caption={Two Stacks}]
class MaxStack
{
public:
    /** initialize your data structure here. */
    MaxStack()
    {
    }

    void push( int x )
    {
        //push x and max so far
        xs.push_back( x );
        if( x_max.empty() )
        {
            x_max.push_back( x );
        }
        else
        {
            int y = ( max )( x_max.back(), x );
            x_max.push_back( y );
        }
    }

    int pop()
    {
        int y = xs.back();
        xs.pop_back();

        x_max.pop_back();

        return y;
    }

    int top()
    {
        return xs.back();
    }

    int peekMax()
    {
        return x_max.back();
    }

    int popMax()
    {
        int y = peekMax();

        //remove the first one in xs
        //that is equal to y

        //all popped elements are saved in buffer
        stack<int> buffer;

        while( top() != y )
        {
            buffer.push( pop() );
        }

        //now top()=y, we need to pop it
        pop();

        //push back all popped elements
        //during the process
        while( !buffer.empty() )
        {
            push( buffer.top() );
            buffer.pop();
        }

        return y;
    }

    vector<int> xs;
    vector<int> x_max;
};
\end{lstlisting}