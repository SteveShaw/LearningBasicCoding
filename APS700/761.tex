\section{761 --- Special Binary String}
Special binary strings are binary strings with the following two properties:

\begin{itemize}
\item The number of 0s is equal to the number of 1s.
\item Every prefix of the binary string has at least as many 1s as 0s.
\end{itemize}

Given a special string $S$, a move consists of choosing two consecutive, non-empty, special substrings of $S$, and swapping them. (Two strings are consecutive if the last character of the first string is exactly one index before the first character of the second string.)

At the end of any number of moves, what is the lexicographically largest resulting string possible?

\paragraph{Example 1:}

\begin{flushleft}
\textbf{Input}: \fcj{S = "11011000"}

\textbf{Output}: \fcj{"11100100"}

\textbf{Explanation}:

The strings \fcj{"10"} (occuring at \fcj{S[1]}) and \fcj{"1100"} (at \fcj{S[3]}) are swapped.

This is the lexicographically largest string possible after some number of swaps.
\end{flushleft}

\paragraph{Note:}

\begin{itemize}
\item $S$ has length at most 50.
\item $S$ is guaranteed to be a special binary string as defined above.
\end{itemize}

\subsection{DFS}
If we regard 1, 0 in the definition of the special string as \fcj{'('} and \fcj{')'} separately,

the problem is actually to get the string which is so-called valid parenthesis and meanwhile is the lexicographically largest.

It is intuitive that we prefer deeper valid parenthesis to be in front (\textbf{deeper} means the string surrounded with more pairs of parenthesis, e.g., \fcj{'(())'} is deeper than \fcj{'()'} ). We can achieve that by sorting them reversely.

we go through $S$. Whenever the parentheses we met can be balanced, we construct valid parentheses by putting \fcj{'('} on the left boundary, \fcj{')'} on the right boundary, and doing with the inner part following the same pattern.

\setcounter{lstlisting}{0}
\begin{lstlisting}[style=customc, caption={DFS}]
string makeLargestSpecial( string S )
{
    size_t start = 0;
    vector<string> segs;
    int bal = 0;
    for( size_t i = 0; i < S.size(); ++i )
    {
        bal += ( S[i] == '0' ) ? -1 : 1;
        if( bal == 0 )
        {
            //this special string can be rearranged recursively
            //add to the array
            segs.push_back( "1" + makeLargestSpecial( S.substr( start + 1, i - start - 1 ) ) + "0" );
            //the next special string start at i+1
            start = i + 1;
        }
    }
    //sort generated largest special strings
    sort( segs.begin(), segs.end() );
    string init;
    //add the string from end to beginning
    return accumulate( rbegin( segs ), rend( segs ), init );
}
\end{lstlisting}