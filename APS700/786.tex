\section{786 --- K-th Smallest Prime Fraction}
A sorted list $A$ contains 1, plus some number of primes. Then, for every $p < q$ in the list, we consider the fraction $p/q$.

What is the $K$-th smallest fraction considered?  Return your answer as an array of ints, where \fcj{answer[0] = p} and \fcj{answer[1] = q}.

\paragraph{Examples:}
\begin{flushleft}


\textbf{Input}: \fcj{A = [1, 2, 3, 5]}, $K = 3$

\textbf{Output}: \fcj{[2, 5]}

\textbf{Explanation}:

The fractions to be considered in sorted order are:

$1/5$, $1/3$, $2/5$, $1/2$, $3/5$, $2/3$.

The third fraction is 2/5.

\textbf{Input}: \fcj{A = [1, 7]}, \fcj{K = 1}

\textbf{Output}: \fcj{[1, 7]}

\end{flushleft}

\paragraph{Note:}

\begin{enumerate}
\item $A$ will have length between 2 and 2000.

\item Each \fcj{A[i]} will be between 1 and 30000.

\item $K$ will be between 1 and \fcj{A.length * (A.length - 1) / 2}.
\end{enumerate}

\subsection{Binary Search}
The fraction range will be $(0, 1)$. Suppose $f(x)$ be the number of fractions below $x$. It's an increasing function on $x$, so we can binary search for the correct answer.

During the finding of $x$ such that there are exactly $K$ fractions below $x$, we record the largest fraction which is less than $x$.

\setcounter{lstlisting}{0}
\begin{lstlisting}[style=customc, caption={Binary Search}]
vector<int> kthSmallestPrimeFraction( vector<int>& A, int K )
{
    auto hi( 1.0 );
    auto lo( 0.0 );
    auto norm( 0 );
    auto denorm( 1 );
    auto count( 0 );
    vector<int> ans( 2 );
    while( hi - lo > 1e-9 )
    {
        auto mid( hi * .5 + lo * .5 );
        //get the count of fractions less than x
        get_count( A, mid, norm, denorm, count );

        if( count < K )
        {
            //search in higher half
            lo = mid;
        }
        else
        {
            //search in lower half
            hi = mid;
            //record norm/denorm
            ans[0] = norm;
            ans[1] = denorm;
        }
    }
    return ans;
}
//helper function to get
//number of fractions less than x
void get_count( vector<int>& A, double x, int& norm, int& denorm, int& count )
{
    norm = 0;
    denorm = 1;
    count = 0;
    auto L( static_cast< int >( A.size() ) );
    int left = -1;
    for( int right = 1; right < L; ++right )
    {
        while( static_cast< double >( A[left + 1] ) / static_cast< double >( A[right] ) < x )
        {
            ++left;
        }
        //when left = -1
        //left +1=0, this is a very good trick
        count += ( left + 1 );
        //get maximum fraction so far
        if( ( left >= 0 ) && ( norm * A[right] < denorm * A[left] ) )
        {
            norm = A[left];
            denorm = A[right];
        }
    }
}
\end{lstlisting}