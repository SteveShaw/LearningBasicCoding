\section{710 --- Random Pick with Blacklist}
Given a blacklist $B$ containing unique integers from $[0, N)$, write a function to return a uniform random integer from $[0, N)$ which is \textbf{NOT} in $B$.

Optimize it such that it minimizes the call to system's \fcj{Math.random()}.

\paragraph{Note:}

\begin{enumerate}
\item $1 \leq N \leq 10^9$
\item $0 \leq \lvert B\rvert < \min(10^5, N)$
\item $[0, N)$ does NOT include $N$. 
\end{enumerate}

\paragraph{Example 1:}

\begin{flushleft}
\textbf{Input}: 

\fcj{["Solution","pick","pick","pick"]}

\fcj{[[1,[]],[],[],[]]}

\textbf{Output}: \fcj{[null,0,0,0]}
\end{flushleft}

\section{Example 2:}

\begin{flushleft}
\textbf{Input}: 

\fcj{["Solution","pick","pick","pick"]}

\fcj{[[2,[]],[],[],[]]}

\textbf{Output}: \fcj{[null,1,1,1]}
\end{flushleft}

\paragraph{Example 3:}

\begin{flushleft}
\textbf{Input}: 
\fcj{["Solution","pick","pick","pick"]}

\fcj{[[3,[1]],[],[],[]]}

\textbf{Output}: \fcj{[null,0,0,2]}
\end{flushleft}

\paragraph{Example 4:}

\begin{flushleft}
\textbf{Input}: 

\fcj{["Solution","pick","pick","pick"]}

\fcj{[[4,[2]],[],[],[]]}

\textbf{Output}: \fcj{[null,1,3,1]}
\end{flushleft}

\subsection{Binary Search And Interval Selection}
We try to find out what the interval is for the generated random number and make sure it does not equal to the number in the blacklist.

Assume $L$ is the length of the blacklist $B$ and $M = N -L$. We can easily know that $B[i] \geq i$ for $i\in [0, L-1]$. 

If $B$ is sorted, then $B[i] - B[i-1] \geq 1 = (i-(i-1))$, i.e. $B[i] - i\geq B[i-1] - (i-1)$. Thus, $0\leq B[i] - i\leq B[L-1] - (L-1) \leq N - (L-1)$.

Notice, the count of numbers we can select is $N-L$. If we select $k$th number where $k$ is one of $0, 1, 2, \ldots, N-L-1$, we need to find the first $B[x]$ such that $B[x] - x > k$. The reason is showing below. 

Suppose we get a random number $r$ from the range $[0, N-L-1]$.

\begin{itemize}
    \item $0 \leq r < B[0] - 0 \longrightarrow$ just return $r$  
    \item $B[0] \leq r < B[1] - 1 \longrightarrow$ return $r+1$, this will make sure we do not get $B[0]$
    \item $B[1]-1 \leq r < B[2] - 2 \longrightarrow$ return $r+2$, this will make sure we do not get $B[1]$
    \item \dots
    \item $B[i]-i \leq r < B[i+1] - (i+1) \longrightarrow$ return $r+(i+1)$, this will make sure we do not get $B[i]$
    \item \dots
    \item $B[L-1]-(L-1) \leq r < N - L = M \longrightarrow$ return $r+(L)$, this will make sure we do not get $B[L-1]$
\end{itemize}

Therefore, after we find the first $B[x]-x > k$, we only need to return $k+x$ as the answer.

\setcounter{lstlisting}{0}
\begin{lstlisting}[style=customc, caption={Binary Search}]
class Solution
{
public:

    Solution( int N, vector<int>& blacklist )
        : seed( random_device{}() )
    {
        swap( B, blacklist );

        //sort the blacklist
        sort( B.begin(), B.end() );

        int L = static_cast<int>( B.size() );

        //set the distribution range to [0, N-L-1]
        uniform_int_distribution<>::param_type p( 0, N - L - 1 );
        dist.param( p );
    }

    int pick()
    {

        int L = static_cast<int>( B.size() );

        //get the random number in [0, N-L-1]
        int k = dist( seed );

        //find first B[l] such that B[l]-l > k
        int l = 0;
        int r = L;

        //rightmost binary search
        while( l < r )
        {
            int mid = ( l + r ) / 2;

            if( B[mid] - mid <=  k )
            {
                l = mid + 1;
            }
            else
            {
                r = mid;
            }
        }


        //this is the number we get
        return l + k;
    }

    mt19937 seed;
    uniform_int_distribution<> dist;
    vector<int> B;
};
\end{lstlisting}

Another one with STL

\begin{lstlisting}[style=customc, caption={STL}]
class Solution
{
public:
    Solution( int N, vector<int>& blacklist ): B( blacklist )
    {
        sort( begin( B ), end( B ) );
        for( size_t i = 0; i < B.size(); ++i )
        {
            //get number of whitelist numbers
            //that are less than B[i]
            B[i] -= i;
        }
        n = N;
    }
    int pick()
    {
        int k = ( rand() % ( n - ( int )B.size() ) );
        //get first B[i] that is B[i] - i > k
        auto it = upper_bound( begin( B ), end( B ), k );
        return k + distance( B.begin(), it );
    }
    vector<int>& B;
    int n;
};
\end{lstlisting}

\subsection{Virtual Whitelist}
We can re-map all blacklist numbers in $[0,N−\lvert B\rvert-1]$ to whitelist numbers such that when we randomly pick a number from $[0,N−\lvert B\rvert-1]$, we actually randomly pick amongst all whitelist numbers.

For example, for $N=6$ and $B=[0,2,3]$: the blacklist numbers in $[0,N−\lvert B\rvert-1] = [0, 2]$ are 0 and 2. We map them to two white list number such as $0\to 4$ and $2\to 5$.

\paragraph{Algorithm} 

Split $B$ into two blacklists, $X$ and $Y$, such that $X$ contains all blacklist numbers less than $N−\lvert B\rvert$ and $Y$ contains the rest.

Use $Y$ to create a list of all whitelist numbers, $W$ in $[N−\lvert B\rvert, N-1]$.

We make use of a hash map $M$, where $M[i]=i$ by default.

Then, iterate through all numbers in $X$, and set $M[X[i]]=W[i]$. Note that $\lvert X\rvert=\rvert W\lvert$.

Now, $M[0]$, $\ldots$, $M[N−\lvert B\rvert−1]$ maps to all whitelist numbers, so we can randomly pick a index $k$ in $[0,N-\lvert B\rvert-1]$ to get a random whitelist number.

\begin{lstlisting}[style=customc, caption={Re-map}]
class Solution
{
public:
    Solution( int N, vector<int>& blacklist )
        : seed( random_device{}() )
    {
        //get the length of whitelist number
        int LB = static_cast<int>( blacklist.size() );
        int LW = N - LB;

        //we create whitelist to cover
        //the numbers in [LW, N-1]
        unordered_set<int> w_cands;
        for( int i = LW; i < N; ++i )
        {
            w_cands.insert( i );
        }

        //remove all numbers belongs to blacklist
        for( int b : blacklist )
        {
            w_cands.erase( b );
        }

        //map blacklist numbers in [0, N-LW-1]
        //to the white numbers in [LW, N-1]
        auto it = w_cands.begin();

        for( int b : blacklist )
        {
            if( b < LW )
            {
                m_white[b] = *it;
                ++it;
            }
        }

        //set random number generator
        uniform_int_distribution<>::param_type p( 0, LW - 1 );
        dis.param( p );
    }

    int pick()
    {
        //generate random index k from [0, LW-1]
        int k = dis( seed );
        //check if k is mapped to a
        //number in the white list
        //of range[LW, N-1]
        auto it = m_white.find( k );
        if( it != m_white.end() )
        {
            return it->second;
        }

        //k is in [0, LW-1]
        return k;
    }

private:

    mt19937 seed;
    uniform_int_distribution<> dis;
    //map of white list numbers
    unordered_map<int, int> m_white;
}
\end{lstlisting}