\section{1028 --- Recover a Tree From Preorder Traversal}
We run a preorder depth first search on the root of a binary tree.
\par
At each node in this traversal, we output $ D $ dashes (where $ D $ is the depth of this node), then we output the value of this node.  (If the depth of a node is $ D $, the depth of its immediate child is $ D+1 $.  The depth of the root node is 0.)
\par
If a node has only one child, that child is guaranteed to be the left child.
\par
Given the output $ S $ of this traversal, recover the tree and return its root.
\paragraph{Example 1:}
\begin{flushleft}
\begin{figure}[H]
\begin{tikzpicture}
[my/.style={draw, circle, fill=gray!20, minimum size=5mm}]
\node[my] (1) at (0,0) {1};
\node[my] (2) [below=8mm of 1, xshift=-11mm] {2};
\node[my] (3) [below=8mm of 1, xshift=11mm] {5};
\node[my] (4) [below=8mm of 2, xshift=-6mm] {3};
\node[my] (5) [below=8mm of 2, xshift=6mm] {4};
\node[my] (6) [below=8mm of 3, xshift=-6mm] {6};
\node[my] (7) [below=8mm of 3, xshift=6mm] {7};
\draw[thick,red] (1) -- (2);
\draw[thick,red] (1) -- (3);
\draw[thick,red] (2) -- (4);
\draw[thick,red] (2) -- (5);
\draw[thick,red] (3) -- (6);
\draw[thick,red] (3) -- (7);
\end{tikzpicture}
\end{figure}
\textbf{Input}: $1D2DD3DD4D5DD6DD7$ ($ D $ is a dash)
\\
\textbf{Output}: $[1,2,5,3,4,6,7]$
\end{flushleft}

\paragraph{Example 2:}
\begin{flushleft}
\begin{figure}[H]
\begin{tikzpicture}
[my/.style={draw, circle, fill=gray!20, minimum size=5mm}]
\node[my] (1) at (0,0) {1};
\node[my] (2) [below=8mm of 1, xshift=-11mm] {2};
\node[my] (3) [below=8mm of 1, xshift=11mm] {5};
\node[my] (4) [below=8mm of 2, xshift=-6mm] {3};
\node[my] (5) [below=8mm of 4, xshift=-6mm] {4};
\node[my] (6) [below=8mm of 3, xshift=-6mm] {6};
\node[my] (7) [below=8mm of 6, xshift=-6mm] {7};
\draw[thick,red] (1) -- (2);
\draw[thick,red] (1) -- (3);
\draw[thick,red] (2) -- (4);
\draw[thick,red] (4) -- (5);
\draw[thick,red] (3) -- (6);
\draw[thick,red] (6) -- (7);
\end{tikzpicture}
\end{figure}
\textbf{Input}: $1D2DD3DDD4D5DD6DDD7$ ($ D $ is a dash)
\\
\textbf{Output}: $[1,2,5,3,\varnothing,6,\varnothing,4,\varnothing,7]$
\end{flushleft}

\paragraph{Example 3:}
\begin{flushleft}
\begin{figure}[H]
\begin{tikzpicture}
[my/.style={draw, circle, fill=gray!20, minimum size=5mm}]
\node[my] (1) at (0,0) {1};
\node[my] (2) [below=8mm of 1, xshift=-3mm] {401};
\node[my] (3) [below=8mm of 2, xshift=-8mm] {349};
\node[my] (4) [below=8mm of 2, xshift=8mm] {88};
\node[my] (5) [below=8mm of 3, xshift=-6mm] {90};
\draw[thick,red] (1) -- (2);
\draw[thick,red] (2) -- (3);
\draw[thick,red] (2) -- (4);
\draw[thick,red] (3) -- (5);
\end{tikzpicture}
\end{figure}
\textbf{Input}: $1D401DD349DDD90DD88$ ($ D $ is a dash)
\\
\textbf{Output}: $[1,401,\varnothing,349,88,90]$
\end{flushleft}

\paragraph{Note:}

\begin{itemize}
\item The number of nodes in the original tree is between 1 and 1000. 
\item Each node will have a value between 1 and $ 10^9 $.
\end{itemize}

\subsection{Hash Map}
\begin{itemize}
\item 创建一个hash map $M$,其key为level的number,而value为数组,存放当前level所有create出的tree node以及该tree node对应的parent node。每个node和其parent node是紧邻着的。
\item 遍历原数组,对于每个数字,创建相应的Tree Node。然后根据相应的level,找到上一层即level-1对应的parent。由于preorder的特性,以及我们设计的数据结构的特点,上一层的level对应的数组的倒数第二个node即为当前create出的node的parent。(最后一个node是这个parent的parent node)。
\item 最后从level = 1开始,将每个node与其parent node相连,先检查parent node的left child是否为null,如果是,将当前node设置为parent node的left child,否则设置为right child。(题目中说明了如果只有一个child,必定是left child)。
\end{itemize}

\setcounter{lstlisting}{0}
\begin{lstlisting}[style=customc, caption={Hash Map}]
TreeNode* recoverFromPreorder( string S )
{
    if( S.empty() )
    {
        return nullptr;
    }

    //add dash to let state change can
    //happen at the end of S
    S.push_back( '-' );

    //the hash map to
    //store each level's node and its parent
    unordered_map<int, vector<TreeNode*>> m_levels;

    int level = 0;
    int x = 0;

    int max_level = 0;
    TreeNode* parent = nullptr;

    int state = 1; //1=digit,2=dash

    for( auto c : S )
    {
        if( c != '-' )
        {
            if( state == 2 )
            {
                //update maximum level so far
                max_level = ( max )( level, max_level );
                //change state to 1
                //to indicate we are collecting numbers
                state = 1;
            }

            x = x * 10 + ( c - '0' );
        }
        else
        {
            if( state == 1 )
            {
                auto node = new TreeNode( x );

                if( level > 0 )
                {
                    //get node's parent
                    auto it = m_levels.find( level - 1 );
                    auto sz = it->second.size();
                    //the last element if the parent of parent
                    //the one to the last element is the parent
                    parent = it->second[sz - 2];
                }

                auto it = m_levels.find( level );
                if( it == m_levels.end() )
                {
                    //we put node and its parent along each other
                    m_levels.emplace( level, initializer_list<TreeNode*> {node, parent} ).first;
                }
                else
                {
                    it->second.emplace_back( node );
                    it->second.emplace_back( parent );
                }


                //reset level and x
                level = 0;
                x = 0;

                //change state to 2
                //to indicate we are counting dashes
                state = 2;
            }

            ++level;
        }
    }

    //now, we link the nodes to its parent
    for( int level = 1; level <= max_level; ++level )
    {
        auto& nodes = m_levels[level];

        for( size_t i = 0; i < nodes.size(); i += 2 )
        {
            if( !nodes[i + 1]->left )
            {
                //left child has not been set
                //set to left child
                nodes[i + 1]->left = nodes[i];
            }
            else
            {
                //left child has been set
                //set to right child
                nodes[i + 1]->right = nodes[i];
            }
        }
    }

    //root is at the level 0
    return m_levels[0][0];
}
\end{lstlisting}


\subsection{Stack}
\begin{itemize}
\item 创建一个stack,其中存放create出来的tree node。
\item 同样是从$S$中获取node的value和对应的level。然后根据value create一个新的node。
\item 如果当前level比stack的size还要小,说明stack中包含了和当前node以及其child nodes相同level的nodes,将这些 nodes 从 stack 中移除, 直到stack的size等于当前level。
\item 如果这时候stack不为空,那么这时候stack的top元素就是当前node的parent。(简单的推理: 当stack的size为1时,就只有root node。而root node的level为零。) 根据这个parent的左右子节点是否为null,来设定当前node为left child还是right child。
\item 最后如果$S$的size不为1,需要将多余的元素弹出,因为最终需要的是$S$最底下的root node。
\end{itemize}

\setcounter{lstlisting}{0}
\begin{lstlisting}[style=customc, caption={Stack}]
TreeNode* recoverFromPreorder( string S )
{
    if( S.empty() )
    {
        return nullptr;
    }

    S.push_back( '-' );

    stack<TreeNode*> stk;

    size_t level = 0;
    int x = 0;

    size_t max_level = 0;
    TreeNode* parent = nullptr;

    int state = 1; //1=digit,2=dash

    for( auto c : S )
    {
        if( c != '-' )
        {
            if( state == 2 )
            {
                max_level = ( max )( level, max_level );
                state = 1;
            }

            x = x * 10 + ( c - '0' );
        }
        else
        {
            if( state == 1 )
            {
                auto node = new TreeNode( x );

                //find the parent of node
                while( !stk.empty() && stk.size() > level )
                {
                    stk.pop();
                }

                if( !stk.empty() )
                {
                    //stk.top() is the parent of node
                    if( !stk.top()->left )
                    {
                        stk.top()->left = node;
                    }
                    else
                    {
                        stk.top()->right = node;
                    }

                }

                //push node into the stack
                stk.push( node );


                level = 0;
                x = 0;

                state = 2;
            }

            ++level;
        }
    }

    //we need to pop unnecessary nodes
    //to get root node
    while( stk.size() > 1 )
    {
        stk.pop();
    }

    return stk.top();
}
\end{lstlisting}