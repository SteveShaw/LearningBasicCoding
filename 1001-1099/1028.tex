\section{1028 --- Recover a Tree From Preorder Traversal}
We run a preorder depth first search on the root of a binary tree.
\par
At each node in this traversal, we output $ D $ dashes (where $ D $ is the depth of this node), then we output the value of this node.  (If the depth of a node is $ D $, the depth of its immediate child is $ D+1 $.  The depth of the root node is 0.)
\par
If a node has only one child, that child is guaranteed to be the left child.
\par
Given the output $ S $ of this traversal, recover the tree and return its root.
\paragraph{Example 1:}
\begin{flushleft}
\begin{figure}[H]
\begin{tikzpicture}
[my/.style={draw, circle, fill=gray!20, minimum size=5mm}]
\node[my] (1) at (0,0) {1};
\node[my] (2) [below=8mm of 1, xshift=-11mm] {2};
\node[my] (3) [below=8mm of 1, xshift=11mm] {5};
\node[my] (4) [below=8mm of 2, xshift=-6mm] {3};
\node[my] (5) [below=8mm of 2, xshift=6mm] {4};
\node[my] (6) [below=8mm of 3, xshift=-6mm] {6};
\node[my] (7) [below=8mm of 3, xshift=6mm] {7};
\draw[thick,red] (1) -- (2);
\draw[thick,red] (1) -- (3);
\draw[thick,red] (2) -- (4);
\draw[thick,red] (2) -- (5);
\draw[thick,red] (3) -- (6);
\draw[thick,red] (3) -- (7);
\end{tikzpicture}
\end{figure}
\textbf{Input}: $1D2DD3DD4D5DD6DD7$ ($ D $ is a dash)
\\
\textbf{Output}: $[1,2,5,3,4,6,7]$
\end{flushleft}

\paragraph{Example 2:}
\begin{flushleft}
\begin{figure}[H]
\begin{tikzpicture}
[my/.style={draw, circle, fill=gray!20, minimum size=5mm}]
\node[my] (1) at (0,0) {1};
\node[my] (2) [below=8mm of 1, xshift=-11mm] {2};
\node[my] (3) [below=8mm of 1, xshift=11mm] {5};
\node[my] (4) [below=8mm of 2, xshift=-6mm] {3};
\node[my] (5) [below=8mm of 4, xshift=-6mm] {4};
\node[my] (6) [below=8mm of 3, xshift=-6mm] {6};
\node[my] (7) [below=8mm of 6, xshift=-6mm] {7};
\draw[thick,red] (1) -- (2);
\draw[thick,red] (1) -- (3);
\draw[thick,red] (2) -- (4);
\draw[thick,red] (4) -- (5);
\draw[thick,red] (3) -- (6);
\draw[thick,red] (6) -- (7);
\end{tikzpicture}
\end{figure}
\textbf{Input}: $1D2DD3DDD4D5DD6DDD7$ ($ D $ is a dash)
\\
\textbf{Output}: $[1,2,5,3,\varnothing,6,\varnothing,4,\varnothing,7]$
\end{flushleft}

\paragraph{Example 3:}
\begin{flushleft}
\begin{figure}[H]
\begin{tikzpicture}
[my/.style={draw, circle, fill=gray!20, minimum size=5mm}]
\node[my] (1) at (0,0) {1};
\node[my] (2) [below=8mm of 1, xshift=-3mm] {401};
\node[my] (3) [below=8mm of 2, xshift=-8mm] {349};
\node[my] (4) [below=8mm of 2, xshift=8mm] {88};
\node[my] (5) [below=8mm of 3, xshift=-6mm] {90};
\draw[thick,red] (1) -- (2);
\draw[thick,red] (2) -- (3);
\draw[thick,red] (2) -- (4);
\draw[thick,red] (3) -- (5);
\end{tikzpicture}
\end{figure}
\textbf{Input}: $1D401DD349DDD90DD88$ ($ D $ is a dash)
\\
\textbf{Output}: $[1,401,\varnothing,349,88,90]$
\end{flushleft}

\paragraph{Note:}

\begin{itemize}
\item The number of nodes in the original tree is between 1 and 1000. 
\item Each node will have a value between 1 and $ 10^9 $.
\end{itemize}

\subsection{Hash Map}
\begin{itemize}
\item 创建一个hash map $M$,其key为level的number,而value为数组,存放当前level所有create出的tree node。
\item 遍历原数组,对于每个数字,创建相应的Tree Node。然后根据相应的level,找到上一层即level-1对应的parent。由于preorder的特性,上一层的level对应的数组的最后一个node即为当前create出的node的parent。
\item 
\end{itemize}

\setcounter{lstlisting}{0}
\begin{lstlisting}[style=customc, caption={TODO}]
/**
 * Definition for a binary tree node.
 * struct TreeNode {
 *     int val;
 *     TreeNode *left;
 *     TreeNode *right;
 *     TreeNode(int x) : val(x), left(NULL), right(NULL) {}
 * };
 */
TreeNode* recoverFromPreorder( string S )
{
}
\end{lstlisting}