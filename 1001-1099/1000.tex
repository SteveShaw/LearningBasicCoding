\section{1000 --- Minimum Cost to Merge Stones}
There are $N$ piles of stones $S$ arranged in a row. The $i$-th pile has $S[i]$ stones.
\par
A move consists of merging exactly $K$ consecutive piles into one pile, and the cost of this move is equal to the total number of stones in these $K$ piles.
\par
Find the minimum cost to merge all piles of stones into one pile.  If it is impossible, return $-1$.

 

\paragraph{Example 1:}

\begin{flushleft}
\textbf{Input}: $S = [3,2,4,1]$, $K = 2$
\\
\textbf{Output}: 20
\\
\textbf{Explanation}: 
We start with [3, 2, 4, 1].
\\
We merge [3, 2] for a cost of 5, and we are left with [5, 4, 1].
\\
We merge [4, 1] for a cost of 5, and we are left with [5, 5].
\\
We merge [5, 5] for a cost of 10, and we are left with [10].
\\
The total cost was 20, and this is the minimum possible.
\end{flushleft}

\paragraph{Example 2:}

\begin{flushleft}
\textbf{Input}: $S = [3,2,4,1]$, $K = 3$
\\
\textbf{Output}: $-1$
\\
\textbf{Explanation}:
\\
After any merge operation, there are 2 piles left, and we can't merge anymore.  So the task is impossible.
\end{flushleft}

\paragraph{Example 3:}

\begin{flushleft}
\textbf{Input}: stones = [3,5,1,2,6], K = 3
\\
\textbf{Output}: 25
\\
\textbf{Explanation}: 
\\
We start with [3, 5, 1, 2, 6].
\\
We merge [5, 1, 2] for a cost of 8, and we are left with [3, 8, 6].
\\
We merge [3, 8, 6] for a cost of 17, and we are left with [17].
\\
The total cost was 25, and this is the minimum possible.
\end{flushleft}
 
\paragraph{Note:
}
\begin{itemize}
\item $1 \leq |S| \leq 30$
\item $2 \leq K \leq 30$
\item $1 \leq S[i] \leq 100$
\end{itemize}
\subsection{Dynamic Programming}
\begin{itemize}
\item 实际上这是一个3-D Dynamic Programming,对应的function $F[i][j][p]$表示minimum cost to merge from $S[i]$ to $S[j]$ to $p$ piles。为了方便起见, $p$的范围是$[1,K]$而不是$[0,K-1]$。
\item 首先要看$S$能否通过merge 连续$K$个元素的方式最终得到一个元素。假设经过$x$次merge后,$S$变为一个元素的数组,那么就有$L-K+1-K+1-\ldots-K+1=L-x\times(K-1)=1$,即$L-1=x\times(K-1)$,因此$(L-1)\bmod(K-1)=0$,如果不满足这个条件,是不能merge成一个元素的。
\item 首先需要找到$F[i][j][K]$,这个需要通过双重循环来做到,其中$p$从2一直到$K$循环,然后逐个测试在$[i,j]$范围内的每个位置$t$,看$F[i][j][p]$和$F[i][t][p-1]+F[t+1][j][1]$哪个更小。
\item $F[i][t][p-1]+F[t+1][j][1]$的意思是从$S[i]$到$S[t]$形成$p-1$个piles所需的minimum cost和从$S[t+1]$到$S[j]$形成1个pile所需的minimum cost之和。
\item 得到$F[i][j][K]$后,就知道了$F[i][j][1]$的cost了,即$F[i][j][1] = F[i][j][K] + \sum\limits_{y=i}^{j}S[y]$,因为这个$K$个能形成一个pile
\end{itemize}
\setcounter{algorithm}{0}
\begin{algorithm}[H]
\caption{}
\begin{algorithmic}[1]
\Procedure{MergeStones}{$S, L, K$}
\If{$(L-1)\bmod (K-1)\neq 0$} \Comment 这种merge成一个pile的操作无法完成
\State \Return $-1$
\EndIf
\State $\star$ Initialize a 3-d array $F$. $F$ has dimension $L\times L \times (K+1)$ and $F[i][j][p]:=+\infty$.
\State $\ast$ 很显然$F[i][i][1]=0$,因为已经$S[i]$本身就是一个pile了,不需要merge
\For{$i:=0$ \textbf{to} $L-1$}
\State $F[i][i][1]\gets 0$
\EndFor 
\State $\ast$ 接下来进行bottom-up 的Dynamic Programming
\For{$l:=2$ \textbf{to} $L$} \Comment 这种3D的DP一般都以长度作为最外层循环
\For{$i:=0$ \textbf{to} $i+l-1<L$} \Comment $[i, i+l-1]$长度为$l$
\State $j:=i+l-1$
\algstore{1000algo}
\end{algorithmic}
\end{algorithm}
\begin{algorithm}[H]
\begin{algorithmic}[1]
\algrestore{1000algo}
\State $\ast$ 这时候进行$F[i][j][K]$的DP计算过程
\For{$p:=2$ \textbf{to} $K$} \Comment piles 从2到$K$
\State $\ast$ 对于$[i,j-1]$中的每个位置$t$,
\State $\ast$ 获得区间$[i,t]$中获得$p-1$个堆和$[t+1,j]$中获得一个堆的cost总和
\For{$t:=i$ \textbf{to} $j$} 
\State $F[i][j][p]\gets \min(F[i][j][p], F[i][t][p-1] + F[t+1][j][1])$
\EndFor
\EndFor
\State $\ast$ 这时候已经得到$F[i][j][K]$,接下来计算merge 这$K$个piles到1个pile的cost了
\State $F[i][j][1]\gets F[i][j][K] + \sum\limits_{y=i}^{j}S[y]$
\EndFor
\EndFor
\State $\ast$ $F[0][L-1][1]$就是minimum cost to merge $S[0]$ to $S[L-1]$ into one pile
\State \Return $F[0][L-1][1]$
\EndProcedure
\end{algorithmic}
\end{algorithm}
在代码实现中,为了快速获得$\sum\limits_{y=i}^{j}S[y]$,采用prefix sum array的方法。
\setcounter{lstlisting}{0}
\begin{lstlisting}[style=customc, caption={Dynamic Programming}]
int mergeStones( vector<int>& stones, int K )
{
    int L = static_cast< int >( stones.size() );

    if( ( L - 1 ) % ( K - 1 ) != 0 )
    {
        return -1;
    }

    vector<vector<vector<int>>> F( L, vector<vector<int>> ( L, vector<int> ( K + 1, 99999 ) ) );

    for( int i = 0; i < L; ++i )
    {
        F[i][i][1] = 0;
    }

    vector<int> v_presum( L, 0 );
    v_presum[0] = stones[0];

    for( int i = 1; i < L; ++i )
    {
        v_presum[i] = stones[i] + v_presum[i - 1];
    }

    for( int l = 2; l <= L; ++l )
    {
        for( int i = 0; i + l - 1 < L; ++i )
        {
            int j = i + l - 1;


            for( int t = i; t < j; ++t )
            {
                for( int piles = 2; piles <= K; ++piles )
                {
                    F[i][j][piles] = ( min )( F[i][j][piles], F[i][t][piles - 1] + F[t + 1][j][1] );
                }
            }

            //v_presum[0] = S[0]
            //So we need to check if i=0, otherwise i-1 is invalid
            if( i == 0 )
            {
                F[i][j][1] = F[i][j][K] + v_presum[j];
            }
            else
            {
                F[i][j][1] = F[i][j][K] + v_presum[j] - v_presum[i - 1];
            }

        }
    }

    return F[0][L - 1][1];
}
\end{lstlisting}