\section{435 --- Non-overlapping Intervals}
Given a collection of intervals $I$, find the minimum number of intervals you need to remove to make the rest of the intervals non-overlapping.

\paragraph{Note:}

\begin{itemize}
\item You may assume the interval's end point is always bigger than its start point.
\item Intervals like [1,2] and [2,3] have borders \textbf{touching} but they don't overlap each other.
\end{itemize}
 

\paragraph{Example 1:}

\begin{flushleft}
\textbf{Input}: [ [1,2], [2,3], [3,4], [1,3] ]

\textbf{Output}: 1

\textbf{Explanation}: [1,3] can be removed and the rest of intervals are non-overlapping.
\end{flushleft}
 

\paragraph{Example 2:}

\begin{flushleft}
\textbf{Input}: [ [1,2], [1,2], [1,2] ]

\textbf{Output}: 2

\textbf{Explanation}: You need to remove two [1,2] to make the rest of intervals non-overlapping.
\end{flushleft}
 

\paragraph{Example 3:}

\begin{flushleft}
\textbf{Input}: [ [1,2], [2,3] ]

\textbf{Output}: 0

\textbf{Explanation}: You don't need to remove any of the intervals since they're already non-overlapping.
\end{flushleft}

\subsection{Count Non-Overlapped Intervals}
The following greedy algorithm does find a set of non-overlapping intervals of maximum size:

\begin{itemize}
\item Select the interval, $x$, with the earliest finishing time.
\item Remove $x$, and all intervals intersecting x, from the set of candidate intervals.
\item Repeat until the set of candidate intervals is empty.
\end{itemize}


\setcounter{lstlisting}{0}
\begin{lstlisting}[style=customc, caption={Count Maximum Non-overlapping Intervals}]
int eraseOverlapIntervals( vector<vector<int>>& intervals )
{
    if( intervals.empty() )
    {
        return 0;
    }

    //according to the greedy algorithm
    //we need sort the intervals per the end
    sort( intervals.begin(), intervals.end(), []( const vector<int>& v1, const vector<int>& v2 )
    {
        if( v1[1] < v2[1] )
        {
            return true;
        }
        else if( v1[1] == v2[1] )
        {
            return v1[0] < v2[0];
        }

        return false;
    } );

    //count is the number of non-overlapping intervals
    int count = 1;

    int last = intervals[0][1];

    for( size_t i = 1; i < intervals.size(); ++i )
    {
        if( intervals[i][0] >= last )
        {
            //only count the ones does not touch with last
            ++count;
            last = intervals[i][1];
        }
    }


    return static_cast<int>( intervals.size() ) - count;

}
\end{lstlisting}