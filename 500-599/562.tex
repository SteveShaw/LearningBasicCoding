\section{562 --- Longest Line of Consecutive One in Matrix}
Given a 01 matrix $M$, find the longest line of consecutive one in the matrix. The line could be horizontal, vertical, diagonal or anti-diagonal.
\paragraph{Example:}

\begin{flushleft}
\textbf{Input}:
[[0,1,1,0],
 [0,1,1,0],
 [0,0,0,1]]
\textbf{Output}: 3

\end{flushleft}

\paragraph{Hint:} 
\begin{itemize}
\item The number of elements in the given matrix will not exceed 10,000.
\end{itemize}

\subsection{Dynamic Programming}
\begin{itemize}
\item 用$F(i,j,k)$代表the length of consecutive ones at $M[i][j]$,其中
\begin{itemize}
\item $k=0$ 表示horizontal
\item $k=1$ 表示vertical
\item $k=2$ 表示diagonal
\item $k=3$ 表示anti-diagonal
\end{itemize}
\item 如果$M[i][j]=1$,分别 check $M[i-1][j]$, $M[i][j-1]$, $M[i-1][j-1]$ 和 $M[i-1][j+1]$是否为1。 如果是1,那么$F(i,j,k)\gets F(i,j,k)+1$,否则$F(i,j,k)=1$。
\item 如果$M[i][j]=0$, 那么$F(i,j,k)=0$
\item Maintain一个变量,记录遇到的最大长度。
\end{itemize}

\setcounter{lstlisting}{0}
\begin{lstlisting}[style=customc, caption={Dynamic Programming}]
int longestLine( vector<vector<int>>& M )
{

    if( M.empty() || M[0].empty() )
    {
        return 0;
    }

    using count_t = array<int, 4>;

    vector<vector<count_t>> F( M.size(), vector<count_t>( M[0].size(), count_t{0, 0, 0, 0} ) );


    int ans = 0;

    for( size_t r = 0; r < M.size(); ++r )
    {
        for( size_t c = 0; c < M[0].size(); ++c )
        {
            if( M[r][c] == 1 )
            {
                //the length of each direction
                //is at least 1
                F[r][c][0] = 1;
                F[r][c][1] = 1;
                F[r][c][2] = 1;
                F[r][c][3] = 1;

                if( ( r > 0 ) || ( c > 0 ) )
                {
                    if( ( r > 0 ) && ( M[r - 1][c] == 1 ) )
                    {
                        //the horizon length increments
                        F[r][c][0] = F[r - 1][c][0] + 1;
                    }

                    if( ( c > 0 ) && ( M[r][c - 1] == 1 ) )
                    {
                        //the vertical length increments
                        F[r][c][1] = F[r][c - 1][1] + 1;
                    }

                    if( ( r > 0 ) && ( c > 0 ) && ( M[r - 1][c - 1] == 1 ) )
                    {
                        //the diag length increments
                        F[r][c][2] = F[r - 1][c - 1][2] + 1;
                    }

                    if( ( r > 0 ) && ( c < M[0].size() - 1 ) && ( M[r - 1][c + 1] == 1 ) )
                    {
                        //the anti-diag length increments
                        F[r][c][3] = F[r - 1][c + 1][3] + 1;
                    }
                }

                //get the maximum length so far
                ans = ( max )( ans, F[r][c][0] );
                ans = ( max )( ans, F[r][c][1] );
                ans = ( max )( ans, F[r][c][2] );
                ans = ( max )( ans, F[r][c][3] );
            }//outmost if
        } //inner for
    }//outer for

    return ans;
}
\end{lstlisting}