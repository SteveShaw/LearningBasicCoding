\section{504 --- Base 7}
Given an integer $n$, return its base 7 string representation.

\paragraph{Example 1:}

\begin{flushleft}
\textbf{Input}: 100

\textbf{Output}: 202

\end{flushleft}

\paragraph{Example 2:}

\begin{flushleft}
\textbf{Input}: $-7$

\textbf{Output}: $-10$
\end{flushleft}

\paragraph{Note:} 

\begin{itemize}
\item The input will be in range of $[-10^{7}, 10^{7}]$.
\end{itemize}

\subsection{Iterative}
每次除以7,余数放入结果字符串,从得到的商继续。

\setcounter{lstlisting}{0}
\begin{lstlisting}[style=customc, caption={Iterative}]
string convertToBase7( int num )
{
    if( num == 0 )
    {
        return "0";
    }

    string ans;

    int n = num;
    int sign = 1;

    if( num < 0 )
    {
        n *= -1;
        sign = -1;
    }

    while( n )
    {
        int q = n / 7;
        int r = n - q * 7;

        ans.push_back( r + '0' );
        n = q;
    }

    if( sign < 0 )
    {
        ans.push_back( '-' );
    }

    //since we add low positions first
    //reverse whole result
    reverse( begin( ans ), end( ans ) );

    return ans;
}
\end{lstlisting}



\subsection{Recursive}
\begin{itemize}
\item $n$可以分为除以7的商和对7的余数两个部分。
\item 分别对这两个部分递归调用函数。
\end{itemize}

\begin{lstlisting}[style=customc, caption={Recursion}]
string convertToBase7( int num )
{
    if( num < 0 )
    {
        return "-" + convertToBase7( -num );
    }

    if( num < 7 )
    {
        //value less than 7 is itself
        return to_string( num );
    }

    return convertToBase7( num / 7 ) + convertToBase7( num % 7 );
}
\end{lstlisting}