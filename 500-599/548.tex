\section{548 --- Split Array with Equal Sum}
Given an array with $n$ integers, $A$, you need to find if there are triplets $(i, j, k)$ which satisfies following conditions:

\begin{itemize}
\item$ 0 < i, i + 1 < j$, $j + 1 < k < n - 1$
\item Sum of subarrays $(0, i - 1)$, $ (i + 1, j - 1) $, $(j + 1, k - 1)$ and $(k + 1, n - 1)$ should be equal.
\end{itemize}

where we define that subarray $ (L, R) $ represents a slice of the original array starting from the element indexed $L$ to the element indexed $R$.

\paragraph{Example:}

\begin{flushleft}
\textbf{Input}: $[1,2,1,2,1,2,1]$

\textbf{Output}: \texttt{True}

\textbf{Explanation}:

$i = 1, j = 3, k = 5$.

$\sum A(0, i - 1) = \sum A(0, 0) = 1$

$\sum A(i + 1, j - 1) = \sum A(2, 2) = 1$

$\sum A(j + 1, k - 1) = \sum A(4, 4) = 1$

$\sum A(k + 1, n - 1) = \sum A(6, 6) = 1$
\end{flushleft}

\paragraph{Note:}

\begin{itemize}
\item $1 \leq n \leq 2000$
\item Elements in the given array will be in range $[-1,000,000, 1,000,000]$.
\end{itemize}

\subsection{Hash Map and Mathematical Induction}
\begin{itemize}
\item From the given condition, we can get $i\geq 1$, $j \geq 3$ and $k\geq 5$, and the length of the array must be no less than 7.
\item Create a prefix summation array $P$ with same length as the input array.
\item From the given condition, we can get the following equations
\begin{align*}
P[j-1] - P[i] &= P[i-1] \\
P[k-1] - P[j] &= P[j-1] - P[i]\\
P[n-1] - P[k] &= P[k-1] - P[j]\\
\end{align*}
By arrange these equations we can get
\begin{align*}
P[j-1] &=  P[i] + P[i-1] \\
P[k-1] + P[i] &= P[j] + P[j-1]\\
P[n-1] + P[j] &= P[k] + P[k-1]\\
\end{align*}
As we can see, $P[x]+P[x-1]$ is the important characteristic.
\item Therefore, create a hash map with the key equal to $P[x] + P[x-1]$ and the value is the array of associated indexes.
\item From $j=3$ until $n-4$, we try each $j$
 \begin{enumerate}
 \item Find in the hash map if any key equal to $P[j-1]$. If there is a key equal to $P[j-1]$, there exists some $i$ that $P[i] + P[i-1] = P[j-1]$ (the first equation). We must make sure that $i+1 < j$ by the given condition. Since the array of associated indexes are sorted, we can use a binary search to find the candidates of $i$ in the array
 \item Then we find if any key equal to $P[n-1]+P[j]$. If there is a key equal to $P[n-1]+P[j]$, there exists some $k$ that $P[n-1] + P[j] = P[k] + P[k-1]$ (the third equation). We must make sure that $j+1<k$ by the given condition. Similarly, we use a binary search to find the candidates $k$ in the array.
 \item Finally, we need to traverse each $i$ and $k$ to make sure that $P[k-1] + P[i] = P[j] + P[j-1]$ (the second equation)
 \end{enumerate}
\end{itemize}

\setcounter{lstlisting}{0}
\begin{lstlisting}[style=customc, caption={Hash Map}]
bool splitArray( vector<int>& nums )
{

    //i>=1, j>=3, k>=5, n>=7
    if( nums.empty() || ( nums.size() < 7 ) )
    {
        return false;
    }

    vector<int> ps( nums.size(), 0 );
    ps[0] = nums[0];

    unordered_map<int, vector<int>> m;

    int L = static_cast<int>( nums.size() );

    //get prefix sum
    //create the hash map with key equal to
    //ps[i]+ps[i-1] and associated indexes
    for( int i = 1; i < L; ++i )
    {
        ps[i] = ps[i - 1] + nums[i];

        auto it = m.find( ps[i] + ps[i - 1] );
        if( it == m.end() )
        {
            m.emplace( ps[i] + ps[i - 1], vector<int> {i} );
        }
        else
        {
            it->second.push_back( i );
        }
    }


    //test each j to find
    //if there exists i and k
    for( int j = 3 ; j < L - 3; ++j )
    {
        //i and j must satisfy
        //ps[j-1] = ps[i] + ps[i-1]
        auto it = m.find( ps[j - 1] );
        if( it == m.end() )
        {
            continue;
        }

        //find first index that is
        //no less than j-1;
        //then it->second[0,x] is the candidiates of i
        int x = leftmost( it->second, j - 1 );

        //j and k must satisfy
        //ps[j] + ps[n-1] = ps[k] + ps[k-1]
        auto it_k = m.find( ps[j] + ps.back() );
        if( it_k == m.end() )
        {
            continue;
        }


        //find the first index that is
        //larger than j+1 (k)
        //then it_k->second[y,len-1] is the candidates of k
        int y = rightmost( it_k->second, j + 1 );
        int len = static_cast<int>( it_k->second.size() );

        //i,j and k must satisfy
        //ps[i] + ps[k-1] = ps[j] + ps[j-1]
        int z = ps[j] + ps[j - 1];

        for( int u = 0; u < x; ++u )
        {
            int i = it->second[u];

            for( int v = y; v < len; ++v )
            {
                int k = it_k->second[v];

                if( ps[k - 1] + ps[i] == z )
                {
                    return true;
                }
            }
        }
    }

    return false;
}

int leftmost( vector<int>& A, int target )
{
    int l = 0;
    int r = static_cast<int>( A.size() );

    while( l < r )
    {
        int mid = ( l + r ) / 2;
        if( A[mid] < target )
        {
            l = mid + 1;
        }
        else
        {
            r = mid;
        }
    }

    return l;
}

int rightmost( vector<int>& A, int target )
{
    int l = 0;
    int r = static_cast<int>( A.size() );

    while( l < r )
    {
        int mid = ( l + r ) / 2;
        if( A[mid] <= target )
        {
            l = mid + 1;
        }
        else
        {
            r = mid;
        }
    }

    return l;
}
\end{lstlisting}