\section{592 --- Fraction Addition and Subtraction}
Given a string $S$ representing an expression of fraction addition and subtraction, you need to return the calculation result in string format. The final result should be \textbf{irreducible fraction}. If your final result is an integer, say 2, you need to change it to the format of fraction that has denominator 1. So in this case, 2 should be converted to 2/1.

\paragraph{Example 1:}

\begin{flushleft}

\textbf{Input}: $ -1/2+1/2 $

\textbf{Output}: $ 0/1 $

\end{flushleft}

\paragraph{Example 2:}

\begin{flushleft}
\textbf{Input}: $ -1/2+1/2+1/3 $

\textbf{Output}: $ 1/3 $
\end{flushleft}

\paragraph{Example 3:}

\begin{flushleft}
\textbf{Input}: $ 1/3-1/2 $

\textbf{Output}: $ -1/6 $
\end{flushleft}

\paragraph{Example 4:}

\begin{flushleft}
\textbf{Input}: $ 5/3+1/3 $

\textbf{Output}: $ 2/1 $
\end{flushleft}

\paragraph{Note:}
\begin{itemize}
\item The input string only contains 0 to 9, \textbackslash, $+$ and $-$. So does the output.
\item Each fraction (input and output) has format $\pm$numerator/denominator. If the first input fraction or the output is positive, then $+$ will be omitted.
\item The input only contains valid irreducible fractions, where the numerator and denominator of each fraction will always be in the range $[1,10]$. If the denominator is 1, it means this fraction is actually an integer in a fraction format defined above.
\item The number of given fractions will be in the range $[1,10]$.
\item The numerator and denominator of the final result are guaranteed to be valid and in the range of 32-bit int.
\end{itemize}

\subsection{GCD}
\begin{itemize}
\item Use GCD to reduce the fractions
\item We make use of a class \texttt{Fraction} to store the input fractions.
\item Use an functor class to do addition.
\end{itemize}

\setcounter{lstlisting}{0}
\begin{lstlisting}[style=customc, caption={GCD}]
class Solution
{
public:
    string fractionAddition( string expression )
    {
        if( expression.empty() )
        {
            return "";
        }

        Fraction f1;
        Fraction f2;

        unsigned char state = 0;

        size_t i = 0;

        if( expression[0] == '-' )
        {
            //skip first sign
            ++i;
        }

        //since both minus and plus
        //can exist, we find the earlier one
        auto pos = expression.find( '-', i );
        pos = ( min )( pos, expression.find( '+', i ) );

        i = 0;

        int count = 0;

        while( pos < expression.size() )
        {
            if( count == 0 )
            {
                //set fraction 1
                f1.from_chars( expression.c_str(), i, pos );
            }
            else if( count == 1 )
            {
                //set fraction 2
                f2.from_chars( expression.c_str(), i, pos );

                //f1 = f1+f2;
                AddOper op( f1, f2 );
                f1 = op();

                //reset count to zero
                count = 1;
            }

            i = pos + 1;

            //save current index of operator
            auto last_pos = pos;

            //find next operator
            pos = expression.find( '-', i );
            pos = ( min )( pos, expression.find( '+', i ) );

            if( expression[last_pos] == '-' )
            {
                //treat fraction as minus
                i = last_pos;
            }
        }

        if( i < expression.size() )
        {
            //process last fraction
            if( count == 1 )
            {
                //f1 has already been set
                //we read to f2
                f2.from_chars( expression.c_str(), i, expression.size() );
                AddOper op( f1, f2 );
                f1 = op();

            }
            else
            {
                //otherwise, only one fracton
                //inside expression
                f1.from_chars( expression.c_str(), i, expression.size() );
            }
        }

        return to_string( f1.norm ) + "/" + to_string( f1.denorm );
    }

    struct Fraction
    {
        int norm;
        int denorm;

        //read norm/denorm from string
        void from_chars( const char* s, size_t start, size_t end )
        {
            int x = 0;
            int sign = 1;
            auto i = start;
            if( s[start] == '-' )
            {
                ++i;
                sign = -1;
            }

            for( ; i < end; ++i )
            {
                if( s[i] == '/' )
                {
                    norm = x * sign;
                    x = 0;
                    continue;
                }

                x = x * 10 + ( s[i] - '0' );
            }

            denorm = x;
        }
    };

    struct AddOper
    {
        const Fraction& f1;
        const Fraction& f2;

        AddOper( const Fraction& f1_, const Fraction& f2_ )
            : f1( f1_ )
            , f2( f2_ )
        {
        }

        Fraction operator()()
        {
            int norm = f1.norm * f2.denorm + f2.norm * f1.denorm;
            int denorm = f1.denorm * f2.denorm;

            if( norm == 0 )
            {
                denorm = 1;
            }
            //get gcd for absolution values of norm/denorm
            int m = gcd( abs( norm ), abs( denorm ) );

            if( denorm < 0 )
            {
                //change the sign of
                //both denorm and norm
                //make sure denorm is positive
                denorm = -denorm;
                norm = -norm;
            }

            return Fraction{norm / m, denorm / m};
        }

        //get gcd of x and y
        int gcd( int x, int y )
        {
            if( y == 0 )
            {
                return x;
            }

            return gcd( y, x % y );
        }
    };
};
\end{lstlisting}
