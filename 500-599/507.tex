\section{507 --- Perfect Number}
We define the Perfect Number is a positive integer that is equal to the sum of all its positive divisors except itself.

Now, given an integer $n$, write a function that returns true when it is a perfect number and false when it is not.

\paragraph{Example:}

\begin{flushleft}
\textbf{Input}: 28

\textbf{Output}: \texttt{true}

\textbf{Explanation}: $28 = 1 + 2 + 4 + 7 + 14$
\end{flushleft}

\paragraph{Note:} 
\begin{itemize}
\item The input number n will not exceed 100,000,000. $(10^8)$
\end{itemize}

\subsection{Optimal Solution}
\begin{itemize}
\item Suppose the given number $n$ has $m$ distinct factors, $f_1, f_2, \ldots, f_m$. 
\item Since the number $n$ divisible by $f_i$, it is also divisible by $f_j=n/f_i$.
\item Also, the largest number in such a pair can only be up to $\sqrt{n}$ ( because $\sqrt{n} \times \sqrt{n}=n$ ). Thus, we can get a significant reduction in the run-time by iterating only up to $\sqrt{n}$ and get such $f_i$ and $f_j$ in a single pass directly.
\item \textbf{Note}: If $\sqrt{n}$ is also a factor, we have to consider the factor only once. 
\item \textbf{Note}: While considering 1 as such a factor, $n$ will also be considered as the other factor. Thus, we need to subtract $n$ from the sum.
\end{itemize}

\setcounter{lstlisting}{0}
\begin{lstlisting}[style=customc, caption={Optimal Approach}]
bool checkPerfectNumber( int num )
{
    if( num <= 0 )
    {
        return false;
    }

    int sum = 0;

    //only up to sqrt(num)
    for( int i = 1; i * i <= num; ++i )
    {
        int q = num / i; //factor pair(i, q)
        int r = num - q * i;

        if( r == 0 )
        {
            if( q != i )
            {
                sum += q;
                sum += i;
            }
        }
    }

    //for factor 1, the factor pair is (1, num)
    //so need to subtract num from sum
    return sum - num == num;
}
\end{lstlisting}