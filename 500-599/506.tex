\section{506 --- Relative Ranks}
Given scores of $N$ athletes, find their relative ranks and the people with the top three highest scores, who will be awarded medals: \textbf{Gold Medal}, \textbf{Silver Medal} and \textbf{Bronze Medal}.

\paragraph{Example 1:}

\begin{flushleft}
\textbf{Input}: $[5, 4, 3, 2, 1]$

\textbf{Output}: [Gold Medal, Silver Medal, Bronze Medal, 4, 5]

\textbf{Explanation}: 

The first three athletes got the top three highest scores, so they got Gold Medal, Silver Medal and Bronze Medal. 

For the left two athletes, you just need to output their relative ranks according to their scores.
\end{flushleft}

\paragraph{Note:}
\begin{itemize}
\item $N$ is a positive integer and won't exceed 10,000.
\item All the scores of athletes are guaranteed to be unique.
\end{itemize}

\subsection{Priority Queue}
其实就是找到排序后原来的元素在排序后的位置。可以用priority queue作为数据结构。

\setcounter{lstlisting}{0}
\begin{lstlisting}[style=customc, caption={Priority Queue}]
vector<string> findRelativeRanks( vector<int>& nums )
{
    priority_queue<pair<int, size_t>> pq;

    for( size_t i = 0; i < nums.size(); ++i )
    {
        pq.emplace( nums[i], i );
    }

    vector<string> ans( nums.size() );

    if( !pq.empty() )
    {
        auto t = pq.top();
        ans[t.second] = "Gold Medal";
        pq.pop();
    }

    if( !pq.empty() )
    {
        auto t = pq.top();
        ans[t.second] = "Silver Medal";
        pq.pop();
    }

    if( !pq.empty() )
    {
        auto t = pq.top();
        ans[t.second] = "Bronze Medal";
        pq.pop();
    }

    int next = 4;

    while( !pq.empty() )
    {
        auto t = pq.top();
        pq.pop();
        ans[t.second] = to_string( next );
        ++next;
    }

    return ans;
}
\end{lstlisting}