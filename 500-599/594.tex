\section{594 --- Longest Harmonious Subsequence}
We define a harmounious array as an array where the difference between its maximum value and its minimum value is exactly 1.

Now, given an integer array, $A$, you need to find the length of its longest harmonious subsequence among all its possible subsequences.

\paragraph{Example 1:}
\begin{flushleft}

\textbf{Input}: $[1,3,2,2,5,2,3,7]$

\textbf{Output}: 5

\textbf{Explanation}: The longest harmonious subsequence is $[3,2,2,2,3]$.
 

\end{flushleft}

\paragraph{Note:} 

\begin{itemize}
\item The length of the input array will not exceed 20,000.
\end{itemize}

\subsection{Hash Map}
\begin{itemize}
\item We make use of a hash map $M$ which stores the number of times an element occurs in the array. Traverse $A$ and fill $M$ once.
\item After this, we traverse over the keys of $M$. For every key considered, say $k$, we find out if the map contains the $k + 1$. We need not care about $k-1$, because if $k$ is present in the harmonic subsequence, at one time either $k+1$ or $k - 1$ only could be included in the harmonic subsequence. The case of $k$being in the harmonic subsequence will automatically be considered, when $k- 1$ is encountered as the current key.
\item Whenver we find that $k+ 1$ exists in $M$, we determine the count of the current harmonic subsequence as count of $k$ plus count of $k+1$
\end{itemize}

注意,这种方法可以不用考虑$k-1$是因为首先遍历完了$A$。如果要边做遍历,边进行$M$的更新,那么需要同时考虑$k+1$和$k-1$。
\setcounter{lstlisting}{0}
\begin{lstlisting}[style=customc, caption={Hash Map With One Pass}]
int findLHS( vector<int>& nums )
{
    unordered_map<int, int>m;
	
    int res = 0;
	
    for( auto i : nums )
    {
        m[i]++;
        if( m.count( i + 1 ) )
        {
            res = max( res, m[i] + m[i + 1] );
        }
        if( m.count( i - 1 ) )
        {
            res = max( res, m[i] + m[i - 1] );
        }
    }
	
    return res;
}
\end{lstlisting}

\begin{lstlisting}[style=customc, caption={Hash Map With TWo Pass}]
int findLHS( vector<int>& nums )
{
    unordered_map<int, int> m;

    for( int n : nums )
    {
        m[n] += 1;
    }

    int ans = 0;

    for( const auto& p : m )
    {
        //only consider k+1
        if( m.count( p.first + 1 ) )
        {
            ans = ( max )( p.second + m[p.first + 1], ans );
        }
    }

    return ans;
}
\end{lstlisting}