\section{564 --- Find the Closest Palindrome}
Given an integer $n$, find the closest integer (not including itself), which is a palindrome.

The \textbf{closest} is defined as absolute difference minimized between two integers.

\paragraph{Example 1:}
\begin{flushleft}
\textbf{Input}: $ 123 $

\textbf{Output}: $ 121 $
\end{flushleft}

\paragraph{Note:}
\begin{itemize}
\item The input $ n $ is a positive integer represented by string, whose length will not exceed 18.
\item If there is a tie, return the smaller one as answer.
\end{itemize}

\subsection{Mathematical Approach}
首先可以得到结论, 如果需要通过copy一半字符的方式从$n$得到 Palindrome number,那么一定是copy first half 到 second half.

其次,有三种方式对$n$进行转换得到距离最近的Palindrome number
\begin{enumerate}
\item Copy first half to second half. For example:  Suppose $n=10987$. After this kind of transformation, we can get $10901$.
\item Decrement the first half and then copy first half.
\item Increment the first half and then copy first half.
\item If the digit at the middle index is zero, decrement the first half and then copy first half: For example, when $n=20001$, decrements the first half $200$ we get $199$, then copy to second half we get $19991$. .
\item If the digit at the middle index is $9$, increments the second half and then copy first half: For example, when $n=10987$, increments the first half $109$ we get $110$, then copy to second half we get $11011$. We do not decrement first half in this case because the generated number must have larger difference than just copying the first half to second half.
\end{enumerate}
