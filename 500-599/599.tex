\section{599 --- Minimum Index Sum of Two Lists}
Suppose Andy and Doris want to choose a restaurant for dinner, and they both have a list of favorite restaurants represented by strings.

You need to help them find out their common interest with the least list index sum. If there is a choice tie between answers, output all of them with no order requirement. You could assume there always exists an answer.

\paragraph{Example 1:}
\begin{flushleft}
\textbf{Input}:

[ Shogun, Tapioca Express, Burger King, KFC ]

[ Piatti, The Grill at Torrey Pines, Hungry Hunter Steakhouse, Shogun ]

\textbf{Output}: [ Shogun ]

\textbf{Explanation}: The only restaurant they both like is Shogun.
\end{flushleft}

\paragraph{Example 2:}

\begin{flushleft}
\textbf{Input}:

[Shogun, Tapioca Express, Burger King, KFC]

[KFC, Shogun, Burger King]

\textbf{Output}: [Shogun]

\textbf{Explanation}: 

The restaurant they both like and have the least index sum is Shogun with index sum 1 ($0+1$).
\end{flushleft}

\paragraph{Note:}
\begin{itemize}
\item The length of both lists will be in the range of $[1, 1000]$.
\item The length of strings in both lists will be in the range of $[1, 30]$.
\item The index is starting from 0 to the list length minus 1.
\item No duplicates in both lists.
\end{itemize}