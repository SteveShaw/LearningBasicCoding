\section{547 --- Friend Circles}
There are $ N $ students in a class. Some of them are friends, while some are not. Their friendship is transitive in nature. For example, if $ A $ is a direct friend of $ B $, and $ B $ is a direct friend of $ C $, then $ A $ is an indirect friend of $ C $. And we defined a friend circle is a group of students who are direct or indirect friends.

Given a $N\times N$ matrix $ M $ representing the friend relationship between students in the class. If $M[i][j] = 1$, then the $i$th and $j$th students are direct friends with each other, otherwise not. And you have to output the total number of friend circles among all the students.

\paragraph{Example 1:}
\begin{flushleft}

\textbf{Input}:
\begin{table}[H]
\begin{tabular}{ccc}
1 & 1 & 0\\
1 & 1 & 0\\
0 & 0 & 1
\end{tabular}
\end{table} 

\textbf{Output}: 2

\textbf{Explanation}:

The 0th and 1st students are direct friends, so they are in a friend circle. 

The 2nd student himself is in a friend circle. So return 2.
\end{flushleft}

\paragraph{Example 2:}
\begin{flushleft}

\textbf{Input}:
\begin{table}[H]
\begin{tabular}{ccc}
1 & 1 & 0\\
1 & 1 & 1\\
0 & 1 & 1
\end{tabular}
\end{table} 

\textbf{Output}: 1

\textbf{Explanation}:

The 0th and 1st students are direct friends, the 1st and 2nd students are direct friends, 

so the 0th and 2nd students are indirect friends. All of them are in the same friend circle, so return 1.
\end{flushleft}

\paragraph{Note:}
\begin{itemize}
\item $N$ is in range $[1,200]$.
\item $M[i][i] = 1$ for all students.
\item If $M[i][j] = 1$, then $M[j][i] = 1$.
\end{itemize}

\subsection{DFS}
\begin{itemize}
\item The given matrix can be viewed as the Adjacency Matrix of a graph. By viewing the matrix in such a manner, our problem reduces to the problem of finding the number of connected components in an undirected graph.
\item In order to find the number of connected components in an undirected graph, one of the simplest methods is to make use of \textbf{Depth First Search} starting from every node. We make use of an array of size $N\times N$, $V$, to indicate that the $i^{\texttt{th}}$ node has already been visited while undergoing a \textbf{Depth First Search} from some node.
\item To undergo DFS, we pick up a node and visit all its directly connected nodes. But, as soon as we visit any of those nodes, we recursively apply the same process to them as well. Thus, we try to go as deeper into the levels of the graph as possible starting from a current node first, leaving the other direct neighbour nodes to be visited later on.
\end{itemize}

\setcounter{lstlisting}{0}
\begin{lstlisting}[style=customc, caption={DFS}]
int findCircleNum( vector<vector<int>>& M )
{
    auto N = static_cast<int>( M.size() );

    vector<int> seen( N, 0 );

    int ans = 0;

    //test each person
    for( int i = 0; i < N; ++i )
    {
        if( seen[i] == 0 )
        {
            //person i has not been
            //visited, dfs from him
            dfs( M, i, seen );
            ++ans;
        }
    }

    return ans;
}

//depth first search
void dfs( vector<vector<int>>& M, int x, vector<int>& seen )
{
    auto N = static_cast<int>( M.size() );

    for( int y = 0; y < N; ++y )
    {
        if( ( M[x][y] == 1 ) && ( seen[y] == 0 ) )
        {
            //y is the friend of x
            //and y has not been visited
            seen[y] = 1;
            dfs( M, y, seen );
        }
    }
}
\end{lstlisting}

\subsection{BFS}
\begin{itemize}
\item In case of \textbf{Breadth First Search}, we start from a particular node and visit all its directly connected nodes first. After all the direct neighbours have been visited, we apply the same process to the neighbour nodes as well. Thus, we exhaust the nodes of a graph on a level by level basis.
\item We also make use of an array, $V$, to keep a track of the already visited nodes. We increment the count of connected components whenever we need to start off with a new node as the root node for applying BFS which hasn't been already visited.
\end{itemize}

\setcounter{lstlisting}{0}
\begin{lstlisting}[style=customc, caption={BFS}]
int findCircleNum( vector<vector<int>>& M )
{
    int N = static_cast<int>( M.size() );

    queue<int> q;

    vector<int> seen( N, 0 );

    int ans = 0;

    for( int i = 0; i < N; ++i )
    {
        if( seen[i] == 0 )
        {
            //for each unvisited persion
            //we apply BFS from this person
            //q will be reused
            q.push( i );

            while( !q.empty() )
            {
                auto x = q.front();
                q.pop();

                seen[x] = 1;

                for( int y = 0; y < N; ++y )
                {
                    if( ( M[x][y] == 1 ) && ( seen[y] == 0 ) )
                    {
                        //y is friend of x
                        //and has not been visited
                        q.push( y );
                    }
                }
            }

            ++ans;
        }
    }

    return ans;
}
\end{lstlisting}

\subsection{Union Find}
\begin{itemize}
\item Another method that can be used to determine the number of connected components in a graph is the \textbf{union find} method.
\item We make use of an array of size $N$, $P$. We traverse over all the nodes of the graph. For every node traversed, we traverse over all the nodes directly connected to it and assign them to a single group which is represented by their parent node. This process is called forming a union. Every group has a single parent node.
\item For every new pair of nodes found, we look for the parents of both the nodes. If the parents nodes are the same, it indicates that they have already been united into the same group. If the parent nodes differ, it means they are yet to be united. Thus, for the pair of nodes $ (x, y) $, while forming the union, we assign $P[P[y]] \gets P[x]$, which ultimately combines them into the same group.
\end{itemize}

\setcounter{lstlisting}{0}
\begin{lstlisting}[style=customc, caption={Union Find}]
int findCircleNum( vector<vector<int>>& M )
{
    auto N = static_cast<int>( M.size() );

    vector<int> parents( N, 0 );

    for( int i = 1; i <= N; ++i )
    {
        parents[i - 1] = i - 1;
    }

    for( int r = 0; r < N; ++r )
    {
        for( int c = 0; c < N; ++c )
        {
            if( M[r][c] == 1 )
            {
                //put r and c into one
                //parent
                group( parents, r, c );
            }
        }

    }

    int ans = 0;

    for( int i = 1; i <= N; ++i )
    {
        if( parents[i - 1] == i - 1 )
        {
            //each group's parent
            //is the parent itself
            ++ans;
        }
    }

    return ans;
}

//union x and y
void group( vector<int>& parents, int x, int y )
{
    int px = find_parent( parents, x );
    int py = find_parent( parents, y );

    if( px != py )
    {
        parents[py] = px;
    }
}

//find parent of x
int find_parent( vector<int>& parents, int x )
{
    while( parents[x] != x )
    {
        x = parents[x];
    }

    return x;
}
\end{lstlisting}