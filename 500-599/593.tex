\section{593 --- Valid Square}
Given the coordinates of four points in 2D space, return whether the four points could construct a square.

The coordinate $(x,y)$ of a point is represented by an integer array with two integers.

\paragraph{Example:}

\begin{flushleft}

\textbf{Input}: $p_1 = [0,0]$, $p_2 = [1,1]$, $p_3 = [1,0]$, $p_4 = [0,1]$

\textbf{Output}: \texttt{True}

\end{flushleft} 

\paragraph{Note:}

\begin{itemize}
\item All the input integers are in the range $[-10000, 10000]$.
\item A valid square has four equal sides with positive length and four equal angles (90-degree angles).
\item Input points have no order.
\end{itemize}

\subsection{Sorting}
If we sort the given set of points based on their $x$-coordinate values, and in the case of a tie, based on their $y$-coordinate value, we can obtain an arrangement, which directly reflects the arrangement of points on a valid square boundary possible.

Consider the only possible cases as shown in the figure below:

\begin{figure}[H]
\begin{tikzpicture}
[every node/.style={draw, rectangle, blue, minimum size=2cm}, very thick]
\draw[very thick, >=stealth, ->] (0,0) -- ++(0,4);
\draw[very thick, >=stealth, ->] (0,0) -- ++(4,0);
\node(0) at (0.5, 1.5) [anchor=south west, rotate=-30] {};
\draw[fill=red] (0.south west) circle (3pt);
\draw[fill=red] (0.north west) circle (3pt);
\draw[fill=red] (0.north east) circle (3pt);
\draw[fill=red] (0.south east) circle (3pt);
\node[draw=none] at ($(0.south west) + (0, -3.5mm)$) {$p_0$};
\node[draw=none] at ($(0.north west) + (-3.5mm, -2.5mm)$) {$p_1$};
\node[draw=none] at ($(0.south east) + (3.5mm, -2.5mm)$) {$p_2$};
\node[draw=none] at ($(0.north east) + (3mm, -3.5mm)$) {$p_3$};
\draw[green] (0.south west) -- (0.north east);
\draw[green] (0.south east) -- (0.north west);
\draw[very thick, >=stealth, ->] (6,0) -- ++(0,4);
\draw[very thick, >=stealth, ->] (6,0) -- ++(4,0);
\node(1) at (6.5, 2.5) [anchor=south west, rotate=-50] {};
\draw[fill=red] (1.south west) circle (3pt);
\draw[fill=red] (1.north west) circle (3pt);
\draw[fill=red] (1.north east) circle (3pt);
\draw[fill=red] (1.south east) circle (3pt);
\node[draw=none] at ($(1.south west) + (0, -3.5mm)$) {$p_0$};
\node[draw=none] at ($(1.north west) + (-3.5mm, 0)$) {$p_2$};
\node[draw=none] at ($(1.south east) + (3.5mm, -2.5mm)$) {$p_1$};
\node[draw=none] at ($(1.north east) + (3mm, -3.5mm)$) {$p_3$};
\draw[green] (1.south west) -- (1.north east);
\draw[green] (1.south east) -- (1.north west);
\draw[very thick, >=stealth, ->] (12,0) -- ++(0,4);
\draw[very thick, >=stealth, ->] (12,0) -- ++(4,0);
\node(2) at (13.2, 1) [anchor=south west] {};
\draw[fill=red] (2.south west) circle (3pt);
\draw[fill=red] (2.north west) circle (3pt);
\draw[fill=red] (2.north east) circle (3pt);
\draw[fill=red] (2.south east) circle (3pt);
\node[draw=none] at ($(2.south west) + (-3.5mm, 0)$) {$p_0$};
\node[draw=none] at ($(2.north west) + (-3.5mm, 0)$) {$p_1$};
\node[draw=none] at ($(2.south east) + (3.5mm, 0)$) {$p_2$};
\node[draw=none] at ($(2.north east) + (3.5mm, 0)$) {$p_3$};
\draw[green] (2.south west) -- (2.north east);
\draw[green] (2.south east) -- (2.north west);
\end{tikzpicture}
\end{figure}

In each case, after sorting, we obtain the following conclusion regarding the connections of the points:

\begin{itemize}
\item $(p_0,p_1)$, $(p_1,p_3)$, $ (p_3, p_2)  $,  and $(p_2, p_0)$  form the four sides of any valid square.
\item $(p_0, p_3)$ and $ (p_1, p_2) $ form the diagonals of the square.

\end{itemize}
Thus, once the sorting of the points is done, based on the above knowledge, we can directly compare  $(p_0,p_1)$, $(p_1,p_3)$, $ (p_3, p_2p)  $,  and $(p_2, p_0)$
for equality of lengths(corresponding to the sides); and $(p_0, p_3)$ and $ (p_1, p_2) $ for equality of lengths(corresponding to the diagonals).

\setcounter{lstlisting}{0}
\begin{lstlisting}[style=customc, caption={Sort}]
bool validSquare( vector<int>& p1, vector<int>& p2, vector<int>& p3, vector<int>& p4 )
{
    vector<vector<int>> pts( 4 );

    swap( pts[0], p1 );
    swap( pts[1], p2 );
    swap( pts[2], p3 );
    swap( pts[3], p4 );

    //sort the points per the x-coordinate first
    //if tie, per y-coordinate
    sort( begin( pts ), end( pts ), []( const vector<int>& a, const vector<int>&b )
    {
        if( a[0] < b[0] )
        {
            return true;
        }

        if( a[0] == b[0] )
        {
            return a[1] < b[1];
        }

        return false;
    } );

    //get the square distance
    auto sqd = [&pts]( int a, int b )
    {
        int x = pts[a][0] - pts[b][0];
        int y = pts[a][1] - pts[b][1];
        return x * x + y * y;
    };

    array<int, 4> da;

    da[0] = sqd( 0, 1 );
    da[1] = sqd( 1, 3 );
    da[2] = sqd( 3, 2 );
    da[3] = sqd( 2, 0 );

    //if all values are equal
    //the result will be 4
    int cnt = count( begin( da ), end( da ), da[0] );

    if( ( cnt != 4 ) || ( da[0] == 0 ) )
    {
        //if they are not equal
        //or all have same coordinates
        return false;
    }

    //make sure the diagonal lengths are equal
    return sqd( 0, 3 ) == sqd( 1, 2 );
}
\end{lstlisting}