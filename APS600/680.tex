\section{680 --- Valid Palindrome II}
Given a non-empty string $s$, you may delete at most one character. Judge whether you can make it a palindrome.

\paragraph{Example 1:}

\begin{flushleft}
\textbf{Input}: \lstinline[language=Java, basicstyle=\small\ttfamily, keywordstyle=\bfseries\color{green!40!black}]|"aba"|

\textbf{Output}: \lstinline[language=Java, basicstyle=\small\ttfamily, keywordstyle=\bfseries\color{green!40!black}]|True|

\end{flushleft}

\section{Example 2:}

\begin{flushleft}
\textbf{Input}: \lstinline[language=Java, basicstyle=\small\ttfamily, keywordstyle=\bfseries\color{green!40!black}]|"abca"|

\textbf{Output}: \lstinline[language=Java, basicstyle=\small\ttfamily, keywordstyle=\bfseries\color{green!40!black}]|True|

\textbf{Explanation}: You could delete the character \lstinline[language=Java, basicstyle=\small\ttfamily, keywordstyle=\bfseries\color{green!40!black}]|'c'|.

\end{flushleft}

\paragraph{Note:}

\begin{itemize}
\item  The string will only contain lowercase characters \textbf{a} to \textbf{z}. The maximum length of the string is 50000.
\end{itemize}

\subsection{Greedy}
Starting with $l=0$ and $r = \left|s\right|-1$. When $s[l]\neq s[r]$, we check if $s[l+1, r]$ or $s[l, r-1]$ is palindrome. If any of the two is palindrome, we can be sure that the string can be made palindrome by either delete $s[l]$ or $s[r]$. 

\setcounter{lstlisting}{0}
\begin{lstlisting}[style=customc, caption={Greedy}]
bool validPalindrome( string s )
{

    //helper function to check if s[start,end] is a palindrome
    auto is_pal = []( const string & s, size_t start, size_t end )
    {
        auto left = start;
        auto right = end;

        while( left < right )
        {
            if( s[left] != s[right] )
            {
                return false;
            }

            ++left;
            --right;

        }

        return true;
    };

    size_t l = 0;
    size_t r = s.size() - 1;

    while( l < r )
    {
        if( s[l] != s[r] )
        {
            //we check if we can delete s[l] or s[r]
            //to make s as a palindrome
            return is_pal( s, l + 1, r ) || is_pal( s, l, r - 1 );
        }

        ++l;
        --r;
    }

    return true;
}
\end{lstlisting} 