\section{625 --- Minimum Factorization}
Given a positive integer $a$, find the smallest positive integer $b$ whose multiplication of each digit equals to $a$.

If there is no answer or the answer is not fit in 32-bit signed integer, then return 0.

\paragraph{Example 1}

\begin{flushleft}
\textbf{Input}:

48 

\textbf{Output}:

68
\end{flushleft}

\paragraph{Example 2}
\begin{flushleft}
\textbf{Input}:

15

\textbf{Output}:

35
\end{flushleft}

\subsection{Factorization}
\begin{itemize}
    \item We know that the final number generated, $x$, should be such that its digits should have a product equal to the given number $a$. In other words, the digits of $x$ will be the factors of the given number $a$. Thus, our problem reduces to finding the factors(not necessarily prime) of $a$ and finding their smallest possible arrangement. 
    \item We start with trying with the largest possible factor $9$, obtain as many such counts of this factor as possible in $x$ and place such factors obtained at its least significant positions. Then, we go on decrementing the number currently considered as the possible factor and if it is a factor, we keep on placing it at relatively more significant positions in $x$. We go on getting such factors till we are done considering all the numbers from 9 to 2. \item At the end, $x$ gives the required result.
\end{itemize}