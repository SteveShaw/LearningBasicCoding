\section{683 --- K Empty Slots}
You have $N$ bulbs in a row numbered from 1 to $N$. Initially, all the bulbs are turned off. We turn on exactly one bulb everyday until all bulbs are on after N days.

You are given an array bulbs of length $N$ where \lstinline[language=C++, basicstyle=\small\ttfamily, keywordstyle=\bfseries\color{green!40!black}]|bulbs[i] = x| means that on the $(i+1)$-th day, we will turn on the bulb at position $x$ where $i$ is 0-indexed and $x$ is 1-indexed.

Given an integer $K$, find out the minimum day number such that there exists two turned on bulbs that have exactly $K$ bulbs between them that are all turned off.

If there isn't such day, return $-1$.

 

\paragraph{Example 1:}
\begin{flushleft}


\textbf{Input}: \lstinline[language=C++, basicstyle=\small\ttfamily, keywordstyle=\bfseries\color{green!40!black}]|bulbs = [1,3,2]|, $K=1$

\textbf{Output}: 2

\textbf{Explanation}:

On the first day: \lstinline[language=C++, basicstyle=\small\ttfamily, keywordstyle=\bfseries\color{green!40!black}]|bulbs[0] = 1|, first bulb is turned on: \lstinline[language=C++, basicstyle=\small\ttfamily, keywordstyle=\bfseries\color{green!40!black}]|[1,0,0]|

On the second day: \lstinline[language=C++, basicstyle=\small\ttfamily, keywordstyle=\bfseries\color{green!40!black}]|bulbs[1] = 3|, third bulb is turned on: \lstinline[language=C++, basicstyle=\small\ttfamily, keywordstyle=\bfseries\color{green!40!black}]|[1,0,1]|

On the third day: \lstinline[language=C++, basicstyle=\small\ttfamily, keywordstyle=\bfseries\color{green!40!black}]|bulbs[2] = 2|, second bulb is turned on: \lstinline[language=C++, basicstyle=\small\ttfamily, keywordstyle=\bfseries\color{green!40!black}]|[1,1,1]|

We return 2 because on the second day, there were two on bulbs with one off bulb between them.
\end{flushleft}

\paragraph{Example 2:}
\begin{flushleft}


\textbf{Input}: \lstinline[language=C++, basicstyle=\small\ttfamily, keywordstyle=\bfseries\color{green!40!black}]|bulbs = [1,2,3]|, $K = 1$

\textbf{Output}: $-1$
\end{flushleft}

\paragraph{Note:}

\begin{enumerate}
\item $1 \leq N \leq 20000$
\item each \lstinline[language=C++, basicstyle=\small\ttfamily, keywordstyle=\bfseries\color{green!40!black}]|bulbs[i]| is in range \lstinline[language=C++, basicstyle=\small\ttfamily, keywordstyle=\bfseries\color{green!40!black}]|[1, N]|.
\item \lstinline[language=C++, basicstyle=\small\ttfamily, keywordstyle=\bfseries\color{green!40!black}]|bulbs| is a permutation of numbers from 1 to $N$.
\item $0 \leq K \leq 20000$
\end{enumerate}

\subsection{Insert Into Sorted Structure }
Let's add bulbs in the order they light on. When each bulb lights on, we check it's neighbors to see if they can satisfy the condition with the current bulb.

We maintain a tree set $S$, then insert index of each bulb from the given array. Apparently, the elements in $S$ are those indices of the bulbs that have been lighted before current bulb. For current bulb, we use a binary search to check the first lighted bulb that has larger index, and the first one that has smaller index than the index of current one.


\setcounter{lstlisting}{0}
\begin{lstlisting}[style=customc, caption={Tree Set}]
int kEmptySlots( vector<int>& bulbs, int K )
{
    if( bulbs.size() < 2 )
    {
        return -1;
    }

    set<int> S;

    //day is starting from 1
    int day = 1;

    for( int x : bulbs )
    {
        //find the first lighted bulb
        //has a index less than current lighted bulb
        //index x
        auto it = S.lower_bound( x );
        if( it != S.begin() )
        {
            //This is a lower bound
            //the element before it
            //is the one less than x
            --it;
            if( x - *it == K + 1 )
            {
                return day;
            }
        }

        //find the firsrt lighted bulb
        //has a index larger than x
        it = S.upper_bound( x );

        if( it != S.end() )
        {
            if( *it - x == K + 1 )
            {
                return day;
            }
        }

        //add current index of lighted bulb
        //into the set
        S.insert( x );

        //increments the day
        ++day;
    }

    return -1;
}
\end{lstlisting}

\subsection{Sliding Window}
Suppose $D[x]=i$ means the bulb at position $x$ is on at day $i$. We want to find intervals $[a,b]$ such that $D[a]$, $D[b]$ are the two smallest values in $(D[a], D[a+1], \ldots, D[b])$ and $b-a=k+1$

This means bulb at position $a$ and $b$ are on before $(a+1, \ldots, b-1)$, thus there will be $K$ bulbs off on day $\max(D[a], D[b])$.

\paragraph{Algorithm}

We construct array $D$ from the input. 

Starting from $a=0$ and $b=k+1$, For each possible interval $(a,b)$, we check if any bulb at position $i$ where $a<i<b$ can meet the condition: $D[i] > D[a] $ and $D[i] > D[b]$. When the condition is meet, $\max(D[a], D[b])$ is one of candidates. Then, we search in the next interval $(b, b+k+1)$.

Otherwise, bulb at position $i$ is \textbf{on} before $a$ and $b$, we need to search in another interval which is $(b, b+i+1)$.

Finally, we will return the minimum from the candidate answers or $-1$ if no any such interval exist.

\begin{lstlisting}[style=customc, caption={Sliding Window}]
int kEmptySlots( vector<int>& bulbs, int K )
{
    if( bulbs.size() < 2 )
    {
        return -1;
    }

    int day = -1;

    vector<int> D( bulbs.size() );

    int n_bulbs = static_cast< int >( bulbs.size() );

    //change to array D
    //where D[i]=x means
    //bulb at index (i) is on day (x)
    for( int i = 0; i < n_bulbs; ++i )
    {
        D[bulbs[i] - 1] = i + 1;
    }

    //starting from
    //the first candidate interval (0, k+1)
    int a = 0;
    int b = K + 1;

    while( b < n_bulbs )
    {
        bool flag = true;

        int i = a + 1;

        for( ; i < b; ++i )
        {
            if( ( D[i] > D[a] ) && ( D[i] > D[b] ) )
            {
                continue;
            }

            //bulb at index (i)
            //is on before bulb[a] and bulb[b]
            //break the search
            flag = false;
            break;
        }

        if( flag )
        {
            //we found a candidate interval
            //update the result as the minimum
            //so far
            if( day < 0 )
            {
                day = ( max )( D[a], D[b] );
            }
            else
            {
                day = ( min )( day, ( max )( D[a], D[b] ) );
            }

            //try next interval (b, b+K+1)
            a = b;
            b = a + K + 1;
        }
        else
        {
            //current interval (a,b)
            //cannot be a candidate
            //try next interval (i, i+K+1)
            a = i;
            b = a + K + 1;
        }
    }

    return day;
}
\end{lstlisting}