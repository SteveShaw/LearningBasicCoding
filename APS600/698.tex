\section{698 --- Partition to K Equal Sum Subsets}
Given an array of integers $A$ and a positive integer $k$, find whether it is possible to divide this array into $k$ non-empty subsets whose sums are all equal.
\paragraph{Example 1:}
\begin{flushleft}
\textbf{Input}:

$A$:  \fcc{[4, 3, 2, 3, 5, 2, 1]}, $k = 4$

\textbf{Output}: \fcc{true}

\textbf{Explanation}: It is possible to divide it into 4 subsets $(5)$, $(1, 4)$, $(2,3)$, $(2,3)$ with equal sums.

\end{flushleft}


\paragraph{Note:}

\begin{itemize}

\item $1 \leq k \leq |A| <= 16$
\item $0 < A[i] < 10000$
\end{itemize}

\subsection{Recursion}

This problem is almost same as \textbf{473 --- Matchsticks to Square}. Two approaches are available: recursion and dynamic programming with memorization.

In recursion approach, we make use of an array $S$ with size equal to $k$ where $S[i]$ is the accumulate sum of each possible set. In the recursion function, for current number $A[i]$, we test each $S[j]$ to see if $A[i]+S[j]\leq T$ where $T=\lvert A\rvert/k$. If we can, update $S[j]$ as $S[j] +A[i]$ and recursively from $A[i+1]$. When reach the end of $A$, we check if each $S[j]$ is equal to $T$. 

To speed up, we need to sort the input array by descending order.
\setcounter{lstlisting}{0}
\begin{lstlisting}[style=customc, caption={Recursion}]
bool canPartitionKSubsets( vector<int>& nums, int k )
{

    int total = accumulate( begin( nums ), end( nums ), 0 );

    if( total % k )
    {
        return false;
    }

    int subset_sum = total / k ;

    //speed up: sort by descending order
    sort( begin( nums ), end( nums ), greater<int>() );

    vector<int> S( k, 0 );

    return dfs( nums, 0, subset_sum, S );
}

bool dfs( vector<int>& A, size_t start, int T, vector<int>& S )
{
    if( start == A.size() )
    {
        bool flag = true;

        for( int sum : S )
        {
            if( sum != T )
            {
                //one subset sum
                //is not equal to target
                flag = false;
                break;
            }
        }

        return flag;
    }

    for( size_t i = 0; i < S.size(); ++i )
    {
        if( S[i] + A[start] <= T )
        {
            //increment S[i] by A[start]
            S[i] += A[start];

            if( dfs( A, start + 1, T, S ) )
            {
                return true;
            }

            //backtracking
            S[i] -= A[start];
        }
    }

    return false;
}
\end{lstlisting}

\subsection{Dynamic Programming With Memorization}
In this approach, we will use a integer, say, $U$, as a mask. if $A[i]$ is used, the $i$-th bit of $U$ is set. Same as problem 473, the key for the memorization is current mask along with number of completed set. Since $k$ is less than 16, we can use 20 bits of $U$ with high 16 bits to store the mask and low 4 bits to store the number of completed set.

In the recursion function, we track the number of completed set, say $C$. If $C=k-1$, remaining numbers will be sure form another subset with same total as the other subsets, so we return \fcc{true}. We check what we need to reach the subset sum $T = \lvert A\rvert / k $ and use this value to select remaining elements in $A$.  

In the same manner, we need to sort $A$ by descending order.

\begin{lstlisting}[style=customc, caption={DP With Memorization}]
bool canPartitionKSubsets( vector<int>& nums, int k )
{

    int sum = accumulate( begin( nums ), end( nums ), 0 );

    if( sum % k )
    {
        return false;
    }

    unordered_map<int, unsigned char> memo;

    //to speed up
    sort( begin( nums ), end( nums ), greater<int>() );

    int set_sum = sum / k;

    return dfs( nums, k, set_sum, 0, 0, 0, memo );
}

bool dfs( vector<int>& A, int k, int set_sum, int mask, int total, int comp_set, unordered_map<int, unsigned char>& memo )
{
    //form the key
    int key = mask;
    key <<=  4;
    key |=  comp_set;

    auto it = memo.find( key );

    if( it != memo.end() )
    {
        return it->second == 1;
    }

    //get number of complete set
    if( ( total > 0 ) && ( ( total % set_sum ) == 0 ) )
    {
        ++comp_set;
    }

    if( comp_set == k - 1 )
    {
        //we find k subsets with equal sum
        memo.emplace( key, 1 );
        return true;
    }

    //how much we need to form a required subset sum
    int needs = ( total / set_sum + 1 ) * set_sum  - total;

    for( size_t i = 0; i < A.size(); ++i )
    {
        if( ( 1 << i ) & mask )
        {
            //A[i] is selected
            //continue to next one
            continue;
        }

        if( A[i] > needs )
        {
            //We cannot use A[i]
            //as it is larger than what we need
            continue;
        }

        if( dfs( A, k, set_sum, ( 1 << i ) | mask, total + A[i], comp_set, memo ) )
        {
            //success
            //directly return here
            memo.emplace( key, 1 );
            return true;
        }

    }

    memo.emplace( key, 0 );
    return false;

}
\end{lstlisting}