\section{664 --- Strange Printer}
There is a strange printer with the following two special requirements:

\begin{enumerate}
\item The printer can only print a sequence of the same character each time.
\item At each turn, the printer can print new characters starting from and ending at any places, and will cover the original existing characters.
\end{enumerate}

Given a string consists of lower English letters only, your job is to count the minimum number of turns the printer needed in order to print it.

\paragraph{Example 1:}

\begin{flushleft}


\textbf{Input}:  \lstinline[language=C++, basicstyle=\small\ttfamily, keywordstyle=\bfseries\color{green!40!black}]|"aaabbb"|

\textbf{Output}: 2

\textbf{Explanation}: Print  \lstinline[language=C++, basicstyle=\small\ttfamily, keywordstyle=\bfseries\color{green!40!black}]|"aaa"| first and then print  \lstinline[language=C++, basicstyle=\small\ttfamily, keywordstyle=\bfseries\color{green!40!black}]|"bbb"|.

\end{flushleft}

\paragraph{Example 2:}

\begin{flushleft}

\textbf{Input}:  \lstinline[language=C++, basicstyle=\small\ttfamily, keywordstyle=\bfseries\color{green!40!black}]|"aba"|

\textbf{Output}: 2

\textbf{Explanation}: Print  \lstinline[language=C++, basicstyle=\small\ttfamily, keywordstyle=\bfseries\color{green!40!black}]|"aaa"| first and then print  \lstinline[language=C++, basicstyle=\small\ttfamily, keywordstyle=\bfseries\color{green!40!black}]|"b"| from the second place of the string, which will cover the existing character  \lstinline[language=C++, basicstyle=\small\ttfamily, keywordstyle=\bfseries\color{green!40!black}]|'a'|.

\end{flushleft}

\paragraph{Hint:} 
\begin{itemize}
\item Length of the given string will not exceed 100.
\end{itemize}

\subsection{Dynamic Programming}
Suppose $F[i][j]$ is the minimum number of turns the printers needs in order to print $S[i, j]$. 

The base cases are direct

\begin{enumerate}
\item $i=j$: $F[i][i]:=1$
\item $j=i+1$: $F[i][i+1]$ depends on if $S[i]=S[i+1]$.When $S[i]=S[i+1]$, $F[i][i+1]:=1$. Otherwise, $F[i][i+1]:=2$.
\end{enumerate}

Next, we iterate the sub-string length from $3$ to $\ell$. $\ell$ is the length of $s$. For any sub-string $S[i,j]$, the initial value of $F[i][j]$ can be set to $1+F[i+1][j]$ since we can print $S[i]$ first and then print $s[i+1,j]$.

Then, we check each index $k$ from $i+1$ to $j$. The total turns for printing $s[i,j]$ will be $F[i,k-1]+F[k, j]$. However, if $s[i]=s[k]$, we don't need to print $s[k]$ separately. We can print $s[i]$ with $s[k]$ in one turn. Thus the total turns can be decreased by one time, i.e. $F[i,k-1]+F[k, j]-1$.

We choose the minimum turns from above as the result for $F[i][j]$. Finally, $F[0][\ell-1]$ gives the result for $s$.