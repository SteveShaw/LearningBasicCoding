\section{644 --- Maximum Average Subarray II}
Given an array consisting of $n$ integers, $A$, find the contiguous subarray whose length is greater than or equal to $k$ that has the maximum average value. And you need to output the maximum average value.

\paragraph{Example 1:}

\begin{flushleft}
\textbf{Input}: $[1,12,-5,-6,50,3]$, $k = 4$

\textbf{Output}: 12.75

\textbf{Explanation}:

when length is 5, maximum average value is 10.8,

when length is 6, maximum average value is 9.16667.

Thus return 12.75.
\end{flushleft}

\paragraph{Note:}

\begin{itemize}
\item $1 \leq k \leq n \leq 10,000$.
\item Elements of the given array will be in range $[-10,000, 10,000]$.
\item The answer with the calculation error less than $10^{-5}$ will be accepted.
\end{itemize}

\subsection{Binary Search}
\begin{enumerate}
\item Notice that the value of the average could lie between a range $[\min(A),\max(A)]$. Here, $x$ and $y$ refer to the minimum and the maximum values out of the given array $A$. \item To find the maximum average of a subarray with at least $k$ elements, we can try to guess and test if a value close to the real maximum average. This will fits a Binary Search approach since we know the range of the real maximum average.
\item During the search, suppose $z$ is the middle value in current range $[x,y]$, i.e., $z=(x+y)/2$, then the task is to find if a subarray with length greater than or equal to $k$ is possible with an average greater than $z$.
\begin{itemize}
\item Now assume there is a subarray $A[i, i+m-1]$ (length is $m\geq k$) such that its average is greater than $z$, i.e, 

$$(A[i] + A[i+1] + \ldots +A[i+m-1])/ m \geq z$$

By transforming this inequality equation, we have 
\begin{align*}
A[i] + A[i+1] + \ldots +A[i+m-1] & \geq m\times z \\
(A[i] - z) + (A[i+1]-z) + \ldots + (A[i+m-1]-z) &\geq 0
\end{align*} 

Thus, for any subarray with length no less than $k$, if subtracting $z$ from each element inside and the summation after is still no less than zero, this means the average value of this subarray is greater than the middle value $z$. Hence, we need to search in the upper half of current range. Otherwise, we will find in the lower hal.
\end{itemize}

\end{enumerate}

In order to determine if such a subarray exists in a linear manner, we maintain a variable, say $\alpha$, to store the summation of $A[i]-z$ until the $i$-th element while traversing over $A$. 

If $(A[0]-z) +\ldots +(A[k-1]-z)\geq 0$, we know the average can be increased over $z$, thus we can directly search in upper half range. Otherwise, we continue from  $A[k]$ until last element, by making use of the following idea.

\begin{itemize}
\item If we know the summation of $A[i]-z$ and $A[j]-z$, say $\alpha_i$ and $\alpha_j$, then we have $(A[i+1]-z) + (A[i+2]-z)+\ldots+(A[j]-z) = \alpha_j - \alpha_i$. Thus we need to find if $\alpha_j - \alpha_i\geq 0 $ for $j-i \geq k$.
\item Instead of checking all possible values of $\alpha$ for each index, we just consider the minimum $\alpha$ until the index $j-k$. 
\item So, we maintain a variable $\beta$ to store the cumulative summation of $A[i-k]-z$ for each index $i$, and we also record the minimum of $\beta$.
\end{itemize}

There is one more thing to note: in this binary search, we are comparing \lstinline[language=Java, basicstyle=\small\ttfamily, keywordstyle=\bfseries\color{green!40!black}]|double| type. Thus, we need to compare the difference with some predefined precision. The problem requires less than $10^{-5}$, so the process will stop when we get error below than $0.00001$.
