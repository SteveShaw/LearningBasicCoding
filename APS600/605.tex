\section{605 --- Can Place Flowers}
Suppose you have a long flowerbed in which some of the plots are planted and some are not. However, flowers cannot be planted in adjacent plots -- they would compete for water and both would die.

Given a flowerbed (represented as an array, $A$, containing $0$ and $1$, where $0$ means empty and $1$ means not empty), and a number $n$, return if $n$ new flowers can be planted in it without violating the \textbf{no-adjacent-flower}s rule.

\paragraph{Example 1:}

\begin{flushleft}

\textbf{Input}: 

flowerbed: $[1,0,0,0,1]$, $n = 1$

\textbf{Output}: \texttt{True}

\end{flushleft}

\paragraph{Example 2:}

\begin{flushleft}
\textbf{Input}: 

flowerbed: $[1,0,0,0,1]$, $n = 2$

\textbf{Output}: \texttt{False}

\end{flushleft}

\paragraph{Note:}

\begin{itemize}
\item The input array won't violate \textbf{no-adjacent-flowers} rule.
\item The input array size is in the range of $[1, 20000]$.
\item $n$ is a non-negative integer which won't exceed the input array size.
\end{itemize}

\subsection{Simulation}
\begin{itemize}
    \item Between two 1s, suppose the number of zeros is $z$. We cannot plant at the place immediately right to the first 1 and immediately left to the right 1. Therefore, the total zeros that can be used to plant is $z-2$. 
    \item However, we have to plant every two zeros to avoid adjacent flowers. Therefore, we can plant at most $(z-2+1)/2$.
    \item However, if a plant is not plant at the beginning, and $z$ is the number of zeros from index 0 to the first 1 we meet. In this case, we cannot plant immediately to the position of this 1. We can plant at mose $(z-1+1)/2 = z/2$
\end{itemize}

Based on the above discussion, the algorithm steps include
\begin{enumerate}
    \item Iterate the input array, 
\end{enumerate}