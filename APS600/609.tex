\section{609 --- Find Duplicate File in System}
Given a list of directory info including directory path, and all the files with contents in this directory, you need to find out all the groups of duplicate files in the file system in terms of their paths.

A group of duplicate files consists of at least two files that have exactly the same content.

A single directory info string in the input list has the following format:

\path{root/d1/d2/.../dm f1.txt(f1_content) f2.txt(f2_content) ... fn.txt(fn_content)}

It means there are $n$ files (\path{f1.txt}, \path{f2.txt}, \textellipsis, \path{fn.txt} with content f1\_content, f2\_content \textellipsis fn\_content, respectively) in directory \path{root/d1/d2/.../dm}. Note that $n \geq 1$ and $m \geq 0$. If $m = 0$, it means the directory is just the root directory.

The output is a list of group of duplicate file paths. For each group, it contains all the file paths of the files that have the same content. A file path is a string that has the following format:

\path{directory_path/file_name.txt}

\paragraph{Example 1:}

\textbf{Input}:

[\path{root/a} \path{1.txt}(abcd) \path{2.txt}(efgh), \path{root/c} \path{3.txt}(abcd), \path{root/c/d} \path{4.txt}(efgh), \path{root} \path{4.txt}(efgh)]

\textbf{Output}:
  
[[\path{root/a/2.txt}, \path{root/c/d/4.txt}, \path{root/4.txt}],[\path{root/a/1.txt}, \path{root/c/3.txt}]]

 

\paragraph{Note:}

\begin{itemize}
\item No order is required for the final output.
\item You may assume the directory name, file name and file content only has letters and digits, and the length of file content is in the range of $[1,50]$.
\item The number of files given is in the range of $[1,20000]$.
\item You may assume no files or directories share the same name in the same directory.
\item You may assume each given directory info represents a unique directory. Directory path and file info are separated by a single blank space.
\end{itemize}

 
\paragraph{Follow-up:}

\begin{itemize}
\item Imagine you are given a real file system, how will you search files? DFS or BFS?
\item If the file content is very large (GB level), how will you modify your solution?
\item If you can only read the file by 1kb each time, how will you modify your solution?
\item What is the time complexity of your modified solution? What is the most time-consuming part and memory consuming part of it? How to optimize?
\item How to make sure the duplicated files you find are not false positive?
\end{itemize}

\subsection{Hash Map}
\begin{itemize}
    \item Get directory, file name and content from the input strings.
    \item Group files per their contents.
\end{itemize}

\subsection{Follow Up Questions:}
\begin{enumerate}
    \item If depth of directory is not too deeper, which is suitable to use DFS, comparing with BFS.
    \item core idea: make use of meta data, like file size before really reading large content. 
    \begin{itemize}
        \item  DFS to map each size to a set of paths that have that size: Map<Integer, Set>
    \item For each size, if there are more than 2 files there, compute hashCode of every file by MD5, if any files with the same size have the same hash, then they are identical files: Map<String, Set>, mapping each hash to the Set of filepaths+filenames. 
    \end{itemize}
\end{enumerate}
