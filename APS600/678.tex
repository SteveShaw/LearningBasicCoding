\section{678 --- Valid Parenthesis String}
Given a string, $s$, containing only three types of characters: \lstinline[language=Java, basicstyle=\small\ttfamily, keywordstyle=\bfseries\color{green!40!black}]|'('|, \lstinline[language=Java, basicstyle=\small\ttfamily, keywordstyle=\bfseries\color{green!40!black}]|')'| and \lstinline[language=Java, basicstyle=\small\ttfamily, keywordstyle=\bfseries\color{green!40!black}]|'*'|, write a function to check whether this string is valid. We define the validity of a string by these rules:

\begin{enumerate}
\item Any left parenthesis \lstinline[language=Java, basicstyle=\small\ttfamily, keywordstyle=\bfseries\color{green!40!black}]|'('| must have a corresponding right parenthesis \lstinline[language=Java, basicstyle=\small\ttfamily, keywordstyle=\bfseries\color{green!40!black}]|')'|.
\item Any right parenthesis \lstinline[language=Java, basicstyle=\small\ttfamily, keywordstyle=\bfseries\color{green!40!black}]|')'| must have a corresponding left parenthesis \lstinline[language=Java, basicstyle=\small\ttfamily, keywordstyle=\bfseries\color{green!40!black}]|'('|.
\item Left parenthesis \lstinline[language=Java, basicstyle=\small\ttfamily, keywordstyle=\bfseries\color{green!40!black}]|'('| must go before the corresponding right parenthesis \lstinline[language=Java, basicstyle=\small\ttfamily, keywordstyle=\bfseries\color{green!40!black}]|')'|.
\item \lstinline[language=Java, basicstyle=\small\ttfamily, keywordstyle=\bfseries\color{green!40!black}]|'*'| could be treated as a single right parenthesis \lstinline[language=Java, basicstyle=\small\ttfamily, keywordstyle=\bfseries\color{green!40!black}]|')'| or a single left parenthesis \lstinline[language=Java, basicstyle=\small\ttfamily, keywordstyle=\bfseries\color{green!40!black}]|'('| or an empty string.
\item An empty string is also valid.
\end{enumerate}

\paragraph{Example 1:}

\begin{flushleft}
\textbf{Input}: \lstinline[language=Java, basicstyle=\small\ttfamily, keywordstyle=\bfseries\color{green!40!black}]|"()"|

\textbf{Output}: \lstinline[language=Java, basicstyle=\small\ttfamily, keywordstyle=\bfseries\color{green!40!black}]|True|

\end{flushleft}

\paragraph{Example 2:}

\begin{flushleft}
\textbf{Input}: \lstinline[language=Java, basicstyle=\small\ttfamily, keywordstyle=\bfseries\color{green!40!black}]|"(*)"|

\textbf{Output}: \lstinline[language=Java, basicstyle=\small\ttfamily, keywordstyle=\bfseries\color{green!40!black}]|True|
\end{flushleft}

\paragraph{Example 3:}

\begin{flushleft}
\textbf{Input}: \lstinline[language=Java, basicstyle=\small\ttfamily, keywordstyle=\bfseries\color{green!40!black}]|"(*))"|

\textbf{Output}: \lstinline[language=Java, basicstyle=\small\ttfamily, keywordstyle=\bfseries\color{green!40!black}]|True|
\end{flushleft}

\paragraph{Note:}

\begin{itemize}
\item The string size will be in the range \lstinline[language=Java, basicstyle=\small\ttfamily, keywordstyle=\bfseries\color{green!40!black}]|[1, 100]|.
\end{itemize}


\subsection{Greedy}
When checking whether the string is valid, we only cared about the number of extra open left parenthesis as parsing through the string. 

For example, when checking whether \lstinline[language=Java, basicstyle=\small\ttfamily, keywordstyle=\bfseries\color{green!40!black}]|'(()())'| is valid, we had a number of extra left parenthesis of \lstinline[language=Java, basicstyle=\small\ttfamily, keywordstyle=\bfseries\color{green!40!black}]|[1, 2, 1, 2, 1, 0]| as we parse through the string: \lstinline[language=Java, basicstyle=\small\ttfamily, keywordstyle=\bfseries\color{green!40!black}]|(| has 1 left bracket, \lstinline[language=Java, basicstyle=\small\ttfamily, keywordstyle=\bfseries\color{green!40!black}]|'(('| has 2, \lstinline[language=Java, basicstyle=\small\ttfamily, keywordstyle=\bfseries\color{green!40!black}]|'(()'| has 1, and so on. This means that after parsing the first $x$ symbols, (which may include asterisks,) we only need to keep track of how many there could have extra left parenthesis.

For example, if we have string \lstinline[language=Java, basicstyle=\small\ttfamily, keywordstyle=\bfseries\color{green!40!black}]|'(***)'|, then as we parse each symbol, the set of possible values for the balance is
\begin{itemize}
\item \lstinline[language=Java, basicstyle=\small\ttfamily, keywordstyle=\bfseries\color{green!40!black}]|[1]| for \lstinline[language=Java, basicstyle=\small\ttfamily, keywordstyle=\bfseries\color{green!40!black}]|'('|;
\item \lstinline[language=Java, basicstyle=\small\ttfamily, keywordstyle=\bfseries\color{green!40!black}]|[0, 1, 2]| for \lstinline[language=Java, basicstyle=\small\ttfamily, keywordstyle=\bfseries\color{green!40!black}]|'(*'|;
\item \lstinline[language=Java, basicstyle=\small\ttfamily, keywordstyle=\bfseries\color{green!40!black}]|[0, 1, 2, 3]| for \lstinline[language=Java, basicstyle=\small\ttfamily, keywordstyle=\bfseries\color{green!40!black}]|'(**'|;
\item \lstinline[language=Java, basicstyle=\small\ttfamily, keywordstyle=\bfseries\color{green!40!black}]|[0, 1, 2, 3, 4]| for \lstinline[language=Java, basicstyle=\small\ttfamily, keywordstyle=\bfseries\color{green!40!black}]|'(***'|, 
\item and \lstinline[language=Java, basicstyle=\small\ttfamily, keywordstyle=\bfseries\color{green!40!black}]|[0, 1, 2, 3]| for \lstinline[language=Java, basicstyle=\small\ttfamily, keywordstyle=\bfseries\color{green!40!black}]|'(***)'|.

\end{itemize}
Furthermore, we can prove these states always form a contiguous interval. Thus, we only need to know the left and right bounds of this interval. That is, we would keep those intermediate states described above as 

\lstinline[language=Java, basicstyle=\small\ttfamily, keywordstyle=\bfseries\color{green!40!black}]|[lo, hi] = [1, 1], [0, 2], [0, 3], [0, 4], [0, 3]|.

\paragraph{Algorithm}

We maintain two varaibles $l$ and $h$ where 

\begin{enumerate}
\item $l$ 表示在有左括号的情况下,将星号当作右括号时左括号的个数(这样做的原因是尽量不多增加右括号的个数)
\item $h$ 表示将星号当作左括号时左括号的个数
\end{enumerate}

When iterating over the string
\begin{itemize}
\item 当遇到左括号时,$l\gets l+1$ and $h\gets h+1$
\item 当遇到右括号时,$l\gets l-1$ only when $l >0$ to make sure $l\geq 0$, and $h\gets h-1$
\item 当遇到星号时,$l\gets l-1$ only when $l >0$ to make sure $l\geq 0$, and $h\gets h+1$ because we treat asterisk as left parenthesis.
\item When $h<0$, 说明就算把星号都当作左括号了,还是不够抵消右括号,直接返回\lstinline[language=C++, basicstyle=\small\ttfamily, keywordstyle=\bfseries\color{green!40!black}]|false|.
\end{itemize}

At the end, we check if $l=0$.

\setcounter{lstlisting}{0}
\begin{lstlisting}[style=customc, caption={Greedy}]
bool checkValidString( string s )
{
    //the number of actual parenthesis
    int l = 0;

    //the number of left parenthesis
    //including asterisks treated as left parenthesis
    int h = 0;

    for( auto c : s )
    {
        if( c == '(' )
        {
            ++l;
            ++h;
        }
        else if( c == ')' )
        {
            //make sure l always
            //no less than 0
            if( l > 0 )
            {
                --l;
            }

            --h;
        }
        else
        {
            if( l > 0 )
            {
                --l;
            }

            //we can treat asterisk
            //as asterisk
            ++h;
        }

        if( h < 0 )
        {
            //too many right parenthesis
            //even if we treat asterisks as
            //left parenthesis
            return false;
        }
    }

    //only valid when
    //actual left parenthesis
    //are complete offset
    //by right parenthesis and
    //asterisk
    return l == 0;
}
\end{lstlisting}

\subsection{Two Stacks}
We can make use of two stacks, $\alpha$ and $\beta$ to track left parenthesis and asterisks. Since the positions are critical, we push each index of left parenthesis and asterisk into corresponding stack.

While iterating 
\begin{enumerate}
\item If it is \lstinline[language=C++, basicstyle=\small\ttfamily, keywordstyle=\bfseries\color{green!40!black}]|')'|, push the index into $\alpha$
\item If it is \lstinline[language=C++, basicstyle=\small\ttfamily, keywordstyle=\bfseries\color{green!40!black}]|'*'|, push the index into $\beta$
\item Otherwise, it is \lstinline[language=C++, basicstyle=\small\ttfamily, keywordstyle=\bfseries\color{green!40!black}]|')'|, we choose to pop from $\alpha$. If $\alpha$ is empty, pop from $\beta$. If both stacks are empty, the expression contains too many \lstinline[language=C++, basicstyle=\small\ttfamily, keywordstyle=\bfseries\color{green!40!black}]|')'|, just return \lstinline[language=C++, basicstyle=\small\ttfamily, keywordstyle=\bfseries\color{green!40!black}]|false|.
\end{enumerate}

After iterations, we may have remaining unmatched \lstinline[language=C++, basicstyle=\small\ttfamily, keywordstyle=\bfseries\color{green!40!black}]|')'|. We need to check if they can be matched by \lstinline[language=C++, basicstyle=\small\ttfamily, keywordstyle=\bfseries\color{green!40!black}]|'*'|. Thus, we iterate $\alpha$ and $\beta$. If we find that a asterisk appears before a left parenthesis, the expression is invalid. 

At the end, check if $\alpha$ is empty. 

\begin{lstlisting}[style=customc, caption={Two Stacks}]
bool checkValidString( string s )
{
    //record index of '('
    stack<size_t> left;
    //record index of '*'
    stack<size_t> ast;

    for( size_t i = 0; i < s.size(); ++i )
    {
        if( s[i] == '(' )
        {
            left.push( i );
        }
        else if( s[i] == '*' )
        {
            ast.push( i );
        }
        else
        {
            if( !left.empty() )
            {
                //match a '('
                left.pop();
            }
            else if( !ast.empty() )
            {
                //match a '*'
                ast.pop();
            }
            else
            {
                //too many ')'
                return false;
            }
        }
    }

    while( !left.empty() && !ast.empty() )
    {
        if( left.top() < ast.top() )
        {
            //match a '(' with '*'
            left.pop();
            ast.pop();
        }
        else
        {
            //'(' cannot be matched
            return false;
        }
    }

    //check if all '(' are matched
    return left.empty();
}
\end{lstlisting}

\subsection{Scan Twice}
We scan the input string $s$ twice

When scan from left to right, we treat \lstinline[language=C++, basicstyle=\small\ttfamily, keywordstyle=\bfseries\color{green!40!black}]|'*'| as \lstinline[language=C++, basicstyle=\small\ttfamily, keywordstyle=\bfseries\color{green!40!black}]|'('|. We make use of a counter $\delta$ to count \lstinline[language=C++, basicstyle=\small\ttfamily, keywordstyle=\bfseries\color{green!40!black}]|'('|. For each \lstinline[language=C++, basicstyle=\small\ttfamily, keywordstyle=\bfseries\color{green!40!black}]|'('| or \lstinline[language=C++, basicstyle=\small\ttfamily, keywordstyle=\bfseries\color{green!40!black}]|'*'|, increments $\delta$. Otherwise, $\delta\gets\delta-1$. Whenever $\delta < 0$, just return \lstinline[language=C++, basicstyle=\small\ttfamily, keywordstyle=\bfseries\color{green!40!black}]|false| because it is showing that there are excess \lstinline[language=C++, basicstyle=\small\ttfamily, keywordstyle=\bfseries\color{green!40!black}]|')'|.  When $\delta=0$, just return \lstinline[language=C++, basicstyle=\small\ttfamily, keywordstyle=\bfseries\color{green!40!black}]|true| because all opening and closing parenthesis are matched.

Otherwise, we scan $s$ again from right to left. At this time, we treat \lstinline[language=C++, basicstyle=\small\ttfamily, keywordstyle=\bfseries\color{green!40!black}]|'*'| as \lstinline[language=C++, basicstyle=\small\ttfamily, keywordstyle=\bfseries\color{green!40!black}]|')'|. We make use of another counter $x$ to count \lstinline[language=C++, basicstyle=\small\ttfamily, keywordstyle=\bfseries\color{green!40!black}]|')'|. For each \lstinline[language=C++, basicstyle=\small\ttfamily, keywordstyle=\bfseries\color{green!40!black}]|')'| or \lstinline[language=C++, basicstyle=\small\ttfamily, keywordstyle=\bfseries\color{green!40!black}]|'*'|, increments $x$. Otherwise, $x\gets x-1$. When $x<0$, just return \lstinline[language=C++, basicstyle=\small\ttfamily, keywordstyle=\bfseries\color{green!40!black}]|false| because $s$ contains excess \lstinline[language=C++, basicstyle=\small\ttfamily, keywordstyle=\bfseries\color{green!40!black}]|'('|. At the end, as long as $x\geq 0$, return \lstinline[language=C++, basicstyle=\small\ttfamily, keywordstyle=\bfseries\color{green!40!black}]|true|. $x=0$ is apparent. For $x>0$, since we treat \lstinline[language=C++, basicstyle=\small\ttfamily, keywordstyle=\bfseries\color{green!40!black}]|'*'| as \lstinline[language=C++, basicstyle=\small\ttfamily, keywordstyle=\bfseries\color{green!40!black}]|')'|, we can reverse some of asterisks to represent \lstinline[language=C++, basicstyle=\small\ttfamily, keywordstyle=\bfseries\color{green!40!black}]|'('|. If there are no enough \lstinline[language=C++, basicstyle=\small\ttfamily, keywordstyle=\bfseries\color{green!40!black}]|'*'| to match \lstinline[language=C++, basicstyle=\small\ttfamily, keywordstyle=\bfseries\color{green!40!black}]|')'|, it should already be detected by $x<0$.

\begin{lstlisting}[style=customc, caption={Scan Twice}]
bool checkValidString( string s )
{
    int x = 0;

    for( auto c : s )
    {
        if( ( c == '(' ) || ( c == '*' ) )
        {
            ++x;
        }
        else
        {
            --x;
        }

        if( x < 0 )
        {
            //there are excess '('
            return false;
        }
    }

    if( x == 0 )
    {
        //all '(' and ')' are matched
        return true;
    }

    x = 0;

    for( size_t i = s.size(); i >= 1; --i )
    {
        auto c = s[i - 1];

        if( ( c == ')' ) || ( c == '*' ) )
        {
            ++x;
        }
        else
        {
            --x;
        }

        if( x < 0 )
        {
            //there are excess '('
            return false;
        }
    }
    //all can be matched.
    return true;
}
\end{lstlisting}