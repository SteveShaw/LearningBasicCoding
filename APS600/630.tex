\section{630 --- Course Schedule III}
There are $n$ different online courses numbered from 1 to $n$. Each course has some duration (course length) $t$ and closed on $d$-th day. A course should be taken continuously for $t$ days and must be finished before or on the $d$-th day. You will start at the 1st day.

Given $n$ online courses represented by pairs $(t,d)$, your task is to find the maximal number of courses that can be taken.

\paragraph{Example:}
\begin{flushleft}

\textbf{Input}: $[[100, 200], [200, 1300], [1000, 1250], [2000, 3200]]$

\textbf{Output}: 3

\textbf{Explanation}: 

There are totally 4 courses, but you can take 3 courses at most:

First, take the 1st course, it costs 100 days so you will finish it on the 100th day, and ready to take the next course on the 101st day.

Second, take the 3rd course, it costs 1000 days so you will finish it on the 1100th day, and ready to take the next course on the 1101st day. 

Third, take the 2nd course, it costs 200 days so you will finish it on the 1300th day.
 
The 4th course cannot be taken now, since you will finish it on the 3300th day, which exceeds the closed date.

\end{flushleft}

\paragraph{Note:}

\begin{itemize}
\item The integer $1 \leq d, t, n \leq 10000$.
\item You can't take two courses simultaneously.
\end{itemize}

\subsection{Greedy}
\begin{itemize}
\item Sort the input per the end time of the course.
\item Maintain a variable $t$ to indicate current start time.
\item Go over the sorted array, if current course $c$ can be taken, i.e. $t + d_c \leq e_c$ ($d_c$ is the duration of course $c$ and $e_c$ is the day that $c$ has to be finished). We add $c$'s duration into a maximum priority queue (i.e., the top of the queue is the maximum one). 
\item If current course $c$ cannot be taken, we need to replace the largest duration from the priority queue with current course. At the same time, we have adjust start time $t$ by subtracting the largest duration from the queue because we don't take this lesson and adding $c$'s duration to $t$. Also, we need to push $c$'s duration into the queue.
\item Finally, the number of items in the queue gives us the answer.
\end{itemize}