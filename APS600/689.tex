\section{689 --- Maximum Sum of 3 Non-Overlapping Subarrays}
In a given array $A$ of positive integers, find three non-overlapping subarrays with maximum sum.

Each subarray will be of size $k$, and we want to maximize the sum of all $3k$ entries.

Return the result as a list of indices representing the starting position of each interval (0-indexed). If there are multiple answers, return the lexicographically smallest one.

\paragraph{Example:}

\begin{flushleft}
\textbf{Input}: \lstinline[language=C++, basicstyle=\small\ttfamily, keywordstyle=\bfseries\color{green!40!black}]|[1,2,1,2,6,7,5,1]|,  $k=2$

\textbf{Output}: \lstinline[language=C++, basicstyle=\small\ttfamily, keywordstyle=\bfseries\color{green!40!black}]|[0, 3, 5]|

Explanation: Subarrays \lstinline[language=C++, basicstyle=\small\ttfamily, keywordstyle=\bfseries\color{green!40!black}]|[1, 2], [2, 6], [7, 5]| correspond to the starting indices \lstinline[language=C++, basicstyle=\small\ttfamily, keywordstyle=\bfseries\color{green!40!black}]|[0, 3, 5]|.

We could have also taken \lstinline[language=C++, basicstyle=\small\ttfamily, keywordstyle=\bfseries\color{green!40!black}]|[2, 1]|, but an answer of \lstinline[language=C++, basicstyle=\small\ttfamily, keywordstyle=\bfseries\color{green!40!black}]|[1, 3, 5]| would be lexicographically larger.
\end{flushleft}
 

\paragraph{Note:}

\begin{itemize}
\item Size of $A$ will be between 1 and 20000.
\item $A[i]$ will be between 1 and 65535.
\item $k$ will be between 1 and $\lfloor \lvert A\rvert\ /3 \rfloor$.
\end{itemize}

\subsection{Dynamic Programming}
Suppose we have an array $W$, where each $W[s]$ represents sum of interval $A[s, s+K-1]$. To create $W$, we can either use prefix sums, or manage the sum of the interval as a window slides along the array.

Then, we have a reduced problem: Given array $W$ and an integer $K$, what is the lexicographically smallest tuple of indices $(x, y, z)$ with $x + K <\leq y$ and $y + K \leq z$ that maximizes $W[x] + W[y] + W[z]$ ?

Suppose we fixed $y$. We would like to know on the intervals $x \in [0, y-K]$ and $z \in [j+K, \lvert W\rvert-1$, where the largest value of $W[x]$ (and respectively $W[z]$) occurs first. (Here, \textbf{first} means the smaller index.)

We can solve these problems with dynamic programming. We make use of two arrays $B$ and $E$ where

\begin{enumerate}
\item Each $B[i]$ stores the first index of the largest of $W[x]$ for $x \in [0, i]$, and 
\item Each $E[j]$ stores the first index of the largest of $W[z]$ for $z \in [j, \lvert W\rvert - 1]$.
\end{enumerate}

This means that for some choice $y$, the candidate must be $(B[y-K], y, E[j+K])$. We take the candidate that produces the maximum $W[x] + W[y] + W[z]$.

\setcounter{lstlisting}{0}
\begin{lstlisting}[style=customc, caption={Dynamic Programming}]
vector<int> maxSumOfThreeSubarrays( vector<int>& nums, int k )
{
    //fill S that record each interval's sum
    int N = static_cast<int>( nums.size() );

    vector<int> S( N - k + 1, 0 );

    int sum = 0;

    for( int i = 0; i < N; ++i )
    {
        sum += nums[i];

        if( i >= k )
        {
            sum -= nums[i - k];
        }

        if( i >= k - 1 )
        {
            S[i - k + 1] = sum;
        }
    }

    //make array B
    //which records the first index
    //of maximum S from left to right
    int nS = N - k + 1;
    int best = 0;
    vector<int> B( nS );
    for( int x = 0; x < nS; ++x )
    {
        if( S[x] > S[best] )
        {
            best = x;
        }

        B[x] = best;
    }

    //make array E
    //which record the first index
    //of maximum S from right to left

    best = nS - 1;
    vector<int> E( nS );
    for( int z = nS - 1; z >= 0; --z )
    {
        //since this is from right to left
        //and we need first occurence
        //we also need update when the S[z] = S[best]
        if( S[z] >= S[best] )
        {
            best = z;
        }

        E[z] = best;
    }

    //find out best tuple (x,y,z) which can make
    //largest S[x]+S[y]+S[z]

    vector<int> ans {0, k, k + k};
    int max3S = S[ans[0]] + S[ans[1]] + S[ans[2]];
    for( int y = k; y < nS - k; ++y )
    {
        int x = B[y - k]; //x <=y-k
        int z = E[y + k]; //z >= y+k

        int total = S[x] + S[y] + S[z];

        if( total > max3S )
        {
            ans.assign( {x, y, z} );
            max3S = total;
        }
    }

    return ans;
}
\end{lstlisting}