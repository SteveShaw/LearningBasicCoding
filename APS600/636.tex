\section{636 --- Exclusive Time of Functions}
On a single threaded CPU, we execute some functions.  Each function has a unique id between 0 and $N-1$.

We store logs in timestamp order that describe when a function is entered or exited.

Each log is a string with this format: \lstinline[language=Java, basicstyle=\small\ttfamily, keywordstyle=\bfseries\color{green!40!black}]|{function_id}:{["start","end"]}:{timestamp}|.  

For example, \texttt{0:start:3} means the function with id 0 started at the beginning of timestamp 3.  \texttt{1:end:2} means the function with id 1 ended at the end of timestamp 2.

A function's exclusive time is the number of units of time spent in this function.  Note that this does not include any recursive calls to child functions.

The CPU is single threaded which means that only one function is being executed at a given time unit.

Return the exclusive time of each function, sorted by their function id.
 

\paragraph{Example 1:}

\begin{flushleft}
\textbf{Input}: $n = 2$, \texttt{logs}: [\texttt{0:start:0}, \texttt{1:start:2}, \texttt{1:end:5}, \texttt{0:end:6}]

\textbf{Output}: $[3, 4]$

\textbf{Explanation}:

Function 0 starts at the beginning of time 0, then it executes 2 units of time and reaches the end of time 1.

Now function 1 starts at the beginning of time 2, executes 4 units of time and ends at time 5.

Function 0 is running again at the beginning of time 6, and also ends at the end of time 6, thus executing for 1 unit of time. 

So function 0 spends $2 + 1 = 3$ units of total time executing, and function 1 spends 4 units of total time executing.


\end{flushleft} 

\paragraph{Note:}

\begin{itemize}
\item $1 \leq n \leq 100$
\item Two functions won't start or end at the same time.
\item Functions will always log when they exit.
\end{itemize}

\subsection{Stack}
First, we need to understand the meaning of this problem:

\begin{itemize}
\item We are given a list of string, $A$, which depict how $n$ functions are executed in a single thread CPU.
\item A function $f_1$ could call function $f_2$, and $f_2$ may call another function $f_3$. As a result, $f_1$ starts first and end third, $f_2$ starts second and end second, and $f_3$ starts third, and end first. In the string, we may see start and end time of $f_2$ and $f_3$ before $f_1$'s end time.
\item This perfectly fits a stack property.
\end{itemize}

Based on the analysis, we make use of a stack to track the functions calling path.
\begin{itemize}
\item At start, we push the first function's id into the stack.
\item Proceed to the next log,
\begin{enumerate}
\item If the next function log a start label, we push this function id on the top of the stack, since the last function would need to be revisited again in the future. 
\item On the other hand, if the next function log an end label, it means the last function on the top of the stack is terminating.
\end{enumerate}
\item We need to record last function's start or end time.
\begin{itemize}
\item If the next function log a start label, we update the running time of the function at the top of the stack as current time minus last recorded time since the function is running from last recorded time to one time unit before current time.
\item Otherwise, the function at the top of the stack is finished , we update the running time of this function by adding one plus the difference between current time and last recorded time. The reason to add one is that this function is running from last recorded time to current time.
\end{itemize}
\end{itemize}