\section{642 --- Design Search Autocomplete System}
Design a search autocomplete system for a search engine. Users may input a sentence (at least one word and end with a special character \#). For \textbf{each character} they type \textbf{except} \#, you need to return the \textbf{top 3} historical hot sentences that have prefix the same as the part of sentence already typed. Here are the specific rules:

\begin{enumerate}
\item The hot degree for a sentence is defined as the number of times a user typed the exactly same sentence before.
\item The returned top 3 hot sentences should be sorted by hot degree (The first is the hottest one). If several sentences have the same degree of hot, you need to use ASCII-code order (smaller one appears first).
\item If less than 3 hot sentences exist, then just return as many as you can.
\item When the input is a special character, it means the sentence ends, and in this case, you need to return an empty list.
\end{enumerate}

Your job is to implement the following functions:

The constructor function:

\lstinline[language=Java, basicstyle=\small\ttfamily, keywordstyle=\bfseries\color{green!40!black}]|AutocompleteSystem(String[] sentences, int[] times)|: This is the constructor. The input is \textbf{historical data}. \texttt{Sentences} is a string array consists of previously typed sentences. \texttt{Times} is the corresponding times a sentence has been typed. Your system should record these historical data.

Now, the user wants to input a new sentence. The following function will provide the next character the user types:

\lstinline[language=Java, basicstyle=\small\ttfamily, keywordstyle=\bfseries\color{green!40!black}]|List<String> input(char c)|: The input \texttt{c} is the next character typed by the user. The character will only be lower-case letters (\texttt{a} to \texttt{z}), blank space or a special character (\#). Also, the previously typed sentence should be recorded in your system. The output will be the \textbf{top 3} historical hot sentences that have prefix the same as the part of sentence already typed.
 

\paragraph{Example:}

\begin{flushleft}

\textbf{Operation}:
 \lstinline[language=Java, basicstyle=\small\ttfamily, keywordstyle=\bfseries\color{green!40!black}]|AutocompleteSystem(["i love you", "island", "ironman", "i love leetcode"],[5, 3, 2, 2])|

The system have already tracked down the following sentences and their corresponding times:

\lstinline[language=Java, basicstyle=\small\ttfamily, keywordstyle=\bfseries\color{green!40!black}]|"i love you"| : 5 times

\lstinline[language=Java, basicstyle=\small\ttfamily, keywordstyle=\bfseries\color{green!40!black}]|"island"| : 3 times

\lstinline[language=Java, basicstyle=\small\ttfamily, keywordstyle=\bfseries\color{green!40!black}]|"ironman"| : 2 times

\lstinline[language=Java, basicstyle=\small\ttfamily, keywordstyle=\bfseries\color{green!40!black}]|"i love leetcode"| : 2 times

Now, the user begins another search:

\textbf{Operation}: \lstinline[language=Java, basicstyle=\small\ttfamily, keywordstyle=\bfseries\color{green!40!black}]|input('i')|

\textbf{Output}: \lstinline[language=Java, basicstyle=\small\ttfamily, keywordstyle=\bfseries\color{green!40!black}]|["i love you", "island", "i love leetcode"]|

\textbf{Explanation}:

There are four sentences that have prefix \texttt{i} . Among them, \lstinline[language=Java, basicstyle=\small\ttfamily, keywordstyle=\bfseries\color{green!40!black}]|"ironman"| and \lstinline[language=Java, basicstyle=\small\ttfamily, keywordstyle=\bfseries\color{green!40!black}]|"i love leetcode"| have same hot degree. Since letter \lstinline[language=Java, basicstyle=\small\ttfamily, keywordstyle=\bfseries\color{green!40!black}]|' '| has ASCII code 32 and \lstinline[language=Java, basicstyle=\small\ttfamily, keywordstyle=\bfseries\color{green!40!black}]|'r'| has ASCII code 114, \lstinline[language=Java, basicstyle=\small\ttfamily, keywordstyle=\bfseries\color{green!40!black}]|"i love leetcode"| should be in front of \lstinline[language=Java, basicstyle=\small\ttfamily, keywordstyle=\bfseries\color{green!40!black}]|"ironman"|. Also we only need to output top 3 hot sentences, so \lstinline[language=Java, basicstyle=\small\ttfamily, keywordstyle=\bfseries\color{green!40!black}]|"ironman"| will be ignored.

\textbf{Operation}: \lstinline[language=Java, basicstyle=\small\ttfamily, keywordstyle=\bfseries\color{green!40!black}]|input(' ')|

\textbf{Output}: \lstinline[language=Java, basicstyle=\small\ttfamily, keywordstyle=\bfseries\color{green!40!black}]|["i love you", "i love leetcode"]|

\textbf{Explanation}:

There are only two sentences that have prefix \lstinline[language=Java, basicstyle=\small\ttfamily, keywordstyle=\bfseries\color{green!40!black}]|"i "|

\textbf{Operation}: \lstinline[language=Java, basicstyle=\small\ttfamily, keywordstyle=\bfseries\color{green!40!black}]|input('a')|

\textbf{Output}: \lstinline[language=Java, basicstyle=\small\ttfamily, keywordstyle=\bfseries\color{green!40!black}]|[]|

\textbf{Explanation}:

There are no sentences that have prefix \lstinline[language=Java, basicstyle=\small\ttfamily, keywordstyle=\bfseries\color{green!40!black}]|"i a"|.

\textbf{Operation}: \lstinline[language=Java, basicstyle=\small\ttfamily, keywordstyle=\bfseries\color{green!40!black}]|input('#')|

\textbf{Output}: \lstinline[language=Java, basicstyle=\small\ttfamily, keywordstyle=\bfseries\color{green!40!black}]|[]|

\textbf{Explanation}:

The user finished the input, the sentence \lstinline[language=Java, basicstyle=\small\ttfamily, keywordstyle=\bfseries\color{green!40!black}]|"i a"| should be saved as a historical sentence in system. And the following input will be counted as a new search.
 

\end{flushleft}

\paragraph{Note:}

\begin{itemize}
\item The input sentence will always start with a letter and end with \#, and only one blank space will exist between two words.
\item The number of complete sentences that to be searched won't exceed 100. The length of each sentence including those in the historical data won't exceed 100.
\item Please use double-quote instead of single-quote when you write test cases even for a character input.
\end{itemize}

\subsection{Brute Force}
We make use of a hash map $M$ to store entries in the form $(S[i], T[i])$. Here, $T[i]$ is the number of times the sentence $S[i]$​ has been typed earlier.

\begin{itemize}
\item \lstinline[language=Java, basicstyle=\small\ttfamily, keywordstyle=\bfseries\color{green!40!black}]|AutocompleteSystem|: We pick up each sentence  and their corresponding times from the given inputs, and add to $M$.

\item \lstinline[language=Java, basicstyle=\small\ttfamily, keywordstyle=\bfseries\color{green!40!black}]|input(c)|: We make use of a variable, $x$ to store the sentence entered till now as the input. For current input character $c$, we append it to $x$ and then iterate over all the keys of $M$ to check if a key exists whose initial characters match with $x$ (In C++, we can use function \lstinline[language=Java, basicstyle=\small\ttfamily, keywordstyle=\bfseries\color{green!40!black}]|compare| to check if a key starts with $x$). We add all such keys to a list, $\ell$. Then, we sort $\ell$ as per our requirements, and obtain the first three values from it.
\end{itemize}

\subsection{Two Levels Hash Map}
Brute force is not an optimal one. To improve the efficiency, we instead make use of a two level hash map, i.e., an array, $A$, with each element as a hash map. The array's size is 26. For example, $A[0]$ is a hash map which stores the sentences starting with letter \texttt{a}.

The process remains the same as in the brute force approach, except the one level indexing of $A$ which needs to be done prior to accessing the required hash map.

\subsection{Trie}
Using Trie is very straight forward. We build a trie tree from the input sentences. We store the times and the string into the leaf node of the tree. 

Then for each input, we traverse the tree until a start node. Then from this node, we collect all reachable leaf nodes, put the times and string into a tree map to automatically sort by the times. The value of each node in the tree map is a tree set so that we can sort the strings with same times automatically. 