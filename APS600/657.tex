\section{657. Robot Return to Origin}
There is a robot starting at position \lstinline[language=C++, basicstyle=\small\ttfamily, keywordstyle=\bfseries\color{green!40!black}]|(0, 0)|, the origin, on a 2D plane. Given a sequence of its moves, judge if this robot \textbf{ends up at} \lstinline[language=C++, basicstyle=\small\ttfamily, keywordstyle=\bfseries\color{green!40!black}]|(0, 0)| after it completes its moves.

The move sequence is represented by a string, and the character \lstinline[language=C++, basicstyle=\small\ttfamily, keywordstyle=\bfseries\color{green!40!black}]|moves[i]| represents its $i$-th move. Valid moves are \lstinline[language=Java, basicstyle=\small\ttfamily, keywordstyle=\bfseries\color{green!40!black}]|R| (right), \lstinline[language=Java, basicstyle=\small\ttfamily, keywordstyle=\bfseries\color{green!40!black}]|L| (left), \lstinline[language=Java, basicstyle=\small\ttfamily, keywordstyle=\bfseries\color{green!40!black}]|U| (up), and \lstinline[language=Java, basicstyle=\small\ttfamily, keywordstyle=\bfseries\color{green!40!black}]|D| (down). If the robot returns to the origin after it finishes all of its moves, return \lstinline[language=C++, basicstyle=\small\ttfamily, keywordstyle=\bfseries\color{green!40!black}]|true|. Otherwise, return \lstinline[language=C++, basicstyle=\small\ttfamily, keywordstyle=\bfseries\color{green!40!black}]|false|.

\paragraph{Note:} 
\begin{itemize}
\item The way that the robot is \lstinline[language=Java, basicstyle=\small\ttfamily, keywordstyle=\bfseries\color{green!40!black}]|"facing"| is irrelevant. \lstinline[language=C++, basicstyle=\small\ttfamily, keywordstyle=\bfseries\color{green!40!black}]|"R"| will always make the robot move to the right once, \lstinline[language=C++, basicstyle=\small\ttfamily, keywordstyle=\bfseries\color{green!40!black}]|"L"| will always make it move left, etc. Also, assume that the magnitude of the robot's movement is the same for each move.

\end{itemize}

\paragraph{Example 1:}

\begin{flushleft}
\textbf{Input}: \lstinline[language=C++, basicstyle=\small\ttfamily, keywordstyle=\bfseries\color{green!40!black}]|"UD"|

\textbf{Output}: \lstinline[language=C++, basicstyle=\small\ttfamily, keywordstyle=\bfseries\color{green!40!black}]|true|

\textbf{Explanation}: The robot moves up once, and then down once. All moves have the same magnitude, so it ended up at the origin where it started. Therefore, we return \lstinline[language=C++, basicstyle=\small\ttfamily, keywordstyle=\bfseries\color{green!40!black}]|true|.

\end{flushleft}
 

\paragraph{Example 2:}

\begin{flushleft}
\textbf{Input}: \lstinline[language=C++, basicstyle=\small\ttfamily, keywordstyle=\bfseries\color{green!40!black}]|"LL"|

\textbf{Output}: \lstinline[language=C++, basicstyle=\small\ttfamily, keywordstyle=\bfseries\color{green!40!black}]|false|

\textbf{Explanation}: The robot moves left twice. It ends up two \lstinline[language=C++, basicstyle=\small\ttfamily, keywordstyle=\bfseries\color{green!40!black}]|"moves"| to the left of the origin. We return \lstinline[language=C++, basicstyle=\small\ttfamily, keywordstyle=\bfseries\color{green!40!black}]|false| because it is not at the origin at the end of its moves.
\end{flushleft}

\subsection{Simulation}
A very easy problem. Just simulate the coordinate changes for each direction.