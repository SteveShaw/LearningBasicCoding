\section{695 --- Max Area of Island}
Given a non-empty 2D array grid of 0's and 1's, an island is a group of 1's (representing land) connected 4-direction (horizontal or vertical.) You may assume all four edges of the grid are surrounded by water.
\par
Find the \textbf{maximum area} of an island in the given 2D array. (If there is no island, the maximum area is 0.)
\paragraph{Example 1:}
\begin{flushleft}
\begin{figure}[H]
\centering
\begin{tikzpicture} 
\matrix (m) [matrix of nodes]
{
0 & 0 & 1 & 0 & 0 & 0 & 0 & 1 & 0 & 0 & 0 & 0 & 0 \\
0 & 0 & 0 & 0 & 0 & 0 & 0 & 1 & 1 & 1 & 0 & 0 & 0 \\
0 & 1 & 1 & 0 & 1 & 0 & 0 & 0 & 0 & 0 & 0 & 0 & 0 \\
0 & 1 & 0 & 0 & 1 & 1 & 0 & 0 & 1 & 0 & 1 & 0 & 0 \\
0 & 1 & 0 & 0 & 1 & 1 & 0 & 0 & 1 & 1 & 1 & 0 & 0 \\
0 & 0 & 0 & 0 & 0 & 0 & 0 & 0 & 0 & 0 & 1 & 0 & 0 \\
0 & 0 & 0 & 0 & 0 & 0 & 0 & 1 & 1 & 1 & 0 & 0 & 0 \\
0 & 0 & 0 & 0 & 0 & 0 & 0 & 1 & 1 & 0 & 0 & 0 & 0 \\
};
\draw [color=green](m-1-8.north west)--(m-2-8.south west)--(m-2-10.south east)--(m-2-10.north east)--(m-2-8.north east)--(m-1-8.north east)--cycle;
\draw [color=green](m-3-2.north west)--(m-5-2.south west)--(m-5-2.south east)--(m-3-2.south east)--(m-3-3.south east)--(m-3-3.north east)--cycle;
\draw [color=green](m-3-5.north west)--(m-5-5.south west)--(m-5-6.south east)--(m-4-6.north east)--(m-4-5.north east)--(m-3-5.north east)--cycle;
\draw [color=red](m-4-9.north west)--(m-5-9.south west)--(m-5-11.south west)--(m-6-11.south west)--(m-6-11.south east)--(m-4-11.north east)--cycle;
\draw [color=green](m-7-8.north west)--(m-8-8.south west)--(m-8-9.south east)--(m-7-9.south east)--(m-7-10.south east)--(m-7-10.north east)--cycle;
\draw [color=blue](m-1-3.north west)--(m-1-3.south west)--(m-1-3.south east)--(m-1-3.north east)--cycle;
\end{tikzpicture}
\end{figure}

Given the above grid, return 6. 

Note the answer is not 11, because the island must be connected 4-direction.
\end{flushleft}

\paragraph{Example 2:}
\begin{flushleft}
\[
\begin{bmatrix}
0 & 0 & 0 & 0 & 0 & 0 & 0 & 0
\end{bmatrix}
\]

Given the above grid, return 0.
\end{flushleft}

\paragraph{Note:}
\begin{itemize}
\item The length of each dimension in the given grid does not exceed 50.
\end{itemize}

\subsection{DFS}

很显然这时寻找四连通区域,在每一个1处进行DFS搜索,搜索完成后,即可获得当前连通区域中1的个数,也就是area大小。这里的一个技巧是,可以把访问到的元素设为$-1$,这样避免了使用额外的二维数组来记录是否访问过。

\setcounter{lstlisting}{0}
\begin{lstlisting}[style=customc, caption={DFS}]
int maxAreaOfIsland( vector<vector<int>>& grid )
{
    int M = static_cast< int >( grid.size() );
    int N = static_cast< int >( grid[0].size() );

    int best = 0;

    for( int r = 0; r < M; ++r )
    {
        for( int c = 0; c < N; ++c )
        {
            if( grid[r][c] == 1 )
            {
                int area = 0;

                dfs( grid, r, c, area );

                best = ( max )( area, best );
            }
        }
    }

    return best;
}

void dfs( vector<vector<int>>& G, int r, int c, int& area )
{
    if( G[r][c] != 1 )
    {
        return;
    }

    //this is a point we have not seen
    //and also belong to an island
    //increment the area
    ++area;

    //mark it as visited
    G[r][c] = 2;

    int dr[] = { -1, 0, 1, 0 };
    int dc[] = { 0, -1, 0, 1 };

    int M = static_cast< int >( G.size() );
    int N = static_cast< int >( G[0].size() );

    for( int i = 0; i < 4; ++i )
    {
        int nr = r + dr[i];
        int nc = c + dc[i];

        if( ( nr >= 0 ) && ( nr < M ) && ( nc >= 0 ) && ( nc < N ) )
        {
            dfs( G, nr, nc, area );
        }
    }
}
\end{lstlisting}