\section{673 --- Number of Longest Increasing Subsequence}
Given an unsorted array of integers, $A$,  find the number of longest increasing subsequence.

\paragraph{Example 1:}
\begin{flushleft}


\textbf{Input}: \lstinline[language=C++, basicstyle=\small\ttfamily, keywordstyle=\bfseries\color{green!40!black}]|[1,3,5,4,7]|

\textbf{Output}: 2

\textbf{Explanation}: The two longest increasing subsequence are \fcj{[1, 3, 4, 7]} and \fcj{[1, 3, 5, 7]}.
\end{flushleft}

\paragraph{Example 2:}
\begin{flushleft}

\textbf{Input}: \fcj{[2,2,2,2,2]}

\textbf{Output}: 5

\textbf{Explanation}: 

The length of longest continuous increasing subsequence is 1, and there are 5 subsequences' length is 1, so output 5.
\end{flushleft}

\paragraph{Note:} 
\begin{itemize}
\item Length of the given array will be not exceed 2000 and the answer is guaranteed to be fit in 32-bit \fcc{signed int}.
\end{itemize}

\subsection{DP}
It is hard to find the recursive relationship if we define the DP array $F[i]$ as the number of LIS ending at index $i$.

We need to define two dp arrays: $L$ and $C$ where 
\begin{itemize}
\item $L[i]$ is the length of LIS ending with \fcj{nums[i]}
\item $C[i]$ is the number of LIS ending with \fcj{nums[i]}.
\end{itemize}

Then starting the normal LIS finding process with two loops.
For a given index $i$, we search all elements at index $j$ where $j <i$. 

If \fcj{nums[j] < nums[i]}, 
\begin{itemize}
\item if $L[i]=L[j]+1$, \fcj{nums[i]} can be added to LIS ending with \fcj{nums[j]}. Then the number of LIS ending with \fcj{nums[j]} can be added to the number of LIS ending with \fcj{nums[i]}, i.e., $C[i] = C[i] + C[j]$.
\item if $L[i] < L[j]+1$, a longer LIS is found. Then we update $L[i]$ as $L[j]+1$, and $C[i]$ as $C[j]$. 
\end{itemize}

After updating $L[i]$ and $C[i]$, compare with the maximum LIS length so far. If current maximum LIS length is equal to $L[i]$, we add $C[i]$ to the result. Otherwise, if $L[i]$ is larger than current maximum LIS length, we update result as $C[i]$ and also the maximum LIS length.

\setcounter{lstlisting}{0}
\begin{lstlisting}[style=customc, caption={DP}]
int findNumberOfLIS( vector<int>& nums )
{
    if( nums.empty() )
    {
        return 0;
    }
    //lens is used to record the length of LIS
    //ending with nums[i]
    vector<int> lens( nums.size(), 1 );
    //counts is used to record the counts of LIS
    //ending with nums[i]
    vector<int> counts( nums.size(), 1 );
    //best is the maximum length of LIS
    int best = 1;
    int ans = 1;
    for( int i = 1; i < nums.size(); ++i )
    {
        for( int j = 0; j < i; ++j )
        {
            if( nums[j] < nums[i] )
            {
                if( lens[i] == lens[j] + 1 )
                {
                    //update counts ending with nums[i]
                    counts[i] += counts[j];
                }
                else if( lens[i] < lens[j] + 1 )
                {
                    //previous lens[i] and counts[i] cannot be used
                    //update LIS length ending with nums[i]
                    lens[i] = lens[j] + 1;
                    //update counts ending with nums[i]
                    counts[i] = counts[j];
                }
            }
        }
        if( best == lens[i] )
        {
            //add counts[i] to the result
            ans += counts[i];
        }
        else if( best < lens[i] )
        {
            //previous result and maximum LIS length
            //must be discarded
            best = lens[i];
            ans = counts[i];
        }
    }
    return ans;
}
\end{lstlisting}