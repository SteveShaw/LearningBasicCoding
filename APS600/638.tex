\section{638 --- Shopping Offers}
In LeetCode Store, there are some kinds of items to sell. Each item has a price.

However, there are some special offers, and a special offer consists of one or more different kinds of items with a sale price.

You are given the each item's price, $P$, a set of special offers, $S$, and the number we need to buy for each item, $N$. The job is to output the lowest price you have to pay for exactly certain items as given, where you could make optimal use of the special offers.

Each special offer is represented in the form of an array, the last number represents the price you need to pay for this special offer, other numbers represents how many specific items you could get if you buy this offer.

You could use any of special offers as many times as you want.

\paragraph{Example 1:}

\begin{flushleft}
\textbf{Input}: [2,5], [[3,0,5],[1,2,10]], [3,2]

\textbf{Output}: 14

\textbf{Explanation}:
 
There are two kinds of items, A and B. Their prices are \$2 and \$5 respectively. 

In special offer 1, you can pay \$5 for 3A and 0B

In special offer 2, you can pay \$10 for 1A and 2B. 

You need to buy 3A and 2B, so you may pay \$10 for 1A and 2B (special offer \#2), and \$4 for 2A.
\end{flushleft}

\paragraph{Example 2:}

\begin{flushleft}
\textbf{Input}: [2,3,4], [[1,1,0,4],[2,2,1,9]], [1,2,1]

\textbf{Output}: 11

\textbf{Explanation}: 

The price of A is \$2, and \$3 for B, \$4 for C. 

You may pay \$4 for 1A and 1B, and \$9 for 2A ,2B and 1C. 

You need to buy 1A ,2B and 1C, so you may pay \$4 for 1A and 1B (special offer \#1), and \$3 for 1B, \$4 for 1C. 

You cannot add more items, though only \$9 for 2A ,2B and 1C.
\end{flushleft}

\paragraph{Note:}

\begin{itemize}
\item There are at most 6 kinds of items, 100 special offers.
\item For each item, you need to buy at most 6 of them.
\item You are not allowed to buy more items than you want, even if that would lower the overall price.
\end{itemize}

\subsection{Backtracking}
From the question, 
\begin{itemize}
\item $N$ needs to be updated when applying any special offer.
\item One special offer can only be used only if the number items of each type is no larger than the ones available in the current $N$.
\item One special offer can be used more than once.
\end{itemize}

Therefore, in the recursion function to backtrack, 
\begin{enumerate}
\item We must test each special offer for current $N$. This means there will be a loop from given start index.
\item For each special offer that can be applied to current needs list $N$, updated $N$ and then recursively starting from this offer to test again. When the recursive calling is end, do backtrack, i.e., restore back $N$ to the ones before the recursive call.
\item After the loop ended, we may still have some items left to be purchased. Thus we add the remaining costs to the total cost and update the minimum one.
\end{enumerate}
