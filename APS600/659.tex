\section{659 --- Split Array into Consecutive Subsequences}
Given an array \fcj{nums} sorted in ascending order, return \fcc{true} if and only if you can split it into 1 or more subsequences such that each subsequence consists of consecutive integers and has length at least 3.

\paragraph{Example 1:}

\begin{flushleft}
\textbf{Input}: \fcj{[1,2,3,3,4,5]}

\textbf{Output}: \fcc{true}

\textbf{Explanation}:

You can split them into two consecutive subsequences:
 
1, 2, 3

3, 4, 5
\end{flushleft}

\paragraph{Example 2:}

\begin{flushleft}
\textbf{Input}: \fcj{[1,2,3,3,4,4,5,5]}

\textbf{Output}: \fcc{true}

\textbf{Explanation}:

You can split them into two consecutive subsequences : 

1, 2, 3, 4, 5

3, 4, 5

\end{flushleft}

\paragraph{Example 3:}

\begin{flushleft}
\textbf{Input}: \fcj{[1,2,3,4,4,5]}


\textbf{Output}: \fcc{false}
\end{flushleft}

 

\paragraph{Constraints:}

\begin{itemize}
\item  $ 1 \leq \lvert A\rvert \leq 10000$
\end{itemize}

\subsection{Greedy}
The basic idea is that, for each distinct element $x$ in the input array, we only need to maintain three variables, i.e., the number of consecutive sub-sequences ending at $x$ with length of 1, 2 and larger than 2.

Suppose $x$ is the current number we are processing, and the count of $x$ is $\Delta_x$. We also assume $t$ is the number we have processed immediately before $x$.

We denote the the number of consecutive sub-sequences ending at $t0$ with length 1, 2 and larger than 2 are $\ell_1$, $\ell_2$ and $\ell_3$ respectively.

Then, we have two scenarios:

$\star$  When $x\neq t+1$: in this case, we cannot add $x$ to any consecutive subsequence ending at $t$. Thus $\ell_1=\ell_2=0$ (i.e., the consecutive sequence ending at $t$ with length 1 and 2).
 
Suppose $l_1$, $l_2$ and $l_3$ are number of consecutive sub-sequences ending at $x$ with length 1, 2 and no less than 3, we will have $l_1=\Delta_x$, $l_2=0$ and $l_3=0$, which means we only have $\Delta_x$ consecutive subsequence with length 1 ending at $x$.

$\star$ When $x=t+1$. This allows us to add $x$ to consecutive sub-sequences ending at $t$ and thus extend those sub-sequences.

But we should add these $x$ to those sub-sequences with length 1 first, and length 2 secondly, and finally to those with length larger than 2. We also need $\Delta_x\geq \ell_1+\ell_2$. If this condition is violated, some consecutive sequences will extend length over 2.

As in first case, we use $l_1$, $l_2$ and $l_3$ to represent the number of consecutive sub-sequences ending at $x$ with length of 1, 2 and over 2 respectively. We can get following relationships
\begin{align*}
    l_2 &= \ell_1 \\
    l_3 &= \ell_2 + \min(\ell_3, \Delta_x-(\ell_1+\ell_2)) \\
   l_1 &= \max(\Delta_x - (\ell_1+\ell_2+\ell_3), 0)
\end{align*}

Since we add $x$ to the consecutive sub-sequence ending at $t$ with length 1 firstly, the consecutive sub-sequence ending $x$ will have length 2. Thus, $l_2 \gets \ell_1$.

Next, we add remaining $x$ to those consecutive sub-sequences ending at $t$ with length 2. These will give consecutive sub-sequences ending at $x$ with length 3. Thus, $l_3\geq \ell_2$. If $\Delta_x > \ell_1 + \ell_2$, we still have some $x$ to use and can add them to the subsequences ending at $t$ with length over 2. All these subsequences ending at $t$ have length over 2, and the total number of such subsequences is $\ell_2 + \min(\ell_3, \Delta_x - (\ell_1 + \ell_2))$. 

If $\Delta_x > \ell_1 + \ell_2 + \ell_3$, we still have remaining $x$s will form the sub-sequences ending at $x$ with length 1, and thus $l_1 = \max(\Delta_x - (\ell_1 + \ell_2 + \ell_3), 0)$.

Finally, if we get $\ell_1=0$ and $\ell_2=0$, we can split the array into consecutive parts. Otherwise, it cannot be done.

\setcounter{lstlisting}{0}
\begin{lstlisting}[style=customc, caption={Greedy}]
bool isPossible( vector<int>& nums )
{
    //number of subsequence with length 1 ending at x
    long long l1{};
    //number of subsequence with length 2 ending at x
    long long l2{};
    //number of subsequence with length over 2 ending at x
    long long l3{};
    //previous element visited
    int prev{};
    for( auto s = begin( nums ); s != end( nums ); )
    {
        int x = *s;
        auto next_s = upper_bound( s, end( nums ), x );
        auto count_x = distance( s, next_s );
        if( s == begin( nums ) )
        {
            //we have count_x subsequence with length 1
            //which are ending at x
            l1 = count_x;
            s = next_s;
            prev = x;
            continue;
        }
        if( x != prev + 1 )
        {
            //x cannot be added to subsequence ending at prev
            if( ( l1 > 0 ) || ( l2 > 0 ) )
            {
                //we only have subsequence with length 1 or 2
                //ending at prev
                //Thus we cannot get conseuctive part with length >= 3
                return false;
            }
            //subsequence with length 1 ending at x
            l1 = count_x;
            //subsequence with length 2 ending at x
            l2 = 0;
            //subsequence with length over 2 ending at x
            l3 = 0;
        }
        else
        {
            if( count_x < l1 + l2 )
            {
                //We don't have enough x to add to
                //subsequence with length 1 and 2 ending at prev
                //which will leave some subsequence ending at prev
                //with length less than 3
                return false;
            }
            //x can be added to subsequence ending at prev
            //add to subsequence with length 1 ending at prev first
            //Thus we have
            auto cur_l2 = l1;
            //then add to subsequence with length 2 ending at prev
            auto cur_l3 = l2;
            //if there still some x left
            //we can extend subsequence with length 3 ending at prev
            if( count_x > l1 + l2 )
            {
                //we can only extend to minimum of l3 and count_x - l1 - l2
                //subsequences with length over 3
                cur_l3 += ( min )( l3, count_x - l1 - l2 );
            }
            //If we still have some x left,
            //these x will form a subsequence with length 1 ending at x
            auto cur_l1 = ( max )( 0ll, count_x - l1 - l2 - l3 );

            //update to l1, l2 and l3
            l1 = cur_l1;
            l2 = cur_l2;
            l3 = cur_l3;
        }

        s = next_s;
        prev = x;
    }
    if( ( l1 == 0 ) && ( l2 == 0 ) )
    {
        //we don't have subsequence with length less than 3
        //ending at last element in the given array
        return true;
    }
    return false;
}
\end{lstlisting}

\subsection{Greedy With Two Pass}
In first pass we get the counts of each number.

During the second pass, make use of another hash map to record the number of consecutive subsequences with length over 2 found so far.

For each number $i$, we try to put it to the tail of one of found subsequence ending at $i-1$. 

If we cannot, we may use $i+1$ and $i+2$ along with $i$ to generate a new subsequence. 

\begin{lstlisting}[style=customc, caption={Two Pass}]
bool isPossible( vector<int>& nums )
{
    unordered_map<int, int> counts;
    unordered_map<int, int> need;
    for( int n : nums )
    {
        counts[n] += 1;
    }
    for( int n : nums )
    {
        auto it = counts.find( n );
        if( it->second == 0 )
        {
            //skip n, since no subsequence
            //can be formed
            continue;
        }
        auto p = need.find( n );
        if( ( p != need.end() ) && ( p->second > 0 ) )
        {
            //consume n
            p->second -= 1;
            //we need another n+1
            need[n + 1] += 1;
        }
        else if( auto c1 = counts.find( n + 1 ), c2 = counts.find( n + 2 );
                 ( ( c1 != counts.end() ) && ( c1->second > 0 ) && ( c2 != counts.end() ) && ( c2->second > 0 ) ) )
        {
            //we consume n+1 and n+2
            --c1->second;
            --c2->second;
            //we need another n+3
            need[n + 3] += 1;
        }
        else
        {
            return false;
        }
        --it->second;
    }
    return true;
}
\end{lstlisting}