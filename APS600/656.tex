\section{656 --- Coin Path}
Given an array $A$ (index starts at 1) consisting of $N$ integers: $A_1$, $A_2$, $\ldots$, $A_N$ and an integer $B$. The integer $B$ denotes that from any place (suppose the index is $i$) in the array $A$, you can jump to any one of the place in the array A indexed $i+1$, $i+2$, $\ldots$ $i+B$ if this place can be jumped to. Also, if you step on the index $i$, you have to pay $A_i$ coins. If $A_i$ is $-1$, it means you can’t jump to the place indexed $i$ in the array.

Now, you start from the place indexed 1 in the array $A$, and your aim is to reach the place indexed $N$ using the minimum coins. You need to return the path of indexes (starting from 1 to $N$) in the array you should take to get to the place indexed $N$ using minimum coins.

If there are multiple paths with the same cost, return the lexicographically smallest such path.

If it's not possible to reach the place indexed $N$ then you need to return an empty array.

\paragraph{Example 1:}

\begin{flushleft}
\textbf{Input}: \lstinline[language=Java, basicstyle=\small\ttfamily, keywordstyle=\bfseries\color{green!40!black}]|[1,2,4,-1,2], 2|

\textbf{Output}: \lstinline[language=Java, basicstyle=\small\ttfamily, keywordstyle=\bfseries\color{green!40!black}]|[1,3,5]|

\end{flushleft}
 

\paragraph{Example 2:}

\begin{flushleft}
\textbf{Input}: \lstinline[language=Java, basicstyle=\small\ttfamily, keywordstyle=\bfseries\color{green!40!black}]|[1,2,4,-1,2], 1|

\textbf{Output}: \lstinline[language=Java, basicstyle=\small\ttfamily, keywordstyle=\bfseries\color{green!40!black}]|[]|

\end{flushleft}
 

\paragraph{Note:}

\begin{itemize}
\item Path $a_1, a_2, \ldots, a_n$ is lexicographically smaller than $b_1, b_2, \ldots, b_m$, if and only if at the first $i$ where $a_i$ and $b_i$ differ, $a_i < b_i$; when no such $i$ exists, then $n < m$.
\item $A_1 \geq 0$. $A_2, \ldots, A_N$ (if exist) will in the range of \lstinline[language=Java, basicstyle=\small\ttfamily, keywordstyle=\bfseries\color{green!40!black}]|[-1, 100]|.
\item Length of $A$ is in the range of \lstinline[language=Java, basicstyle=\small\ttfamily, keywordstyle=\bfseries\color{green!40!black}]|[1, 1000]|.
\item $B$ is in the range of \lstinline[language=Java, basicstyle=\small\ttfamily, keywordstyle=\bfseries\color{green!40!black}]|[1, 100]|.

\end{itemize}

\subsection{Backtrace}
This is a typical backtracking problem. We try each possible selection, and check if we can acheive the best result.

For this problem, we make use of an array $P$ with size $N$ to record next index of current index $i$, i.e, $P[i]$ is the next index from index $i$. At start, $P$ is filled  with all $-1$. 

In the backtracking process, With $i$ as the current index, we can consider every possible index from $i+1$ to $i+B$ as the next place to be jumped to. For every such next index, $j$, if this place can be jumped to, we determine the cost of reaching the end of the array, say $c$, which is starting from the index $i$, and with $j$ as the next index jumped from $i$. If $c$ is less than the global minimum cost, say $\kappa$, we can update $\kappa$ as $c$ and update $P[i]$ as $j$.

After the backtracking process finished, we traverse over the array $P$ starting from the index $0$. At every step, we add the current index to the result list and also move to the index pointed by $P[i]$, since this refers to the next index for the minimum cost.

\subsection{Backtracking With Memory}
In the brute force backtracking approach, a lot of duplicate function calls are made, since we are considering the same index through multiple paths. To remove this redundancy, we can make use of memorization.

We make use of an array, say $M$, to store the minimum cost of jumps to reach the end of the array $A$. Whenever the value for any index is calculated once, it is stored in its appropriate location. Thus, next time whenever the same function call is made, we can return the result directly from $M$, pruning the search space to a great extent.