\section{677 --- Map Sum Pairs}
Implement a \lstinline[language=Java, basicstyle=\small\ttfamily, keywordstyle=\bfseries\color{green!40!black}]|MapSum| class with \lstinline[language=Java, basicstyle=\small\ttfamily, keywordstyle=\bfseries\color{green!40!black}]|insert|, and \lstinline[language=Java, basicstyle=\small\ttfamily, keywordstyle=\bfseries\color{green!40!black}]|sum| methods.

For the method \lstinline[language=Java, basicstyle=\small\ttfamily, keywordstyle=\bfseries\color{green!40!black}]|insert|, you'll be given a pair of \lstinline[language=Java, basicstyle=\small\ttfamily, keywordstyle=\bfseries\color{green!40!black}]|(string, integer)|. The \lstinline[language=Java, basicstyle=\small\ttfamily, keywordstyle=\bfseries\color{green!40!black}]|string| represents the key and the \lstinline[language=Java, basicstyle=\small\ttfamily, keywordstyle=\bfseries\color{green!40!black}]|integer| represents the value. If the key already existed, then the original key-value pair will be overridden to the new one.

For the method \lstinline[language=Java, basicstyle=\small\ttfamily, keywordstyle=\bfseries\color{green!40!black}]|sum|, you'll be given a string representing the prefix, and you need to return the sum of all the pairs' value whose key starts with the prefix.

\paragraph{Example 1:}

\begin{flushleft}


\textbf{Input}: \lstinline[language=Java, basicstyle=\small\ttfamily, keywordstyle=\bfseries\color{green!40!black}]|insert("apple", 3)|, \textbf{Output}: \lstinline[language=Java, basicstyle=\small\ttfamily, keywordstyle=\bfseries\color{green!40!black}]|Null|

\textbf{Input}: \lstinline[language=Java, basicstyle=\small\ttfamily, keywordstyle=\bfseries\color{green!40!black}]|sum("ap")|, \textbf{Output}: 3

\textbf{Input}: \lstinline[language=Java, basicstyle=\small\ttfamily, keywordstyle=\bfseries\color{green!40!black}]|insert("app", 2)|, Output: \lstinline[language=Java, basicstyle=\small\ttfamily, keywordstyle=\bfseries\color{green!40!black}]|Null|

\textbf{Input}: \lstinline[language=Java, basicstyle=\small\ttfamily, keywordstyle=\bfseries\color{green!40!black}]|sum("ap")|, Output: 5

\end{flushleft}

\subsection{Trie}
Since this is for prefix searching, Trie structure is the natural way. For \lstinline[language=Java, basicstyle=\small\ttfamily, keywordstyle=\bfseries\color{green!40!black}]|sum| function, we need a depth first search approach to get the total of all the trie nodes each represents a key.

\setcounter{lstlisting}{0}
\begin{lstlisting}[style=customc, caption={Trie}]
class MapSum
{
public:
    /** Initialize your data structure here. */
    MapSum()
    {
    }

    void insert( string key, int val )
    {
        add( key, val );
    }

    int sum( string prefix )
    {
        int ans = 0;

        auto node = m_root;

        //go to the end node of prefix
        for( const auto c : prefix )
        {
            auto it = node->children.find( c );
            if( it == node->children.end() )
            {
                return 0;
            }

            node = it->second;
        }

        //using dfs to sum all the values
        dfs( node, ans );

        return ans;
    }

private:

    struct Trie
    {
        unordered_map<char, Trie*> children;
        int val = 0;
        unsigned char flag = 0;
    };

    Trie* m_root = new Trie;
    void add( const string& s, int val )
    {
        auto node = m_root;

        for( auto c : s )
        {
            auto it = node->children.find( c );
            if( it == node->children.end() )
            {
                it = node->children.emplace( c, new Trie ).first;
            }
            node = it->second;
        }

        node->flag = 1;
        node->val = val;
    }

    void dfs( Trie* trie, int& total )
    {
        if( !trie )
        {
            return;
        }

        if( trie->flag )
        {
            //add the value
            // of this key to total
            total += trie->val;
        }

        for( const auto& p : trie->children )
        {
            //recursively deeper into the children
            dfs( p.second, total );
        }
    }
};

/**
 * Your MapSum object will be instantiated and called as such:
 * MapSum* obj = new MapSum();
 * obj->insert(key,val);
 * int param_2 = obj->sum(prefix);
 */
\end{lstlisting}