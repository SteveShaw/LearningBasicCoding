\section{686 --- Repeated String Match}
Given two strings $A$ and $B$, find the minimum number of times $A$ has to be repeated such that $B$ is a substring of it. If no such solution, return $-1$.

For example, with \lstinline[language=C++, basicstyle=\small\ttfamily, keywordstyle=\bfseries\color{green!40!black}]|A = "abcd"| and \lstinline[language=C++, basicstyle=\small\ttfamily, keywordstyle=\bfseries\color{green!40!black}]|B = "cdabcdab"|.

Return 3, because by repeating $A$ three times (\lstinline[language=C++, basicstyle=\small\ttfamily, keywordstyle=\bfseries\color{green!40!black}]|"abcdabcdabcd"|), $B$ is a substring of it; and $B$ is not a substring of $A$ repeated two times (\lstinline[language=C++, basicstyle=\small\ttfamily, keywordstyle=\bfseries\color{green!40!black}]|"abcdabcd"|).

\paragraph{Note:}

\begin{itemize}
\item The length of $ A $ and $ B $ will be between 1 and 10000.
\end{itemize}

\subsection{Sliding Window}
We keep track of index $i$ and $j$ for $A$ and $B$, then compare $A[i]$ with $B[j]$. Since $A$ can repeat, the index that we will scanning in $A$ will be $\bmod(i+j, L_A)$ where $L_A$ is the length of $A$.

We increments $j$ whenever $A[\bmod(i+j, L_A)]=B[j]$ until the end of $B$. If $j=L_B$ ($L_B$ is the length of $B$), we know $B$ will appear as a substring in a string with repeated $A$. The count of repeat will be $(i+j-1)/L_A+1$. The reason is when $i+j=L_A$, the count of repeating $A$ is 1.

\setcounter{lstlisting}{0}
\begin{lstlisting}[style=customc, caption={Sliding Window}]
int repeatedStringMatch( string A, string B )
{
    size_t LA = A.size();

    size_t LB = B.size();

    for( size_t i = 0; i < LA; ++i )
    {
        size_t  j = 0;

        while( j < LB )
        {
            //we start from (i+j) % LA
            auto k = ( i + j ) % LA;

            if( A[k] == B[j] )
            {
                ++j;
            }
            else
            {
                break;
            }
        }

        //all are matched
        //LB is the substring
        if( j == LB )
        {
            return ( i + j - 1 ) / LA + 1;
        }
    }

    return -1;
}
\end{lstlisting}

\subsection{KMP}
We build a KMP table $T$ for string $B$. The comparing scheme is same as the sliding window approach, i.e.

\begin{lstlisting}[style=customc, caption={Scanning Scheme}]
while( j < LB )
{
    //we start from (i+j) % LA
    //when j is incremented
    //i+j is also incremented
    //because i is fixed.
    auto k = ( i + j ) % LA;

    if( A[k] == B[j] )
    {
        ++j;
    }
    else
    {
        break;
    }
}
\end{lstlisting}

When the matching is failed at some index $j$, we will utilize the KMP table to find the next matching index. In this problem, since the index to scan in $A$, $i$, is fixed when comparing, $i$ will be changed to $i + j - T[j-1]$. (In the normal string matching KMP, the scanning index $i$ in the string to be searched is incremented when matched).


\begin{lstlisting}[style=customc, caption={KMP}]
int repeatedStringMatch( string A, string B )
{
    auto LA = static_cast< int >( A.size() );
    auto LB = static_cast< int >( B.size() );

    //build KMP table
    vector<int> F( B.size(), 0 );
    int t = 0;
    for( int i = 1; i < LB; ++i )
    {
        t = F[i - 1];

        while( ( t > 0 ) && ( B[i] != B[t] ) )
        {
            t = F[t - 1];

        }

        if( B[i] == B[t] )
        {
            t = t + 1;
        }

        F[i] = t;
    }

    int i = 0; //index in A
    int j = 0; //index in B

    while( i < LA )
    {
        while( j < LB )
        {
            //during this process
            //i is fixed
            //(i+j) is incremented

            auto k = ( i + j ) % LA;
            if( A[k] == B[j] )
            {
                ++j;

            }
            else
            {
                break;
            }
        }


        if( j == LB )
        {
            //found match
            return ( i + j - 1 ) / LA + 1;
        }


        if( j > 0 )
        {
            //since i is fixed during comparing
            //we need to advance i to the
            //first index that does not match
            //which is i + j - F[j-1]

            //In normal KMP, since i is
            //incremented during comparing
            //the following is not needed
            i += j - F[j - 1];

            //normal KMP operation
            j = F[j - 1];
        }
        else
        {
            //same as normal KMP operation
            ++i;
        }

    }
    return -1;
}
\end{lstlisting}
