\section{691 --- Stickers to Spell Word}
We are given $N$ different types of \textbf{stickers}. Each sticker has a \textbf{lowercase} English word on it.
\par
You would like to spell out the given \textbf{target} string $T$ by cutting individual letters from your collection of stickers and rearranging them.
\par
You can use each sticker more than once if you want, and you have infinite quantities of each sticker.
\par
What is the minimum number of stickers that you need to spell out the target? If the task is impossible, return -1.
\paragraph{Example 1:}
\begin{flushleft}
\textbf{Input:}

\lstinline[language=C++, basicstyle=\small\ttfamily, keywordstyle=\bfseries\color{green!40!black}]|stickers = ["with", "example", "science"]|, \lstinline[language=C++, basicstyle=\small\ttfamily, keywordstyle=\bfseries\color{green!40!black}]|target = "thehat"|.

\textbf{Output}: 3

\textbf{Explanation}:

We can use 2 \lstinline[language=C++, basicstyle=\small\ttfamily, keywordstyle=\bfseries\color{green!40!black}]|"with"| stickers, and 1 \lstinline[language=C++, basicstyle=\small\ttfamily, keywordstyle=\bfseries\color{green!40!black}]|"example"| sticker.

After cutting and rearrange the letters of those stickers, we can form the target \lstinline[language=C++, basicstyle=\small\ttfamily, keywordstyle=\bfseries\color{green!40!black}]|"thehat"|.

Also, this is the minimum number of stickers necessary to form the target string.
\end{flushleft}

\paragraph{Example 2:}
\begin{flushleft}
\textbf{Input}:

\lstinline[language=C++, basicstyle=\small\ttfamily, keywordstyle=\bfseries\color{green!40!black}]|stickers=["notice", "possible"]|, \lstinline[language=C++, basicstyle=\small\ttfamily, keywordstyle=\bfseries\color{green!40!black}]|target = "basicbasic"|

\textbf{Output}: $-1$

\textbf{Explanation}:

We can't form the target \lstinline[language=C++, basicstyle=\small\ttfamily, keywordstyle=\bfseries\color{green!40!black}]|"basicbasic"| from cutting letters from the given stickers.
\end{flushleft}


\paragraph{Note:}
\begin{itemize}
\item \textbf{stickers} has length in the range $[1, 50]$.
\item \textbf{stickers} consists of \textbf{lowercase} English words (without apostrophes).
\item \textbf{target} has length in the range $[1, 15]$, and consists of \textbf{lowercase} English letters.
\item In all test cases, all words were chosen randomly from the 1000 most common US English words, and the target was chosen as a concatenation of two random words.
\item The time limit may be more challenging than usual. It is expected that a 50 sticker test case can be solved within 35ms on average.
\end{itemize}

\subsection{Dynamic Programming}
In this approach, suppose the length of $T$ is $L$. Since the given $L$ is no large than 15, we can make use of a integer $s$  such that if the $i$--th bit of $s$ is 1, it means $T[i]$ can be spelled by the sticker. 


For each state $s$, we look at what happens to it after applying a sticker. For each letter in the sticker that can satisfy an unset bit of $s$, we set the bit. At the end, we know now is the result of applying that sticker to state. 

Thus, we make use of an array $F$. For each state $s$, $F[s]$ gives the minimum number of stickers such that: for each set bit $i$ in $s$, $T[i]$ can be spelled by the stickers. 

We starting from zero state. For each current state $s$, we apply a sticker and will get next state $s_1$. Then $F[s_1]=\min(F[s_1], F[s]+1)$. After checking all $2^L$ states, $F[2^L-1]$ is the answer.

\subsubsection{Algorithm}
The inputs of the following procedure are
\begin{itemize}
    \item $S$ --- The input sticker array
    \item $N$ --- The length of $S$
    \item $T$ --- The target string
    \item $L$ --- The length of $T$
\end{itemize}

\setcounter{lstlisting}{0}
\begin{lstlisting}[style=customc, caption={Dynamic Progamming}]
int minStickers( vector<string>& stickers, string target )
{
    int L = static_cast<int>( target.size() );
    //the space of states
    int NS = ( 1 << L );

    vector<int> F( NS, -1 );
    //at start, all bits are 0
    //no any sticker is needed
    F[0] = 0;

    //try each state
    for( int state = 0; state < NS; ++state )
    {
        if( F[state] < 0 )
        {
            //this state cannot
            //be reached
            continue;
        }

        for( const auto& sticker : stickers )
        {
            //apply one sticker to
            //current state
            //to get next_state
            int next_state = state;

            for( auto ch : sticker )
            {
                for( size_t i = 0; i < target.size(); ++i )
                {
                    if( next_state & ( 1 << i ) )
                    {
                        //we already spell out target[i]
                        continue;
                    }
                    if( target[i] == ch )
                    {
                        //we can spell letter target[i]
                        next_state |= ( 1 << i );
                        //we try next letter in the sticker
                        break;
                    }
                }//loop target
            }//loop letter in sticker

            //now next_state is the result of applying the sticker
            if( ( F[next_state] < 0 ) || ( F[next_state] > F[state] + 1 ) )
            {
                //the number of stickers
                //for reach next_state
                //is the number of stickers
                //to reach state plus 1
                F[next_state] = F[state] + 1;
            }
        }// loop stickers
    }//loop state

    return F.back();
}
\end{lstlisting}