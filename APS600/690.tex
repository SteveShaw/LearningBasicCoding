\section{690 --- Employee Importance}

\textbf{Easy}

You are given a data structure of employee information, which includes the employee's \textbf{unique id}, his \textbf{importance} value and his \textbf{direct} subordinates' id.

For example, employee 1 is the leader of employee 2, and employee 2 is the leader of employee 3. They have importance value 15, 10 and 5, respectively. Then employee 1 has a data structure like \fcj{[1, 15, [2]]}, and employee 2 has \fcj{[2, 10, [3]]}, and employee 3 has \fcj{[3, 5, []]}. Note that although employee 3 is also a subordinate of employee 1, the relationship is not direct.

Now given the employee information of a company, and an employee id, you need to return the total importance value of this employee and all his subordinates.

\paragraph{Example 1:}

\begin{flushleft}
\textbf{Input}: \fcj{[[1, 5, [2, 3]], [2, 3, []], [3, 3, []]]}, 1

\textbf{Output}: 11

\textbf{Explanation}:

Employee 1 has importance value 5, and he has two direct subordinates: employee 2 and employee 3. They both have importance value 3. So the total importance value of employee 1 is $5 + 3 + 3 = 11$.
\end{flushleft}
 

\paragraph{Note:}

\begin{itemize}
\item One employee has at most one direct leader and may have several subordinates.

\item The maximum number of employees won't exceed 2000.
\end{itemize}

\subsection{DFS}
We need a hash map which map a employee's id to itself. Then we make use of DFS to recursively get all the subordinates of the given id.

\setcounter{lstlisting}{0}
\begin{lstlisting}[style=customc, caption={DFS}]
int getImportance( vector<Employee*> employees, int id )
{
    unordered_map<int, Employee*> m;
    //build the map between id and Employee
    for( auto p : employees )
    {
        m[p->id] = p;
    }
    return dfs( id, m );
}
int dfs( int id, const unordered_map<int, Employee*>& m )
{
    //recursively get the
    //total importance values of
    //all subordinates of id
    auto p = m[id];
    int sum = p->importance;
    for( int id : p->subordinates() )
    {
        sum += dfs( id, m );
    }
    return sum;
}
\end{lstlisting}

\paragraph{Related Problems}
\begin{itemize}
\item \textbf{339. Nested List Weight Sum}.
\end{itemize}