\section{672 --- Bulb Switcher II}
There is a room with $n$ lights which are turned on initially and 4 buttons on the wall. After performing exactly $m$ unknown operations towards buttons, you need to return how many different kinds of status of the $n$ lights could be.

Suppose $n$ lights are labeled as number $[1, 2, 3, \ldots, n]$, function of these 4 buttons are given below:

\begin{enumerate}
\item Flip all the lights.
\item Flip lights with even numbers.
\item Flip lights with odd numbers.
\item Flip lights with $(3k + 1)$ numbers, $k = 0, 1, 2, \ldots$

\end{enumerate}
 

\paragraph{Example 1:}
\begin{flushleft}

\textbf{Input}: $n = 1$, $m = 1$.

\textbf{Output}: 2

\textbf{Explanation}: Status can be: \lstinline[language=Java, basicstyle=\small\ttfamily, keywordstyle=\bfseries\color{green!40!black}]|[on], [off]|
\end{flushleft}
 

\paragraph{Example 2:}
\begin{flushleft}

\textbf{Input}: $n = 2$, $m = 1$.

\textbf{Output}: 3

\textbf{Explanation}: Status can be: \lstinline[language=Java, basicstyle=\small\ttfamily, keywordstyle=\bfseries\color{green!40!black}]|[on, off], [off, on], [off, off]|
\end{flushleft}
 

\paragraph{Example 3:}

\begin{flushleft}


\textbf{Input}: $n = 3$, $m = 1$.

\textbf{Output}: 4

\textbf{Explanation}: 

Status can be: \lstinline[language=Java, basicstyle=\small\ttfamily, keywordstyle=\bfseries\color{green!40!black}]|[off, on, off], [on, off, on], [off, off, off], [off, on, on]|.
\end{flushleft}

\paragraph{Note:} 
\begin{itemize}
\item $n$ and $m$ both fit in range \lstinline[language=Java, basicstyle=\small\ttfamily, keywordstyle=\bfseries\color{green!40!black}]|[0, 1000]|.
\end{itemize}

\subsection{Reduce Search Space}
Give the conditions, we know the search space is very large: there will be $2^N$ states of lights and $4^M$ operation sequences. Reducing the search space must be done first.

Actually, the first 6 lights uniquely determine the rest of the lights. This is because every operation that modifies the $x$-th light also changes the $(x+6)$-th light.

Regarding to operations: 

\begin{itemize}
\item operations are commutative: doing operation $A$ followed by $B$ is the same as doing operation $B$ followed by $A$. So we can assume we do all the operations in order. 
\item Doing the same operation twice in a row is the same as doing nothing. Hence, we can mark the operation state as 1 when it is running odd times and 0 for even times. For example, if current operation state is $9$ (\lstinline[language=C++, basicstyle=\small\ttfamily, keywordstyle=\bfseries\color{green!40!black}]|0b1001|), the operation 1 and 4 will perform odd times.

For the four operations, we will have total $2^4=16$ operation states by treating each operation as a bit position in a integer. Thus if $m > 4$, the results will repeat the results from $m \leq 4$. 
\end{itemize}

Now, given $m$, we will try all 16 operation states to see if a state is possible to make $m$ operations. In each state, we firstly check how many operations will be taken by odd times, say the result is $x$.
\begin{itemize}
\item If $x > m$, we will skip this state.
\item If $x \leq m$, this state is possible only if $\bmod(x, 2)=\bmod(m,2)$. The reason is that: in this case, we need to perform $x$ operations odd times. If $x$ is odd number, the total number of operations would be odd. Conversely, if $x$ is even number, total number of operations must be even.  


\item For each possible state, since the first 6 lights uniquely determine the rest of the digits, we can make use of 6-bits in a integer to record the states of the bulbs. Put these states into a hash set. Finally, the number of elements in this hash set will be the answer
\end{itemize}

\setcounter{lstlisting}{0}
\begin{lstlisting}[style=customc, caption={Reduce Search Space}]
int flipLights( int n, int m )
{
    n = ( min )( n, 6 );

    auto get_ones = []( int n )
    {
        int ones = 0;
        while( n )
        {
            ++ones;
            n = n & ( n - 1 );
        }

        return ones;
    };

    //n may be less than 6
    //in this case, we don't need
    //6 bits, we only n bits
    int shift = ( max )( 0, 6 - n );

    int parity = ( m & 1 );

    unordered_set<int> l_states;

    for( int state = 0; state < 16; ++state )
    {
        int odd_opers = get_ones( state );

        //the number of odd operations should less than m
        //and the odd/even should is same as m
        if( ( odd_opers <= m ) && ( ( odd_opers & 1 ) == parity ) )
        {
            int mask = 1;

            int lights = 0;

            if( mask & state )
            {
                //do operation 1: 111111
                lights ^= ( 63 >> shift );
            }

            mask <<= 1;

            if( mask & state )
            {
                //do operation 2: 010101
                lights ^= ( 21 >> shift );
            }

            mask <<= 1;

            if( mask & state )
            {
                //do operation 3: 101010
                lights ^= ( 42 >> shift );
            }

            mask <<= 1;

            if( mask & state )
            {
                //do operation 4: 100100
                lights ^= ( 36 >> shift );
            }

            l_states.insert( lights );
        }

    }

    return static_cast< int >( l_states.size() );
}
\end{lstlisting}


