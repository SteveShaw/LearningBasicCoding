\section{600 --- Non-negative Integers without Consecutive Ones}
Given a positive integer $n$, find the number of non-negative integers less than or equal to $n$, whose binary representations do NOT contain consecutive ones.

\paragraph{Example 1:}

\begin{flushleft}
\textbf{Input}: 5

\textbf{Output}: 5

\textbf{Explanation}: 

Here are the non-negative integers $\leq$ 5 with their corresponding binary representations:

0 : 0

1 : 1

2 : 10

3 : 11

4 : 100

5 : 101

Among them, only integer 3 disobeys the rule (two consecutive ones) and the other 5 satisfy the rule.
\end{flushleft} 

\paragraph{Note:} 
\begin{itemize}
\item $1 \leq n \leq 10^9$
\end{itemize}

\subsection{Bit Manipulation}
Suppose $F[n]$ is the count of binary numbers with $n$ bits such that these numbers don't contain consecutive 1's. 

If we know the value of $F[n−1]$ and $F[n-2]$, in order to generate the required binary numbers with $n$ bits, 
\begin{itemize}
\item we can append a 0 to all the binary numbers with $n-1$ bits without creating an invalid number. These numbers give a number of $F[n−1]$ to be included in $F[n]$.
\item But, we can't append a 1 to all these numbers with $n-1$ bits, since it could lead to the presence of two consecutive ones. We can only append 1 to all the numbers with $n-2$ bits ending with $01$. This gives a number of $F[n−2]$ to be included in $F[n]$. 
\item Thus, in total, $F[n]=F[n−1]+F[n−2]$.
\end{itemize}
Based on $F$, the approach to find the count of non-negative integers less than given number $x$ is demonstrated by an example as below:
\begin{itemize}
\item Suppose $x=1010100$ which is a 7-bit number. Starting with MSB of $x$. If we fix a 0 at the MSB position, and find out the count of 6 bit numbers with no two consecutive 1's, these 6-bit numbers will lie in the range $0000000 \longrightarrow 0111111$. The count of these numbers is actually $F[6]$.
\item But, if we try to fix 1 at the MSB, the numbers considered will lie in the range $1000000 \longrightarrow 1111111$. But the integers no larger than $x$ is from $1000000 \longrightarrow 1010100$. Hence, if fix 1 at MSB only, we will include numbers that may larger than $x$. Therefor, we can't fix 1 at the MSB and consider all the 6-bit numbers at the LSBs. We need to go forward to the next bit.
\item To do that, fix 1 at the MSB, and move forward to proceed to the next bit. Since the bit is zero, we can't substitute a 1 here, since doing so will lead to generation of numbers exceeding $x$. Thus, the only option left here is to keep 0 at this bit position.
\item However, by starting with $10$, the numbers in the range $1000000 \longrightarrow 1011111$ can still lead to number exceeding $x$. Thus, we cannot only keep 0 at the second bit position, and we have to proceed further.
\item Now, the 3rd bit position is 1. Now, we can fix 0 at this position and find out the count of 4-bit consecutive numbers with no two consecutive 1's, which is $F[4]$. This will cover the range $1000000\longrightarrow 1001111$.
\item Based on the above discussion
\begin{enumerate}
\item Fix 1 at the 3rd bit position, and proceed with the 4th bit, which is 0.
\item Fix 0 at the 4th bit position, and proceed with the 5th bit, which is 1.
\item We fix 0 here and find out the count of 2-bit consecutive numbers with no two consecutive 1's, which is $F[2]$. Then proceed with the 6th bit, which is 0.
\item Fix 0 at the 6th bit position, and proceed to the 7th bit, which is 0.
\item Fix 0 at the 7th bit position.
\end{enumerate}
\item Now, we can see, that based on the above procedure, the numbers in the range $1000000\longrightarrow 1111111$, $1000000\longrightarrow1111111$, $1000000\longrightarrow1001111$ have been considered and the counts for these ranges have been obtained as $F[6]$, $F[4]$ and $F[2]$ respectively. Now, only $1010100$ is pending to be considered in the required count. Since, it doesn't contain any consecutive 1's, we add a 1 to the total count obtained till now to consider this number. Thus, the result returned is $F[6]+F[4]+F[2]+1$.
\item When scanning the input number $x$, if we meet two consecutive 1's at $b$-th and $(b-1)$th bit, we know that we cannot append either 0 or 1 after $(b-1)$th bit. Therefore, the process of scanning will be stopped.
\end{itemize}