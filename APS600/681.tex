\section{681 --- Next Closest Time}
Given a time represented in the format \lstinline[language=Java, basicstyle=\small\ttfamily, keywordstyle=\bfseries\color{green!40!black}]|"HH:MM"|, form the next closest time by reusing the current digits. There is no limit on how many times a digit can be reused.

You may assume the given input string is always valid. For example, \lstinline[language=Java, basicstyle=\small\ttfamily, keywordstyle=\bfseries\color{green!40!black}]|"01:34"|, \lstinline[language=Java, basicstyle=\small\ttfamily, keywordstyle=\bfseries\color{green!40!black}]|"12:09"| are all valid. \lstinline[language=Java, basicstyle=\small\ttfamily, keywordstyle=\bfseries\color{green!40!black}]|"1:34"|, \lstinline[language=Java, basicstyle=\small\ttfamily, keywordstyle=\bfseries\color{green!40!black}]|"12:9"| are all invalid.

\section{Example 1:}

\begin{flushleft}

\textbf{Input}: \lstinline[language=Java, basicstyle=\small\ttfamily, keywordstyle=\bfseries\color{green!40!black}]|"19:34"|

\textbf{Output}: \lstinline[language=Java, basicstyle=\small\ttfamily, keywordstyle=\bfseries\color{green!40!black}]|"19:39"|

\textbf{Explanation}: 

The next closest time choosing from digits 1, 9, 3, 4, is \lstinline[language=Java, basicstyle=\small\ttfamily, keywordstyle=\bfseries\color{green!40!black}]|"19:39"|, which occurs 5 minutes later.  It is not 19:33, because this occurs 23 hours and 59 minutes later.

\end{flushleft}

\paragraph{Example 2:}

\begin{flushleft}
\textbf{Input}: \lstinline[language=Java, basicstyle=\small\ttfamily, keywordstyle=\bfseries\color{green!40!black}]|"23:59"|

\textbf{Output}: \lstinline[language=Java, basicstyle=\small\ttfamily, keywordstyle=\bfseries\color{green!40!black}]|"22:22"|

\textbf{Explanation}: 

The next closest time choosing from digits 2, 3, 5, 9, is \lstinline[language=Java, basicstyle=\small\ttfamily, keywordstyle=\bfseries\color{green!40!black}]|"22:22"|. It may be assumed that the returned time is next day's time since it is smaller than the input time numerically.

\end{flushleft}

\subsection{Simulation}
Since there are $24 \times 60-1=1339$ possible minute from current time, we can try each minute and comparing the digits to the input time string.

\setcounter{lstlisting}{0}
\begin{lstlisting}[style=customc, caption={Simulation}]
string nextClosestTime( string time )
{
    int m[10] =  {0};

    for( auto c : time )
    {
        if( c != ':' )
        {
            m[c - '0'] = 1;
        }
    }
    int hour = ( time[0] - '0' ) * 10 + ( time[1] - '0' );
    int minute = ( time[3] - '0' ) * 10 + ( time[4] - '0' );

    int total_minute = hour * 60 + minute;

    int tm[4];


    for( int i = 1; i < 24 * 60; ++i )
    {
        ++total_minute;

        //important: we have to
        //make sure the minutes
        //are in one day
        if( total_minute > 24 * 60 )
        {
            total_minute -= 24 * 60;
        }

        hour = total_minute / 60;
        minute = total_minute - 60 * hour;

        tm[0] = hour / 10;
        tm[1] = hour - 10 * tm[0];

        tm[2] = minute / 10;
        tm[3] = minute - 10 * tm[2];

        bool flag = true;
        for( int t : tm )
        {
            if( m[t] == 0 )
            {
                flag = false;
                break;
            }
        }

        if( flag )
        {
            //this is the closest time
            //share digits from the input
            string fmt( "00:00" );
            fmt[0] = tm[0] + '0';
            fmt[1] = tm[1] + '0';

            fmt[3] = tm[2] + '0';
            fmt[4] = tm[3] + '0';

            return fmt;
        }
    }

    return time;
}
\end{lstlisting}

\subsection{Try All Possible Combinations}
We have up to 4 different allowed digits, which naively gives us $4^4=256$ possible times. For each possible time, let's check that it can be displayed on a clock: ie., hours is less than 24 and minutes less than 60. The best possible time $t$ which is not equal to start is the one with the smallest $ (t - s)\bmod  (24 \times 60)$, as this represents the time that has elapsed since start, and where the modulo operation is taken to be always non-negative. Notice when $t < s$, we need to add $24\times 60$ to $t$ to represent the time in the next day.

\begin{lstlisting}[style=customc, caption={Try Every Given Digit}]
string nextClosestTime( string time )
{
    int m[10] =  {0};

    for( auto c : time )
    {
        if( c != ':' )
        {
            m[c - '0'] = 1;
        }
    }
    int hour = ( time[0] - '0' ) * 10 + ( time[1] - '0' );
    int minute = ( time[3] - '0' ) * 10 + ( time[4] - '0' );

    int start = hour * 60 + minute;

    int min_diff = 24 * 60;

    int ans = start;

    //try all possible combinations
    //from the digits in the given time string
    for( int i = 0; i < 10; ++i )
    {
        for( int j = 0; j < 10; ++j )
        {
            if( !m[i] || !m[j] )
            {
                continue;
            }

            hour = i * 10 + j;

            if( hour >= 24 )
            {
                continue;
            }

            for( int x = 0; x < 10; ++x )
            {
                for( int y = 0; y < 10; ++y )
                {
                    if( !m[x] || !m[y] )
                    {
                        continue;
                    }


                    minute = x * 10 + y;

                    if( minute >= 60 )
                    {
                        continue;
                    }

                    int cur = hour * 60 + minute;

                    int diff = cur - start;

                    if( diff < 0 )
                    {
                        //we will treat this time
                        //as next day's time
                        diff += 24 * 60;
                    }


                    if( ( diff > 0 ) && ( diff < min_diff ) )
                    {
                        //record minimum difference
                        min_diff = diff;
                        ans = cur;
                    }
                }
            }
        }
    }


    //change to string
    string fmt( "00:00" );
    hour = ans  / 60;
    minute = ans -  60 * hour;

    fmt[0] = hour / 10 + '0';
    fmt[1] = hour % 10 + '0';

    fmt[3] = minute / 10 + '0';
    fmt[4] = ( minute % 10 ) + '0';

    return fmt;
}
\end{lstlisting}