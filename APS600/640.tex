\section{640 --- Solve the Equation}
Solve a given equation and return the value of $x$ in the form of string $x=n$. The equation contains only plus, minus operation, the variable $x$ and its coefficient.

If there is no solution for the equation, return \textbf{No solution}.

If there are infinite solutions for the equation, return \textbf{Infinite solutions}.

If there is exactly one solution for the equation, we ensure that the value of $x$ is an integer.

\paragraph{Example 1:}

\begin{flushleft}
\textbf{Input}: $x+5-3+x=6+x-2$

\textbf{Output}: $x=2$

\end{flushleft}

\paragraph{Example 2:}

\begin{flushleft}
\textbf{Input}: $x=x$

\textbf{Output}: \textbf{Infinite solutions}

\end{flushleft}

\paragraph{Example 3:}

\begin{flushleft}
Input: $2x=x$

Output: $x=0$

\end{flushleft}

\paragraph{Example 4:}

\begin{flushleft}
\textbf{Input}: $2x+3x-6x=x+2$

\textbf{Output}: $x=-1$

\end{flushleft}

\paragraph{Example 5:}

\begin{flushleft}
\textbf{Input}: $x=x+2$

\textbf{Output}: \textbf{No solution}

\end{flushleft}

\subsection{Group Coefficients}
\begin{itemize}
\item Split the function per the equal sign.
\item Parse two expressions split by the equal sign.
\item Iterating the expression, whenever a plus or minus sign is met
\begin{enumerate}
\item Accumulate the coefficients of $x$ by checking if last letter is $x$.
\item If it is not, accumulate the values.
\end{enumerate}
\item To dealing with $0\times x$, we set the starting number as $-1$ to indicate whether a $x$ has a coefficient or not.
\end{itemize}