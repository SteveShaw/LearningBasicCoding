\section{87 -- Scramble String}
Given a string $S_1$, we may represent it as a binary tree by partitioning it to two non-empty substrings recursively.
\par
Below is one possible representation of $S_1 = \texttt{great}$:
\begin{figure}[H]
\begin{tikzpicture}
[mynode/.style={circle, draw, minimum size=1.2cm, fill=gray!15}]
\node(){};
\node[mynode] (t1){\texttt{great}};
\node[mynode, below = 8mm of t1, xshift=-24mm] (t2) {\texttt{gr}};
\node[mynode, below = 8mm of t2, xshift=-8mm] (t4) {\texttt{g}};
\node[mynode, below = 8mm of t2, xshift=8mm] (t5) {\texttt{r}};
\node[mynode, below = 8mm of t1, xshift=24mm] (t3) {\texttt{eat}};
\node[mynode, below = 8mm of t3, xshift=-16mm] (t6) {\texttt{e}};
\node[mynode, below = 8mm of t3, xshift=16mm] (t7) {\texttt{at}};
\node[mynode, below = 8mm of t7, xshift=-8mm] (t8) {\texttt{a}};
\node[mynode, below = 8mm of t7, xshift=8mm] (t9) {\texttt{t}};
\draw[->,>=stealth'] (t1)--(t2);
\draw[->,>=stealth'] (t1)--(t3);
\draw[->,>=stealth'] (t2)--(t4);
\draw[->,>=stealth'] (t2)--(t5);
\draw[->,>=stealth'] (t3)--(t6);
\draw[->,>=stealth'] (t3)--(t7);
\draw[->,>=stealth'] (t7)--(t8);
\draw[->,>=stealth'] (t7)--(t9);
\end{tikzpicture}
\end{figure}
To scramble the string, we may choose any non-leaf node and swap its two children.
\par
For example, if we choose the node \texttt{gr} and swap its two children, it produces a scrambled string \texttt{rgeat}.
\begin{figure}[H]
\begin{tikzpicture}
[mynode/.style={circle, draw, minimum size=1.2cm, fill=gray!15}]
\node(){};
\node[mynode] (t1){\texttt{rgeat}};
\node[mynode, below = 8mm of t1, xshift=-24mm] (t2) {\texttt{rg}};
\node[mynode, below = 8mm of t2, xshift=-8mm] (t4) {\texttt{r}};
\node[mynode, below = 8mm of t2, xshift=8mm] (t5) {\texttt{g}};
\node[mynode, below = 8mm of t1, xshift=24mm] (t3) {\texttt{eat}};
\node[mynode, below = 8mm of t3, xshift=-16mm] (t6) {\texttt{e}};
\node[mynode, below = 8mm of t3, xshift=16mm] (t7) {\texttt{at}};
\node[mynode, below = 8mm of t7, xshift=-8mm] (t8) {\texttt{a}};
\node[mynode, below = 8mm of t7, xshift=8mm] (t9) {\texttt{t}};
\draw[->,>=stealth'] (t1)--(t2);
\draw[->,>=stealth'] (t1)--(t3);
\draw[->,>=stealth'] (t2)--(t4);
\draw[->,>=stealth'] (t2)--(t5);
\draw[->,>=stealth'] (t3)--(t6);
\draw[->,>=stealth'] (t3)--(t7);
\draw[->,>=stealth'] (t7)--(t8);
\draw[->,>=stealth'] (t7)--(t9);
\end{tikzpicture}
\end{figure}
We say that \texttt{rgeat} is a scrambled string of \texttt{great}.
\par
Similarly, if we continue to swap the children of nodes \texttt{eat} and \texttt{at}, it produces a scrambled string \texttt{rgtae}.
\begin{figure}[H]
\begin{tikzpicture}
[mynode/.style={circle, draw, minimum size=1.2cm, fill=gray!15}]
\node(){};
\node[mynode] (t1){\texttt{rgtae}};
\node[mynode, below = 8mm of t1, xshift=-24mm] (t2) {\texttt{rg}};
\node[mynode, below = 8mm of t2, xshift=-8mm] (t4) {\texttt{r}};
\node[mynode, below = 8mm of t2, xshift=8mm] (t5) {\texttt{g}};
\node[mynode, below = 8mm of t1, xshift=24mm] (t3) {\texttt{tae}};
\node[mynode, below = 8mm of t3, xshift=16mm] (t6) {\texttt{e}};
\node[mynode, below = 8mm of t3, xshift=-16mm] (t7) {\texttt{ta}};
\node[mynode, below = 8mm of t7, xshift=-8mm] (t8) {\texttt{t}};
\node[mynode, below = 8mm of t7, xshift=8mm] (t9) {\texttt{a}};
\draw[->,>=stealth'] (t1)--(t2);
\draw[->,>=stealth'] (t1)--(t3);
\draw[->,>=stealth'] (t2)--(t4);
\draw[->,>=stealth'] (t2)--(t5);
\draw[->,>=stealth'] (t3)--(t6);
\draw[->,>=stealth'] (t3)--(t7);
\draw[->,>=stealth'] (t7)--(t8);
\draw[->,>=stealth'] (t7)--(t9);
\end{tikzpicture}
\end{figure}
We say that \texttt{rgtae} is a scrambled string of \texttt{great}.
\par
Given two strings $S_1$ and $S_2$ of the same length, determine if $S_2$ is a scrambled string of $S_1$
\paragraph{Example 1:}
\begin{flushleft}
\textbf{Input}: $S_1 = \texttt{great}$, $S_2 = \texttt{rgeat}$
\\
\textbf{Output}: 1 (\texttt{true})
\end{flushleft}
\paragraph{Example 2:}
\begin{flushleft}
\textbf{Input}: $S_1 = \texttt{abcde}$, $S_2 = \texttt{caebd}$
\\
\textbf{Output}: 0 (\texttt{false})
\end{flushleft}
\subsection{Recrusive Approach}
基本思路是将$S_1$分为两个substring,长度分别为$\ell$和$L-\ell$。那么 $S_1$和$S_2$为scramble当且仅当
\begin{itemize}
\item $S_1[0\ldots\ell-1]$和$S_2[0\ldots \ell-1]$ 是scramble,同时$S_1[\ell\ldots L-1]$和$S_2[\ell\ldots L-1]$也是scramble. 或者
\item $S_1[0\ldots\ell-1]$和$S_2[\ell\ldots L-1]$ 是scramble,同时$S_1[L-\ell\ldots L-1]$和$S_2[0\ldots L-\ell-1]$也是scramble
\end{itemize} 
以 \texttt{rgeat} 和 \texttt{great} 为例,\texttt{rgeat} 可分成 \texttt{rg} 和 \texttt{eat} 两段, \texttt{great} 可分成 \texttt{gr} 和 \texttt{eat} 两段,\texttt{rg} 和 \texttt{gr} 是scrambled的, \texttt{eat} 和 \texttt{eat} 当然是scrambled。因此递归算法需要测试每个index进行上述判断。

\setcounter{lstlisting}{0}
\begin{lstlisting}[style=customc, caption={Recursion}]
bool isScramble( string s1, string s2 )
{

    if( s1.size() != s2.size() )
    {
        //if the size of both string are different
        //cannot be scramble strings
        return false;
    }
    string_view vs1( s1.c_str(), s1.size() );
    string_view vs2( s2.c_str(), s2.size() );
    return dfs( vs1, vs2 );
}
//helper function
bool dfs( string_view vs1, string_view vs2 )
{
    if( vs1 == vs2 )
    {
        return true;
    }
    //check if vs1 and vs2 has same letters
    int count[128] = {0};
    for( size_t i = 0; i < vs1.size(); ++i )
    {
        char c1 = vs1[i];
        char c2 = vs2[i];
        count[c1 - '\0'] += 1;
        count[c2 - '\0'] -= 1;
    }
    for( int i = 0; i < 128; ++i )
    {
        if( count[i] )
        {
            //vs1 and vs2 have different letters
            return false;
        }
    }
    auto L = vs1.size();
    for( size_t i = 1; i < L; ++i )
    {
        auto a1 = vs1.substr( 0, i );
        auto a2 = vs1.substr( i );
        auto b1 = vs2.substr( 0, i );
        auto b2 = vs2.substr( i );

        //test s1[0,i-1] vs s2[0,i-1]
        //and s1[i, L-1] vs s2[i, L-1]
        if( dfs( a1, b1 ) && dfs( a2, b2 ) )
        {
            return true;
        }

        auto c1 = vs2.substr( L - i );
        auto c2 = vs2.substr( 0, L - i );
        //test s1[0,i-1] vs s2[L-i,L-1]
        //and s1[L-i, L-1] vs s2[0, i-1]
        if( dfs( a1, c1 ) && dfs( a2, c2 ) )
        {
            return true;
        }
    }
    return false;
}
\end{lstlisting}
%Procedure $B$ check if two strings $S_1$ and $S_2$ are scramble string or not. The length of $S_1$ and $S_2$ are all $L$
%\setcounter{algorithm}{0}
%\begin{algorithm}[H]
%\caption{Recursive Approach Main Procedure}
%\begin{algorithmic}[1]
%\Procedure{$B$}{$S_1, S_2, L$}
%\State \Return $\Gamma(S_1, S_2, L, L)$
%\EndProcedure
%\end{algorithmic}
%\end{algorithm}
%Function $\Gamma$ recursively check if two strings $S_1$ with length $L_1$ and $S_2$ with length $L_2$ are scramble strings or not.
%\begin{algorithm}[H]
%\caption{Recursive Approach Function}
%\begin{algorithmic}[1]
%\Function{$\Gamma$}{$S_1, S_2$}
%\If{$|S_1|\neq |S_2|$} \Comment The length of two strings are not equal
%\State \Return 0 \Comment \texttt{false}
%\EndIf
%\State $\delta$ as $\delta[0]=\ldots=\delta[255]:=0$ The character counter array
%\State $L := |S_1|$
%\For{$i:=0$ to $L-1$}
%\State $\delta[S_1[i]]\gets \delta[S_1[i]] + 1$
%\State $\delta[S_2[i]]\gets \delta[S_2[i]] - 1$
%\EndFor
%\For{$j:=0$ \textbf{to} 255}
%\If{$\delta[j]\neq 0$}
%\State \Return 0 \Comment Found different character in $S_1$ and $S_2$
%\EndIf
%\EndFor
%\For{$\ell:=0$ \textbf{to} $L$ }
%\State $\alpha_1:=S_1[0,\ell-1], \alpha_2:=S_1[\ell, L-1]$
%\State $\beta_1:=S_2[0,\ell-1], \beta_2:=S_2[\ell, L-1]$ \Comment Check $S_2$ from left to right
%\If{$\Gamma(\alpha_1, \beta_1)=1$ \textbf{and} $\Gamma(\alpha_2, \beta_2)=1$} \Comment Recursively check $\alpha_1,\;\beta_1$ and $\alpha_2,\;\beta_2$
%\State \Return 1 \Comment $S_1$ and $S_2$ are scramble
%\EndIf
%\algstore{87algo}
%\end{algorithmic}
%\end{algorithm}
%\begin{algorithm}[H]
%\begin{algorithmic}[1]
%\algrestore{87algo}
%\State $\beta_1:=S_2[0,L-\ell-1], \beta_2:=S_2[L-\ell, L-1]$ \Comment Check $S_2$ from right to left
%\If{$\Gamma(\alpha_1, \beta_2)=1$ \textbf{and} $\Gamma(\alpha_2, \beta_1)=1$} \Comment Recursively check $\alpha_1,\;\beta_2$ and $\alpha_2,\;\beta_1$
%\State \Return 1 \Comment $S_1$ and $S_2$ are scramble
%\EndIf
%\EndFor
%\State \Return 0 \Comment $S_1$ and $S_2$ are not scramble
%\EndFunction
%\end{algorithmic}
%\end{algorithm}
\subsection{Dynamic Programming Approach}
上面递归算法复杂度会达到指数级因为重复进行了很多次判断。如果用Dynamic Programming方法,这里的Recursive function $F$其实是一个三维数组。$F[i][j][\ell]$表示字符串$S_1[i, i+\ell-1]$和$S_2[j,j+\ell-1]$是否为scramble。
\par
接着就是建立recursive formula,也就是怎么根据历史信息来得到$F[i][j][\ell]$。首先是把当前$S_1[i, i+\ell-1]$字符串分成左右两部分,假设左半部分长度为$k$,即划分成$S_1[i, i+k-1]$ and $S_1[i+k, i+\ell-1]$。分两种情况进行比较:
\begin{enumerate}
    \item 分别比较$S_1[i,i+k-1]$ with $S_2[j, j+k-1]$, $S_1[i+k, i+\ell-1]$ with $S_2[j+k, j+\ell-1]$,即$S_1[i, i+\ell-1]$的前$k$个字符和$S_2[j, j+\ell-1]$的前$k$个字符进行比较,同时比较$S_1[i, i+\ell-1]$的后$\ell-k$个字符和$S_2[j, j+\ell-1]$的后$\ell-k$个字符
    \item 分别比较$S_1[i, i+k-1]$ with $S_2[j+\ell-k, j+\ell-1]$, $S_1[i+k, i+\ell-1]$ with $S_2[j, j+\ell-k-1]$,即$S_1[i, i+\ell-1]$的前$k$个字符和$S_2[j, j+\ell-1]$的后$k$个字符进行比较,同时比较$S_1[i, i+\ell-1]$的后$\ell-k$个字符和$S_2[j, j+\ell-1]$的前$\ell-k$个字符
\end{enumerate}
如果以上两种情况有一种成立,说明$S_1[i,i+\ell-1]$和$S_2[j, j+\ell-1]$是scramble的。
\par
而对于判断这些左右部分是不是scramble则是已经计算并保存在$F$中的,因为长度小于$\ell$的所有情况都在前面计算过了(也就是长度是最外层循环)。
\par
上面说的是在按照一个给定的长度把$S_1$分成两部分的情况,而对于$S_1[i\ldots i+\ell-1]$来说左边的长度可以从1到$\ell-1$,在这些划分中只要有一个满足上面两种情况任何一个,那么$S_1[i, i+\ell-1]$和$S_2[j, j+\ell-1]$就是scramble。
\par
注意到$S_1[i,i+k-1]$是否和$S_2[j, j+k-1]$为scramble string其实是$F[i][j][k]$,同理$S_1[i+k, i+\ell-1]$是否和$S_2[j+k, j+\ell-1]$为scramble string即为$F[i+k][j+k][\ell-k]$。
\par
而$S_1[i, i+k-1]$ with $S_2[j+\ell-k, j+\ell-1]$的比较结果存放在$F[i][j+\ell-k][k]$,而$S_1[i+k, i+\ell-1]$ with $S_2[j, j+\ell-k-1]$则存放在$F[i+k][j][\ell-k]$。
\par
因此总结起来Recursive Formula 如下
\[
F[i][j][\ell] = \bigcup_{k=1}^{\ell-1}\{(F[i][j][k]\land F[i+k][j+k][\ell-k])\lor(F[i][j+\ell-k][k]\land F[i+k][j][\ell-k])\}
\]
总时间复杂度因为是三维动态规划,需要三层循环,加上每一步需要线行时间求解递推式,所以是$O(n^4)$。虽然已经比较高了,但是至少不是指数量级的,动态规划还是有很大优势的,空间复杂度是$O(n^3)$。

\begin{lstlisting}[style=customc, caption={Dynamic Programming}]
bool isScramble( string s1, string s2 )
{
    using v2d = vector<vector<unsigned char>>;
    if( s1.size() != s2.size() )
    {
        return false;
    }
    if( s1 == s2 )
    {
        return true;
    }
    auto L = s1.size();
    //F[i][j][l] = 1
    //indicate s1[i,i+l-1] and s2[j,j+l-1] are scramble
    vector<v2d> F( L, v2d( L, vector<unsigned char>( L + 1, 0 ) ) );
    //fill the length equal to 1
    for( size_t i = 0; i < L; ++i )
    {
        for( size_t j = 0; j < L; ++j )
        {
            F[i][j][1] = s1[i] == s2[j] ? 1 : 0;
        }
    }
    //fill from l=2 to L
    for( size_t l = 2; l <= L; ++l )
    {
        for( size_t i = 0; i + l <= L; ++i )
        {
            for( size_t j = 0; j + l <= L; ++j )
            {
                for( size_t k = 1; k < l; ++k )
                {
                    if( F[i][j][k] && F[i + k][j + k][l - k] )
                    {
                        F[i][j][l] = 1;
                    }
                    if( F[i][j][l] == 0 )
                    {
                        if( F[i + k][j][l - k] && F[i][j + l - k][k] )
                        {
                            F[i][j][l] = 1;
                        }
                    }
                } //end for (k)
            } //end for(j)
        }//end for(i)
    }//end for(l)
    return F[0][0][L];
}
\end{lstlisting}