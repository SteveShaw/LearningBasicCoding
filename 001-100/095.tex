\section{95 --- Unique Binary Search Trees II}
Given an integer $n$, generate all structurally unique BST's (binary search trees) that store values $1 \ldots n$.
\paragraph{Example:}
\begin{flushleft}
\textbf{Input}: 3
\\
\textbf{Output}:
\begin{table}[H]
\begin{tabular}{|l|}
\hline
1,$\varnothing$,3,2      \\ \hline
3,2,$\varnothing$,1      \\ \hline
3,1,$\varnothing$,$\varnothing$,2 \\ \hline
2,1,3           \\ \hline
1,$\varnothing$,2,$\varnothing$,3 \\ \hline
\end{tabular}
\end{table}
\textbf{Explanation}:
\\
The above output corresponds to the 5 unique BST's shown below:
\begin{figure}[H]
\begin{tikzpicture}
[mynode/.style={draw,circle,minimum size=5mm, fill=gray!20!}]
\node(){};
\node[mynode](1) {1};
\node[mynode](3)[below = 8mm of 1, xshift=5mm] {3};
\node[mynode](2)[below = 8mm of 3, xshift=-5mm] {2};
\draw[>=stealth,->] (1) -- (3);
\draw[>=stealth,->] (3) -- (2);
\node[mynode](31) [right = 2cm of 1]{3};
\node[mynode](21)[below = 8mm of 31, xshift=-5mm] {2};
\node[mynode](11)[below = 8mm of 21, xshift=-5mm] {1};
\draw[>=stealth,->] (31) -- (21);
\draw[>=stealth,->] (21) -- (11);
\node[mynode](32) [right = 1.8cm of 31]{3};
\node[mynode](12)[below = 8mm of 32, xshift=-5mm] {1};
\node[mynode](22)[below = 8mm of 12, xshift=5mm] {2};
\draw[>=stealth,->] (32) -- (12);
\draw[>=stealth,->] (12) -- (22);
\node[mynode](24) [right = 1.8cm of 32]{2};
\node[mynode](14)[below = 8mm of 24, xshift=-8mm] {1};
\node[mynode](34)[below = 8mm of 24, xshift=8mm] {3};
\draw[>=stealth,->] (24) -- (14);
\draw[>=stealth,->] (24) -- (34);
\node[mynode](15) [right = 2cm of 24]{1};
\node[mynode](25)[below = 8mm of 15, xshift=5mm] {2};
\node[mynode](35)[below = 8mm of 25, xshift=5mm] {3};
\draw[>=stealth,->] (15) -- (25);
\draw[>=stealth,->] (25) -- (35);
\end{tikzpicture}
\end{figure}
\end{flushleft}
\subsection{Recursive}
Let's pick up number $ i $ out of the sequence $1$, $\ldots$, $n$ and use it as the root of the current tree. Then there are $ i - 1 $ elements available for the construction of the left subtree and n - i elements available for the right subtree. As we already discussed that results in $ G(i - 1) $ different left subtrees and $ G(n - i) $ different right subtrees, where $G$ is a Catalan number.

BST

Now let's repeat the step above for the sequence $1$, $\ldots$, $i - 1$ to construct all left subtrees, and then for the sequence $i + 1$, $\ldots$, $n$ to construct all right subtrees.

This way we have a root $i$ and two lists for the possible left and right subtrees. The final step is to loop over both lists to link left and right subtrees to the root.

与96题类似的分析,从序列$1,\ldots,n$中取出一个数$i$,将其作为root。然后其左分支有$1,\ldots,i-1$,右分支上有$i+1,\ldots,n$。从96题的分析直到左边有$G(i-1)$个不同的subtree,右边有$G(n-i)$个不同的subtree。因此,重复上述步骤到$1,\ldots, i-1$上construct所有的left subtree,然后在$i+1,\ldots,n$上construct 所有的right subtree。 这时候就得到一个以$i$为节点的root,以及两个subtree array,即左子树和右子树array,最后就是loop over both array to link left and right subtree to the root
%
%\subsubsection{Algorithm}
%\setcounter{algorithm}{0}
%\begin{algorithm}[H]
%\caption{Recursive Construction}
%\begin{algorithmic}[1]
%\Procedure{GenerateTrees}{$n$}
%\State $\texttt{ans}:=\texttt{DFS}(1, n)$
%\State \Return \texttt{ans}
%\EndProcedure
%\end{algorithmic}
%\end{algorithm}
%Function \texttt{DFS} recursively create unique BST from sequence $(\alpha, \beta)$
%\begin{algorithm}[H]
%\caption{Recursively Construct BST}
%\begin{algorithmic}[1]
%\Function{DFS}{$\alpha,\;\beta$}
%\State $\texttt{ans}:=\emptyset$
%\If{$\alpha > \beta$}
%\State $\texttt{ans} \gets \texttt{ans} + \varnothing$ \Comment Add an empty tree
%\State \Return \texttt{ans}
%\EndIf
%\For{$k:=\alpha$ \textbf{to} $\beta$}
%\State $\texttt{left}:=\texttt{DFS}(\alpha, k-1)$
%\State $\texttt{right}:=\texttt{DFS}(k+1, \beta)$
%\For{$l \in \texttt{left}$  }
%\For{$r \in \texttt{right}$}
%\State Create a node $T$ with $\texttt{Value}(T):=k$
%\State $\texttt{LEFT}(T)\gets l$ \Comment Set $T$'s left subtree
%\State $\texttt{RIGHT}(T)\gets r$ \Comment Set $T$'s right subtree
%\State $\texttt{ans} \gets \texttt{ans} + T $ \Comment Add $T$ to result array
%\EndFor
%\EndFor
%\algstore{95algo}
%\end{algorithmic}
%\end{algorithm}
%\begin{algorithm}[H]
%\begin{algorithmic}[1]
%\algrestore{95algo}
%\EndFor
%\State \Return \texttt{ans}
%\EndFunction
%\end{algorithmic}
%\end{algorithm}
\subsection{Morris Traversal}
In this method, we have to use a new data structure --- \textbf{Threaded Binary Tree}, and the strategy is as follows:
\begin{itemize}
    \item Initialize $P$ as root
    \item While current node $P$ is not \textbf{NULL}, If $P$ does not have left child
    \begin{enumerate}
        \item Add current $P$’s value
        \item Go to the right, i.e., $P \gets \text{RIGHT}(P)$
    \end{enumerate}
    Else
    \begin{enumerate}
        \item In current node $P$'s left subtree, make $P$ the right child of the rightmost node
        \item Go to this left child, i.e., $T\gets \text{RIGHT}(T)$
    \end{enumerate}
\end{itemize}
For example:
\begin{figure}[H]
\begin{tikzpicture}
[mynode/.style={draw,circle,minimum size=5mm, fill=gray!20!}]
\node(){};
\node[mynode](1) {1};
\node[mynode](2) [below = 8mm of 1, xshift=-11mm] {2};
\node[mynode](3) [below = 8mm of 1, xshift=11mm] {3};
\node[mynode](4) [below = 8mm of 2, xshift=-6mm] {4};
\node[mynode](5) [below = 8mm of 2, xshift=6mm] {5};
\node[mynode](6) [below = 8mm of 3, xshift=-6mm] {6};
\draw[>=stealth,->] (1) -- (2);
\draw[>=stealth,->] (1) -- (3);
\draw[>=stealth,->] (2) -- (4);
\draw[>=stealth,->] (2) -- (5);
\draw[>=stealth,->] (3) -- (6);
\end{tikzpicture}
\end{figure}
\begin{enumerate}
    \item 1 is the root, so initialize 1 as $P$, 1 has left child which is 2, the current's left subtree is
\begin{figure}[H]
\begin{tikzpicture}
[mynode/.style={draw,circle,minimum size=5mm, fill=gray!20!}]
\node(){};
\node[mynode](1) {2};
\node[mynode](2) [below = 8mm of 1, xshift=-6mm] {4};
\node[mynode](3) [below = 8mm of 1, xshift=6mm] {5};
\draw[>=stealth,->] (1) -- (2);
\draw[>=stealth,->] (1) -- (3);
\end{tikzpicture}
\end{figure}
\item So in this subtree, the rightmost node is 5, then make $P$ (1) as the right child of 5. Set $P \gets \text{LEFT}(P)$ ($\text{Value}(P)$ is 2). The tree now looks like:
\begin{figure}[H]
\begin{tikzpicture}
[mynode/.style={draw,circle,minimum size=5mm, fill=gray!20!}]
\node(){};
\node[mynode](2) {2};
\node[mynode](4) [below = 8mm of 2, xshift=-6mm] {4};
\node[mynode](5) [below = 8mm of 2, xshift=6mm] {5};
\node[mynode](1) [below = 8mm of 5, xshift=6mm] {1};
\node[mynode](3) [below = 8mm of 1, xshift=6mm] {3};
\node[mynode](6) [below = 8mm of 3, xshift=-6mm] {6};
\draw[>=stealth,->] (2) -- (4);
\draw[>=stealth,->] (2) -- (5);
\draw[>=stealth,->] (5) -- (1);
\draw[>=stealth,->] (1) -- (3);
\draw[>=stealth,->] (3) -- (6);
\end{tikzpicture}
\end{figure}
\item For current 2, which has left child 4, we can continue with the same process as we did above
\begin{figure}[H]
\begin{tikzpicture}
[mynode/.style={draw,circle,minimum size=5mm, fill=gray!20!}]
\node(){};
\node[mynode](2) {2};
\node[mynode](4) [below = 8mm of 2, xshift=6mm] {4};
\node[mynode](5) [below = 8mm of 4, xshift=6mm] {5};
\node[mynode](1) [below = 8mm of 5, xshift=6mm] {1};
\node[mynode](3) [below = 8mm of 1, xshift=6mm] {3};
\node[mynode](6) [below = 8mm of 3, xshift=-6mm] {6};
\draw[>=stealth,->] (2) -- (4);
\draw[>=stealth,->] (4) -- (5);
\draw[>=stealth,->] (5) -- (1);
\draw[>=stealth,->] (1) -- (3);
\draw[>=stealth,->] (3) -- (6);
\end{tikzpicture}
\end{figure}
\item then add 4 because it has no left child, then add 2, 5, 1, 3 one by one, for node 3 which has left child 6, do the same as above. Finally, the inorder taversal is $(4,2,5,1,6,3)$.
\end{enumerate}
