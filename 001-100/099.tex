\section{99 --- Recover Binary Search Tree}
Two elements of a binary search tree (BST) are swapped by mistake. Recover the tree without changing its structure.
\paragraph{Example 1:}
\begin{flushleft}
\textbf{Input}:
\begin{figure}[H]
\begin{tikzpicture}
[mynode/.style={draw,circle,minimum size=5mm, fill=gray!20!}]
\node(){};
\node[mynode](1) {1};
\node[mynode](3) [below = 8mm of 1, xshift=-6mm] {3};
\node[mynode](2) [below = 8mm of 3, xshift=6mm] {2};
\draw[>=stealth,->] (1) -- (3);
\draw[>=stealth,->] (3) -- (2);
\end{tikzpicture}
\end{figure}
\textbf{Output}:
\begin{figure}[H]
\begin{tikzpicture}
[mynode/.style={draw,circle,minimum size=5mm, fill=gray!20!}]
\node(){};
\node[mynode](3) {3};
\node[mynode](1) [below = 8mm of 3, xshift=-6mm] {1};
\node[mynode](2) [below = 8mm of 1, xshift=6mm] {2};
\draw[>=stealth,->] (3) -- (1);
\draw[>=stealth,->] (1) -- (2);
\end{tikzpicture}
\end{figure}
\end{flushleft}
\paragraph{Example 2:}
\begin{flushleft}
\textbf{Input}:
\begin{figure}[H]
\begin{tikzpicture}
[mynode/.style={draw,circle,minimum size=5mm, fill=gray!20!}]
\node(){};
\node[mynode](3) {3};
\node[mynode](1) [below = 8mm of 3, xshift=-8mm] {1};
\node[mynode](4) [below = 8mm of 3, xshift=8mm] {4};
\node[mynode](2) [below = 8mm of 4, xshift=-6mm] {2};
\draw[>=stealth,->] (3) -- (1);
\draw[>=stealth,->] (3) -- (4);
\draw[>=stealth,->] (4) -- (2);
\end{tikzpicture}
\end{figure}
\textbf{Output}:
\begin{figure}[H]
\begin{tikzpicture}
[mynode/.style={draw,circle,minimum size=5mm, fill=gray!20!}]
\node(){};
\node[mynode](2) {2};
\node[mynode](1) [below = 8mm of 2, xshift=-8mm] {1};
\node[mynode](4) [below = 8mm of 2, xshift=8mm] {4};
\node[mynode](3) [below = 8mm of 4, xshift=-6mm] {3};
\draw[>=stealth,->] (2) -- (1);
\draw[>=stealth,->] (2) -- (4);
\draw[>=stealth,->] (4) -- (3);
\end{tikzpicture}
\end{figure}
\end{flushleft}
\paragraph{Follow up:}
\begin{flushleft}
A solution using $O(n)$ space is pretty straight forward. Could you devise a constant space solution?
\end{flushleft}
\subsection{Inorder Traverse}
\begin{CJK*}{UTF8}{gbsn}
用中序遍历树,并将所有nodes存到一个array $A$中,把所有节点值存到另一个array $B$中,然后对存节点值的$B$排序,然后将$A$中保存的node的value按照排序好的$B$的顺序进行赋值。这种最一般的解法可针对任意个数目的节点错乱的情况
\end{CJK*}
\subsubsection{Algorithm}
\setcounter{algorithm}{0}
\begin{algorithm}[H]
\caption{Inorder Traverse}
\begin{algorithmic}[1]
\Procedure{RecoverTree}{$T$}
\State Initialize $A$ and $B$ as two 1-d array
\State $A:=\emptyset,\ B:=\emptyset$
\State $\texttt{Inorder}(T, A, B)$ \Comment Call inorder function to save nodes and related values into $A$ and $B$
\State $\texttt{Sort}(B)$ \Comment Sort the values saved in $B$
\For{$i:=0\to |A|-1$}
\State $\texttt{Value}(A[i])\gets B[i]$ \Comment Assign the values in $B$ to nodes in $A$
\EndFor
\EndProcedure
\end{algorithmic}
\end{algorithm}
\begin{CJK*}{UTF8}{gbsn}
下面的函数\texttt{Inorder}中序遍历$T$,然后按照顺序把访问到的node及其value存入$A$和$B$中。
\end{CJK*}
\begin{algorithm}[H]
\caption{Inorder Traverse Function}
\begin{algorithmic}[1]
\Function{Inorder}{$T, A, B$}
\If{$T=\text{NULL}$}
\State \Return
\EndIf
\algstore{99algo}
\end{algorithmic}
\end{algorithm}
\begin{algorithm}[H]
\begin{algorithmic}[1]
\algrestore{99algo}
\State $\texttt{Inorder}(\text{LEFT}(T), A, B)$ \Comment Visit left child tree of $T$
\State $A\gets A + T$ \Comment Save node to $A$
\State $B\gets B + \text{Value}(T)$ \Comment Save node's value to $B$
\State  $\texttt{Inorder}(\text{RIGHT}(T), A, B)$ \Comment Visit left child tree of $T$
\EndFunction
\end{algorithmic}
\end{algorithm}
