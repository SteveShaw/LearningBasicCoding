\section{99 --- Recover Binary Search Tree}
Two elements of a binary search tree (BST) are swapped by mistake. Recover the tree without changing its structure.
\paragraph{Example 1:}
\begin{flushleft}
\textbf{Input}:
\begin{figure}[H]
\begin{tikzpicture}
[mynode/.style={draw,circle,minimum size=5mm, fill=gray!20!}]
\node(){};
\node[mynode](1) {1};
\node[mynode](3) [below = 8mm of 1, xshift=-6mm] {3};
\node[mynode](2) [below = 8mm of 3, xshift=6mm] {2};
\draw[>=stealth,->] (1) -- (3);
\draw[>=stealth,->] (3) -- (2);
\end{tikzpicture}
\end{figure}
\textbf{Output}:
\begin{figure}[H]
\begin{tikzpicture}
[mynode/.style={draw,circle,minimum size=5mm, fill=gray!20!}]
\node(){};
\node[mynode](3) {3};
\node[mynode](1) [below = 8mm of 3, xshift=-6mm] {1};
\node[mynode](2) [below = 8mm of 1, xshift=6mm] {2};
\draw[>=stealth,->] (3) -- (1);
\draw[>=stealth,->] (1) -- (2);
\end{tikzpicture}
\end{figure}
\end{flushleft}
\paragraph{Example 2:}
\begin{flushleft}
\textbf{Input}:
\begin{figure}[H]
\begin{tikzpicture}
[mynode/.style={draw,circle,minimum size=5mm, fill=gray!20!}]
\node(){};
\node[mynode](3) {3};
\node[mynode](1) [below = 8mm of 3, xshift=-8mm] {1};
\node[mynode](4) [below = 8mm of 3, xshift=8mm] {4};
\node[mynode](2) [below = 8mm of 4, xshift=-6mm] {2};
\draw[>=stealth,->] (3) -- (1);
\draw[>=stealth,->] (3) -- (4);
\draw[>=stealth,->] (4) -- (2);
\end{tikzpicture}
\end{figure}
\textbf{Output}:
\begin{figure}[H]
\begin{tikzpicture}
[mynode/.style={draw,circle,minimum size=5mm, fill=gray!20!}]
\node(){};
\node[mynode](2) {2};
\node[mynode](1) [below = 8mm of 2, xshift=-8mm] {1};
\node[mynode](4) [below = 8mm of 2, xshift=8mm] {4};
\node[mynode](3) [below = 8mm of 4, xshift=-6mm] {3};
\draw[>=stealth,->] (2) -- (1);
\draw[>=stealth,->] (2) -- (4);
\draw[>=stealth,->] (4) -- (3);
\end{tikzpicture}
\end{figure}
\end{flushleft}
\paragraph{Follow up:}
\begin{flushleft}
A solution using $O(n)$ space is pretty straight forward. Could you devise a constant space solution?
\end{flushleft}

\subsection{Sort an Almost Sorted Array Where Two Elements Are Swapped}
Inorder traversal of BST is an array sorted in the ascending order. Here two nodes are swapped, and hence inorder traversal is an almost sorted array where only two elements are swapped. To identify these two elements in a sorted array is a classical problem that could be solved in linear time.

Here is the algorithm:

\begin{enumerate}
\item Construct inorder traversal of the tree. It should be an almost sorted list where only two elements are swapped.

\item Identify two swapped elements $x$ and $y$ in an almost sorted array in linear time.

\item Traverse the tree again. Change value $x$ to $y$ an value $y$ to $x$.
\end{enumerate}

\setcounter{lstlisting}{0}
\begin{lstlisting}[style=customc, caption={Sort an Almost Sorted Array}]
void recoverTree( TreeNode* root )
{
    vector<int> A;
    inorder( root, A );
    int x = -1;
    int y = -1;
    tie( x, y ) = find_swapped( A );
    recover( root, 2, x, y );

}
//inorder traverse
void inorder( TreeNode* root, vector<int>& A )
{
    if( !root )
    {
        return;
    }

    inorder( root->left, A );
    A.push_back( root->val );
    inorder( root->right, A );
}
//find the swapped items in the almost sorted array
tuple<int, int> find_swapped( vector<int>& A )
{
    int x = -1;
    int y = -1;
    bool flag = false;

    for( size_t i = 0; i < A.size() - 1; ++i )
    {
        if( A[i] > A[i + 1] )
        {
            y = A[i + 1];

            if( !flag )
            {
                //this the first swap occurred
                x = A[i];
                flag = true;
            }
            else
            {
                //this is the second swap occurred
                break;
            }

        }
    }

    return {x, y};
}
//recover the swapped nodes helper function
void recover( TreeNode* root, int count, int x, int y )
{
    if( !root )
    {
        return;
    }
    if( root->val == x )
    {
        //change it to y
        root->val = y;
        //decrement count
        --count;
        if( count == 0 )
        {
            //we have changed two nodes
            //complete
            return;
        }
    }
    else if( root->val == y )
    {
        //change it to x
        root->val = x;
        //decrement count
        --count;

        if( count == 0 )
        {
            //we have changed two nodes
            //complete
            return;
        }
    }

    //recursively process left child
    recover( root->left, count, x, y );
    //recursively process right child
    recover( root->right, count, x, y );
}
\end{lstlisting}

\subsection{Iterative Inorder Traversal}
In this approach, we construct inorder traversal by iterations and identify swapped nodes at the same time, in one pass. 

\textbf{Iterative inorder traversal} is simple: go left as far as you can, then one step right. Repeat till the end of nodes in the tree.

To identify swapped nodes, track the last node \fcj{pre} in the inorder traversal (i.e. the predecessor of the current node) and compare it with current node value. If the current node value is smaller than its predecessor \fcj{pre} value, the swapped node is here.

There are only two swapped nodes here, and hence one could break after having the second node identified.

Doing so, one could get directly nodes (and not only their values), and hence swap node values in $O(1)$ time.

\begin{lstlisting}[style=customc, caption={Iterative Inorder Traversal}]
void recoverTree( TreeNode* root )
{
    TreeNode* x = nullptr;
    TreeNode* y = nullptr;
    TreeNode* pre = nullptr;
    vector<TreeNode*> stk;
    auto node = root;
    while( !stk.empty() || node )
    {
        while( node )
        {
            stk.push_back( node );
            node = node->left;
        }
        node = stk.back();
        stk.pop_back();
        if( pre && ( node->val < pre->val ) )
        {
            y = node;
            if( !x )
            {
                //in this case, we only find one
                //node is out of order
                //we have to find another one
                x = pre;
            }
            else
            {
                //we have found the swapped nodes
                break;
            }
        }
        pre = node;
        //move to right child
        node = node->right;
    }
    if( x && y )
    {
        //only swap the values
        swap( x->val, y->val );
    }
}
\end{lstlisting}

\subsection{Recursive Inorder Traversal}
This approach is the recursive version of last approach

\begin{lstlisting}[style=customc, caption={Recursive Way}]
void recoverTree( TreeNode* root )
{
    //The swapped nodes
    TreeNode* x = nullptr;
    TreeNode* y = nullptr;
    //last visited node
    TreeNode* pre = nullptr;
    inorder( root, x, y, pre );
    //swap the values
    swap( x->val, y->val );
}
//helper recursive function
void inorder( TreeNode* root, TreeNode*& x, TreeNode*& y, TreeNode*& pre )
{
    if( !root )
    {
        return;
    }

    inorder( root->left, x, y, pre );
    if( pre && ( root->val < pre->val ) )
    {
        y = root;
        if( !x )
        {
            //first time find out of order item
            x = pre;
        }
        else
        {
            //second time find out of item
            //x and y have been found
            return;
        }
    }
    pre = root;
    inorder( root->right, x, y, pre );
}
\end{lstlisting}

\subsection{Morris Inorder Traversal}
The idea of Morris algorithm is to set the \textit{temporary link} between the node and its \fcj{predecessor}: 

\fcj{predecessor.right = root}. 

So one starts from the node, computes its predecessor and verifies if the link is present.

\begin{itemize}
\item There is no link? Set it and go to the left subtree.
\item There is a link? Break it and go to the right subtree.
\end{itemize}

There is one small issue to deal with : what if there is no left child, i.e. there is no left subtree? Then go straightforward to the right subtree.


\setcounter{lstlisting}{0}
\begin{lstlisting}[style=customc, caption={Morris Travel}]
void recoverTree( TreeNode* root )
{
    //the swapped nodes
    TreeNode* x = nullptr;
    TreeNode* y = nullptr;
    //the previous node of current visited one
    TreeNode* pre = nullptr;
    //morris node
    TreeNode* morris = nullptr;
    auto node = root;
    while( node )
    {
        if( node->left )
        {
            morris = node->left;
            while( morris->right && ( morris->right != node ) )
            {
                morris = morris->right;
            }
            if( !morris->right )
            {
                //we have not set morris node
                //link it to current node
                morris->right = node;
                //go to current node's left subtree again
                node = node->left;
            }
            else
            {
                //we have already explored node's left
                //break the link
                morris->right = nullptr;
                //now we check the swapped nodes
                if( pre && ( pre->val > node->val ) )
                {
                    y = node;
                    if( !x )
                    {
                        x = pre;
                    }
                }
                //set previous node
                pre = node;
                //move to right subtree
                node = node->right;
            }
        }//end if(node)
        else
        {
            //current node has no left child
            //check the swapped nodes first
            if( pre && ( pre->val > node->val ) )
            {
                y = node;
                if( !x )
                {
                    x = pre;
                }
            }
            //set previous node
            pre = node;
            //move to right subtree
            node = node->right;
        }//end else
    }//end while
    //swap the values of the two nodes
    if( x && y )
    {
        swap( x->val, y->val );
    }
}
\end{lstlisting}