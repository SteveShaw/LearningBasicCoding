\section{96 --- Unique Binary Search Trees}
Given $n$, how many structurally unique BST's (binary search trees) that store values $1 \ldots n$?
\paragraph{Example:}
\begin{flushleft}
\textbf{Input}: 3
\\
\textbf{Output}: 5
\\
\textbf{Explanation}:
\\
Given $n = 3$, there are a total of $5$ unique BST's:
\end{flushleft}
\begin{figure}[H]
\begin{tikzpicture}
[mynode/.style={draw,circle,minimum size=5mm, fill=gray!20!}]
\node(){};
\node[mynode](1) {1};
\node[mynode](3)[below = 8mm of 1, xshift=5mm] {3};
\node[mynode](2)[below = 8mm of 3, xshift=-5mm] {2};
\draw[>=stealth,->] (1) -- (3);
\draw[>=stealth,->] (3) -- (2);
\node[mynode](31) [right = 2cm of 1]{3};
\node[mynode](21)[below = 8mm of 31, xshift=-5mm] {2};
\node[mynode](11)[below = 8mm of 21, xshift=-5mm] {1};
\draw[>=stealth,->] (31) -- (21);
\draw[>=stealth,->] (21) -- (11);
\node[mynode](32) [right = 1.8cm of 31]{3};
\node[mynode](12)[below = 8mm of 32, xshift=-5mm] {1};
\node[mynode](22)[below = 8mm of 12, xshift=5mm] {2};
\draw[>=stealth,->] (32) -- (12);
\draw[>=stealth,->] (12) -- (22);
\node[mynode](24) [right = 1.8cm of 32]{2};
\node[mynode](14)[below = 8mm of 24, xshift=-8mm] {1};
\node[mynode](34)[below = 8mm of 24, xshift=8mm] {3};
\draw[>=stealth,->] (24) -- (14);
\draw[>=stealth,->] (24) -- (34);
\node[mynode](15) [right = 2cm of 24]{1};
\node[mynode](25)[below = 8mm of 15, xshift=5mm] {2};
\node[mynode](35)[below = 8mm of 25, xshift=5mm] {3};
\draw[>=stealth,->] (15) -- (25);
\draw[>=stealth,->] (25) -- (35);
\end{tikzpicture}
\end{figure}

\subsection{Dynamic Programming}
\begin{CJK*}{UTF8}{gbsn}
为了从有序序列$1\ldots n$中构造出BST,可以enumerate each number $i$, 然后用这个$i$作为 root, 接着剩下的数字$1, \ldots, i-1$作为$i$的left branch,而 $i+1, \ldots, n$则作为$i$的right branch。然后用recursive的方式构建出subtree。通过上述这种方法,可以保证构造出的BST是unique的,因为这些BST的roots都是unqiue的。
\par
从上述分析可以看出,这个问题可以 reduced into problems with smaller sizes,即subproblems。因此可以保存这些 subproblem 的 solution 并且 reuse them later。这正好就是Dynamic Programming Way。
\par
Define 两个functions:
\begin{enumerate}
\item $G(n)$: the number of unique BST for a sequence of length n.
\item $F(i, n)$: the number of unique BST, where the number $i$ is served as the root of BST $(1 \leq i \leq n)$.
\end{enumerate}
后面将看到起始$G(n)$可以 deducted from $F(i, n)$。首先,从上述定义即分析可以看到 $G(n)$ 其实就是 sum of BST $F(i,n)$,写成表达式如下
\[
G(n) = \sum_{i=1}^{n} F(i, n) \qquad \qquad
\]
对于base case,即$n=0$和$n=1$时,都只有一种方式 construct BST,即只有一个 root 的 BST 和一个 empty tree,即
\[
G(0) = 1, \qquad G(1) = 1
\]
对于$n>1$,也就是对于序列$1, \ldots, n$中,选择一个number $i$ 做为 root, 那么 the number of unique BST with this number as root 其实就是 $F(i,n)$。 这个$F(i,n)$ 也称之为 the \textbf{cartesian product} of the number of BST for its left and right subtrees。如下图所示
\begin{figure}[H]
    \centering
    \includegraphics[width=15cm]{96BST.png}
\end{figure}
例如 $F(3,7)$,即the number of unique BST tree with the number 3 as its root。这就需要从其left subsequence $(1,2)$ 中construct left subtree,并且从其 right subsequence $(4,5,6,7)$中construct right subtree。然后combine them together (i.e. cartesian product)。现在比较tricky的是 the number of unique BST out of sequence $(1,2)$ 其实就是 $G(2)$, 而同样  the number of of unique BST out of sequence $(4, 5, 6, 7)$ 就是 $G(4)$。对于$G(n)$来说,只是需要直到所从中构建的sequence的长度。因此$F(3,2) = G(2)\cdot G(4)$。其中$\cdot$是cartesian product。因此$F(i,n)$和$G(n)$的关系就是
\[
F(i,n) = G(i-1)\cdot G(n-i)
\]
综合上述分析,可以得到如下的递推公式
\[
G(n) = \sum\limits_{i=1}^{n}G(i-1) \cdot G(n-i)
\]
计算过程从lower number开始,因为$G(n)$ depends on $G(0), \ldots, G(n-1)$。
\end{CJK*}
\subsubsection{Algorithm}
\setcounter{algorithm}{0}
\begin{algorithm}[H]
\caption{Dynamic Programming}
\begin{algorithmic}[1]
\Procedure{NumTrees}{$n$}
\State $G$ as $G[0]:=1,\;G[1]:= 1,\; G[2]=\ldots=G[n]:=0$
\For{$x:=2$ \textbf{to} $n$}
\For{$k:=1$ \textbf{to} $x$}
\State $G[x] \gets G[x] + G[k-1]\times G[x-k]$
\EndFor
\EndFor
\State \Return $G[n]$
\EndProcedure
\end{algorithmic}
\end{algorithm}
