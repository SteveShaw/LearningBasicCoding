\section{91 --- Decode Ways}
A message $S$ containing letters from A--Z is being encoded to numbers using the following mapping:
\par
$A\to 1, B\to 2, \cdots, Z\to 26$
\par
Given a non-empty string containing only digits, determine the total number of ways to decode it.
\paragraph{Example 1:}
\begin{flushleft}
\textbf{Input}: 12
\\
\textbf{Output}: 2
\\
\textbf{Explanation}: It could be decoded as \texttt{AB} (1 and 2) or \texttt{L} (12 itself).
\end{flushleft}
\paragraph{Example 2:}
\begin{flushleft}
\textbf{Input}: 226
\\
\textbf{Output}: 3
\\
\textbf{Explanation}: It could be decoded as \texttt{BZ} (2 and 26), \texttt{VF} (22 and 6), or \texttt{BBF} (2, 2 and 6).
\end{flushleft}
\subsection{Dynamic Programming}
与斐波那契数列很像,但是有限制条件,假设$F[i]$表示number of decode ways for $S[0\ldots i]$。初始化时为0,即$F[i] = 0$
\begin{itemize}
\item 如果$S[i]$不为0,则$F[i] \gets F[i] + F[i-1]\times 1$,因为$S[i]$可以decode为一个letter,即有一种方式。
\item 如果$S[i-1]$为1,或者$S[i-1]=2$ and $0\leq S[i]\leq 6$,则$F[i] \gets F[i] + F[i-2]\times 1$,即$S[i-1,i-2]$可以decode为一个letter,即有一种方式。
\end{itemize}
由于$F[i]$仅仅取决于$F[i-1]$和$F[i-2]$,因此可以用两个变量$x_0$ and $x_1$来分别代表$F[i-2]$ and $F[i-1]$,然后逐步更新。

\setcounter{lstlisting}{0}
\begin{lstlisting}[style=customc, caption={Dynamic Programming}]
int numDecodings( string s )
{
    //empty string has one way to decode
    int x0 = 1;
    //the first letter has 1 way to decode if it is not '0'
    int x1 = s[0] == '0' ? 0 : 1;
    for( size_t i = 1; i < s.size(); ++i )
    {
        int x2 = 0;
        if( s[i] != '0' )
        {
            //we have one way to decode s[i]
            //so x2=x1 x 1
            x2 = x1;
        }
        if( s[i - 1] == '1' )
        {
            //we have one way to decode s[i-1,i]
            //so x2+=x0 times 1
            x2 += x0;
        }
        if( s[i - 1] == '2' )
        {
            //we have one way to decode s[i-1,i]
            //so x2+=x0 times 1
            if( ( s[i] >= '0' ) && ( s[i] <= '6' ) )
            {
                x2 += x0;
            }
        }
        //overwrite previous values
        x0 = x1;
        x1 = x2;
    }
    return x1;
}
\end{lstlisting}