\section{49 --- Group Anagrams}
Given an array of strings $A$, group anagrams together.

\paragraph{Example:}
\begin{flushleft}

\textbf{Input}: [eat, tea, tan, ate, nat, bat]


\textbf{Output}:

\begin{table}[H]
\begin{tabular}{lll}
ate & eat & tea \\
nat & tan & \\
bat & &
\end{tabular}
\end{table}

\end{flushleft}
\paragraph{Note:}

\begin{itemize}
\item All inputs will be in lowercase.
\item The order of your output does not matter.
\end{itemize}

\subsection{Counting Sort}
\begin{itemize}
\item 遍历输入字符串数组,对每一个字符串$S$,用一个大小为26的数组来统计每个单词中字符出现的次数,然后按照从$a$--$z$的顺序来组成一个字符串作为hash map的key。
\item 为了优化,hash map的value是存放每个字符串在字符串数组中的index。
\end{itemize}

\setcounter{lstlisting}{0}
\begin{lstlisting}[style=customc, caption={Counting Sort}]
vector<vector<string>> groupAnagrams( vector<string>& strs )
{
    //optimize memory and runtime performance
    //use array of index as the value of the
    //hash map
    unordered_map<string, vector<size_t>> m;

    int chars[26] = {0};

    for( size_t i = 0; i < strs.size(); ++i )
    {
        const auto&str = strs[i];

        for( auto c : str )
        {
            chars[c - 'a'] += 1;
        }

        string key;

        //contruct the key
        for( int i = 0; i < 26; ++i )
        {
            key.append( chars[i], i + 'a' );
            chars[i] = 0;
        }

        auto it = m.find( key );

        if( it == m.end() )
        {
            m.emplace( key, initializer_list<size_t> {i} );
        }
        else
        {
            it->second.emplace_back( i );
        }
    }

    vector<vector<string>> ans( m.size() );

    size_t j = 0;

    for( const auto&p : m )
    {
        for( size_t i : p.second )
        {
            ans[j].emplace_back( strs[i].c_str() );
        }

        ++j;
    }

    return ans;
}
\end{lstlisting}