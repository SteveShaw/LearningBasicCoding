\section{50 --- Pow(x, n)}
Implement \texttt{pow}($x, n$), which calculates $x$ raised to the power $n$ ($x^n$).

\paragraph{Example 1:}

\begin{flushleft}
\textbf{Input}: $2.00000, 10$

\textbf{Output}: $1024.00000$
\end{flushleft}

\paragraph{Example 2:}

\begin{flushleft}
\textbf{Input}: $2.10000, 3$

\textbf{Output}: $9.26100$
\end{flushleft}

\paragraph{Example 3:}

\begin{flushleft}
\textbf{Input}: $2.00000, -2$

\textbf{Output}: $0.25000$

\textbf{Explanation}: $2^{-2} = 1/2^2 = 1/4 = 0.25$
\end{flushleft}


\paragraph{Note:}

\begin{itemize}
\item $-100.0 < x < 100.0$
\item $n$ is a 32-bit signed integer, within the range $[−2^{31}, 2^{31} − 1]$
\end{itemize}

\subsection{Divide And Conquer}
The most difficult is dealing with maximum integer type and minimum integer type. Therefore, we need to compute the result of half of the exponent at the first time.
\setcounter{algorithm}{0}
\begin{algorithm}[H]
\caption{Divide And Conquer}
\begin{algorithmic}[1]
\Procedure{MyPow}{$x, n$}
\If{$n = 0$}
\State \Return 1 \Comment $x^0 = 1$
\EndIf
\State $\alpha :=$ \Call{MyPow}{$x, n/2$} \Comment Get $x^{n/2}$
\algstore{50algo}
\end{algorithmic}
\end{algorithm}
\begin{algorithm}[H]
\begin{algorithmic}[1]
\algrestore{50algo}
\If{$n \bmod 2 = 0$}
\State \Return $\alpha^2$ \Comment $x^n = x^{n/2} \times x^{n/2}$
\ElsIf{$n > 0$}
\State \Return $\alpha^2\times x$ \Comment $x^n = {x^{n/2}}^2\times x$
\Else
\State \Return $\alpha^2 / x$ \Comment $x^n = {x^{n/2}}^2 / x$
\EndIf
\EndProcedure
\end{algorithmic}
\end{algorithm}

\setcounter{lstlisting}{0}
\begin{lstlisting}[style=customc, caption={Divide And Conquer}]
double myPow( double x, int n )
{

    //convert to long long type
    //to avoid data type overflow
    auto exp = static_cast<long long>( n );

    if( exp >= 0 )
    {

        return f( x, exp );
    }

    exp *= -1;
    return 1 / f( x, exp );
}

double f( double x, long long n )
{
    if( n == 0 )
    {
        return 1;
    }

    if( n == 1 )
    {
        return x;
    }

    auto exp = n / 2;
    auto y = f( x, exp );

    y *= y;

    if( n & 1 )
    {
        //odd exponent
        return x * y;
    }

    return y;
}
\end{lstlisting}


\subsection{Iterative Method}
假设需要计算$x^{13}$。
\begin{itemize}
\item 13的二进制为$1101$。由于有4个bits,所以需要4 iterations。
\item First, initialize the result to 1: $r\gets 1 = x^0$。然后从左到右扫描bits
\begin{enumerate}
\item $r\gets r^2 = x^0$, bit 1 is 1, so $r\gets r\times x = x^1$
\item $r\gets r^2 = x^2$, bit 2 is 0, so $r\gets r\times x = x^3$
\item $r\gets r^2 = x^6$, bit 3 is 0, so $r\gets r\times x^0 = x^6$
\item $r\gets r^2 = x^{12}$, bit 4 is 1, so $r\gets r\times x = x^{13}$
\end{enumerate} 
\end{itemize}
\begin{algorithm}[H]
\caption{Iterative Method}
\begin{algorithmic}[1]
\Procedure{MyPow}{$x,n$}
\State $r := 1$ \Comment $r = x^n$ 
\State $i:=n$
\While{$i \neq 0$}
\If{$i \bmod 2 = 1$}
\State $r \gets r \times x$
\EndIf
\State $x\gets x^2$
\State $i \gets i / 2$
\EndWhile
\If{$n < 0$}
\State \Return $1/r$
\Else
\State \Return $r$
\EndIf
\EndProcedure
\end{algorithmic}
\end{algorithm}

\setcounter{lstlisting}{0}
\begin{lstlisting}[style=customc, caption={Iterative}]
double myPow( double x, int n )
{
    double r = 1.0;
    auto i = static_cast<long long>( n );

    if( i < 0 )
    {
        i = -i;
    }

    while( i )
    {

        if( i & 1 )
        {
            //multiply x
            //for bit 1
            r *= x;
        }

        //always do x<--x*x
        x *= x;

        i /= 2;
    }

    if( n < 0 )
    {
        return 1 / r;
    }

    return r;
}
\end{lstlisting}
