\section{68 --- Text Justification}
Given an array of words and a width \fcj{maxWidth}, format the text such that each line has exactly \fcj{maxWidth} characters and is fully (left and right) justified.

You should pack your words in a greedy approach; that is, pack as many words as you can in each line. Pad extra spaces \fcj{' '} when necessary so that each line has exactly \fcj{maxWidth} characters.

Extra spaces between words should be distributed as evenly as possible. If the number of spaces on a line do not divide evenly between words, the empty slots on the left will be assigned more spaces than the slots on the right.

For the last line of text, it should be left justified and no extra space is inserted between words.

\paragraph{Note:}

\begin{itemize}
\item A word is defined as a character sequence consisting of non-space characters only.

\item Each word's length is guaranteed to be greater than 0 and not exceed \fcj{maxWidth}.

\item The input array \fcj{words} contains at least one word.

\end{itemize}

\paragraph{Example 1:}
\begin{flushleft}


\textbf{Input}:

\fcj{words = ["This", "is", "an", "example", "of", "text", "justification."]}

\fcj{maxWidth = 16}

\textbf{Output}:

\begin{lstlisting}[style=customc]
[
   "This    is    an",
   "example  of text",
   "justification.  "
]
\end{lstlisting}
\end{flushleft}

\paragraph{Example 2:}
\begin{flushleft}

\textbf{Input}:

\fcj{words = ["What","must","be","acknowledgment","shall","be"]}

\fcj{maxWidth = 16}

\textbf{Output}:
\begin{lstlisting}[style=customc]
[
  "What   must   be",
  "acknowledgment  ",
  "shall be        "
]
\end{lstlisting}

\textbf{Explanation}: 

Note that the last line is \fcj{"shall be    "} instead of \fcj{"shall     be"},

because the last line must be left-justified instead of fully-justified.

Note that the second line is also left-justified becase it contains only one word.

\end{flushleft}

\paragraph{Example 3:}
\begin{flushleft}


\textbf{Input}:
\begin{lstlisting}[style=customc]
words = ["Science","is","what","we","understand","well","enough","to",
"explain", "to","a","computer.",
"Art","is","everything","else","we","do"]
\end{lstlisting}

\fcj{maxWidth = 20}

\textbf{Output}:
\begin{lstlisting}[style=customc]
[
  "Science  is  what we",
  "understand      well",
  "enough to explain to",
  "a  computer.  Art is",
  "everything  else  we",
  "do                  "
]
\end{lstlisting}
\end{flushleft}

\subsection{Greedy}
The algorithm can be divided into three steps:
\begin{enumerate}
\item Check if current word can be added into current line
\item Determine how many spaces we need to insert
\item Process the last line.
\end{enumerate}

We make use of an array $T$ to record added words.

To check if current word can be added to $T$
\begin{itemize}
\item Get total characters $x$ By assuming insert only one space between words in $T$ and current word.
\item If $x < L$ ($L$ is \fcj{maxWidth}), we add current word into $T$. Otherwise, we pad all the words in $T$ with spaces.
\end{itemize}

To pad the words in $T$, 
\begin{enumerate}
\item Get total length of all words, say $\ell$, in $T$.
\item Get number of words in $T$, say $y$.
\item The average number of spaces, say, $z$, inserted between two consecutive words are $\ell/(y-1)$.
\item The extra spaces, say $r$, will be $\bmod(\ell, y-1)$.
\item To greedily used extra spaces, for the first $r+1$ words, the inserted spaces between two consecutive ones are $ z + 1$
\end{enumerate}

If there is only word in $T$, we just print the word and pad with required spaces.

\setcounter{lstlisting}{0}
\begin{lstlisting}[style=customc, caption={Greedy}]
vector<string> fullJustify( vector<string>& words, int maxWidth )
{
    size_t L = static_cast< size_t >( maxWidth );
    //record selected word in current line
    vector<string_view> text;
    //total length of selected word in current line
    size_t text_len = 0;

    vector<string> ans;

    for( const auto& word : words )
    {
        //assume we put only one space between already selected words
        //and this word, we may need <text.size() - 1 + 1 = text.size()> spaces
        auto spaces = text.size();
        //check if we can select <word>
        if( text_len + spaces + word.size() <= L )
        {
            //add <word> into current line <text>
            text.emplace_back( word.c_str(), word.size() );
            //update <text_len>
            text_len += word.size();
        }
        else
        {
            //we will arrange current line <text>
            string line = gen_line( text, text_len, L );
            ans.push_back( "" );
            ans.back().swap( line );

            //remove all current words in the <text>
            text.clear();
            //add current word
            text.emplace_back( word.c_str(), word.size() );
            //reset total length of selected words
            text_len = word.size();
        }
    }//end for(word)

    //add final line
    string line;
    line.reserve( L );

    //we add each word plus one space
    auto spaces = L - text_len;

    for( size_t i = 0; i < text.size() - 1; ++i )
    {
        line += text[i];
        line.append( 1, ' ' );
        --spaces;
    }
    //add last word
    line += text.back();
    if( spaces )
    {
        //pad spaces
        line.append( spaces, ' ' );
    }

    ans.emplace_back( line );
    return ans;
}
//helper function to generate left justified line
string gen_line( vector<string_view>& text, size_t text_len, size_t L )
{
    //the required spaces to be inserted
    auto spaces = L - text_len;

    string line;
    line.reserve( L );

    if( text.size() == 1 )
    {
        //we only have one word in this line
        //just add the word and required spaces
        line += text[0];
        line.append( spaces, ' ' );

        return line;
    }
    //the number of spaces between two words
    auto sep_spaces = spaces / ( text.size() - 1 );
    //the number of extra spaces will put in the front of line
    auto extra_spaces = spaces - sep_spaces * ( text.size() - 1 );

    for( size_t i = 0; i < text.size() - 1; ++i )
    {
        line += text[i];
        auto insert_spaces = sep_spaces;
        if( extra_spaces )
        {
            //we insert one more space
            //to consume extra spaces
            ++insert_spaces;
            --extra_spaces;
        }
        line.append( insert_spaces, ' ' );
    }

    //add final word
    line += text.back();

    return line;
}
\end{lstlisting}