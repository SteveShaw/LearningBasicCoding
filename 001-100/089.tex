\section{89 --- Gray Code}
The gray code is a binary numeral system where two successive values differ in only one bit.
\par
Given a non-negative integer $n$ representing the total number of bits in the code, print the sequence of gray code. A gray code sequence must begin with 0.
\paragraph{Example 1:}
\begin{flushleft}
\textbf{Input}: 2
\\
\textbf{Output}: $[0,1,3,2]$
\\
\textbf{Explanation}:
\\
00 -- 0
\\
01 -- 1
\\
11 -- 3
\\
10 -- 2
\\
For a given $n$, a gray code sequence may not be uniquely defined. For example, $[0,2,3,1]$ is also a valid gray code sequence.
\\
00 -- 0
\\
10 -- 2
\\
11 -- 3
\\
01 -- 1
\end{flushleft}
\paragraph{Example 2:}
\begin{flushleft}
\textbf{Input}: 0
\\
\textbf{Output}: $[0]$
\\
\textbf{Explanation}: We define the gray code sequence to begin with 0. A gray code sequence of $n$ has size $= 2^n$, which for $n = 0$ the size is $2^0 = 1$. Therefore, for $n = 0$ the gray code sequence is $[0]$.
\end{flushleft}
\subsection{Analysis}
\begin{CJK*}{UTF8}{gbsn}
维基百科给出了gray code的转换代码
\end{CJK*}
\begin{lstlisting}[backgroundcolor=\color{blue!80!green!10}, keywordstyle=\bfseries\color{green!40!black}, commentstyle=\itshape\color{purple!40!black},language=C++]
/*
 * This function converts an unsigned binary
 * number to reflected binary Gray code.
 *
 * The operator >> is shift right. The operator ^ is exclusive or.
 */
unsigned int BinaryToGray(unsigned int num)
{
    return num ^ (num >> 1);
}
\end{lstlisting}
\begin{CJK*}{UTF8}{gbsn}
只要从0到$2^n-1$用上述函数计算处gray code放到输出数组中即可。
\end{CJK*}