\section{89 --- Gray Code}
The gray code is a binary numeral system where two successive values differ in only one bit.
\par
Given a non-negative integer $n$ representing the total number of bits in the code, print the sequence of gray code. A gray code sequence must begin with 0.
\paragraph{Example 1:}
\begin{flushleft}
\textbf{Input}: 2
\\
\textbf{Output}: $[0,1,3,2]$
\\
\textbf{Explanation}:
\\
00 -- 0
\\
01 -- 1
\\
11 -- 3
\\
10 -- 2
\\
For a given $n$, a gray code sequence may not be uniquely defined. For example, $[0,2,3,1]$ is also a valid gray code sequence.
\\
00 -- 0
\\
10 -- 2
\\
11 -- 3
\\
01 -- 1
\end{flushleft}
\paragraph{Example 2:}
\begin{flushleft}
\textbf{Input}: 0
\\
\textbf{Output}: $[0]$
\\
\textbf{Explanation}: We define the gray code sequence to begin with 0. A gray code sequence of $n$ has size $= 2^n$, which for $n = 0$ the size is $2^0 = 1$. Therefore, for $n = 0$ the gray code sequence is $[0]$.
\end{flushleft}
\subsection{Analysis}
维基百科给出了gray code的转换代码
\setcounter{lstlisting}{0}
\begin{lstlisting}[style=customc, caption={}]
/*
 * This function converts an unsigned binary
 * number to reflected binary Gray code.
 *
 * The operator >> is shift right. The operator ^ is exclusive or.
 */
unsigned int BinaryToGray(unsigned int num)
{
    return num ^ (num >> 1);
}
\end{lstlisting}
只要从0到$2^n-1$用上述函数计算gray code放到输出数组中即可。

\setcounter{lstlisting}{0}
\begin{lstlisting}[style=customc, caption={Math}]
vector<int> grayCode( int n )
{
    int end = ( 1 << n ) - 1;
    vector<int> ans( end + 1, 0 );
    for( int i = 0; i <= end; ++i )
    {
        //apply the formula (x ^ (x >>1))
        ans[i] = ( i ^ ( i >> 1 ) );
    }
    return ans;
}
\end{lstlisting}

\subsection{Iterative}
We can generate the sequence iteratively. 

For example, when $n=2$, the sequence is \fcj{00, 01, 11, 10}.

To obtain sequence for $n=3$, the process is as following:

\fcj{00, 01, 11, 10 -> 000, 001, 011, 010} --- This is same as $n=2$

Then add 1 to the symmetric of the sequence of \fcj{00, 01, 11, 10}

\fcj{10, 11, 01, 10 -> 110, 111, 101, 100}

\begin{lstlisting}[style=customc, caption={Iteratively Generating}]
vector<int> grayCode( int n )
{
    int L = ( 1 << n );
    vector<int> ans;
    ans.reserve( L );
    //add first code into output
    ans.push_back( 0 );
    for( int i = 0; i < n; ++i )
    {
        //for each x in current ans
        //tranform to next code by x | (1<<i)
        transform( ans.rbegin(), ans.rend(), back_inserter( ans ), [i]( int x )
        {
            return x | ( 1 << i );
        } );
    }
    return ans;
}
\end{lstlisting}