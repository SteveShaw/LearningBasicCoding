\section{71 --- Simplify Path}
Given an absolute path for a file (Unix-style), simplify it. Or in other words, convert it to the canonical path.

In a UNIX-style file system, a period \fcj{.} refers to the current directory. Furthermore, a double period \fcj{..} moves the directory up a level.

Note that the returned canonical path must always begin with a slash \fcj{/}, and there must be only a single slash \fcj{/} between two directory names. The last directory name (if it exists) must not end with a trailing \fcj{/}. Also, the canonical path must be the shortest string representing the absolute path.
 

\paragraph{Example 1:}

\begin{flushleft}
\textbf{Input}: \fcj{"/home/"}

\textbf{Output}: \fcj{"/home"}

\textbf{Explanation}: Note that there is no trailing slash after the last directory name.
\end{flushleft}

\paragraph{Example 2:}
\begin{flushleft}
\textbf{Input}: \fcj{"/../"}

\textbf{Output}: \fcj{"/"}

\textbf{Explanation}: Going one level up from the root directory is a no-op, as the root level is the highest level you can go.
\end{flushleft}
\paragraph{Example 3:}
\begin{flushleft}


\textbf{Input}: \fcj{"/home//foo/"}

\textbf{Output}: \fcj{"/home/foo"}

\textbf{Explanation}: In the canonical path, multiple consecutive slashes are replaced by a single one.
\end{flushleft}
\paragraph{Example 4:}
\begin{flushleft}


\textbf{Input}: \fcj{"/a/./b/../../c/"}

\textbf{Output}: \fcj{"/c"}
\end{flushleft}

\paragraph{Example 5:}

\begin{flushleft}
Input: \fcj{"/a/../../b/../c//.//"}

Output: \fcj{"/c"}
\end{flushleft}

\paragraph{Example 6:}

\begin{flushleft}
\textbf{Input}: \fcj{"/a//b////c/d//././/.."}

\textbf{Output}: \fcj{"/a/b/c"}
\end{flushleft}

\subsection{Iterative Parsing}
\begin{itemize}
\item The separator is \fcj{'/'}. We can treat whole path are comprised of multiple substrings between \fcj{'/'}.
\item If there is only \fcj{'.'} between two \fcj{'/'}, we just ignore it.
\item If there is \fcj{'..'} between two \fcj{'/'}, we delete the last path name.
\end{itemize}

Thus, we make use of a stack to push found path name between two \fcj{'/'}. 

\setcounter{lstlisting}{0}
\begin{lstlisting}[style=customc, caption={Stack}]
string simplifyPath( string path )
{
    string_view sv_path( path.c_str(), path.size() );
    vector<string_view> groups;
    size_t start = 0;
    while( start < sv_path.size() )
    {
        auto p = start;
        //find next '/'
        while( ( start < sv_path.size() ) && ( sv_path[start] != '/' ) )
        {
            ++start;
        }
        if( p == start )
        {
            //no sub path
            ++start;
            continue;
        }
        if( ( start == p + 1 ) && ( sv_path[p] == '.' ) )
        {
            //current directory (one period)
            //ignore
            ++start;
            continue;
        }
        auto sub_path = sv_path.substr( p, start - p );
        if( sub_path == ".." )
        {
            //go to up level
            //pop up last path
            if( !groups.empty() )
            {
                groups.pop_back();
            }
        }
        else
        {
            //this is a sub path name
            //save it
            groups.push_back( sub_path );
        }
    }
    if( groups.empty() )
    {
        return "/";
    }
    string ans;
    for( const auto& sub_path : groups )
    {
        ans.push_back( '/' );
        ans += sub_path;
    }
    return ans;
}
\end{lstlisting}