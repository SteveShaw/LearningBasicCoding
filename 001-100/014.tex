\section{14 --- Longest Common Prefix}
Write a function to find the longest common prefix string amongst an array of strings $ A $.
\par
If there is no common prefix, return an empty string.

\paragraph{Example 1:}

\begin{flushleft}
\textbf{Input}: [\texttt{flower}, \texttt{flow}, \texttt{flight}]
\\
\textbf{Output}: \texttt{fl}
\end{flushleft}

\paragraph{Example 2:}

\begin{flushleft}
\textbf{Input}: [\texttt{dog}, \texttt{racecar}, \texttt{car}]
\\
\textbf{Output}: $ \emptyset $
\\
\textbf{Explanation}: There is no common prefix among the input strings.
\end{flushleft}

\paragraph{Note:}

\begin{itemize}
\item All given inputs are in lowercase letters.
\end{itemize}

\subsection{Horizontal scanning}
\begin{itemize}
\item Set prefix equal to $S[0]$
\item Iterate through the strings $S[1],\ldots S[n-1]$.
\item At each iteration, check if $S[i]$ can find the prefix at position 0. If it is not, {\color{red}shorten} the prefix by \textbf{\color{red}1} character until prefix is empty or found the prefix at \textbf{position} \textbf{\color{red}0}.
\end{itemize}
\setcounter{algorithm}{0}
\begin{algorithm}[H]
\caption{Scanning horizontal to find prefix}
\begin{algorithmic}[1]
\Procedure{LongestCommonPrefix}{$S,L$}
\State $P := S[0]$ \Comment Set prefix to first string of the array
\For{$i:=1$ \textbf{to} $L-1$}
\While{$S[i][0\ldots \lvert P\rvert-1] \neq P$}
\If{$ \lvert P\rvert = 1$}
\State \Return $\emptyset$ \Comment Return empty string since $P$ will become empty.
\EndIf
\State $P\gets P[0\ldots \lvert P\rvert-2]$ \Comment Shorten current prefix $P$ by one character
\EndWhile
		\algstore{014algo}
		\end{algorithmic}
	\end{algorithm}
\begin{algorithm}[H]
	\begin{algorithmic}[1]
		\algrestore{014algo}
		\EndFor
\State \Return $P$
\EndProcedure
\end{algorithmic}
\end{algorithm}
\setcounter{lstlisting}{0}
\begin{lstlisting}[style=customc, caption={Horizontal scanning}]
string longestCommonPrefix( vector<string>& strs )
{

    if( strs.empty() )
    {
        return "";
    }

    string prefix = strs[0];

    for( size_t i = 1; i < strs.size(); ++i )
    {
        auto sz = ( min )( strs[i].size(), prefix.size() );

        size_t x = sz;

        for( size_t j = 0; j < sz; ++j )
        {
            if( strs[i][j] != prefix[j] )
            {
                if( j == 0 )
                {
                    return "";
                }

                x = j;

                break;
            }
        }

        if( x < prefix.size() )
        {
            prefix = prefix.substr( 0, x );
        }
    }

    return prefix;
}
\end{lstlisting}

\subsection{Vertical Scanning}
Imagine a very short string is at the end of the array. The above approach will still do $\lvert S\rvert$ comparisons. One way to optimize this case is to do vertical scanning by comparing characters from top to bottom on the same column (same \textbf{\color{red}index} of the strings) before moving on to the next column.
\begin{algorithm}[H]
\caption{Scanning vertically to find prefix}
\begin{algorithmic}[1]
\Procedure{LongestCommonPrefix}{$S,L$}
\If{$S = \emptyset$}
\State \Return $\emptyset$
\EndIf
\For{$i:=0$ \textbf{to} $\lvert S[0]\rvert-1$} 
\State $C := S[0][i]$ 
\For{$k:=1$ \textbf{to} $L-1$}
\If{$\lvert S[k]\rvert = i$ \textbf{or} $S[k][i] \neq C$} 
\State \Return $S[0][0\ldots i-1]$ \Comment This is the common prefix
\EndIf
\EndFor
\EndFor
\State \Return $S[0]$ \Comment $S[0]$ match all prefix in the array.
\EndProcedure
\end{algorithmic}
\end{algorithm}
\begin{lstlisting}[style=customc, caption={Vertical Scanning}]
string longestCommonPrefix( vector<string>& strs )
{
    if( strs.empty() )
    {
        return "";
    }

    string &p = strs[0];

    for( size_t i = 0; i < p.size(); ++i )
    {
        for( size_t j = 1; j < strs.size(); ++j )
        {
            if( ( strs[j].size() == i ) || ( strs[j][i] != p[i] ) )
            {
                //If the code can run here
                //then strs[1...j] has been matched
                //for prefix strs[0][0...i]
                //we can directly return the prefix string here
                if( i == 0 )
                {
                    return "";
                }

                return p.substr( 0, i );
            }
        }
    }
    return p;
}
\end{lstlisting}
\subsection{Divide and conquer}
\begin{itemize}
\item Split problem of finding \texttt{LCP} in $S[i\ldots j]$ into 2 subproblems, i.e. find \texttt{LCP} in $S[i,x]$ and $S[x+1\ldots j]$ where $x = (i+j)/2$.
\item 根据上述两个部分得到的\texttt{LCP},construct \texttt{LCP} for $S[i\ldots j]$.
\item 方法是逐个字符比较这两个\texttt{LCP}直到没有匹配的letter,也就是这两个\texttt{LCP}的\texttt{LCP}。这个就是\texttt{LCP} for $S[i\ldots j]$。
\end{itemize}
\begin{algorithm}[H]
\caption{Divide and conquer}
\begin{algorithmic}[1]
\Procedure{LongestCommonPrefix}{$A, L$}
\State \Return \Call{Find}($A, 0, L-1$)
\EndProcedure
\end{algorithmic}
\end{algorithm}
Function \texttt{find} 包括三个参数
\begin{itemize}
\item $A, L$ --- input array and its length
\item $l$, $r$ --- 处理的范围即$A[l\ldots r]$
\end{itemize}
\begin{algorithm}[H]
\caption{Helper Function}
\begin{algorithmic}[1]
\Procedure{Find}{$A, L, l, r$}
\If{$l=r$}
\State \Return $A[l]$
\EndIf
\State $x:=(l+r)/2$
\State $S_1 :=$ \Call{Find}{$A, l, x$}
\State $S_2 :=$ \Call{Find}{$A, x+1, r$}
\State \Return \Call{CommonPrefix}($S_{1}, S_{2}$)
\EndProcedure
\Statex
\end{algorithmic}
\end{algorithm}

Function \texttt{CommonPrefix}则寻找两个输入string $S_1$和$ S_2 $的common prefix。
\begin{algorithm}[H]
\caption{Helper Function To Get Common Prefix}
\begin{algorithmic}[1]
\Procedure{CommonPrefix}{$S_{1}, L_1, S_{2}, L_2$}  
\If{$S_{1}$ is empty \textbf{or} $S_2$ is empty}
\State \Return A empty string
\EndIf
\State $\ell:= \min(L_1, L_2)$ \Comment Get shorter length
\For{$i:=0$ \textbf{to} $\ell$}
\If{$S_1[i] \neq S_2[i]$}
\State \Return $S_1[0\ldots i-1]$ 
\EndIf
\EndFor
\State \Return $S_1[0\ldots \ell-1]$ 
\EndProcedure
\Statex
\end{algorithmic}
\end{algorithm}