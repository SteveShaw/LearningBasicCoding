\section{60 --- Permutation Sequence}
The set $[1,2,3,...,n]$ contains a total of $n!$ unique permutations.
\par
By listing and labeling all of the permutations in order, we get the following sequence for $n = 3$:
\[
123, 132, 213, 231, 312, 321
\]
\par
Given $n$ and $k$, return the $k$th permutation sequence.
\par
\textbf{Note}:
\par
Given $n$ will be between 1 and 9 inclusive.
\par
Given $k$ will be between 1 and n! inclusive.
\paragraph{Example 1:}
\begin{flushleft}
\textbf{Input}: $n = 3, k = 3$
\par
\textbf{Output}: $213$
\end{flushleft}
\paragraph{Example 2:}
\begin{flushleft}
\textbf{Input}: $n = 4, k = 9$
\par
\textbf{Output}: $2314$
\end{flushleft}
\subsection{Find the N-th permutation (Using factorial number system)} 
In this section, we are talking about the background of how to find $n$th permutation of a string. 
\subsection{Factorial Number System}
Instead of finding all permutations and looking up the $n$th, we can directly calculate the $n$th permutation. We need to first understand the \textbf{Factorial Number System.} A factorial number system uses \textbf{factorial} values instead of powers of numbers (binary system uses powers of 2, decimal uses powers of 10) to denote place-values (or base). The place values (base) are:
\[
0!=1, 1!=1, 2!=2, 3!=6, 4!=24, 5!=120, \ldots
\]
The digit in the zeroth place is always 0. The digit in the 1st place (with base = 1!) can be 0 or 1. The digit in the 2nd place (with base 2!) can be 0, 1 or 2 and so on. Generally speaking, the digit at $n$th place can take any value between $0\to n$.
\par
First few numbers represented as factoradics
\begin{align*}
    0 & \implies 0 = 0\times 0! \\
    1 & \implies 10 = 1*1! + 0*0! \\
    2 & \implies 100 = 1*2! + 0*1! + 0*0! \\
    3 & \implies 110 = 1\times 2! + 1\times 1! + 0\times 0! \\
    4 & \implies 200 = 2\times 2! + 0\times 1! + 0\times 0! \\
    5 & \implies 210 = 2\times 2! + 1\times 1! + 0\times 0! \\
    6 & \implies 1000 = 1\times 3! + 0\times 2! + 0\times 1! + 0\times 0! \\
    7 & \implies 1010 = 1\times 3! + 0\times 2! + 1\times 1! + 0\times 0! \\
    8 & \implies 1100 = 1\times 3! + 1\times 2! + 0\times 1! + 0\times 0! \\
    9 & \implies 1110 \\
    10& \implies 1200
\end{align*}
\par
There is a direct relationship between $n$th lexicographical permutation of a string and its factoradic representation.
\par
For example, here are the permutations of the string \textit{abcd}.
\begin{figure} [H]
    \centering
    \begin{tikzpicture}
    \matrix (p) [matrix of nodes]
    {
    0  abcd & 6  bacd &  12  cabd & 18  dabc \\
    1  abdc & 7  badc &  13  cadb & 19  dacb \\
    2  acbd & 8  bcad &  14  cbad & 20  dbac \\
    3  acdb & 9  bcda &  15  cbda & 21  dbca \\
    4  adbc & 10  bdac & 16  cdab & 22  dcab \\
    5  adcb & 11  bdca & 17  cdba & 23  dcba \\
    };    
    \end{tikzpicture}
\end{figure}
\par
We can see a pattern here, if observed carefully. The first letter changes after every 6th ($3!$) permutation. The second letter changes after 2($2!$) permutation. The third letter changed after every ($1!$) permutation and the fourth letter changes after every ($0!$) permutation. We can use this relation to directly find the nth permutation.
\subsection{Algorithm}
Once we represent $n$ in factoradic representation, we consider each digit in it and add a character from the given string to the output. If we need to find the 14th permutation of \textbf{ABCD}. 14 in \textbf{factoradics} is 2100.
\begin{itemize}
    \item Start with the first digit \textbf{2}. Take the element at position 2 from \textbf{ABCD} which is \textbf{C}, and add it to the Output. Therefore, the output now is \textbf{C} and The previous string becomes \textbf{ABD}
    \item The next digit is \textbf{1}.String is now \textbf{ABCD}. Again, get the character at position 1 which is \textbf{B} and add it to the Output. Now, the output is \textbf{CB} and the previous string is \textbf{AD}
    \item Next digit is \textbf{0}. String is \textbf{AD}. Add the character at position 0 to the Output, we get output \textbf{CBA}. The previous string reduces to \textbf{D}
    \item Next digit is \textbf{0}. String is \textbf{D}. Add the character at position 0 to the Output, we get output \textbf{CBAD} and we consumed all characters from previous string.
\end{itemize}
In summary, we get the following process
\begin{figure}[H]
\centering
\begin{tikzpicture}
\node(0) {};
\node(2) at (2,0) [text=red, draw, minimum width=1cm, minimum height=1cm] {2};
\node(a) at (3.5,0) [text=violet, draw, minimum width=2cm, minimum height=1cm] {\textbf{ABCD}};
\node(c) at (6.5,0) [text=purple, draw, minimum width=2cm, minimum height=1cm] {\textbf{C}};
\node(d) at (8.5,0) [text=purple, draw, minimum width=2cm, minimum height=1cm] {\textbf{ABD}};
\draw[-{Stealth[length=2mm]}] (a) -- (c);
\node(1) at (2,-1) [text=red, draw, minimum width=1cm, minimum height=1cm] {1};
\node(a1) at (3.5,-1) [text=violet, draw, minimum width=2cm, minimum height=1cm] {\textbf{ABD}};
\node(c1) at (6.5,-1) [text=purple, draw, minimum width=2cm, minimum height=1cm] {\textbf{CB}};
\node(d1) at (8.5,-1) [text=purple, draw, minimum width=2cm, minimum height=1cm] {\textbf{AD}};
\draw[-{Stealth[length=2mm]}] (a1) -- (c1);
\node(0) at (2,-2) [text=red, draw, minimum width=1cm, minimum height=1cm] {0};
\node(a2) at (3.5,-2) [text=violet, draw, minimum width=2cm, minimum height=1cm] {\textbf{AD}};
\node(c2) at (6.5,-2) [text=purple, draw, minimum width=2cm, minimum height=1cm] {\textbf{CBA}};
\node(d2) at (8.5,-2) [text=purple, draw, minimum width=2cm, minimum height=1cm] {\textbf{D}};
\draw[-{Stealth[length=2mm]}] (a2) -- (c2);
\node at (2,-3) [text=red, draw, minimum width=1cm, minimum height=1cm] {0};
\node(a3) at (3.5,-3) [text=violet, draw, minimum width=2cm, minimum height=1cm] {\textbf{D}};
\node(c3) at (6.5,-3) [text=purple, draw, minimum width=2cm, minimum height=1cm] {\textbf{CBAD}};
\node at (8.5,-3) [text=purple, draw, minimum width=2cm, minimum height=1cm] {$\emptyset$};
\draw[-{Stealth[length=2mm]}] (a3) -- (c3);
\end{tikzpicture}
\end{figure}
\subsection{Convert To Factorial Number}
To convert a given number to Factorial Number System,successively divide the number by $1,2,3,4,5,6$ and so on until the quotient becomes zero. The reminders at each step forms the factoradic representation.
\par
For example, to convert $349$ to factoradic, we have the following process
\begin{itemize}
    \item $\dfrac{349}{1} = 349$, we have remainders $r = 0$ and quotient $q = 349$, the generated factoradic number $\digamma = 0$
    \item $\dfrac{349}{2} = 174$, $r = 1, q = 174,\; \digamma = 10$
    \item $\dfrac{174}{3} = 58$, $r = 0, q = 58,\; \digamma = 010$
    \item $\dfrac{58}{4} = 14$, $r = 2, q = 14,\; \digamma = 2010$
    \item $\dfrac{14}{5} = 2$, $r = 4, q = 2,\; \digamma = 42010$
    \item $\dfrac{2}{6} = 0$, $r = 2, q = 0,\; \digamma = 242010$
\end{itemize}
Hence, Factoradic representation of 349 is \textbf{242010}. \textbf{Note}: The length of factoradic representation should be equal to the length of the character array. Prepend zeros to ensure the length is same.

\setcounter{lstlisting}{0}
\begin{lstlisting}[style=customc, caption={Factorial Number System}]
string getPermutation( int n, int k )
{
    vector<int> factors( n, 0 );

    int factor = 1;

    //k is in [0, n!-1]
    --k;

    //fill the coefficients of factorial number
    while( k )
    {
        int q = k / factor;
        int r = k - q * factor;

        factors[factor - 1] = r;

        k = q;
        ++factor;
    }

    string s;
    for( int i = 1; i <= n; ++i )
    {
        s.push_back( i + '0' );
    }

    string ans;

    //generate the k-th permutation
    //based on factors array
    for( size_t i = 0; i < factors.size(); ++i )
    {

        auto j = factors.size() - i - 1;

        int sel = factors[j];
        ans.push_back( s[sel] );

        s = s.substr( 0, sel ) + s.substr( sel + 1 );
    }

    return ans;
}
\end{lstlisting}