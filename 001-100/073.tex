\section{73 --- Set Matrix Zeroes}
Given a $m \times n$ matrix, if an element is 0, set its entire row and column to 0. Do it in-place.

\paragraph{Example 1:}
\begin{flushleft}


\textbf{Input}: 

\[
\begin{bmatrix}
1 & 1 & 1\\
1 & 0 & 1\\
1 & 1 & 1\\
\end{bmatrix}
\]

\textbf{Output}: 

\[
\begin{bmatrix}
1 & 0 & 1\\
0 & 0 & 0\\
1 & 0 & 1\\
\end{bmatrix}
\]

\end{flushleft}


\paragraph{Example 2:}
\begin{flushleft}


\textbf{Input}: 

\[
\begin{bmatrix}
0 & 1 & 2 & 0\\
3 & 4 & 5 & 2\\
1 & 3 & 1 & 5
\end{bmatrix}
\]

\textbf{Output}: 

\[
\begin{bmatrix}
0 & 0 & 0 & 0\\
0 & 4 & 5 & 0\\
0 & 3 & 1 & 0
\end{bmatrix}
\]
\end{flushleft}
\paragraph{Follow up:}

\begin{itemize}
\item A straight forward solution using $O(mn)$ space is probably a bad idea.
\item A simple improvement uses $O(m + n)$ space, but still not the best solution.
\item Could you devise a constant space solution?
\end{itemize}

\subsection{Constant Space}
We can use the first cell of every row and column as a flag. This flag would determine whether a row or column has been set to zero. This means for every cell instead of going to $M+N$ cells and setting it to zero we just set the flag in two cells.

These flags are used later to update the matrix. 

\begin{enumerate}
\item If the first cell of a row is set to zero this means the row should be marked zero. 

\item If the first cell of a column is set to zero this means the column should be marked zero.
\end{enumerate}

Thus, the algorithm steps are clear:

\begin{enumerate}
\item Iterate over the matrix and mark the first cell of a row $r$ and first cell of a column $c$ when $M(r,c)=0$.
\item Special treatment for the $M(0,0)$. Since row 0 and column 0 have the same first cell $M(0,0)$, we make use of an additional flag to tell us if the column 0 is marked or not, while $M(0,0)$ is still used for mark row 0.
\item Iterate over $M$ starting from 2nd row and 2nd column i.e. starting from $M(1,1)$. For every cell, we check if the row $r$ or column $c$ had been marked earlier by checking the respective first row cell or first column cell. If any of them was marked, we set the value in the cell to 0.
\item Then check if $M(0,0)=0$, if it is, mark the cells in the row 0 as zeros.
\item Finally, when column 0 is marked, and fill zeros when it is.
\end{enumerate}

\setcounter{lstlisting}{0}
\begin{lstlisting}[style=customc, caption={Constant Space}]
void setZeroes( vector<vector<int>>& matrix )
{
    auto M = matrix.size();
    auto N = matrix[0].size();
    //the flag indicate if column 0 needs to be set to zeros
    bool zeros_col0 = false;
    for( size_t r = 0; r < M; ++r )
    {
        // Since first cell for both first row and first column
        // is the same i.e. matrix[0][0]
        // We can use an additional variable for either the first row/column.
        // We choose to use zeros_col0 for the first column
        // and using matrix[0][0] for the first row.
        if( matrix[r][0] == 0 )
        {
            zeros_col0 = true;
        }
        for( size_t c = 1; c < N; ++c )
        {
            // If an element is zero,
            //we set the first element of the corresponding row and column to 0
            if( matrix[r][c] == 0 )
            {
                matrix[r][0] = 0;
                matrix[0][c] = 0;
            }
        }
    }//end for(r)

    // Iterate over the array once again and using the first row and first column,
    // update the elements.
    for( size_t r = 1; r < M; ++r )
    {
        for( size_t c = 1; c < N; ++c )
        {
            if( ( matrix[r][0] == 0 ) || ( matrix[0][c] == 0 ) )
            {
                matrix[r][c] = 0;
            }
        }
    }
    //matrix[0][0] is the flag for the first row
    // See if the first row needs to be set to zero as well
    if( matrix[0][0] == 0 )
    {
        for( size_t c = 1; c < N; ++c )
        {
            matrix[0][c] = 0;
        }
    }
    // See if the first column needs to be set to zero as well
    if( zeros_col0 )
    {
        for( size_t r = 0; r < M; ++r )
        {
            matrix[r][0] = 0;
        }
    }
}
\end{lstlisting}