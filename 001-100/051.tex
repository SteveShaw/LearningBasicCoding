\section{51 --- N-Queens}
The $n$-queens puzzle is the problem of placing $n$ queens on an $n\times n$ chessboard such that no two queens attack each other.


Given an integer $n$, return all distinct solutions to the $n$-queens puzzle.

Each solution contains a distinct board configuration of the n-queens' placement, where \texttt{Q} and dot both indicate a queen and an empty space respectively.

\subsection{Backtrack}
\begin{itemize}
\item There could be the only one queen in a row and the only one queen in a column. This means that there is no need to consider all squares on the board. One could just iterate over the columns.
\item For all diagonals the sum of row and column of each square is constant
\item For all anti-diagonals the difference between row and column of each square is constant.
\end{itemize}

The backtrack algorithm is listed as below

\setcounter{algorithm}{0}
\begin{algorithm}[H]
\caption{$N$-Queen Backtarck}
\begin{algorithmic}[1]
\Procedure{Backtrack}{$n$}
\State $\star$ Start from the first $r = 0$.
\State $\star$ Iterate over the columns and try to put a queen in each column.
\If{Square $(r,c)$ is not under attack}
\State $\star$ Place the queen in $(r, c)$ square.
\State Exclude one row, one column and two diagonals from further consideration.
\If{$r = n$} \Comment All rows are filed up
\State $\star$ One solution is found
\Else
\State \Call{Backtrack}{$n+1$} \Comment Proceed to place further queens 
\EndIf
\State $\star$ Now backtrack, remove the queen from $(r,c)$.
\EndIf
\EndProcedure
\end{algorithmic}
\end{algorithm}

\setcounter{lstlisting}{0}
\begin{lstlisting}[style=customc, caption={Backtrack}]
vector<vector<string>> solveNQueens( int n )
{
    vector<int> queens( n );

    vector<unsigned char> rows( n, 0 );
    vector<unsigned char> hills( 2 * n - 1, 0 );
    vector<unsigned char> dales( 2 * n - 1, 0 );

    vector<vector<string>> ans;

    dfs( 0, n, rows, hills, dales, queens, ans );

    return ans;
}

void dfs( int r, int n,
          vector<unsigned char>& rows,
          vector<unsigned char>& hills,
          vector<unsigned char>& dales,
          vector<int>& queens,
          vector<vector<string>>& ans )
{
    for( int c = 0; c < n; ++c )
    {
        if( hills[r + c] + dales[r - c + n - 1] + rows[c] == 0 )
        {
            //place queen at (r,c)
            queens[r] = c;
            hills[r + c] = 1;
            dales[r - c + n - 1] = 1;
            rows[c] = 1;

            if( r + 1 == n )
            {
                //we have filled all rows
                vector<string> chess( n, string( n, '.' ) );

                for( int i = 0; i < n; ++i )
                {
                    chess[i][queens[i]] = 'Q';
                }

                ans.push_back( move( chess ) );
            }
            else
            {
                //depth first search
                dfs( r + 1, n, rows, hills, dales, queens, ans );
            }

            //backtrack
            queens[r] = 0;
            hills[r + c] = 0;
            dales[r - c + n - 1] = 0;
            rows[c] = 0;
        }
    }
}
\end{lstlisting}