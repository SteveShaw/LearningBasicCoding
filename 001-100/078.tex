\section{78 --- Subsets}
Given a set of distinct integers, $A$, return all possible subsets (the power set).
\par
Note: The solution set must not contain duplicate subsets.
\paragraph{Example:}
\begin{flushleft}
\textbf{Input}: $A = [1,2,3]$
\\
\textbf{Output}:
$(3), (1), (2), (1,2,3), (1,3),  (2,3),  (1,2), () $
\end{flushleft}
\subsection{Backtracking}
原来的set中每一个element在subset中有两种状态:要么存在、要么不存在。这样构造subset的过程中每个element有两种选择方法:选择、不选择,因此可以构造一颗二叉树,例如给出的例子$A=[1,2,3]$,构造的二叉树如下(左子树表示选择该层处理的元素,右子树不选择),最后得到的叶子节点就是子集:
\begin{figure}[H]
\begin{tikzpicture}
[mynode/.style={draw, minimum size=1cm, fill=gray!20!}]
\node(){};
\node[mynode](rt){};
\node[mynode](l1) [below = 8mm of rt, xshift=-3.8cm] {$(1)$};
\node[mynode](r1) [below = 8mm of rt, xshift=3.8cm] {};
\node[mynode](ll) [below = 8mm of l1, xshift=-2cm] {$(1,2)$};
\node[mynode](lr) [below = 8mm of l1, xshift=2cm] {$(1)$};
\node[mynode](rl) [below = 8mm of r1, xshift=-2cm] {$(2)$};
\node[mynode](rr) [below = 8mm of r1, xshift=2cm] {};
\node[mynode](lll) [below = 8mm of ll, xshift=-1cm] {$(1,2,3)$};
\node[mynode](llr) [below = 8mm of ll, xshift=1cm] {$(1,2)$};
\node[mynode](lrl) [below = 8mm of lr, xshift=-1cm] {$(1,3)$};
\node[mynode](lrr) [below = 8mm of lr, xshift=1cm] {$(1)$};
\node[mynode](rll) [below = 8mm of rl, xshift=-1cm] {$(2,3)$};
\node[mynode](rlr) [below = 8mm of rl, xshift=1cm] {$(2)$};
\node[mynode](rrl) [below = 8mm of rr, xshift=-1cm] {$(3)$};
\node[mynode](rrr) [below = 8mm of rr, xshift=1cm] {};
\draw[>=stealth,->] (rt) -- (l1.north east);
\draw[>=stealth,->] (rt) -- (r1.north west);
\draw[>=stealth,->] (l1) -- (ll);
\draw[>=stealth,->] (l1) -- (lr);
\draw[>=stealth,->] (ll) -- (lll);
\draw[>=stealth,->] (ll) -- (llr);
\draw[>=stealth,->] (lr) -- (lrl);
\draw[>=stealth,->] (lr) -- (lrr);
\draw[>=stealth,->] (r1) -- (rl);
\draw[>=stealth,->] (r1) -- (rr);
\draw[>=stealth,->] (rl) -- (rll);
\draw[>=stealth,->] (rl) -- (rlr);
\draw[>=stealth,->] (rr) -- (rrl);
\draw[>=stealth,->] (rr) -- (rrr);
\end{tikzpicture}
\end{figure}
根据上述二叉树,递归函数中需要从某个位置开始,向当前level的数组中加入该位置的数字,然后继续递归,结束后,把这个数字移除出数组。类似的题目都有类似这样的递归结构。原来的题目中还有个要求是subset中的数字必须是非递减序列,因此首先需要对数组进行排序。
\subsubsection{Code}
\setcounter{algorithm}{0}
\begin{algorithm}[H]
\caption{Depth First Search}
\begin{algorithmic}[1]
\Procedure{Subsets}{$A, L$}
\State $\texttt{sort}(A)$
\State $V:=\emptyset$ \Comment Empty array to hold current subset
\State $\texttt{ans}:=\emptyset$ \Comment The result subsets
\State $\texttt{DFS}(A, L, V, \texttt{ans}, 0)$ \Comment Call recursive function \texttt{DFS}
\State \Return \texttt{ans}
\EndProcedure
\end{algorithmic}
\end{algorithm}

\begin{algorithm}[H]
\caption{Recursive Function}
\begin{algorithmic}[1]
\Function{DFS}{$A, L, V, \texttt{ans}, \alpha$} \Comment $\alpha$ is the starting index
\State $\texttt{ans}\gets \texttt{ans} + V$ \Comment Add current subset to result subsets
\For{$i:=\alpha$ \textbf{to} $L-1$}
\State $V\gets V + A[i]$ \Comment Add to current subset
\State $\texttt{DFS}(A, L, V, \texttt{ans}, i+1)$ \Comment Recursive for next number
\State $V\gets V\setminus A[i]$ \Comment Remove $A[i]$ from current subset
\EndFor
\EndFunction
\end{algorithmic}
\end{algorithm}