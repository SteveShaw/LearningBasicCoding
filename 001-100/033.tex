\section{33 --- Search in Rotated Sorted Array}
Suppose an array $A$ sorted in ascending order is rotated at some pivot unknown to you beforehand.

(i.e., $[0,1,2,4,5,6,7]$ might become $[4,5,6,7,0,1,2]$).

You are given a target value, $T$, to search. If found in the array return its index, otherwise return $-1$.

You may assume no duplicate exists in the array.

Your algorithm's runtime complexity must be in the order of $ O(\log n)$.

\paragraph{Example 1:}

\begin{flushleft}
\textbf{Input}: $A = [4,5,6,7,0,1,2]$, $T = 0$

\textbf{Output}: 4
\end{flushleft}

\paragraph{Example 2:}

\begin{flushleft}
\textbf{Input}: $A = [4,5,6,7,0,1,2]$, $T = 3$

\textbf{Output}: $-1$

\end{flushleft}

\subsection{Binary Search}

Notice the input array is ascending, so except at the smallest number, we always have $A[i] < A[i+1]$

\begin{itemize}
\item Find a rotation index $z$, i.e. index of the smallest element in the array. Binary search works just perfect here. Since the array is rotated, so, there is two parts: one part is ascending and another one is descending. 
\begin{itemize}
\item If at index $i$, $A[i] > A[i+1]$, then $A[i+1]$ is the smallest number. 
\item If $A[i] < A[0]$, then smallest number is in the range $A[0, i]$
\item otherwise, the smallest number is in the range $A[i, L-1]$
\end{itemize}

\item index $z$ splits array in two parts. Compare $A[0]$ and target, $T$, to identify in which part one has to look for target. If $A[0] > T$, search in the right of rotation index. Otherwise, search in the left of rotation index.

\item Perform a binary search in the chosen part of the array.
\end{itemize}

\setcounter{lstlisting}{0}
\begin{lstlisting}[style=customc, caption={Binary Search}]
int search( vector<int>& nums, int target )
{
	//in find_rotation_index, we need to check
	//nums[mid] and nums[mid+1]
	//therefore size of nums must be greater than 1
	//we have to deal with the edge case for empty and 1 element array
    if( nums.empty() )
    {
        return -1;
    }

    if( nums[0] == target )
    {
        return 0;
    }

    int L = static_cast<int>( nums.size() );

	//find the rotation index
    int z = find_rotation_index( nums );

    if( nums[z] == target )
    {
        return z;
    }


    if( z == 0 )
    {
		//no rotation
        return bs( nums, 0, L - 1, target );
    }

    if( nums[0] > target )
    {
        //target in the right of rotation index
        return bs( nums, z + 1, L - 1, target );
    }

    return bs( nums, 0, z - 1, target );
}

int find_rotation_index( vector<int>& A )
{
    int l = 0;
    int r = static_cast<int>( A.size() ) - 1;

    if( A[l] < A[r] )
    {
        //no rotation
        return 0;
    }

    while( l <= r )
    {
        int mid = ( l + r ) / 2;

        if( A[mid] > A[mid + 1] )
        {
            return mid + 1;
        }

        if( A[mid] < A[l] )
        {
            //the smallest value happens in
            //left part
            r = mid - 1;
        }
        else
        {
            //the smallest value happens in
            //the right part
            l = mid + 1;
        }
    }

    return 0;
}

int bs( vector<int>& A, int l, int r, int T )
{
    while( l <= r )
    {
        int mid = ( l + r ) / 2;

        if( A[mid] == T )
        {
            return mid;
        }

        if( A[mid] < T )
        {
            l = mid + 1;
        }
        else
        {
            r = mid - 1;
        }
    }

    return -1;
}
\end{lstlisting}