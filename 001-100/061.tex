\section{61 --- Rotate List}
Given a linked list $\eta_0$, rotate the list to the right by $k$ places, where $k$ is non-negative.
\paragraph{Example 1:}
\begin{flushleft}
\textbf{Input:}
\\
\begin{figure}[H]
\begin{tikzpicture}
\node{};
\node(1) at (2,0)[draw, circle] {1};
\node(2) [draw, circle, right = 0.7cm of 1] {2};
\node(3) [draw, circle, right = 0.7cm of 2] {3};
\node(4) [draw, circle, right = 0.7cm of 3] {4};
\node(5) [draw, circle, right = 0.7cm of 4] {5};
\node(6) [draw, circle, right = 0.7cm of 5] {$\emptyset$};
\draw[-{Stealth}] (1) -- (2);
\draw[-{Stealth}] (2) -- (3);
\draw[-{Stealth}] (3) -- (4);
\draw[-{Stealth}] (4) -- (5);
\draw[-{Stealth}] (5) -- (6);
\end{tikzpicture}
\end{figure}
$k=2$
\\
\textbf{Output:}
\\
\begin{figure}[H]
\begin{tikzpicture}
\node{};
\node(1) at (2,0)[draw, circle] {4};
\node(2) [draw, circle, right = 0.7cm of 1] {5};
\node(3) [draw, circle, right = 0.7cm of 2] {1};
\node(4) [draw, circle, right = 0.7cm of 3] {2};
\node(5) [draw, circle, right = 0.7cm of 4] {3};
\node(6) [draw, circle, right = 0.7cm of 5] {$\emptyset$};
\draw[-{Stealth}] (1) -- (2);
\draw[-{Stealth}] (2) -- (3);
\draw[-{Stealth}] (3) -- (4);
\draw[-{Stealth}] (4) -- (5);
\draw[-{Stealth}] (5) -- (6);
\end{tikzpicture}
\end{figure}
\textbf{Explanation}:
\par
rotate 1 steps to the right:
\\
\begin{figure}[H]
\begin{tikzpicture}
\node{};
\node(1) at (2,0)[draw, circle] {5};
\node(2) [draw, circle, right = 0.7cm of 1] {1};
\node(3) [draw, circle, right = 0.7cm of 2] {2};
\node(4) [draw, circle, right = 0.7cm of 3] {3};
\node(5) [draw, circle, right = 0.7cm of 4] {4};
\node(6) [draw, circle, right = 0.7cm of 5] {$\emptyset$};
\draw[-{Stealth}] (1) -- (2);
\draw[-{Stealth}] (2) -- (3);
\draw[-{Stealth}] (3) -- (4);
\draw[-{Stealth}] (4) -- (5);
\draw[-{Stealth}] (5) -- (6);
\end{tikzpicture}
\end{figure}
rotate 2 steps to the right:
\\
\begin{figure}[H]
\begin{tikzpicture}
\node{};
\node(1) at (2,0)[draw, circle] {4};
\node(2) [draw, circle, right = 0.7cm of 1] {5};
\node(3) [draw, circle, right = 0.7cm of 2] {1};
\node(4) [draw, circle, right = 0.7cm of 3] {2};
\node(5) [draw, circle, right = 0.7cm of 4] {3};
\node(6) [draw, circle, right = 0.7cm of 5] {$\emptyset$};
\draw[-{Stealth}] (1) -- (2);
\draw[-{Stealth}] (2) -- (3);
\draw[-{Stealth}] (3) -- (4);
\draw[-{Stealth}] (4) -- (5);
\draw[-{Stealth}] (5) -- (6);
\end{tikzpicture}
\end{figure}
\end{flushleft}
\paragraph{Example 2:}
\begin{flushleft}
\textbf{Input}:
\\
\begin{figure}[H]
\begin{tikzpicture}
\node{};
\node(1) at (2,0)[draw, circle] {0};
\node(2) [draw, circle, right = 0.7cm of 1] {1};
\node(3) [draw, circle, right = 0.7cm of 2] {2};
\node(4) [draw, circle, right = 0.7cm of 3] {$\emptyset$};
\draw[-{Stealth}] (1) -- (2);
\draw[-{Stealth}] (2) -- (3);
\draw[-{Stealth}] (3) -- (4);
\end{tikzpicture}
\end{figure}
$k = 4$
\\
\textbf{Output:}
\\
\begin{figure}[H]
\begin{tikzpicture}
\node{};
\node(1) at (2,0)[draw, circle] {2};
\node(2) [draw, circle, right = 0.7cm of 1] {0};
\node(3) [draw, circle, right = 0.7cm of 2] {1};
\node(4) [draw, circle, right = 0.7cm of 3] {$\emptyset$};
\draw[-{Stealth}] (1) -- (2);
\draw[-{Stealth}] (2) -- (3);
\draw[-{Stealth}] (3) -- (4);
\end{tikzpicture}
\end{figure}
\textbf{Explanation}:
\\
rotate 1 steps to the right:
\\
\begin{figure}[H]
\begin{tikzpicture}
\node{};
\node(1) at (2,0)[draw, circle] {2};
\node(2) [draw, circle, right = 0.7cm of 1] {0};
\node(3) [draw, circle, right = 0.7cm of 2] {1};
\node(4) [draw, circle, right = 0.7cm of 3] {$\emptyset$};
\draw[-{Stealth}] (1) -- (2);
\draw[-{Stealth}] (2) -- (3);
\draw[-{Stealth}] (3) -- (4);
\end{tikzpicture}
\end{figure}
rotate 2 steps to the right: 
\\
\begin{figure}[H]
\begin{tikzpicture}
\node{};
\node(1) at (2,0)[draw, circle] {1};
\node(2) [draw, circle, right = 0.7cm of 1] {2};
\node(3) [draw, circle, right = 0.7cm of 2] {0};
\node(4) [draw, circle, right = 0.7cm of 3] {$\emptyset$};
\draw[-{Stealth}] (1) -- (2);
\draw[-{Stealth}] (2) -- (3);
\draw[-{Stealth}] (3) -- (4);
\end{tikzpicture}
\end{figure}
rotate 3 steps to the right:
\\
\begin{figure}[H]
\begin{tikzpicture}
\node{};
\node(1) at (2,0)[draw, circle] {0};
\node(2) [draw, circle, right = 0.7cm of 1] {1};
\node(3) [draw, circle, right = 0.7cm of 2] {2};
\node(4) [draw, circle, right = 0.7cm of 3] {$\emptyset$};
\draw[-{Stealth}] (1) -- (2);
\draw[-{Stealth}] (2) -- (3);
\draw[-{Stealth}] (3) -- (4);
\end{tikzpicture}
\end{figure}
rotate 4 steps to the right:
\\
\begin{figure}[H]
\begin{tikzpicture}
\node{};
\node(1) at (2,0)[draw, circle] {2};
\node(2) [draw, circle, right = 0.7cm of 1] {0};
\node(3) [draw, circle, right = 0.7cm of 2] {1};
\node(4) [draw, circle, right = 0.7cm of 3] {$\emptyset$};
\draw[-{Stealth}] (1) -- (2);
\draw[-{Stealth}] (2) -- (3);
\draw[-{Stealth}] (3) -- (4);
\end{tikzpicture}
\end{figure}
\end{flushleft}

\subsection{Analysis}
The nodes in the list are already linked, and hence the rotation basically means

\begin{itemize}
\item To close the linked list into the ring.
\item To break the ring after the new tail and just in front of the new head
\end{itemize}

Suppose $k=2$ and the list is shown as below:

\begin{figure}[H]
\begin{tikzpicture}
[start chain, 
every node/.style={draw, circle,
 minimum size=6mm, fill=gray!20!, on chain},
  node distance=8mm, 
  every join/.style={>=stealth,->},
  every on chain/.style={join=by ->},
 thick
]
\node{1};
\node{2};
\node{3};
\node{4};
\node{5};
\node{};
\end{tikzpicture}
\end{figure}

The steps to rotate the list by $k=2$ places are shown as below:

\begin{enumerate}
\item Link the head and the tail
\begin{figure}[H]
\begin{tikzpicture}
[start chain, 
every node/.style={draw, circle,
 minimum size=6mm, fill=gray!20!, on chain},
  node distance=8mm, 
  every join/.style={>=stealth,->},
  every on chain/.style={join=by ->},
 thick
]
\node(1){1};
\node(2){2};
\node(3){3};
\node(4){4};
\node(5){5};
\draw[very thick, red, >=stealth, ->] (5.south) -- ++(0, -10mm) -| (1.south);
\end{tikzpicture}
\end{figure}
\item Break 3 and 4, and set node 3 as the final non-empty node.
\begin{figure}[H]
\begin{tikzpicture}
[every node/.style={draw, circle,
 minimum size=6mm, fill=gray!20!, on chain},
  node distance=8mm, 
  every join/.style={>=stealth,->},
 thick
]
\node(1){1};
\node(2)[join]{2};
\node(3)[join]{3};
\node[join]{};
\node(4){4};
\node(5)[join]{5};
\draw[very thick, red, >=stealth, ->] (5.south) -- ++(0, -10mm) -| (1.south);
\end{tikzpicture}
\end{figure}
\item The new head is in the position $n-k$, where $n$ is a number of nodes in the list. The new tail is just before the new head, which is in the position $n - k - 1$.
\item When $k\geq n$, just use $k\bmod n$ to change $k$ to less than $n$.
\end{enumerate}

Thus, the algorithm is quite straightforward :

\begin{itemize}
\item Find the old tail and connect it with the head to form a close ring. Compute the length of the list n at the same time.

\item Find the new tail, which is $(n - (k\bmod n) - 1)$th node from the head, and the new head, which is $(n - (k\bmod n))$th node.

\item Break the ring by set the new tail's next node as empty node, and the new head.
\end{itemize}

\setcounter{lstlisting}{0}
\begin{lstlisting}[style=customc, caption={Direct Way}]
ListNode* rotateRight( ListNode* head, int k )
{
    if( !head )
    {
        return head;
    }

    auto node = head;

    int n = 1;

    while( node->next )
    {
        ++n;
        node = node->next;
    }

    //close the ring
    node->next = head;

    //the new tail's position
    auto new_tail = head;

    int pos = n - ( k % n ) - 1;
    for( int i = 0; i < pos; ++i )
    {
        new_tail = new_tail->next;
    }

    auto new_head = new_tail->next;
    new_tail->next = nullptr;

    return new_head;
}
\end{lstlisting}