\section{97 --- Interleaving String}
Given $s_1$, $s_2$, $s_3$, find whether $s_3$ is formed by the interleaving of $s_1$ and $s_2$.
\paragraph{Example 1:}
\begin{flushleft}
\textbf{Input}: $s_1 = \texttt{aabcc}$, $s_2 = \texttt{dbbca}$, $s_3 = \texttt{aadbbcbcac}$
\\
\textbf{Output}: \texttt{true}
\end{flushleft}
\paragraph{Example 2:}
\begin{flushleft}
\textbf{Input}: $s_1 = \texttt{aabcc}$, $s_2 = \texttt{dbbca}$, $s_3 = \texttt{aadbbbaccc}$
\\
\textbf{Output}: \texttt{false}
\end{flushleft}
\subsection{Dynamic Programming}
\begin{CJK*}{UTF8}{gbsn}
Define $F(r,c)$表示$s_3[0\ldots r+c-1]$是否是$s_1[0\ldots r-1]$ and $s_2[0\ldots c-1]$ interleave的结果。这时候会有下列几种情况
\begin{itemize}
\item $s_1[r-1]\neq s_3[r+c-1]$ and $s_2[c-1]\neq s_3[r+c-1]$,这种情况下,很容易知道$s_3[0\ldots r+c-1]$肯定不是$s_1[0\ldots r-1]$ and $s_2[0\ldots c-1]$ interleave的结果
\item 要么$s_1[r-1] = s_3[r+c-1]$ 或者 $s_2[c-1]\neq s_3[r+c-1]$,或者两者都是,这种情况下, 看$s_3[0\ldots r+c-1]$是否是$s_1[0\ldots r-1]$ and $s_2[0\ldots c-1]$ interleave的结果就要看$s_3[0\ldots r+c-2]$能否从下面两种情况下得到
\begin{enumerate}
\item $s_1[r-1] = s_3[r+c-1]$,这时候看$s_1[0\ldots r-2]$ and $s_2[0\ldots c-1]$ interleave的结果,也就是$F[r-1,c]$的值。
\item $s_2[c-1] = s_3[r+c-1]$,这时候看$s_1[0\ldots r-1]$ and $s_2[0\ldots c-2]$ interleave的结果,也就是$F[r,c-1]$的值。
\end{enumerate}
\end{itemize}
综上所述, $F(r,c)$的递推公式就是
\[
F(r,c) = 
\begin{cases}
\texttt{false} &\qquad s_1[r-1]\neq s_3[r+c-1], \;s_2[c-1]\neq s_3[r+c-1]\\
F[r-1][c] &\qquad s_1[r-1]= s_3[r+c-1] \; \texttt{or} \\
F[r][c-1] &\qquad s_2[c-1]= s_3[r+c-1] 
\end{cases}
\]
另外$F(r,c)$中,$r \in [0, L_1]$ and $c\in[0, L_2]$,$L_1$ and $L_2$分别是$s_1$ and $s_2$的长度。这样$F(0,0)$就意味着$s_1$, $s_2$ and $s_3$都是空字符串,肯定为1。最后看$F[L_1][L_2]$是否为\texttt{true}。
\end{CJK*}

\subsubsection{Algorithm}
\setcounter{algorithm}{0}
\begin{algorithm}[H]
\caption{Dynamic Programming}
\begin{algorithmic}[1]
\Procedure{IsInterleave}{$s_1, L_1, s_2, L_2, s_3, L_3$}
\If{$s_1=\varnothing$}
\State \Return $\texttt{StrCmp}(s_2, s_3)$ \Comment $s_1$ is empty, just compare $s_2$ against $s_3$ directly
\EndIf
\If{$s_2=\varnothing$}
\State \Return $\texttt{StrCmp}(s_1 s_3)$ \Comment $s_2$ is empty, just compare $s_1$ against $s_3$ directly
\EndIf
\If{$L_1+L_2 \neq L_3$}
\State \Return \texttt{false}
\EndIf
\State $F$ as $F[0][0]=\ldots=F[L_1][L_2]:=0$
\State $F[0][0]\gets 1$
\For{$r:=1$ \textbf{to} $L_1$}
\If{$s_1[r-1] = s_3[r-1]$} \Comment $c=0$ so only compare $s_1$ with $s_3$
\State $F[r][0] \gets F[r-1][0]$
\EndIf
\EndFor
\For{$c:=1$ \textbf{to} $L_2$}
\If{$s_2[c-1] = s_3[c-1]$} \Comment $r=0$ so only compare $s_2$ with $s_3$
\State $F[0][c] \gets F[0][c-1]$
\EndIf
\EndFor
\For{$r:=1$ \textbf{to} $L_1$}
\For{$c:=1$ \textbf{to} $L_2$}
\If{$s_1[r-1] = s_3[r+c-1]$}
\State $F[r][c] \gets F[r-1][c]$
\EndIf
\If{$s_2[c-1] = s_3[r+c-1]$}
\State $F[r][c] \gets F[r][c] + F[r][c-1]$
\EndIf
\algstore{97algo}
\end{algorithmic}
\end{algorithm}
\begin{algorithm}[H]
\begin{algorithmic}[1]
\algrestore{97algo}
\EndFor
\EndFor
\If{$F[L_1][L_2] > 0$}
\State \Return \texttt{true}
\Else
\State \Return \texttt{false}
\EndIf
\EndProcedure
\end{algorithmic}
\end{algorithm}
\begin{CJK*}{UTF8}{gbsn}
也可以只用一个one dimension array作为$F$,这时候$F$的长度为$L_2$(也可以是$L_1$)。那么类似的有
\end{CJK*}
\[
F(c) = 
\begin{cases}
\texttt{false} &\qquad s_1[r-1]\neq s_3[r+c-1], \;s_2[c-1]\neq s_3[r+c-1]\\
F[c] &\qquad s_1[r-1]= s_3[r+c-1] \; \texttt{or} \\
F[c-1] &\qquad s_2[c-1]= s_3[r+c-1] 
\end{cases}
\]
\begin{CJK*}{UTF8}{gbsn}
不过循环仍然是二维循环。
\end{CJK*}