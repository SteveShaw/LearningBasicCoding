\section{18 --- 4Sum}
Given an array $A$ of $ n $ integers and an integer target $ T $, are there elements $ a $, $ b $, $ c $, and $ d $ in $ A $ such that $ a + b + c + d = T $? Find all unique quadruplets in the array which gives the sum of target.

\paragraph{Note:}

\begin{itemize}
\item The solution set must not contain duplicate quadruplets.
\end{itemize}

\paragraph{Example:}

\begin{flushleft}
\textbf{Input}: $A = [1, 0, -1, 0, -2, 2]$, $T = 0$
\\
\textbf{Output}: 
\begin{table}[H]
\begin{tabular}{lcccc}
1): & -1 &  0 & 0 & 1  \\
2): & -2 & -1 & 1 & 2  \\
3): & -2  & 0 & 0 & 2 
\end{tabular}
\end{table}
\end{flushleft}

\subsection{Sort}
为了避免重复项,使用了Tree Set保证新加入的数没有重复。解法思路跟三数之和基本没啥区别,就是多加了一层for循环,其他的都一样,当然同样都要先对数组进行排序。

\setcounter{lstlisting}{0}
\begin{lstlisting}[style=customc, caption={Sorting}]
vector<vector<int>> fourSum( vector<int>& nums, int target )
{
    sort( nums.begin(), nums.end() );

    int L = static_cast< int >( nums.size() );

    vector<vector<int>> ans;

    for( int i = 0; i <= L - 4; ++i )
    {
        if( ( i > 0 ) && ( nums[i] == nums[i - 1] ) )
        {
            continue;
        }

        for( int j = i + 1; j <= L - 3; ++j )
        {
            //skip duplicate numbers
            if( ( j > i + 1 ) && ( nums[j] == nums[j - 1] ) )
            {
                continue;
            }

            int l = j + 1;
            int r = L - 1;

            while( l < r )
            {
                if( nums[i] + nums[j] + nums[l] + nums[r] == target )
                {
                    ans.emplace_back( initializer_list<int> {nums[i], nums[j], nums[l], nums[r]} );

                    //skip duplicate numbers
                    while( ( l < r ) && ( nums[l] == nums[l + 1] ) )
                    {
                        ++l;
                    }

                    //skip duplicate numbers
                    while( ( l < r ) && ( nums[r] == nums[r - 1] ) )
                    {
                        --r;
                    }

                    //important: l+1 and r-1 is the first number
                    //that is not same as before
                    ++l;
                    --r;
                }
                else if( nums[i] + nums[j] + nums[l] + nums[r] < target )
                {
                    ++l;
                }
                else
                {
                    --r;
                }
            }

        }

    }

    return ans;
}

\end{lstlisting}
