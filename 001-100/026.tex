\section{26 --- Remove Duplicates from Sorted Array}

Given a sorted array $A$, remove the duplicates in-place such that each element appear only once and return the new length.

Do not allocate extra space for another array, you must do this by modifying the input array in-place with $O(1)$ extra memory.

\paragraph{Example 1:}

\begin{flushleft}
\textbf{Input}: $A = [1,1,2]$

\textbf{Output}: 2

\textbf{Explanation}: Your function should return length equal to 2 with the first two elements of $A$ being 1 and 2 respectively.
\end{flushleft}

\paragraph{Example 2:}

\begin{flushleft}
\textbf{Input}: $A = [0,0,1,1,1,2,2,3,3,4]$

\textbf{Output}: 5

\textbf{Explanation}: 
Your function should return length equal to 5, with the first five elements of nums being modified to 0, 1, 2, 3, and 4 respectively. 

It doesn't matter what values are set beyond the returned length.
\end{flushleft}

\subsection{Binary Search With Two Pointers}
\begin{itemize}
\item 用一个write指针代表写入的位置,用read指针从$A$中读取数字。
\item 遇到重复数字,用rightmost binary search快速定位第一个与当前数字不同的数字。
\end{itemize}

\setcounter{lstlisting}{0}
\begin{lstlisting}[style=customc, caption={Binary Search With Two Pointers}]
int removeDuplicates( vector<int>& nums )
{
    int L = static_cast<int>( nums.size() );

    int write = 0;

    int i = 0;

    while( i < L )
    {
        if( i == L - 1 )
        {
            nums[write] = nums[i];
            ++write;
            break;
        }

        if( nums[i] == nums[i + 1] )
        {
            //search for the end of duplicate
            //by binary search

            int l = i;
            int r = L;

            //rightmost binary search
            //find the first element
            //that is larger than nums[i]

            while( l < r )
            {
                int m = ( l + r ) / 2;
                if( nums[m] <= nums[i] )
                {
                    l = m + 1;
                }
                else
                {
                    r = m;
                }
            }

            //write current element into write
            nums[write] = nums[i];
            //increments the write
            ++write;

            //set next start at l
            i = l;
        }
        else
        {
            //nums[i] is not duplicate
            //directly put into write
            nums[write] = nums[i];
            //increments the write
            ++write;
            //increments i
            ++i;
        }
    }

    //nums[0...write-1] are all unique elements
    return write;
}
\end{lstlisting}


