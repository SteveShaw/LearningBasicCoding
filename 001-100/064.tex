\section{64 --- Minimum Path Sum}
Given a $m \times n$ grid filled with non-negative numbers, find a path from top left to bottom right which minimizes the sum of all numbers along its path.

Note: You can only move either down or right at any point in time.

\paragraph{Example:}
\begin{flushleft}


\textbf{Input}:
\[
\begin{bmatrix}
1 & 3 & 1\\
1 & 5 & 1\\
4 & 2 & 1
\end{bmatrix}
\]

\textbf{Output}: 7

\textbf{Explanation}: Because the path $1\to 3 \to 1 \to 1 \to 1$ minimizes the sum.


\end{flushleft}

\subsection{Dynamic Programming 2D}
Suppose $F[i][j]$ is the minimum sum of the path from the index $(i,j)$ to the bottom rightmost element. 

We start by initializing the bottom rightmost element of $F$ as the last element of the given matrix. Then for each element starting from the bottom right, we traverse backwards and fill in the matrix with the required minimum sums. 

Now, we need to note that at every element, we can move either rightwards or downwards. Therefore, for filling in the minimum sum, we use the equation: $F[i][j] = G[i][j] + \min(F[i+1][j], F[i][j+1])$.

\setcounter{lstlisting}{0}
\begin{lstlisting}[style=customc, caption={2D DP}]
int minPathSum( vector<vector<int>>& grid )
{
    auto M = grid.size();
    auto N = grid[0].size();

    vector<vector<int>> F( M, vector<int>( N, 0 ) );

    F[M - 1][N - 1] = grid[M - 1][N - 1];
    //fill the right column
    for( size_t r = M - 1; r >= 1; --r )
    {
        F[r - 1][N - 1] = F[r][N - 1] + grid[r - 1][N - 1];
    }
    //fill the last row
    for( size_t c = N - 1; c >= 1; --c )
    {
        F[M - 1][c - 1] = F[M - 1][c] + grid[M - 1][c - 1];
    }
    //Fill DP from bottom right to to left
    for( size_t r = M - 1; r >= 1; --r )
    {
        for( size_t c = N - 1; c >= 1; --c )
        {
            //current value plus minimum of
            //its right and down neighbors
            F[r - 1][c - 1] = grid[r - 1][c - 1] + ( min )( F[r][c - 1], F[r - 1][c] );
        }
    }

    return F[0][0];
}
\end{lstlisting}