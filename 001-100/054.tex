\section{54 --- Spiral Matrix}
Given a matrix of $m \times n$ elements ($m$ rows, $n$ columns), return all elements of the matrix in spiral order.
\paragraph{Example 1:}
\begin{flushleft}
\textbf{Input}:
\begin{figure}[H]
    \begin{tikzpicture}
    \matrix [matrix of nodes]
    {
        1 & 2 & 3 \\
        4 & 5 & 6 \\
        7 & 8 & 9\\
    };
    \end{tikzpicture}
\end{figure}
\par
\textbf{Output}: $[1,2,3,6,9,8,7,4,5]$
\end{flushleft}
\paragraph{Example 2:}
\begin{flushleft}
\textbf{Input}:
\par
\begin{figure}[H]
\begin{tikzpicture}
    \matrix [matrix of nodes]
    {
          1 & 2 & 3 & 4\\
          5 & 6 & 7 & 8\\
          9 & 10 & 11 & 12 \\
    };
\end{tikzpicture}
\end{figure}
\par
\textbf{Output}: $[1,2,3,4,8,12,11,10,9,5,6,7]$
\end{flushleft}

\subsection{Simulation}
Draw the path that the spiral makes. We know that the path should turn clockwise whenever it would go out of bounds or into a cell that was previously visited.
\par
Suppose the matrix $M$ have $R$ rows and $C$ columns. $\nu[r][c]$ is a $0/1$ auxiliary matrix denotes that the cell on the $r$-th row and $c$-th column was previously visited. Our current position is $(r_0, c_0)$, facing direction $\theta$, and we want to visit $R\times C$ total cells.
\par
As we move through the matrix, our candidate next position is $(r_1, c_1)$. If the candidate is in the bounds of the matrix and not seen before, i.e. not in $\nu$, then it becomes our next position; otherwise, our next position is the one after performing a clockwise turn.

The input parameter includes the matrix $A$, number of rows $R$ and number of columns $C$. The return value is array $\phi$, the elements that visited through spiral order.


\setcounter{lstlisting}{0}
\begin{lstlisting}[style=customc, caption={Simulation}]
vector<int> spiralOrder( vector<vector<int>>& matrix )
{
    if( matrix.empty() || matrix[0].empty() )
    {
        return {};
    }

    vector<vector<unsigned char>> seen( matrix.size(), vector<unsigned char>( matrix[0].size(), 0 ) );

    //the sequence cannot be
    //changed.
    int dr[] = {0, 1, 0, -1};
    int dc[] = {1, 0, -1, 0};

    vector<int> ans;

    int r = 0;
    int c = 0;
    //direction
    int d = 0;

    int rows = static_cast<int>( matrix.size() );
    int cols = static_cast<int>( matrix[0].size() );
    int total = static_cast<int>( rows * cols );

    ans.reserve( total );

    for( int i = 0; i < total; ++i )
    {
        ans.push_back( matrix[r][c] );
        seen[r][c] = 1;

        int nr = r + dr[d];
        int nc = c + dc[d];

        if( ( nr >= 0 ) && ( nr < rows ) && ( nc >= 0 ) && ( nc < cols ) && ( !seen[nr][nc] ) )
        {
            //(nr,nc) is inside range
            //and (nr,nc) has not seen
            r = nr;
            c = nc;
        }
        else
        {
            //change the direction
            d = ( ( d + 1 ) % 4 );
            r += dr[d];
            c += dc[d];
        }
    }

    return ans;
}
\end{lstlisting}
