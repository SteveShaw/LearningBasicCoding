\section{54 --- Spiral Matrix}
Given a matrix of $m \times n$ elements ($m$ rows, $n$ columns), return all elements of the matrix in spiral order.
\paragraph{Example 1:}
\begin{flushleft}
\textbf{Input}:
\begin{figure}[H]
    \begin{tikzpicture}
    \matrix [matrix of nodes]
    {
        1 & 2 & 3 \\
        4 & 5 & 6 \\
        7 & 8 & 9\\
    };
    \end{tikzpicture}
\end{figure}
\par
\textbf{Output}: $[1,2,3,6,9,8,7,4,5]$
\end{flushleft}
\paragraph{Example 2:}
\begin{flushleft}
\textbf{Input}:
\par
\begin{figure}[H]
\begin{tikzpicture}
    \matrix [matrix of nodes]
    {
          1 & 2 & 3 & 4\\
          5 & 6 & 7 & 8\\
          9 & 10 & 11 & 12 \\
    };
\end{tikzpicture}
\end{figure}
\par
\textbf{Output}: $[1,2,3,4,8,12,11,10,9,5,6,7]$
\end{flushleft}

\subsection{Simulation}
Draw the path that the spiral makes. We know that the path should turn clockwise whenever it would go out of bounds or into a cell that was previously visited.
\par
Suppose the matrix $M$ have $R$ rows and $C$ columns. $\nu[r][c]$ is a $0/1$ auxiliary matrix denotes that the cell on the $r$-th row and $c$-th column was previously visited. Our current position is $(r_0, c_0)$, facing direction $\theta$, and we want to visit $R\times C$ total cells.
\par
As we move through the matrix, our candidate next position is $(r_1, c_1)$. If the candidate is in the bounds of the matrix and not seen before, i.e. not in $\nu$, then it becomes our next position; otherwise, our next position is the one after performing a clockwise turn.

The input parameter includes the matrix $A$, number of rows $R$ and number of columns $C$. The return value is array $\phi$, the elements that visited through spiral order.
\setcounter{algorithm}{0}
\begin{algorithm}[H]
\caption{Simulation Based Solution}
\begin{algorithmic}[1]
\Procedure{SpiralOrder}{$A,R,C$}
\If{$R=0$ \textbf{or} $C=0$}
\State \Return $\emptyset$ \Comment Boundary case.
\EndIf
\State $\phi := \emptyset$
\State $\nu$ as $\nu[0][0] = \ldots  = \nu[R-1][C-1] := 0$
\State $\delta_R$ as $\delta_R[4] = (0, 1, 0, -1)$ \Comment The direction order will be right, down, left and up
\State $\delta_C$ as $\delta_C[4] = (1, 0, -1, 0)$
\State $r :=0$
\State $c :=0$
\State $\theta := 0$
\For{$i := 0$ \textbf{to} $R\times C -1$}
\State $\phi \gets \phi \cup A[r][c]$
\State $\nu[r][c] \gets 1$ \Comment Mark this element as visited
\State $r_n := r + \delta_R[\theta]$ \Comment Get next element row in the direction $\theta$
\State $c_n := c + \delta_C[\theta]$ \Comment Get next element column in the direction $\theta$ 
\If{$0 \leq r_n < R$ \textbf{and} $0 \leq c_n < C$ \textbf{and} $\nu[r_n][c_n] = 0$ }
\State $r\gets r_n$
\State $c\gets c_n$
\Else
\State $\theta \gets (\theta + 1) \bmod 4$ \Comment Get next direction
\State $r\gets r + \delta_R[\theta]$
\State $c\gets c + \delta_R[\theta]$
\EndIf
\State \Return $\phi$
\EndFor
\EndProcedure
\end{algorithmic}
\end{algorithm}
