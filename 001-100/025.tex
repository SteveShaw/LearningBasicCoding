\section{25 --- Reverse Nodes in k-Group}
Given a linked list, reverse the nodes of a linked list $k$ at a time and return its modified list.

$k$ is a positive integer and is less than or equal to the length of the linked list. If the number of nodes is not a multiple of k then left-out nodes in the end should remain as it is.

\paragraph{Example:}

\begin{flushleft}
Given this linked list: 

\begin{figure}[H]
\begin{tikzpicture}
[start chain, every node/.style={draw, circle, minimum size=6mm, fill=gray!20!}, node distance=8mm, every join/.style={>=stealth,->}, thick]
\node[on chain] {1};
\node[on chain, join] {2};
\node[on chain, join] {3};
\node[on chain, join] {4};
\node[on chain, join] {5};
\end{tikzpicture}
\end{figure}
For k = 2, you should return:

\begin{figure}[H]
\begin{tikzpicture}
[start chain, every node/.style={draw, circle, minimum size=6mm, fill=gray!20!}, node distance=8mm, every join/.style={>=stealth,->}, thick]
\node[on chain] {2};
\node[on chain, join] {1};
\node[on chain, join] {4};
\node[on chain, join] {3};
\node[on chain, join] {5};
\end{tikzpicture}
\end{figure}

For k = 3, you should return: 
\begin{figure}[H]
\begin{tikzpicture}
[start chain, every node/.style={draw, circle, minimum size=6mm, fill=gray!20!}, node distance=8mm, every join/.style={>=stealth,->}, thick]
\node[on chain] {3};
\node[on chain, join] {2};
\node[on chain, join] {1};
\node[on chain, join] {4};
\node[on chain, join] {5};
\end{tikzpicture}
\end{figure}

\end{flushleft}


\paragraph{Note:}

\begin{itemize}
\item Only constant extra memory is allowed.
\item You may not alter the values in the list's nodes, only nodes itself may be changed.
\end{itemize}

\subsection{Pointer Operation}
以图中给出的链表为例, 创建一个\texttt{dummy node},因为翻转链表时头结点可能会变化,那么加入这个node后的链表变为
\begin{figure}[H]
\begin{tikzpicture}
[start chain, 
every node/.style={draw, circle,
 minimum size=6mm, fill=gray!20!},
  node distance=8mm, 
  every join/.style={>=stealth,->},
 thick
]
\node[on chain] {$-1$};
\node[on chain, join] {2};
\node[on chain, join] {3};
\node[on chain, join] {4};
\node[on chain, join] {5};
\end{tikzpicture}
\end{figure}

如果$k$为3的话,目标是将1, 2, 3翻转一下,需要维护两个 pointers: $x$和$y$,分别指向需要翻转的$k$个nodes部分的前后位置,如下图所示

\begin{figure}[H]
\begin{tikzpicture}
[start chain, 
every node/.style={draw, circle,
 minimum size=8mm, fill=gray!20!},
  node distance=8mm and 8mm, 
  every join/.style={>=stealth,->},
 thick
]
\node(1)[on chain] {$-1$};
\node(x)[below=of 1] {$x$};
\node[on chain, join] {1};
\node[on chain, join] {2};
\node[on chain, join] {3};
\node(4)[on chain, join] {4};
\node(y) [below=of 4] {$y$};
\node[on chain, join] {5};
\draw[>=stealth,->] (x) -- (1);
\draw[>=stealth,->] (y) -- (4);
\end{tikzpicture}
\end{figure}

reverse完成后,$x$更新到下图所示的位置:

\begin{figure}[H]
\begin{tikzpicture}
[start chain, 
every node/.style={draw, circle,
 minimum size=8mm, fill=gray!20!},
  node distance=8mm and 8mm, 
  every join/.style={>=stealth,->},
 thick
]
\node(1)[on chain] {$-1$};
\node[on chain, join] {3};
\node(2)[on chain, join] {2};
\node(3)[on chain, join] {1};
\node(4)[on chain, join] {4};
\node[on chain, join] {5};
\node(x)[below=of 3] {$x$};
\node(y) [below=of 4] {$y$};
\draw[>=stealth,->] (x) -- (3);
\draw[>=stealth,->] (y) -- (4);
\end{tikzpicture}
\end{figure}

以此类推,只要\texttt{y}走过$k$个节点,就可以进行局部翻转了。反转过程可参见下面的示意图。

给定的链表如下图所示: 其中start node is $S$, end node is $E$. To reverse node between $S$ and $E$

\begin{figure}[H]
\centering
\begin{tikzpicture}
[start chain, 
every node/.style={draw, circle,
 minimum size=6mm, fill=gray!20!},
  node distance=8mm, 
  every join/.style={>=stealth,->},
 thick
]
\node[on chain] {$\ldots$};
\node[on chain, join](1) {$S$};
\node[on chain, join](2) {$P$};
\node[on chain, join](3) {$C$};
\node[on chain, join](4) {$N$};
\node[on chain, join] {$\ldots$};
\node[on chain, join](5) {$E$};
\node[on chain, join] {$\ldots$};
\end{tikzpicture}
\end{figure}

\begin{figure}[H]
\label{fig01}
\caption{Break $P$ with $C$ and link $P$ to $N$}
\centering
\begin{tikzpicture}
[start chain, 
every node/.style={draw, circle,
 minimum size=6mm, fill=gray!20!},
  node distance=8mm, 
  every join/.style={>=stealth,->},
 thick
]
\node[on chain] {$\ldots$};
\node[on chain, join](1) {$S$};
\node[on chain, join](2) {$P$};
\node[on chain](3) {$C$};
\node[on chain, join](4) {$N$};
\node[on chain, join] {$\ldots$};
\node[on chain, join](5) {$E$};
\node[on chain, join] {$\ldots$};
\draw[>=stealth, ->] (2) [bend left=60] to (4);
\end{tikzpicture}
\end{figure}

\begin{figure}[H]
\caption{Break $C$ with $N$ and set $C$ to next node of $S$}
\centering
\begin{tikzpicture}
[start chain, 
every node/.style={draw, circle,
 minimum size=6mm, fill=gray!20!},
  node distance=8mm, 
  every join/.style={>=stealth,->},
 thick
]
\node[on chain] {$\ldots$};
\node[on chain, join](1) {$S$};
\node[on chain, join](2) {$P$};
\node[on chain](3) {$C$};
\node[on chain](4) {$N$};
\node[on chain, join] {$\ldots$};
\node[on chain, join](5) {$E$};
\draw[>=stealth, ->] (2) [bend left=60] to (4);
\draw[>=stealth, ->] (3) [bend left=60] to (2);
\end{tikzpicture}
\end{figure}

\begin{figure}[H]
\caption{Break $S$ and $P$, link $S$ to $C$}
\centering
\begin{tikzpicture}
[start chain, 
every node/.style={draw, circle,
 minimum size=6mm, fill=gray!20!},
  node distance=8mm, 
  every join/.style={>=stealth,->},
 thick
]
\node[on chain] {$\ldots$};
\node[on chain, join](1) {$S$};
\node[on chain](2) {$P$};
\node[on chain](3) {$C$};
\node[on chain](4) {$N$};
\node[on chain, join] {$\ldots$};
\node[on chain, join](5) {$E$};
\node[on chain, join] {$\ldots$};
\draw[>=stealth, ->] (2) [bend left=60] to (4);
\draw[>=stealth, ->] (3) [bend left=60] to (2);
\draw[>=stealth, ->] (1) [bend left=280] to (3);
\end{tikzpicture}
\end{figure}


\begin{figure}[H]
\caption{Set next node of $S$ to $C$}
\centering
\begin{tikzpicture}
[start chain, 
every node/.style={draw, circle,
 minimum size=6mm, fill=gray!20!},
  node distance=8mm, 
  every join/.style={>=stealth,->},
 thick
]
\node[on chain] {$\ldots$};
\node[on chain, join](1) {$S$};
\node[on chain](2) {$P$};
\node[on chain](3) {};
\node[on chain](4) {$C$};
\node[on chain, join] {$\ldots$};
\node[on chain, join](5) {$E$};
\node[on chain, join] {$\ldots$};
\draw[>=stealth, ->] (2) [bend left=60] to (4);
\draw[>=stealth, ->] (3) [bend left=60] to (2);
\draw[>=stealth, ->] (1) [bend left=280] to (3);
\end{tikzpicture}
\end{figure}

By the figures shows above, we put previous $C$ before $P$ and link $S$ to $C$. Then move $C$ to previous $N$, and repeat the process by staring similarly to figure \ref{fig01}.

At the end of the reverse, $C$ will be at $E$, while $P$ is not moved and the next node of $S$ is the previous node of $E$.

\begin{figure}[H]
\caption{Update $C$ to $P$'s next node}
\centering
\begin{tikzpicture}
[start chain, 
every node/.style={draw, circle,
 minimum size=6mm, fill=gray!20!},
  node distance=8mm, 
  every join/.style={>=stealth,->},
 thick
]
\node[on chain] {$\ldots$};
\node[on chain, join](1) {$S$};
\node[on chain](2) {$P$};
\node[on chain](3) {};
\node[on chain](4) {$\ldots$};
\node[on chain](T) {};
\node[on chain, label=above:$C$](5) {$E$};
\node[on chain, join] {$\ldots$};
\draw[>=stealth,->] (2) [bend left=60] to (5);
\draw[>=stealth,->] (1) [bend left=300] to (T);
\draw[>=stealth,->] (T) -- (4);
\draw[>=stealth,->] (4) -- (3);
\draw[>=stealth,->] (3) -- (2);
\end{tikzpicture}
\end{figure}


\setcounter{algorithm}{0}
\begin{algorithm}[H]
\caption{Reverse link list by $k$ steps}
\begin{algorithmic}[1]
\Procedure{ReverseKGroup}{$L, K$}
\State $\star$ Create a dummy node $d$.
\State $s:=d$ \Comment The pointer to the previous node of the group to be reversed
\State $\star$ Link $s$ to $L$ \Comment Link dummy node to the linked list
\State $\ast$ Start the process
\State $c:=L$ \Comment Current Node
\State $\delta:=0$ \Comment The counter
\While{$c$ is not null}
\State $\delta\gets\delta + 1$
\If{$\delta=K$}
\State $\delta=0$
\State $s\gets$\Call{Reverse}{$s, c_n$} \Comment Reverse the group starting with next node of $s$ and ending with $c$
\State $\star$ Update $c$ as the next node of $x$
\Else
\State $\star$ Update $c$ to its next node
\EndIf
\EndWhile
\EndProcedure
\end{algorithmic}
\end{algorithm}

Function \texttt{Reverse} reverse the segment in list between start node $s$ and end node $e$

\begin{algorithm}[H]
\caption{Reverse one segment of link list }
\begin{algorithmic}[1]
\Function{Reverse}{$s,e$}
\State $p := s_n$ \Comment $x$ is the next node of $s$
\State $\ast$ Starting reverse from next node of $x$
\State $c := p_n$ 
\While{$c \neq e$}
\State $\ast$ Break $p$ with $c$, and link $p$ to next node of $c$
\State $p_n \gets c_n $
\State $\ast$ Set next node of $c$ to next node of $s$
\State $c_n \gets s_n $ 
\State $\ast$ Set next node of $s$ to $c$
\State $s_n \gets c$ 
\State $\ast$ Update $c$ as next node of $p$
\State $c\gets p_n$
\EndWhile
\State \Return $p$ \Comment next node of $p$ is the start of next group
\EndFunction
\end{algorithmic}
\end{algorithm}


\setcounter{lstlisting}{0}
\begin{lstlisting}[style=customc, caption={Pointer operation}]
ListNode* reverseKGroup( ListNode* head, int k )
{
    ListNode* dummy = new ListNode( 0 );

    auto S = dummy;

    //link S to head
    S->next = head;

    auto x = head;

    int count = 0;

    while( x != nullptr )
    {
        //since count starting with zero
        //increment count first
        ++count;

        if( count == k )
        {
            count = 0;
            //reverse S to x->next
            S = reverse( S, x->next );
            //update x to S->next
            x = S->next;
        }
        else
        {
            x = x->next;
        }
    }

    return dummy->next;
}
//Helper function to reverse between S and E
ListNode* reverse( ListNode* S, ListNode* E )
{
    auto P = S->next;

    auto C = P->next;

    while( C != E )
    {
        //break P with C and link to C->next
        P->next = C->next;

        //link C to S->next
        C->next = S->next;

        //link S to C
        S->next = C;

        //update C to P->next
        C = P->next;
    }

    //P->next is the start of next group
    return P;
}
\end{lstlisting}