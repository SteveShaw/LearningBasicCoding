\section{56 --- Merge Intervals}
Given a collection of intervals, merge all overlapping intervals.

\paragraph{Example 1:}

\begin{flushleft}
\textbf{Input}: $[[1,3],[2,6],[8,10],[15,18]]$

\textbf{Output}: $[[1,6],[8,10],[15,18]]$

\textbf{Explanation}: Since intervals $ [1,3] $ and $ [2,6] $ overlaps, merge them into $ [1,6] $.
\end{flushleft}

\paragraph{Example 2:}

\begin{flushleft}
\textbf{Input}: $ [[1,4],[4,5]] $

\textbf{Output}:$  [[1,5]] $

\textbf{Explanation}: Intervals $ [1,4] $ and $ [4,5] $ are considered overlapping.
\end{flushleft}

\subsection{Sorting}
\begin{itemize}
\item If we sort the intervals by their start value, then each set of intervals that can be merged by scanning the list.
\item First, we sort the list as described. Then, we insert the first interval into the merged list and continue considering each interval in turn as follows: If the current interval begins after the previous interval ends, then they do not overlap and we can append the current interval to merged. Otherwise, they do overlap, and we merge them by updating the end of the previous interval if it is less than the end of the current interval.
\end{itemize}

\setcounter{lstlisting}{0}
\begin{lstlisting}[style=customc, caption={Sort}]
vector<vector<int>> merge( vector<vector<int>>& intervals )
{
    if( intervals.empty() )
    {
        return {};
    }

    //we sort per begin of each interval
    //This is differenct from the problem where we count how many overlapped ranges
    //where we sort per the end of each interval
    sort( intervals.begin(), intervals.end(), []( const vector<int>& i, const vector<int>& j )
    {
        if( i[0] < j[0] )
        {
            return true;
        }

        if( i[0] == j[0] )
        {
            return i[1] < j[1];
        }

        return false;
    } );

    vector<vector<int>> ans;

    ans.emplace_back( intervals[0].begin(), intervals[0].end() );

    for( size_t i = 1; i < intervals.size(); ++i )
    {
        auto& last = ans.back();
        //current interval begin is less than
        //last interval's end
        if( intervals[i][0] <= last[1] )
        {
            //can merge
            last[1] = ( max )( last[1], intervals[i][1] );
        }
        else
        {
            //no overlap
            ans.emplace_back( intervals[i].begin(), intervals[i].end() );
        }
    }

    return ans;
}
\end{lstlisting}