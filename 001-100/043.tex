\section{43 --- Multiply Strings}
Given two non-negative integers $A$ and $B$ represented as strings, return the product of $A$ and $B$, also represented as a string.

\paragraph{Example 1:}

\begin{flushleft}
\textbf{Input}: $A = 2$, $B = 3$

\textbf{Output}: 6
\end{flushleft}

\paragraph{Example 2:}

\begin{flushleft}
\textbf{Input}: $A = 123$, $B = 456$

\textbf{Output}: 56088
\end{flushleft}

\paragraph{Note:}

\begin{itemize}
\item The length of both $A$ and $B$ is less than 110.
\item Both $A$ and $B$ contain only digits $0--9$.
\item Both $A$ and $B$ do not contain any leading zero, except the number 0 itself.
\item You must not use any built-in \textbf{BigInteger} library or convert the inputs to integer directly.
\end{itemize}

\subsection{Mathematical Way}
\begin{itemize}
\item 需要一个额外的数组用来存放中间的运算过程。
\item 最后需要移除前面的零。
\end{itemize}


\setcounter{lstlisting}{0}
\begin{lstlisting}[style=customc, caption={Mathematical Way}]
string multiply( string num1, string num2 )
{
    size_t len_1 = num1.size();
    size_t len_2 = num2.size();

    //the maximum size of the product
    //is len_1 + len_2
    vector<int> product( len_1 + len_2, 0 );

    //reverse input numbers
    //so that the multiply is starting
    //from the beginning
    reverse( begin( num1 ), end( num1 ) );
    reverse( begin( num2 ), end( num2 ) );

    for( size_t i = 0; i < len_1; ++i )
    {
        for( size_t j = 0; j < len_2; ++j )
        {
            int v = ( num1[i] - '0' ) * ( num2[j] - '0' );

            product[i + j] += v;
            //accumulate the carry into next position
            product[i + j + 1] += product[i + j] / 10;
            //keep the number less than 10 in current position
            product[i + j] %= 10;
        }
    }

    //remove prefix zeros
    while( !product.empty() && ( product.back() == 0 ) )
    {
        product.pop_back();
    }

    if( product.empty() )
    {
        return "0";
    }

    string ans;

    //we need to reverse the array
    for( size_t i = 0; i < product.size(); ++i )
    {
        auto j = product.size() - 1 - i;
        ans.push_back( product[j] + '0' );
    }

    return ans;
}
\end{lstlisting}