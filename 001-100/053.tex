\section{53 --- Maximum Subarray}
Given an integer array $A$, find the contiguous subarray (containing at least one number) which has the largest sum and return its sum.
\paragraph{Example:}
\begin{flushleft}
\textbf{Input}: $[-2,1,-3,4,-1,2,1,-5,4]$,
\\
\textbf{Output}: 6
\\
\textbf{Explanation}: 
\\
$[4,-1,2,1]$ has the largest sum 6.
\end{flushleft}

\paragraph{Follow up:}
\begin{flushleft}
If you have figured out the $O(n)$ solution, try coding another solution using the divide and conquer approach, which is more subtle.
\end{flushleft}

\subsection{Kadane Algorithm}
\setcounter{algorithm}{0}
\begin{algorithm}[H]
\caption{Kadane Algorithm}
\begin{algorithmic}[1]
\Procedure{MaxSubArray}{$A$, $n$}
\State $\alpha := A[0]$ \Comment Current maximum
\State $\beta:=A[0]$ \Comment Maximum So far
\For{$i:=1$ \textbf{to} $n-1$}
\State $\alpha \gets A[i] + \max(\alpha, 0)$
\State $\beta \gets \max(\beta, \alpha)$
\EndFor
\State \Return $\beta$
\EndProcedure
\end{algorithmic}
\end{algorithm}

\setcounter{lstlisting}{0}
\begin{lstlisting}[style=customc, caption={Kadane Algorithm}]
int maxSubArray( vector<int>& nums )
{
    int cur_max = nums[0];
    int global_max = nums[0];

    for( size_t i = 1; i < nums.size(); ++i )
    {
        cur_max = nums[i] + ( max )( cur_max, 0 );
        global_max = ( max )( global_max, cur_max );
    }

    return global_max;
}
\end{lstlisting}

\subsection{Divide And Conquer}
The problem is a classical example of divide and conquer approach, and can be solved with the algorithm similar with the merge sort.

The solution template for the divide and conquer problems is as follows:

\begin{itemize}
\item Define the base case(s).

\item Split the problem into subproblems and solve them recursively.

\item Merge the solutions for the subproblems to obtain the solution for the original problem.
\end{itemize}

应用到这个问题中,算法如下:

\begin{enumerate}
\item Base case: if $n=\lvert A\rvert=1$, return $A[0]$
\item Get result $l$ for left sub-array, i.e., the first $n/2$ elements.
\item Get result $r$ for right sub-array, i.e., the last $n/2$ elements.
\item Get result $c$ that crossing the middle point.
\item Merge the solutions above, i.e, return $\max(l, r, c)$
\end{enumerate}

\begin{lstlisting}[style=customc, caption={Divide And Conquer}]
int maxSubArray( vector<int>& nums )
{
    int n = static_cast<int>( nums.size() );
    return f( nums, 0, n - 1 );
}

//helper function for recursion
int f( vector<int>& nums, int low, int high )
{
    if( low == high )
    {
        //base case:
        return nums[low];
    }

    int middle = ( low + high ) / 2;

    //find left half result
    int l = f( nums, low, middle );
    //find right half result
    int r = f( nums, middle + 1, high );
    //find cross sum
    int c = max_crossing_sum( nums, low, high, middle );

    //return the maximum result of (l, r, c)
    int x = ( max )( l, r );
    x = ( max )( x, c );
    return x;
}

//get maximum sum cross the middle point
int max_crossing_sum( vector<int>& A, int low, int high, int middle )
{
    //get maximum sum in [low,middle]
    int left_sum = INT_MIN;
    int sum = 0;

    for( int i = middle; i >= low; --i )
    {
        sum += A[i];
        left_sum = ( max )( sum, left_sum );
    }

    //get maximum sum in [middle+1, high]
    int right_sum = INT_MIN;
    sum = 0;
    int max_right = middle;

    for( int k = middle + 1; k <= high; ++k )
    {
        sum += A[k];
        right_sum = ( max )( sum, right_sum );
    }

    //combine the two sum
    return left_sum + right_sum;
}
\end{lstlisting}