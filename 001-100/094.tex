\section{94 --- Binary Tree Inorder Traversal}
Given a binary tree $T$, return the \textbf{inorder} traversal of its nodes' values.
\paragraph{Example:}
\begin{flushleft}
\textbf{Input}: 
\begin{figure}[H]
\begin{tikzpicture}
[mynode/.style={draw,circle,minimum size=5mm, fill=gray!20!}]
\node(){};
\node[mynode](1) {1};
\node[mynode](2)[below=8mm of 1, xshift=8mm] {2};
\node[mynode](3)[below=8mm of 2, xshift=-8mm] {3};
\draw[-] (1) -- (2);
\draw[-] (2) -- (3);
\end{tikzpicture}
\end{figure}
\textbf{Output}: $[1,3,2]$
\end{flushleft}
\paragraph{Follow up}
\begin{flushleft}
Recursive solution is trivial, could you do it iteratively?
\end{flushleft}
\subsection{Iterative Method Based On Stack}
\begin{CJK*}{UTF8}{gbsn}
从root开始,先将root压入栈,然后再将其所有left child nodes压入栈,然后取出栈顶node,保存节点值,再将当前指针移到其right child node上,若存在right child node,则在下次循环时又可将其所有left child nodes压入栈中。这样就保证了访问顺序为left--root--right。
\end{CJK*}
\setcounter{algorithm}{0}
\begin{algorithm}[H]
\caption{Iterative With Stack}
\begin{algorithmic}[1]
\Procedure{InorderTraversal}{$T$}
\State $S:=\emptyset$ \Comment The stack
\State $p:=T$ \Comment Start with root $T$
\State $\texttt{ans}:=\emptyset$
\While{$p\neq \texttt{NULL}$ \textbf{or} $S\neq \emptyset$}
\While{$p\neq \texttt{NULL}$} \Comment Push all left child nodes $p$ to $S$
\State $S\gets S + p$
\State $p\gets \texttt{Left}(p)$
\EndWhile
\algstore{94algo}
\end{algorithmic}
\end{algorithm}
\begin{algorithm}[H]
\begin{algorithmic}[1]
\algrestore{94algo}
\State $p\gets \texttt{Top}(S)$
\State $\texttt{Pop}(S)$
\State $\texttt{ans}\gets \texttt{ans}+ \texttt{Val}(p)$ \Comment Visit $p$
\State $p\gets \texttt{Right}(p)$
\EndWhile
\State \Return \texttt{ans}
\EndProcedure
\end{algorithmic}
\end{algorithm}