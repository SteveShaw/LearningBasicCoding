\section{55 --- Jump Game}
Given an array of non-negative integers, $A$, you are initially positioned at the first index of the array.

Each element in the array represents your maximum jump length at that position.

Determine if you are able to reach the last index.

\paragraph{Example 1:}

\begin{flushleft}
\textbf{Input}: $[2,3,1,1,4]$

\textbf{Output}: \texttt{true}

\textbf{Explanation}: Jump 1 step from index 0 to 1, then 3 steps to the last index.

\end{flushleft}

\paragraph{Example 2:}

\begin{flushleft}
\textbf{Input}: $[3,2,1,0,4]$

\textbf{Output}: \texttt{false}

\textbf{Explanation}: 

You will always arrive at index 3 no matter what. Its maximum jump length is 0, which makes it impossible to reach the last index.
\end{flushleft}


\subsection{Graphic Level Equivalent}
和45题Jump Game II解法基本一致,同样将其看作是多个level的图, $\alpha_0$表示当前level的ending position,$\alpha_1$为下个level的ending position。遍历当前level中的each position,更新下个level的ending position。一旦当前位置到达$\alpha_0$,即表示进入下一个level了,那么$\alpha_0$更新为$\alpha_1$

\setcounter{lstlisting}{0}
\begin{lstlisting}[style=customc, caption={Graphic Approach}]
bool canJump( vector<int>& nums )
{
    //current level end position
    int cur_end = 0;
    //next level end position
    int next_end = 0;

    int p = 0;

    int L = static_cast<int>( nums.size() );

    while( p <= cur_end )
    {
        //update next level end position
        next_end = ( max )( next_end, p + nums[p] );

        if( next_end >= L - 1 )
        {
            return true;
        }

        if( p == cur_end )
        {
            cur_end = next_end;
        }

        ++p;
    }

    return false;
}
\end{lstlisting}

\subsection{Dynamic Programming Top-down}
Top-down Dynamic Programming can be thought of as optimized backtracking. It relies on the observation that once we determine that a certain index is good / bad, this result will never change. This means that we can store the result and not need to recompute it every time.

Therefore, for each position in the array, we store whether the index is good or bad into an array, say $M$ and let its values be either one of: \texttt{GOOD}, \texttt{BAD} or \texttt{UNKNOWN}.

Then, the steps of the algorithm are 
\begin{enumerate}
\item Initially, all elements of the memo table are \texttt{UNKNOWN}, except for the last one, which is \texttt{GOOD} (it can reach itself)
\item First checks if a index $i$ is known (\texttt{GOOD} / \texttt{BAD})
\begin{itemize}
\item If it is known then return the stored result.
\item Otherwise perform the DFS and backtracking steps.
\end{itemize}
\item Once we determine the value of the current index, we store it in the $M$
\end{enumerate}

However, this approach will cause \textbf{TLE}.

\begin{lstlisting}[style=customc, caption={Top Down Dynamic Programming}]
//This will cause TLE
//only for reference
bool canJump( vector<int>& nums )
{

    if( nums.empty() || ( nums.size() == 1 ) )
    {
        return true;
    }

    vector<unsigned char> memo( nums.size(), 0 );

    memo.back() = 2; //Good

    return dfs( nums, 0, memo );
}
//recursive helper function
bool dfs( vector<int>& A, size_t start, vector<unsigned char>& memo )
{
    if( memo[start] > 0 )
    {
        return memo[start] == 2;
    }

    //next_jump is the maximum index from start can reach.
    auto next_jump = ( min )( A.size() - 1, start + A[start] );
    for( size_t i = start + 1; i <= next_jump; ++i )
    {
        if( dfs( A, i, memo ) )
        {
            memo[start] = 2;
            return true;
        }
    }

    memo[start] = 1;
    return false;
}
\end{lstlisting}

\subsection{Bottom Up Dynamic Programming}
We know that we only ever jump to the right. This means that if we start from the \textbf{right} of the array, every time we will query a position to our right, that position has already be determined as being \texttt{GOOD} or \texttt{BAD}. This means we don't need to recurse anymore, as we will always hit the memo table.

\setcounter{lstlisting}{0}
\begin{lstlisting}[style=customc, caption={Bottom Up Dynamic Programming}]
bool canJump( vector<int>& nums )
{
    vector<unsigned char> memo( nums.size(), 0 );

    memo.back() = 2; //Good

    int L = static_cast<int>( nums.size() );

    //process from right
    for( int i = L - 2; i >= 0; --i )
    {
        //the maximum index that from index i can reach
        int end_pos = ( min )( L - 1, i + nums[i] );

        for( int j = i + 1; j <= end_pos; ++j )
        {
            if( memo[j] == 2 )
            {
                //can jump
                memo[i] = 2;
                break;
            }
        }
    }

    return memo[0] == 2;
}
\end{lstlisting}