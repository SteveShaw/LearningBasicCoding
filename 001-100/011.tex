\section{11 --- Container With Most Water}
\definecolor{lightblue}{RGB}{153, 205,255}
Given $n$ non-negative integers $A =[a_1, a_2, \ldots, a_n]$, where each represents a point at coordinate $(i, a_i)$. $n$ vertical lines are drawn such that the two endpoints of line $i$ is at $(i, a_i)$ and $(i, 0)$. Find two lines, which together with $x$-axis forms a container, such that the container contains the most water.
\par
\textbf{Note}: You may not slant the container and $n$ is at least 2.
\paragraph{Example:}
\begin{flushleft}
	\textbf{Input}: $[1,8,6,2,5,4,8,3,7]$
\begin{figure}[H]
\begin{tikzpicture}
\draw [decoration={markings,mark=at position 1 with
	{\arrow[scale=2,>=stealth]{>}}},postaction={decorate}] (0,0) -- (15cm,0);
\draw [decoration={markings,mark=at position 1 with
	{\arrow[scale=2,>=stealth]{>}}},postaction={decorate}] (0,0) -- (0, 9cm);
\foreach \y in {1,...,8}
{
	\draw (0, \y) -- (0.3, \y);
	\node at (0,\y)[anchor=east] {\y};
}
\node at (0,0)[anchor=east] {0};
\draw[fill=blue] (1.44, 0) rectangle (1.56, 1);
\draw[fill=red] (2.94, 0) rectangle (3.06, 8);
\draw[fill=blue] (4.44, 0) rectangle (4.56, 6);
\draw[fill=blue] (5.94, 0) rectangle (6.06, 2);
\draw[fill=blue] (7.44, 0) rectangle (7.56, 5);
\draw[fill=blue] (8.94, 0) rectangle (9.06, 4);
\draw[fill=blue] (10.44, 0) rectangle (10.56, 8);
\draw[fill=blue] (11.94, 0) rectangle (12.06, 3);
\draw[fill=red] (13.44, 0) rectangle (13.56, 7);
\draw[fill=lightblue, fill opacity=0.4] (3.06,0) rectangle(13.44, 7);
\end{tikzpicture}
\end{figure}
\textbf{Output}: 49
\end{flushleft}
\subsection{Left And Right Pointers}
\begin{itemize}
	\item 竖线之间围起来的区域面积是由较短的那条竖线决定的。如果这条线离左边的线越远,围起来的面积也就越大。
	\item 用$x$和$y$分别指向给定高度数组的开始和末尾。同时maintain一个变量$z$表示到目前位置获得的最大面积。
	\item 接下来进行循环,只要$A[x]$小于$A[y]$,那么计算出当前$x$和$y$围起来的面积,更新$z$。然后根据$x$和$y$的大小将较小的那个pointer向较大的那个pointer移动。
	\item 如上所属,由于区域面积是由较低的那条竖线决定,因此将较小的pointer向较大的pointer方向移动,虽然宽度减小,但是有可能获得更大的高度。
\end{itemize}
\setcounter{algorithm}{0}
\begin{algorithm}[H]
\caption{Get maximum contained water}
\begin{algorithmic}[1]
\Procedure{MaxArea}{$H, L$}
\State $z:=0$ \Comment $z$ is the maximum area until now
\State $\ast$ 初始化左右两个pointer $x$和$y$
\State $x:=0$ \textbf{and} $y:=L - 1$
\While{$x < y$}
\State $z_0 := \min(H[x], H[y])\times(y-x)$ \Comment Get current area  between $x$ and $y$
\State $z \gets \max(z, z_0)$ \Comment Update maximum area so far.
\If{$H[x] < H[y]$}
\State $x \gets x+1$
\Else
\State $y \gets y-1$
\EndIf
\EndWhile
\State \Return $z$
\EndProcedure
\end{algorithmic}
\end{algorithm}
\setcounter{lstlisting}{0}
\begin{lstlisting}[style=customc, caption={Two Pointers}]
int maxArea( vector<int>& height )
{
    //global maximum area
    int z = 0;

    //left pointer
    int x = 0;

    //right pointer
    int y = static_cast<int>( height.size() - 1 );

    while( x < y )
    {
        int h = ( min )( height[x], height[y] );
        int z0 = h * ( y - x );
        z = ( max )( z, z0 );

        if( height[x] < height[y] )
        {
            //move shorter to longer
            ++x;
        }
        else
        {
            //move shorter to longer
            --y;
        }
    }

    return z;
}
\end{lstlisting}