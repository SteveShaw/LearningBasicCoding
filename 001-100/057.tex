\section{57 --- Insert Interval}
Given a set of non-overlapping intervals $I$, insert a new interval $\mu$ into the intervals (merge if necessary).
\par
You may assume that the intervals were initially sorted according to their start times.
\paragraph{Example 1:}
\begin{flushleft}
\textbf{Input}:
\par
$ I = [[1,3],[6,9]]$, $\mu = [2,5]$
\par
\begin{tikzpicture}
\node(0) {};
\node(1)[draw, rectangle, minimum size=8mm, below = 3mm of 0] {$[1,3]$};
\node(2)[draw, rectangle, minimum size=8mm, right= 6mm of 1.east, anchor=west] {$[6, 9]$};
\end{tikzpicture}
\par
\textbf{Output}:
\par
$[[1,5],[6,9]]$
\end{flushleft}
\paragraph{Example 2:}
\begin{flushleft}
\textbf{Input}:
\par
$I= [[1,2],[3,5],[6,7],[8,10],[12,16]]$, $\mu= [4,8]$
\par
\textbf{Output}: $[[1,2],[3,10],[12,16]]$
\par
\textbf{Explanation}:
\par
Because the new interval $[4,8]$ overlaps with $[3,5],[6,7],[8,10]$.
\end{flushleft} 

\subsection{Binary Search}
由于题目给出的interval都是按照起始位置按大小排序的,因为这些inteval都是不overlapp的,所以这些interval的end也是顺序排序的。所以自然而然就会用binary search。Denote $I$ 的长度为 $N$。
\par
首先找到一个interval $I[s]$,其起点$\alpha$是第一个大于或者等于$\mu$的起点 $\mu_{\alpha}$,即$I[s]_{\alpha} \geq \mu_{\alpha}$。如果$s=0$,意味着$I$中所有的interval的起点都不小于$\mu_{\alpha}$,很显然merge之后的interval的起点$\alpha$也就是$\mu_{\alpha}$。如果$s\neq 0$,那么我们需要比较$\mu_{\alpha}$与$I[s-1]$的end也就是$I[s-1]_{\beta}$。
\begin{itemize}
    \item If $\mu_{\alpha} \leq I[s-1]_{\beta}$,那么$I[s-1]$可以被merge,merge之后的interval的起始点$\alpha = I[s-1]_{\alpha}$。这种情况可以从下面的示图分析得出
    \begin{figure}[H]
        \centering
            \begin{tikzpicture}
            \node(0) {};
            \node (1) at (5, 0)  [draw, minimum width=4cm, minimum height=0.7cm, label = $\mu$, label=right:$\beta$, label=left:$\alpha$, anchor = south west] {};
            \node(2) at (3, 1.3cm) [draw, minimum width=3cm, minimum height=0.7cm, label = $I(s-1)$,label=right:$\beta$, label=left:$\alpha$, anchor = south west] {};
            \node(3) at (8, 1.3cm) [draw, minimum width=3cm, minimum height=0.7cm, label = $I(s)$, label=right:$\beta$, label=left:$\alpha$, anchor = south west] {};
            \end{tikzpicture}
        \caption{Previous Interval $I[s-1]$ is overlapped with $\mu$}
    \end{figure}
    \item If $\mu_{\alpha} > I[s-1]_{\beta}$,那么$I[s-1]$,merge之后的interval的起始点$\alpha$仍然是 $\mu_{\alpha}$。
        \begin{figure}[H]
        \centering
            \begin{tikzpicture}
            \node(0) {};
            \node (1) at (5, 0)  [draw, minimum width=4cm, minimum height=0.7cm, label = $\mu$, label=right:$\beta$, label=left:$\alpha$, anchor = south west] {};
            \node(2) at (1, 1.3cm) [draw, minimum width=3cm, minimum height=0.7cm, label = $I(s-1)$,label=right:$\beta$, label=left:$\alpha$, anchor = south west] {};
            \node(3) at (8, 1.3cm) [draw, minimum width=3cm, minimum height=0.7cm, label = $I(s)$, label=right:$\beta$, label=left:$\alpha$, anchor = south west] {};
            \end{tikzpicture}
        \caption{Previous Interval $I[s-1]$ Is Not Overlap With $\mu$}
    \end{figure}
\end{itemize}
接下来寻找$I[e]$,其end $I[e]_{\beta}$是第一个大于或者等于$\mu$的end $\mu_{\beta}$。如果$e = N$,就是说$I$中所有的interval的end都不大于$\mu_{\beta}$,显然merge之后interval的end就是$\mu_{\beta}$。如果$e < N$,则需要比较$I[e]$的起点即$I[e]_{\alpha}$与$\mu_{\beta}$的关系。
\begin{itemize}
    \item If $I[e]_{\alpha} \leq \mu_{\beta}$,那么$I[e]$就要被merge,merge之后的end 即为$I[e]_{\beta}$。
           \begin{figure}[H]
        \centering
            \begin{tikzpicture}
            \node(0) {};
            \node (1) at (3, 0)  [draw, minimum width=4cm, minimum height=0.7cm, label = $\mu$, label=right:$\beta$, label=left:$\alpha$, anchor = south west] {};
            \node(3) at (6, 1.3cm) [draw, minimum width=3cm, minimum height=0.7cm, label = $I(e)$, label=right:$\beta$, label=left:$\alpha$, anchor = south west] {};
            \end{tikzpicture}
        \caption{The Interval $I[e]$ Is Overlap With $\mu$}
    \end{figure}
    \item If $I[e]_{\alpha} > \mu_{\beta}$,两个interval没有overlap,$I[e]$不会被merge,这时merge之后的interval的end即为$\mu_{\beta}$
        \begin{figure}[H]
        \centering
            \begin{tikzpicture}
            \node(0) {};
            \node (1) at (3, 0)  [draw, minimum width=4cm, minimum height=0.7cm, label = $\mu$, label=right:$\beta$, label=left:$\alpha$, anchor = south west] {};
            \node(3) at (8, 1.3cm) [draw, minimum width=3cm, minimum height=0.7cm, label = $I(e)$, label=right:$\beta$, label=left:$\alpha$, anchor = south west] {};
            \end{tikzpicture}
        \caption{The Interval $I[e]$ Is Not Overlap With $\mu$}
    \end{figure}
\end{itemize}
这样$I[s\ldots e]$就是要被merge的intervals,不过有几个边界情况需要考虑:
\begin{itemize}
    \item $s = N$,这意味着新的interval $\mu$位于$I$中最后一个interval的右边,所以最后的结果就是将$\mu$放在$I$的末尾,i.e. $I \gets I + \mu$。
    \item $e = 0$,这表示$\mu$位于$I$中第一个interval的左边,所以最后的结果就是将$\mu$插入到$I$的第一个位置,i.e. $I \gets \mu + I$
    \item $s = e$,这表示$\mu$没有和$I$中任何一个interval overlap,但是由于$I[s-1]$位于$\mu$的左侧,$I[s]$肯定不会在$\mu$的左侧,所以需要把$\mu$插入到$I[s-1]$和$I[s]$中间,i.e. $I \gets I[0\ldots s-1] + \mu + I[s\ldots N-1]$
\end{itemize}

\setcounter{lstlisting}{0}
\begin{lstlisting}[style=customc, caption={Binary Search}]
vector<vector<int>> insert( vector<vector<int>>& intervals, vector<int>& newInterval )
{

    //find the first interval whose start is no less than newInterval's start

    int L = static_cast<int>( intervals.size() );

    int s = bs( intervals, newInterval[0], 0 );

    int merge_l = newInterval[0];

    if( s > 0 )
    {
        //compare end of intervals[s-1]
        //against start of newInterval
        if( intervals[s - 1][1] >= newInterval[0] )
        {
            //newInterval will cover intervals[s-1]
            //update the merged interval's left boundary
            merge_l = intervals[s - 1][0];

            s = s - 1; //the start index of intervals to be merged
        }
    }

    int merge_r = newInterval[1];

    //find the first interval whose end is no less than newInterval's end
    int e = bs( intervals, newInterval[1], 1 );

    if( e < L )
    {
        //compare start of intervals[e]
        //against end of newInterval
        if( intervals[e][0] <= newInterval[1] )
        {
            //intervals[e] can be merged with newInterval
            //update the merged interval's right boundary
            merge_r = intervals[e][1];
            e = e + 1;
        }
    }


    vector<vector<int>> ans;

    //remove intervals in range [s...e]
    for( int i = 0; i < s; ++i )
    {
        ans.emplace_back( intervals[i].begin(), intervals[i].end() );
    }

    ans.push_back( vector<int> {merge_l, merge_r} );

    for( int i = e; i < L; ++i )
    {
        ans.emplace_back( intervals[i].begin(), intervals[i].end() );
    }

    return ans;
}

//leftmost binary search
int bs( vector<vector<int>>& A, int T, int index )
{
    int l = 0;

    int r = static_cast<int>( A.size() );


    while( l < r )
    {
        int mid = ( l + r ) / 2;

        if( A[mid][index] < T )
        {
            l = mid + 1;
        }
        else
        {
            r = mid;
        }
    }

    return l;
}
\end{lstlisting}