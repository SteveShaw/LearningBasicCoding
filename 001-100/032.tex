\section{32 --- Longest Valid Parentheses}
Given a string $s$ containing just the characters left and right parenthesis, find the length of the longest valid (well-formed) parentheses substring.

\paragraph{Example 1:}

\begin{flushleft}
\textbf{Input}: \texttt{(()}

\textbf{Output}: 2

\textbf{Explanation}: The longest valid parentheses substring is \texttt{()}
\end{flushleft}

\paragraph{Example 2:}
\begin{flushleft}

\textbf{Input}: \texttt{)()())}

\textbf{Output}: 4

\textbf{Explanation}: The longest valid parentheses substring is \texttt{()()}
\end{flushleft}

\subsection{Dynamic Programming}

\begin{itemize}
\item Denote $F(i)$ as the length of the longest valid substring ending at index $i$.
\item The valid substring must end with \texttt{)}.
\item Thus any substring ending with \texttt{)} at index $i$ will have $F(i)=0$ because it is not a valid substring. So, $F(i)$ is updated only when \texttt{)} is encountered.
\end{itemize}

To fill $F$, we will check every two consecutive characters of the string and if

\begin{itemize}
\item When $s[i-1]=\texttt{(}$, $s[i]=\texttt{)}$, obviously we have $F(i) = F(i-2)+2$. 
\item When $s[i] = \text{)}$ and $s[i-1] = \text{)}$, we need to check the letter $c$ at index $i-F(i-1)-1$. If it is \texttt{(}, $F(i)\gets F(i-1) +  F(i-F(i-1)-2)+2$.
\begin{enumerate}
\item if $s[i-1]$ was a part of a valid substring $x$ and the substring $y$ ending at $i$ is also valid, there must be a left parentheses lies before $x$, i.e. $s$ looks like  $\ldots(x)$. Therefore $F(i)$ will include the length of $x$ which is $F(i-1)$ plus 2.
\item Also, at the index $k = i - F(i-1) - 2$ which is before the left parenthesis lies before $x$, i.e. $\ldots s[k](x)$, there is also a valid substring ending at index $k$. This substring length is thus $F(i - F(i-1)-2)$.
\end{enumerate}
\end{itemize}

\setcounter{algorithm}{0}
\begin{algorithm}[H]
\caption{Dynamic Programming}
\begin{algorithmic}[1]
\Procedure{LongestValidParentheses}{$S, L$}
\State $\star$ Create an array $F$ with size $L$ and intialized all elements to zero
\State $\ell:=0$ \Comment The result of maximum length
\For{$i:=1$ \textbf{to} $L-1$}
\If{$S[i]$ is \texttt{)}}
\If{$S[i-1]$ is \texttt{(}} \Comment Condition 1
\If{$i \ge 2$}
\State $F[i] \gets F[i-2]+2 $
\Else
\State $F[i]\gets 2$
\EndIf
\Else 
\State $p := i - F[i-1] - 1$
\If{$p \ge 0$ \textbf{and} $S[p]$ is \texttt{(}} \Comment Condition 2
\State $F[i] \gets F[i-1]+2 $
\State $k := p-1$
\If{$k\ge 0$} \Comment Add the length of valid substring ending at $k$
\State $F[i] \gets F[i] + F[k]$
\EndIf
\EndIf
\EndIf
\EndIf
\State $\ell \gets \max\left(F[i], \ell\right)$
\EndFor
\State \Return $\ell$
\EndProcedure
\end{algorithmic}
\end{algorithm}

\setcounter{lstlisting}{0}
\begin{lstlisting}[style=customc, caption={Dynamic Programming}]
int longestValidParentheses( string s )
{

    //DP array
    vector<int> F( s.size(), 0 );

    int L = static_cast<int>( s.size() );

    int ans = 0;

    for( int i = 1; i < L ; ++i )
    {
        if( s[i] == ')' )
        {
            if( s[i] == '(' )
            {
                //condition 1
                //...()
                F[i] = i >= 2 ? ( F[i - 2] + 2 ) : 2;
            }
            else
            {
                int p = i - F[i - 1] - 1;

                if( ( p >= 0 ) && ( s[p] == '(' ) )
                {
                    //condition 2
                    //....(x)
                    F[i] = F[i - 1] + 2;

                    if( p >= 1 )
                    {
                        //...s[p-1](x)
                        F[i] += F[p - 1];
                    }
                }
            }
        }

        //get the maximum length so far
        ans = ( max )( ans, F[i] );
    }

    return ans;
}
\end{lstlisting}
%
\subsection{Stack}

\begin{itemize}
\item For every left parenthesis encountered, we push its index onto the stack $\Phi$.
\item For every right parenthesis encountered, 
\begin{enumerate}
\item pop the top element $t_0$ from the stack.
\item If $\Phi$ is empty, push current index $i$ into the stack.
\item otherwise, $i-t$ gives the length of the current valid string of parentheses ($t$ is the current top of the stack). The reason is that $t$ must be on the left of $t_0$, i.e. $t = t_0-1$. Otherwise, it means there are some other left parenthesis between $t$ and $t_0$, but these left parenthesis are popped by some right parenthesis, this is \textbf{impossible}. Therefore, the length of current valid substring should be $i - t_0 + 1 = i-t$
\end{enumerate}
\item At beginning, push $-1$ onto the stack. It will bring two benefits: 
\begin{enumerate}
\item It will let the computation of valid substring length to be consistent, i.e. the length always be the current index minus the top element of stack.
\item If the number of right parenthesis is larger than the number of left parenthesis, the pop of stack can always happen before checking if the stack is empty.
\end{enumerate}
\end{itemize}

\begin{algorithm}[H]
\caption{Stack Based Algorithm}
\begin{algorithmic}[1]
\Procedure{LongestValidParentheses}{$S, L$}
\State $\star$ Initialize an empty stack $\Phi$
\State $\star$ Push $-1$ to $\Phi$
\State $\ell:=0$ \Comment The maximum length of valid substring
\For{$i:=0$ \textbf{to} $L-1$}
\If{$S[i]$ is left parenthesis}
\State $\star$ Push $i$ into $\Phi$
\Else
\State $\star$ Pop from $\Phi$
\If{$\Phi$ is empty}
\State $\star$ Push $i$ into $\Phi$
\Else
\State $x := i - \Phi(0)$ \Comment The length of valid substring is the $i$ minus the top of $\Phi$
\State $\ell\gets \max(\ell, x)$ \Comment Update the maximum length so far
\EndIf
\EndIf
\EndFor
\State \Return $\ell$
\EndProcedure
\end{algorithmic}
\end{algorithm}

\begin{lstlisting}[style=customc, caption={Stack}]
int longestValidParentheses( string s )
{
    stack<int> stk;
    //add -1 to unify the
    //length calculation
    stk.push( -1 );

    int L = static_cast<int>( s.size() );
    int ans = 0;
    for( int i = 0; i < L; ++i )
    {
        if( s[i] == '(' )
        {
            //add index of left parenthesis to
            //the stack
            stk.push( i );
        }
        else
        {
            stk.pop();
            if( stk.empty() )
            {
                //add current index to
                //stack
                stk.push( i );
            }
            else
            {
                //we found a valid substring
                //with length i - stk.top()
                ans = ( max )( ans, i - stk.top() );
            }
        }
    }

    return ans;
}
\end{lstlisting}


\subsection{Two Counters}
使用两个counter $x$和$y$分别记录左括号和右括号的个数。
\begin{enumerate}
\item 从左到右遍历$s$。当前是左括号时,increments $x$。否则,increments $y$。
\item 如果$x=y$,那么找到了一个valid substring,记录其长度,并更新输出的最大长度。如果$x>y$,则将$x$和$y$都reset为零。
\item 接着, 从右到左遍历$s$。$x$和$y$也是和上述方法一样更新,不过当$x<y$时,则将两个都reset为零。
\end{enumerate}

\begin{algorithm}[H]
\caption{Two Counters}
\begin{algorithmic}[1]
\Procedure{LongestValidParentheses}{$S, L$}
\State $\ell := 0$
\State $\ast$ Initialize two counters $x :=0$ and $y := 0$ 
\State $\ast$ Scanning from left to right
\For{$i:=0$ \textbf{to} $L-1$}
\If{$S[i]$ is left parenthesis}
\State $x \gets x+1$
\Else
\State $y \gets y+1$
\EndIf
\If{$x<y$}
\State $\ast$ Invalid substring, reset both counters
\State $x \gets 0$
\State $y\gets 0$
\ElsIf{$x = y$} 
\State $\ast$ Found a valid substring, the length is $x\times 2$
\State $\ell \gets \max(\ell, 2x)$
\EndIf
\EndFor
\State $\ast$ Scanning from right to left
\State $x\gets 0$
\State $y\gets 0$
\For{$i:=L-1$ \textbf{to} $0$}
\If{$S[i]$ is left parenthesis}
\State $x \gets x+1$
\Else
\State $y \gets y+1$
\EndIf
\algstore{032algo}
\end{algorithmic}
\end{algorithm}
\begin{algorithm}[H]
\begin{algorithmic}[1]
\algrestore{032algo}
\If{$x > y$}
\State $\ast$ Invalid substring, reset both counters
\State $x \gets 0$
\State $y\gets 0$
\ElsIf{$x = y$}
\State $\ast$ Found a valid substring, the length is $x\times 2$
\State $\ell \gets \max(\ell, 2x)$
\EndIf
\EndFor
\State \Return $\ell$
\EndProcedure
\end{algorithmic}
\end{algorithm}

\begin{lstlisting}[style=customc, caption={Double Counters}]
int longestValidParentheses( string s )
{

    //number of left p when scanning from left to right
    int x1 = 0;
    //number of right p when scanning from left to right
    int y1 = 0;

    //number of left p when scanning from right to left
    int x2 = 0;
    //number of right p when scanning from right to left
    int y2 = 0;

    int ans = 0;

    for( size_t i = 0; i < s.size(); ++i )
    {
        if( s[i] == '(' )
        {
            ++x1;
        }
        else
        {
            ++y1;
        }

        if( x1 < y1 )
        {
            //scanning from left to right
            //x1 < y1 means we cannot find enough
            //left p to match right p
            x1 = 0;
            y1 = 0;
        }
        else if( x1 == y1 )
        {
            ans = ( max )( ans, 2 * x1 );
        }

        //get information from right to left
        auto j = s.size() - i - 1;

        if( s[j] == '(' )
        {
            ++x2;
        }
        else
        {
            ++y2;
        }


        if( x2 > y2 )
        {
            //scanning from right to left
            //y2 < x2 means we cannot find enough
            //right p to match left p
            x2 = 0;
            y2 = 0;
        }
        else if( x2 == y2 )
        {
            ans = ( max )( ans, 2 * x2 );
        }

    }

    return ans;
}

\end{lstlisting}
