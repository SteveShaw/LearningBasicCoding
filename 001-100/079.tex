\section{79 --- Word Search}
Given a 2D \fcj{board} and a \fcj{word}, find if the word exists in the grid.

The word can be constructed from letters of sequentially adjacent cell, where \fcj{"adjacent"} cells are those horizontally or vertically neighboring. The same letter cell may not be used more than once.

\paragraph{Example:}
\begin{flushleft}


\fcj{board}: =

\begin{lstlisting}[style=customc]
{
  {'A','B','C','E'},
  {'S','F','C','S'},
  {'A','D','E','E'}
}
\end{lstlisting}

Given \fcj{word = "ABCCED"}, return \fcj{true}.

Given \fcj{word = "SEE"}, return \fcj{true}.

Given \fcj{word = "ABCB"}, return \fcj{false}.

\end{flushleft}

\subsection{Backtracking}
This is still the backtracking algorithm. For each cell match the first letter of the given word, we start the recursive matching of remaining letters.

If a cell matches against current letter, we first change this cell to a unique value to indicate it is used, then recursively matching the remaining letters. If match is failed, we reset the cell to its previous letter.

\setcounter{lstlisting}{0}
\begin{lstlisting}[style=customc, caption={Backtracking}]
bool exist( vector<vector<char>>& board, string word )
{
    auto M = static_cast< int >( board.size() );
    auto N = static_cast< int >( board[0].size() );
    for( int r = 0; r < M; ++r )
    {
        for( int c = 0; c < N; ++c )
        {
            if( board[r][c] == word[0] )
            {
                auto b = board[r][c];
                //change to 0 to indicate
                //board[r][c] is visited
                board[r][c] = '\0';
                if( dfs( board, r, c, word, 1 ) )
                {
                    return true;
                }
                //backtracking
                board[r][c] = b;
            }
        }
    }
    return false;
}
//backtracking helper function
bool dfs( vector<vector<char>>& B, int r, int c, string_view word_view, size_t start )
{
    if( start == word_view.size() )
    {
        return true;
    }
    auto M = B.size();
    auto N = B[0].size();
    //four adjacent cells offsets
    int dr[] = { -1, 0, 1, 0 };
    int dc[] = { 0, -1, 0, 1 };
    for( int i = 0; i < 4; ++i )
    {
        int nr = r + dr[i];
        int nc = c + dc[i];
        if( ( nr >= 0 ) && ( nr < M ) && ( nc >= 0 ) && ( nc < N ) )
        {
            if( B[nr][nc] == word_view[start] )
            {
                auto b = B[nr][nc];
                //change to 0 to indicate
                //board[nr][nc] is visited
                B[nr][nc] = '\0';
                if( dfs( B, nr, nc, word_view, start + 1 ) )
                {
                    return true;
                }
                //backtracking
                B[nr][nc] = b;
            }
        }
    }
    return false;
}
\end{lstlisting} 