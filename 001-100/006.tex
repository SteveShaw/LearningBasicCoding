\section{6 --- ZigZag Conversion}
The string \texttt{PAYPALISHIRING} is written in a zigzag pattern on a given number of rows like this: (you may want to display this pattern in a fixed font for better legibility)
\begin{table}[H]
	\begin{tabular}{lllllll}
P & &  A & & H & &  N\\
A  & P  & L  & S  & I  & I &  G\\
Y &  I & R &  &  &  &
	\end{tabular}
\end{table}
And then read line by line: \texttt{PAHNAPLSIIGYIR}
\par
Write the code that will take a string and make this conversion given a number of rows:
\begin{lstlisting}[style=customc]
string convert(string s, int numRows);
\end{lstlisting}

\paragraph{Example 1:}

\begin{flushleft}
	\textbf{Input}: $S$ = \texttt{PAYPALISHIRING}, $N = 3$
	\\
\textbf{Output}: \texttt{PAHNAPLSIIGYIR}
\\
\end{flushleft}


\paragraph{Example 2:}

\begin{flushleft}
	\textbf{Input}: $S$ = \texttt{PAYPALISHIRING}, $N = 4$
	\\
\textbf{Output}: \texttt{PINALSIGYAHRPI}
\\
\textbf{Explanation}:
\begin{table}[H]
	\begin{tabular}{lllllll}
		P & &   &  I &  & &  N\\
		A  &  & L  & S   &  & I &  G\\
		Y &  A &  & H & R &  &  \\
		P &  &  & I &  &  &
	\end{tabular}
\end{table}
\end{flushleft}
\subsection{Visit By Row}
Visit all characters in row 0 first, then row 1, then row 2, and so on $\dots$. Supper number of rows is $N$
\begin{itemize}
	\item Characters in row $0$ are located at indexes $0, \left(2 \times N - 2\right), 2\times\left(2 \times N - 2\right), \ldots$
	\item Characters in row $N-1$ are located at indexes 
	\[N-1, \left(2 \times N - 2\right) + N - 1, 2\times\left(2 \times N - 2\right) + N - 1, \ldots\]
	\item Characters in inner row $0<i<N-1$ are located at indexes 
	\[
	i, \left(2 \times N - 2\right)-i, \left(2 \times N - 2\right) + i, 2\times\left(2 \times N - 2\right)- i, 2\times\left(2 \times N - 2\right)+ i, 3\times\left(2 \times N - 2\right)- i, \dots
	\]
\end{itemize}
\setcounter{algorithm}{0}
\begin{algorithm}[H]
	\caption{Print letters in Zigzag way}
	\begin{algorithmic}[1]
		\Procedure{Zigzag}{$S, L, N$}
		\If{$N = 1$}
		\State \Return $S$ \Comment one row is $S$ itself
		\EndIf
		\State $\delta := 2 \times N -2$
		\State $Z:= \emptyset$ \Comment $Z$ will be the print string
		\For{$n:=0$ \textbf{to} $N-1$}
		\State $k:= 0$
		\While{$k+n < L$}
		\State $Z \gets Z + S[k+n]$ \Comment $S[m\times \left(2 \times N - 2\right)] + i$ for $m = 1,2, \ldots$ and $i = 0, \ldots, N-1$
		\If{$n\neq 0$ \textbf{and} $n\neq N-1$ \textbf{and} $k+\delta - n < L$} \Comment For inner rows
		\State $Z \gets Z + S[k+\delta-n]$ \Comment get $S[m\times\left(2 \times N - 2\right)- i]$
		\EndIf
		\State $k \gets k+\delta$
		\EndWhile
		\EndFor
		\State \Return $Z$
		\EndProcedure
		\Statex
	\end{algorithmic}
\end{algorithm}
\subsection{Sort by Row}
By iterating through the string from left to right, we can easily determine which row in the Zig-Zag pattern that a character belongs to.
\par
We can use $\min(N,L)$  lists to represent the non-empty rows of the Zig-Zag Pattern.

\begin{itemize}
	\item Iterate through $S$ from left to right, appending each character to the appropriate row. 
	\item The appropriate row can be tracked using two variables: the current row $r$ and the current direction $d$.
	\item The current direction $d$ changes only when moved up to the top-most row or moved down to the bottom-most row.
\end{itemize}

\setcounter{lstlisting}{0}
\begin{lstlisting}[style=customc, caption={Sort By Row}]
string convert( string s, int numRows )
{
    if( numRows == 1 )
    {
        return s;
    }

    //Determine the rows for print
    int l = static_cast<int>( s.size() );
    l = ( min )( l, numRows );

    vector<string> v_rows( l );
    int curRow = 0;

    bool goingDown = false;

    for( char c : s )
    {
        v_rows[curRow].push_back( c );

        if( curRow == 0 || curRow == numRows - 1 )
        {
            //Change the direction
            goingDown = !goingDown;
        }

        //up: -1
        //down: 1
        curRow += goingDown ? 1 : -1;
    }

    string ans;
    for( const string &row : v_rows )
    {
        ans += row;
    }

    return ans;
}
\end{lstlisting}
