\section{7 --- Reverse Integer}
Given a 32-bit signed integer $x$, reverse digits of an integer.

\paragraph{Example 1:}

\begin{flushleft}
	\textbf{Input}: 123
	\\
\textbf{Output}: 321
\end{flushleft}

\paragraph{Example 2:}

\begin{flushleft}
	\textbf{Input}: $-123$
	\\
\textbf{Output}: $-321$
\end{flushleft}

\paragraph{Example 3:}

\begin{flushleft}
	\textbf{Input}: 120
	\\
\textbf{Output}: 21
\end{flushleft}

\paragraph{Note:}
\begin{itemize}
	\item Assume we are dealing with an environment which could only store integers within the 32-bit signed integer range: $[−2^{31},  2^{31} − 1]$. 
	\item For the purpose of this problem, assume that your function returns 0 when the reversed integer overflows.
\end{itemize}

\subsection{Pop and Push Digits And Check before Overflow}
We can build up the reverse integer one digit at a time. While doing so, we can check beforehand whether or not appending another digit would cause overflow.
\begin{itemize}
	\item We want to repeatedly \textbf{pop} the last digit off of $x$ and \textbf{push} it to the back of the result number $y$. In the end, $y$ will be the reverse of the $x$.
	\item To \textbf{pop} and \textbf{push} digits without the help of some auxiliary stack/array
	\begin{itemize}
		\item \textbf{pop}: 
		
		\begin{align*}
			z &= x \bmod 10 \\
			x &= \frac{x}{10}
		\end{align*}
		
		\item \textbf{push}:
		\begin{align*}
			t &= y \times 10 + z \\
			y &= t
		\end{align*}
	\end{itemize}
	\item However, $t$ could be overflow, Suppose $I$ is the maximum value represented by integer type.
	\begin{enumerate}
		\item if $t = y \times 10 + z$ has overflow, it must be $y \geq \dfrac{I}{10}$
		\item if $y > \dfrac{I}{10} $, $t$ will be overflow for sure.
		\item if $y = \dfrac{I}{10}$, $t$ will be overflow $\iff z > 7$, because $I = 2^{31} - 1 = 2147483647$
	\end{enumerate}
	\item similar when $t$ is underflow, and the minimum value represented by integer type is $J = -2^{31} = -2147483648$
\end{itemize}

\setcounter{algorithm}{0}
\begin{algorithm}[H]
	\caption{Reverse an integer}
	\begin{algorithmic}[1]
		\Procedure{ReverseInteger}{$x$}
		\State $y:=0$ \Comment $y$ is the result of reverse
		\State $y_0:= I/10$ \Comment threshold for $y$ to avoid overflow
		\State $y_1:= J/10$ \Comment threshold for $y$ to avoid underflow
		\While{$x \neq 0$}
		\State $z := x \bmod 10$
		\State $x = x/10$
		\If{$y > y_0$ \textbf{or} ($y=y_1$ \textbf{and} $z > 7$)} \Comment overflow
		\State \Return 0
		\ElsIf{$y < y_1$ \textbf{or} ($y=y_1$ \textbf{and} $z < -8$)} \Comment underflow
		\State \Return 0
		\Else
		\State $y\gets y\times 10 + z$
		\EndIf
		\EndWhile
		\State \Return $y$
		\EndProcedure
	\end{algorithmic}
\end{algorithm}