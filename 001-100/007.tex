\section{7 --- Reverse Integer}
Given a 32-bit signed integer $x$, reverse digits of an integer.

\paragraph{Example 1:}

\begin{flushleft}
	\textbf{Input}: 123
	\\
\textbf{Output}: 321
\end{flushleft}

\paragraph{Example 2:}

\begin{flushleft}
	\textbf{Input}: $-123$
	\\
\textbf{Output}: $-321$
\end{flushleft}

\paragraph{Example 3:}

\begin{flushleft}
	\textbf{Input}: 120
	\\
\textbf{Output}: 21
\end{flushleft}

\paragraph{Note:}
\begin{itemize}
	\item Assume we are dealing with an environment which could only store integers within the 32-bit signed integer range: $[−2^{31},  2^{31} − 1]$. 
	\item For the purpose of this problem, assume that your function returns 0 when the reversed integer overflows.
\end{itemize}

\subsection{Pop and Push Digits And Check before Overflow}
We can build up the reverse integer one digit at a time. While doing so, we can check beforehand whether or not appending another digit would cause overflow.


\setcounter{lstlisting}{0}
\begin{lstlisting}[style=customc, caption={Check before over/underflow}]
int reverse( int x )
{
    int n = x;
    int res = 0;
    //helper function to check if
    //k is larger than INT_MAX
    auto is_overflow = []( int k, int r, int thr ) -> bool
    {
        return ( k > thr ) || ( ( k == thr ) && ( r > 7 ) );
    };
    //helper function to check if
    //k is smaller than INT_MIN
    auto is_underflow = []( int k, int r, int thr ) -> bool
    {
        return ( k < thr ) || ( ( k == thr ) && ( r < -8 ) );
    };
    if( x > 0 )
    {
        //process positive number
        //threshold
        int thr = INT_MAX / 10;
        while( n != 0 )
        {
            //get quotient q
            //and remainder r
            int q = n / 10;
            int r = n - q * 10;
            n = q;
            //check if res will be larger than INT_MIN
            if( is_overflow( res, r, thr ) )
            {
                return 0;
            }
            res = res * 10 + r;
        }
    }
    else
    {
        //process negative number
        //the threshold
        int thr = INT_MIN / 10;
        while( n != 0 )
        {
            //get quotient q
            //and remainder r
            int q = n / 10;
            int r = n - q * 10;
            n = q;
            //check if res will be less than INT_MIN
            if( is_underflow( res, r, thr ) )
            {
                return 0;
            }
            res = res * 10 + r;
        }
    }
    return res;
}
\end{lstlisting}