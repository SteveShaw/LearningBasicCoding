\section{48 --- Rotate Image}
You are given an $n \times n$ 2D matrix representing an image.

Rotate the image by 90 degrees (clockwise).

\paragraph{Note:}

\begin{itemize}
\item You have to rotate the image in-place, which means you have to modify the input 2D matrix directly. \textbf{DO NOT} allocate another 2D matrix and do the rotation.
\end{itemize}

\paragraph{Example 1:}
\begin{flushleft}
\textbf{Input}:
\begin{table}[H]
\begin{tabular}{ccc}
1 & 2 & 3\\
4 & 5 & 6\\
7 & 8 & 9
\end{tabular}
\end{table}
\textbf{Output}:
\begin{table}[H]
\begin{tabular}{ccc}
7 & 4 & 1\\
8 & 5 & 2\\
9 & 6 & 3
\end{tabular}
\end{table}
\end{flushleft}

\paragraph{Example 2:}
\begin{flushleft}
\textbf{Input}:
\begin{table}[H]
\begin{tabular}{cccc}
5 &  1 &  9 & 11\\
2 &  4 &  8 & 10\\
13 &  3 &  6 &  7\\
15 & 14 & 12 & 16
\end{tabular}
\end{table}
\textbf{Output}:
\begin{table}[H]
\begin{tabular}{cccc}
15 & 13 &  2 &  5\\
14 &  3 &  4 &  1\\
12 &  6 &  8 &  9\\
16 &  7 & 10 & 11
\end{tabular}
\end{table}
\end{flushleft}

\subsection{Transpose and then reverse}
\begin{enumerate}
\item The obvious idea would be to transpose the matrix first and then reverse each row. 
\end{enumerate}

\setcounter{lstlisting}{0}
\begin{lstlisting}[style=customc, caption={Transpose Then Reverse}]
void rotate( vector<vector<int>>& matrix )
{
    int n = static_cast<int>( matrix.size() );

    //transpose
    for( int x = 0; x < n; ++x )
    {
        //notice each column starting
        //from x
        for( int y = x; y < n; ++y )
        {
            swap( matrix[x][y], matrix[y][x] );
        }
    }

    //reverse each row

    for( int x = 0; x < n; ++x )
    {
        reverse( matrix[x].begin(), matrix[x].end() );
    }

}
\end{lstlisting}

\subsection{Rotate four numbers}
每次移动4个数字

\setcounter{lstlisting}{0}
\begin{lstlisting}[style=customc, caption={Rotate 4 Numbers Each Time}]
void rotate( vector<vector<int>>& matrix )
{
    auto n = matrix.size();

    //move 4 number at each time
    for( size_t x = 0; x < n / 2; ++x )
    {
        for( size_t y = x; y < n - 1 - x; ++y )
        {
            int tmp = matrix[x][y];
            //overwrite top left by bottom left
            matrix[x][y] = matrix[n - 1 - y][x];
            //overwrite bottom left by bottom right
            matrix[n - 1 - y][x] = matrix[n - 1 - x][n - 1 - y];
            //overwrite bottom right by top right
            matrix[n - 1 - x][n - 1 - y] = matrix[y][n - 1 - x];
            //overwrite top right by value from top left
            matrix[y][n - 1 - x] = tmp;
        }
    }
}
\end{lstlisting}