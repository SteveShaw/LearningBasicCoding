\section{34 --- Find First and Last Position of Element in Sorted Array}
Given an array of integers $A$ sorted in ascending order, find the starting and ending position of a given target value.

Your algorithm's runtime complexity must be in the order of $O(log n)$.

If the target is not found in the array, return $[-1, -1]$.

\paragraph{Example 1:}

\begin{flushleft}
\textbf{Input}: $A = [5,7,7,8,8,10]$, $T = 8$

\textbf{Output}: $[3,4]$
\end{flushleft}

\paragraph{Example 2:}

\begin{flushleft}
\textbf{Input}: $A = [5,7,7,8,8,10]$, $T = 6$

\textbf{Output}: $[-1,-1]$
\end{flushleft}

\subsection{Binary Search}
实际上是有重复元素的二分搜索,或者是lower bound和upper bound的结合,

Given an array $A$ of $n$ elements with values $A_0, A_1, \ldots, A_{n-1}$, sorted such that $A_0 \leq A_1 \leq \ldots \leq A_{n-1}$, and target value $T$
To find the leftmost element, the following procedure can be used:
\begin{enumerate}
    \item Set $L$ to 0 and $R$ to $n$.
    \item If $L \geq R$, go to step \ref{bs-step6}. \label{bs-step2}
    \item Set $m$ (the position of the middle element) to the floor of $\dfrac{L + R}{2}$, which is the greatest integer less than or equal to $\dfrac{L + R}{2}$
    \item If $A_m < T$, set $L$ to $m + 1$ and go to step \ref{bs-step2}.
    \item Otherwise, if $A_m \geq T$, set $R$ to $m$ and go to step \ref{bs-step2}.
    \item Now $L = R$, the search is done, return $L$. \label{bs-step6}
\end{enumerate}
If $L < n$ and $A_L = T$, then $A_L$ is the \textbf{leftmost} element that equals $T$. Even if $T$ is not in the array, $L$ is the rank of in the array, or the number of elements in the array that are less than T.
\setcounter{algorithm}{0}
\begin{algorithm}[H]
\caption{Leftmost Binary Search}
\begin{algorithmic}[1]
\Function{BinarySearchLeftmost}{$A$, $n$, $T$}
\State $L := 0$
\State $R := n$
\While{$L < R$}
\State $M := \lfloor\dfrac{L+R}{2}\rfloor$
\If{$A[M] < T$}
\State $L\gets M+1$
\Else
\State $R\gets M$
\EndIf
\EndWhile
\State \Return $L$
\EndFunction
\end{algorithmic}
\end{algorithm}

To find the rightmost element, the following procedure can be used
\begin{enumerate}
    \item Set $L$ to 0 and $R$ to $n$.
    \item If $L \geq R$, go to step \ref{bsr-step6}. \label{bsr-step2}
    \item Set $m$ (the position of the middle element) to the floor of $\dfrac{L + R}{2}$, which is the greatest integer less than or equal to $\dfrac{L + R}{2}$
    \item If $A_m \leq T$, set $L$ to $m + 1$ and go to step \ref{bs-step2}.
    \item Otherwise, if $A_m \geq T$, set $R$ to $m$ and go to step \ref{bsr-step2}.
    \item Now $L = R$, the search is done, return $L$. \label{bsr-step6}
\end{enumerate}
If $L > 0$ and $A_{L-1} = T$, then $A_{L-1}$ is the \textbf{rightmost} element that equals $T$. Even if $T$ is not in the array, $(n - 1) - L$ is the number of elements in the array that are greater than T.
\begin{algorithm}[H]
\caption{Rightmost Binary Search}
\begin{algorithmic}[1]
\Function{BinarySearchRightmost}{$A$, $n$, $T$}
\State $L := 0$
\State $R := n$
\While{$L < R$}
\State $M := \lfloor\dfrac{L+R}{2}\rfloor$
\If{$A[M] \geq T$}
\State $L\gets M+1$
\Else
\State $R\gets M$
\EndIf
\EndWhile
\State \Return $L$
\EndFunction
\end{algorithmic}
\end{algorithm}

\setcounter{lstlisting}{0}
\begin{lstlisting}[style=customc, caption={Leftmost And Rightmost Binary Search}]
vector<int> searchRange( vector<int>& nums, int target )
{
    if( nums.empty() )
    {
        return {-1, -1};
    }

    int l = 0;
    int r = static_cast<int>( nums.size() );

    //find the first number that is >= target
    while( l < r )
    {
        int mid = ( l + r ) / 2;
        if( nums[mid] < target )
        {
            l = mid + 1;
        }
        else
        {
            r = mid;
        }
    }

    if( l == static_cast<int>( nums.size() ) )
    {
        //no element equal to target
        return {-1, -1};
    }

    if( nums[l] != target )
    {
        //nums[l] > target not equal
        return {-1, -1};
    }

    vector<int> ans{l, l};

    r = static_cast<int>( nums.size() );

    //find the first number that > target
    while( l < r )
    {
        int mid = ( l + r ) / 2;
        if( nums[mid] <= target )
        {
            l = mid + 1;
        }
        else
        {
            r = mid;
        }
    }

    //appreantly, the nums[l-1]=target
    ans[1] = l - 1;

    return ans;
}
\end{lstlisting}