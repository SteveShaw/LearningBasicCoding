\section{77 --- Combinations}
Given two integers $n$ and $k$, return all possible combinations of $k$ numbers out of $1, 2, \ldots, n$.

\paragraph{Example:}

\begin{flushleft}
\textbf{Input}: $n = 4$, $k = 2$

\textbf{Output}:

\[
\begin{bmatrix}
2 & 4\\
3 & 4\\
2 & 3\\
1 & 2\\
1 & 3\\
1 & 4
\end{bmatrix}
\]
\end{flushleft}

\subsection{Backtracking}
Typical backtrack algorithm.

For this problem, we make use of a helper function to do backtracking. 

\begin{enumerate}
\item If the number of current selected integers is $k$, add these selections into output.
\item Iterate over the integers from a start number $s$ to $n$.
\item Add current integer $i$ into the current selected integers, say $A$.
\item Recursively add more integers into $A$.
\item Backtrack by removing $i$ from $A$.
\end{enumerate}

\setcounter{lstlisting}{0}
\begin{lstlisting}[style=customc, caption={Backtracking}]
vector<vector<int>> combine( int n, int k )
{
    vector<int> sels;
    vector<vector<int>> ans;
    //backtracking
    dfs( 1, n, k, sels, ans );
    return ans;
}
//backtracking helper function
void dfs( int start, int n, size_t k, vector<int>& cur, vector<vector<int>>& ans )
{
    if( cur.size() == k )
    {
        //we have k numbers
        //add to oo
        ans.emplace_back( begin( cur ), end( cur ) );
        return;
    }
    for( int i = start; i <= n; ++i )
    {
        //add i to selections
        cur.push_back( i );
        //go to next integer recursively
        dfs( i + 1, n, k, cur, ans );
        //backtrack
        cur.pop_back();
    }
}
\end{lstlisting}