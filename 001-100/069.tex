\section{69 --- Sqrt(x)}
Implement \fcc{int sqrt(int x)}.

Compute and return the square root of $x$, where $x$ is guaranteed to be a non-negative integer.

Since the return type is an integer, the decimal digits are truncated and only the integer part of the result is returned.

\paragraph{Example 1:}

\begin{flushleft}
\textbf{Input}: 4

\textbf{Output}: 2
\end{flushleft}

\paragraph{Example 2:}

\begin{flushleft}
\textbf{Input}: 8

\textbf{Output}: 2

\textbf{Explanation}: 

The square root of 8 is $2.82842\ldots$, and since the decimal part is truncated, 2 is returned.
\end{flushleft}

\subsection{Binary Search}
For $x \ge 2$, the square root is always smaller than $x / 2$ and larger than 0: $0 < a < x / 2$.

Since $a$ is an integer, the problem goes down to the iteration over the sorted set of integer numbers. Here the binary search enters the scene.

We can use a leftmost binary search to find the square root

\setcounter{lstlisting}{0}
\begin{lstlisting}[style=customc, caption={Binary Search}]
int mySqrt( int x )
{
    if( x < 2 )
    {
        return x;
    }
    //the square root
    //is in [2, x/2]
    int l = 2;
    int r = x / 2 + 1;
    //leftmost binary search
    while( l < r )
    {
        auto mid = static_cast<long long>( ( l + r ) / 2 );

        if( mid * mid < x )
        {
            l = mid + 1;
        }
        else
        {
            r = mid;
        }
    }
    //check if l*l=x
    auto y = static_cast<long long>( l );
    if( y * y == x )
    {
        return l;
    }
    //return the previous one
    return l - 1;
}
\end{lstlisting}