\section{52 --- N-Queens II}
The $n$-queens puzzle is the problem of placing $n$ queens on an $n\times n$ chessboard such that no two queens attack each other.

Given an integer $n$, return the number of distinct solutions to the $n$-queens puzzle.

\subsection{Backtrack}

\begin{itemize}
\item Almost same as Problem 51. The only difference in this problem, we just count the total possible placements.
\end{itemize}

\setcounter{lstlisting}{0}
\begin{lstlisting}[style=customc, caption={Backtrack}]
int totalNQueens( int n )
{
    vector<int> queens( n );

    vector<unsigned char> rows( n, 0 );
    vector<unsigned char> hills( 2 * n - 1, 0 );
    vector<unsigned char> dales( 2 * n - 1, 0 );

    int ans = 0;

    dfs( 0, n, rows, hills, dales, queens, ans );

    return ans;
}

void dfs( int r, int n,
          vector<unsigned char>& rows,
          vector<unsigned char>& hills,
          vector<unsigned char>& dales,
          vector<int>& queens,
          int& ans )
{
    for( int c = 0; c < n; ++c )
    {
        if( hills[r + c] + dales[r - c + n - 1] + rows[c] == 0 )
        {
            //place queen at (r,c)
            queens[r] = c;
            hills[r + c] = 1;
            dales[r - c + n - 1] = 1;
            rows[c] = 1;

            if( r + 1 == n )
            {
                //all queens are placed
                //increments the result
                ++ans;
            }
            else
            {
                //try next row
                dfs( r + 1, n, rows, hills, dales, queens, ans );
            }

            //backtrack
            queens[r] = 0;
            hills[r + c] = 0;
            dales[r - c + n - 1] = 0;
            rows[c] = 0;
        }
    }
}
\end{lstlisting}