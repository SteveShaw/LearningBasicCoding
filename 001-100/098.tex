\section{98 --- Validate Binary Search Tree}
Given a binary tree $T$, determine if it is a valid binary search tree (BST).
\par
Assume a BST is defined as follows:
\begin{itemize}
\item The left subtree of a node contains only nodes with keys less than the node's key.
\item The right subtree of a node contains only nodes with keys greater than the node's key.
\item Both the left and right subtrees must also be binary search trees.
\end{itemize}
\paragraph{Example 1:}
\begin{flushleft}
\textbf{Input}:
\begin{figure}[H]
\begin{tikzpicture}
[mynode/.style={draw,circle,minimum size=5mm, fill=gray!20!}]
\node(){};
\node[mynode](2) {2};
\node[mynode](1)[below=8mm of 2, xshift=-8mm] {1};
\node[mynode](3)[below=8mm of 2, xshift=8mm] {3};
\draw[>=stealth,->] (2) -- (1);
\draw[>=stealth,->] (2) -- (3);
\end{tikzpicture}
\end{figure}
\textbf{Output}: \texttt{true}
\end{flushleft}
\paragraph{Example 2:}
\begin{flushleft}
\begin{figure}[H]
\begin{tikzpicture}
[mynode/.style={draw,circle,minimum size=5mm, fill=gray!20!}]
\node(){};
\node[mynode](5) {5};
\node[mynode](1)[below=8mm of 5, xshift=-8mm] {1};
\node[mynode](4)[below=8mm of 5, xshift=8mm] {4};
\node[mynode](3)[below=8mm of 4, xshift=-8mm] {3};
\node[mynode](6)[below=8mm of 4, xshift=8mm] {6};
\draw[>=stealth,->] (5) -- (1);
\draw[>=stealth,->] (5) -- (4);
\draw[>=stealth,->] (4) -- (3);
\draw[>=stealth,->] (4) -- (6);
\end{tikzpicture}
\end{figure}
\textbf{Output}: \texttt{false}
\\
\textbf{Explanation}: The root node's value is 5 but its right child's value is 4.
\end{flushleft}
\subsection{Recursion}
验证二叉搜索树有很多种解法,可以利用它本身的性质来做,即$\text{left} < \text{root} < \text{right}$,也可以通过利用中序遍历结果为有序数列来做,在Recursive方法中,初始化时带入系统最大值和最小值,在递归过程中换成它们自己的节点值,具体实现时为了避免出现节点值出现\fcj{int}的最大值和最小值的情况,用\fcj{long long} 代替\fcj{int}

\setcounter{lstlisting}{0}
\begin{lstlisting}[style=customc, caption={Recurision}]
bool isValidBST( TreeNode* root )
{
    return dfs( root, false, -1, false, -1 );
}
//flagl: indicate if low limit is set
//flagh: indicate if high limit is set
bool dfs( TreeNode* node, bool flagl, int lo, bool flagh, int hi )
{
    if( !node )
    {
        return true;
    }
    if( flagl && ( lo >= node->val ) )
    {
        //value <= lower
        return false;
    }
    if( flagh && ( hi <= node->val ) )
    {
        //value >= higher
        return false;
    }
    //set flagh to true
    //flagl is still same
    if( !dfs( node->left, flagl, lo, true, node->val ) )
    {
        return false;
    }
    //set flagl to true
    //flagh is still same
    if( !dfs( node->right, true, node->val, flagh, hi ) )
    {
        return false;
    }
    return true;
}
\end{lstlisting}

\subsection{In--order Traverse}
一般的二叉搜索树是$\text{LEFT}\leq\text{ROOT}\leq\text{RIGHT}$,而这道题设定为$\text{LEFT}<\text{ROOT}<\text{RIGHT}$,那么就可以用中序遍历来做。如果不去掉$\text{LEFT}=\text{ROOT}$这个条件的话,那么下边两个数用中序遍历无法区分:
\begin{figure}[H]
\begin{tikzpicture}
[mynode/.style={draw,circle,minimum size=5mm, fill=gray!20!}]
\node(){};
\node[mynode](1) {20};
\node[mynode](2)[right=8mm of 1] {20};
\node[mynode](3)[below=8mm of 1, xshift=-6mm] {20};
\node[mynode](4)[below=8mm of 2, xshift=6mm] {20};
\draw[>=stealth,->] (1) -- (3);
\draw[>=stealth,->] (2) -- (4);
\end{tikzpicture}
\end{figure}
它们的中序遍历结果都一样,但是左边的是BST,右边的不是BST。去掉等号的条件则相当于去掉了这种限制条件。中序遍历方法思路很直接,每当遍历到一个新节点时和其上一个节点比较,如果不大于上一个节点那么则返回\fcj{false},全部遍历完成后返回\fcj{true}
%\subsubsection{Algorithm}
%\begin{algorithm}[H]
%\begin{algorithmic}[1]
%\Procedure{IsValidBST}{$T$}
%\State $\alpha:=\text{NULL}$ \Comment The pointer point to the previous node
%\If{$T=\text{NULL}$}
%\State \Return \texttt{true}
%\EndIf
%\State $\texttt{ans}:=\texttt{Inorder}(T, \alpha)$
%\State \Return \texttt{ans}
%\EndProcedure
%\end{algorithmic}
%\end{algorithm}
%Function \texttt{Inorder} traverse the tree recursively and check the order between current node and previous node $\alpha$
%\begin{algorithm}[H]
%\caption{Inorder Traverse}
%\begin{algorithmic}[1]
%\Function{Inorder}{$T,\alpha$}
%\If{$T=\text{NULL}$}
%\State \Return \texttt{true}
%\EndIf
%\State $\texttt{ans}:=\texttt{Inorder}(\texttt{LEFT}(T), \alpha)$ \Comment Traverse left child tree
%\If{$\texttt{ans}=\texttt{false}$}
%\State \Return \texttt{false}
%\EndIf
%\If{$\alpha\neq\text{NULL}$}
%\State $v_0:=\texttt{Value}(\alpha)$
%\State $v_1:=\texttt{Value}(T)$
%\algstore{98algo}
%\end{algorithmic}
%\end{algorithm}
%\begin{algorithm}[H]
%\begin{algorithmic}[1]
%\algrestore{98algo}
%\If{$v_0\geq v_1$} \Comment Violate the property of BST
%\State \Return \texttt{false} 
%\EndIf
%\EndIf
%\State $\alpha\gets T$ \Comment Update previous node as current node
%\State $\texttt{ans}\gets \texttt{Inorder}(\texttt{RIGHT}(T), \alpha)$ \Comment Traverse right child tree
%\State \Return \texttt{ans}
%\EndFunction
%\end{algorithmic}
%\end{algorithm}
