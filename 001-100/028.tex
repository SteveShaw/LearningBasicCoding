\section{28 --- Implement strStr()}
Implement function \texttt{strStr}().

Return the index of the first occurrence of \textbf{needle} in \textbf{haystack}, or $-1$ if \textbf{needle} is not part of \textbf{haystack}.

\paragraph{Example 1:}

\begin{flushleft}
\textbf{Input}: \textbf{haystack} = \texttt{hello}, \textbf{needle} = \texttt{ll}


\textbf{Output}: 2
\end{flushleft}

\paragraph{Example 2:}

\begin{flushleft}
\textbf{Input}: \textbf{haystack} = \texttt{aaaaa}, \textbf{needle} = \texttt{bba}

\textbf{Output}: $-1$
\end{flushleft}

\paragraph{Clarification}:

\begin{flushleft}
What should we return when needle is an empty string? This is a great question to ask during an interview.

For the purpose of this problem, we will return 0 when needle is an empty string. 
\end{flushleft}

\subsection{KMP}
\begin{itemize}
\item 网上很多KMP算法介绍中,table包含$-1$。
\item 这里的KMP算法中的table不包含$-1$,因此当找到两个不匹配的字符时,如果pattern的当前位置$y$不为零,则$y\gets f(y-1)$,其中$f$为得到的pattern字符串的table。而如果$y$为零,则increments被搜寻字符串中的index $x$。
\end{itemize}

\setcounter{lstlisting}{0}
\begin{lstlisting}[style=customc, caption={KMP}]
int strStr( string haystack, string needle )
{
    if( needle.empty() )
    {
        return 0;
    }

    if( haystack.empty() )
    {
        return -1;
    }

    vector<int> f( needle.size(), 0 );

    int t = 0;

    //generate pattern table for needle
    for( size_t i = 1; i < needle.size(); ++i )
    {
        t = f[i - 1];

        while( ( t > 0 ) && ( needle[i] != needle[t] ) )
        {
            t = f[t - 1];
        }

        if( needle[i] == needle[t] )
        {
            t = t + 1;
        }

        f[i] = t;
    }

    //index in haystack
    size_t x = 0;
    //index in needle
    size_t y = 0;

    while( x < haystack.size() )
    {
        if( haystack[x] == needle[y] )
        {
            ++x;
            ++y;

            if( y == needle.size() )
            {
                //returns the start index
                //of the matching
                return x - y;
            }
        }
        else
        {
            if( y > 0 )
            {
                //get the next search index
                //in the pattern from f[y-1]
                y = f[y - 1];
            }
            else
            {
                //otherwise
                //increments the index in
                //haystack
                ++x;
            }
        }

    }

    return -1;

}
\end{lstlisting}