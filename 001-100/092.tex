\section{92 --- Reverse Linked List II}
Reverse a linked list $\ell$ from position $m$ to $n$. Do it in one-pass.
\par
Note: $1 \leq m \leq n \leq L$. $L$ is the length of $\ell$
\paragraph{Example:}
\begin{flushleft}
\textbf{Input}:
\end{flushleft}
\begin{figure}[H]
\begin{tikzpicture}
[mynode/.style={draw,circle,minimum size=5mm, fill=gray!20!}]
\node(){};
\node[mynode](1){$1$};
\node[mynode](2) [right=8mm of 1]{$2$};
\node[mynode](3) [right=8mm of 2]{$3$};
\node[mynode](4) [right=8mm of 3]{$4$};
\node[mynode](5) [right=8mm of 4]{$5$};
\node[mynode](e) [right=8mm of 5]{$\varnothing$};
\draw[>=stealth,->] (1) -- (2);
\draw[>=stealth,->] (2) -- (3);
\draw[>=stealth,->] (3) -- (4);
\draw[>=stealth,->] (4) -- (5);
\draw[>=stealth,->] (5) -- (e);
\end{tikzpicture}
\end{figure}
\begin{flushleft}
$m=2, n=4$
\\
\textbf{Output}:
\end{flushleft}
\begin{figure}[H]
\begin{tikzpicture}
[mynode/.style={draw,circle,minimum size=5mm, fill=gray!20!}]
\node(){};
\node[mynode](1){$1$};
\node[mynode](2) [right=8mm of 1]{$4$};
\node[mynode](3) [right=8mm of 2]{$3$};
\node[mynode](4) [right=8mm of 3]{$2$};
\node[mynode](5) [right=8mm of 4]{$5$};
\node[mynode](e) [right=8mm of 5]{$\varnothing$};
\draw[>=stealth,->] (1) -- (2);
\draw[>=stealth,->] (2) -- (3);
\draw[>=stealth,->] (3) -- (4);
\draw[>=stealth,->] (4) -- (5);
\draw[>=stealth,->] (5) -- (e);
\end{tikzpicture}
\end{figure}
\subsection{Analysis}
基本的思路有两点
\begin{enumerate}
\item 创建一个dummy node,然后其next指针指向head。然后用变量$\alpha$代表previous node,而变量$\beta$代表current node。在开始时,$\alpha$就是dummy node,而$\beta$就是head。设定counter $\delta$起始为1。然后移动$\alpha$和$\beta$直到$\delta=m$。
\item 这个时候,$\beta$指向reverse的起始node,reverse的过程就是把$\beta$的下一个node $n$ 移动到$\alpha$的下一个节点位置。同时increment $\delta$直到$\delta=n$。移动的过程中,$\alpha$和$\beta$保持不动,每次都会更新$n$为$\beta$的下一个节点。
\end{enumerate}

\setcounter{lstlisting}{0}
\begin{lstlisting}[style=customc, caption={Iteraitve}]
ListNode* reverseBetween( ListNode* head, int m, int n )
{
    auto dummy = make_unique<ListNode>( -1 );
    dummy->next = head;
    auto pre = dummy.get();
    auto cur = head;
    int count = 1;
    //get the start node
    while( count < m )
    {
        pre = pre->next;
        cur = cur->next;
        ++count;
    }
    //perform reverse
    //in this process, cur and pre are fixed
    while( count < n )
    {
        auto nx = cur->next;
        cur->next = nx->next;
        nx->next = pre->next;
        pre->next = nx;
        ++count;
    }
    //return the new head;
    auto new_head = dummy->next;
    dummy->next = nullptr;
    return new_head;
}
\end{lstlisting}

\subsection{Recursion}
