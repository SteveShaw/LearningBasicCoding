\section{80 --- Remove Duplicates from Sorted Array II}
Given a sorted array $A$, remove the duplicates in--place such that duplicates appeared at most twice and return the new length.
\par
Do not allocate extra space for another array, you must do this by modifying the input array in-place with $O(1)$ extra memory.
\paragraph{Example 1:}
\begin{flushleft}
Given $A = [1,1,1,2,2,3]$, Your function should return $5$, with the first five elements of nums being 1, 1, 2, 2 and 3 respectively. It does not matter what you leave beyond the returned length.
\end{flushleft}
\paragraph{Example 2:}
\begin{flushleft}
Given $A = [0,0,1,1,1,1,2,3,3]$, Your function should return 7, with the first seven elements of $A$ being modified to 0, 0, 1, 1, 2, 3 and 3 respectively. It does not matter what values are set beyond the returned length.
\end{flushleft}
\subsection{Read And Write Pointers}
这里需要一个counter来记录重复元素出现的次数。如果碰到不同的数字,把counter重新设为1。如果遇到重复的数字,需要首先检查这个counter,如果这个counter已经是2(如果进行推广,这里的2也可以换成任意指定的所允许的重复个数$k$), 我们就跳过这个数字, 否则仍然保留这个数字。这里需要两个index:\fcj{write} and \fcj{read},前者用于指定拷贝元素的位置,而后者则是扫描数组用的。

\setcounter{lstlisting}{0}
\begin{lstlisting}[style=customc, caption={Read And Write Pointers}]
int removeDuplicates( vector<int>& nums )
{
    if( nums.empty() )
    {
        return 0;
    }
    //write index
    int write = 0;
    int n = static_cast<int>( nums.size() );
    //count of duplicate numbers
    int count = 1;
    for( int read = 1; read < n; ++read )
    {
        if( nums[read] == nums[write] )
        {
            if( count == 1 )
            {
                //count is going to be 2
                //so we can still keep nums[read]
                //write to next index
                ++write;
                nums[write] = nums[read];
                ++count;
            }
        }
        else
        {
            //reset counter to 1
            count = 1;
            //write to next index
            ++write;
            nums[write] = nums[read];
        }
    }
    //when read index is finished
    //write index is in the final
    //valid number.
    //The length is write + 1
    return write + 1;
}
\end{lstlisting}
