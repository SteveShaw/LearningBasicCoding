\section{82 -- Remove Duplicates from Sorted List II}
\tikzset
{
llstyle/.style={>=stealth',auto,node distance=7mm},
mynode/.style={circle, draw, minimum size=5mm, fill=gray!15}
}
Given a sorted linked list with head node $H$, delete all nodes that have duplicate numbers, leaving only distinct numbers from the original list.
\paragraph{Example 1:}
\begin{flushleft}
\textbf{Input}:
\begin{figure}[H]
\begin{tikzpicture}[llstyle]
\node(){};
\node[mynode](1){1};
\node[mynode, right=of 1](2){2};
\node[mynode, right=of 2](3){3};
\node[mynode, right=of 3](31){3};
\node[mynode, right=of 31](4){4};
\node[mynode, right=of 4](41){4};
\node[mynode, right=of 41](5){5};
\draw[->,>=stealth'] (1)--(2);
\draw[->,>=stealth'] (2)--(3);
\draw[->,>=stealth'] (3)--(31);
\draw[->,>=stealth'] (31)--(4);
\draw[->,>=stealth'] (4)--(41);
\draw[->,>=stealth'] (41)--(5);
\end{tikzpicture}
\end{figure}
\textbf{Output:}
\begin{figure}[H]
\begin{tikzpicture}[llstyle]
\node(){};
\node[mynode](1){1};
\node[mynode, right=7mm of 1](2){2};
\node[mynode, right=7mm of 2](5){5};
\draw[->,>=stealth'] (1)--(2);
\draw[->,>=stealth'] (2)--(5);
\end{tikzpicture}
\end{figure}
\end{flushleft}
\paragraph{Example 2:}
\begin{flushleft}
\textbf{Input}:
\begin{figure}[H]
\begin{tikzpicture}[llstyle]
\node(){};
\node[mynode](1){1};
\node[mynode, right=of 1](a1){1};
\node[mynode, right=of a1](b1){1};
\node[mynode, right=of b1](2){2};
\node[mynode, right=of 2](3){3};
\draw[->,>=stealth'] (1)--(a1);
\draw[->,>=stealth'] (a1)--(b1);
\draw[->,>=stealth'] (b1)--(2);
\draw[->,>=stealth'] (2)--(3);
\end{tikzpicture}
\end{figure}
\textbf{Output:}
\begin{figure}[H]
\begin{tikzpicture}[llstyle]
\node(){};
\node[mynode](2){2};
\node[mynode, right=of 2](3){3};
\draw[->,>=stealth'] (2)--(3);
\end{tikzpicture}
\end{figure}
\end{flushleft}
\subsection{Linear Scan}
\begin{CJK*}{UTF8}{gbsn}
首先创建一个dummy节点,指向\texttt{head},然后从\texttt{head}开始,我们维护三个节点,分别指向当前节点$\alpha$,当前节点的前一个节点$\beta$以及当前节点的下一个节点$\gamma$。同时用一个计数器$\delta$记录遇到的重复节点个数。
\par
然后从头节点出发,如果$\alpha$和$\gamma$有不同的value,$\alpha$会保留,所以这时候$\beta$前进到$\alpha$而$\alpha$前进到$\gamma$。否则,从$\alpha$开始搜索相同的节点,直至链表末尾或者$\gamma$和$\alpha$具有不同的值。这时候如果$\alpha$是链表末尾,表示从$\beta$后的下一个节点开始直至链表末尾都是相同的元素,直接把$\beta$的指向下一个节点的指针指向\texttt{null}然后返回。否则,将$\beta$的指向下一个节点的指针指向$\gamma$。但这时候$\beta$并不前进还是停留在当前位置,因为我们并不确定从$\gamma$开始的节点是否有重复。而$\alpha$仍旧前进到$\gamma$的位置。最后返回dummy的节点的下一个节点作为头节点。 
\end{CJK*}
\subsubsection{Code}
Procedure \texttt{DeleteDuplicate} delete all duplicate nodes from linked list with head $H$.
\setcounter{algorithm}{0}
\begin{algorithm}[H]
\caption{Linear Scanning}
\begin{algorithmic}[1]
\Procedure{DeleteDuplicate}{$H$}
\State Create a dummny node $D$
\State $D_n\gets H$ \Comment The next pointer of $D$ points to the list head $H$
\State $\alpha:=D$ \Comment The previous node
\State $\beta:=H$ \Comment The current node
\While{$\beta \neq 0$} \label{82while1}
\State $\delta:=0$ \Comment The counter of duplicate nodes
\State $\gamma:=\beta_n$ \Comment The next node of $\beta$
\While{$\beta\neq 0$ \textbf{and} $\gamma \neq 0$} \label{82while2}
\If{$V(\beta)\neq V(\gamma)$} \Comment The two nodes are not duplicate
\State \texttt{break} \Comment Stop while loop [\ref{82while2}]
\EndIf
\State $\delta\gets \delta+1$ \Comment Increments the counter
\State $\beta \gets \gamma$ \Comment $\beta$ move forward to $\gamma$
\State $\gamma \gets \gamma_n$ \Comment $\gamma$ move forward to next node
\EndWhile \Comment End[\ref{82while2}]
\If{$\delta > 0$} \Comment Found duplicate nodes
\State $\delta\gets 0$ \Comment Reset the counter
\If{$\beta \neq 0$} \Comment Not at end of the list
\State $\alpha_n\gets \gamma$ \Comment Change previous node's next pointer to $\gamma$
\Else
\State $\alpha_n\gets 0$ \Comment At the end of the list
\State \texttt{break} \Comment Stop while loop[\ref{82while1}]
\EndIf
\Else
\State $\alpha \gets \beta$ \Comment Move previous node to current node
\EndIf
\State $\beta\gets \beta_n$ \Comment Move current node to next node
\EndWhile
\algstore{82algo}
\end{algorithmic}
\end{algorithm}
\begin{algorithm}[H]
\begin{algorithmic}[1]
\algrestore{82algo}
\State \Return $D_n$ \Comment Return the next node of dummny node
\EndProcedure
\end{algorithmic}
\end{algorithm}
\subsection{Recursive}
\begin{CJK*}{UTF8}{gbsn}
递归算法比较简洁。如果当前点$\beta$以及其下一个节点$\gamma$都不为空,那么如果两个不是重复的,递归处理$\gamma$,然后将返回的头节点设置为$\beta$的\texttt{next}指针。如果是重复的,就从$\gamma$开始找到第一个不是重复的节点,然后递归处理该节点,由于$\beta$和$\gamma$是重复的,所以它们就被放弃掉,直接返回递归处理后的头节点。
\end{CJK*}
\subsubsection{Code}
\begin{algorithm}[H]
\caption{Recursive Approach}
\begin{algorithmic}[1]
\Procedure{DeleteDuplicate}{$H$}
\State \Return $\Gamma(H)$ \Comment Recursive function $\Gamma$
\EndProcedure
\end{algorithmic}
\end{algorithm}

Function $\Gamma$ recursively process the linked list $\beta$ to remove duplicate elements
\begin{algorithm}[H]
\caption{Recursively Removing Duplicate Elements}
\begin{algorithmic}[1]
\Function{$\Gamma$}{$\beta$}
\If{$\beta=0$} \Comment The current list is empty
\State \Return 0
\EndIf
\If{$\beta_n=0$} \Comment The current list has only one node
\State \Return $\beta$
\EndIf
\State $\gamma:=\beta_n$
\If{$V(\gamma)\neq V(\beta)$} \Comment They are not duplicate
\State $\beta_n \gets \Gamma(\gamma)$ \Comment Recursively process linked list with head $\gamma$
\State \Return $\beta$ \Comment Return current head
\EndIf
\State $v:=V(\beta_n)$
\algstore{82algo}
\end{algorithmic}
\end{algorithm}
\begin{algorithm}[H]
\begin{algorithmic}[1]
\algrestore{82algo}
\While{$\gamma\neq 0$ \textbf{and} $V(\gamma)=v$}
\State $\gamma \gets \gamma_n$ \Comment Move to next node
\EndWhile
\State \Return $\Gamma(\gamma)$ \Comment Recursively process linked list with head $\gamma$ and exclude duplicates as $\beta$
\EndFunction
\end{algorithmic}
\end{algorithm}