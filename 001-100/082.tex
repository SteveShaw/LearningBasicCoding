\section{82 -- Remove Duplicates from Sorted List II}
\tikzset
{
llstyle/.style={>=stealth',auto,node distance=7mm},
mynode/.style={circle, draw, minimum size=5mm, fill=gray!15}
}
Given a sorted linked list with head node $H$, delete all nodes that have duplicate numbers, leaving only distinct numbers from the original list.
\paragraph{Example 1:}
\begin{flushleft}
\textbf{Input}:
\begin{figure}[H]
\begin{tikzpicture}[llstyle]
\node(){};
\node[mynode](1){1};
\node[mynode, right=of 1](2){2};
\node[mynode, right=of 2](3){3};
\node[mynode, right=of 3](31){3};
\node[mynode, right=of 31](4){4};
\node[mynode, right=of 4](41){4};
\node[mynode, right=of 41](5){5};
\draw[->,>=stealth'] (1)--(2);
\draw[->,>=stealth'] (2)--(3);
\draw[->,>=stealth'] (3)--(31);
\draw[->,>=stealth'] (31)--(4);
\draw[->,>=stealth'] (4)--(41);
\draw[->,>=stealth'] (41)--(5);
\end{tikzpicture}
\end{figure}
\textbf{Output:}
\begin{figure}[H]
\begin{tikzpicture}[llstyle]
\node(){};
\node[mynode](1){1};
\node[mynode, right=7mm of 1](2){2};
\node[mynode, right=7mm of 2](5){5};
\draw[->,>=stealth'] (1)--(2);
\draw[->,>=stealth'] (2)--(5);
\end{tikzpicture}
\end{figure}
\end{flushleft}
\paragraph{Example 2:}
\begin{flushleft}
\textbf{Input}:
\begin{figure}[H]
\begin{tikzpicture}[llstyle]
\node(){};
\node[mynode](1){1};
\node[mynode, right=of 1](a1){1};
\node[mynode, right=of a1](b1){1};
\node[mynode, right=of b1](2){2};
\node[mynode, right=of 2](3){3};
\draw[->,>=stealth'] (1)--(a1);
\draw[->,>=stealth'] (a1)--(b1);
\draw[->,>=stealth'] (b1)--(2);
\draw[->,>=stealth'] (2)--(3);
\end{tikzpicture}
\end{figure}
\textbf{Output:}
\begin{figure}[H]
\begin{tikzpicture}[llstyle]
\node(){};
\node[mynode](2){2};
\node[mynode, right=of 2](3){3};
\draw[->,>=stealth'] (2)--(3);
\end{tikzpicture}
\end{figure}
\end{flushleft}
\subsection{Linear Scan}
Create a dummy node as the previous node of head. When iterating the list, skip duplicate nodes and correctly process \fcj{next} field of each node.

\setcounter{lstlisting}{0}
\begin{lstlisting}[style=customc, caption={Iterative}]
ListNode* deleteDuplicates( ListNode* head )
{
    //create a dummy node
    auto dummy = new ListNode( -1 );
    dummy->next = head;
    auto pre = dummy;
    while( pre->next )
    {
        auto cur = pre->next;
        //skip duplicate nodes
        while( cur->next && ( cur->next->val == cur->val ) )
        {
            cur = cur->next;
        }
        //change pre->next to the first
        //node with different value
        if( cur != pre->next )
        {
            pre->next = cur->next;
        }
        else
        {
            //no duplicate is found
            //move to next node of pre
            pre = pre->next;
        }
    }
    auto ans = dummy->next;
    dummy->next = nullptr;
    return ans;
}
\end{lstlisting}

\subsection{Recursive}
递归算法比较简洁。如果当前node以及其下一个node都不为\fcj{null},那么如果两个不是重复的,递归处理下一个node,然后将返回的头节点设置为当前节点的\fcj{next}。如果是重复的,就从下一个node开始找到第一个不是重复的节点,然后递归处理该节点,由于当前和下一个node是重复的,所以它们就被放弃掉,直接返回递归处理后的头节点。

\begin{lstlisting}[style=customc, caption={Recursive}]
ListNode* deleteDuplicates( ListNode* head )
{
    if( !head )
    {
        return head;
    }
    if( head->next && ( head->next->val == head->val ) )
    {
        //skip duplicate nodes
        while( head->next && ( head->next->val == head->val ) )
        {
            head = head->next;
        }
        //now head->next is different from head
        return deleteDuplicates( head->next );
    }
    //head and head->next are not equal
    //set head->next as the result of deleteDuplicates
    head->next = deleteDuplicates( head->next );
    return head;
}
\end{lstlisting}
