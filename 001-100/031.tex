\section{31 --- Next Permutation}
Implement \textbf{next permutation}, which rearranges numbers into the lexicographically next greater permutation of numbers.
\par
If such arrangement is not possible, it must rearrange it as the lowest possible order (ie, sorted in ascending order).
\par
The replacement must be in-place and use only constant extra memory.
\par
Here are some examples. Inputs are in the left-hand column and its corresponding outputs are in the right-hand column.
\begin{table}[H]
\begin{tabular}{lcl}
(1,2,3) & $\Longrightarrow$ & (1,3,2) \\
(3,2,1) & $\Longrightarrow$ & (1,2,3) \\
(1,1,5) & $\Longrightarrow$ & (1,5,1)
\end{tabular}
\end{table}
\subsection{Single Pass}
\begin{enumerate}
    \item 如果给定的sequence已经是降序排列,就不会有next permutation。例如 $[9, 5, 4, 3, 1]$
    \item 从右到左寻找the first pair of two successive numbers $a[i]$ and $a[i-1]$, which satisfy $A[i] > A[i-1]$. 由于从$A[i]$到最后一个数都是降序排列,因此这个部分无需进行rearrange。需要rearrang的是从$A[i-1]$到最后一个数。
    \item 由于next permutation is the one just larger than the current one,因此方法就是用$A[i-1]$的右边所有数中,即从$A[i]$到$A[L-1]$中大于$A[i-1]$的数中最小的那个数替换$A[i-1]$。而由于$A[i]$到$A[L-1]$都是降序排列,因此只需从位置$i$开始寻找最后一个大于$A[i-1]$的数即可,假设这个数是$A[j]$。 
    \item Swap $A[i-1]$ and $A[j]$。 这时候$A[i]$到$A[L-1]$仍然是降序排列。为了得到更小的permutation, 需要将其变为升序排列。因此 Reverse $A[i]$ to $A[L-1]$就能得到next permutation了。
\end{enumerate}
\begin{figure}[H]
\begin{tikzpicture}
[mynode/.style={draw,minimum height=1cm, inner sep=0.5 cm, node distance=0.6cm, fill=gray!20!}]
\node[rectangle split, rectangle split horizontal, rectangle split parts=9] (a)[mynode] {$1$ \nodepart{two} $5$ \nodepart{three} $8$ \nodepart{four} $4$ \nodepart{five} $7$ \nodepart{six} $6$ \nodepart{seven} $5$ \nodepart{eight} $3$ \nodepart{nine} $1$};
\node (b) [below=8mm of a.nine south] {};
\node (c) [above=1cm of a] {从右到左寻找第一个decreasing number};
\draw[red, >=stealth,->, line width=2pt] (b) -- (a.nine south);
\end{tikzpicture}
\end{figure}
\begin{figure}[H]
\begin{tikzpicture}
[mynode/.style={draw,minimum height=1cm, inner sep=0.5 cm, node distance=0.6cm, fill=gray!20!}]
\node[rectangle split, rectangle split horizontal, rectangle split parts=9] (a)[mynode] {$1$ \nodepart{two} $5$ \nodepart{three} $8$ \nodepart{four} $4$ \nodepart{five} $7$ \nodepart{six} $6$ \nodepart{seven} $5$ \nodepart{eight} $3$ \nodepart{nine} $1$};
\node (b) [below=8mm of a.eight south] {};
\draw[red, >=stealth,->, line width=2pt] (b) -- (a.eight south);
\end{tikzpicture}
\end{figure}
\begin{figure}[H]
\begin{tikzpicture}
[mynode/.style={draw,minimum height=1cm, inner sep=0.5 cm, node distance=0.6cm, fill=gray!20!}]
\node[rectangle split, rectangle split horizontal, rectangle split parts=9] (a)[mynode] {$1$ \nodepart{two} $5$ \nodepart{three} $8$ \nodepart{four} $4$ \nodepart{five} $7$ \nodepart{six} $6$ \nodepart{seven} $5$ \nodepart{eight} $3$ \nodepart{nine} $1$};
\node (b) [below=8mm of a.seven south] {};
\draw[red, >=stealth,->, line width=2pt] (b) -- (a.seven south);
\end{tikzpicture}
\end{figure}
\begin{figure}[H]
\begin{tikzpicture}
[mynode/.style={draw,minimum height=1cm, inner sep=0.5 cm, node distance=0.6cm, fill=gray!20!}]
\node[rectangle split, rectangle split horizontal, rectangle split parts=9] (a)[mynode] {$1$ \nodepart{two} $5$ \nodepart{three} $8$ \nodepart{four} $4$ \nodepart{five} $7$ \nodepart{six} $6$ \nodepart{seven} $5$ \nodepart{eight} $3$ \nodepart{nine} $1$};
\node (b) [below=8mm of a.six south] {};
\draw[red, >=stealth,->, line width=2pt] (b) -- (a.six south);
\end{tikzpicture}
\end{figure}
\begin{figure}[H]
\begin{tikzpicture}
[mynode/.style={draw,minimum height=1cm, inner sep=0.5 cm, node distance=0.6cm, fill=gray!20!}]
\node[rectangle split, rectangle split horizontal, rectangle split parts=9] (a)[mynode] {$1$ \nodepart{two} $5$ \nodepart{three} $8$ \nodepart{four} $4$ \nodepart{five} $7$ \nodepart{six} $6$ \nodepart{seven} $5$ \nodepart{eight} $3$ \nodepart{nine} $1$};
\node (b) [below=8mm of a.five south] {};
\draw[red, >=stealth,->, line width=2pt] (b) -- (a.five south);
\end{tikzpicture}
\end{figure}
\begin{figure}[H]
\begin{tikzpicture}
[mynode/.style={draw,minimum height=1cm, inner sep=0.5 cm, node distance=0.6cm, fill=gray!20!}]
\node[rectangle split, rectangle split horizontal, rectangle split parts=9] (a)[mynode] {$1$ \nodepart{two} $5$ \nodepart{three} $8$ \nodepart{four} $4$ \nodepart{five} $7$ \nodepart{six} $6$ \nodepart{seven} $5$ \nodepart{eight} $3$ \nodepart{nine} $1$};
\node (b) [below=8mm of a.four south] {};
\draw[red, >=stealth,->, line width=2pt] (b) -- (a.four south);
\node (c) [above=1cm of a] {找到了第一个decreasing number = $4$};
\end{tikzpicture}
\end{figure}
\begin{figure}[H]
\begin{tikzpicture}
[mynode/.style={draw,minimum height=1cm, inner sep=0.5 cm, node distance=0.6cm, fill=gray!20!}]
\node[rectangle split, rectangle split horizontal, rectangle split parts=9] (a)[mynode] {$1$ \nodepart{two} $5$ \nodepart{three} $8$ \nodepart{four} $4$ \nodepart{five} $7$ \nodepart{six} $6$ \nodepart{seven} $5$ \nodepart{eight} $3$ \nodepart{nine} $1$};
\node (b) [below=10mm of a.four south] {\textcolor{red}{$A[i-1]$}};
\draw[red, >=stealth,->, line width=2pt] (b) -- (a.four south);
\node (d) [above=1cm of a.five north] {};
\draw[blue, >=stealth,->, line width=2pt] (d) -- (a.five north);
\node (c) [above=2cm of a] {从4开始往右寻找最后一个大于4的number};
\end{tikzpicture}
\end{figure}
\begin{figure}[H]
\begin{tikzpicture}
[mynode/.style={draw,minimum height=1cm, inner sep=0.5 cm, node distance=0.6cm, fill=gray!20!}]
\node[rectangle split, rectangle split horizontal, rectangle split parts=9] (a)[mynode] {$1$ \nodepart{two} $5$ \nodepart{three} $8$ \nodepart{four} $4$ \nodepart{five} $7$ \nodepart{six} $6$ \nodepart{seven} $5$ \nodepart{eight} $3$ \nodepart{nine} $1$};
\node (b) [below=10mm of a.four south] {\textcolor{red}{$A[i-1]$}};
\draw[red, >=stealth,->, line width=2pt] (b) -- (a.four south);
\node (d) [above=1cm of a.six north] {};
\draw[blue, >=stealth,->, line width=2pt] (d) -- (a.six north);
\end{tikzpicture}
\end{figure}
\begin{figure}[H]
\begin{tikzpicture}
[mynode/.style={draw,minimum height=1cm, inner sep=0.5 cm, node distance=0.6cm, fill=gray!20!}]
\node[rectangle split, rectangle split horizontal, rectangle split parts=9] (a)[mynode] {$1$ \nodepart{two} $5$ \nodepart{three} $8$ \nodepart{four} $4$ \nodepart{five} $7$ \nodepart{six} $6$ \nodepart{seven} $5$ \nodepart{eight} $3$ \nodepart{nine} $1$};
\node (b) [below=10mm of a.four south] {\textcolor{red}{$A[i-1]$}};
\draw[red, >=stealth,->, line width=2pt] (b) -- (a.four south);
\node (d) [above=1.1cm of a.seven north] {\textcolor{blue}{$A[j]$}};
\draw[blue, >=stealth,->, line width=2pt] (d) -- (a.seven north);
\node (c) [above=2cm of a] {找到了从4往右最后一个大于4的number=5};
\end{tikzpicture}
\end{figure}
%swap
\begin{figure}[H]
\begin{tikzpicture}
[mynode/.style={draw,minimum height=1cm, inner sep=0.5 cm, node distance=0.6cm, fill=gray!20!}]
\node[rectangle split, rectangle split horizontal, rectangle split parts=9] (a)[mynode] {$1$ \nodepart{two} $5$ \nodepart{three} $8$ \nodepart{four} $4$ \nodepart{five} $7$ \nodepart{six} $6$ \nodepart{seven} $5$ \nodepart{eight} $3$ \nodepart{nine} $1$};
\node (b) [below=10mm of a.four south] {\textcolor{red}{$A[i-1]$}};
\draw[red, >=stealth,->, line width=2pt] (b) -- (a.four south);
\node (d) [above=1.1cm of a.seven north] {\textcolor{blue}{$A[j]$}};
\draw[blue, >=stealth,->, line width=2pt] (d) -- (a.seven north);
\node (c) [above=2cm of a] {将4和5的位置进行互换};
\draw [blue,>=stealth,->,line width=2pt,out=105,in=30] (a.seven north) to (a.four north);
\draw [red,>=stealth,->,line width=2pt,out=-15,in=-60] (a.four south) to (a.seven south);
\end{tikzpicture}
\end{figure}
\begin{figure}[H]
\begin{tikzpicture}
[mynode/.style={draw,minimum height=1cm, inner sep=0.5 cm, node distance=0.6cm, fill=gray!20!}]
\node[rectangle split, rectangle split horizontal, rectangle split parts=9] (a)[mynode] {$1$ \nodepart{two} $5$ \nodepart{three} $8$ \nodepart{four} $5$ \nodepart{five} $7$ \nodepart{six} $6$ \nodepart{seven} $4$ \nodepart{eight} $3$ \nodepart{nine} $1$};
\node (b) [below=10mm of a.four south] {\textcolor{red}{$A[i-1]$}};
\draw[red, >=stealth,->, line width=2pt] (b) -- (a.four south);
\node (d) [above=1.1cm of a.seven north] {\textcolor{blue}{$A[j]$}};
\draw[blue, >=stealth,->, line width=2pt] (d) -- (a.seven north);
\end{tikzpicture}
\end{figure}
%Reverse
\begin{figure}[H]
\begin{tikzpicture}
[mynode/.style={draw,minimum height=1cm, inner sep=0.5 cm, node distance=0.6cm, fill=gray!20!}]
\node[rectangle split, rectangle split horizontal, rectangle split parts=9] (a)[mynode] {$1$ \nodepart{two} $5$ \nodepart{three} $8$ \nodepart{four} $5$ \nodepart{five} $7$ \nodepart{six} $6$ \nodepart{seven} $4$ \nodepart{eight} $3$ \nodepart{nine} $1$};
\node (b) [below=10mm of a.four south] {\textcolor{red}{$A[i-1]$}};
\draw[red, >=stealth,->, line width=2pt] (b) -- (a.four south);
\node (c) [above=1cm of a] {Reverse $A[i-1]$ 后面的数字};
\node (d) [below=1.5cm of a.seven south] {Reverse};
\draw[blue, line width=2pt] (a.four split south) ..  controls +(0,-1.5cm) and +(0,1.5cm) .. (d.north);
\draw[blue, line width=2pt] (a.south east) ..  controls +(0,-1.5cm) and +(0,1.5cm) .. (d.north);
\end{tikzpicture}
\end{figure}
\begin{figure}[H]
\begin{tikzpicture}
[mynode/.style={draw,minimum height=1cm, inner sep=0.5 cm, node distance=0.6cm, fill=gray!20!}]
\node[rectangle split, rectangle split horizontal, rectangle split parts=9] (a)[mynode] {$1$ \nodepart{two} $5$ \nodepart{three} $8$ \nodepart{four} $5$ \nodepart{five} $1$ \nodepart{six} $3$ \nodepart{seven} $4$ \nodepart{eight} $6$ \nodepart{nine} $7$};
\node (c) [above=1cm of a] {Next Permutation};
\end{tikzpicture}
\end{figure}

\setcounter{algorithm}{0}
\begin{algorithm}[H]
\caption{Find Next Permutation}
\begin{algorithmic}[1]
\Procedure{NextPermutation}{$A, L$}
\State $\star$ 从右到左检查是否存在$A[i]>A[i-1]$
\State $p:=L$ \Comment At $p$, $A[p] > A[p-1]$
\For{$i := L-1$ \textbf{to} $1$}
\If{$A[i] > A[i-1]$}
\State $p\gets i$ \Comment Found the first descending number
\State \textbf{Break}
\EndIf
\EndFor
\If{$p \neq L_A$} \Comment There exits $A[i] > A[i-1]$
\State $\ast$ 寻找$A[p]$到$A[L-1]$中大于$A[p-1]$的所有数中的最小值。
\State $q:=p$ \Comment $A[q] =\min(A[j]\;|\;A[j] > A[p],\ p\leq j<L)$
\For{$j:=p$ \textbf{to} $L-1$}
\If{$A[j] > A[p-1]$}
\State $q\gets j$ \Comment Update $q$ 
\Else
\State \texttt{break}
\EndIf
\EndFor
\algstore{31algo}
\end{algorithmic}
\end{algorithm}
\begin{algorithm}[H]
\begin{algorithmic}[1]
\algrestore{31algo}
\State $\star$ Swap $A[p-1]$ and $A[q]$
\State $\star$ Reverse $A[p]$ to $A[L-1]$
\Else
\State $\ast$ 不存在next permutation 因为$A$已经是降序排列了
\State $\star$ 直接reverse $A$
\EndIf
\EndProcedure
\end{algorithmic}
\end{algorithm}
\setcounter{lstlisting}{0}
\begin{lstlisting}[style=customc, caption={Next Permutation}]
void nextPermutation( vector<int>& nums )
{
    size_t p = nums.size();

    for( size_t i = nums.size() - 1; i >= 1; --i )
    {
        if( nums[i] > nums[i - 1] )
        {
            p = i;
            break;
        }
    }

    if( p != nums.size() )
    {
        //Exists A[i] > A[i-1]
        //Need to find the minimum number thar is larger than A[p-1]

        auto q = p;

        for( size_t j = p; j < nums.size(); ++j )
        {
            if( nums[j] > nums[p - 1] )
            {
                q = j;
            }
            else
            {
                break;
            }
        }

        swap( nums[q], nums[p - 1] );
        reverse( nums.begin() + p, nums.end() );
    }
    else
    {
        //A is decending
        //just reverse A
        reverse( nums.begin(), nums.end() );
    }

}
\end{lstlisting}