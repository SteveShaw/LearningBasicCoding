\section{27 --- Remove Element}
Given an array $A$ and a value $x$, remove all instances of that value in-place and return the new length.

Do not allocate extra space for another array, you must do this by modifying the input array in-place with $O(1)$ extra memory.

The order of elements can be changed. It doesn't matter what you leave beyond the new length.

\paragraph{Example 1:}
\begin{flushleft}
\textbf{Input}: $A=[3,2,2,3]$, $x=3$

\textbf{Output}: 2

\textbf{Explanation}: 

Your function should return 2 with the first two elements of $A$ being 2. 

It doesn't matter what you leave beyond the returned length.
\end{flushleft}

\paragraph{Example 2:}

\begin{flushleft}
\textbf{Input}: $A = [0,1,2,2,3,0,4,2]$, $x = 2$,

\textbf{Output}: 5

\end{flushleft}

\subsection{Double Pointers}
\begin{itemize}
\item 同样是双指针,遇到与$x$不相等的,写入write指针的位置,然后increments write指针。
\end{itemize}

\setcounter{lstlisting}{0}
\begin{lstlisting}[style=customc, caption={Double Pointers}]
int removeElement( vector<int>& nums, int val )
{
    size_t write = 0;

    size_t read = 0;

    while( read < nums.size() )
    {
        if( nums[read] != val )
        {
            //put to write position
            nums[write] = nums[read];
            ++write;
        }

        ++read;
    }

    return write;
}
\end{lstlisting}