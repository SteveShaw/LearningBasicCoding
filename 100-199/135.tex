\section{135 --- Candy}
There are $N$ children standing in a line. Each child is assigned a rating value.
\par
You are giving candies to these children subjected to the following requirements:
\begin{enumerate}
\item Each child must have at least one candy.
\item Children with a higher rating get more candies than their neighbors.
\item What is the minimum candies you must give?
\end{enumerate}
\paragraph{Example 1:}
\begin{flushleft}
\textbf{Input}: \fcj{[1,0,2]}
\\
\textbf{Output}: 5
\\
\textbf{Explanation}: You can allocate to the first, second and third child with 2, 1, 2 candies respectively.
\end{flushleft}
\paragraph{Example 2:}
\begin{flushleft}
\textbf{Input}: \fcj{[1,2,2]}
\\
\textbf{Output}: 4
\\
\textbf{Explanation}: You can allocate to the first, second and third child with 1, 2, 1 candies respectively. The third child gets 1 candy because it satisfies the above two conditions.
\end{flushleft}
\subsection{Brute Force}
这是最直接的方法,首先创建一个1维数组$C$用来跟踪分配的candies。
\begin{itemize}
\item 首先,给每个student一个candy。然后从左至右扫描$C$。对于每个位置,如果其rating $R[i]$ 比左边的大,即$R[i]>R[i-1]$,并且$C[i]\leq C[i-1]$,那么就要update $C[i]$为$C[i-1]+1$。
\item 同样的,我们也需要检查其rating是否比右边的大,即$R[i]>R[i+1]$,如果是,并且$C[i] \leq C[i+1]$,那么$C[i]$还要被再次update为$C[i+1]+1$。
\item 继续上述步骤直到没有发生任何update $C[i]$的操作。
\item 最后将数组$C$中的元素加起来就是最终的结果。
\end{itemize}
\subsection{Using One Array}
上述Brute Force的方法虽然直接,但是效率不高。为了提高效率,首先用两个1维数组$L$ and $R$。$L$用来保存从左至右所需要的candies的数量,即如果当前student的rating比其左边的高,他应该得到比左边更多的candies。类似的,$R$就是保存从右至左所需要的candies的数量,即如果当前student的rating比其右边的高,他应该得到比右边更多的candies。
\par
开始时,$L$ and $R$中都是1, 即每个student都是1个candy。首先从左到右遍历ratings数组$A$,在当前位置$i$,如果$A[i]>A[i-1]$,那么$L[i]\gets L[i-1]+1$。因为$L[i]$在update前总是小于或者等于$L[i-1]$的。
\par
接着从右到左遍历$R$,类似的在当前位置$i$,如果$A[i]>A[i+1]$,那么$R[i]\gets R[i+1]+1$。
\par
这时候,在位置$i$处,需要得到$\max(L[i],R[i])$以满足左右两侧rating的大小关系。因此最后的minimum number of candies就是
\[
\sum_{0}^{|A|-1}\max(L[i], R[i])
\]
不过在从右到左时,可以不用$R$数组,直接更新$L[i]$为$\max(L[i], L[i+1]+1)$即可。这样就只需要一个数组了。

\setcounter{lstlisting}{0}
\begin{lstlisting}[style=customc, caption={Bidirectional Scanning}]
int candy( vector<int>& ratings )
{
    auto L( static_cast< int >( ratings.size() ) );
    vector<int> candies( ratings.size(), 1 );
    //scan from left to right
    //to check left side of each person
    for( auto i = 1; i < L; ++i )
    {
        if( ratings[i] > ratings[i - 1] )
        {
            //person i need more candies than person i-1
            candies[i] = candies[i - 1] + 1;
        }
    }
    //scan from right to left
    //to check right side of each person
    auto ans( 0 );
    for( auto i = L - 2; i >= 0; --i )
    {
        if( ratings[i] > ratings[i + 1] )
        {
            //person i need more candies than person i+1
            candies[i] = ( max )( candies[i], candies[i + 1] + 1 );
        }
        ans += candies[i];
    }
    //we have not added the last person's candies.
    ans += candies.back();
    return ans;
}
\end{lstlisting}

\subsection{One Pass With Constant Space}
上述方法仍然需要线性空间。如果不需要线性空间,那么首先需要观察到如下信息:
\begin{itemize}
\item 分配candies都是按1递增进行的。因为目标是最少的candies。
\item 分配candies,local minimum处的student得到的candy必须是1。
\end{itemize}
因此分配candy要么是以$1,2,3,\ldots,n$的形式,或者是$n, n-1, \ldots, 2, 1$的形式,这两种形式的和都是$n\times(n+1)/2$。

因此我们可以将ratings array看作是由一些上坡和下坡部分构成的。当在上坡的时候,candy的分配就是$1,2,3,\ldots,n$的形式。类似的,下坡的分配形式就是$k, k-1, \ldots, 2,1$。现在的问题就是local maximum或者说local peak到底应该包含在上坡或者下坡的哪一个中?很明显,为了同时大于左右两侧的candy数量,local peak所赋予的candy数必须由其所在上坡和下坡中的长度最长的那个来决定。而对于local valley即波谷,我们把它包含在下一个上坡段中,其所assign的candy数量必当为1( which can be subtracted from the next slope's count calculation )

在实现时, maintain当前段和上一段的类型即是上坡还是下坡。另外keep track上坡和下坡的长度,即所包含的element的个数。当遇到如下情形时,
\begin{enumerate}
\item 一段上坡,然后紧接着是相等的rating
\item 一段下坡,然后紧接着是一段上坡
\item 一段下坡,然后紧接着是相等的rating
\end{enumerate}
这时候,就需要更新当前当前所分配的candy的总的数量。


\begin{lstlisting}[style=customc, caption={Up And Down Slope}]
int candy( vector<int>& ratings )
{
    if( ratings.size() <= 1ULL )
    {
        return ratings.empty() ? 0 : 1;
    }
    int candies = 0;
    int up_count = 0;
    int down_count = 0;
    unsigned char last_slope( 0 );
    //helper lambad to get (1+..+count)
    auto sum = []( int count )
    {
        return ( count * ( count + 1 ) ) / 2;
    };
    for( auto i = 1ull; i < ratings.size(); ++i )
    {
        unsigned char cur_slope( 0 );
        if( ratings[i] > ratings[i - 1] )
        {
            //up slope
            cur_slope = 1;
        }
        else if( ratings[i] < ratings[i - 1] )
        {
            //down slope
            cur_slope = 2;
        }

        //find end of mountain point
        //a mountain contains a up slope and a down slope
        //or a up slope with leveling
        bool end_mountain = false;
        if( ( last_slope == 1 ) && ( cur_slope == 0 ) )
        {
            //up and then level
            end_mountain = true;
        }
        else if( ( last_slope == 2 ) && ( cur_slope < 2 ) )
        {
            //down and then level or up
            end_mountain = true;
        }

        if( end_mountain )
        {
            //the candies in each mountain
            //will have candies for up slope and down slope
            //and the candies for the local peak which is ( max )( up_count, down_count ) + 1
            //but we ignore 1 here because we the end of moutain
            //is local valley and it will be used as the next start of mountain
            candies += sum( up_count ) + sum( down_count ) + ( max )( up_count, down_count );
            up_count = 0;
            down_count = 0;
        }
        switch( cur_slope )
        {
        case 1:
            //this is up slope, increments the length of up counts
            ++up_count;
            break;

        case 2:
            //this is down slope, increments the length of down counts
            ++down_count;
            break;

        case 0:
            //this is leveling, just increments the candies
            ++candies;
            break;
        }

        last_slope = cur_slope;

    } //end for
    //process last segment containing turning point
    //in this case, we have to add 1 to ( max )( up_count, down_count )
    //because this end of mountain is the final local valley
    candies += sum( up_count ) + sum( down_count ) + ( max )( up_count, down_count ) + 1;
    return candies;
}
\end{lstlisting}
