\section{150 --- Evaluate Reverse Polish Notation}
Evaluate the value of an arithmetic expression in Reverse Polish Notation.
\par
Valid operators are $+$, $-$, $\times$, $/$. Each operand may be an integer or another expression.
\paragraph{Note:}
\begin{itemize}
    \item Division between two integers should truncate toward zero.
    \item The given \texttt{RPN} expression is always valid. That means the expression would always evaluate to a result and there won't be any divide by zero operation.
\end{itemize}
\paragraph{Example 1:}
\begin{flushleft}
\textbf{Input}: $[2, 1, +, 3, \times]$
\\
\textbf{Output}: 9
\textbf{Explanation}: $((2 + 1) \times 3) = 9$
\end{flushleft}
\paragraph{Example 2:}
\begin{flushleft}
\textbf{Input}: $[4, 13, 5, /, +]$
\\
\textbf{Output}: 6
\\
\textbf{Explanation}: $(4 + (13 / 5)) = 6$
\end{flushleft}
\paragraph{Example 3:}
\begin{flushleft}
\textbf{Input}: $[10, 6, 9, 3, +, -11, \times, /, \times, 17, +, 5, +]$
\\
\textbf{Output}: 22
\\
\textbf{Explanation}:
\begin{align*}
& ((10 \times (6 / ((9 + 3) \times -11))) + 17) + 5 \\
&= ((10 \times (6 / (12 \times -11))) + 17) + 5 \\
&= ((10 \times (6 / -132)) + 17) + 5 \\
&= ((10 \times 0) + 17) + 5\\
&= (0 + 17) + 5\\
&= 17 + 5\\
&= 22\\
\end{align*}
\end{flushleft}
算法相对比较简单,用一个stack保存遇到的数字即数字运算的结果,步骤如下:
\begin{enumerate}
    \item 遍历字符串数组,遇到数字入栈
    \item 如果是操作符,则连续将栈顶的两个元素弹出,注意,对于减法和除法,第二个弹出的是左操作数,第一个弹出的是右操作数,将计算得到的结果压入栈中。
    \item 遍历完后,返回栈顶元素即为表达式结果。
\end{enumerate}
\setcounter{lstlisting}{0}
\begin{lstlisting}[style=customc, caption={Stack Based Approach}]
int evalRPN(vector<string>& tokens)
{
	stack<int> stk;
	
	for (const auto& s : tokens)
	{
		if (s[0] >= '0' && s[0] <= '9') // positive integers
		{
			int val = stoi(s);
			stk.push(val);
		}
		else if (s[0] == '-' && s.size() > 1) // negative integers with minus symbol
		{
			int val = stoi(s);
			stk.push(val);
		}
		else
		{
			int op1 = stk.top();
			stk.pop();
			int op2 = stk.top();
			stk.pop();

			switch (s[0])
			{
			case '+':
				stk.push(op1 + op2);
				break;

			case '-':
				stk.push(op2 - op1);
				break;

			case '/':
				stk.push(op2 / op1);
				break;

			case '*':
				stk.push(op1 * op2);
				break;
			}
		}
	}

	return stk.top();
}
\end{lstlisting}