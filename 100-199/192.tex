\section{192 --- Word Frequency}
Write a bash script to calculate the frequency of each word in a text file words.txt.
\par
For simplicity sake, you may assume:
\begin{itemize}
\item words.txt contains only lowercase characters and space characters (\textvisiblespace).
\item Each word must consist of lowercase characters only.
Words are separated by one or more whitespace characters.
\end{itemize}
\paragraph{Example:}
\begin{flushleft}
\begin{mdframed}[style=mymdf]
the day is sunny the the
\\
the sunny is is
\end{mdframed}
Your script should output the following, sorted by descending frequency:
\begin{mdframed}[style=mymdf]
the 4
\\
is 3
\\
sunny 2
\\
day 1
\end{mdframed}
\end{flushleft}
\paragraph{Note:}
\begin{itemize}
\item Don't worry about handling ties, it is guaranteed that each word's frequency count is unique.
\item Could you write it in one-line using Unix pipes?
\end{itemize}
\subsection{Solution}
\begin{mdframed}[style=mymdf]
\begin{lstlisting}[style=customc]
cat words.txt | tr -s ' ' '\n' | sort | uniq -c | sort -r | awk '{ print $2, $1 }'
\end{lstlisting}
\end{mdframed}
\begin{itemize}
\item \textbf{tr -s}: truncate the string with target string, but only remaining one instance (e.g. multiple whitespaces)
\item \textbf{sort}: To make the same string successive so that \texttt{uniq} could count the same string fully and correctly.
\item \textbf{uniq -c}: \texttt{uniq} is used to filter out the repeated lines which are successive, \texttt{-c} means counting
\item \textbf{sort -r}: \texttt{-r} means sorting in descending order
\item \texttt{awk} is used to format output
\end{itemize}
