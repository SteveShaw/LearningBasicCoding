\section{199. Binary Tree Right Side View}
Given a binary tree, imagine yourself standing on the right side of it, return the values of the nodes you can see ordered from top to bottom.
\paragraph{Example:}
\begin{flushleft}
\textbf{Input}:
\begin{figure}[H]
\begin{tikzpicture}
[mynode/.style={draw,circle,minimum size=7mm, fill=gray!20!}]
\node(){};
\node[mynode](1) {1};
\node[mynode](2)[below=8mm of 1, xshift=-8mm] {2};
\node[mynode](3)[below=8mm of 1, xshift=8mm] {3};
\node[mynode](4)[below=8mm of 3, xshift=8mm] {4};
\node[mynode](5)[below=8mm of 2, xshift=8mm] {5};
\draw (1) -- (2);
\draw (1) -- (3);
\draw (2) -- (5);
\draw (3) -- (4);
\end{tikzpicture}
\end{figure}
\textbf{Output}: $[1,3,4]$
\end{flushleft}
\subsection{Breath First Search}
给每一个node进行编号,同时记录node所在的层级编号。
\begin{lstlisting}[style=customc]
vector<int> rightSideView( TreeNode* root )
{
    if( !root )
    {
        return {};
    }

    vector<pair<int, int>> v;

    queue<tuple<TreeNode*, int, int>> q;

    q.emplace( root, 0, 0 );

    while( !q.empty() )
    {
        auto t = q.front();
        q.pop();

        auto node = get<0>( t );

        int index = get<1>( t ); //The index of current node
        int l = get<2>( t ); //The level of current node

        if( v.empty() || v.size() <= static_cast<size_t>( l ) )
        {
            v.emplace_back( index, node->val );
        }
        else
        {
            if( index > v[l].first )
            {
                v[l].second = node->val;
            }
        }

        if( node->left )
        {
            //index *2 +1 is the index of left child
            q.emplace( node->left, index * 2 + 1, l + 1 );
        }

        if( node->right )
        {
            //index *2 +2 is the index of right child
            q.emplace( node->right, index * 2 + 2, l + 1 );
        }
    }

    vector<int> ans;
    ans.reserve( v.size() );

    for( const auto& p : v )
    {
        ans.push_back( p.second );
    }

    return ans;
}
\end{lstlisting}