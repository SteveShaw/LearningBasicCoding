\section{199. Binary Tree Right Side View}
Given a binary tree, imagine yourself standing on the right side of it, return the values of the nodes you can see ordered from top to bottom.
\paragraph{Example:}
\begin{flushleft}
\textbf{Input}:
\begin{figure}[H]
\begin{tikzpicture}
[mynode/.style={draw,circle,minimum size=7mm, fill=gray!20!}]
\node(){};
\node[mynode](1) {1};
\node[mynode](2)[below=8mm of 1, xshift=-8mm] {2};
\node[mynode](3)[below=8mm of 1, xshift=8mm] {3};
\node[mynode](4)[below=8mm of 3, xshift=8mm] {4};
\node[mynode](5)[below=8mm of 2, xshift=8mm] {5};
\draw (1) -- (2);
\draw (1) -- (3);
\draw (2) -- (5);
\draw (3) -- (4);
\end{tikzpicture}
\end{figure}
\textbf{Output}: $[1,3,4]$
\end{flushleft}
\subsection{Breath First Search}
Typical application of BFS with level traversing.
\begin{lstlisting}[style=customc, caption={BFS}]
vector<int> rightSideView( TreeNode* root )
{
    if( !root )
    {
        return {};
    }
    queue<TreeNode*> q;
    q.push( root );
    vector<int> ans;
	//BFS
    while( !q.empty() )
    {
        auto sz( q.size() );
		//record the rightmost node's value
        auto rightmost( 0 );
		//level traverse
        for( auto i = 0ULL; i < sz; ++i )
        {
            auto t = q.front();
            q.pop();
            if( t->left )
            {
                q.push( t->left );
            }
            if( t->right )
            {
                q.push( t->right );
            }
			//the last updated value
			//is the one we need
            rightmost = t->val;
        }
		//add the value to the output
        ans.push_back( rightmost );
    }
    return ans;
}

\end{lstlisting}