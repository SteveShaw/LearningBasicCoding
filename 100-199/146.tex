\section{146 --- LRU Cache}
Design and implement a data structure for \textcolor{red}{Least Recently Used (LRU)} cache. It should support the following operations: \fcj{get} and \fcj{put}.
\begin{itemize}
    \item \fcj{get(key)} --- Get the value (will always be positive) of the key if the key exists in the cache, otherwise return $-1$.
    \item \fcj{put(key, value)} --- Set or insert the value if the key is not already present. When the cache reached its capacity, it should invalidate the least recently used item before inserting a new item.
\end{itemize}
\paragraph{Example:}
\begin{flushleft}
\fcj{LRUCache cache = new LRUCache( 2 /* capacity */ );}

\fcj{cache.put(1, 1);}

\fcj{cache.put(2, 2);}

\fcj{cache.get(1);       // returns 1}

\fcj{cache.put(3, 3);    // evicts key 2}

\fcj{cache.get(2);       // returns -1 (not found)}

\fcj{cache.put(4, 4);    // evicts key 1}

\fcj{cache.get(1);       // returns -1 (not found)}

\fcj{cache.get(3);       // returns 3}

\fcj{cache.get(4);       // returns 4}


\end{flushleft}
\paragraph{Follow up:}
\begin{itemize}
    \item Could you do both operations in $O(1)$ time complexity?
\end{itemize}

\subsection{Hash Table Plus Double Linked List}
为了能够在$O(1)$的时间复杂度情况下,实现这两个函数,首先需要用一个hash table来建立key和value的映射。但是仅有这个还不够,为了能够在$O(1)$时间内获取LRU item,必须借助链表,将LRU item放置在链表队尾,而刚刚进行 \fcj{get} 和 \fcj{put} 操作的key则要放在链表的队头。
\begin{itemize}
    \item \fcj{get}: 首先看查询的key是否在hash table中,如果不存在,返回$-1$。否则,则把这个key和其对应的value从链表中删除,然后放置在链表的头位置。
    \item \fcj{put}: 首先看要放入的key是否在hash table中,
    \begin{enumerate}
        \item 如果存在,则同样把这个key和value组成的元素从链表中删除,然后放置在链表的头位置,然后退出。
        \item 反之,则将这个key和对应的value插入到hash table中,同时,在链表的头位置插入这对key和value组成的元素。同时检查是否超出允许的最大个数,如果超出了,将list队尾的key从hash table中删除,然后删除队尾元素。
    \end{enumerate}
\end{itemize}

\setcounter{lstlisting}{0}
\begin{lstlisting}[style=customc, caption={Hash Table With Linked List}]
class LRUCache
{
public:
    LRUCache( int capacity )
    {
        max_c = capacity;
    }
    int get( int key )
    {
        auto it = m.find( key );
        if( it == m.end() )
        {
            return -1;
        }
        //remove the (key,value) and put into the
        //begin of the list
        cache.splice( cache.begin(), cache, it->second );
        return it->second->second;
    }
    void put( int key, int value )
    {
        auto it = m.find( key );
        //this key exists
        if( it != m.end() )
        {
            //remove from the hash map
            //since the position in the list
            //is changed
            cache.erase( it->second );
        }
        //add (key,value) to the begin
        cache.emplace_front( key, value );
        m[key] = cache.begin();
        if( m.size() > max_c )
        {
            //remove the LRU
            int remove_key = cache.back().first;
            m.erase( remove_key );
            cache.pop_back();
        }
    }
    int max_c;
    list<pair<int, int>> cache;
    unordered_map<int, list<pair<int, int>>::iterator> m;
};
\end{lstlisting}
