\section{146 --- LRU Cache}
Design and implement a data structure for \textcolor{red}{Least Recently Used (LRU)} cache. It should support the following operations: \textcolor{red}{get} and \textcolor{red}{put}.
\begin{itemize}
    \item \texttt{get}(key) --- Get the value (will always be positive) of the key if the key exists in the cache, otherwise return $-1$.
    \item \texttt{put}(key, value) --- Set or insert the value if the key is not already present. When the cache reached its capacity, it should invalidate the least recently used item before inserting a new item.
\end{itemize}
\paragraph{Example:}
\begin{flushleft}
\begin{lstlisting}[style=customc]
LRUCache cache = new LRUCache( 2 /* capacity */ );

cache.put(1, 1);
cache.put(2, 2);
cache.get(1);       // returns 1
cache.put(3, 3);    // evicts key 2
cache.get(2);       // returns -1 (not found)
cache.put(4, 4);    // evicts key 1
cache.get(1);       // returns -1 (not found)
cache.get(3);       // returns 3
cache.get(4);       // returns 4
\end{lstlisting}
\end{flushleft}
\paragraph{Follow up:}
\begin{itemize}
    \item Could you do both operations in $O(1)$ time complexity?
\end{itemize}
\subsection{Hash Table Plus Double Linked List}
为了能够在$O(1)$的时间复杂度情况下,实现这两个函数,首先需要用一个hash table来建立key和value的映射。但是仅有这个还不够,为了能够在$O(1)$时间内获取LRU item,必须借助链表,将LRU item放置在链表队尾,而刚刚进行\textcolor{red}{get}和\textcolor{red}{put}操作的key则要放在链表的队头。
\begin{itemize}
    \item \textcolor{red}{get}: 首先看查询的key是否在hash table中,如果不存在,返回$-1$。否则,则把这个key和其对应的value从链表中删除,然后放置在链表的头位置。
    \item \textcolor{red}{put}: 首先看要放入的key是否在hash talbe中,
    \begin{enumerate}
        \item 如果存在,则同样把这个key和value组成的元素从链表中删除,然后放置在链表的头位置,然后退出。
        \item 反之,则将这个key和对应的value插入到hash table中,同时,在链表的头位置插入这对key和value组成的元素。同时检查是否超出允许的最大个数,如果超出了,将list队尾的key从hash table中删除,然后删除队尾元素。
    \end{enumerate}
\end{itemize}
\begin{lstlisting}[style=customc]
class LRUCache {
public:
    LRUCache(int capacity) {
        max_c = capacity;
    }
    
    int get(int key) {
        
        auto it = m.find(key);
        if(it==m.end())
        {
            return -1;
        }
        
        cache.splice(cache.begin(), cache, it->second);
        return it->second->second;
    }
    
    void put(int key, int value) {
        
        auto it = m.find(key);
        
        if(it!=m.end())
        {
            cache.erase(it->second);
        }
        
        cache.emplace_front(key, value);
        m[key] = cache.begin();
        
        if(m.size()> max_c)
        {
            int remove_key = cache.back().first;
            m.erase(remove_key);
            cache.pop_back();
        }
    }
    
    int max_c;
		
	list<pair<int,int>> cache;
		
	unordered_map<int, list<pair<int,int>>::iterator> m;
};
\end{lstlisting}