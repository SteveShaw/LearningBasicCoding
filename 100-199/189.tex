\section{189 --- Rotate Array}
Given an array $A$, rotate the array to the right by $k$ steps, where $k$ is non-negative.
\paragraph{Example 1:}
\begin{flushleft}
\textbf{Input}: $[1,2,3,4,5,6,7]$ and $k = 3$
\\
\textbf{Output}: $[5,6,7,1,2,3,4]$
\\
\textbf{Explanation}:
\\
rotate 1 steps to the right: $[7,1,2,3,4,5,6]$
\\
rotate 2 steps to the right: $[6,7,1,2,3,4,5]$
\\
rotate 3 steps to the right: $[5,6,7,1,2,3,4]$
\end{flushleft}
\paragraph{Example 2:}
\begin{flushleft}
\textbf{Input}: $[-1,-100,3,99]$ and $k = 2$
\\
\textbf{Output}: $[3,99,-1,-100]$
\\
\textbf{Explanation}:
\\
rotate 1 steps to the right: $[99,-1,-100,3]$
\\
rotate 2 steps to the right: $[3,99,-1,-100]$
\end{flushleft}
\paragraph{Note:}
\begin{itemize}
\item Try to come up as many solutions as you can, there are at least 3 different ways to solve this problem.
\item Could you do it in-place with $O(1)$ extra space?
\end{itemize}
\subsection{Cyclic Replacement}
Start from one element and keep rotating until we have rotated n different elements.
\setcounter{algorithm}{0}
\begin{algorithm}[H]
\caption{Cyclic Replacement}
\begin{algorithmic}[1]
\Procedure{Rotate}{$A, L, k$}
\State $\delta:=0$ \Comment The number of elements have been shifted
\State $x:=0$ \Comment The start index of elements to be shifted
\State $y:=x$ \Comment The current index of element that is shifted
\algstore{189algo}
\end{algorithmic}
\end{algorithm}
\begin{algorithm}[H]
\begin{algorithmic}[1]
\algrestore{189algo}
\State $z:=0$ \Comment The variable used as saving number to be shifted
\While{$\delta < n$}
\State $n:=A[x]$ \Comment The first number that will be shifted in the following loop
\Repeat
\State $y:=(y+k)\bmod L$ \Comment The next index that place $n$
\State $z:=A[y]$ \Comment Save the value to $z$
\State $A[y]\gets n$ \Comment Replace $A[y]$ by $n$
\State $n\gets z$ \Comment The value will replace next element
\State $\delta\gets\delta+1$ \Comment Increments the number of elements shifted
\Until{$y\neq x$}
\EndWhile
\State $x\gets x+1$ \Comment Move to next element to be shifted
\State $y\gets x$
\EndProcedure
\end{algorithmic}
\end{algorithm}
\setcounter{lstlisting}{0}
\begin{lstlisting}[style=customc, caption={Cyclic Replacement Code}]
void rotate( vector<int>& nums, int k )
{
    if( nums.empty() )
    {
        return;
    }

    if( k == 0 )
    {
        return;
    }

    int L = static_cast<int>( nums.size() );

    int x = 0;
    int y = x;

    int delta = 0;

    int z = -1;

    while( delta < L )
    {
        int n = nums[x];
        do
        {
            y = ( y + k ) % L;
            z = nums[y];
            nums[y] = n;

            n = z;
            ++delta;
        }
        while( y != x );

        ++x;
        y = x;
    }
}
\end{lstlisting}
\subsection{Reverse}
Reverse the first $n-k$ elements, then reverse last $k$ elements, finally reverse whole array.
\subsection{Swap}
Swap the last $k$ elements with the first $k$ elements.
\begin{algorithm}[H]
\caption{Swap}
\begin{algorithmic}[1]
\Procedure{Rotate}{$A,L,k$}
\State $\theta:=0$ \Comment The start index
\State $\ell:=L$
\State $\kappa:=k$
\While{$\ell\neq0$ \textbf{and} $\kappa\neq 0$}
\State $\kappa\gets \kappa\bmod L$
\For{$i:=0\to \kappa-1$}
\State \Call{swap}{$A[\theta + i]$, $A[\theta + \ell - \kappa + i]$}
\State $\theta\gets \theta + \kappa$ \Comment Change the start index
\State $\ell\gets \ell - \kappa$ \Comment Change the length of numbers to be shifted
\EndFor
\EndWhile
\EndProcedure
\end{algorithmic}
\end{algorithm}
\begin{lstlisting}[style=customc, caption={Swap}]
void rotate( vector<int>& nums, int k )
{
    int theta = 0;
    int ell = static_cast<int>( nums.size() );
    int kappa = k;

    while( ( ell != 0 ) && ( kappa != 0 ) )
    {
        kappa = kappa % ell;

        for( int i = 0; i < kappa; ++i )
        {
            swap( nums[theta + i], nums[theta + ell - kappa + i] );
        }

        theta += kappa;
        ell -= kappa;
    }
}
\end{lstlisting}