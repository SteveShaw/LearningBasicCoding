\section{189 --- Rotate Array}
Given an array $A$, rotate the array to the right by $k$ steps, where $k$ is non-negative.
\paragraph{Example 1:}
\begin{flushleft}
\textbf{Input}: $[1,2,3,4,5,6,7]$ and $k = 3$
\\
\textbf{Output}: $[5,6,7,1,2,3,4]$
\\
\textbf{Explanation}:
\\
rotate 1 steps to the right: $[7,1,2,3,4,5,6]$
\\
rotate 2 steps to the right: $[6,7,1,2,3,4,5]$
\\
rotate 3 steps to the right: $[5,6,7,1,2,3,4]$
\end{flushleft}
\paragraph{Example 2:}
\begin{flushleft}
\textbf{Input}: $[-1,-100,3,99]$ and $k = 2$
\\
\textbf{Output}: $[3,99,-1,-100]$
\\
\textbf{Explanation}:
\\
rotate 1 steps to the right: $[99,-1,-100,3]$
\\
rotate 2 steps to the right: $[3,99,-1,-100]$
\end{flushleft}
\paragraph{Note:}
\begin{itemize}
\item Try to come up as many solutions as you can, there are at least 3 different ways to solve this problem.
\item Could you do it in-place with $O(1)$ extra space?
\end{itemize}
\subsection{Cyclic Replacement}
Start from one element and keep rotating until we have rotated $n$ different elements.
\setcounter{lstlisting}{0}
\begin{lstlisting}[style=customc, caption={Cyclic Replacement Code}]
void rotate( vector<int>& nums, int k )
{
    if( nums.empty() )
    {
        return;
    }
    auto L = static_cast<int>( nums.size() );
    k = ( k % L );
    //this approach will throw exception
    //when k=0,so we check if k is zero first
    if( k == 0 )
    {
        return;
    }
    //
    int last = 0;
    int cur = nums[0];
    //current index to be rotated
    int rot_idx = 0;
    //the original index
    auto anchor( rot_idx );
    //rotate each index until hit its original index
    for( int i = 0; i < L; ++i )
    {
        last = cur;
        //move to rotated position
        rot_idx = ( rot_idx + k ) % L;
        cur = nums[rot_idx];
        nums[rot_idx] = last;
        //check if hit original index
        if( rot_idx == anchor )
        {
            //try rotate next index
            ++rot_idx;
            anchor = rot_idx;
            cur = nums[rot_idx];
        }
    }
}
\end{lstlisting}
\subsection{Reverse}
Reverse the first $n-k$ elements, then reverse last $k$ elements, finally reverse whole array.

\subsection{Swap}
Swap the last $k$ elements with the first $k$ elements.

\begin{lstlisting}[style=customc, caption={Swap}]
void rotate( vector<int>& nums, int k )
{
    auto start( 0 );
    auto L( static_cast<int>( nums.size() ) );
    while( L && k )
    {
        k = ( k % L );
        //swap first k elements with last k elements
        for( int i = 0; i < k; ++i )
        {
            swap( nums[start + i], nums[start + ( L - k + i )] );
        }
        //shrink the range to be swapped
        //decrease the range length
        L -= k;
        //increase the range start
        start += k;
    }
}
\end{lstlisting}