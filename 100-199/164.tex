\section{164 --- Maximum Gap}
Given an unsorted array $A$, find the maximum difference between the successive elements in its sorted form.
\par
Return 0 if the array contains less than 2 elements.
\paragraph{Example 1:}
\begin{flushleft}
\textbf{Input}: $[3,6,9,1]$
\\
\textbf{Output}: 3
\\
\textbf{Explanation}: The sorted form of the array is $[1,3,6,9]$, either $(3,6)$ or $(6,9)$ has the maximum difference 3.
\end{flushleft}
\paragraph{Example 2:}
\begin{flushleft}
\textbf{Input}: $[10]$
\\
\textbf{Output}: 0
\\
\textbf{Explanation}: The array contains less than 2 elements, therefore return 0.
\end{flushleft}
\paragraph{Note:}
\begin{itemize}
    \item You may assume all elements in the array are non-negative integers and fit in the 32-bit signed integer range.
    \item Try to solve it in linear time/space.
\end{itemize}


\subsection{Bucket Sort}
Suppose for each of the $n$ elements in our array, there was a bucket. Then each element would occupy one bucket. Now If we reduced the number of buckets, some buckets would have to accommodate more than one element.

Suppose all elements of the array are sorted and have a uniform gap between them. This means every adjacent pair of elements differ by the same constant value. So for $n$ elements of the array, there are $n−1$ gaps with width, say $t$. It is trivial to deduce that $t=(\max−\min)/(n−1)$ (where $\max$ and $\min$ are the minimum and maximum elements of the array). This width is the maximal width/gap between two adjacent elements in the array!

Actually, $t$ is the smallest value for any array with the same number of elements (i.e. $n$) and the same range (i.e. ($\max−\min$)). We can start with a uniform width array (as described above) and try to reduce the gap between any two adjacent elements. If the gap is reduced between $A[i−1]$ and $A[i]$ to some value $t−\delta$, then you will notice that the gap between $A[i]$ and $A[i+1]$ would have increased to $t+\delta$. Hence the maximum attainable gap would have become $t+p$ from $t$. Thus the value of the maximum gap $t$ can only increase.

From the Pigeonhole Principle, if we used buckets instead of individual elements for comparison, the number of comparisons would reduce if we could accommodate more than one element in a single bucket. In this way, we only had to compare among the buckets. 

To make sure we only compare between buckets, we have to ensure that the gap between the buckets represent the maximal gap in the input array. We could do that just by setting the gap inside any bucket is smaller than $t=(\max−\min)/(n−1)$ (as described above). Since the gaps (between elements) within the same bucket would only be $\leq t$, we could deduce that the maximal gap would indeed occur only between two adjacent buckets.

Hence by setting bucket size $b$to be $1<b\leq(\max−\min)/(n−1)$, we can ensure that at least one of the gaps between adjacent buckets would serve as the maximal gap.

\setcounter{lstlisting}{0}
\begin{lstlisting}[style=customc, caption={Bucket Sort}]
int maximumGap( vector<int>& nums )
{
    if( nums.empty() || ( nums.size() < 2 ) )
    {
        return 0;
    }
    //1. find maximum and minimum of nums
    //to get the bucket size
    auto p = minmax_element( begin( nums ), end( nums ) );
    const auto& [p_min, p_max] = p;
    //2. compute the bucket size and number of buckets
    auto L = static_cast< int >( nums.size() );
    int bucket_size = ( *p_max - *p_min ) / ( L - 1 );
    bucket_size = ( max )( 1, bucket_size );
    //3. fill buckets
    int num_buckets = ( *p_max - *p_min ) / bucket_size + 1;
    vector<array<int, 3>> buckets( num_buckets, array<int, 3> { *p_max, * p_min, 0 } );
    for( int n : nums )
    {
        int bucket_index = ( n - *p_min ) / bucket_size;
        auto& bucket = buckets[bucket_index];
        //indicate this bucket is used
        bucket[2] = 1;
        //update minimum inside the bucket
        bucket[0] = ( min )( n, bucket[0] );
        //update maximum inside the bucket
        bucket[1] = ( max )( n, bucket[1] );
    }
    //4. get the maximum gap
    int last_max = *p_min;
    int max_gap = 0;
    for( const auto& bucket : buckets )
    {
        if( !bucket[2] )
        {
            //this bucket is not used
            continue;
        }
        //get the gap between buckets
        max_gap = ( max )( max_gap, bucket[0] - last_max );
        //update last_max
        last_max = bucket[1];
    }
    return max_gap;
}
\end{lstlisting}