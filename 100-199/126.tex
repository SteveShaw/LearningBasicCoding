\section{126 --- Word Ladder II}
Given two words (\fcj{beginWord} and \fcj{endWord}), and a dictionary's word list, find all shortest transformation sequence(s) from \fcj{beginWord} to \fcj{endWord}, such that:
\begin{itemize}
\item Only one letter can be changed at a time
\item Each transformed word must exist in the word list. Note that \fcj{beginWord} is not a transformed word.
\end{itemize}
\paragraph{Note:}
\begin{itemize}
\item Return an empty list if there is no such transformation sequence.
\item All words have the same length.
\item All words contain only lowercase alphabetic characters.
\item You may assume no duplicates in the word list.
\item You may assume \fcj{beginWord} and \fcj{endWord} are non-empty and are not the same.
\end{itemize}
\paragraph{Example 1:}
\begin{flushleft}
\textbf{Input}: 

\fcj{beginWord = "hit"}

\fcj{endWord = "cog"}

\fcj{wordList = ["hot","dot","dog","lot","log","cog"]}

\textbf{Output}:

\fcj{["hit","hot","dot","dog","cog"]}

\fcj{["hit","hot","lot","log","cog"]}

\end{flushleft}

\paragraph{Example 2:}
\begin{flushleft}
\textbf{Input}: 

\fcj{beginWord = "hit"}

\fcj{endWord = "cog"}

\fcj{wordList = ["hot","dot","dog","lot","log"]}

\textbf{Output}: 

\fcj{[]}


\textbf{Explanation}: The endWord \fcj{"cog"} is not in wordList, therefore no possible transformation.
\end{flushleft}
\subsection{Bidirectional BFS}
假设同时从$\alpha$和$\beta$出发进行宽度优先扩展,如果可以在两个树的某一层找到相同的word,这时候就能确定最短距离了,即最早产生相同word的层数。以给出的例子说明,从begin word \textbf{hot}出发,其扩展出的BFS tree前三层如下所示
\begin{figure}[H]
\begin{tikzpicture}
[mynode/.style={draw,circle,minimum size=10mm, fill=gray!20!}]
\node(){};
\node[mynode](1) {\scriptsize hit};
\node[mynode](2)[below=8mm of 1] {\scriptsize hot};
\node[mynode](3)[below=8mm of 2, xshift=-8mm] {\scriptsize dot};
\node[mynode](4)[below=8mm of 2, xshift=8mm] {\scriptsize lot};
\draw[>=stealth,->] (1) -- (2);
\draw[>=stealth,->] (2) -- (3);
\draw[>=stealth,->] (2) -- (4);
\end{tikzpicture}
\end{figure}
而从end word \textbf{cog}出发,其扩展出的BFS tree 前三层如下所示。
\begin{figure}[H]
\begin{tikzpicture}
[mynode/.style={draw,circle,minimum size=10mm, fill=gray!20!}]
\node(){};
\node[mynode](1) {\scriptsize cog};
\node[mynode](2)[below=8mm of 1, xshift=-8mm] {\scriptsize dog};
\node[mynode](3)[below=8mm of 1, xshift=8mm] {\scriptsize log};
\node[mynode](4)[below=8mm of 2] {\scriptsize dot};
\node[mynode](5)[below=8mm of 3] {\scriptsize lot};
\draw[>=stealth,->] (1) -- (2);
\draw[>=stealth,->] (1) -- (3);
\draw[>=stealth,->] (2) -- (4);
\draw[>=stealth,->] (3) -- (5);
\end{tikzpicture}
\end{figure}
注意在上图中虽然\textbf{dog}也能transform到\textbf{log},但是由于\textbf{log}也同样能从\textit{cog}中得到,而我们是为了得到最短的路径,所以\textbf{log}选择从\textit{cog}变换得到。同理,\textbf{dog}也由\textit{cog}转换得到,而不是从\textbf{log}。这样在具体实现时,我们需要将已经transform得到的word从保存的可transform的word list $A$中删除。这时候可以看到两个word扩展出的树的第三层有相同的word,因此这一层就是最短的路径。

在实现时,用一个hash map记录每个word所transform出的单词,可能是一个也可能是多个,因此这个hash map的value就是一个array。由于我们是从两个方向同时进行DFS,因此每次选择DFS的开始层时,总是选择node最少的层开始,另外就是注意由于是两个方向,而hash map所记录的转换关系是单向的,即是从$\alpha$到$\beta$,所以如果当前处理的方向是从$\beta$到$\alpha$,那么在更新hash map时,要把tranform出的word作为key,而作为transform源头的word放入key对应的array中。

有了最短路径后,我们就得到了每个word及其transform出的word,接下来就可以用递归即DFS的方式构建出path了。
\setcounter{algorithm}{0}
\begin{algorithm}[H]
\caption{Bidirectional BFS Search + DFS Path Building}
\begin{algorithmic}[1]
\Procedure{FindLadders}{$\alpha,\;\beta,\;A$}
\If{$\beta\notin A$} \Comment The end word is not in the word list
\State \Return $\emptyset$ \Comment Cannot reach end word through $A$
\EndIf
\State $A\gets A\setminus \beta$ \Comment Remove end word from $A$
\algstore{126algo}
\end{algorithmic}
\end{algorithm}
\begin{algorithm}[H]
\begin{algorithmic}[1]
\algrestore{126algo}
\State $x:=\emptyset,\ y:=\emptyset$ \Comment Two hash sets to simulate BFS tree
\State $x\gets x + \alpha$ \Comment The forward BFS tree root
\State $y\gets y + \alpha$ \Comment The backward BFS tree root
\State $C:=\emptyset$ \Comment The hash map contains the words and transformed words from each word
\State $b:=\texttt{false}$ \Comment Indicate if the two trees have same words in their respective current level
\State $d:=0$ \Comment Indicate the type of current tree (0 is forward tree and 1 is backward tree)
\While{$x\neq \emptyset$ \textbf{and} $y\neq\emptyset$ and $b:=\texttt{false}$}
\For{$w \in x$}
\State $A\gets A\setminus w$ \Comment Remove transformed words in $x$ from $A$ to avoid repeating search
\EndFor
\For{$w \in y$}
\State $A\gets A\setminus w$ \Comment Remove transformed words in $y$ from $A$ to avoid repeating search
\EndFor
\If{$|x| \leq |y|$} \Comment $x$ has fewer or equal nodes than $y$
\State $\texttt{Process}(x, y, C, A, b, \texttt{forward})$ \Comment Forward direction
\Else 
\State $\texttt{Process}(y, x, C, A, b, \texttt{backward})$ \Comment Backward direction
\EndIf
\EndWhile
\State $P:=\emptyset$ \Comment The building path during recursive
\State $\pi:=\emptyset$ \Comment The final results
\State $\texttt{BuildPath}(\alpha, \beta, C, P, \pi)$ \Comment Build paths from $\alpha$ to $\beta$ from parent/children map $C$
\State \Return $pi$
\EndProcedure
\end{algorithmic}
\end{algorithm}
\texttt{Process}将BFS的tree当前level $L_0$ transform到$A$中允许的words,同时比较另外一个tree的当前Level $L_1$,看是否有相同的word,如果有相同的word,就表示已经找到最短路径的transform path了。如果没有,但是是$A$中的word,就把transform出的word加入到下一层level $L_n$中,最后swap $L_n$和$L$,使得$L$变为下一个level。

两种情况下,都要根据是forward还是barward将word和transform出的word加入到parent--children map $C$中。

最后,如果在$L_0$和$L_1$中找到相同的word,\text{Process}会修改$b$为\texttt{true}。
\begin{algorithm}[H]
\caption{Helper Function To Process BFS Tree}
\begin{algorithmic}[1]
\Function{Process}{$L_0, L_1, C, A, b, \texttt{Direction}$}
\State $L_n:=\emptyset$ \Comment The hash set of next level from $L_0$
\For{$w\in L_0$}
\State $s:=w$ 
\For{$i:=0 \to |s|$} \label{126for02}
\State $e:=s[i]$ \Comment Current character
\For{$g:=\texttt{char}(a)\to \texttt{char}(z)$ \textbf{and} $g\neq e$} \Comment try every letter from `a' to `z' \label{126for01}
\State $s[i]\gets g$
\If{$s\in L_1$} \Comment Found in current level of another tree \label{126if02}
\State $b\gets \texttt{true}$
\If{\texttt{Direction}=\texttt{Forward}} \Comment $L_0$ is current level of forward tree \label{126if01}
\State $C[w]\gets C[w] + s$ \Comment Add $s$ as $w$'s children list
\Else \Comment $L_0$ is current level of backward tree
\State $C[s]\gets C[s] + w$ \Comment Add $w$ as $s$'s children list
\EndIf \Comment End [\ref{126if01}]
\Else \Comment $s$ is not current level of another tree
\If{$b=\texttt{false}$} \Comment No matched word in current level of two trees \label{126if03}
\State $L_n\gets L_n+s$ \Comment Add $s$ to next level
\If{\texttt{Direction}=\texttt{Forward}} 
\State $C[w]\gets C[w] + s$ 
\Else 
\State $C[s]\gets C[s] + w$ 
\EndIf 
\EndIf \Comment End [\ref{126if03}]
\EndIf \Comment End [\ref{126if02}]
\EndFor \Comment End [\ref{126for01}]
\State $s[i]\gets e$ \Comment Restore $s$ to original word 
\EndFor \Comment End [\ref{126for02}]
\EndFor
\State $\texttt{swap}(L_0, L_n)$ \Comment Update $L_0$ as its next level
\EndFunction
\end{algorithmic}
\end{algorithm}
\texttt{BuildPath}则根据parent--children map $C$ 以及给定的begin word $w$和end word $\beta$,构建出所有可能的path,并将生成的path加入到最终结果$\pi$中。初始时,$w=\alpha$
\begin{algorithm}[H]
\caption{DFS Build Path}
\begin{algorithmic}[1]
\Function{BuildPath}{$w, \beta, P, C, \pi$}
\If{$w:=\beta$} \Comment Reach the end
\State $\pi\gets \pi + P$ \Comment Add current path
\State \Return
\EndIf
\If{$w\notin C$} \Comment The begin word is not in $C$
\State \Return \Comment Do nothing
\EndIf
\For{$s\in C[w]$} \Comment Test each transformed word from $w$
\State $P\gets P + s$ \Comment Add $s$ to current path
\State $\texttt{BuildPath}(s, \beta, P, C, \pi)$ \Comment Proceed to next word $s$
\State $P\gets P - s$ \Comment Remove $s$ from current path
\EndFor
\EndFunction
\end{algorithmic}
\end{algorithm}