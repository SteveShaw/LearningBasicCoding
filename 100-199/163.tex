\section{163 --- Missing Ranges}
Given a sorted integer array \fcj{nums}, where the range of elements are in the \textbf{inclusive range} \fcj{[lower, upper]}, return its missing ranges.

\paragraph{Example:}

\begin{flushleft}
\textbf{Input}: 

\fcj{nums = [0, 1, 3, 50, 75]} 

\fcj{lower = 0} and \fcj{upper = 99}

\textbf{Output}: \fcj{["2", "4->49", "51->74", "76->99"]}
\end{flushleft}

\subsection{One Pass}

Maintain a variable \fcj{pre} to record last range's start. \fcj{pre} starts from \fcj{lower - 1}. The reason is that we can unify the processing of \fcj{lower} and \fcj{higher}

By traversing each number in \fcj{nums}, if the difference between current number $x$ and last range end \fcj{pre} is larger than 2, we found a missing range which has \fcj{pre+1} as the start and $x-1$ as the end. Then update \fcj{pre} as $x$.

After completion of traversing, we have to check if there is a missing range between \fcj{upper} and latest \fcj{pre}. If \fcj{upper - pre >= 1}, a missing range \fcj{[pre+1, upper-1]} does exist.

\setcounter{lstlisting}{0}
\begin{lstlisting}[style=customc, caption={One Pass}]
vector<string> findMissingRanges( vector<int>& nums, int lower, int upper )
{
    using ll = long long;
    auto to_str = []( ll s, ll e )
    {
        if( s == e )
        {
            return to_string( s );
        }

        return to_string( s ) + "->" + to_string( e );
    };
    vector<string> ans;
    auto pre = static_cast<ll>( lower );
    --pre;//unify boundary and in range comparison
    for( int n : nums )
    {
        auto ll_n = static_cast< ll >( n );
        if( n - pre >= 2 )
        {
            ans.emplace_back( to_str( pre + 1, ll_n - 1 ) );
        }
        pre = n;
    }
    //process the case for upper range
    auto hi = static_cast< ll >( upper );
    if( hi - pre >= 1 )
    {
        ans.emplace_back( to_str( pre + 1, hi ) );
    }
    return ans;
}
\end{lstlisting}