\section{115 --- Distinct Subsequences}
Given a string $S$ and a string $T$, count the number of distinct subsequences of $S$ which equals $T$.

A subsequence of a string is a new string which is formed from the original string by deleting some (can be none) of the characters without disturbing the relative positions of the remaining characters.  (i.e, \fcj{"ACE"} is a subsequence of \fcj{"ABCDE"} while \fcj{"AEC"} is not).
\paragraph{Example 1:}
\begin{flushleft}
\textbf{Input}: \fcj{S = "rabbbit"}, \fcj{T = "rabbit"}

\textbf{Output}: 3

\textbf{Explanation}:

As shown below, there are 3 ways you can generate \fcj{"rabbit"} from $S$.
\begin{itemize}
    \item \textbf{rabb}b\textbf{it}
    \item \textbf{ra}b\textbf{bbit}
    \item \textbf{rab}b\textbf{bit}
\end{itemize}
\end{flushleft}
\paragraph{Example 2:}
\begin{flushleft}
\textbf{Input}: \fcj{S = "babgbag"}, \fcj{T = "bag"}

\textbf{Output}: 5

\textbf{Explanation}:

As shown below, there are 5 ways you can generate \fcj{"bag"} from $S$.
\begin{itemize}
    \item \textbf{ba}b\textbf{g}bag
    \item \textbf{ba}bgba\textbf{g}
    \item \textbf{b}abgb\textbf{ag}
    \item ba\textbf{b}gb\textbf{ag}
    \item babg\textbf{bag}
    \end{itemize}
\end{flushleft}
\subsection{Dynamic Programming}
这种问题一般依靠Dynamic Programming来解决,假设$F[i][j]$表示$S[0,i-1]$中能够表示$T[0,j-1]$的subsequence的个数,$i=0$或者$j=0$则表示$S[0,-1]$或者$T[0,-1]$是空字符串。很显然$F[i][0]=1$因为对于$T$为空字符串,$S$中只有一个subsequence也就是空字符串能表示空字符串。对于$i>0$ and $j>0$,首先$F[i][j] = 0 + F[i-1][j]$即首先要加上$S[0,i-2]$中能够表示$T[0,j-1]$的subsequence的个数,接着看$S[i-1]$是否和$T[j-1]$相等,如果相等,$F[i][j]$还要加上$F[i-1][j-1]$,即$S[0,i-2]$中能够表示$T[0,j-2]$的subsequence的个数。不相等就不用加了。

\setcounter{lstlisting}{0}
\begin{lstlisting}[style=customc, caption={DP With 2D Array}]
int numDistinct( string s, string t )
{
    using ll_t = long long;
    vector<vector<ll_t>> F( s.size() + 1, vector<ll_t>( t.size() + 1, 0 ) );
    //fill the base case
    //for any substring of s
    //only one way to represent empty string
    for( size_t i = 0; i < s.size(); ++i )
    {
        F[i][0] = 1;
    }
    for( size_t x = 1; x <= s.size(); ++x )
    {
        for( size_t y = 1; y <= t.size(); ++y )
        {
            //for current letter s[x-1]
            //we can choose use it or not use it
            //if we do not use it
            //we add the way of substring S[0,x-2] to represent t[0,y-1]
            F[x][y] = F[x - 1][y];
            if( s[x - 1] == t[y - 1] )
            {
                //only when s[x-1]=t[y-1]
                //we can use s[x-1]
                //add the way of substring S[0,x-2] to represent t[0, y-2]
                F[x][y] += F[x - 1][y - 1];
            }
        }
    }
    return F[s.size()][t.size()];
}
\end{lstlisting}

Since $F[x][y]$ only depend on $F[x-1][y-1]$ and $F[x][y-1]$, we can reduce 2D to 1D array.

\begin{lstlisting}[style=customc, caption={DP With 1D Array}]

int numDistinct( string s, string t )
{
    //F is updated for each S[0,i]
    vector<int> F( t.size() + 1, 0 );
    F[0] = 1;
    for( size_t x = 1; x <= s.size(); ++x )
    {
        //each time we start from empty string of t
        //so the first element in F is 1
        size_t last = 1;
        for( size_t y = 1; y <= t.size(); ++y )
        {
            auto temp = F[y];
            if( s[x - 1] == t[y - 1] )
            {
                //we have to add the way
                //when add s[x-1] to the subsequence
                F[y] += last;
            }
            //update last value
            last = temp;
        }
    }
    return F.back();
}
\end{lstlisting}

