\definecolor{mygreen}{rgb}{0,0.6,0}
\definecolor{mygray}{rgb}{0.5,0.5,0.5}
\definecolor{mymauve}{rgb}{0.58,0,0.82}

\lstset{ 
  backgroundcolor=\color{gray!20!white},   % choose the background color; you must add \usepackage{color} or \usepackage{xcolor}; should come as last argument
  basicstyle=\footnotesize,        % the size of the fonts that are used for the code
  breaklines=true,                 % sets automatic line breaking
  commentstyle=\color{mygreen},    % comment style
  deletekeywords={...},            % if you want to delete keywords from the given language
  escapeinside={\%*}{*)},          % if you want to add LaTeX within your code
  extendedchars=true,              % lets you use non-ASCII characters; for 8-bits encodings only, does not work with UTF-8
  keepspaces=true,                 % keeps spaces in text, useful for keeping indentation of code (possibly needs columns=flexible)
  keywordstyle=\color{blue},       % keyword style
  language=C++,                 % the language of the code
  numbers=left,                    % where to put the line-numbers; possible values are (none, left, right)
  numbersep=5pt,                   % how far the line-numbers are from the code
  numberstyle=\tiny\color{mygray}, % the style that is used for the line-numbers
  showspaces=false,                % show spaces everywhere adding particular underscores; it overrides 'showstringspaces'
  showstringspaces=false,          % underline spaces within strings only
  showtabs=false,                  % show tabs within strings adding particular underscores
  stepnumber=2,                    % the step between two line-numbers. If it's 1, each line will be numbered
  stringstyle=\color{mymauve},     % string literal style
  tabsize=2,	                   % sets default tabsize to 2 spaces
  title=\lstname                   % show the filename of files included with \lstinputlisting; also try caption instead of title
}

\section{116 --- Populating Next Right Pointers in Each Node}
Given a binary tree $T$
\begin{lstlisting}
struct TreeLinkNode {
  TreeLinkNode *left;
  TreeLinkNode *right;
  TreeLinkNode *next;
}
\end{lstlisting}
Populate each next pointer to point to its next right node. If there is no next right node, the next pointer should be set to \texttt{null}.
\par
Initially, all next pointers are set to \texttt{null}.
\paragraph{Note:}
\begin{itemize}
    \item You may only use constant extra space.
    \item Recursive approach is fine, implicit stack space does not count as extra space for this problem.
    \item You may assume that it is a perfect binary tree (i.e., all leaves are at the same level, and every parent has two children).
\end{itemize}
\paragraph{Example:}
\begin{flushleft}
\textbf{Input}
\begin{figure}[H]
\begin{tikzpicture}
[mynode/.style={draw,circle,minimum size=5mm, fill=gray!20!}]
\node(){};
\node[mynode](1) {1};
\node[mynode](2) [below = 8mm of 1, xshift=-12mm] {2};
\node[mynode](3) [below = 8mm of 1, xshift=12mm] {3};
\node[mynode](4) [below = 8mm of 2, xshift=-7mm] {4};
\node[mynode](5) [below = 8mm of 2, xshift=7mm] {5};
\node[mynode](6) [below = 8mm of 3, xshift=-7mm] {6};
\node[mynode](7) [below = 8mm of 3, xshift=7mm] {7};
\draw[>=stealth,->] (1) -- (2);
\draw[>=stealth,->] (1) -- (3);
\draw[>=stealth,->] (2) -- (4);
\draw[>=stealth,->] (2) -- (5);
\draw[>=stealth,->] (3) -- (6);
\draw[>=stealth,->] (3) -- (7);
\end{tikzpicture}
\end{figure}
\textbf{Output}:
\begin{figure}[H]
\begin{tikzpicture}
[mynode/.style={draw,circle,minimum size=5mm, fill=gray!20!}]
\node(){};
\node[mynode](1) {1};
\node[mynode](r1) [right = 12mm of 1] {\texttt{null}};
\node[mynode](2) [below = 8mm of 1, xshift=-12mm] {2};
\node[mynode](3) [below = 8mm of 1, xshift=12mm] {3};
\node[mynode](r2) [right = 12mm of 3] {\texttt{null}};
\node[mynode](4) [below = 8mm of 2, xshift=-7mm] {4};
\node[mynode](5) [below = 8mm of 2, xshift=7mm] {5};
\node[mynode](6) [below = 8mm of 3, xshift=-7mm] {6};
\node[mynode](7) [below = 8mm of 3, xshift=7mm] {7};
\node[mynode](r3) [right = 12mm of 7] {\texttt{null}};
\draw[>=stealth,->] (1) -- (2);
\draw[>=stealth,->] (1) -- (3);
\draw[>=stealth,->] (2) -- (4);
\draw[>=stealth,->] (2) -- (5);
\draw[>=stealth,->] (3) -- (6);
\draw[>=stealth,->] (3) -- (7);
\draw[>=stealth,->] (1) -- (r1);
\draw[>=stealth,->] (3) -- (r2);
\draw[>=stealth,->] (7) -- (r3);
\draw[>=stealth,->] (2) -- (3);
\draw[>=stealth,->] (4) -- (5);
\draw[>=stealth,->] (5) -- (6);
\draw[>=stealth,->] (6) -- (7);
\end{tikzpicture}
\end{figure}
\end{flushleft}
\subsection{Analysis}
\begin{CJK*}{UTF8}{gbsn}
由于要求只能用$O(1)$的存储空间,因此需要有效利用$\texttt{next}$指针来联系左右节点。由于是否存在左右节点只能由父节点来判断,因此需要根据parent node的next来判断是否旁边还有node。由于已经明确了是完全二叉树,因此每个节点除了根节点之外都会有兄弟节点。而循环只需要走到叶子节点上层即可。
\end{CJK*}
\setcounter{algorithm}{0}
\begin{algorithm}[H]
\caption{Tricky Approach}
\begin{algorithmic}[1]
\Procedure{Connect}{$T$}
\If{$T=\texttt{null}$}
\State \Return
\EndIf
\State $x:=T$
\While{$\texttt{Left}(x)\neq \texttt{null}$} \Comment If the child is not leaf
\State $y:=x$ 
\While{$y\neq \texttt{null}$} \Comment Connect siblings in current level
\State $\texttt{Next}(\texttt{Left}(y))\gets \texttt{Right}(y)$
\If{$\texttt{Next}(y)\neq \texttt{null}$}
\State $\texttt{Next}(\texttt{Right}(y))\gets\texttt{Left}(\texttt{Next}(y))$ \Comment Connect right to right sibling's left
\EndIf
\State $y\gets \texttt{Next}(y)$ \Comment Move to right sibling
\EndWhile
\State $x\gets \texttt{Left}(x)$ \Comment Move to the leftmost child node 
\EndWhile
\EndProcedure
\end{algorithmic}
\end{algorithm}
