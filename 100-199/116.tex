\definecolor{mygreen}{rgb}{0,0.6,0}
\definecolor{mygray}{rgb}{0.5,0.5,0.5}
\definecolor{mymauve}{rgb}{0.58,0,0.82}

\section{116 --- Populating Next Right Pointers in Each Node}
Given a binary tree $T$

\begin{lstlisting}[style=customc]
struct TreeLinkNode {
  TreeLinkNode *left;
  TreeLinkNode *right;
  TreeLinkNode *next;
}
\end{lstlisting}
Populate each next pointer to point to its next right node. If there is no next right node, the next pointer should be set to \texttt{null}.
\par
Initially, all next pointers are set to \texttt{null}.
\paragraph{Note:}
\begin{itemize}
    \item You may only use constant extra space.
    \item Recursive approach is fine, implicit stack space does not count as extra space for this problem.
    \item You may assume that it is a perfect binary tree (i.e., all leaves are at the same level, and every parent has two children).
\end{itemize}
\paragraph{Example:}
\begin{flushleft}
\textbf{Input}
\begin{figure}[H]
\begin{tikzpicture}
[mynode/.style={draw,circle,minimum size=5mm, fill=gray!20!}]
\node(){};
\node[mynode](1) {1};
\node[mynode](2) [below = 8mm of 1, xshift=-12mm] {2};
\node[mynode](3) [below = 8mm of 1, xshift=12mm] {3};
\node[mynode](4) [below = 8mm of 2, xshift=-7mm] {4};
\node[mynode](5) [below = 8mm of 2, xshift=7mm] {5};
\node[mynode](6) [below = 8mm of 3, xshift=-7mm] {6};
\node[mynode](7) [below = 8mm of 3, xshift=7mm] {7};
\draw[>=stealth,->] (1) -- (2);
\draw[>=stealth,->] (1) -- (3);
\draw[>=stealth,->] (2) -- (4);
\draw[>=stealth,->] (2) -- (5);
\draw[>=stealth,->] (3) -- (6);
\draw[>=stealth,->] (3) -- (7);
\end{tikzpicture}
\end{figure}
\textbf{Output}:
\begin{figure}[H]
\begin{tikzpicture}
[mynode/.style={draw,circle,minimum size=5mm, fill=gray!20!}]
\node(){};
\node[mynode](1) {1};
\node[mynode](r1) [right = 12mm of 1] {\texttt{null}};
\node[mynode](2) [below = 8mm of 1, xshift=-12mm] {2};
\node[mynode](3) [below = 8mm of 1, xshift=12mm] {3};
\node[mynode](r2) [right = 12mm of 3] {\texttt{null}};
\node[mynode](4) [below = 8mm of 2, xshift=-7mm] {4};
\node[mynode](5) [below = 8mm of 2, xshift=7mm] {5};
\node[mynode](6) [below = 8mm of 3, xshift=-7mm] {6};
\node[mynode](7) [below = 8mm of 3, xshift=7mm] {7};
\node[mynode](r3) [right = 12mm of 7] {\texttt{null}};
\draw[>=stealth,->] (1) -- (2);
\draw[>=stealth,->] (1) -- (3);
\draw[>=stealth,->] (2) -- (4);
\draw[>=stealth,->] (2) -- (5);
\draw[>=stealth,->] (3) -- (6);
\draw[>=stealth,->] (3) -- (7);
\draw[>=stealth,->] (1) -- (r1);
\draw[>=stealth,->] (3) -- (r2);
\draw[>=stealth,->] (7) -- (r3);
\draw[>=stealth,->] (2) -- (3);
\draw[>=stealth,->] (4) -- (5);
\draw[>=stealth,->] (5) -- (6);
\draw[>=stealth,->] (6) -- (7);
\end{tikzpicture}
\end{figure}
\end{flushleft}
\subsection{Analysis}
由于要求只能用$O(1)$的存储空间,因此需要有效利用$\texttt{next}$指针来联系左右节点。由于是否存在左右节点只能由父节点来判断,因此需要根据parent node的next来判断是否旁边还有node。由于已经明确了是完全二叉树,因此每个节点除了根节点之外都会有兄弟节点。而循环只需要走到叶子节点上层即可。

\setcounter{lstlisting}{0}
\begin{lstlisting}[style=customc, caption={Iterative}]
Node* connect( Node* root )
{
    if( !root )
    {
        return root;
    }
    auto node = root;
    while( node->left )
    {
        auto t = node;
        while( t )
        {
            //connect left and right
            t->left->next = t->right;
            if( t->next )
            {
                //connect right to right sibling's left
                t->right->next = t->next->left;
            }
            //move to its sibling
            t = t->next;
        }
        //move to its child node
        node = node->left;
    }
    return root;
}
\end{lstlisting}
