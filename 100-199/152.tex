\section{152 --- Maximum Product Subarray}
Given an integer array $A$, find the contiguous subarray within an array (containing at least one number) which has the largest product.
\paragraph{Example 1:}
\begin{flushleft}
\textbf{Input}: $[2,3,-2,4]$
\\
\textbf{Output}: 6
\\
\textbf{Explanation}: $[2,3]$ has the largest product 6.
\end{flushleft}
\paragraph{Example 2:}
\begin{flushleft}
\textbf{Input}: $[-2,0,-1]$
\\
\textbf{Output}: 0
\\
\textbf{Explanation}: The result cannot be 2, because $[-2,-1]$ is not a subarray.
\end{flushleft}
\subsection{Kadane's Algorithm}
\begin{CJK*}{UTF8}{gbsn}
给予Kadane's algorithm,由于product受到正负数的影响,因此maintain两个变量:$\alpha$和$\beta$,其中$\alpha$是maximum subarray product ending at $i$,而$\beta$则是minimum subarray product ending at $i$。,同时用变量$P$代表目前为止所获得的最大product。遍历每一个数字时,首先确定当前数字$A[i]$,$\alpha\times A[i]$和$\beta\times A[i]$三者中的最大值,由于这时候$\alpha$会更新到这三者中的最大值,因此每次更新前将$\alpha$保存到一个临时变量$\gamma$中。然后确定数字$A[i]$, $\gamma\times A[i]$和$\beta\times A[i]$三者中的最小值,将$\beta$更新为这个最小值。最后更新$P$为$P$和更新后的$\alpha$两者中的最大值。遍历结束后,$P$即为所求最大product。
\end{CJK*}
\setcounter{lstlisting}{0}
\begin{lstlisting}[style=customc, caption={Modified Kadane's Algorithm}]
int maxProduct( vector<int>& A )
{
	//alpha: max_ending_here
	int max_ending_here = A[0];
	//beta: min_endign_here
	int min_ending_here = A[0];
	//P: max_so_far
	int max_so_far = A[0];

	for ( size_t i = 1; i < A.size(); ++i )
	{
		//save current alpha to a temp variable
		//gamma: cur_max
		int cur_max = max_ending_here;

		//update alpha
		max_ending_here = ( max )( A[i], min_ending_here * A[i] );
		max_ending_here = ( max )( max_ending_here, cur_max * A[i] );

		//update beta
		min_ending_here = ( min )( A[i], min_ending_here * A[i] );
		min_ending_here = ( min )( min_ending_here, cur_max * A[i] );

		//update P
		max_so_far = ( max )( max_so_far, max_ending_here );
	}

	return max_so_far;
}
\end{lstlisting}