\section{148 --- Sort List}
Sort a linked list $H$ in $O(n\log_2 n)$ time using constant space complexity.
\paragraph{Example 1:}
\begin{flushleft}
\textbf{Input}:
\begin{figure}[H]
\begin{tikzpicture}
[mynode/.style={draw,circle,minimum size=8mm, fill=gray!20!}]
\node(){};
\node[mynode](1) {4};
\node[mynode](2)[right=12mm of 1] {2};
\node[mynode](3)[right=12mm of 2] {1};
\node[mynode](4)[right=12mm of 3] {3};
\draw[>=stealth,->] (1) -- (2);
\draw[>=stealth,->] (2) -- (3);
\draw[>=stealth,->] (3) -- (4);
\end{tikzpicture}
\end{figure}
\textbf{Output}: 
\begin{figure}[H]
\begin{tikzpicture}
[mynode/.style={draw,circle,minimum size=8mm, fill=gray!20!}]
\node(){};
\node[mynode](1) {1};
\node[mynode](2)[right=12mm of 1] {2};
\node[mynode](3)[right=12mm of 2] {3};
\node[mynode](4)[right=12mm of 3] {4};
\draw[>=stealth,->] (1) -- (2);
\draw[>=stealth,->] (2) -- (3);
\draw[>=stealth,->] (3) -- (4);
\end{tikzpicture}
\end{figure}
\end{flushleft}
\paragraph{Example 2:}
\begin{flushleft}
\textbf{Input}:
\begin{figure}[H]
\begin{tikzpicture}
[mynode/.style={draw,circle,minimum size=8mm, fill=gray!20!}]
\node(){};
\node[mynode](1) {$-1$};
\node[mynode](2)[right=12mm of 1] {5};
\node[mynode](3)[right=12mm of 2] {3};
\node[mynode](4)[right=12mm of 3] {4};
\node[mynode](5)[right=12mm of 4] {0};
\draw[>=stealth,->] (1) -- (2);
\draw[>=stealth,->] (2) -- (3);
\draw[>=stealth,->] (3) -- (4);
\draw[>=stealth,->] (4) -- (5);
\end{tikzpicture}
\end{figure}
\textbf{Output}: 
\begin{figure}[H]
\begin{tikzpicture}
[mynode/.style={draw,circle,minimum size=8mm, fill=gray!20!}]
\node(){};
\node[mynode](1) {$-1$};
\node[mynode](2)[right=12mm of 1] {0};
\node[mynode](3)[right=12mm of 2] {3};
\node[mynode](4)[right=12mm of 3] {4};
\node[mynode](5)[right=12mm of 4] {5};
\draw[>=stealth,->] (1) -- (2);
\draw[>=stealth,->] (2) -- (3);
\draw[>=stealth,->] (3) -- (4);
\draw[>=stealth,->] (4) -- (5);
\end{tikzpicture}
\end{figure}
\end{flushleft}
\subsection{Merge Sort}
由于链表的特性,Merge Sort是最适合的排序方法。首先把原链表分成左右两部分,分别对这两部分调用排序函数,然后将这两部分排序号的链表进行merge。

\setcounter{lstlisting}{0}
\begin{lstlisting}[style=customc, caption={Merge Sort}]
ListNode* sortList( ListNode* head )
{
    if( !head || !head->next )
    {
        return head;
    }
    auto slow = head;
    auto fast = head;
    auto pre = slow;
    //after the while loop
    //slow will be at the next of middle node
    //so we require a (pre) node to track the previous
    //node of slow
    //for example: 4->2->null, slow will be at 2.
    while( fast && fast->next )
    {
        pre = slow;
        slow = slow->next;
        fast = fast->next->next;
    }
    //break up the list
    pre->next = nullptr;
    //recursively sort left and right
    auto t1 = sortList( head );
    auto t2 = sortList( slow );
    //merge the result
    return merge( t1, t2 );
}
//merge routine
ListNode* merge( ListNode* l1, ListNode* l2 )
{
    auto dummy = new ListNode( -1 );
    auto merged = dummy;
    while( l1 && l2 )
    {
        if( l1->val <= l2->val )
        {
            merged->next = l1;
            l1 = l1->next;
        }
        else
        {
            merged->next = l2;
            l2 = l2->next;
        }

        merged = merged->next;
    }
    if( l1 )
    {
        merged->next = l1;
    }
    if( l2 )
    {
        merged->next = l2;
    }
    auto head = dummy->next;
    dummy->next = nullptr;
    delete dummy;
    return head;
}
\end{lstlisting}



