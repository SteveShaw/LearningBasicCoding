\section{148 --- Sort List}
Sort a linked list $H$ in $O(n\log_2 n)$ time using constant space complexity.
\paragraph{Example 1:}
\begin{flushleft}
\textbf{Input}:
\begin{figure}[H]
\begin{tikzpicture}
[mynode/.style={draw,circle,minimum size=8mm, fill=gray!20!}]
\node(){};
\node[mynode](1) {4};
\node[mynode](2)[right=12mm of 1] {2};
\node[mynode](3)[right=12mm of 2] {1};
\node[mynode](4)[right=12mm of 3] {3};
\draw[>=stealth,->] (1) -- (2);
\draw[>=stealth,->] (2) -- (3);
\draw[>=stealth,->] (3) -- (4);
\end{tikzpicture}
\end{figure}
\textbf{Output}: 
\begin{figure}[H]
\begin{tikzpicture}
[mynode/.style={draw,circle,minimum size=8mm, fill=gray!20!}]
\node(){};
\node[mynode](1) {1};
\node[mynode](2)[right=12mm of 1] {2};
\node[mynode](3)[right=12mm of 2] {3};
\node[mynode](4)[right=12mm of 3] {4};
\draw[>=stealth,->] (1) -- (2);
\draw[>=stealth,->] (2) -- (3);
\draw[>=stealth,->] (3) -- (4);
\end{tikzpicture}
\end{figure}
\end{flushleft}
\paragraph{Example 2:}
\begin{flushleft}
\textbf{Input}:
\begin{figure}[H]
\begin{tikzpicture}
[mynode/.style={draw,circle,minimum size=8mm, fill=gray!20!}]
\node(){};
\node[mynode](1) {$-1$};
\node[mynode](2)[right=12mm of 1] {5};
\node[mynode](3)[right=12mm of 2] {3};
\node[mynode](4)[right=12mm of 3] {4};
\node[mynode](5)[right=12mm of 4] {0};
\draw[>=stealth,->] (1) -- (2);
\draw[>=stealth,->] (2) -- (3);
\draw[>=stealth,->] (3) -- (4);
\draw[>=stealth,->] (4) -- (5);
\end{tikzpicture}
\end{figure}
\textbf{Output}: 
\begin{figure}[H]
\begin{tikzpicture}
[mynode/.style={draw,circle,minimum size=8mm, fill=gray!20!}]
\node(){};
\node[mynode](1) {$-1$};
\node[mynode](2)[right=12mm of 1] {0};
\node[mynode](3)[right=12mm of 2] {3};
\node[mynode](4)[right=12mm of 3] {4};
\node[mynode](5)[right=12mm of 4] {5};
\draw[>=stealth,->] (1) -- (2);
\draw[>=stealth,->] (2) -- (3);
\draw[>=stealth,->] (3) -- (4);
\draw[>=stealth,->] (4) -- (5);
\end{tikzpicture}
\end{figure}
\end{flushleft}
\subsection{Merge Sort}
由于链表的特性,Merge Sort是最适合的排序方法。首先把原链表分成左右两部分,分别对这两部分调用排序函数,然后将这两部分排序号的链表进行merge。
\par
另外有一点需要注意,用快慢指针寻找链表的中间位置应该是
\begin{lstlisting}[style=customc]
fast = head;
slow = head;
while(fast->next && fast->next->next)
{
    fast = fast->next->next;
    slow = slow->next;
}
//now slow is the middle node of the list, i.e.,  
//if we count 0 from head, the number of the slow is
//floor(L/2). L is the number of nodes in the list
\end{lstlisting}
\setcounter{algorithm}{0}
\begin{algorithm}[H]
\caption{Merge Sort}
\begin{algorithmic}[1]
\Procedure{SortList}{$H$}
\If{$H=\texttt{null}$ \textbf{or} $H.\texttt{next}=\texttt{null}$}
\State \Return $H$
\EndIf
\State $F:=H$, $S:=H$ \Comment Fast and Slow pointers
\While{$F.\texttt{next}\neq\texttt{null}$ \textbf{and} $F.\texttt{next}.\texttt{next}\neq\texttt{null}$ }
\State $F\gets F.\texttt{next}$
\State $F\gets F.\texttt{next}$
\State $S\gets S.\texttt{next}$
\EndWhile
\State $H_1:=S.\texttt{next}$ \Comment The header of right list
\State $S.\texttt{next}\gets\texttt{null}$ \Comment Break the original list
\State $t_0:=\texttt{SortList}(H)$ \Comment Sort left half
\State $t_1:=\texttt{SortList}(H_1)$ \Comment Sort right half
\State $\hat{H}:=\texttt{Merge}(H, H_1)$ \Comment Merge the sorted two lists
\State \Return $\hat{H}$
\EndProcedure
\end{algorithmic}
\end{algorithm}
Function \texttt{Merge} merge two linked lists $L_1$ and $L_2$ by ascending order and return the header of the merged list
\begin{algorithm}[H]
\caption{Merge Helper Function}
\begin{algorithmic}[1]
\Function{Merge}{$L_1, L_2$}
\State Create a dummy header $D$
\State $\ell:=D$
\algstore{148algo}
\end{algorithmic}
\end{algorithm}
\begin{algorithm}[H]
\begin{algorithmic}[1]
\algrestore{148algo}
\While{$L_1\neq\texttt{null}$ \textbf{and} $L_2\neq\texttt{null}$}
\If{$L_1.\texttt{value}<L_2.\texttt{value}$}
\State $\ell.\texttt{next}\gets L_1$
\State $L_1\gets L_1.\texttt{next}$
\Else
\State $\ell.\texttt{next}\gets L_2$
\State $L_2\gets L_2.\texttt{next}$
\EndIf
\State $\ell\gets\ell.\texttt{next}$ \Comment Move the merged list pointer to next node
\EndWhile
\If{$L_1\neq\texttt{null}$} \Comment $L_1$ still has elements 
\State $\ell.\texttt{next}\gets L_1$ \Comment Link the merged list to the remainings of $L_1$
\EndIf
\If{$L_2\neq\texttt{null}$} \Comment $L_2$ still has elements 
\State $\ell.\texttt{next}\gets L_2$ \Comment Link the merged list to the remainings of $L_2$
\EndIf
\State \Return $D.\texttt{next}$ \Comment The dummy header's next node is the header of merged list
\EndFunction
\end{algorithmic}
\end{algorithm}
