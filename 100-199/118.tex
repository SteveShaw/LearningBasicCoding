\section{118 --- Pascal's Triangle}
Given a non-negative integer \fcj{numRows}, generate the first \fcj{numRows} of Pascal's triangle.

In Pascal's triangle, each number is the sum of the two numbers directly above it.
\paragraph{Example:}
\begin{flushleft}
\textbf{Input}: 5
\\
\textbf{Output}:

\fcj{[[1], [1,1], [1,2,1], [1,3,3,1], [1,4,6,4,1]]}
\end{flushleft}
\subsection{Analysis}

每一层的首个和结尾一个数字都是1,从第三层开始,中间的每个数字都是上一行的左右两个数字之和

\setcounter{lstlisting}{0}
\begin{lstlisting}[style=customc, caption={DP}]
vector<vector<int>> generate( int numRows )
{
    vector<vector<int>> ans;
    if( numRows == 0 )
    {
        return ans;
    }
    ans.emplace_back( vector<int> {1} );
    for( int r = 2; r <= numRows; ++r )
    {
        vector<int> cur( r, 1 );
        //the head and tail are 1
        //fill from index = 1 to r-2
        for( int i = 1; i < r - 1; ++i )
        {
            cur[i] = ans.back()[i - 1] + ans.back()[i];
        }
        ans.emplace_back( vector<int> {} );
        ans.back().swap( cur );
    }
    return ans;
}
\end{lstlisting}
