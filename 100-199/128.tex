\section{128 --- Longest Consecutive Sequence}
Given an unsorted array of integers $A$, find the length of the longest consecutive elements sequence.
\par
Your algorithm should run in $O(n)$ complexity.
\paragraph{Example:}
\begin{flushleft}
\textbf{Input}: $[100, 4, 200, 1, 3, 2]$
\\
\textbf{Output}: 4
\\
\textbf{Explanation}: The longest consecutive elements sequence is $[1, 2, 3, 4]$. Therefore its length is 4.
\end{flushleft}

\subsection{Hash Map}
由于$A$可能是乱序的,所以需要用一个hash map来记录每个数字当前所找到包含其在内的连续数的序列长度。遍历数组中的每个数字$x$时,然后在hash map中寻找$x-1$对应的序列长度$\ell_1$,以及$x+1$对应的序列长度$\ell_2$,这时候由于$x$的加入,之前分开的$x-1$和$x+1$可以连接在一起了,因此这时候就要把这个三个数字对应的长度统一更新为$\ell_1+\ell_2+1$。除此以外,还需要将$x - \ell_1$和$x+\ell_2$这两个数字对应的长度更新为$\ell_1+\ell_2+1$,因为这是当前连续数序列的起始和终止值。如果不更新这两个值,以后碰到比如$x-\ell_1+1$时,这个时候长度计算就错误了。当然,如果hash map中找不到对应的数字,那么返回的长度相应为0。


\setcounter{lstlisting}{0}
\begin{lstlisting}[style=customc, caption={Hash Map}]
int longestConsecutive( vector<int>& nums )
{
    if( nums.empty() )
    {
        return 0;
    }

    unordered_map<int, int> m;

    int ans = 1;

    for( int n : nums )
    {
        if( m.find( n ) != m.end() )
        {
            //n already has processed
            //in previous number
            continue;
        }

        auto it_l = m.find( n - 1 );
        int left_len = 0;


        if( it_l != m.end() )
        {
            //[n-left_len, n-1] is consecutive
            left_len = it_l->second;
        }

        int right_len = 0;

        auto it_r = m.find( n + 1 );

        if( it_r != m.end() )
        {
            //[n+1, n+right_len] is consecutive
            right_len = it_r->second;
        }

        //found new consecutive sequence
        int len = left_len + 1 + right_len;

        ans = ( max )( ans, len );

        m.emplace( n, len );

        if( it_l != m.end() )
        {
            //update number n-1
            //related length
            it_l->second = len;
        }

        if( it_r != m.end() )
        {
            //update number n+1
            //related length
            it_r->second = len;
        }

        //we have to update
        //y = n-left_len related length
        it_l = m.find( n - left_len );
        if( left_len && ( it_l != m.end() ) )
        {
            m[n - left_len] = len;
        }


        //we also have to update
        //z=n+right_len related length
        it_r = m.find( n + right_len );
        if( right_len && ( it_r != m.end() ) )
        {
            m[n + right_len] = len;
        }
    }

    return ans;
}
\end{lstlisting}

\subsection{Hash Set And Intelligent Sequence Building}
The numbers are stored in a hash set to allow $O(1)$ lookup, and we only attempt to build sequences from numbers that are not part of a longer sequence. This is accomplished by 
\begin{enumerate}
\item ensuring that the number that would immediately precede the current number in a sequence is not present, as that number would necessarily be part of a longer sequence.
\end{enumerate}

\begin{lstlisting}[style=customc, caption={Hash Set}]
int longestConsecutive( vector<int>& nums )
{
    unordered_set<int> seen( begin( nums ), end( nums ) );
    //the longest consecutive numbers
    int best_seq = 0;
    for( int n : nums )
    {
        if( seen.find( n - 1 ) == seen.end() )
        {
            //a new sequence starting from n
            int cur = n;
            int cur_seq = 1;
            //continue build the sequence
            while( seen.find( cur + 1 ) != seen.end() )
            {
                ++cur;
                ++cur_seq;
            }
            //update the global longest sequence length
            best_seq = ( max )( best_seq, cur_seq );
        }
    }
    return best_seq;
}
\end{lstlisting}