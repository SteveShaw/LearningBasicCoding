\section{173 --- Binary Search Tree Iterator}
Implement an iterator over a binary search tree (BST). Your iterator will be initialized with the root node of a BST.
\par
Calling \texttt{next}() will return the next smallest number in the BST.
\begin{figure}[H]
\begin{tikzpicture}
[mynode/.style={draw,circle,minimum size=8mm, fill=gray!20!}]
\node(){};
\node[mynode](1) {{\footnotesize 7}};
\node[mynode](2)[below=8mm of 1, xshift=-8mm] {{\footnotesize 3}};
\node[mynode](3)[below=8mm of 1, xshift=8mm] {{\footnotesize 15}};
\node[mynode](4)[below=8mm of 3, xshift=-8mm] {{\footnotesize 9}};
\node[mynode](5)[below=8mm of 3, xshift=8mm] {{\footnotesize 20}};
\draw[>=stealth,->] (1) -- (2);
\draw[>=stealth,->] (1) -- (3);
\draw[>=stealth,->] (3) -- (4);
\draw[>=stealth,->] (3) -- (5);
\end{tikzpicture}
\end{figure}
\begin{lstlisting}[style=customc]
BSTIterator iterator = new BSTIterator( root );
iterator.next();    // return 3
iterator.next();    // return 7
iterator.hasNext(); // return true
iterator.next();    // return 9
iterator.hasNext(); // return true
iterator.next();    // return 15
iterator.hasNext(); // return true
iterator.next();    // return 20
iterator.hasNext(); // return false
\end{lstlisting}
\paragraph{Note:}
\begin{itemize}
\item \texttt{next}() and \texttt{hasNext}() should run in average $O(1)$ time and uses $O(h)$ memory, where $h$ is the height of the tree.
\item You may assume that \texttt{next}() call will always be valid, that is, there will be at least a next smallest number in the BST when \texttt{next}() is called.
\end{itemize}
\subsection{In-Order Traverse}
\begin{CJK*}{UTF8}{gbsn}
其实这就是iterative in-order traverse。下面是iterative in-order traverse的代码
\begin{lstlisting}[style=customc]
TreeNode visit = root;
Stack<TreeNode> stack = new Stack();

while( visit != null || !stack.empty() )
{
    while( visit != null )
    {
        stack.push( visit );
        visit = visit.left;
    }
    TreeNode next = stack.pop();
    visit = next.right;
    doSomethingWith( next.val );
}
\end{lstlisting}
而这里需要实现的如下所示
\begin{lstlisting}[style=customc]
BSTIterator i = new BSTIterator( root );
while( i.hasNext() )
{
    doSomethingWith( i.next() );
}
\end{lstlisting}
可以看到这两个有相同的结构,即包括
\begin{itemize}
\item 初始化.
\item A while-loop with \textbf{a condition that tells whether there is more}.
\item The loop body \textbf{gets the next value} and does something with it.
\end{itemize}
这些就组成了这个类的三个函数。
\end{CJK*}
\setcounter{lstlisting}{0}
\begin{lstlisting}[style=customc, caption={Class Definition}]
/**
 * Definition for a binary tree node.
 * struct TreeNode {
 *     int val;
 *     TreeNode *left;
 *     TreeNode *right;
 *     TreeNode(int x) : val(x), left(NULL), right(NULL) {}
 * };
 */
class BSTIterator
{
public:
    BSTIterator( TreeNode* root )
    {
        node_ = root;
    }

    /** @return the next smallest number */
    int next()
    {
        while( node_ )
        {
            stk_.push( node_ );
            node_ = node_->left;
        }

        auto next = stk_.top();
        stk_.pop();
        node_ = next->right;

        return next->val;
    }

    /** @return whether we have a next smallest number */
    bool hasNext()
    {
        return ( node_ != nullptr ) || ( !stk_.empty() );
    }

    TreeNode* node_;
    stack<TreeNode*> stk_;
};

/**
 * Your BSTIterator object will be instantiated and called as such:
 * BSTIterator* obj = new BSTIterator(root);
 * int param_1 = obj->next();
 * bool param_2 = obj->hasNext();
 */
\end{lstlisting}