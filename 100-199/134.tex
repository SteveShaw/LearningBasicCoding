\section{134 --- Gas Station}
There are $N$ gas stations along a circular route, where the amount of gas at station i is $G[i]$.
\par
You have a car with an unlimited gas tank and it costs $C[i]$ of gas to travel from station $i$ to its next station $i+1$. You begin the journey with an empty tank at one of the gas stations.
\par
Return the starting gas station's index if you can travel around the circuit once in the clockwise direction, otherwise return $-1$.
\paragraph{Note:}
\begin{itemize}
\item If there exists a solution, it is guaranteed to be unique.
\item Both input arrays are non-empty and have the same length.
\item Each element in the input arrays is a non-negative integer.
\end{itemize}
\paragraph{Example 1:}
\begin{flushleft}
\textbf{Input}:
\\
$G  = [1,2,3,4,5]$, $C =[3,4,5,1,2]$
\\
\textbf{Output}: 3
\\
\textbf{Explanation}:
\begin{itemize}
\item Start at station 3 (index 3) and fill up with 4 unit of gas. Your tank $= 0 + 4 = 4$
\item Travel to station 4. Your tank $= 4 - 1 + 5 = 8$
\item Travel to station 0. Your tank $= 8 - 2 + 1 = 7$
\item Travel to station 1. Your tank $= 7 - 3 + 2 = 6$
\item Travel to station 2. Your tank $= 6 - 4 + 3 = 5$
\item Travel to station 3. The cost is 5. Your gas is just enough to travel back to station 3.
\end{itemize}
Therefore, return 3 as the starting index.
\end{flushleft}
\paragraph{Example 2:}
\begin{flushleft}
\textbf{Input}:
\\ 
$G  = [2,3,4]$, $C = [3,4,3]$
\\
\textbf{Output}: $-1$
\\
\textbf{Explanation}:
\begin{itemize}
\item You cannot start at station 0 or 1, as there is not enough gas to travel to the next station.
\item Let's start at station 2 and fill up with 4 unit of gas. Your tank $= 0 + 4 = 4$
\item Travel to station 0. Your tank $= 4 - 3 + 2 = 3$
\item Travel to station 1. Your tank $= 3 - 3 + 3 = 3$
\item You cannot travel back to station 2, as it requires 4 unit of gas but you only have 3.
\end{itemize}
Therefore, you cannot travel around the circuit once no matter where you start.
\end{flushleft}
\subsection{Analysis}
首先能走完整个环的前提是gas的总量要大于cost的总量,这样才会有起点的存在。假设开始设置起点$s = 0$, 并从这里出发,如果当前的gas值大于cost值,就可以继续前进,此时到下一个station,剩余的gas加上当前的gas再减去cost,看是否大于0,若大于0,则继续前进。当到达某一个station时,若这个值小于0了,则说明从起点到这个点中间的任何一个点都不能作为起点,则把起点设为下一个点,继续遍历。当遍历完整个环时,当前保存的起点即为所求。
\setcounter{algorithm}{0}
\begin{algorithm}[H]
\caption{One Pass}
\begin{algorithmic}[1]
\Procedure{CanCompleteCircuit}{$G,C,L$}
\State $t:=0$ \Comment Total sum of difference between gas and cost at each station
\State $s:=0$ \Comment The start station
\State $z:=0$ \Comment Current sum of difference between gas and cost at each station
\For{$i:=0\to L$} \Comment $G$ and $C$ lengths are equal to $L$
\State $t\gets t + G[i] - C[i]$
\State $z\gets z + G[i] - C[i]$
\If{$z<0$} 
\State $z\gets 0$ \Comment Reset to zero
\State $s\gets i+1$ \Comment Move start station to next
\EndIf
\EndFor
\If{$t<0$} \Comment Total gass are less than total costs
\State \Return $-1$ \Comment Cannot complete the circuit
\Else
\State \Return $z$
\EndIf
\EndProcedure
\end{algorithmic}
\end{algorithm}
