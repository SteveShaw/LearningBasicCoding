\section{120 --- Triangle}
Given a triangle $T$, find the minimum path sum from top to bottom. Each step you may move to adjacent numbers on the row below.
\paragraph{Example}
\begin{flushleft}
\textbf{Input}:
\begin{table}[H]
\begin{tabular}{llll}
2 &   &   &   \\
3 & 4 &   &   \\
6 & 5 & 7 &   \\
4 & 1 & 8 & 3
\end{tabular}
\end{table}
\textbf{Output}: 11

\textbf{Explanation}:
The minimum path sum from top to bottom is 11 (i.e., $2 + 3 + 5 + 1 = 11$).
\end{flushleft}
\paragraph{Note:}
\begin{itemize}
\item Bonus point if you are able to do this using only $O(n)$ extra space, where $n$ is the total number of rows in the triangle.
\end{itemize}
\subsection{Dynamic Programming}

表面上看,这个三角形看起来像一个树状结构,这很自然让我们想到用树的遍历算法比如DFS。但是,如果仔细观察就会发现,其实相邻的nodes构成了一个选择分支。也就是说,存在\textbf{overlapping subproblems}。而且,如果$x, y$是$k$的adjacent nodes,一旦找到了minimum path starting from $x$ and $y$,那么minimum path starting from $k$也就找到了,这就构成了一个\textbf{optimal substructure}。因此,Dynamic Programming是最佳的solution。



如果采用\textbf{Top--Down}的方式,从最顶层的node出发,递归寻找每个node的minium path sum。每次计算出一个path sum,就存放在一个数组中。下一次需要计算同一个node的minimum path sum时,只需要直接从存放的数组中获得。但是这个数组的大小至少要与输入的三角形数组的大小相等,而这需要$O(N^2)$存储空间。虽然也能通过某些trick优化这个存储空间,但是这在recursive方法中很难直接想到。


反之,如果采用\textbf{Bottom--Up}的方式,优化存储空间就很直接了。首先从最底层的node出发,最开始每个node的minimum path sum就是node本身的值。接着,在$k$th row的$i$th node的minimum path sum就是这个节点的两个children的minimum path sum中的最小值加上这个节点的值。即:

\[
F(k,i)=\min(F(k+1, i), F(k+1, i+1)) + T[k][i]
\]

接下来就是如何优化存储空间,我们需要将$F$修改成1D array,然后iteratively逐步更新$F$,即对于$k$th level

\[
F[i] = \min(F[i], F[i+1]) + T[k][i];
\]
最后$F[0]$就是最终的结果。

\setcounter{lstlisting}{0}
\begin{lstlisting}[style=customc, caption={DP}]
int minimumTotal( vector<vector<int>>& triangle )
{
    vector<int> F( triangle.back() );
    auto levels = static_cast<int>( triangle.size() );
    for( int level = levels - 2; level >= 0; --level )
    {
        for( int i = 0; i <= level; ++i )
        {
            F[i] = ( min )( F[i], F[i + 1] ) + triangle[level][i];
        }
    }
    return F[0];
}
\end{lstlisting}