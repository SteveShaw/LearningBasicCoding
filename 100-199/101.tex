\section{101 --- Symmetric Tree}
Given a binary tree $T$, check whether it is a mirror of itself (ie, symmetric around its center).
\par
For example, this binary tree is symmetric:
\begin{figure}[H]
\begin{tikzpicture}
[mynode/.style={draw,circle,minimum size=5mm, fill=gray!20!}]
\node(){};
\node[mynode](1) {1};
\node[mynode](2l) [below = 8mm of 1, xshift=-15mm] {2};
\node[mynode](2r) [below = 8mm of 1, xshift=15mm] {2};
\node[mynode](3l) [below = 8mm of 2l, xshift=-6mm] {3};
\node[mynode](4l) [below = 8mm of 2l, xshift=6mm] {4};
\node[mynode](3r) [below = 8mm of 2r, xshift=6mm] {3};
\node[mynode](4r) [below = 8mm of 2r, xshift=-6mm] {4};
\draw[>=stealth,->] (1) -- (2l);
\draw[>=stealth,->] (1) -- (2r);
\draw[>=stealth,->] (2l) -- (3l);
\draw[>=stealth,->] (2l) -- (4l);
\draw[>=stealth,->] (2r) -- (3r);
\draw[>=stealth,->] (2r) -- (4r);
\end{tikzpicture}
\end{figure}
But the following is not:
\begin{figure}[H]
\begin{tikzpicture}
[mynode/.style={draw,circle,minimum size=5mm, fill=gray!20!}]
\node(){};
\node[mynode](1) {1};
\node[mynode](2l) [below = 8mm of 1, xshift=-12mm] {2};
\node[mynode](2r) [below = 8mm of 1, xshift=12mm] {2};
\node[mynode](3l) [below = 8mm of 2l, xshift=6mm] {3};
\node[mynode](3r) [below = 8mm of 2r, xshift=6mm] {3};
\draw[>=stealth,->] (1) -- (2l);
\draw[>=stealth,->] (1) -- (2r);
\draw[>=stealth,->] (2l) -- (3l);
\draw[>=stealth,->] (2r) -- (3r);
\end{tikzpicture}
\end{figure}
\paragraph{Note:}
Bonus points if you could solve it both recursively and iteratively.
\subsection{Recursion}
首先注意到,用于做比较的函数肯定需要两个nodes $n_1$ and $n_2$,因此需要一个额外的辅助函数来进行递归,因为主函数只有一个输入就是根节点。接下来,如果$n_1$ and $n_2$都是空节点,显然是相等的,否则如果都是非空节点并且$n_1$的值和$n_2$的值相等,则递归比较$n_1$的左子树和$n_2$的右子树以及$n_1$的右子树和$n_2$的左子树。

\subsection{Iteration}
这里需要两个queue,一个存放根节点的左子树,另外一个存放根节点的右子树,然后算法基本就和recursion一样。
\begin{algorithm}[H]
\caption{Iteration}
\begin{algorithmic}[1]
\Procedure{IsSymmetric}{$T$}
\If{$T=\text{NULL}$}
\State \Return \texttt{true}
\EndIf
\State Initialize $Q_1$ and $Q_2$ as two empty queues
\State $Q_1\gets Q_1 + \text{LEFT}(T)$
\State $Q_2\gets Q_2 + \text{RIGHT}(T)$
\algstore{102algo}
\end{algorithmic}
\end{algorithm}
\begin{algorithm}[H]
\begin{algorithmic}[1]
\algrestore{102algo}
\While{$Q_1\neq \emptyset$ \textbf{and} $Q_2\neq \emptyset$}
\State $n_1:=\text{FRONT}(Q_1)$
\State $\text{POP}(Q_1)$
\State $n_2:=\text{FRONT}(Q_2)$
\State $\text{POP}(Q_2)$
\If{$n_1=\text{NULL}$ \textbf{and} $n_2=\text{NULL}$}
\State \texttt{continue} \Comment Continue to next items
\EndIf
\If{$n_1\neq\text{NULL}$ \textbf{and} $n_2\neq\text{NULL}$ \textbf{and} $\text{Value}(n_1)=\text{Value}(n_2)$}
\State $Q_1\gets Q_1 + \text{LEFT}(n_1)$
\State $Q_2\gets Q_2 + \text{RIGHT}(n_2)$
\State $Q_1\gets Q_1 + \text{RIGHT}(n_1)$
\State $Q_2\gets Q_2 + \text{LEFT}(n_2)$
\Else
\State \Return \texttt{false}
\EndIf
\EndWhile
\State \Return \texttt{true}
\EndProcedure
\end{algorithmic}
\end{algorithm}