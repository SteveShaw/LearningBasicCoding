\section{140 --- Word Break II}
Given a non-empty string $s$ and a dictionary \fcj{wordDict} containing a list of non-empty words, add spaces in $s$ to construct a sentence where each word is a valid dictionary word. Return all such possible sentences.
\paragraph{Note:}
\begin{itemize}
\item The same word in the dictionary may be reused multiple times in the segmentation.
\item You may assume the dictionary does not contain duplicate words.
\end{itemize}
\paragraph{Example 1:}
\begin{flushleft}

\textbf{Input}: 

$s$: \fcj{"catsanddog"}

\fcj{wordDict = ["cat", "cats", "and", "sand", "dog"]}

$S$ = catsanddog, $A = [\text{cat}, \text{cats}, \text{and}, \text{sand}, \text{dog}]$

\textbf{Output}:

\fcj{[cats and dog",  "cat sand dog"]}

\end{flushleft}

\paragraph{Example 2:}

\begin{flushleft}

$s$: \fcj{"pineapplepenapple"}

\fcj{wordDict = ["apple", "pen", "applepen", "pine", "pineapple"]}

\textbf{Output}:
\fcj{["pine apple pen apple", "pineapple pen apple", "pine applepen apple"]}

\textbf{Explanation}: 

Note that you are allowed to reuse a dictionary word.
\end{flushleft}


\paragraph{Example 3:}
\begin{flushleft}
\textbf{Input}: 

$s$: \fcj{"catsandog"}

\fcj{wordDict = ["cats", "dog", "sand", "and", "cat"]}


\textbf{Output}: \fcj{[]}
\end{flushleft}


\subsection{Recursion with memoization}
The naive approach to solve this problem is to use recursion. 

For finding the solution, we check every possible prefix of that string $s$ in the dictionary of words, if it is found in the dictionary (say $s_1$), then the recursive function is called for the remaining portion of that string. This function returns the prefix $s_1$ appended by the result of the recursive call using the remaining portion of $s$ if the remaining portion is a sub-string per the dictionary. Otherwise, empty list is returned.

To avoid duplicate sub-problems, we can use memorization to prune the branches. In this case, we are making use of a hash map to store the results. In this map, the key is the starting index of the string currently considered and the value contains all the sentences which can be formed using the substring from this starting index. 

Thus, if we encounter the same starting index from different function calls, we can return the result directly from the hash map rather than going for redundant function calls.
\setcounter{lstlisting}{0}
\begin{lstlisting}[style=customc, caption={DP With Memorization}]
vector<string> wordBreak( string s, vector<string>& wordDict )
{
    //hash map used for DP
    unordered_map<size_t, vector<string>> memo;
    unordered_set<string_view> dict( wordDict.begin(), wordDict.end() );
    return dfs( s, 0, memo, dict );
}
//dp helper function
vector<string> dfs( string_view s, size_t start, unordered_map<size_t, vector<string>>& memo, unordered_set<string_view>& dict )
{
    if( start == s.size() )
    {
        return { "" };
    }

    //if we have memorized
    //use it
    auto it = memo.find( start );
    if( it != memo.end() )
    {
        return it->second;
    }
    vector<string> res;
    for( size_t end = start; end < s.size(); ++end )
    {
        if( dict.find( s.substr( start, end - start + 1 ) ) != dict.end() )
        {
            //s[start,end] is in dictionary
            //deeply test for s[end+1,...]
            auto subs = dfs( s, end + 1, memo, dict );
            for( auto& sub : subs )
            {
                res.emplace_back( s.substr( start, end - start + 1 ) );
                if( !sub.empty() )
                {
                    //add space
                    res.back().push_back( ' ' );
                    res.back().append( sub.c_str() );
                }
            }
        }
    }
    memo[start] = res;
    return res;
}
\end{lstlisting}

%\subsection{Dynamic Programming}
%The given problem ($s$) can be divided into sub-problems $s_1$ and $s_2$. If these sub problems individually satisfy the required conditions, the complete problem, $s$ also satisfies the same conditions. 
%
%For example, \fcj{"catsanddog"} can be split into two substrings \fcj{"catsand"}, \fcj{"dog"}. The subproblem \fcj{"catsand"} can be further divided into \fcj{"cats"}, \fcj{"and"}, which individually are a part of the dictionary making \fcj{"catsand"} satisfy the condition. Going further backwards, \fcj{"catsand"}, \fcj{"dog"} also satisfy the required criteria individually leading to the complete string \fcj{"catsanddog"} also to satisfy the criteria.
%
%We make use of an array $F$ with size $n+1$, where $n$ is the length of $s$. $F[k]$ is used to store every possible sentence having all valid dictionary words using the substring $s[0:k-1]$. We also use two index pointers $i$ and $j$, where $i$ refers to the length of the substring considered currently starting from the beginning (i.e. $s[0:i-1]$), and $j$ refers to the index partitioning this substring into smaller ones (i.e. $s[0:j-1]$ and $s[j:i-1]$)
%
%To fill $F$, we initialize the element $F[0]$ as an empty string, since no sentence can be formed using a word of size 0. We consider substrings of all possible lengths starting from the beginning by making use of index $i$. For every such substring, we partition the string into two smaller ones in all possible ways using the index $j$. To fill $F[i]$, 
%
%\begin{enumerate}
%\item we check if the $F[j]$ contains a non-empty string i.e. if some valid sentence can be formed using $s[0:j-1]$. 
%\item If so, we further check if $s[j:i-1]$ is present in the dictionary. 
%\item If both the conditions are satisfied, we append the substring $s[j:i-1]$  to every possible sentence that can be formed up to the index $j$ (which is already stored in $F[j]$). 
%\item These newly formed sentences are stored in $F[i]$. 
%\item Finally the element $F[n]$ gives all possible valid sentences using the complete string $s$.
%\end{enumerate}