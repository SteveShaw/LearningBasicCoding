\section{154 --- Find Minimum in Rotated Sorted Array II}
Suppose an array $A$ sorted in ascending order is rotated at some pivot unknown to you beforehand. (i.e.,  \fcj{[0,1,2,4,5,6,7]} might become  \fcj{[4,5,6,7,0,1,2]}).

Find the minimum element.

The array may contain \textbf{duplicates}.
\paragraph{Example 1:}
\begin{flushleft}
\textbf{Input}: \fcj{[1,3,5]}

\textbf{Output}: 1
\end{flushleft}

\paragraph{Example 2:}
\begin{flushleft}
\textbf{Input}: \fcj{[2,2,2,0,1]}

\textbf{Output}: 0
\end{flushleft}
\paragraph{Note:}
\begin{itemize}
    \item This is a follow up problem to \textcolor{blue}{Find Minimum in Rotated Sorted Array}.
    \item Would allow duplicates affect the run-time complexity? How and why?
\end{itemize}
\subsection{Binary Search}

这里和153基本类似,不过这里允许重复数字存在。因此算法和153基本类似,唯一不同的是,当$A[m]=A[r]$时,需要将$r$往左移动一个位置,然后继续进行比较。由于重复元素的存在,当然$A[m]=A[r]$时,实际上就成了线性搜索,特别的,当$A$中所有数字都相等时,时间复杂度就是$O(n)$。

\setcounter{lstlisting}{0}
\begin{lstlisting}[style=customc, caption={Binary Search}]
int findMin( vector<int>& A )
{
	//right boundry of searching
	int r = static_cast< int >( A.size() - 1 ); 
	//left boundry of searching
	int l = 0;

	//We need l to be a valid index 
	//so the condition is (l < r)
	//not (l<=r)
	while ( l < r )
	{
		int mid = ( l + r ) / 2;

		if ( A[mid] > A[r] )
		{
			//The minimum is in [mid+1,r]
			l = mid + 1;
		}
		else if( A[mid] < A[r] )
		{
			//The minimum is in [l, mid]
			r = mid;
		}
		else  // A[m]=A[r]
		{
			//Left move r by one index
			--r;
		}
	}

	return A[l];
}
\end{lstlisting}

Since for duplicate numbers, we will fall down to linear scan. The best method would be just linear scan.

\setcounter{lstlisting}{0}
\begin{lstlisting}[style=customc, caption={Linear Scan}]
int findMin( vector<int>& nums )
{
    auto p_min = min_element( begin( nums ), end( nums ) );
    return *p_min;
}
\end{lstlisting}