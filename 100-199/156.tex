\section{156 --- Binary Tree Upside Down}
Given a binary tree with root $R$, where all the right nodes are either leaf nodes with a sibling (a left node that shares the same parent node) or empty, flip it upside down and turn it into a tree where the original right nodes turned into left leaf nodes. Return the new root.
\paragraph{For example:}
\begin{flushleft}
Given a binary tree
\begin{figure}[H]
\begin{tikzpicture}
[mynode/.style={draw,circle,minimum size=5mm, fill=gray!20!}]
\node(){};
\node[mynode](1) {1};
\node[mynode](2)[below=8mm of 1, xshift=-8mm] {2};
\node[mynode](3)[below=8mm of 1, xshift=8mm] {3};
\node[mynode](4)[below=8mm of 2, xshift=-8mm] {4};
\node[mynode](5)[below=8mm of 2, xshift=8mm] {5};
\draw[>=stealth,->] (1) -- (2);
\draw[>=stealth,->] (1) -- (3);
\draw[>=stealth,->] (2) -- (4);
\draw[>=stealth,->] (2) -- (5);
\end{tikzpicture}
\end{figure}
return the root of the binary tree
\begin{figure}[H]
\begin{tikzpicture}
[mynode/.style={draw,circle,minimum size=5mm, fill=gray!20!}]
\node(){};
\node[mynode](1) {4};
\node[mynode](2)[below=8mm of 1, xshift=-8mm] {5};
\node[mynode](3)[below=8mm of 1, xshift=8mm] {2};
\node[mynode](4)[below=8mm of 3, xshift=-8mm] {3};
\node[mynode](5)[below=8mm of 3, xshift=8mm] {1};
\draw[>=stealth,->] (1) -- (2);
\draw[>=stealth,->] (1) -- (3);
\draw[>=stealth,->] (3) -- (4);
\draw[>=stealth,->] (3) -- (5);
\end{tikzpicture}
\end{figure}
\end{flushleft}
\subsection{Recursive Approach}
\begin{CJK*}{UTF8}{gbsn}
对于一个根节点来说,目标是将其左子节点变为根节点,右子节点变为左子节点,原来的根节点变为右子节点,那么我们判断这个root是否存在,且其有没有left child node,如果不满足这两个条件的话,直接返回即可,否则对left child node调用递归函数,直到到达most left开始翻转,翻转好most left child node后,开始回到上一个left child node继续翻转即可,直至翻转完The whole tree。
\par
另外注意到题目中给出的特性,所有的右节点要么没有,要么就是叶子节点并且一定有个兄弟节点。
\end{CJK*}
\setcounter{algorithm}{0}
\begin{algorithm}[H]
\caption{Recursive Approach}
\begin{algorithmic}[1]
\Procedure{UpsideDownBinaryTree}{$R$}
\If{$R=\texttt{null}$ \textbf{or} $R.\texttt{left}=\texttt{null}$}
\State \Return $R$ \Comment root or left child is empty
\EndIf
\State $l:=R.\texttt{left}$ \Comment The left child node of current root $R$
\State $r:=R.\texttt{right}$ \Comment The right child node of current root $R$
\State $\hat{R}:=\texttt{UpsideDownBinaryTree}(l)$ \Comment Recursively operating on left tree
\State $l.\texttt{left}\gets r$ \Comment Change left child node of $l$ to $r$
\State $l.\texttt{right}\gets R$ \Comment Change right child node $l$ to $R$
\State $R.\texttt{left}\gets\texttt{null}$ \Comment Set left and right child nodes of current root $R$ to empty
\State $R.\texttt{right}\gets\texttt{null}$
\State \Return $\hat{R}$
\EndProcedure
\end{algorithmic}
\end{algorithm}
\setcounter{lstlisting}{0}
\begin{lstlisting}[style=customc, caption={Recursive Approach}]
TreeNode *upsideDownBinaryTree ( TreeNode *root )
{
	if( !root || !root->left )
	{
		return root;
	}

	//save left child node
	auto l = root->left;
	//save right child node
	auto r = root->right;
	//Recursive call the function to get the 
	//root of upside-down left child tree
	auto l_root = upsideDownBinaryTree ( l );
	//set left child node of l to r
	l->left = r;
	//set right child node of l to current root
	l->right = root;
	//since root becomes leaf, 
	//set its left and child node to null
	root->left = nullptr;
	root->right = nullptr;
	//This is the new root of the changed tree
	return l_root;
}
\end{lstlisting}
