\section{156 --- Binary Tree Upside Down}
Given a binary tree with root $R$, where all the right nodes are either leaf nodes with a sibling (a left node that shares the same parent node) or empty, flip it upside down and turn it into a tree where the original right nodes turned into left leaf nodes. Return the new root.
\paragraph{For example:}
\begin{flushleft}
Given a binary tree
\begin{figure}[H]
\begin{tikzpicture}
[mynode/.style={draw,circle,minimum size=5mm, fill=gray!20!}]
\node(){};
\node[mynode](1) {1};
\node[mynode](2)[below=8mm of 1, xshift=-8mm] {2};
\node[mynode](3)[below=8mm of 1, xshift=8mm] {3};
\node[mynode](4)[below=8mm of 2, xshift=-8mm] {4};
\node[mynode](5)[below=8mm of 2, xshift=8mm] {5};
\draw[>=stealth,->] (1) -- (2);
\draw[>=stealth,->] (1) -- (3);
\draw[>=stealth,->] (2) -- (4);
\draw[>=stealth,->] (2) -- (5);
\end{tikzpicture}
\end{figure}
return the root of the binary tree
\begin{figure}[H]
\begin{tikzpicture}
[mynode/.style={draw,circle,minimum size=5mm, fill=gray!20!}]
\node(){};
\node[mynode](1) {4};
\node[mynode](2)[below=8mm of 1, xshift=-8mm] {5};
\node[mynode](3)[below=8mm of 1, xshift=8mm] {2};
\node[mynode](4)[below=8mm of 3, xshift=-8mm] {3};
\node[mynode](5)[below=8mm of 3, xshift=8mm] {1};
\draw[>=stealth,->] (1) -- (2);
\draw[>=stealth,->] (1) -- (3);
\draw[>=stealth,->] (3) -- (4);
\draw[>=stealth,->] (3) -- (5);
\end{tikzpicture}
\end{figure}
\end{flushleft}
\subsection{Recursive Approach}

对于一个根节点来说,目标是将其左子节点变为根节点,右子节点变为左子节点,原来的根节点变为右子节点,那么我们判断这个root是否存在,且其有没有left child node,如果不满足这两个条件的话,直接返回即可,否则对left child node调用递归函数,直到到达most left开始翻转,翻转好most left child node后,开始回到上一个left child node继续翻转即可,直至翻转完The whole tree。

另外注意到题目中给出的特性,所有的右节点要么没有,要么就是叶子节点并且一定有个兄弟节点。

%\setcounter{algorithm}{0}
%\begin{algorithm}[H]
%\caption{Recursive Approach}
%\begin{algorithmic}[1]
%\Procedure{UpsideDownBinaryTree}{$R$}
%\If{$R=\texttt{null}$ \textbf{or} $R.\texttt{left}=\texttt{null}$}
%\State \Return $R$ \Comment root or left child is empty
%\EndIf
%\State $l:=R.\texttt{left}$ \Comment The left child node of current root $R$
%\State $r:=R.\texttt{right}$ \Comment The right child node of current root $R$
%\State $\hat{R}:=\texttt{UpsideDownBinaryTree}(l)$ \Comment Recursively operating on left tree
%\State $l.\texttt{left}\gets r$ \Comment Change left child node of $l$ to $r$
%\State $l.\texttt{right}\gets R$ \Comment Change right child node $l$ to $R$
%\State $R.\texttt{left}\gets\texttt{null}$ \Comment Set left and right child nodes of current root $R$ to empty
%\State $R.\texttt{right}\gets\texttt{null}$
%\State \Return $\hat{R}$
%\EndProcedure
%\end{algorithmic}
%\end{algorithm}
\setcounter{lstlisting}{0}
\begin{lstlisting}[style=customc, caption={Recursive Approach}]
TreeNode *upsideDownBinaryTree( TreeNode *root )
{
    if( !root || !root->left )
    {
        return root;
    }
    //save left child node
    auto l = root->left;
    //save right child node
    auto r = root->right;
    //Recursive call the function to get the
    //root of upside-down left child tree
    auto ans = upsideDownBinaryTree( l );
    //left's left child node is its parent's right child: right
    l->left = r;
    //left's right child node is its parent: root
    l->right = root;
    //root become leaf
    root->left = nullptr;
    root->right = nullptr;
    //This is the new root of the changed tree
    return ans;
}
\end{lstlisting}

\subsection{Iterative}
In this approach, we visit each left child node and change its left and child node. More specifically, suppose we are visiting a node $t$ and its parent node is $p$

\begin{enumerate}
\item Change \fcj{t->left} to \fcj{p->right}.
\item Change \fcj{t->right} to \fcj{p}.
\end{enumerate}

Thus, we need to save current node and its right node before moving to current node's left child.


\begin{lstlisting}[style=customc, caption={Iterative}]
TreeNode* upsideDownBinaryTree( TreeNode* root )
{
    //save right child of last visited node
    TreeNode* x = nullptr;
    //save last visited node
    TreeNode* y = nullptr;
    auto node = root;
    while( node )
    {
        //save left child as next
        //for successive visit
        auto next = node->left;
        //change left child to the right child of its parent
        //i.e. the sibling: x
        node->left = x;
        //save its right child to x
        x = node->right;
        //change right child to its parent: y
        node->right = y;
        //save node to y
        y = node;
        //move to its previous left child
        node = next;
    }
    //now y is the new root
    return y;
}
\end{lstlisting}
