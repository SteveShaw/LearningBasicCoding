\section{143 --- Reorder List}
Given a singly linked list $L$: 
\begin{figure}[H]
\begin{tikzpicture}
[mynode/.style={draw,circle,minimum size=10mm, fill=gray!20!}]
\node(){};
\node[mynode](0) {\scalebox{0.6}{$L_0$}};
\node[mynode](1)[right=15mm of 0] {\scalebox{0.6}{$L_1$}};
\node[mynode](2)[right=15mm of 1] {$\ldots$};
\node[mynode](n1)[right=15mm of 2] {\scalebox{0.6}{$L_{n-1}$}};
\node[mynode](n2)[right=15mm of n1] {\scalebox{0.6}{$L_n$}};
\draw[>=stealth,->] (0) -- (1);
\draw[>=stealth,->] (1) -- (2);
\draw[>=stealth,->] (2) -- (n1);
\draw[>=stealth,->] (n1) -- (n2);
\end{tikzpicture}
\end{figure}
reorder it to:
\begin{figure}[H]
\begin{tikzpicture}
[mynode/.style={draw,circle,minimum size=10mm, fill=gray!20!}]
\node(){};
\node[mynode](0) {\scalebox{0.6}{$L_0$}};
\node[mynode](n2)[right=15mm of 0] {\scalebox{0.6}{$L_n$}};
\node[mynode](1)[right=15mm of n2] {\scalebox{0.6}{$L_1$}};
\node[mynode](n1)[right=15mm of 1] {\scalebox{0.6}{$L_{n-1}$}};
\node[mynode](2)[right=15mm of n1] {$\ldots$};
\draw[>=stealth,->] (0) -- (n2);
\draw[>=stealth,->] (n2) -- (1);
\draw[>=stealth,->] (1) -- (n1);
\draw[>=stealth,->] (n1) -- (2);
\end{tikzpicture}
\end{figure}
You may \textbf{not} modify the values in the list's nodes, only nodes itself may be changed.
\paragraph{Example 1:}
\begin{flushleft}
Given:
\begin{figure}[H]
\begin{tikzpicture}
[mynode/.style={draw,circle,minimum size=5mm, fill=gray!20!}]
\node(){};
\node[mynode](1) {1};
\node[mynode](2)[right=8mm of 1] {2};
\node[mynode](3)[right=8mm of 2] {3};
\node[mynode](4)[right=8mm of 3] {4};
\draw[>=stealth,->] (1) -- (2);
\draw[>=stealth,->] (2) -- (3);
\draw[>=stealth,->] (3) -- (4);
\end{tikzpicture}
\end{figure}
Reorder to:
\begin{figure}[H]
\begin{tikzpicture}
[mynode/.style={draw,circle,minimum size=5mm, fill=gray!20!}]
\node(){};
\node[mynode](1) {1};
\node[mynode](4)[right=8mm of 1] {4};
\node[mynode](2)[right=8mm of 4] {2};
\node[mynode](3)[right=8mm of 2] {3};
\draw[>=stealth,->] (1) -- (4);
\draw[>=stealth,->] (4) -- (2);
\draw[>=stealth,->] (2) -- (3);
\end{tikzpicture}
\end{figure}
\end{flushleft}
\paragraph{Example 2:}
\begin{flushleft}
Given:
\begin{figure}[H]
\begin{tikzpicture}
[mynode/.style={draw,circle,minimum size=5mm, fill=gray!20!}]
\node(){};
\node[mynode](1) {1};
\node[mynode](2)[right=8mm of 1] {2};
\node[mynode](3)[right=8mm of 2] {3};
\node[mynode](4)[right=8mm of 3] {4};
\node[mynode](5)[right=8mm of 4] {5};
\draw[>=stealth,->] (1) -- (2);
\draw[>=stealth,->] (2) -- (3);
\draw[>=stealth,->] (3) -- (4);
\draw[>=stealth,->] (4) -- (5);
\end{tikzpicture}
\end{figure}
Reorder to:
\begin{figure}[H]
\begin{tikzpicture}
[mynode/.style={draw,circle,minimum size=5mm, fill=gray!20!}]
\node(){};
\node[mynode](1) {1};
\node[mynode](5)[right=8mm of 1] {5};
\node[mynode](2)[right=8mm of 5] {2};
\node[mynode](4)[right=8mm of 2] {4};
\node[mynode](3)[right=8mm of 4] {3};
\draw[>=stealth,->] (1) -- (5);
\draw[>=stealth,->] (5) -- (2);
\draw[>=stealth,->] (2) -- (4);
\draw[>=stealth,->] (4) -- (3);
\end{tikzpicture}
\end{figure}
\end{flushleft}
\subsection{Two Pointers + Reverse}
首先用快慢指针找到链表的中间位置的node,然后从这个位置,将这个链表断开。然后对后半部分的链表进行reverse操作,最后合并两个链表。
\setcounter{lstlisting}{0}
\begin{lstlisting}[style=customc, caption={Two Pointers}]
void reorderList( ListNode* head )
{
    if( !head )
    {
        return;
    }
    //find middle node
    auto slow = head;
    auto fast = head;
    while( fast && fast->next )
    {
        slow = slow->next;
        fast = fast->next->next;
    }
    //right is the start of right half
    auto right = slow->next;
    //break left and right
    slow->next = nullptr;
    //reverse right half
    ListNode* pre = nullptr;
    auto cur = right;
    while( cur )
    {
        auto next = cur->next;
        cur->next = pre;
        pre = cur;
        cur = next;
    }
    //interleave with left and right
    auto left = head;
    right = pre;
    while( left && right )
    {
        //change left->next
        auto next = left->next;
        left->next = right;
        left = next;
        //change right->next
        next = right->next;
        right->next = left;
        right = next;
    }

}
\end{lstlisting}