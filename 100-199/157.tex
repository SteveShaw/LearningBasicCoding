\section{157 --- Read N Characters Given Read4}
Given a file and assume that you can only read the file using a given method \fcj{read4}, implement a method to read $n$ characters.
 

\paragraph{Method read4:}

\begin{flushleft}
The API \fcj{read4} reads 4 consecutive characters from the file, then writes those characters into the buffer array \fcj{buf}.

The return value is the number of actual characters read.

Note that \fcj{read4()} has its own file pointer, much like \fcj{FILE *fp} in C.

Definition of \fcj{read4}: \fcj{int read4(char[] buf)}
\end{flushleft}

\paragraph{Note:} 

\begin{itemize}
\item \fcj{buf[]} is destination not source, the results from \fcj{read4} will be copied to \fcj{buf[]}
\end{itemize}

Below is a high level example of how read4 works:

\begin{flushleft}
\fcj{File file("abcdefghijk"); // initially file pointer (fp) points to 'a'}

\fcj{char[] buf = new char[4]; // Create buffer with enough space to store characters}

\fcj{read4(buf); // read4 returns 4. Now buf = "abcd", fp points to 'e'}

\fcj{read4(buf); // read4 returns 4. Now buf = "efgh", fp points to 'i'}

\fcj{read4(buf); // read4 returns 3. Now buf = "ijk", fp points to end of file}
\end{flushleft}
 
\paragraph{Method read:}

\begin{flushleft}
By using the \fcj{read4} method, implement the method \fcj{read} that reads $n$ characters from the file and store it in the buffer array \fcj{buf}. Consider that you \textbf{cannot} manipulate the file directly.

The return value is the number of actual characters read.

Definition of \fcj{read}: \fcj{int read(char[] buf, int n)}
\end{flushleft}

\paragraph{Note:} 

\begin{itemize}
\item \fcj{buf[]} is destination not source, you will need to write the results to \fcj{buf[]}.
\end{itemize}
 

\paragraph{Example 1:}

\begin{flushleft}
\textbf{Input}: \fcj{file = "abc"}, $n = 4$

\textbf{Output}: 3

\textbf{Explanation}: 

After calling your \fcj{read} method, \fcj{buf} should contain \fcj{"abc"}. We read a total of 3 characters from the file, so return 3. Note that \fcj{"abc"} is the file's content, not \fcj{buf}. \fcj{buf} is the destination buffer that you will have to write the results to.

\end{flushleft}
\paragraph{Example 2:}

\begin{flushleft}
\textbf{Input}: \fcj{file = "abcde"}, $n = 5$

\textbf{Output}: 5

\textbf{Explanation}: 

After calling your \fcj{read} method, buf should contain \fcj{"abcde"}. We read a total of 5 characters from the file, so return 5.

\end{flushleft}

\paragraph{Example 3:}

\begin{flushleft}
\textbf{Input}: \fcj{file = "abcdABCD1234"}, $n = 12$

\textbf{Output}: 12

\textbf{Explanation}: After calling your read method, \fcj{buf} should contain \fcj{"abcdABCD1234"}. We read a total of 12 characters from the file, so return 12.
\end{flushleft}

\paragraph{Example 4:}

\begin{flushleft}
\textbf{Input}: \fcj{file = "leetcode"}, $n = 5$

\textbf{Output}: 5

\textbf{Explanation}: After calling your read method, \fcj{buf} should contain \fcj{"leetc"}. We read a total of 5 characters from the file, so return 5.
\end{flushleft}
 

\paragraph{Note:}

\begin{itemize}
\item Consider that you cannot manipulate the file directly, the file is only accessible for \fcj{read4} but not for \fcj{read}.
\item The \fcj{read} function will only be called once for each test case.
\item You may assume the destination buffer array, \fcj{buf}, is guaranteed to have enough space for storing $n$ characters.
\end{itemize}
\subsection{Analysis}

比较简单的题目,每次调用\fcj{read4},直到返回值变为零,表示已经全部读完。另外每次调用\fcj{read4}的时候,将\fcj{p}指针根据返回值调整到下一个开始读取的位置。

%\setcounter{algorithm}{0}
%\begin{algorithm}[H]
%\caption{Implementation read By read4}
%\begin{algorithmic}[1]
%\Procedure{Read}{$p, n$}
%\State $x:=0$ \Comment The read result of each calling \texttt{read4}
%\While{$\texttt{true}$}
%\State $y:= \texttt{read4(p+x)}$ \Comment $y$ is the count of characters read from the file
%\If{$y = 0$}
%\State \texttt{break} \Comment Finished reading from the file
%\EndIf
%\State $x\gets x+y$ \Comment Update current total counts of characters 
%\EndWhile
%\State \Return $\min(x, n)$ \Comment The file may contain less than $n$ characters
%\EndProcedure
%\end{algorithmic}
%\end{algorithm}

\setcounter{lstlisting}{0}
\begin{lstlisting}[style=customc, caption={Pointer}]
// Forward declaration of the read4 API.
int read4( char *buf );
/**
 * @param buf Destination buffer
 * @param n   Number of characters to read
 * @return    The number of actual characters read
 */
int read( char *buf, int n )
{
    int total = 0;
    while( total < n )
    {
        int x = read4( buf );
        if( x == 0 )
        {
            //no content
            break;
        }
        //add to total
        total += x;
        //move the buffer pointer
        buf += x;
    }
    //total may be large than n
    //we return the minimum of total and n
    return ( min )( total, n );
}
\end{lstlisting}