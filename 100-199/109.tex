\section{109 --- Convert Sorted List to Binary Search Tree}
Given a singly linked list $L$ where elements are sorted in ascending order, convert it to a height balanced BST.
\par
For this problem, a height-balanced binary tree is defined as a binary tree in which the depth of the two subtrees of every node never differ by more than 1.
\paragraph{Example:}
\begin{flushleft}
\textbf{Input}: $L = [-10,-3,0,5,9]$,
\\
One possible answer is:
\begin{figure}[H]
\begin{tikzpicture}
[mynode/.style={draw,circle,minimum size=10mm, fill=gray!20!}]
\node(){};
\node[mynode](0) {0};
\node[mynode](3) [below = 8mm of 0, xshift=-8mm] {-3};
\node[mynode](9) [below = 8mm of 0, xshift=8mm] {9};
\node[mynode](10) [below = 8mm of 3, xshift=-8mm] {-10};
\node[mynode](5) [below = 8mm of 9, xshift=-8mm] {5};
\draw[>=stealth,->] (0) -- (3);
\draw[>=stealth,->] (0) -- (9);
\draw[>=stealth,->] (3) -- (10);
\draw[>=stealth,->] (9) -- (5);
\end{tikzpicture}
\end{figure}
 \end{flushleft}
 \subsection{Recursion}
 \begin{CJK*}{UTF8}{gbsn}
 链表的查找中间点可以通过快慢指针来操。找到中点后,要以中点的值建立一个根节点,然后需要把原链表断开,分为前后两个链表,都不能包含原中点,然后再分别对这两个链表递归调用原函数,分别连上左右子节点即可。因此,仍然需要一个辅助函数来进行递归操作,这个辅助函数有两个输入参数,即起始节点和尾部节点后的第一个节点。
\end{CJK*}
\subsubsection{Algorithm}
\setcounter{algorithm}{0}
\begin{algorithm}[H]
\caption{Recursion}
\begin{algorithmic}[1]
\Procedure{SortedListToBST}{$L$}
\If{$L=\texttt{null}$}
\State \Return \texttt{null}
\EndIf
\State $T:=\texttt{CreateTree}(L, \texttt{null})$
\EndProcedure
\end{algorithmic}
\end{algorithm}
 \begin{CJK*}{UTF8}{gbsn}
 \texttt{CreateTree}根据输入的起始节点$\alpha$和尾部节点后的第一个节点$\beta$构建二叉搜索树
\end{CJK*}
\begin{algorithm}[H]
\caption{Recursively Build Binary Search Tree}
\begin{algorithmic}[1]
\Function{CreateTree}{$\alpha, \beta$}
\If{$\alpha=\beta$} \Comment Cannot compare larger/smaller for linked list nodes
\State \Return \texttt{null}
\EndIf
\State $F:=\alpha$ \Comment Fast pointer
\State $S:=\alpha$ \Comment Slow pointer
\While{$\texttt{Next}(F)\neq \beta$ \textbf{and} $\texttt{Next}(\texttt{Next}(F))\neq \beta$ }
\State $F\gets \texttt{Next}(\texttt{Next}(F))$ \Comment Forward fast pointer by 2 nodes
\State $S\gets \texttt{Next}(S)$ \Comment Forward slow pointer by 1 node
\EndWhile
\State Create a tree node $T$ with value equal to $\texttt{Value}(S)$ 
\State $\texttt{LEFT}(T)\gets \texttt{CreateTree}(\alpha, S)$ \Comment Create left subtree
\State $\texttt{RIGHT}(T)\gets \texttt{CreateTree}(\texttt{Next}(S), \beta)$ \Comment Create right subtree
\State \Return $T$
\EndFunction
\end{algorithmic}
\end{algorithm} 