\section{114 --- Flatten Binary Tree to Linked List}
Given a binary tree $T$, flatten it to a linked list in-place.
\par
For example, given the following tree:
\begin{figure}[H]
\begin{tikzpicture}
[mynode/.style={draw,circle,minimum size=5mm, fill=gray!20!}]
\node(){};
\node[mynode](1) {1};
\node[mynode](2) [below = 8mm of 1, xshift=-10mm] {2};
\node[mynode](5) [below = 8mm of 1, xshift=10mm] {5};
\node[mynode](3) [below = 8mm of 2, xshift=-8mm] {3};
\node[mynode](4) [below = 8mm of 2, xshift=8mm] {4};
\node[mynode](6) [below = 8mm of 5, xshift=8mm] {6};
\draw[>=stealth,->] (1) -- (2);
\draw[>=stealth,->] (1) -- (5);
\draw[>=stealth,->] (2) -- (3);
\draw[>=stealth,->] (2) -- (4);
\draw[>=stealth,->] (5) -- (6);
\end{tikzpicture}
\end{figure}
The flattened tree should look like:
\begin{figure}[H]
\begin{tikzpicture}
[mynode/.style={draw,circle,minimum size=5mm, fill=gray!20!}]
\node(){};
\node[mynode](1) {1};
\node[mynode](2) [below = 8mm of 1, xshift=5mm] {2};
\node[mynode](3) [below = 8mm of 2, xshift=5mm] {3};
\node[mynode](4) [below = 8mm of 3, xshift=5mm] {4};
\node[mynode](5) [below = 8mm of 4, xshift=5mm] {5};
\node[mynode](6) [below = 8mm of 5, xshift=5mm] {6};
\draw[>=stealth,->] (1) -- (2);
\draw[>=stealth,->] (2) -- (3);
\draw[>=stealth,->] (3) -- (4);
\draw[>=stealth,->] (4) -- (5);
\draw[>=stealth,->] (5) -- (6);
\end{tikzpicture}
\end{figure}
\subsection{Depth First Search}
\begin{CJK*}{UTF8}{gbsn}
根据展开后形成的链表的顺序分析出是使用preorder traverse,如果用recursive depth first search,那么首先对左右子树分别进行flatten,接着保存当前节点的右子树,然后将当前节点的右子树替换成左子树,然后把当前节点的左子树设为\texttt{nullptr},接着,沿着新的右子树(其实就是之前的左子树)的right一直走到叶子节点,然后将这个叶子节点的右子树设为之前保存的原来的右子树。
\par
而对于iterative方法,则从root开始出发,先检测root的left node是否存在,如存在则将root和其right node断开,将left node及其子树一起连到原right node的位置,然后再把原right node连到原来的left node最右面的right node之后
\end{CJK*}
\subsubsection{Algorithm}
\setcounter{algorithm}{0}
\begin{algorithm}[H]
\caption{Depth First Search --- Recursion}
\begin{algorithmic}[1]
\Procedure{Flattern}{$T$}
\If{$T=\texttt{nullptr}$}
\State \Return
\EndIf
\State $\texttt{Flattern}(\texttt{Left}(T))$ \Comment Recursive on left subtree
\State $\texttt{Flattern}(\texttt{Right}(T))$ \Comment Recursive on right subtree
\State $x:=\texttt{Right}(T)$ \Comment Save right subtree of $T$
\State $\texttt{Right}(T)\gets \texttt{Left}(T)$ \Comment Replace right subtree by left subtree
\State $\texttt{Left}(T)\gets \texttt{nullptr}$ \Comment Set left subtree of $T$ as empty
\State $y:=T$
\While{$\texttt{Right}(y)$}
\State $y\gets \texttt{Right}(y)$ \Comment Move right until right child is null
\EndWhile
\State $\texttt{Right}(y)\gets x$ \Comment Link previous right child
\EndProcedure
\end{algorithmic}
\end{algorithm}
The process of transformation of the given example is shown as below:
\begin{figure}[H]
\begin{tikzpicture}
[mynode/.style={draw,circle,minimum size=5mm, fill=gray!20!}]
\node(){};
\node[mynode](1) {1};
\node[mynode](2) [below = 8mm of 1, xshift=-10mm] {2};
\node[mynode](5) [below = 8mm of 1, xshift=10mm] {5};
\node[mynode](3) [below = 8mm of 2, xshift=-8mm] {3};
\node[mynode](4) [below = 8mm of 2, xshift=8mm] {4};
\node[mynode](6) [below = 8mm of 5, xshift=8mm] {6};
\draw[>=stealth,->] (1) -- (2);
\draw[>=stealth,->] (1) -- (5);
\draw[>=stealth,->] (2) -- (3);
\draw[>=stealth,->] (2) -- (4);
\draw[>=stealth,->] (5) -- (6);
\end{tikzpicture}
\caption{Original Tree}
\end{figure}
\begin{figure}[H]
\begin{tikzpicture}
[mynode/.style={draw,circle,minimum size=5mm, fill=gray!20!}]
\node(){};
\node[mynode](1) {1};
\node[mynode](2) [below = 8mm of 1, xshift=-10mm] {2};
\node[mynode](5) [below = 8mm of 1, xshift=10mm] {5};
\node[mynode](3) [below = 8mm of 2, xshift=8mm] {3};
\node[mynode](4) [below = 8mm of 3, xshift=8mm] {4};
\node[mynode](6) [below = 8mm of 5, xshift=8mm] {6};
\draw[>=stealth,->] (1) -- (2);
\draw[>=stealth,->] (1) -- (5);
\draw[>=stealth,->] (2) -- (3);
\draw[>=stealth,->] (3) -- (4);
\draw[>=stealth,->] (5) -- (6);
\end{tikzpicture}
\caption{Recursive On Left And Right Subtree}
\end{figure}
\begin{figure}[H]
\begin{tikzpicture}
[mynode/.style={draw,circle,minimum size=5mm, fill=gray!20!}]
\node(){};
\node[mynode](1) {1};
\node[mynode](2) [below = 8mm of 1, xshift=5mm] {2};
\node[mynode](3) [below = 8mm of 2, xshift=5mm] {3};
\node[mynode](4) [below = 8mm of 3, xshift=5mm] {4};
\node[mynode](5) [below = 8mm of 4, xshift=5mm] {5};
\node[mynode](6) [below = 8mm of 5, xshift=5mm] {6};
\draw[>=stealth,->] (1) -- (2);
\draw[>=stealth,->] (2) -- (3);
\draw[>=stealth,->] (3) -- (4);
\draw[>=stealth,->] (4) -- (5);
\draw[>=stealth,->] (5) -- (6);
\end{tikzpicture}
\caption{Replace Right By Left Subtree}
\end{figure}
The following is the implementation of iterative way
\begin{algorithm}[H]
\caption{Iterative Way}
\begin{algorithmic}[1]
\Procedure{Flattern}{$T$}
\State $x:=T$
\While{$x\neq \texttt{null}$}
\If{$\texttt{Left}(x)\neq null$}
\State $y:=\texttt{Left}(x)$
\While{$\texttt{Right}(y)\neq \texttt{null}$} \Comment Move to the rightmost node of left subtree
\State $y\gets \texttt{Right}(y)$
\EndWhile
\State $\texttt{Right}(y)\gets \texttt{Right}(x)$ \Comment Set right subtree of $x$ as $y$'s right child
\State $\texttt{Right}(x) \gets \texttt{Left}(x)$ \Comment Update right subtree of $x$ as its left subtree
\State $\texttt{Left}(x) \gets \texttt{null}$ \Comment Update left subtree of $x$ as empty
\EndIf
\EndWhile
\State $x\gets \texttt{Right}(x)$ \Comment Move to right subtree of $x$
\EndProcedure
\end{algorithmic}
\end{algorithm}
The process of transformation for the given example tree is shown as below
\begin{figure}[H]
\begin{tikzpicture}
[mynode/.style={draw,circle,minimum size=5mm, fill=gray!20!}]
\node(){};
\node[mynode](1) {1};
\node[mynode](2) [below = 8mm of 1, xshift=8mm] {\textcolor{red}{2}};
\node[mynode](3) [below = 8mm of 2, xshift=-8mm] {\textcolor{red}{3}};
\node[mynode](4) [below = 8mm of 2, xshift=8mm] {\textcolor{red}{4}};
\node[mynode](5) [below = 8mm of 4, xshift=8mm] {5};
\node[mynode](6) [below = 8mm of 5, xshift=8mm] {6};
\draw[>=stealth,->] (1) -- (2);
\draw[>=stealth,->] (2) -- (3);
\draw[>=stealth,->] (2) -- (4);
\draw[>=stealth,->] (4) -- (5);
\draw[>=stealth,->] (5) -- (6);
\end{tikzpicture}
\caption{Change Left And Right Subtree}
\end{figure}
\begin{figure}[H]
\begin{tikzpicture}
[mynode/.style={draw,circle,minimum size=5mm, fill=gray!20!}]
\node(){};
\node[mynode](1) {1};
\node[mynode](2) [below = 8mm of 1, xshift=5mm] {2};
\node[mynode](3) [below = 8mm of 2, xshift=5mm] {\textcolor{red}{3}};
\node[mynode](4) [below = 8mm of 3, xshift=5mm] {4};
\node[mynode](5) [below = 8mm of 4, xshift=5mm] {5};
\node[mynode](6) [below = 8mm of 5, xshift=5mm] {6};
\draw[>=stealth,->] (1) -- (2);
\draw[>=stealth,->] (2) -- (3);
\draw[>=stealth,->] (3) -- (4);
\draw[>=stealth,->] (4) -- (5);
\draw[>=stealth,->] (5) -- (6);
\end{tikzpicture}
\caption{The Final Result}
\end{figure}