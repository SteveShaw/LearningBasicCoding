\section{114 --- Flatten Binary Tree to Linked List}
Given a binary tree $T$, flatten it to a linked list in-place.
\par
For example, given the following tree:
\begin{figure}[H]
\begin{tikzpicture}
[mynode/.style={draw,circle,minimum size=5mm, fill=gray!20!}]
\node(){};
\node[mynode](1) {1};
\node[mynode](2) [below = 8mm of 1, xshift=-10mm] {2};
\node[mynode](5) [below = 8mm of 1, xshift=10mm] {5};
\node[mynode](3) [below = 8mm of 2, xshift=-8mm] {3};
\node[mynode](4) [below = 8mm of 2, xshift=8mm] {4};
\node[mynode](6) [below = 8mm of 5, xshift=8mm] {6};
\draw[>=stealth,->] (1) -- (2);
\draw[>=stealth,->] (1) -- (5);
\draw[>=stealth,->] (2) -- (3);
\draw[>=stealth,->] (2) -- (4);
\draw[>=stealth,->] (5) -- (6);
\end{tikzpicture}
\end{figure}
The flattened tree should look like:
\begin{figure}[H]
\begin{tikzpicture}
[mynode/.style={draw,circle,minimum size=5mm, fill=gray!20!}]
\node(){};
\node[mynode](1) {1};
\node[mynode](2) [below = 8mm of 1, xshift=5mm] {2};
\node[mynode](3) [below = 8mm of 2, xshift=5mm] {3};
\node[mynode](4) [below = 8mm of 3, xshift=5mm] {4};
\node[mynode](5) [below = 8mm of 4, xshift=5mm] {5};
\node[mynode](6) [below = 8mm of 5, xshift=5mm] {6};
\draw[>=stealth,->] (1) -- (2);
\draw[>=stealth,->] (2) -- (3);
\draw[>=stealth,->] (3) -- (4);
\draw[>=stealth,->] (4) -- (5);
\draw[>=stealth,->] (5) -- (6);
\end{tikzpicture}
\end{figure}
\subsection{Depth First Search}
根据展开后形成的链表的顺序分析出是使用preorder traverse,如果用recursive depth first search,那么首先对左右子树分别进行flatten,接着保存当前节点的右子树,然后将当前节点的右子树替换成左子树,然后把当前节点的左子树设为\texttt{nullptr},接着,沿着新的右子树(其实就是之前的左子树)的right一直走到叶子节点,然后将这个叶子节点的右子树设为之前保存的原来的右子树。

而对于iterative方法,则从root开始出发,先检测root的left node是否存在,如存在则将root和其right node断开,将left node及其子树一起连到原right node的位置,然后再把原right node连到原来的left node最右面的right node之后
\setcounter{lstlisting}{0}
\begin{lstlisting}[style=customc, caption={Recursion}]
void flatten( TreeNode* root )
{
    if( !root )
    {
        return;
    }
    //flatten left child
    flatten( root->left );
    //flatten right child
    flatten( root->right );
    //save current right child first
    auto x = root->right;
    //move left child to right child
    root->right = root->left;
    root->left = nullptr;
    //link to previous right child
    auto node = root;
    while( node->right )
    {
        node = node->right;
    }
    node->right = x;
}
\end{lstlisting}
The process of transformation of the given example is shown as below:
\begin{figure}[H]
\begin{tikzpicture}
[mynode/.style={draw,circle,minimum size=5mm, fill=gray!20!}]
\node(){};
\node[mynode](1) {1};
\node[mynode](2) [below = 8mm of 1, xshift=-10mm] {2};
\node[mynode](5) [below = 8mm of 1, xshift=10mm] {5};
\node[mynode](3) [below = 8mm of 2, xshift=-8mm] {3};
\node[mynode](4) [below = 8mm of 2, xshift=8mm] {4};
\node[mynode](6) [below = 8mm of 5, xshift=8mm] {6};
\draw[>=stealth,->] (1) -- (2);
\draw[>=stealth,->] (1) -- (5);
\draw[>=stealth,->] (2) -- (3);
\draw[>=stealth,->] (2) -- (4);
\draw[>=stealth,->] (5) -- (6);
\end{tikzpicture}
\caption{Original Tree}
\end{figure}
\begin{figure}[H]
\begin{tikzpicture}
[mynode/.style={draw,circle,minimum size=5mm, fill=gray!20!}]
\node(){};
\node[mynode](1) {1};
\node[mynode](2) [below = 8mm of 1, xshift=-10mm] {2};
\node[mynode](5) [below = 8mm of 1, xshift=10mm] {5};
\node[mynode](3) [below = 8mm of 2, xshift=8mm] {3};
\node[mynode](4) [below = 8mm of 3, xshift=8mm] {4};
\node[mynode](6) [below = 8mm of 5, xshift=8mm] {6};
\draw[>=stealth,->] (1) -- (2);
\draw[>=stealth,->] (1) -- (5);
\draw[>=stealth,->] (2) -- (3);
\draw[>=stealth,->] (3) -- (4);
\draw[>=stealth,->] (5) -- (6);
\end{tikzpicture}
\caption{Recursive On Left And Right Subtree}
\end{figure}
\begin{figure}[H]
\begin{tikzpicture}
[mynode/.style={draw,circle,minimum size=5mm, fill=gray!20!}]
\node(){};
\node[mynode](1) {1};
\node[mynode](2) [below = 8mm of 1, xshift=5mm] {2};
\node[mynode](3) [below = 8mm of 2, xshift=5mm] {3};
\node[mynode](4) [below = 8mm of 3, xshift=5mm] {4};
\node[mynode](5) [below = 8mm of 4, xshift=5mm] {5};
\node[mynode](6) [below = 8mm of 5, xshift=5mm] {6};
\draw[>=stealth,->] (1) -- (2);
\draw[>=stealth,->] (2) -- (3);
\draw[>=stealth,->] (3) -- (4);
\draw[>=stealth,->] (4) -- (5);
\draw[>=stealth,->] (5) -- (6);
\end{tikzpicture}
\caption{Replace Right By Left Subtree}
\end{figure}

The following is the implementation of iterative way

\begin{lstlisting}[style=customc, caption={Iterative Approach}]
void flatten( TreeNode* root )
{
    auto node = root;
    while( node )
    {
        if( node->left )
        {
            auto t = node->left;
            //get the rightmost node of left child
            while( t->right )
            {
                t = t->right;
            }
            //link to current node's right child
            t->right = node->right;
            node->right = node->left;
            node->left = nullptr;
        }
        node = node->right; //move to right child
    }
}
\end{lstlisting}
The process of transformation for the given example tree is shown as below
\begin{figure}[H]
\begin{tikzpicture}
[mynode/.style={draw,circle,minimum size=5mm, fill=gray!20!}]
\node(){};
\node[mynode](1) {1};
\node[mynode](2) [below = 8mm of 1, xshift=8mm] {\textcolor{red}{2}};
\node[mynode](3) [below = 8mm of 2, xshift=-8mm] {\textcolor{red}{3}};
\node[mynode](4) [below = 8mm of 2, xshift=8mm] {\textcolor{red}{4}};
\node[mynode](5) [below = 8mm of 4, xshift=8mm] {5};
\node[mynode](6) [below = 8mm of 5, xshift=8mm] {6};
\draw[>=stealth,->] (1) -- (2);
\draw[>=stealth,->] (2) -- (3);
\draw[>=stealth,->] (2) -- (4);
\draw[>=stealth,->] (4) -- (5);
\draw[>=stealth,->] (5) -- (6);
\end{tikzpicture}
\caption{Change Left And Right Subtree}
\end{figure}
\begin{figure}[H]
\begin{tikzpicture}
[mynode/.style={draw,circle,minimum size=5mm, fill=gray!20!}]
\node(){};
\node[mynode](1) {1};
\node[mynode](2) [below = 8mm of 1, xshift=5mm] {2};
\node[mynode](3) [below = 8mm of 2, xshift=5mm] {\textcolor{red}{3}};
\node[mynode](4) [below = 8mm of 3, xshift=5mm] {4};
\node[mynode](5) [below = 8mm of 4, xshift=5mm] {5};
\node[mynode](6) [below = 8mm of 5, xshift=5mm] {6};
\draw[>=stealth,->] (1) -- (2);
\draw[>=stealth,->] (2) -- (3);
\draw[>=stealth,->] (3) -- (4);
\draw[>=stealth,->] (4) -- (5);
\draw[>=stealth,->] (5) -- (6);
\end{tikzpicture}
\caption{The Final Result}
\end{figure}