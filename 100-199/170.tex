\section{170 --- Two Sum III: Data structure design}
Design and implement a \fcj{TwoSum} class. It should support the following operations:\fcj{add} and \fcj{find}.
\begin{itemize}
\item \fcj{add} --- Add the number to an internal data structure.
\item \fcj{find} --- Find if there exists any pair of numbers which sum is equal to the value.
\end{itemize}

\paragraph{Example 1:}

\begin{flushleft}
\fcj{add(1); add(3); add(5);}

\fcj{find(4) -> true}

\fcj{find(7) -> false}
\end{flushleft}

\paragraph{Example 2:}

\begin{flushleft}
\fcj{add(3); add(1); add(2);}

\fcj{find(3) -> true}

\fcj{find(6) -> false}
\end{flushleft}

\subsection{Hash Map}
\textbf{Easy} problem.\setcounter{lstlisting}{0}
\begin{lstlisting}[style=customc, caption={Hash Map}]
class TwoSum
{
public:
    /** Initialize your data structure here. */
    TwoSum() {}
    /** Add the number to an internal data structure.. */
    void add( int number )
    {
        m[number]++;
    }
    /** Find if there exists any pair of numbers which sum is equal to the value. */
    bool find( int value )
    {
        for( const auto[k, v] : m )
        {
            auto it = m.find( value - k );
            if( it != m.end() )
            {
                const auto& [d, count] = *it;
                if( d != k )
                {
                    return true;
                }
                if( count > 1 )
                {
                    //duplicate number
                    return true;
                }
            }
        }
        return false;
    }
    unordered_map<int, int> m;
};
\end{lstlisting}

