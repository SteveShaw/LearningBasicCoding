\section{158 --- Read N Characters Given Read4 II - Call multiple times }
Given a file and assume that you can only read the file using a given method \fcj{read4}, implement a method \fcj{read} to read $n$ characters. Your method \fcj{read} may be called \textbf{multiple times}.

\paragraph{Method read4:}

\begin{flushleft}
The API \fcj{read4} reads 4 consecutive characters from the file, then writes those characters into the buffer array \fcj{buf}.

The return value is the number of actual characters read.

Note that \fcj{read4()} has its own file pointer, much like \fcj{FILE *fp} in C.

Definition of \fcj{read4}: \fcj{int read4(char[] buf)}
\end{flushleft}

\paragraph{Note:} 

\begin{itemize}
\item \fcj{buf[]} is destination not source, the results from \fcj{read4} will be copied to \fcj{buf[]}
\end{itemize}

Below is a high level example of how read4 works:

\begin{flushleft}
\fcj{File file("abcdefghijk"); // initially file pointer (fp) points to 'a'}

\fcj{char[] buf = new char[4]; // Create buffer with enough space to store characters}

\fcj{read4(buf); // read4 returns 4. Now buf = "abcd", fp points to 'e'}

\fcj{read4(buf); // read4 returns 4. Now buf = "efgh", fp points to 'i'}

\fcj{read4(buf); // read4 returns 3. Now buf = "ijk", fp points to end of file}
\end{flushleft}
 
\paragraph{Method read:}

\begin{flushleft}
By using the \fcj{read4} method, implement the method \fcj{read} that reads $n$ characters from the file and store it in the buffer array \fcj{buf}. Consider that you \textbf{cannot} manipulate the file directly.

The return value is the number of actual characters read.

Definition of \fcj{read}: \fcj{int read(char[] buf, int n)}
\end{flushleft}

\paragraph{Note:} 

\begin{itemize}
\item \fcj{buf[]} is destination not source, you will need to write the results to \fcj{buf[]}.
\end{itemize}


\paragraph{Example 1:}

\begin{lstlisting}[style=customc]
File file( "abc" );
Solution sol;
// Assume buf is allocated and guaranteed to have enough space for storing all characters from the file.
sol.read( buf, 1 ); // After calling your read method, buf should contain "a". We read a total of 1 character from the file, so return 1.
sol.read( buf, 2 ); // Now buf should contain "bc". We read a total of 2 characters from the file, so return 2.
sol.read( buf, 1 ); // We have reached the end of file, no more characters can be read. So return 0.
\end{lstlisting}

\paragraph{Example 2:}
\begin{lstlisting}[style=customc]
File file( "abc" );
Solution sol;
sol.read( buf, 4 ); // After calling your read method, buf should contain "abc". We read a total of 3 characters from the file, so return 3.
sol.read( buf, 1 ); // We have reached the end of file, no more characters can be read. So return 0.
\end{lstlisting}

 

\paragraph{Note:}

\begin{itemize}
\item Consider that you \textbf{cannot} manipulate the file directly, the file is only accessible for \fcj{read4} but \textbf{not} for \fcj{read}.
\item The \fcj{read} function may be called \textbf{multiple times}.
\item You may assume the destination buffer array, \fcj{buf}, is guaranteed to have enough space for storing $n$ characters.
\item It is guaranteed that in a given test case the same buffer \fcj{buf} is called by \fcj{read}.
\end{itemize}
\subsection{Analysis}

We should notice that in each calling function \fcj{read}, the destination \fcj{buf} is a new buffer. Thus, we only need to make sure that current calling \fcj{read} can get the correct result from last read's offset by considering $n$ letters are required.

We maintain two pointers \fcj{left} and \fcj{right} with a 4 byte buffer \fcj{chars}. When \fcj{left = right}, we need to read new characters through \fcj{read4} into \fcj{chars} and update \fcj{right} by the return value. Otherwise, we output one letter from \fcj{chars} to \fcj{buf} and increment \fcj{left}.

\setcounter{lstlisting}{0}
\begin{lstlisting}[style=customc, caption={Two Pointers}]
int read4( char *buf );
class Solution
{
public:
    /**
     * @param buf Destination buffer
     * @param n   Number of characters to read
     * @return    The number of actual characters read
     */
    int read( char *buf, int n )
    {
        for( int i = 0; i < n; ++i )
        {
            if( left == right )
            {
                //read letters using read4
                right = read4( chars );
                //reset left
                left = 0;
                if( right == 0 )
                {
                    //no letter is read
                    //return actual letters read
                    return i;
                }
            }
            //output from chars to buf
            buf[i] = chars[left];
            //increment left
            ++left;
        }
        return n;
    }
    int left = 0;
    int right = 0;
    char chars[4];
};
\end{lstlisting}