\section{158 --- Read N Characters Given Read4 II - Call multiple times }
The API: \texttt{int read4(char *buf)} reads 4 characters at a time from a file.
\par
The return value is the actual number of characters read.
\par
For example, it returns 3 if there is only 3 characters left in the file.
\par
By using the read4 API, implement the function \texttt{int read(char *p, int n)} that reads $n$ characters from the file.
\paragraph{Note:}
\begin{itemize}
\item The \texttt{read} function may be called \texttt{multiple} times..
\end{itemize}
\subsection{Analysis}
\begin{CJK*}{UTF8}{gbsn}
这道题是之前157 --- \textbf{Read N Characters Given Read4}的拓展,不同点在于这里\texttt{read}函数可以调用多次,那么难度就增加了,例如:要读取的文件buffer为\texttt{ab},然后执行read(1),read(2),那么第一次调用\texttt{read(1)}后,从buffer中读出一个字符,那么就是第一个字符\texttt{a},然后又调用了一个\texttt{read(2)},但是buffer中只剩一个\texttt{b}了,所以就把取出的结果就是\texttt{b}。
\par
另外一个例子更能说明不同: 文件的buffer为\texttt{a}, 分别执行\texttt{read(0)},\texttt{read(1)},\texttt{read(2)}。第一次调用read(0),不会读取任何字符。第二次调用\texttt{read(1)},取一个字符,而buffer中只有一个字符\texttt{a},因此返回\texttt{a}。然后调用read(2),但是这时候buffer中没有字符了,所以同样不会读取任何字符。
\end{CJK*}