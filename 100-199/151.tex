\section{151 --- Reverse Words in a String}
Given an input string $S$, reverse the string word by word.
\paragraph{Example:}
\begin{flushleft}
\textbf{Input}: \texttt{the sky is blue}
\\
\textbf{Output}: \texttt{blue is sky the}
\end{flushleft}
\paragraph{Note:}
\begin{itemize}
    \item A word is defined as a sequence of non-space characters.
    \item Input string may contain leading or trailing spaces. However, your reversed string should not contain leading or trailing spaces.
    \item You need to reduce multiple spaces between two words to a single space in the reversed string.
\end{itemize}
\paragraph{Follow up:}
\begin{itemize}
    \item For C programmers, try to solve it in-place in $O(1)$ space.
\end{itemize}    
\subsection{Two Pointers}
\begin{CJK*}{UTF8}{gbsn}
先整个字符串整体翻转一次,然后再分别翻转每一个单词(或者先分别翻转每一个单词,然后再整个字符串整体翻转一次),此时就能得到需要的结果了。这里采用双指针的方法或者双index的方法,假设$\alpha$是当前存储到的位置,$L$为字符串的长度。先reverse $S$,然后开始遍历$S$。遍历$S$用$\beta$,即当前访问的位置,遇到空格直接跳过,如果是非空格字符,这时候要判断$\alpha$是否为0,为0的话表示第一个单词,不用增加空格;如果不为0,说明不是第一个单词,需要在单词中间加一个空格,然后需要找到下一个单词的结束位置。这里用一个\texttt{while}循环来找下一个为空格的位置,在此过程中继续更新$\alpha$,相当于利用$S$的空间进行操作。找到单词的结束位置后,就reverse这个单词,然后更新$\beta$为结尾位置,最后遍历结束,最后把$\alpha$以后的部分去除掉就是最终的结果。
\end{CJK*}
\setcounter{lstlisting}{0}
\begin{lstlisting}[style=customc, caption={Two Pointers}]
void reverseWords(string &s)
{
	//reverse the whole string first
	reverse(s.begin(), s.end());

	size_t alpha = 0;

	for (size_t beta = 0; beta < s.size(); ++beta)
	{
		if (s[beta] != ' ')
		{
			if (alpha != 0) //This is not the first word
			{
				s[alpha] = ' ';
				++alpha;
			}

			auto pos = beta;

			auto word_start = alpha;

			while ((pos < s.size()) && (s[pos] != ' '))
			{
				s[alpha] = s[pos];
				++alpha;
				++pos;
			}

			beta = pos;

			//alpha is at the next position of the last word,
			//therefore the reverse range is [word_start, alpha-1]
			reverse(s.begin() + word_start, s.begin() + alpha);
		}

	}

	s.resize(alpha);

}
\end{lstlisting}