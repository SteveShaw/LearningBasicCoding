\section{162 --- Find Peak Element}
A peak element is an element that is greater than its neighbors.
\par
Given an input array $A$, where $A[i] \neq A[i+1]$, find a peak element and return its index.
\par
The array may contain multiple peaks, in that case return the index to any one of the peaks is fine.
\par
You may imagine that $A[-1] = A[n] = -\infty$.
\paragraph{Example 1:}
\begin{flushleft}
\textbf{Input}: $A = [1,2,3,1]$
\\
\textbf{Output}: 2
\\
\textbf{Explanation}: 3 is a peak element and your function should return the index number 2.
\end{flushleft}
\paragraph{Example 2:}
\begin{flushleft}
\textbf{Input}: $A = [1,2,1,3,5,6,4]$
\\
\textbf{Output}: 1 or 5 
\\
\textbf{Explanation}: Your function can return either index number 1 where the peak element is 2, or index number 5 where the peak element is 6.
\end{flushleft}
\paragraph{Note:}
\begin{itemize}
\item Your solution should be in logarithmic complexity.
\end{itemize}
\subsection{Recursive Binary Search}
We can view any given sequence in $A$ as alternating ascending and descending sequences. By making use of this, and the fact that we can return any peak as the result, we can make use of \textbf{Binary Search} to find the required peak element.
\par
In case of simple Binary Search, we work on a sorted sequence of numbers and try to find out the required number by reducing the search space at every step. In this case, we use a modification of this simple Binary Search to our advantage. We start off by finding the middle element $M$ from the given array $A$. If this element happens to be lying in a descending sequence of numbers. or a local falling slope(found by comparing $A[i]$ to its right neighbor), it means that the peak will always lie towards the left of this element. Thus, we reduce the search space to the left of $M$(including itself) and perform the same process on left subarray.
\par
If the middle element $M$ lies in an ascending sequence of numbers, or a rising slope(found by comparing $A[i]$ to its right neighbor), it obviously implies that the peak lies towards the right of this element. Thus, we reduce the search space to the right of $M$ (\textbf{not} including $M$) and perform the same process on the right subarray.
\par
In this way, we keep on reducing the search space till we eventually reach a state where only one element is remaining in the search space. This single element is the \textbf{peak element}.
\par
To see how it works, consider the three cases discussed above again.
\begin{enumerate}
\item $A=[5,4,3,2,1]$: In this case, we firstly find 3 as the middle element. Since it lies on a falling slope ($3>2$), we reduce the search space to $[5, 4, 3]$. For this subarray, 4 happens to be the middle element, which again lies on a falling slope ($4 > 3$), reducing the search space to $[5, 4]$. Now, 5 acts as the middle element and it lies on a falling slope ($5>4$), reducing the search space to $[5]$ only. Thus, 5 is returned as the peak correctly.
\item $A=[1,2,3,4,5]$:  In this case, we firstly find $3$ as the middle element. Since it lies on a rising slope ($3 < 4$), we reduce the search space to $[4, 5]$. Now, $4$ acts as the middle element for this subarray and it lies on a rising slope ($4< 5$), reducing the search space to $[5]$ only. Thus, 5 is returned as the peak correctly.
\item $A=[2,3,4,5,1]$: In this case, the peak lies somewhere in the middle. The first middle element is $4$. It lies on a rising slope ($4<5$), indicating that the peak lies towards its right. Thus, the search space is reduced to $[5, 1]$. Now, 5 happens to be the on a falling slope ($5 > 1$), reducing the search space to $[5]$ only. Thus, 5 is identified as the peak element correctly.
\end{enumerate}

\setcounter{lstlisting}{0}
\begin{lstlisting}[style=customc, caption={Reference Code}]
int findPeakElement ( vector<int>& nums )
{

	int l = 0;
	int r = static_cast< int >( nums.size () ) - 1;

	while( l < r )
	{
		int mid = ( l + r ) / 2;

		if( nums[mid] < nums[mid + 1] )
		{
			l = mid + 1;
		}
		else
		{
			r = mid;
		}
	}

	return l;
}
\end{lstlisting}