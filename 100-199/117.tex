\section{117 --- Populating Next Right Pointers in Each Node II}
Given a binary tree

\setcounter{lstlisting}{0}
\begin{lstlisting}[style=customc]
struct Node
{
    int val;
    Node *left;
    Node *right;
    Node *next;
}
\end{lstlisting}

Populate each next pointer to point to its next right node. If there is no next right node, the next pointer should be set to \fcj{null}.

Initially, all next pointers are set to \fcj{null}.
\paragraph{Note:}
\begin{itemize}
\item You may only use constant extra space.
\item Recursive approach is fine, implicit stack space does not count as extra space for this problem.
\end{itemize}
\paragraph{Example:}
\begin{flushleft}
\textbf{Input}
\begin{figure}[H]
\begin{tikzpicture}
[mynode/.style={draw,circle,minimum size=5mm, fill=gray!20!}]
\node(){};
\node[mynode](1) {1};
\node[mynode](2) [below = 8mm of 1, xshift=-12mm] {2};
\node[mynode](3) [below = 8mm of 1, xshift=12mm] {3};
\node[mynode](4) [below = 8mm of 2, xshift=-7mm] {4};
\node[mynode](5) [below = 8mm of 2, xshift=7mm] {5};
\node[mynode](7) [below = 8mm of 3, xshift=7mm] {7};
\draw[>=stealth,->] (1) -- (2);
\draw[>=stealth,->] (1) -- (3);
\draw[>=stealth,->] (2) -- (4);
\draw[>=stealth,->] (2) -- (5);
\draw[>=stealth,->] (3) -- (7);
\end{tikzpicture}
\end{figure}
\textbf{Output}:
\begin{figure}[H]
\begin{tikzpicture}
[mynode/.style={draw,circle,minimum size=5mm, fill=gray!20!}]
\node(){};
\node[mynode](1) {1};
\node[mynode](r1) [right = 12mm of 1] {\texttt{null}};
\node[mynode](2) [below = 8mm of 1, xshift=-12mm] {2};
\node[mynode](3) [below = 8mm of 1, xshift=12mm] {3};
\node[mynode](r2) [right = 12mm of 3] {\texttt{null}};
\node[mynode](4) [below = 8mm of 2, xshift=-7mm] {4};
\node[mynode](5) [below = 8mm of 2, xshift=7mm] {5};
\node[mynode](7) [below = 8mm of 3, xshift=7mm] {7};
\node[mynode](r3) [right = 12mm of 7] {\texttt{null}};
\draw[>=stealth,->] (1) -- (2);
\draw[>=stealth,->] (1) -- (3);
\draw[>=stealth,->] (2) -- (4);
\draw[>=stealth,->] (2) -- (5);
\draw[>=stealth,->] (3) -- (7);
\draw[>=stealth,->] (1) -- (r1);
\draw[>=stealth,->] (3) -- (r2);
\draw[>=stealth,->] (7) -- (r3);
\draw[>=stealth,->] (2) -- (3);
\draw[>=stealth,->] (4) -- (5);
\draw[>=stealth,->] (5) -- (7);
\end{tikzpicture}
\end{figure}
\end{flushleft}
\subsection{Level Traversing}
虽然题目要求中可能不是完全二叉树,但是基本思路仍然是用指针分别指向当前节点以及下层节点的首节点以及下层节点的当前节点。保留下层节点的首节点是当遍历完当前level的时候,能够move到下一层。

\setcounter{lstlisting}{0}
\begin{lstlisting}[style=customc, caption={Level}]
Node* connect( Node* root )
{
    auto node( root );
    //the current node in the next level of node
    Node* child = nullptr;
    //the first node int the next level of node
    Node* first_child = nullptr;
    while( node )
    {
        auto t( node );
        //traverse current level
        while( t )
        {
            if( t->left )
            {
                if( child )
                {
                    //child is the last node of t->left
                    child->next = t->left;
                }
                else
                {
                    //we have not set child
                    //so this is the first child in the next level of node
                    first_child = t->left;
                }
                //update current child node
                child = t->left;
            }

            if( t->right )
            {
                if( child )
                {
                    //child is the last node of t->right
                    child->next = t->right;
                }
                else
                {
                    //we have not set child
                    //so this is the first child in the next level of node
                    first_child = t->right;
                }
                //update current child node
                child = t->right;
            }

            t = t->next;
        }
        //move node to the next level
        node = first_child;
        //we have to reset child and first_child to nullptr
        //for processing next level
        child = nullptr;
        first_child = nullptr;
    }
    return root;
}
\end{lstlisting}