\section{104 --- Maximum Depth of Binary Tree}
Given a binary tree $T$, find its maximum depth.
\par
The maximum depth is the number of nodes along the longest path from the root node down to the farthest leaf node.
\par
Note: A leaf is a node with no children.
\paragraph{Example:}
Given binary tree
\begin{figure}[H]
\begin{tikzpicture}
[mynode/.style={draw,circle,minimum size=10mm, fill=gray!20!}]
\node(){};
\node[mynode](3) {3};
\node[mynode](9) [below = 8mm of 3, xshift=-10mm] {9};
\node[mynode](20) [below = 8mm of 1, xshift=10mm] {20};
\node[mynode](15) [below = 8mm of 20, xshift=-6mm] {15};
\node[mynode](7) [below = 8mm of 20, xshift=6mm] {7};
\draw[>=stealth,->] (3) -- (9);
\draw[>=stealth,->] (3) -- (20);
\draw[>=stealth,->] (20) -- (15);
\draw[>=stealth,->] (20) -- (7);
\end{tikzpicture}
\end{figure}
return its depth $D = 3$.
\subsection{Recursion}
A very easy problem. Just return the maximum depth of left substree and right subtree plus 1 (since we need to count the root).
\subsubsection{Algorithm}
\setcounter{algorithm}{0}
\begin{algorithm}[H]
\caption{Recursion}
\begin{algorithmic}[1]
\Procedure{MaxDepth}{$T$}
\If{$T=\texttt{null}$}
\State \Return 0 \Comment Empty tree has height 1
\EndIf
\State $\alpha:=\texttt{MaxDepth}(\texttt{LEFT}(T))$ \Comment The maximum depth of left subtree
\State $\beta:=\texttt{MaxDepth}(\texttt{RIGHT}(T))$ \Comment The maximum depth of right subtree
\State \Return $1+\max(\alpha, \beta)$
\EndProcedure
\end{algorithmic}
\end{algorithm}