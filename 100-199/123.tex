\section{123 --- Best Time to Buy and Sell Stock III}
Say you have an array $P$ for which the $i$th element is the price of a given stock on day $i$.

Design an algorithm to find the maximum profit. You may complete at most \textbf{two} transactions.

Note: You may not engage in multiple transactions at the same time (i.e., you must sell the stock before you buy again).
\paragraph{Example 1:}
\begin{flushleft}
\textbf{Input}: $[3,\;3,\;5,\;0,\;0,\;3,\;1,\;4]$

\textbf{Output}: 6

\textbf{Explanation}: Buy on day 4 (price = 0) and sell on day 6 (price = 3), profit $= 3-0 = 3$. Then buy on day 7 (price = 1) and sell on day 8 (price = 4), profit $= 4-1 = 3$.
\end{flushleft}
\paragraph{Example 2:}
\begin{flushleft}
\textbf{Input}: $[1,\;2,\;3,\;4,\;5]$

\textbf{Output}: 4

\textbf{Explanation}: Buy on day 1 (price = 1) and sell on day 5 (price = 5), profit $= 5-1 = 4$.Note that you cannot buy on day 1, buy on day 2 and sell them later, as you are engaging multiple transactions at the same time. You must sell before buying again.
\end{flushleft}
\paragraph{Example 3:}
\begin{flushleft}
\textbf{Input}: $[7,\;6,\;4,\;3,\;1]$
\\
\textbf{Output}: 0
\\
\textbf{Explanation}: In this case, no transaction is done, i.e. max profit $= 0$.
\end{flushleft}
\subsection{General Approach}
解决类似的问题有一个一般性的方法可以处理。基本思想是创建两个table $H$ and $U$。其中
\begin{itemize}
\item $H[i][j]$表示当持有股票时,从0到$i$th天最多发生$j$次transaction所能获得的最大profit。而
\item $U[i][j]$表示当不持有股票时,从0到$i$th天最多发生$j$次transaction所能获得的最大profit。
\end{itemize}
对应的递推公式为
\begin{align*}
H[i][j] &= \max(U[i-1][j]-P[i], H[i-1][j]) \\
U[i][j] &= \max(H[i-1][j-1] + P[i], U[i-1][j])
\end{align*}
公式1的意义表示,如果在第$i$天手头持有股票,那么要么是第$i$天买入股票或者在第$i$天前已经买入了股票,
\begin{enumerate}
\item 如果在第$i$天买入股票,由于只是买入,并没有卖出,因此不算一个transaction,而这时候购买股票需要花费$P[i]$,而第$i-1$天手头没有股票,因此相应的利润就是$U[i-1][j] - P[i]$。
\item 如果第$i$天没有买入股票,这时候第$i-1$天肯定持有股票了,因此这时候的利润就和第$i-1$天持有股票的利润相等,即$H[i-1][j]$
\end{enumerate}
公式2也是基于类似的理由,唯一不同的是,如果在$i$天卖出股票,由于手头持有股票才能卖,所以这算一个transaction,因此在$i-1$天的利润就是$H[i-1][j-1]$。

最后就是初始化时,
\begin{itemize}
\item $U[0][i]=U[i][0]:=0$,$U[0][i]$即在第0天手头没有股票的利润很显然是0,$U[i][0]$即没有发生transaction时手头没有股票的利润,很显然也是0。
\item 对于$H[0][0]$,即在第0天买入股票并持有,其利润即为$-P[0]$,对于$i>0$,$H[0][i]$即在第0天买入股票并持有并发生$i>1$次transaction,因为只有一天而且是持有股票,这个利润必须就是$-P[0]$。而$H[i][0]$即状态变成手头持有股票从第0天到第$i$天发生0次transaction所能产生的最大利润,由于是第$i$天状态仍然是持有股票,并且是0交易,所以要么在$i-1$天前已经是持有股票的状态,要么只是在$i$天才持有股票,因此$H[i][0]=\max(H[i-1][0], -P[i])$。
\end{itemize}

\setcounter{lstlisting}{0}
\begin{lstlisting}[style=customc, caption={DP}]
int maxProfit( vector<int>& prices )
{
    if( prices.empty() )
    {
        return 0;
    }
    //the maximum profit when hold a stock at day i with transaction j
    vector<array<int, 3>> stock( prices.size(), {0, 0, 0} );
    //the maximum profit when not holding a stock at day i with transaction j
    vector<array<int, 3>> cash( prices.size(), {0, 0, 0} );
    stock[0][0] = -prices[0];
    for( size_t i = 1; i < prices.size(); ++i )
    {
        //at day i without any transaction
        //we either hold stock at day (i - 1)
        //or we bought a stock at day (i)
        //only buy without sell is not one transaction
        stock[i][0] = ( max )( stock[i - 1][0], -prices[i] );
    }
    stock[0][1] = -prices[0];
    stock[0][2] = -prices[0];
    for( size_t i = 1; i < prices.size(); ++i )
    {
        //stock[i][k]: either we have stock at day (i-1) with k transaction
        //or we bought a stock at day (i-1) with k transaction
        //notice: only buy without sell is not one transaction
        stock[i][1] = ( max )( cash[i - 1][1] - prices[i], stock[i - 1][1] );
        stock[i][2] = ( max )( cash[i - 1][2] - prices[i], stock[i - 1][2] );
        //cash[i][k]: either we have stock at day (i-1) with (k-1) transactions
        //and sell it on day i
        //or we doesn't have stock at day (i-1) with (k) transactions
        cash[i][1] = ( max )( stock[i - 1][0] + prices[i], cash[i - 1][1] );
        cash[i][2] = ( max )( stock[i - 1][1] + prices[i], cash[i - 1][2] );
    }
    //we can do maximum 2 transactions
    return ( max )( stock.back()[2], cash.back()[2] );
}
\end{lstlisting}


