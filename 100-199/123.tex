\section{123 --- Best Time to Buy and Sell Stock III}
Say you have an array $P$ for which the $i$th element is the price of a given stock on day $i$.
\par
Design an algorithm to find the maximum profit. You may complete at most \textbf{two} transactions.
\par
Note: You may not engage in multiple transactions at the same time (i.e., you must sell the stock before you buy again).
\paragraph{Example 1:}
\begin{flushleft}
\textbf{Input}: $[3,\;3,\;5,\;0,\;0,\;3,\;1,\;4]$
\\
\textbf{Output}: 6
\\
\textbf{Explanation}: Buy on day 4 (price = 0) and sell on day 6 (price = 3), profit $= 3-0 = 3$. Then buy on day 7 (price = 1) and sell on day 8 (price = 4), profit $= 4-1 = 3$.
\end{flushleft}
\paragraph{Example 2:}
\begin{flushleft}
\textbf{Input}: $[1,\;2,\;3,\;4,\;5]$
\\
\textbf{Output}: 4
\\
\textbf{Explanation}: Buy on day 1 (price = 1) and sell on day 5 (price = 5), profit $= 5-1 = 4$.Note that you cannot buy on day 1, buy on day 2 and sell them later, as you are engaging multiple transactions at the same time. You must sell before buying again.
\end{flushleft}
\paragraph{Example 3:}
\begin{flushleft}
\textbf{Input}: $[7,\;6,\;4,\;3,\;1]$
\\
\textbf{Output}: 0
\\
\textbf{Explanation}: In this case, no transaction is done, i.e. max profit $= 0$.
\end{flushleft}
\subsection{General Approach}
\begin{CJK*}{UTF8}{gbsn}
解决类似的问题有一个一般性的方法可以处理。基本思想是创建两个table $H$ and $U$。其中
\begin{itemize}
\item $H[i][j]$表示当持有股票时,从0到$i$th天最多发生$j$次transaction所能获得的最大profit。而
\item $U[i][j]$表示当不持有股票时,从0到$i$th天最多发生$j$次transaction所能获得的最大profit。
\end{itemize}
对应的递推公式为
\end{CJK*}
\begin{align*}
H[i][j] &= \max(U[i-1][j]-P[i], H[i-1][j]) \\
U[i][j] &= \max(H[i-1][j-1] + P[i], U[i-1][j])
\end{align*}
\begin{CJK*}{UTF8}{gbsn}
公式1的意义表示,如果在第$i$天手头持有股票,那么要么是第$i$天买入股票或者在第$i$天前已经买入了股票,
\begin{enumerate}
\item 如果在第$i$天买入股票,由于只是买入,并没有卖出,因此不算一个transaction,而这时候购买股票需要花费$P[i]$,而第$i-1$天手头没有股票,因此相应的利润就是$U[i-1][j] - P[i]$。
\item 如果第$i$天没有买入股票,这时候第$i-1$天肯定持有股票了,因此这时候的利润就和第$i-1$天持有股票的利润相等,即$H[i-1][j]$
\end{enumerate}
公式2也是基于类似的理由,唯一不同的是,如果在$i$天卖出股票,由于手头持有股票才能卖,所以这算一个transaction,因此在$i-1$天的利润就是$H[i-1][j-1]$。
\end{CJK*}
\par
\begin{CJK*}{UTF8}{gbsn}
最后就是初始化时,
\begin{itemize}
\item $U[0][i]=U[i][0]:=0$,$U[0][i]$即在第0天手头没有股票的利润很显然是0,$U[i][0]$即没有发生transaction时手头没有股票的利润,很显然也是0。
\item 对于$H[0][0]$,即在第0天买入股票并持有,其利润即为$-P[0]$,对于$i>0$,$H[0][i]$即在第0天买入股票并持有并发生$i>1$次transaction,因为只有一天而且是持有股票,这个利润必须就是$-P[0]$。而$H[i][0]$即状态变成手头持有股票从第0天到第$i$天发生0次transaction所能产生的最大利润,由于是第$i$天状态仍然是持有股票,并且是0交易,所以要么在$i-1$天前已经是持有股票的状态,要么只是在$i$天才持有股票,因此$H[i][0]=\max(H[i-1][0], -P[i])$。
\end{itemize}
\end{CJK*}
\setcounter{algorithm}{0}
\begin{algorithm}[H]
\caption{Dynamic Programming}
\begin{algorithmic}[1]
\Procedure{MaxProfit}{$P,\;L_P$}
\State $H_{L_P\times 3}$ as $H[0][0]:=-P[0]$  \Comment Hold status
\State $U_{L_P\times 3}$ as $U[0][0]=\ldots=U[L_P-1][2]:=0$ \Comment Hold none status
\For{$i:=1\to L_P-1$}
\State $H[i][0]\gets\max(H[i-1][0], -P[i])$
\EndFor 
\For{$k:=1\to 2$}
\State $H[0][k]\gets -P[0]$
\EndFor
\algstore{123algo}
\end{algorithmic}
\end{algorithm}
\begin{algorithm}[H]
\begin{algorithmic}[1]
\algrestore{123algo}
\For{$i:=1\to L_P-1$}
\For{$k:=1\to 2$}
\State $H[i][k]\gets \max(U[i-1][k]-P[i], H[i-1][k])$ \Comment Update hold status
\State $U[i][k]\gets \max(H[i-1][k-1] + P[i], U[i-1][k])$ \Comment Update hold none status
\EndFor
\EndFor
\State \Return $\max(H[L_P-1][2], U[L_P-1][2])$ \Comment The maximum of hold and none-hold status
\EndProcedure
\end{algorithmic}
\end{algorithm}
