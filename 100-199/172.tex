\section{172 --- Factorial Trailing Zeroes}
Given an integer $n$, return the number of trailing zeroes in $n!$.
\paragraph{Example 1:}
\begin{flushleft}
\textbf{Input}: 3
\\
\textbf{Output}: 0
\\
\textbf{Explanation}: $3! = 6$, no trailing zero.
\end{flushleft}
\paragraph{Example 2:}
\begin{flushleft}
\textbf{Input}: 5
\\
\textbf{Output}: 1
\\
\textbf{Explanation}: $5! = 120$, one trailing zero.
\end{flushleft}
\paragraph{Note:}
\begin{itemize}
\item Your solution should be in logarithmic time complexity.
\end{itemize}
\subsection{Find Number Of Fives}
\begin{CJK*}{UTF8}{gbsn}
这其实就是要找乘数中10的个数,而10可分解为2和5,而2的数量又远大于5的数量,那么也就是找出5的个数。需注意的一点就是,像25,125,这样不只含有一个5的数字需要考虑进去。
\end{CJK*}
\setcounter{algorithm}{0}
\begin{algorithm}[H]
\caption{Find Number Of Fives}
\begin{algorithmic}[1]
\Procedure{TrailingZeroes}{$n$}
\State $\delta:=0$ \Comment The result
\While{$n \neq 0$}
\State $\delta\gets \delta + \dfrac{n}{5}$
\State $n\gets n/5$
\EndWhile
\State \Return $\delta$
\EndProcedure
\end{algorithmic}
\end{algorithm}
\setcounter{lstlisting}{0}
\begin{lstlisting}[style=customc, caption={Source Code}]
int trailingZeroes( int n )
{
    int ans = 0;

    while( n != 0 )
    {
        ans += ( n / 5 );
        n /= 5;
    }

    return ans;
}
\end{lstlisting}