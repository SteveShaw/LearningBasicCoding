\section{139 --- Word Break}
Given a \textbf{non-empty} string $S$ and a dictionary $A$ containing a list of \textbf{non-empty} words, determine if $S$ can be segmented into a space-separated sequence of one or more dictionary words.
\par
\paragraph{Note:}
\begin{itemize}
\item The same word in the dictionary may be reused multiple times in the segmentation.
\item You may assume the dictionary does not contain duplicate words.
\end{itemize}
\paragraph{Example 1:}
\begin{flushleft}
\textbf{Input}:

\fcj{s = "leetcode"}

\fcj{wordDict = ["leet", "code"]}

\textbf{Output}: \fcj{true}

\textbf{Explanation}: Return \fcj{true} because \fcj{"leetcode"} can be segmented as \fcj{"leet code"}.
\end{flushleft}
\paragraph{Example 2:}
\begin{flushleft}
\textbf{Input}:

\fcj{s = "applepenapple"}

\fcj{wordDict = ["apple", "pen"]}

\textbf{Output}: \fcj{true}

\textbf{Explanation}: Return \fcj{true} because \fcj{"applepenapple"} can be segmented as \fcj{"apple pen apple"}. Note that you are allowed to reuse a dictionary word.
\end{flushleft}

\paragraph{Example 3:}
\begin{flushleft}
\textbf{Input}: 

\fcj{s = "catsandog"}

\fcj{wordDict = ["cats", "dog", "sand", "and", "cat"]}

\textbf{Output}: \fcj{false}

\end{flushleft}
\subsection{Dynamic Programming}
本题重要的是如何构建DP函数。假设$F[i]$表示$S[0,i-1]$可以拆分成给定字典中的单词。由于需要考虑空字符串的情况,因此$F$的长度是$S$的长度加1。
\par
算法包含如下几个步骤
\begin{enumerate}
    \item 首先Initialize $F[0]$为 \texttt{true}
    \item 遍历$S$。这里需要两个for循环来遍历,在当前位置$i$,假设在$j$处把$S[0,i-1]$分为$S[0, j-1]$ 和 $S[j, i-1]$两个部分,很显然$S[0,j-1]$对应的就是$F[j]$,这个结果已经有了,现在就是要看$S[j,i-1]$就是是不是在字典中。
    \begin{itemize}
        \item 如果$F[j]$为\texttt{true},并且$S[j,i-1]$也在字典中,那么很显然$S[0,i-1]$也是可以拆分成字典中的单词的,因此$F[i]$也为\texttt{true},这时候就可以终止$j$的循环了。
        \item 如果其中一个都不满足上述条件,继续$j$的循环。
    \end{itemize}
    \item 最后$F$的最后一个值就是需要返回的结果了。
\end{enumerate}


\setcounter{lstlisting}{0}
\begin{lstlisting}[style=customc, caption={DP}]
bool wordBreak( string s, vector<string>& wordDict )
{
    unordered_set<string_view> words( begin( wordDict ), end( wordDict ) );
    //F[i] means if we can break s[0,i-1] into the words
    //found in wordDict
    vector<bool> F( s.size() + 1, false );
    //F[0] corresponds to an empty string
    F[0] = true;
    //use string_view to minimize memory usage
    string_view sv( s );
    for( size_t i = 1; i <= s.size(); ++i )
    {
        //test each separation index j from 0 to (i-1)
        for( size_t j = 0; j < i; ++j )
        {
            if( F[j] )
            {
                //we can break S[0, j-1] into words
                //found in wordDict
                //check if we S[j, i-1] can be found in wordDict
                auto sub = sv.substr( j, i - j );
                if( words.find( sub ) != words.end() )
                {
                    F[i] = true;
                    break;
                }
            }//end if
        }//end for(j)
    }//end for(i)
    return F.back();
}
\end{lstlisting}

\subsection{Breadth-First-Search}
In \textit{BFS} approach, we start by adding index 0 into a queue, say $q$. Each time, we pop an index $x$ from the queue. Then we find from index $x$ to $L-1$ ($L$ is the size of input string) to see if there is a index $y$ such that substring $s[x,y]$ can be found in the dictionary. If $y$ is $L_1$, we know we can break $s$ into words in the dictionary. Otherwise, we add $y$ into $q$.

\begin{lstlisting}[style=customc, caption={BFS}]
bool wordBreak( string s, vector<string>& wordDict )
{
    //queue used for BFS
    queue<size_t> q;
    //a flag vector indicate if an index
    vector<bool> seen( s.size(), false );
    q.push( 0 );
    unordered_set<string_view> words( wordDict.begin(), wordDict.end() );
    string_view sv( s );
    while( !q.empty() )
    {
        auto start = q.front();
        q.pop();
        if( !seen[start] )
        {
            //start has not been processed
            //check each index from start to L-1
            //to see if s[start, end] is in the dictionary
            for( size_t end = start; end < s.size(); ++end )
            {
                auto sub = sv.substr( start, end - start + 1 );
                if( words.find( sub ) != words.end() )
                {
                    if( end == s.size() - 1 )
                    {
                        return true;
                    }
                    q.push( end + 1 );
                }
            }
            seen[start] = true;
        }
    }
    return false;
}
\end{lstlisting}