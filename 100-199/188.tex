\section{188 --- Best Time to Buy and Sell Stock IV}
Say you have an array for which the $i$th element is the price of a given stock on day $i$.
\par
Design an algorithm to find the maximum profit. You may complete at most $k$ transactions.
\paragraph{Note:}
\begin{itemize}
\item You may not engage in multiple transactions at the same time (ie, you must sell the stock before you buy again).
\end{itemize}
\paragraph{Example 1:}
\begin{flushleft}
\textbf{Input}: 
\\
$[2,4,1]$, $k = 2$
\\
\textbf{Output}: 2
\\
\textbf{Explanation}: Buy on day 1 (price = 2) and sell on day 2 (price = 4), profit = $4-2 = 2$.
\end{flushleft}
\paragraph{Example 2:}
\begin{flushleft}
\textbf{Input}:
\\
$[3,2,6,5,0,3]$, $k = 2$
\\
\textbf{Output}: 7
\\
\textbf{Explanation}: Buy on day 2 (price = 2) and sell on day 3 (price = 6), profit $= 6-2 = 4$. Then buy on day 5 (price = 0) and sell on day 6 (price = 3), profit $= 3-0 = 3$.
\end{flushleft}
\subsection{Dynamic Programming}
\begin{CJK*}{UTF8}{gbsn}
算法和123题一模一样,但是需要注意,如果$k$不小于天数,其实就相当与无限次买卖,也就是第122题的情况了。
\end{CJK*}
\setcounter{lstlisting}{0}
\begin{lstlisting}[style=customc, caption={Dynamic Programming}]
int maxProfit( int k, vector<int>& prices )
{
    if( prices.empty() )
    {
        return 0;
    }

	//If allower number of transactions
	//is no less than days, it is same as
	//unlimited transactions case
    if( k >= L )
    {
        int ans = 0;

        for( int i = 0; i < L - 1; ++i )
        {
			//As long as the price of the stock
			//is rising, we just buy on day i
			//and sell on day i+1
            if( prices[i] < prices[i + 1] )
            {
                ans += prices[i + 1] - prices[i];
            }
        }

        return ans;
    }

    //hold stock
    vector<vector<int>> H( prices.size(), vector<int>( k + 1, 0 ) ); //hold
    //no stock in hand
    vector<vector<int>> U( prices.size(), vector<int>( k + 1, 0 ) ); //unhold

    //Initially, to have a stock at day 0
    //we must buy stock first
    H[0][0] = -prices[0];

    for( size_t i = 1; i < prices.size(); ++i )
    {
        H[i][0] = ( max )( H[i - 1][0], -prices[i] );
    }

    //No matter how many transactions
    //we can only buy once each day
    //so at day 0, always be the price of stock
    //at day 0
    for( int i = 1; i <= k; ++i )
    {
        H[0][i] = -prices[0];
    }

    for( size_t i = 1; i < prices.size(); ++i )
    {
        for( int j = 1; j <= k; ++j )
        {
            //Since we must have stock to sell, therefore
            //From unhold to hold is not a complete transaction
            H[i][j] = ( max )( U[i - 1][j] - prices[i], H[i - 1][j] );
            U[i][j] = ( max )( H[i - 1][j - 1] + prices[i], U[i - 1][j] );
        }
    }

    return ( max )( H.back().back(), U.back().back() );
}
\end{lstlisting}