\section{179 --- Largest Number}
Given a list of non negative integers $A$, arrange them such that they form the largest number.
\paragraph{Example 1:}
\begin{flushleft}
\textbf{Input}: $[10,2]$
\\
\textbf{Output}: \textcolor{red}{210}
\end{flushleft}
\paragraph{Example 2:}
\begin{flushleft}
\textbf{Input}: $[3,30,34,5,9]$
\\
\textbf{Output}: \textcolor{red}{9534330}
\end{flushleft}
\paragraph{Note:} 
\begin{itemize}
\item The result may be very large, so you need to return a string instead of an integer.
\end{itemize}
\subsection{Sort}
\begin{CJK*}{UTF8}{gbsn}
\begin{itemize}
\item 需要自定义排序规则
\item 对于两个数字对应的字符串a和b,如果ab > ba,则a排在前面,比如9和34,由于934>349,所以9排在前面,再比如说30和3,由于303<330,所以3排在30的前面。
\item 如果拼接后的字符串以0开头,需要去除。如果去除后成了空字符串,那么说明整个数组都是0。否则就返回从第一个不是0的字符开始的字符串。 
\end{itemize}
\end{CJK*}
\setcounter{lstlisting}{0}
\begin{lstlisting}[style=customc, caption={Sort Based On Custom Rule[]}]
string largestNumber( vector<int>& nums )
{
	//sort based on custom rule
    sort( nums.begin(), nums.end(), []( int a, int b )
    {
        auto sa = to_string( a );
        auto sb = to_string( b );

        auto s1 = sa;
        s1 += sb;

        auto s2 = sb;
        s2 += sa;

        return s1 >  s2;
    });

    string ans;

    for( auto n : nums )
    {
        ans += to_string( n );
    }

	//Check if ans is starting with 0
    size_t i = 0;
    for( ; i < ans.size(); ++i )
    {
        if( ans[i] != '0' )
        {
            break;
        }
    }

    if( i == ans.size() )
    {
        return "0";
    }

    return ans.substr( i );
}
\end{lstlisting}