\section{169 --- Majority Element}
Given an array $A$ of size $n$, find the majority element. The majority element is the element that appears more than $\lfloor \dfrac{n}{2}\rfloor$ times.
\par
You may assume that the array is non-empty and the majority element always exist in the array.
\paragraph{Example 1:}
\begin{flushleft}
\textbf{Input}: $[3,2,3]$
\\
\textbf{Output}: 3
\end{flushleft}
\paragraph{Example 2:}
\begin{flushleft}
\textbf{Input}: $[2,2,1,1,1,2,2]$
\\
\textbf{Output}: 2
\end{flushleft}
\subsection{Boyer-Moore Voting Algorithm}
The algorithm maintains a sequence element $x$ and a counter $\delta$, with $\delta$ initially zero.
\par
It then processes the elements of the sequence, one at a time. When processing an element $e$, 
\begin{itemize}
\item if $\delta$ is zero, the algorithm stores $e$ as its remembered sequence element $x$ and sets $\delta$ to one. 
\item Otherwise, it compares $e$ to the stored element $x$ and either increments the counter (if they are equal) or decrements the counter (otherwise).
\end{itemize}
At the end of this process, if the sequence has a majority, it will be $e$ stored by the algorithm.
\paragraph{Note:}
Even when the input sequence has no majority, the algorithm will report one of the sequence elements as its result. However, it is possible to perform a \textbf{second pass} over the same input sequence in order to count the number of times the reported element occurs and determine whether it is actually a majority.
\par
This second pass is needed, as it is not possible for a sublinear--space algorithm to determine whether there exists a majority element in a single pass through the input.
\setcounter{algorithm}{0}
\begin{algorithm}[H]
\caption{Boyer-Moore Voting Algorithm}
\begin{algorithmic}[1]
\Procedure{MajorityElement}{$A, n$}
\State $x:=A[0]$ \Comment The stored element
\State $\delta:=0$ \Comment The counter
\For{$i:=1\to n-1$}
\If{$A[i]=x$}
\State $\delta\gets\delta+1$ \Comment Increment the counter
\Else
\State $\delta\gets\delta-1$ \Comment Decrement the counter
\EndIf
\State Check if the counter is zero
\If{$\delta=0$}
\State $\delta\gets 1$ \Comment Reset the counter to 1
\State $x\gets A[i]$ \Comment Replace the stored element as current number
\EndIf
\EndFor
\State Second pass to make sure $x$ is majority element indeed
\State $\delta\gets 0$ \Comment Reset counter to 0
\For{$i:=1\to n-1$}
\If{$x=A[i]$}
\State $\delta\gets \delta +1$
\EndIf
\EndFor
\State Determine if the counter is actually larger than half of length
\If{$\delta > \lfloor n/2 \rfloor$}
\State \Return $x$
\Else
\State \Return $\varnothing$ \Comment No majority element exists
\EndIf
\EndProcedure
\end{algorithmic}
\end{algorithm}
\setcounter{lstlisting}{0}
\begin{lstlisting}[style=customc]
int majorityElement( vector<int>& nums )
{
    int x = nums[0];
    int count = 1;

    for( auto n : nums )
    {
        if( n == x )
        {
            ++count;
        }
        else
        {
            --count;
        }

        if( count == 0 )
        {
            x = n;
            count = 1;
        }
    }

    count = 0;
    for( auto n : nums )
    {
        if( x == n )
        {
            ++count;
        }
    }

    int L = static_cast<int>( nums.size() );
    L >>= 1; // L = L/2
    if( count > L )
    {
        return x;
    }

    return nums[0];
}
\end{lstlisting}
