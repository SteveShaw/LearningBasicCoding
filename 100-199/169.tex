\section{169 --- Majority Element}
Given an array $A$ of size $n$, find the majority element. The majority element is the element that appears \textbf{more than} $\lfloor \dfrac{n}{2}\rfloor$ times.

You may assume that the array is non-empty and the majority element always exist in the array.
\paragraph{Example 1:}
\begin{flushleft}
\textbf{Input}: \fcj{[3,2,3]}
\\
\textbf{Output}: 3
\end{flushleft}
\paragraph{Example 2:}
\begin{flushleft}
\textbf{Input}: \fcj{[2,2,1,1,1,2,2]}

\textbf{Output}: 2
\end{flushleft}
\subsection{Boyer-Moore Voting Algorithm}
The algorithm maintains a sequence element $x$ and a counter $\delta$, with $\delta$ initially zero. It then processes the elements of the sequence, one at a time. When processing an element $e$, 
\begin{itemize}
\item if $\delta$ is zero, the algorithm stores $e$ as its remembered sequence element $x$ and sets $\delta$ to one. 
\item Otherwise, it compares $e$ to the stored element $x$ and either increments the counter (if they are equal) or decrements the counter (otherwise).
\end{itemize}
At the end of this process, if the sequence has a majority, it will be $e$ stored by the algorithm.

We have to perform a \textbf{second pass} over the same input sequence in order to count the number of times the reported element occurs and determine whether it is actually a majority.

\setcounter{lstlisting}{0}
\begin{lstlisting}[style=customc, caption={Boyer-Moore Voting Algorithm}]
int majorityElement( vector<int>& nums )
{
    //1. find majority element
    int x = nums[0];
    int count = 0;
    for( int n : nums )
    {
        if( n == x )
        {
            ++count;
        }
        else
        {
            --count;
            if( count == 0 )
            {
                x = n;
                count = 1;
            }
        }
    }
    //2. check if x is the majority element
    int total = std::count( begin( nums ), end( nums ), x );
    auto N = static_cast< int >( nums.size() );
    if( total > N / 2 )
    {
        return x;
    }
    return nums[0];
}
\end{lstlisting}
