\section{190 --- Reverse Bits}
Reverse bits of a given 32 bits unsigned integer.

\paragraph{Example 1:}
\begin{flushleft}


\textbf{Input}: \fcj{00000010100101000001111010011100}

\textbf{Output}: \fcj{00111001011110000010100101000000}

\textbf{Explanation}: 

The input binary string \fcj{00000010100101000001111010011100} represents the unsigned integer \fcj{43261596}, so return \fcj{964176192} which its binary representation is \fcj{00111001011110000010100101000000}.

\end{flushleft}

\paragraph{Example 2:}
\begin{flushleft}


\textbf{Input}: \fcj{11111111111111111111111111111101}

\textbf{Output}: \fcj{10111111111111111111111111111111}

\textbf{Explanation}: 

The input binary string \fcj{11111111111111111111111111111101} represents the unsigned integer \fcj{4294967293}, so return \fcj{3221225471} which its binary representation is \fcj{10111111111111111111111111111111}.
\end{flushleft} 


\paragraph{Follow up:}

\begin{itemize}
\item If this function is called many times, how would you optimize it?
\end{itemize}