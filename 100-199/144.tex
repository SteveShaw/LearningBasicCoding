\section{144 --- Binary Tree Preorder Traversal}
Given a binary tree with root node $R$, return the preorder traversal of its nodes' values.
\paragraph{Example:}
\begin{flushleft}
\textbf{Input}:
\begin{figure}[H]
\begin{tikzpicture}
[mynode/.style={draw,circle,minimum size=5mm, fill=gray!20!}]
\node(){};
\node[mynode](1) {1};
\node[mynode](2)[below=8mm of 1, xshift=8mm] {2};
\node[mynode](3)[below=8mm of 2, xshift=-8mm] {3};
\draw[>=stealth,->] (1) -- (2);
\draw[>=stealth,->] (2) -- (3);
\end{tikzpicture}
\end{figure}
\textbf{Output}: $[1,2,3]$
\end{flushleft}
\paragraph{Follow up:}
\begin{itemize}
\item Recursive solution is trivial, could you do it iteratively?
\end{itemize}
\subsection{Iterative With Stack}
How to traverse the tree
There are two general strategies to traverse a tree:
\begin{itemize}
    \item Breadth First Search (BFS): We scan through the tree level by level, following the order of height, from top to bottom. The nodes on higher level would be visited before the ones with lower levels.
    \item Depth First Search (DFS): In this strategy, we adopt the depth as the priority, so that one would start from a root and reach all the way down to certain leaf, and then back to root to reach another branch.
    \par
    The DFS strategy can further be distinguished as \textbf{preorder}, \textbf{inorder}, and \textbf{postorder} depending on the relative order among the root node, left node and right node.
\end{itemize}
In iterative approach, we simulate the recursive process, i.e., start from the root and then at each iteration pop the current node out of the stack and push its child nodes. In the implemented strategy we push nodes into output list following the order \fcj{top} $\rightarrow$ \fcj{bottom} and \fcj{left} $\rightarrow$ \fcj{right}, that naturally reproduces preorder traversal.

\setcounter{lstlisting}{0}
\begin{lstlisting}[style=customc, caption={Iterative Approach With Stack 1}]
vector<int> preorderTraversal( TreeNode* root )
{
    vector<int> ans;
    if( !root )
    {
        return ans;
    }
    stack<TreeNode*> stk;
    stk.push( root );
    while( !stk.empty() )
    {
        //visit current node first
        auto t = stk.top();
        stk.pop();
        ans.push_back( t->val );
        //add right node
        if( t->right )
        {
            stk.push( t->right );
        }
        //add left node
        if( t->left )
        {
            stk.push( t->left );
        }
    }
    return ans;
}
\end{lstlisting}

Another iterative approach uses a auxiliary node $p$ which is initialized as \fcj{root} at start. This approach can be seen as a template also for postorder and inorder traverse. 

In this approach, the \fcj{while} loop will check if the stack is empty or $p$ is a null node. If one condition is not met, then add non \fcj{null} node $p$ into the stack with adding the value of $p$ to the output. Then we change $p$ to its left child (when $p$ is a non \fcj{null} node). 

If $p$ is a \fcj{null} node, we get the top node $t$ from the stack and set $p$ to the right child of $t$.


\begin{lstlisting}[style=customc, caption={Iterative Approach With Stack 2}]
vector<int> preorderTraversal( TreeNode* root )
{
    vector<int> ans;
    if( !root )
    {
        return ans;
    }
    stack<TreeNode*> stk;
    auto p = root;
    while( !stk.empty() || p )
    {
        if( p )
        {
            //add current node to the stack
            //for future right child traverse
            stk.push( p );
            //we are in preorder
            //get p's value first
            ans.push_back( p->val );
            //change p to its left child
            p = p->left;
        }
        else
        {
            //p is a null node
            //get top node of the stack
            auto t = stk.top();
            stk.pop();
            //change p to its right child
            p = t->right;
        }
    }
    return ans;
}
\end{lstlisting}

\subsection{Morris Traversal}
The idea is to go down from the node to its predecessor $P$, and each predecessor $P$ will be visited twice. Suppose we are at node $T_0$, from this node we check if we can go left
\begin{itemize}
\item If the left child of $T_0$ exists, go right from the right child node of the left child until end. When we visit a leaf (node's predecessor) \textbf{1st time}, it has a zero right child, so we update output and establish the pseudo link \fcj{P.right=T0} to mark the fact the predecessor is visited. When we visit the same predecessor \textbf{the 2nd time}, it already points to $T_0$, thus we remove pseudo link and move right to the next node.
\item If the left child of $T_0$ is \fcj{null}, update output and move right to next node.
\end{itemize}
An example with detailed steps is shown as following:
\begin{figure}[H]
\begin{tikzpicture}
[mynode/.style={draw,circle,minimum size=5mm, fill=gray!20!}]
\node(){};
\node[mynode](1)[fill=red!20!] {1};
\node[mynode](2)[below=8mm of 1, xshift=-10mm] {2};
\node[mynode](5)[below=8mm of 1, xshift=10mm] {5};
\node[mynode](3)[below=8mm of 2, xshift=-8mm] {3};
\node[mynode](4)[below=8mm of 2, xshift=8mm] {4};
\draw[>=stealth,->] (1) -- (2);
\draw[>=stealth,->] (1) -- (5);
\draw[>=stealth,->] (2) -- (3);
\draw[>=stealth,->] (2) -- (4);
\end{tikzpicture}
\end{figure}
Starting from node 1. The status is: $T_0 = 1$. Is left child of $T_0$ \fcj{null}? No, then go left. 
\begin{figure}[H]
\begin{tikzpicture}
[mynode/.style={draw,circle,minimum size=5mm, fill=gray!20!}]
\node(){};
\node[mynode](1)[fill=red!20!] {1};
\node[mynode](2)[below=8mm of 1, xshift=-10mm, fill=blue!20!] {2};
\node[mynode](5)[below=8mm of 1, xshift=10mm] {5};
\node[mynode](3)[below=8mm of 2, xshift=-8mm] {3};
\node[mynode](4)[below=8mm of 2, xshift=8mm] {4};
\draw[>=stealth,->] (1) -- (2);
\draw[>=stealth,->] (1) -- (5);
\draw[>=stealth,->] (2) -- (3);
\draw[>=stealth,->] (2) -- (4);
\end{tikzpicture}
\end{figure}
Current Status: $T_0 = 1$, $P \gets 2$. Go right of $P$ until you meet \fcj{null} or current node $T_0$
\begin{figure}[H]
\begin{tikzpicture}
[mynode/.style={draw,circle,minimum size=5mm, fill=gray!20!}]
\node(){};
\node[mynode](1)[fill=red!20!] {1};
\node[mynode](2)[below=8mm of 1, xshift=-10mm, fill=blue!20!] {2};
\node[mynode](5)[below=8mm of 1, xshift=10mm] {5};
\node[mynode](3)[below=8mm of 2, xshift=-8mm] {3};
\node[mynode](4)[below=8mm of 2, xshift=8mm,fill=blue!20!] {4};
\draw[>=stealth,->] (1) -- (2);
\draw[>=stealth,->] (1) -- (5);
\draw[>=stealth,->] (2) -- (3);
\draw[>=stealth,->] (2) -- (4);
\end{tikzpicture}
\end{figure}
Current Status: $T_0 = 1$, $P \gets 4$, \fcj{P.right=null}. Now the right traverse reaches end, then
\begin{enumerate}
\item Add $T_0$'s value to the output
\item Link \fcj{P.right} to $T_0$, i.e., \fcj{4.right=1}.
\item Move $T_0$ to its left child, i.e. 2
\end{enumerate}
Now, output is $[1]$
\begin{figure}[H]
\begin{tikzpicture}
[mynode/.style={draw,circle,minimum size=5mm, fill=gray!20!}]
\node(){};
\node[mynode](1) {1};
\node[mynode](2)[below=8mm of 1, xshift=-10mm, fill=red!20!] {2};
\node[mynode](5)[below=8mm of 1, xshift=10mm] {5};
\node[mynode](3)[below=8mm of 2, xshift=-8mm] {3};
\node[mynode](4)[below=8mm of 2, xshift=8mm] {4};
\draw[>=stealth,->] (1) -- (2);
\draw[>=stealth,->] (1) -- (5);
\draw[>=stealth,->] (2) -- (3);
\draw[>=stealth,->] (2) -- (4);
\draw[>=stealth,->] (4) to[bend right=15] (1);
\end{tikzpicture}
\end{figure}
Current Status: $T_0 = 2$. Is left child of the node is \fcj{null}? No, then go left.
\begin{figure}[H]
\begin{tikzpicture}
[mynode/.style={draw,circle,minimum size=5mm, fill=gray!20!}]
\node(){};
\node[mynode](1) {1};
\node[mynode](2)[below=8mm of 1, xshift=-10mm, fill=red!20!] {2};
\node[mynode](5)[below=8mm of 1, xshift=10mm] {5};
\node[mynode](3)[below=8mm of 2, xshift=-8mm, fill=blue!20!] {3};
\node[mynode](4)[below=8mm of 2, xshift=8mm] {4};
\draw[>=stealth,->] (1) -- (2);
\draw[>=stealth,->] (1) -- (5);
\draw[>=stealth,->] (2) -- (3);
\draw[>=stealth,->] (2) -- (4);
\draw[>=stealth,->] (4) to[bend right=15] (1);
\end{tikzpicture}
\end{figure}
Current Status: $T_0 = 2$, $P = 3$. Go right of $P$ until you meet \fcj{null} or $T_0$. Since the right of 3 is \fcj{null}, then
\begin{enumerate}
\item Add $T_0$'s value to the output
\item Link \fcj{P.right=T0}, i.e., \fcj{3.right = 2}.
\item Move $T_0$ to its left child, i.e. 3
\end{enumerate}
Now, output is $[1,2]$
\begin{figure}[H]
\begin{tikzpicture}
[mynode/.style={draw,circle,minimum size=5mm, fill=gray!20!}]
\node(){};
\node[mynode](1) {1};
\node[mynode](2)[below=8mm of 1, xshift=-10mm] {2};
\node[mynode](5)[below=8mm of 1, xshift=10mm] {5};
\node[mynode](3)[below=8mm of 2, xshift=-8mm, fill=red!20!] {3};
\node[mynode](4)[below=8mm of 2, xshift=8mm] {4};
\draw[>=stealth,->] (1) -- (2);
\draw[>=stealth,->] (1) -- (5);
\draw[>=stealth,->] (2) -- (3);
\draw[>=stealth,->] (2) -- (4);
\draw[>=stealth,->] (4) to[bend right=15] (1);
\draw[>=stealth,->] (3) to[bend right=30] (2);
\end{tikzpicture}
\end{figure}
Current Status: $T_0 = 3$. Is left child of $T_0$ is \fcj{null}? Yes, then
\begin{enumerate}
\item Add $T_0$'s value to the output
\item Move to the right child of $T_0$, which is a 2.
\end{enumerate}
Now, output is $[1,2,3]$
\begin{figure}[H]
\begin{tikzpicture}
[mynode/.style={draw,circle,minimum size=5mm, fill=gray!20!}]
\node(){};
\node[mynode](1) {1};
\node[mynode](2)[below=8mm of 1, xshift=-10mm,fill=red!20!] {2};
\node[mynode](5)[below=8mm of 1, xshift=10mm] {5};
\node[mynode](3)[below=8mm of 2, xshift=-8mm] {3};
\node[mynode](4)[below=8mm of 2, xshift=8mm] {4};
\draw[>=stealth,->] (1) -- (2);
\draw[>=stealth,->] (1) -- (5);
\draw[>=stealth,->] (2) -- (3);
\draw[>=stealth,->] (2) -- (4);
\draw[>=stealth,->] (4) to[bend right=15] (1);
\draw[>=stealth,->] (3) to[bend right=30] (2);
\end{tikzpicture}
\end{figure}
Current Status: $T_0 = 2$. Is left child of $T_0$ is \fcj{null}? No, then go left. 
\begin{figure}[H]
\begin{tikzpicture}
[mynode/.style={draw,circle,minimum size=5mm, fill=gray!20!}]
\node(){};
\node[mynode](1) {1};
\node[mynode](2)[below=8mm of 1, xshift=-10mm,fill=red!20!] {2};
\node[mynode](5)[below=8mm of 1, xshift=10mm] {5};
\node[mynode](3)[below=8mm of 2, xshift=-8mm, fill=blue!20!] {3};
\node[mynode](4)[below=8mm of 2, xshift=8mm] {4};
\draw[>=stealth,->] (1) -- (2);
\draw[>=stealth,->] (1) -- (5);
\draw[>=stealth,->] (2) -- (3);
\draw[>=stealth,->] (2) -- (4);
\draw[>=stealth,->] (4) to[bend right=15] (1);
\draw[>=stealth,->] (3) to[bend right=30] (2);
\end{tikzpicture}
\end{figure}
Current Status: $T_0 = 2$, $P = 3$. Go right of $P$ until you meet \fcj{null} or $T_0$. Now, go right of 3 will meet 2, so
\begin{enumerate}
\item \textbf{Delete link} \fcj{P.right=T0}, i.e., \fcj{3.right=2}
\item Move to right child of $T_0$, i.e., 4.
\end{enumerate}
\begin{figure}[H]
\begin{tikzpicture}
[mynode/.style={draw,circle,minimum size=5mm, fill=gray!20!}]
\node(){};
\node[mynode](1) {1};
\node[mynode](2)[below=8mm of 1, xshift=-10mm] {2};
\node[mynode](5)[below=8mm of 1, xshift=10mm] {5};
\node[mynode](3)[below=8mm of 2, xshift=-8mm] {3};
\node[mynode](4)[below=8mm of 2, xshift=8mm, fill=red!20!] {4};
\draw[>=stealth,->] (1) -- (2);
\draw[>=stealth,->] (1) -- (5);
\draw[>=stealth,->] (2) -- (3);
\draw[>=stealth,->] (2) -- (4);
\draw[>=stealth,->] (4) to[bend right=15] (1);
\draw[>=stealth,->] (3) to[bend right=30] (2);
\end{tikzpicture}
\end{figure}
Current Status: $T_0 = 4$. Is left child of $T_0$ \fcj{null}? Yes, then
\begin{enumerate}
\item Add $T_0$'s value to output.
\item Move to the right child of $T_0$, which is 1.
\end{enumerate}
Now, output is $[1,2,3,4]$
\begin{figure}[H]
\begin{tikzpicture}
[mynode/.style={draw,circle,minimum size=5mm, fill=gray!20!}]
\node(){};
\node[mynode](1)[fill=red!20!] {1};
\node[mynode](2)[below=8mm of 1, xshift=-10mm] {2};
\node[mynode](5)[below=8mm of 1, xshift=10mm] {5};
\node[mynode](3)[below=8mm of 2, xshift=-8mm] {3};
\node[mynode](4)[below=8mm of 2, xshift=8mm] {4};
\draw[>=stealth,->] (1) -- (2);
\draw[>=stealth,->] (1) -- (5);
\draw[>=stealth,->] (2) -- (3);
\draw[>=stealth,->] (2) -- (4);
\draw[>=stealth,->] (4) to[bend right=15] (1);
\end{tikzpicture}
\end{figure}
Current Status: $T_0 = 1$. Is left child of $T_0$ is \fcj{null}? No, then go left. 
\begin{figure}[H]
\begin{tikzpicture}
[mynode/.style={draw,circle,minimum size=5mm, fill=gray!20!}]
\node(){};
\node[mynode](1)[fill=red!20!] {1};
\node[mynode](2)[below=8mm of 1, xshift=-10mm, fill=blue!20!] {2};
\node[mynode](5)[below=8mm of 1, xshift=10mm] {5};
\node[mynode](3)[below=8mm of 2, xshift=-8mm] {3};
\node[mynode](4)[below=8mm of 2, xshift=8mm] {4};
\draw[>=stealth,->] (1) -- (2);
\draw[>=stealth,->] (1) -- (5);
\draw[>=stealth,->] (2) -- (3);
\draw[>=stealth,->] (2) -- (4);
\draw[>=stealth,->] (4) to[bend right=15] (1);
\end{tikzpicture}
\end{figure}
Current Status: $T_0 = 1$, $P = 2$. Go right of $P$ until we meet \fcj{null} or $T_0$.
\begin{figure}[H]
\begin{tikzpicture}
[mynode/.style={draw,circle,minimum size=5mm, fill=gray!20!}]
\node(){};
\node[mynode](1)[fill=red!20!] {1};
\node[mynode](2)[below=8mm of 1, xshift=-10mm, fill=blue!20!] {2};
\node[mynode](5)[below=8mm of 1, xshift=10mm] {5};
\node[mynode](3)[below=8mm of 2, xshift=-8mm] {3};
\node[mynode](4)[below=8mm of 2, xshift=8mm, fill=blue!20!] {4};
\draw[>=stealth,->] (1) -- (2);
\draw[>=stealth,->] (1) -- (5);
\draw[>=stealth,->] (2) -- (3);
\draw[>=stealth,->] (2) -- (4);
\draw[>=stealth,->] (4) to[bend right=15] (1);
\end{tikzpicture}
\end{figure}
Current Status: $T_0 = 1$, $P = 4$, \fcj{P.right=T0}. We meet $T_0$, therefore
\begin{enumerate}
\item Delete link \fcj{P.right=T0}, (i.e. \fcj{4.right=1}).
\item Move to right child of $T_0$, i.e., 5.
\end{enumerate}
\begin{figure}[H]
\begin{tikzpicture}
[mynode/.style={draw,circle,minimum size=5mm, fill=gray!20!}]
\node(){};
\node[mynode]{1};
\node[mynode](2)[below=8mm of 1, xshift=-10mm] {2};
\node[mynode](5)[below=8mm of 1, xshift=10mm, fill=red!20!] {5};
\node[mynode](3)[below=8mm of 2, xshift=-8mm] {3};
\node[mynode](4)[below=8mm of 2, xshift=8mm] {4};
\draw[>=stealth,->] (1) -- (2);
\draw[>=stealth,->] (1) -- (5);
\draw[>=stealth,->] (2) -- (3);
\draw[>=stealth,->] (2) -- (4);
\end{tikzpicture}
\end{figure}
Current Status: $T_0 = 5$. Is left child of $T_0$ \fcj{null}? Yes, then
\begin{enumerate}
\item Add $T_0$'s value to output.
\item Move to right child of $T_0$
\end{enumerate}
Since we reach a \fcj{null} node, the whole process is finished.


\begin{lstlisting}[style=customc, caption={Morris Traversal}]
vector<int> preorderTraversal( TreeNode* root )
{
    vector<int> ans;
    auto node = root;
    while( node )
    {
        if( !node->left )
        {
            //no left child
            //we output current node's value
            ans.push_back( node->val );
            //and move to the right child
            node = node->right;
        }
        else
        {
            //we have to link the rightmost child of node
            //with it
            auto t = node->left;
            while( t->right && ( t->right != node ) )
            {
                t = t->right;
            }
            if( !t->right )
            {
                //first time to traverse node's child tree
                ans.push_back( node->val );
                t->right = node;
                //traverse node's left child
                node = node->left;
            }
            else
            {
                //second time to traverse node's child tree
                //this time left child tree has been visited
                //we first set t->right to null
                t->right = nullptr;
                //traverse node's right child
                node = node->right;
            }//end(t->right)
        }//end(node->left)
    }//end(while)

    return ans;
}
\end{lstlisting}