\section{137 --- Single Number II}
Given a non-empty array of integers $A$, every element appears three times except for one, which appears exactly once. Find that single one.
\paragraph{Note:}
\begin{flushleft}
Your algorithm should have a linear runtime complexity. Could you implement it without using extra memory?
\end{flushleft}
\paragraph{Example 1:}
\begin{flushleft}
\textbf{Input}: $[2,2,3,2]$
\\
\textbf{Output}: 3
\end{flushleft}
\paragraph{Example 2:}
\begin{flushleft}
\textbf{Input}: $[0,1,0,1,0,1,99]$
\\
\textbf{Output}: 99
\end{flushleft}
\subsection{Bitwise Operation}
首先问题可以推广到如下形式,给定一个array $A$,其中除了一个数字重复了$p$次,其余的都重复了$k$次,找出这个重复$p$次的数字。

首先从简单的例子开始,假定有一组1-bit数字构成的array $A$, 用一个counter记录array中数字1的出现次数,当这个counter到达某个值$k$时,就重置为0,然后继续。

很明显,如果这个counter能够计数到$k$,假设其有$m$个bits,那么就有$2^m\geq k$,变换一下也就是$m\geq \log_{2}(k)$。

现在来看当遍历数组$A$时,counter的每个bit是如何变化的,假设counter的$m$个bits从最高位到最低位分别为$b_{m-1}, \ldots, b_0$,由于当遇到数字0时,counter是保持不变的,那么具有如此性质的bitwise操作只有\textbf{OR}或者是\textbf{XOR}。接下来看这两种操作哪一种最佳。

最开始,counter中的所有bit都是0。 直到遇到数字1,这些bits才开始发生改变。当遇到第一个1时,这时候$b_{m-1}=0,\ldots, b_0=1$。当遇到第二个1时,则变化为$b_{m-1}=0,\ldots, b_1=1, b_0=0$。注意到,$b_0$变成了0。如果用OR操作,那么$b_0\gets b_0\;\texttt{or}\; i$,这时候$b_0$仍然是1。因此,只能选择异或操作。对于$b_1,\ldots,b_{m-1}$,需要找到在什么样的情况下这些bits会发生改变。比如以$b_1$为例,当遇到数字1,这时候$b_0$必须是1才会使得我们必须改变$b_1$的值。因为如果是0,只需要把$b_0$从0变为1即可,并不需要改变$b_1$的值。因此$b_1$只有在$b_0$和$i$都是1的情况下才会改变其值,写成bitwise的形式就是$b_1\gets b_1\;\texttt{XOR}\;(b_0\;\texttt{AND}\;i)$。以此类推,$b_{m-1}$只有当$b_{m-2},\ldots,b_1,b_0$和$i$都是1的情况下才会需要改变,即$b_{m-1}\gets b_{m-1}\;\texttt{XOR}\;(b_{m-2}\;\texttt{AND}\;\ldots\;\texttt{AND}\;b_0\;\texttt{AND}\;i)$。这就是所需要的bitwise operation。
\par
上述bitwise operation是从0一直到$2^{m}-1$,而不是$k$。如果$k<2^m-1$,则需要在counter等于$k$时,将counter重置为0。要做到这一点,需要用一个mask $M$ 来和$b_{m-1},\ldots, b_0$做\texttt{AND} operation,即
\begin{align*}
b_{m-1}&\gets b_{m-1}\;\texttt{AND}\;M\\
b_{m-2}&\gets b_{m-2}\;\texttt{AND}\;M\\
\ldots& \\
b_0&\gets b_0\;\texttt{AND}\;M
\end{align*}
而$M$只有当counter等于$k$时才等于0,其他情况下都为1。假设$k$写成二进制的形式位$x_{m-1}, \ldots, x_{0}$,那么这个mask可以按照如下方法得到
\[
\texttt{NOT}(y_0\;\texttt{AND}\;y_1\;\texttt{AND}\ldots\texttt{AND}\;y_{m-1})
\]
其中$y_i=b_i$ when $x_j=1$,$y_i=\texttt{NOT}(b_i)$ when $x_i=0$。这里$i\in[0,m-1]$。
\par
例如$k=3$,写成二进制形式为$11$,即$x_{1}=1,x_0=1$。那么mask就是$\texttt{NOT}(b_1\;\texttt{AND}\;b_0)$。

%因此算法的框架就像下面所示的那样
%\setcounter{algorithm}{0}
%\begin{algorithm}[H]
%\caption{Overall Description}
%\begin{algorithmic}[1]
%\Procedure{F}{$A, L$}
%\For{$i:=0\to L-1$}
%\State $n:=A[i]$
%\State $b_{m-1}\gets b_{m-1}\;\texttt{XOR}\;(b_{m-2}\;\texttt{AND}\ldots\texttt{AND}\;b_0\;\texttt{AND}\;n)$
%\State $b_{m-2}\gets b_{m-2}\;\texttt{XOR}\;(b_{m-3}\;\texttt{AND}\ldots\texttt{AND}\;b_0\;\texttt{AND}\;n)$
%\State $\ldots$
%\State $b_{0}\gets b_{0}\;\texttt{XOR}\;n$
%\State $M:=\texttt{NOT}(y_0\;\texttt{AND}\ldots\texttt{AND}\;y_{m-1})$ \Comment The mask
%\State $x_{m-1}\gets x_{m-1}\;\texttt{AND}\;M$
%\State $x_{m-2}\gets x_{m-2}\;\texttt{AND}\;M$
%\State $\ldots$
%\State $x_{0}\gets x_{0}\;\texttt{AND}\;M$
%\EndFor
%\EndProcedure
%\end{algorithmic}
%\end{algorithm}
假设$A$中的number都是32bit整数。最直接的方法当然是对于整数中的32个bit分别创建1个counter,也就是有32个counters。不过更好的方法用$m$个32位整数而不是32个$m$位整数,这里$m$是满足$m\geq\log_2{k}$的最小整数。也就是说可以把32个counter的同一个bit分别单独拿出来构成一个整数,这样就只需要$m$个32位整数了.
%%\begin{figure}[H]
%%\includegraphics[width=15cm]{137.png}
%%\end{figure}


以$m_0$为例,它是一个32bit整数。当遍历完整个array时,$m_0$中的第$z$个bit将由array中所有number的第$z$个bit位确定。如果一个number的第$z$个bit位在array中出现了$k$的倍数次,那么当计数到$k$时,由于我们要将counter重置为0,这个number在第$z$个bit也会重置为0。因此只有那个重复了$p$次的number在第$z$个bit位出现次数才最终决定了这个位上的值。由于这个number的出现次数是$p$,那么$p\bmod k\neq0$。当$p>k$时,只有$p\bmod k$次才最终决定在$z$处的值。也就是说在这个number对于第$z$个bit位而言其有效的重复次数是$\hat{p}:=p\bmod k$。

假设这个有效次数$\hat{p}$写成二进制形式位$\hat{p}_{m-1}\hat{p}_{m-2}\ldots\hat{p}_0$,那么如果$m_i$也就是第$i$个counter的第$z$个bit是1,很显然那个重复了$p$的number的有效次数$\hat{p}$的第$z$个bit也为1。而如果$m_i$的第$z$个bit位是0,那么重复了$p$次的数字的有效次数$\hat{p}$的第$z$个bit只能是0。如果是1,那么因为在遍历完整个数组后,这个位上的1出现了$\hat{p}$次,最终结果还是1,就不是我们假设的0了。而这个$z$是从0到32的,因此结论就是如果$\hat{p}_i=1$,那么$m_i$就是那个重复了$p$次的数字。

因此算法首先计算出$\hat{p}:=p\bmod k$,然后只要$\hat{p}$的第$z$个bit位为1, 就返回$m_z$,如果有多个这样的bit位,返回其中任意一个即可。

\begin{itemize}
\item 当$k=2$, $p=1$时, 只需要$m=1$个32位整数作为counter,因为$2^1\geq2$,而因为$2^m=k$,甚至都不要mask。因此算法其实就是每个数进行\texttt{XOR}。
\item 当$k=3$, $p=1$时,由于$2^2>3$,因此$m=2$,即两个32位整数的counter $m_0$ and $m_1$。而$2^m=2^2=4>k=3$,因此需要mask $M$。$k$的二进制形式为$k=b(11)$,因此$M = \texttt{NOT}(m_0\;\texttt{AND}\;m_1)$。算法如下所示:
\end{itemize}


\setcounter{algorithm}{0}
\begin{algorithm}[H]
\caption{K=3 AND P=1}
\begin{algorithmic}[1]
\Procedure{F}{$A, L$}
\State $m_0:=0$, $m_1:=0$, $M:=0$
\For{$i:=0\to L-1$}
\State $n:=A[i]$
\State $m_1\gets m_1\;\texttt{XOR}\;(m_0\;\texttt{AND}\;n)$
\State $m_0\gets m_0\;\texttt{XOR}\;n$
\State $M\gets \texttt{NOT}(m_0\;\texttt{AND}\;m_1)$
\State $m_1\gets m_1\;\texttt{AND}\;M$
\State $m_0\gets m_0\;\texttt{AND}\;M$
\EndFor
\State \Return $m_0$
\EndProcedure
\end{algorithmic}
\end{algorithm}
由于$p\bmod k=1$,因此返回$m_0$。
\begin{itemize}
\item 当$k=5$, $p=3$时, $m=\lfloor\log_25\rfloor=3$。因此需要3个counter,$m_0$,$m_1$ and $m_2$。而$p\bmod k=3$,二进制形式为$b(011)$,因此返回$m_0$或者$m_1$都可以。而$k$的二进制是$b(101)$,因此$M:=\texttt{NOT}(m_2\;\texttt{AND}\;\texttt{NOT}(m_1)\;\texttt{AND}\;m_0)$
\end{itemize}

\setcounter{lstlisting}{0}
\begin{lstlisting}[style=customc, caption={Bitwise}]
int singleNumber( vector<int>& nums )
{
    int m0 = 0;
    int m1 = 0;
    int mask = 0;
    for( int n : nums )
    {
        //set/unset bit 1 of counter
        m1 = m1 ^ ( m0 & n );
        //set/unset bit 0 of counter
        m0 = m0 ^ n;
        //get the mask
        mask = ~( m1 & m0 );
        //set/unset bit 1 and bit 0
        //through mask
        m1 = mask & m1;
        m0 = mask & m0;
    }
    return m0;
}
\end{lstlisting}
