\section{155 --- Min Stack}
Design a stack that supports \texttt{push}, \texttt{pop}, \texttt{top}, and retrieving the minimum element in constant time.
\begin{itemize}
    \item \texttt{push}($x$) -- Push element $x$ onto stack.
    \item \texttt{pop}() -- Removes the element on top of the stack.
    \item \texttt{top}() -- Get the top element.
    \item \texttt{getMin}() -- Retrieve the minimum element in the stack.
\end{itemize}
\paragraph{Example:}
\begin{lstlisting}[style=customc]
MinStack minStack = new MinStack();
minStack.push(-2);
minStack.push(0);
minStack.push(-3);
minStack.getMin();   // Returns -3.
minStack.pop();
minStack.top();      // Returns 0.
minStack.getMin();   // Returns -2.
\end{lstlisting}
\begin{CJK*}{UTF8}{gbsn}
这里涉及到如何利用stack的特性的技巧,为了得到stack中的最小值,同时又能保证弹出操作能够正确的更新最小值,可以将当前最小值和要压入的值一起放入stack中,一种做法是先放入当前值,在放入当前最小值,这样栈顶就是最小值。这样,当弹出时,需要弹出两个。然后如果栈不为空,就用栈顶元素更新最小值。
\par
而对于\texttt{top} function,由于栈顶元素是最小值,因此需要先把当前栈顶元素弹出,然后获得剩下的栈的栈顶元素,再把弹出的最小值重新压入栈中。
\end{CJK*}
\setcounter{lstlisting}{0}
\begin{lstlisting}[style=customc, caption={Push Two Values At Same Time}]
class MinStack
{
public:
	/** initialize your data structure here. */
	MinStack ()
	{
		m_min = -1;
	}

	void push ( int x )
	{
		if( m_stk.empty () )
		{
			//This is the first time
			//to push into stack
			m_min = x;
		}
		else
		{
			//Update current minimum value
			m_min = ( min ) ( x, m_min );
		}

		//Push x first
		m_stk.push ( x );
		//Push current minimum value
		m_stk.push ( m_min );
	}

	void pop ()
	{
		if( !m_stk.empty () )
		{
			//Pop minimum value
			m_stk.pop ();
			//Pop actual value
			m_stk.pop ();

			if( !m_stk.empty () )
			{
				//Update current minimum value
				m_min = m_stk.top ();
			}
		}
	}

	int top ()
	{
		//we have to pop current top first
		//because it is m_min
		m_stk.pop ();
		//This is the actual top
		int x = m_stk.top ();
		//Push m_min to stack again
		m_stk.push ( m_min );
		return x;
	}

	int getMin ()
	{
		return m_min;
	}

	// The minimum value in the stack
	int m_min;
	// The stack
	stack<int> m_stk;
};
\end{lstlisting}