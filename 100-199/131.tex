\section{131 --- Palindrome Partitioning}
Given a string $s$, partition $s$ such that every substring of the partition is a palindrome.
\par
Return all possible palindrome partitioning of $s$.
\paragraph{Example:}
\begin{flushleft}
\textbf{Input}: \fcj{"aab"}

\textbf{Output}:

\fcj{["aa","b"]}

\fcj{[a","a","b"]}
\end{flushleft}

\subsection{DFS}

因为需要找到所有的palindrome字符串,因此\textit{DFS}是最自然的选择。在递归函数中,从给定的起始位置出发,然后逐个搜寻以这个起始位置为开始的\textit{substring}在后面哪个位置可以找到palindrome,如果找到了这个位置,把当前的substring加入当前的选择中,然后从找到的位置继续进行递归。

\setcounter{lstlisting}{0}
\begin{lstlisting}[style=customc, caption={DFS}]
vector<vector<string>> partition( string s )
{
    if( s.empty() )
    {
        return {};
    }
    vector<vector<string>> ans;
    //save the result in one pass scan
    vector<string_view> pals;
    dfs( s, 0, pals, ans );
    return ans;
}
//helper function for backtracking
void dfs( string_view s, size_t start, vector<string_view>& pals, vector<vector<string>>& ans )
{
    if( start == s.size() )
    {
        //add current palindromes to the result
        ans.emplace_back( pals.begin(), pals.end() );
        return;
    }
    //check if each substring [start, i]
    for( size_t i = start; i < s.size(); ++i )
    {
        auto l = start;
        auto r = i;
        bool flag = true;
        //check if substring [start,i] is a
        //palindrome or not
        while( l < r )
        {
            if( s[l] != s[r] )
            {
                flag = false;
                break;
            }
            l = l + 1;
            r = r - 1;
        }
        if( flag )
        {
            //s[start, i] is a palindrome
            //add to current array
            pals.push_back( s.substr( start, i - start + 1 ) );
            //deep into s[i+1, ...]
            dfs( s, i + 1, pals, ans );
            //backtracking
            pals.pop_back();
        }
    }
}
\end{lstlisting}