\section{124 --- Binary Tree Maximum Path Sum}
Given a non-empty binary tree $T$, find the maximum path sum.
\par
For this problem, a path is defined as any sequence of nodes from some starting node to any node in the tree along the parent-child connections. The path must contain \textbf{at least one node} and does not need to go through the root.
\paragraph{Example 1:}
\begin{flushleft}
\textbf{Input}:
\begin{figure}[H]
\begin{tikzpicture}
[mynode/.style={draw,circle,minimum size=5mm, fill=gray!20!}]
\node(){};
\node[mynode](1) {\textcolor{red}{1}};
\node[mynode](2)[below=8mm of 1, xshift=-8mm] {\textcolor{red}{2}};
\node[mynode](3)[below=8mm of 1, xshift=8mm] {\textcolor{red}{3}};
\draw[>=stealth,->] (1) -- (2);
\draw[>=stealth,->] (1) -- (3);
\end{tikzpicture}
\end{figure}
\textbf{Output}: 6
\end{flushleft}
\paragraph{Example 2:}
\begin{flushleft}
\textbf{Input}:
\begin{figure}[H]
\begin{tikzpicture}
[mynode/.style={draw,circle,minimum size=5mm, fill=gray!20!}]
\node(){};
\node[mynode](1) {-10};
\node[mynode](2)[below=8mm of 1, xshift=-8mm] {9};
\node[mynode](3)[below=8mm of 1, xshift=8mm] {\textcolor{red}{20}};
\node[mynode](4)[below=8mm of 3, xshift=-8mm] {\textcolor{red}{15}};
\node[mynode](5)[below=8mm of 3, xshift=8mm] {\textcolor{red}{7}};
\draw[>=stealth,->] (1) -- (2);
\draw[>=stealth,->] (1) -- (3);
\draw[>=stealth,->] (3) -- (4);
\draw[>=stealth,->] (3) -- (5);
\end{tikzpicture}
\end{figure}
\textbf{Output}: 42
\end{flushleft}
\subsection{DFS}
\begin{figure}[H]
\begin{tikzpicture}
[mynode/.style={draw,circle,minimum size=5mm, fill=gray!20!}]
\node(){};
\node[mynode](1) {4};
\node[mynode](2)[below=8mm of 1, xshift=-8mm] {11};
\node[mynode](3)[below=8mm of 1, xshift=8mm] {13};
\node[mynode](4)[below=8mm of 2, xshift=-8mm] {7};
\node[mynode](5)[below=8mm of 2, xshift=8mm] {2};
\draw[>=stealth,->] (1) -- (2);
\draw[>=stealth,->] (1) -- (3);
\draw[>=stealth,->] (2) -- (4);
\draw[>=stealth,->] (2) -- (5);
\end{tikzpicture}
\end{figure}

%\setcounter{algorithm}{0}
%\begin{algorithm}[H]
%\caption{DFS}
%\begin{algorithmic}[1]
%\Procedure{MaxPathSum}{$T$}
%\State $\texttt{ans}:=-\infty$
%\State $\texttt{DFS}(T,\;\texttt{ans})$ \Comment Call helper function \texttt{DFS}
%\State \Return \texttt{ans}
%\EndProcedure
%\end{algorithmic}
%\end{algorithm}
%Function \texttt{DFS}递归遍历二叉树$T$,并更新最大的path sum $S$。
%\begin{algorithm}[H]
%\caption{Helper Function}
%\begin{algorithmic}[1]
%\Function{DFS}{$T,\;S$}
%\If{$T=\texttt{null}$}
%\State \Return 0
%\EndIf
%\State $x:=\texttt{DFS}(\texttt{Left}(T),\;S)$ \Comment Recursive on left subtree
%\State $y:=\texttt{DFS}(\texttt{Right}(T),\;S)$ \Comment Recursive on right subtree
%\State $x\gets \max(0, x)$ \Comment Negative value is neglected
%\State $y\gets \max(0, y)$ \Comment Negative value is neglected
%\State $S\gets \max(S, x+y+\texttt{Value}(T))$ \Comment Update max path sum so far
%\State \Return $\max(x,y) + \texttt{Value}(T)$ \Comment Returns the max path sum starting with $T$
%\algstore{124algo}
%\end{algorithmic}
%\end{algorithm}
%\begin{algorithm}[H]
%\begin{algorithmic}[1]
%\algrestore{124algo}
%\EndFunction
%\end{algorithmic}
%\end{algorithm}

\subsection{Recursion}
To simplify the problem, we first try to find a way to compute a maximum sum path that goes though one node and/or one of its subtrees. Since this path may not go through root node, we have to check at each step what is better: to continue the current path or to start a new path with the current node as a highest node in this new path

以图中所示的树为例子,很容易就能找到最长路径为$7\to11\to4\to13$,由于树的递归解法一般都是递归到叶节点,然后开始处理并回溯到根节点。假设此时已经递归到结点7了,那么其没有左右子节点,所以以结点7为根结点的子树最大路径和就是7。然后回溯到结点11。而以结点11为根结点的子树的最大路径和为$7+11+2=20$。但是当回溯到结点4的时候,对于结点11来说,就不能同时取两条路径了,只能取左路径,或者是右路径,所以当根结点是4的时候,那么结点11只能取其左子结点7,因为7大于2。所以,对于每个结点来说,需要比较经过其left child的path sum以及经过right child的path sum,看这两个值哪个大。因此递归函数返回值就可以定义为以当前结点为根结点,到叶子节点的最大路径之和,然后全局路径最大值放在函数的输入参数中。

在递归函数中,如果当前结点不存在,那么直接返回0。否则就分别对其左右子节点调用递归函数,由于path sum有可能为负数,加上负数只会让结果更小,因此需要和零比较,取较大的那个。然后更新全局最大值,而这个值就是以left child为起点的最大path sum加上以right child为起点的最大path sum,再加上当前结点值,这样就组成了一条完整的路径。而返回值是取左右两个path sum中的较大值加上当前结点值,因为返回值的定义是以当前结点为起点的path sum,所以只能取左右两个path sum中的一个。

\setcounter{lstlisting}{0}
\begin{lstlisting}[style=customc, caption={Recursion}]
int maxPathSum( TreeNode* root )
{
    int best = INT_MIN;
    dfs( root, best );
    return best;
}
int dfs( TreeNode* node, int& best )
{
    if( !node )
    {
        return 0;
    }
    //maximum sum on the left child
    int left_sum = ( max )( dfs( node->left, best ), 0 );
    //maximum sum on the right child
    int right_sum = ( max )( dfs( node->right, best ), 0 );
    //the sum we can obtain by starting a new path
    //where node is the highest one
    int newpath_sum = node->val + left_sum + right_sum;
    //update global maximum sum
    //if new path can bring a greater one
    best = ( max )( best, newpath_sum );
    //for recursion:
    //we have to return the maximum sum
    //if we continue the path from other node
    //which go through current one
    return node->val + ( max )( left_sum, right_sum );
}
\end{lstlisting}