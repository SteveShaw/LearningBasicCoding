\section{145 --- Binary Tree Postorder Traversal}
Given a binary tree $R$, return the \textit{postorder} traversal of its nodes' values.
\paragraph{Example:}
\begin{flushleft}
\textbf{Input}:
\begin{figure}[H]
\begin{tikzpicture}
[mynode/.style={draw,circle,minimum size=5mm, fill=gray!20!}]
\node(){};
\node[mynode](1) {1};
\node[mynode](2)[below=8mm of 1, xshift=8mm] {2};
\node[mynode](3)[below=8mm of 2, xshift=-8mm] {3};
\draw[>=stealth,->] (1) -- (2);
\draw[>=stealth,->] (2) -- (3);
\end{tikzpicture}
\end{figure}
\textbf{Output}: $[3,2,1]$
\end{flushleft}
\paragraph{Follow up:}
\begin{itemize}
\item Recursive solution is trivial, could you do it iteratively?
\end{itemize}
\subsection{Stack}
和preorder一样,也有两种iterative的方法: stack和morris traversal。stack 的 output 过程和 preorder 类似,唯一不同的是需要先把left child放入stack中,另外最后要把得到的数组进行reverse。因为插入的过程中,当前节点的值都放在最前面了。
\subsection{Morris Traversal}
和Preorder方法类似,但是需要做改动,即,这时候需要看其right child是否存在,如果存在,则一直向左,然后建立link。如果不存在,则进入left child,并输出当前节点。