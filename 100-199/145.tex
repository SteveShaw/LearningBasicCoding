\section{145 --- Binary Tree Postorder Traversal}
Given a binary tree $R$, return the \textit{postorder} traversal of its nodes' values.
\paragraph{Example:}
\begin{flushleft}
\textbf{Input}:
\begin{figure}[H]
\begin{tikzpicture}
[mynode/.style={draw,circle,minimum size=5mm, fill=gray!20!}]
\node(){};
\node[mynode](1) {1};
\node[mynode](2)[below=8mm of 1, xshift=8mm] {2};
\node[mynode](3)[below=8mm of 2, xshift=-8mm] {3};
\draw[>=stealth,->] (1) -- (2);
\draw[>=stealth,->] (2) -- (3);
\end{tikzpicture}
\end{figure}
\textbf{Output}: $[3,2,1]$
\end{flushleft}
\paragraph{Follow up:}
\begin{itemize}
\item Recursive solution is trivial, could you do it iteratively?
\end{itemize}
\subsection{Stack}
和preorder一样,也有两种iterative的方法: stack和morris traversal。stack 的 output 过程和 preorder 类似,唯一不同的是需要先把left child放入stack中,另外最后要把得到的数组进行reverse。因为插入的过程中,当前节点的值都放在最前面了。

\setcounter{lstlisting}{0}
\begin{lstlisting}[style=customc, caption={Iteraive Apporach 1}]
vector<int> postorderTraversal( TreeNode* root )
{
    if( !root )
    {
        return {};
    }
    stack<TreeNode*> stk;
    stk.push( root );
    vector<int> ans;
    while( !stk.empty() )
    {
        auto t = stk.top();
        stk.pop();
        //we push left child first
        if( t->left )
        {
            stk.push( t->left );
        }
        //push right child second
        if( t->right )
        {
            stk.push( t->right );
        }
        //add current node's value
        ans.push_back( t->val );
    }
    //the sequence in the adding is root, right, left
    //thus after reverse, the sequence is left, right, root
    reverse( begin( ans ), end( ans ) );
    return ans;
}
\end{lstlisting}

The second approach using stack is with an auxiliary node $p$.

\begin{lstlisting}[style=customc, caption={Stack With Auxiliary Node}]
vector<int> postorderTraversal( TreeNode* root )
{
    if( !root )
    {
        return {};
    }
    stack<TreeNode*> stk;
    auto p = root;
    vector<int> ans;
    while( !stk.empty() || p )
    {
        if( p )
        {
            stk.push( p );
            //add current node's value to the output
            ans.push_back( p->val );
            //we visit right child tree first
            p = p->right;
        }
        else
        {
            auto t = stk.top();
            stk.pop();
            //visit left child tree
            p = t->left;
        }
    }
    //we need to reverse the output
    //since the sequence is root,right,left
    reverse( begin( ans ), end( ans ) );
    return ans;
}
\end{lstlisting} 

%\subsection{Morris Traversal}
%和Preorder方法类似,但是需要做改动,即,这时候需要看其right child是否存在,如果存在,则一直向左,然后建立link。如果不存在,则进入left child,并输出当前节点。