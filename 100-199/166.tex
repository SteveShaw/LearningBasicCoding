\section{166 --- Fraction to Recurring Decimal}
Given two integers $N$ and $D$ representing the numerator and denominator of a fraction, return the fraction in string format.
\par
If the fractional part is repeating, enclose the repeating part in parentheses.
\paragraph{Example 1:}
\begin{flushleft}
\textbf{Input}: $N = 1$, $D = 2$
\\
\textbf{Output}: $0.5$
\end{flushleft}
\paragraph{Example 2:}
\begin{flushleft}
\textbf{Input}: $N = 2$, $D = 1$
\\
\textbf{Output}: 2
\end{flushleft}
\paragraph{Example 3:}
\begin{flushleft}
\textbf{Input}: $N = 2$, $D = 3$
\\
\textbf{Output}: $0.(6)$
\end{flushleft}
\subsection{Hash Map}
\begin{CJK*}{UTF8}{gbsn}
几个要点:
\begin{itemize}
    \item 当余数发生重复时,即可停止计算,因为已经找到循环点了。
    \item 用一个hash map保存每个小数位上的数字。
    \item 由于要算出小数每一位,因此和手动计算一样,每次要把余数乘10,再除以denominator,得到的商即为小数的下一位数字。
    \item 如果余数在之前出现过,则在这个余数之前对应的位置前加左括号,然后在当前位置处加右括号。
    \item 由于商必定是小于10的数字,因此小数点的位置每次只是增加1
\end{itemize}
另外还需要考虑如下几个边界情况
\begin{itemize}
    \item 需要注意符号是否为正或者负。
    \item 可能会造成 overflow 的情况,比如 $N = -2147483648$, $D = -1$,因此具体算法实现时应使用比int type更长的类型,比如long long。
    \item 如果结果为0,而其中一个数字是负数,也要注意不能把负号放入结果中。
\end{itemize}
\end{CJK*}
\setcounter{algorithm}{0}
\begin{algorithm}[H]
\caption{Hash Map}
\begin{algorithmic}[1]
\Procedure{FractionToDecimal}{$N, D$}
\State $S:=\emptyset$ \Comment The result decimal string
\If{$N\times D < 0$}
\State Add minus symbol $-$ to $S$
\State Change any negative number to positive number so that $N\geq0$ and $D\geq0$ 
\EndIf
\State $q:=N/D$ \Comment The quotient
\State $r:=N - D\times q$ \Comment The remainder
\State Transform $q$ to string and append to integer part $S$
\State $P:=\emptyset$ \Comment The string for decimal part
\State $M:=\emptyset$ \Comment The hash map for the remainder and the position inside $P$
\State $t:=0$ \Comment The current position of decimal part
\While{$r\neq 0$} \label{166while}
\If{$r \in M$}
\State Put left parenthesis at position $M[r]$ in $P$
\State Append right parenthesis to $P$
\State \texttt{break} \Comment Stop while loop [\ref{166while}]
\EndIf
\State $M[r]:=t$
\State $t\gets t+1$ \Comment Move to next decimal position
\State $r\gets r\times 10$ \Comment Calculate next quotient
\State $q \gets r / D$
\State Transform $q$ to string and append to decimal part $P$
\State $r\gets r - D\times q$ \Comment New remainder
\EndWhile
\algstore{166algo}
\end{algorithmic}
\end{algorithm}
\begin{algorithm}[H]
\begin{algorithmic}[1]
\algrestore{166algo}
\If{$P=\emptyset$} \Comment $N$ can be divided by $D$
\If{$q=0$} 
\State \Return 0 \Comment 0 does not have negative sign
\Else
\State \Return $S$ \Comment $S$ is the result
\EndIf
\EndIf
\State Append dot symbol to $S$
\State $S\gets S + P$ \Comment Append decimal part to integer part
\State \Return $S$
\EndProcedure
\end{algorithmic}
\end{algorithm}
\setcounter{lstlisting}{0}
\begin{lstlisting}[style=customc, caption={Source Code}]
string fractionToDecimal( int numerator, int denominator )
{
    auto N = static_cast<long long>( numerator );
    auto D = static_cast<long long>( denominator );

    string ans;

	//Negative sign
    if( N * D < 0 )
    {
        ans.push_back( '-' );

        N = ( max )( N, -N );
        D = ( max )( D, -D );
    }

    auto q = N / D;
    auto r = N - q * D;

    ans += to_string( q );

    string decimal;

    size_t pos = 0;
    unordered_map<long long, size_t> m;

    while( r != 0 )
    {
        auto it = m.find( r );

        if( it != m.end() )
        {
            decimal.insert( it->second, "(" );
            decimal.push_back( ')' );
            break;
        }

        m.emplace( r, pos );
        ++pos;

        r *= 10;
        q = r / D;
        r -= q * D;

        decimal += to_string( q );
    }

    if( decimal.empty() )
    {
        if( q == 0 )
        {
            return "0";
        }

        return ans;
    }

    ans.push_back( '.' );
    ans += decimal;

    return ans;
}
\end{lstlisting}