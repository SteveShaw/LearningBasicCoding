\section{191 --- Number of 1 Bits}
Write a function that takes an unsigned integer and return the number of \textbf{1} bits it has (also known as the Hamming weight).
\paragraph{Example 1:}
\begin{flushleft}
\textbf{Input}: 00000000000000000000000000001011
\\
\textbf{Output}: 3

\textbf{Explanation}: 
\\
The input binary string \textbf{00000000000000000000000000001011} has a total of three \textbf{1} bits.
\end{flushleft}
\paragraph{Example 2:}
\begin{flushleft}
\textbf{Input}: 00000000000000000000000010000000
\\
\textbf{Output}: 1
\\
\textbf{Explanation}: The input binary string \textbf{00000000000000000000000010000000} has a total of one \textbf{1} bit.
\end{flushleft}
\textbf{Example 3:}
\begin{flushleft}
\textbf{Input}: 11111111111111111111111111111101
\\
\textbf{Output}: 31
\\
\textbf{Explanation}:
\\
The input binary string \textbf{11111111111111111111111111111101} has a total of thirty one \textbf{1} bits.
\end{flushleft}
\textbf{Follow up:}
\begin{itemize}
\item If this function is called many times, how would you optimize it?
\end{itemize}
\subsection{Trick}
\begin{CJK*}{UTF8}{gbsn}
$n \text{ \& } (n-1)$会把末尾的1消除掉,因此不停的进行此操作直至$n$为0,就可以得到1的个数了。
\end{CJK*}
\setcounter{lstlisting}{0}
\begin{lstlisting}[style=customc, caption={Bit Tricks}]
int hammingWeight( uint32_t n )
{
    int ans = 0;
	
    while( n )
    {
        n = n & ( n - 1 );
        ++ans;
    }

    return ans;
}
\end{lstlisting}