\section{103 --- Binary Tree Zigzag Level Order Traversal}
Given a binary tree $T$, return the zigzag level order traversal of its nodes' values. (ie, from left to right, then right to left for the next level and alternate between). For example: Given binary tree
\begin{figure}[H]
\begin{tikzpicture}
[mynode/.style={draw,circle,minimum size=10mm, fill=gray!20!}]
\node(){};
\node[mynode](3) {3};
\node[mynode](9) [below = 8mm of 3, xshift=-10mm] {9};
\node[mynode](20) [below = 8mm of 1, xshift=10mm] {20};
\node[mynode](15) [below = 8mm of 20, xshift=-6mm] {15};
\node[mynode](7) [below = 8mm of 20, xshift=6mm] {7};
\draw[>=stealth,->] (3) -- (9);
\draw[>=stealth,->] (3) -- (20);
\draw[>=stealth,->] (20) -- (15);
\draw[>=stealth,->] (20) -- (7);
\end{tikzpicture}
\end{figure}
return its zigzag level order traversal as:
\\
$\left((3),\ (20,9),\ (15,7)\right)$
\subsection{BFS}
同样用BFS搜索,记录每层的节点个数即当前queue的size,然后分奇数偶数层确定访问节点的方向,具体说就是对于奇数层,从右往左,对于偶数层则从左往右,层数从0开始。