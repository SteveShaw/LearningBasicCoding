\section{105 --- Construct Binary Tree from Preorder and Inorder Traversal}\label{105algo}
Given preorder and inorder traversal of a tree, construct the binary tree.
\par
\textbf{Note}: You may assume that duplicates do not exist in the tree.
\par
For example, given
\begin{itemize}
    \item preorder $A = (3,9,20,15,7)$
    \item inorder $B = (9,3,15,20,7)$
\end{itemize}
Return the following binary tree:
\begin{figure}[H]
\begin{tikzpicture}
[mynode/.style={draw,circle,minimum size=10mm, fill=gray!20!}]
\node(){};
\node[mynode](3) {3};
\node[mynode](9) [below = 8mm of 3, xshift=-10mm] {9};
\node[mynode](20) [below = 8mm of 1, xshift=10mm] {20};
\node[mynode](15) [below = 8mm of 20, xshift=-6mm] {15};
\node[mynode](7) [below = 8mm of 20, xshift=6mm] {7};
\draw[>=stealth,->] (3) -- (9);
\draw[>=stealth,->] (3) -- (20);
\draw[>=stealth,->] (20) -- (15);
\draw[>=stealth,->] (20) -- (7);
\end{tikzpicture}
\end{figure}
\subsection{Recursion}
由于Preorder的第一个肯定是根,所以原二叉树的根节点可以知道,题目中给了一个很关键的条件就是树中没有相同元素,有了这个条件我们就可以在中序遍历中也定位出根节点的位置,并以根节点的位置将中序遍历拆分为左右两个部分,分别对其递归调用原函数。
\par
递归函数必须指定构建二叉树的preorder数组$A$的区间$[p_0, p_1)$和inorder数组$B$的区间$[i_0,i_1)$,然后由于$A[p_0]$就是当前子树的root,因此在$B$中给定的范围顺序搜寻$A[p_0]$在$B$中的位置$\alpha$。找到这个位置后,就知道以$A[p_0]$为root的左子树的长度$L_0:=\alpha - i_0$,右子树的长度$L_1:=i_1 - \alpha - 1$,因此接下来递归建立左子树时,在preroder数组$A$中的搜寻范围是$[p_0+1, p_0+1+L_0)$,相对应的在$B$中的搜寻范围是$[i_0, \alpha)$,而建立右子树时,在$A$中的搜寻范围是$[p_0+L_0+1, p_1)$,在$B$中的搜寻范围是$[\alpha+1, i_1)$
