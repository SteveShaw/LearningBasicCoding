\section{160 --- Intersection of Two Linked Lists}
Write a program to find the node at which the intersection of two singly linked lists begins.
\par
For example, the following two linked lists:
\begin{figure}[H]
\begin{tikzpicture}
[mynode/.style={draw,circle,minimum size=1cm, fill=gray!20!}]
\node(){};
\node[minimum size=1cm](A) {A:};
\node[minimum size=1cm](E)[right=8mm of A] {};
\node[mynode](a1)[right=8mm of E] {$a_1$};
\node[mynode](a2)[right=8mm of a1] {$a_2$};
\node[minimum size=1cm](B) [below=1.6cm of A]{B:};
\node[mynode](b1)[right=8mm of B] {$b_1$};
\node[mynode](b2)[right=8mm of b1] {$b_2$};
\node[mynode](b3)[right=8mm of b2] {$b_3$};
\node[mynode](c1)[right=8mm of b3, yshift=1.2cm] {$c_1$};
\node[mynode](c2)[right=8mm of c1] {$c_2$};
\node[mynode](c3)[right=8mm of c2] {$c_3$};
\draw[>=stealth,->] (a1) -- (a2);
\draw[>=stealth,->] (a2) to[bend left=15] (c1);
\draw[>=stealth,->] (b1) -- (b2);
\draw[>=stealth,->] (b2) -- (b3);
\draw[>=stealth,->] (b3) to[bend right=15] (c1);
\draw[>=stealth,->] (a2) to[bend left=15] (c1);
\draw[>=stealth,->] (c1) -- (c2);
\draw[>=stealth,->] (c2) -- (c3);
\end{tikzpicture}
\end{figure}

begin to intersect at node $c_1$.
\paragraph{Example 1:}
\begin{flushleft}
\textbf{Input}:
\\
\begin{figure}[H]
\begin{tikzpicture}
[mynode/.style={draw,circle,minimum size=1cm, fill=gray!20!}]
\node(){};
\node[minimum size=1cm](A) {A:};
\node[minimum size=1cm](E)[right=8mm of A] {};
\node[mynode](a1)[right=8mm of E] {4};
\node[mynode](a2)[right=8mm of a1] {1};
\node[minimum size=1cm](B) [below=1.6cm of A]{B:};
\node[mynode](b1)[right=8mm of B] {5};
\node[mynode](b2)[right=8mm of b1] {0};
\node[mynode](b3)[right=8mm of b2] {1};
\node[mynode](c1)[right=8mm of b3, yshift=1.2cm] {8};
\node[mynode](c2)[right=8mm of c1] {4};
\node[mynode](c3)[right=8mm of c2] {5};
\draw[>=stealth,->] (a1) -- (a2);
\draw[>=stealth,->] (a2) to[bend left=15] (c1);
\draw[>=stealth,->] (b1) -- (b2);
\draw[>=stealth,->] (b2) -- (b3);
\draw[>=stealth,->] (b3) to[bend right=15] (c1);
\draw[>=stealth,->] (a2) to[bend left=15] (c1);
\draw[>=stealth,->] (c1) -- (c2);
\draw[>=stealth,->] (c2) -- (c3);
\end{tikzpicture}
\end{figure}
\textbf{Output}: Reference of the node with value $= 8$
\\
\textbf{Input Explanation}: The intersected node's value is 8 (note that this must not be 0 if the two lists intersect). From the head of $A$, it reads as $[4,1,8,4,5]$. From the head of $B$, it reads as \fcj{[5,0,1,8,4,5]}. There are 2 nodes before the intersected node in $A$; There are 3 nodes before the intersected node in $B$.
\end{flushleft}

\paragraph{Example 2:}
\begin{flushleft}
\textbf{Input}:
\begin{figure}[H]
\begin{tikzpicture}
[mynode/.style={draw,circle,minimum size=1cm, fill=gray!20!}]
\node(){};
\node[minimum size=1cm](A) {A:};
\node[mynode](a1)[right=8mm of A] {0};
\node[mynode](a2)[right=8mm of a1] {9};
\node[mynode](a3)[right=8mm of a2] {1};
\node[minimum size=1cm](B) [below=1.6cm of A]{B:};
\node[minimum size=1cm](E1)[right=8mm of B] {};
\node[minimum size=1cm](E2)[right=8mm of E1] {};
\node[mynode](b1)[right=8mm of E2] {3};
\node[mynode](c1)[right=8mm of b3, yshift=1.2cm] {2};
\node[mynode](c2)[right=8mm of c1] {4};
\draw[>=stealth,->] (a1) -- (a2);
\draw[>=stealth,->] (a2) -- (a3);
\draw[>=stealth,->] (a3) to[bend left=15] (c1);
\draw[>=stealth,->] (b1) to[bend right=15] (c1);
\draw[>=stealth,->] (c1) -- (c2);
\end{tikzpicture}
\end{figure}
\textbf{Output}: Reference of the node with value $= 2$
\\
\textbf{Input Explanation}: The intersected node's value is 2 (note that this must not be 0 if the two lists intersect). From the head of $A$, it reads as $[0,9,1,2,4]$. From the head of $B$, it reads as $[3,2,4]$. There are 3 nodes before the intersected node in $A$; There are 1 node before the intersected node in $B$.
\end{flushleft}

\paragraph{Example 3:}
\begin{flushleft}
\textbf{Input}:
\\
\begin{figure}[H]
\begin{tikzpicture}
[mynode/.style={draw,circle,minimum size=1cm, fill=gray!20!}]
\node(){};
\node[minimum size=1cm](A) {A:};
\node[mynode](a1)[right=8mm of A] {2};
\node[mynode](a2)[right=8mm of a1] {6};
\node[mynode](a3)[right=8mm of a2] {4};
\node[minimum size=1cm](B) [below=1.6cm of A]{B:};
\node[minimum size=1cm](E)[right=8mm of B] {};
\node[mynode](b1)[right=8mm of E] {1};
\node[mynode](b2)[right=8mm of b1] {5};
\draw[>=stealth,->] (a1) -- (a2);
\draw[>=stealth,->] (a2) -- (a3);
\draw[>=stealth,->] (b1) -- (b2);
\end{tikzpicture}
\end{figure}
\textbf{Output}: Reference of the node with value $= 8$
\\
\textbf{Input Explanation}: From the head of $A$, it reads as $[2,6,4]$. From the head of $B$, it reads as $[1,5]$. Since the two lists do not intersect, intersectVal must be 0
\\.
\textbf{Explanation}: The two lists do not intersect, so return \texttt{null}.
\end{flushleft}
\paragraph{Notes:}
\begin{itemize}
\item If the two linked lists have no intersection at all, return \texttt{null}.
\item The linked lists must retain their original structure after the function returns.
\item You may assume there are no cycles anywhere in the entire linked structure.
\item Your code should preferably run in $O(n)$ time and use only $O(1)$ memory.
\end{itemize}
\subsection{Two Pointers}
\begin{itemize}
\item Maintain two pointers $x$ and $y$ initialized at the head of $A$ and $B$, respectively. Then let them both traverse through the lists, one node at a time.
\item When $x$ reaches the end of a list, then redirect it to the head of $B$; similarly when $y$ reaches the end of a list, redirect it the head of $A$.
\item If at any point $x$ meets $y$, then $x/y$ is the intersection node.
\item If two lists have intersection, then their last nodes must be the same one. So when $x$ or $y$ reaches the end of a list, record the last element of $A$ and $B$ respectively. If the two last elements are not the same one, then the two lists have no intersections.
\end{itemize}
\setcounter{lstlisting}{0}
\begin{lstlisting}[style=customc, caption={Reference Code}]
ListNode *getIntersectionNode( ListNode *headA, ListNode *headB ) {

    auto x = headA;
    auto y = headB;

    if( !x || !y )
    {
        return nullptr;
    }

    ListNode* lastA = nullptr;

    while( x->next )
    {
        x = x->next;
    }

    while( y->next )
    {
        y = y->next;
    }

	// A and B have no intersection
    if( x != y )
    {
        return nullptr;
    }

    x = headA;
    y = headB;

    int x0 = 0;
    int y0 = 1;

    ListNode* A[2]  = {headA, headB};

    while( x != y )
    {
        x = x->next;

        if( !x )
        {
            x0 = 1 - x0;
            x = A[x0];
        }

        y = y->next;

        if( !y )
        {
            y0 = 1 - y0;
            y = A[y0];
        }
    }

    return x;
}
\end{lstlisting}