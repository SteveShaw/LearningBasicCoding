\section{187 --- Repeated DNA Sequences}
All DNA is composed of a series of nucleotides abbreviated as $A$, $C$, $G$, and $T$, for example: \texttt{ACGAATTCCG}. When studying DNA, it is sometimes useful to identify repeated sequences within the DNA.
\par
Write a function to find all the 10-letter-long sequences (substrings) that occur more than once in a DNA molecule.
\paragraph{Example:}
\begin{flushleft}
\textbf{Input}: $s = \texttt{AAAAACCCCCAAAAACCCCCCAAAAAGGGTTT}$
\\
\textbf{Output}: $[\texttt{AAAAACCCCC}, \texttt{CCCCCAAAAA}]$
\end{flushleft}
\subsection{Hash Set}
\begin{CJK*}{UTF8}{gbsn}
\begin{itemize}
\item 需要用两个hash set,一个用来记录出现过的string,另外一个用来存放出现过至少两次的string。之所以用hash set来存放至少两次的string,是因为结果中也不能出现重复的string。
\item 直接用hash set记录出现的10个字母的string
\item 如果碰到的string在hash set中出现过,加入到存放结果的hash set中。
\end{itemize}
\end{CJK*}
\setcounter{lstlisting}{0}
\begin{lstlisting}[style=customc, caption={Hash Sets}]
vector<string> findRepeatedDnaSequences( string s )
{
    if( s.size() <= 10 )
    {
        return {};
    }

    unordered_set<string> ss;

    auto key = s.substr( 0, 10 );

    ss.emplace( key );

    unordered_set<string> ss_ans;

    for( size_t i = 1; i <= s.size() - 10; ++i )
    {
        key = s.substr( i, 10 );

        if( ss.find( key ) != ss.end() )
        {
            ss_ans.insert( key );
        }
        else
        {
            ss.insert( key );
        }
    }

    return {ss_ans.begin(), ss_ans.end()};
}
\end{lstlisting}