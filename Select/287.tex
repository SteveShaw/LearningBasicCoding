\section{287 --- Find the Duplicate Number}
Given an array $A$ containing $n + 1$ integers where each integer is between 1 and $n$ (inclusive), prove that at least one duplicate number must exist. Assume that there is only one duplicate number, find the duplicate one.

\paragraph{Example 1:}
\begin{flushleft}
\textbf{Input}: $[1,3,4,2,2]$
\\
\textbf{Output}: 2
\end{flushleft}

\paragraph{Example 2:}
\begin{flushleft}
\textbf{Input}: $[3,1,3,4,2]$
\\
\textbf{Output}: 3
\end{flushleft}

\paragraph{Note:}
\begin{itemize}
\item You must not modify the array (assume the array is read only).
\item You must use only constant, $O(1)$ extra space.
\item Your runtime complexity should be less than $O(n^2)$.
\item There is only one duplicate number in the array, but it could be repeated more than once.
\end{itemize}
\subsection{Detect Cycle And Find The Cycle Start}
\begin{itemize}
\item 与检测链表是否有cycle并找到cycle的起始点类似。
\item 由于存在重复元素,并且$A$中每个值的大小都是1到$n$,而长度又是$n+1$,因此如果将$A[i]$看作index,那么根据$A[i]$的值访问下个元素$A[A[i]]$到最后就会形成cycle。
\item 另外由于0不属于1到$n$,因此$A[0]$肯定不在cycle内,这就相当于有cycle的链表里,链表的头指针不在cycle中。
\item 借用链表中的快慢指针,这里用两个index$x$和$y$分别代表快慢两个index,其中$x\gets A[A[x]]$ 而$y\gets A[y]$。如果遇到相等,就终止循环,
\item 然后$x$和$y$分别从0和刚才循环中断的地方开始按照$i\gets A[i]$的方式前进,直到相等,这个相等的位置处即为重复的number。
\end{itemize}
\setcounter{algorithm}{0}
\begin{algorithm}[H]
\caption{Detect Cycle And Find Cycle Start}
\begin{algorithmic}[1]
\Procedure{FindDuplicate}{$A$}
\State $\ast$ 初始化两个index $x$和$y$为0
\State $\ast$ 检测Cycle,找到快慢两个index的相遇点
\Repeat
\State $x\gets A[A[x]]$
\State $y\gets A[y]$
\Until{$x=y$}
\State $\ast$ 寻找Cycle的起始点
\State $z:=0$ \Comment 找起始点也需要两个index,其中一个是0,另外一个就是刚才找到的相遇点
\Repeat
\State $x\gets A[x]$
\State $z\gets A[z]$
\Until{$x=z$}
\State \Return $x$
\EndProcedure
\end{algorithmic}
\end{algorithm}
\setcounter{lstlisting}{0}
\begin{lstlisting}[style=customc, caption={Detect Cycle And Find Cycle Start}]
int findDuplicate( vector<int>& nums )
{
    //x is the fast index
    //y is the slow index
    int x = 0;
    int y = 0;

    //detect cycle and
    //find the meet point of x and y
    do
    {
        x = nums[nums[x]];
        y = nums[y];
    }
    while( x != y );

    //find the start of the cycle
    //one index will start from 0
    //since nums[0] is not inside cycle
    int z = 0;

    while( x != z )
    {
        x = nums[x];
        z = nums[z];
    }

    return x;
}
\end{lstlisting}
\subsection{Binary Search}
\paragraph{Why Binary Search Can Work}
\begin{enumerate}
\item 由于所有数字的范围都是$1$到$n$,那么如果$A$中没有重复数字,那么对于一个数$x$,在$A$小于或者等于$x$的数字个数就为$x$个。
\item 但由于$A$中有重复的数字,假设这个数字是$m$,那么$A$中小于或者等于这个重复数字$m$的至少有$m+1$个。由于$A$的长度为$n+1$,因此对$A$中其他数,小于或者等于这个数的个数都不会大于这个数本身。
\item 因此binary search是要在$1$到$n$这$n$个数中找到第一个数$\alpha$使得在$A$中小于等于$\alpha$的数的个数$z$要大于$\alpha$。典型的\textbf{rightmost} binary search。
\item 注意binary search不是作用于数组$A$。
\end{enumerate}
\paragraph{Algorithm Steps}
\begin{itemize}
\item 初始化左右两端的index $l$和$r$分别为1和$n$。
\item 得到当前的middle value $x=(l+r)/2$,遍历$A$,得到小于等于$x$的数的count $z$,
\begin{enumerate}
\item 如果$z\leq x$,$l\gets x+1$
\item 如果$z > x$,$r\gets x$
\item 循环结束后 $l$就是那个duplicate number。
\end{enumerate}
\end{itemize}
\begin{algorithm}[H]
\caption{Binary Search}
\begin{algorithmic}[1]
\Procedure{FindDuplicate}{$A,L$}
\State $l:=1$, $r:=L-1$ \Comment Binary Search on $[1,n]$ not on $A$ and $L=n+1$
\State $\ast$ Rightmost Binary Search
\While{$l < r$}
\State $x:=(l+r)/2$ \Comment Mid-point
\State $z:=0$
\For{Each number $t$ in $A$}
\If{$t \leq x$}
\State $z\gets z+1$
\EndIf
\EndFor
\If{$z\leq x$}
\State $l\gets x+1$
\Else
\State $r\gets x$
\EndIf
\EndWhile
\State \Return $l$
\EndProcedure
\end{algorithmic}
\end{algorithm}
\begin{lstlisting}[style=customc, caption={Binary Search}]
int findDuplicate( vector<int>& nums )
{
    int l = 1;
    //r=n
    int r = static_cast<int>( nums.size() ) - 1;

    //binary search on $[1...n]$
    //not on nums
    while( l < r )
    {
        int x = ( l + r ) / 2;

        int count = 0;
        for( int n : nums )
        {
            if( n <= x )
            {
                ++count;
            }
        }

        if( count <= x )
        {
            l = x + 1;
        }
        else
        {
            r = x;
        }
    }

    return l;
}
\end{lstlisting}
\subsection{Bit Manipulation}
\begin{itemize}
\item 在总共32个bit位上,分别统计出1到$n$上在每个比特位上的1的个数,记为$x$
\item 同样的在总共32个bit位上,分别统计出$A$中所有数字在每个比特位上的1的个数,记为$y$。
\item 如果在某个bit位上,$y>x$,表明该位上肯定有重复数字的贡献,因此将结果中的对应的bit位也设为1。
\item 这样32个bit位上设置好相应的bit 1后,就能得到那个重复数字。
\end{itemize}
\setcounter{lstlisting}{0}
\begin{lstlisting}[style=customc, caption={Bit Manipulation}]
int findDuplicate( vector<int>& nums )
{
    int mask = 1;

    int L = static_cast<int>( nums.size() );

    int ans = 0;

    for( int i = 0; i < 32; ++i )
    {
        //in fact, x counts for [0...n]
        //but 0 has no any effect
        mask = 1 << i;

        int x = 0;
        int y = 0;

        for( int i = 0; i < L; ++i )
        {
            if( i & mask )
            {
                ++x;
            }

            if( nums[i] & mask )
            {
                ++y;
            }
        }

        if( y > x )
        {
            ans |= mask;
        }
    }

    return ans;
}
\end{lstlisting}