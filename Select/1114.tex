\section{1114 --- Print in Order}
Suppose we have a class:

\begin{lstlisting}[style=customc]
public class Foo {
  public void one() { print(one); }
  public void two() { print(two); }
  public void three() { print(three); }
}
\end{lstlisting}

The same instance of \texttt{Foo} will be passed to three different threads. Thread $A$ will call \texttt{one}, thread $B$ will call \texttt{two}, and thread $C$ will call \texttt{three}. Design a mechanism and modify the program to ensure that \texttt{two} is executed after \texttt{one}, and \texttt{three} is executed after \texttt{two}.

 

\paragraph{Example 1:}

\begin{flushleft}
\textbf{Input}: $[1,2,3]$

\textbf{Output}: \texttt{onetwothree}

\textbf{Explanation}: 

There are three threads being fired asynchronously. 

The input $[1,2,3]$ means thread $A$ calls \texttt{one}(), thread $B$ calls \texttt{two}(), and thread $C$ calls \texttt{three}(). 

\texttt{onetwothree} is the correct output.
\end{flushleft}


\paragraph{Example 2:}

\begin{flushleft}
\textbf{Input}: $[1,3,2]$

\textbf{Output}: \texttt{onetwothree}

\textbf{Explanation}: 

The input $[1,3,2]$ means thread $A$ calls \texttt{one}(), thread $B$ calls \texttt{three}(), and thread $C$ calls \texttt{two}(). \texttt{onetwothree} is the correct output.
\end{flushleft}
 

\paragraph{Note:}

\begin{itemize}
\item We do not know how the threads will be scheduled in the operating system, even though the numbers in the input seems to imply the ordering. The input format you see is mainly to ensure our tests' comprehensiveness.
\end{itemize}

\subsection{Condition Variable}
In C++ 11 and later, The \texttt{condition\_variable} class is a synchronization primitive that can be used to block a thread, or multiple threads at the same time, until another thread both modifies a shared variable (the condition), and notifies the \texttt{condition\_variable}.

\setcounter{lstlisting}{0}
\begin{lstlisting}[style=customc, caption={Condition Variable}]
class Foo
{
public:
    Foo()
    {
        m_flag1 = false;
        m_flag2 = false;
    }

    void first( function<void()> printFirst )
    {
        // printFirst() outputs "first". Do not change or remove this line.
        printFirst();
        m_flag1 = true;
        m_cv1.notify_one();
    }

    void second( function<void()> printSecond )
    {
        // printSecond() outputs "second". Do not change or remove this line.
        unique_lock<mutex> lk( m_mtx );
        m_cv1.wait( lk, [this] {return m_flag1;} ); //when m_flag1 is true, m_cv1 will stop waiting
        printSecond();
        m_flag2 = true;
        m_cv2.notify_one();
    }

    void third( function<void()> printThird )
    {

        unique_lock<mutex> lk( m_mtx );
        m_cv2.wait( lk, [this] {return m_flag2;} ); //when m_flag2 is true, m_cv1 will stop waiting
        // printThird() outputs "third". Do not change or remove this line.
        printThird();
    }

    mutex m_mtx;
    condition_variable m_cv1;
    condition_variable m_cv2;

    bool m_flag1;
    bool m_flag2;
};
\end{lstlisting}