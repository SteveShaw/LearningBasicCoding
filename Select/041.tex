\section{41 --- First Missing Positive}
Given an unsorted integer array $A$, find the smallest missing positive integer.

\paragraph{Example 1:}

\begin{flushleft}
\textbf{Input}: $[1,2,0]$

\textbf{Output}: 3

\end{flushleft}

\paragraph{Example 2:}

\begin{flushleft}
\textbf{Input}: $[3,4,-1,1]$

\textbf{Output}: 2

\end{flushleft}

\paragraph{Example 3:}

\begin{flushleft}
\textbf{Input}: $[7,8,9,11,12]$

\textbf{Output}: 1
\end{flushleft}

\paragraph{Note:}

\begin{itemize}
\item Your algorithm should run in $O(n)$ time and uses constant extra space.
\end{itemize}

\subsection{Fill 1 to n}
$A$的长度记作$n$。方法是将$1$到$n$存入$A$,使得$A[i]=i+1$。这样$A$中如果有数字$k$在$1$到$n$,那么最后这个数字会在$A[k-1]$处。不在$[1,n]$范围内的数字,会占据$[1,n]$中缺失的数字的位置。

\setcounter{algorithm}{0}
\begin{algorithm}[H]
\caption{Swapping Positions To Find Missing Number}
\begin{algorithmic}[1]
\Procedure{FirstMissingPositive}{$A$, $n$}
\For{$i:=0$ \textbf{to} $n-1$}
\State $k:=A[i]$
\State $\ast$ put $k$ to $A[k-1]$
\While{$k>0$ \textbf{and} $k\leq n$ \textbf{and} $A[k-1] \neq k$}
\State \textbf{Swap} $A[k-1]$ and $A[i]$ \Comment Make sure that $A[k-1] = k$
\State $k \gets A[i]$ \Comment $k$ is updated as current $A[i]$
\EndWhile
\EndFor
\algstore{041algo}
\end{algorithmic}
\end{algorithm}
\begin{algorithm}[H]
\begin{algorithmic}[1]
\algrestore{041algo}
\For{$i:=0$ \textbf{to} $n-1$}
\If{$A[i] \neq i+1$}
\State \Return $i+1$ \Comment Found the first missing number in $[1,n]$
\EndIf
\EndFor
\State \Return $n+1$ \Comment All numbers in the array are in $[1,n]$
\EndProcedure
\end{algorithmic}
\end{algorithm}

\setcounter{lstlisting}{0}
\begin{lstlisting}[style=customc, caption={Swap}]
int firstMissingPositive( vector<int>& nums )
{
    int n = static_cast<int>( nums.size() );

    for( size_t i = 0; i < nums.size(); ++i )
    {
        int k = nums[i];

        //put k to nums[k-1]
        while( ( k > 0 ) && ( k <= n ) && ( nums[k - 1] != k ) )
        {
            swap( nums[i], nums[k - 1] );
            //now we need to put nums[i] to
            //appropriate position
            k = nums[i];
        }
    }

    for( int k = 0; k < n; ++k )
    {
        if( nums[k] != k + 1 )
        {
            //found first missing numbers
            //in [1,n]
            return k + 1;
        }
    }

    //all numbers in nums are in [1,n]
    return n + 1;
}
\end{lstlisting}

\subsection{Index As Hash Key}
\begin{itemize}
\item  Get rid of negative numbers and zeros since there is no need of them. One could get rid of all numbers larger than $L$ as well, since the first missing positive is for sure smaller or equal to $L + 1$. The case when the first missing positive is equal to $L + 1$ will be treated separately. We can replace them by 1s to get rid.
\item To ensure that the first missing positive is not 1, we have to verify the presence of 1 before proceeding to this operation.
\item Now there we have an array which contains only positive numbers in a range from 1 to $L$, and the problem is to find a first missing positive in $\mathcal{O}(N)$ time and constant space. 
\item The idea is to use index in $A$ as a hash key and \textbf{sign} of the element as the presence detector. For example, negative sign of $A[2]$ element means that number 3 is present in $A$. The positive sign of $A[3]$ element means that number 4 is not present (missing) in $A$.
\item To achieve that, iterating the array after cleaning, check each element, $A[i]$, and change the sign of $A[A[i]-1]$ to \textbf{negative} to mark that number $A[i]$ is present in $A$. Be careful with duplicates and ensure that the sign was changed only once.
\end{itemize}

As such, the algorithm is clear
\begin{algorithm}[H]
\caption{Index As Hash key}
\begin{algorithmic}[1]
\Procedure{FirstMissingPositive}{$A, L$}
\State $\star$ Check if $1$ is present in $A$. If not, return $1$.
\If{$L=1$}
\State \Return 2 \Comment No $1$
\EndIf
\algstore{41algo}
\end{algorithmic}
\end{algorithm}
\begin{algorithm}[H]
\begin{algorithmic}[1]
\algrestore{41algo}
\State $\ast$ Replace negative numbers, zeros, and numbers larger than $L$ by 1s.
\State $\ast$ Iterating $A$, change the sign of $(x-1)$-th element if you meet the number $x$. 
\State $\ast$ Iterating $A$, return the index of the first positive element plus 1.
\State $\ast$ If no any positive element is found, the only answer is $L+1$
\EndProcedure
\end{algorithmic}
\end{algorithm}

\begin{lstlisting}[style=customc, caption={Index As Hash Key}]
int firstMissingPositive( vector<int>& nums )
{
    int count = 0;

    int L = static_cast<int>( nums.size() );

    //count 1s
    //replace any element that is less than 1
    //or larger than L to 1
    for( int i = 0; i < L; ++i )
    {
        if( nums[i] == 1 )
        {
            ++count;
        }
        else if( ( nums[i] <= 0 ) || ( nums[i] > L ) )
        {
            nums[i] = 1;
        }
    }

    if( count == 0 )
    {
        //no 1 is present
        //the first missing is 1
        return 1;
    }

    if( nums.size() == 1 )
    {
        //this means nums = [1]
        //so answer is 2
        return 2;
    }

    for( int i = 0; i < L; ++i )
    {
        int x = abs( nums[i] );
        if( nums[x - 1] > 0 )
        {
            //change the sign of nums[x-1]
            nums[x - 1] = -nums[x - 1];
        }
    }

    for( int i = 0; i < L; ++i )
    {
        if( nums[i] > 0 )
        {
            //the first missing positive
            //is i+1
            return i + 1;
        }
    }

    //no any positive is found
    //nums are filled with 1--L
    //the first missing is L+1
    return L + 1;
}
\end{lstlisting}
